
\chapter{Simplification}
\label{chap:simplification}

\section{Simplification sets}\index{simplification sets} 

\subsection{Rewrite rules}
\begin{ttbox}
addsimps : simpset * thm list -> simpset \hfill{\bf infix 4}
delsimps : simpset * thm list -> simpset \hfill{\bf infix 4}
\end{ttbox}

\index{rewrite rules|(} Rewrite rules are theorems expressing some
form of equality, for example:
\begin{eqnarray*}
  Suc(\Var{m}) + \Var{n} &=&      \Var{m} + Suc(\Var{n}) \\
  \Var{P}\conj\Var{P}    &\bimp&  \Var{P} \\
  \Var{A} \un \Var{B} &\equiv& \{x.x\in \Var{A} \disj x\in \Var{B}\}
\end{eqnarray*}
Conditional rewrites such as $\Var{m}<\Var{n} \Imp \Var{m}/\Var{n} =
0$ are also permitted; the conditions can be arbitrary formulas.

Internally, all rewrite rules are translated into meta-equalities,
theorems with conclusion $lhs \equiv rhs$.  Each simpset contains a
function for extracting equalities from arbitrary theorems.  For
example, $\neg(\Var{x}\in \{\})$ could be turned into $\Var{x}\in \{\}
\equiv False$.  This function can be installed using
\ttindex{setmksimps} but only the definer of a logic should need to do
this; see {\S}\ref{sec:setmksimps}.  The function processes theorems
added by \texttt{addsimps} as well as local assumptions.

\begin{ttdescription}
  
\item[$ss$ \ttindexbold{addsimps} $thms$] adds rewrite rules derived
  from $thms$ to the simpset $ss$.
  
\item[$ss$ \ttindexbold{delsimps} $thms$] deletes rewrite rules
  derived from $thms$ from the simpset $ss$.

\end{ttdescription}

\begin{warn}
  The simplifier will accept all standard rewrite rules: those where
  all unknowns are of base type.  Hence ${\Var{i}+(\Var{j}+\Var{k})} =
  {(\Var{i}+\Var{j})+\Var{k}}$ is OK.
  
  It will also deal gracefully with all rules whose left-hand sides
  are so-called {\em higher-order patterns}~\cite{nipkow-patterns}.
  \indexbold{higher-order pattern}\indexbold{pattern, higher-order}
  These are terms in $\beta$-normal form (this will always be the case
  unless you have done something strange) where each occurrence of an
  unknown is of the form $\Var{F}(x@1,\dots,x@n)$, where the $x@i$ are
  distinct bound variables. Hence $(\forall x.\Var{P}(x) \land
  \Var{Q}(x)) \bimp (\forall x.\Var{P}(x)) \land (\forall
  x.\Var{Q}(x))$ is also OK, in both directions.
  
  In some rare cases the rewriter will even deal with quite general
  rules: for example ${\Var{f}(\Var{x})\in range(\Var{f})} = True$
  rewrites $g(a) \in range(g)$ to $True$, but will fail to match
  $g(h(b)) \in range(\lambda x.g(h(x)))$.  However, you can replace
  the offending subterms (in our case $\Var{f}(\Var{x})$, which is not
  a pattern) by adding new variables and conditions: $\Var{y} =
  \Var{f}(\Var{x}) \Imp \Var{y}\in range(\Var{f}) = True$ is
  acceptable as a conditional rewrite rule since conditions can be
  arbitrary terms.
  
  There is basically no restriction on the form of the right-hand
  sides.  They may not contain extraneous term or type variables,
  though.
\end{warn}
\index{rewrite rules|)}


\subsection{*The subgoaler}\label{sec:simp-subgoaler}
\begin{ttbox}
setsubgoaler :
  simpset *  (simpset -> int -> tactic) -> simpset \hfill{\bf infix 4}
prems_of_ss  : simpset -> thm list
\end{ttbox}

The subgoaler is the tactic used to solve subgoals arising out of
conditional rewrite rules or congruence rules.  The default should be
simplification itself.  Occasionally this strategy needs to be
changed.  For example, if the premise of a conditional rule is an
instance of its conclusion, as in $Suc(\Var{m}) < \Var{n} \Imp \Var{m}
< \Var{n}$, the default strategy could loop.

\begin{ttdescription}
  
\item[$ss$ \ttindexbold{setsubgoaler} $tacf$] sets the subgoaler of
  $ss$ to $tacf$.  The function $tacf$ will be applied to the current
  simplifier context expressed as a simpset.
  
\item[\ttindexbold{prems_of_ss} $ss$] retrieves the current set of
  premises from simplifier context $ss$.  This may be non-empty only
  if the simplifier has been told to utilize local assumptions in the
  first place, e.g.\ if invoked via \texttt{asm_simp_tac}.

\end{ttdescription}

As an example, consider the following subgoaler:
\begin{ttbox}
fun subgoaler ss =
    assume_tac ORELSE'
    resolve_tac (prems_of_ss ss) ORELSE'
    asm_simp_tac ss;
\end{ttbox}
This tactic first tries to solve the subgoal by assumption or by
resolving with with one of the premises, calling simplification only
if that fails.


\subsection{*The solver}\label{sec:simp-solver}
\begin{ttbox}
mk_solver  : string -> (thm list -> int -> tactic) -> solver
setSolver  : simpset * solver -> simpset \hfill{\bf infix 4}
addSolver  : simpset * solver -> simpset \hfill{\bf infix 4}
setSSolver : simpset * solver -> simpset \hfill{\bf infix 4}
addSSolver : simpset * solver -> simpset \hfill{\bf infix 4}
\end{ttbox}

A solver is a tactic that attempts to solve a subgoal after
simplification.  Typically it just proves trivial subgoals such as
\texttt{True} and $t=t$.  It could use sophisticated means such as {\tt
  blast_tac}, though that could make simplification expensive.
To keep things more abstract, solvers are packaged up in type
\texttt{solver}. The only way to create a solver is via \texttt{mk_solver}.

Rewriting does not instantiate unknowns.  For example, rewriting
cannot prove $a\in \Var{A}$ since this requires
instantiating~$\Var{A}$.  The solver, however, is an arbitrary tactic
and may instantiate unknowns as it pleases.  This is the only way the
simplifier can handle a conditional rewrite rule whose condition
contains extra variables.  When a simplification tactic is to be
combined with other provers, especially with the classical reasoner,
it is important whether it can be considered safe or not.  For this
reason a simpset contains two solvers, a safe and an unsafe one.

The standard simplification strategy solely uses the unsafe solver,
which is appropriate in most cases.  For special applications where
the simplification process is not allowed to instantiate unknowns
within the goal, simplification starts with the safe solver, but may
still apply the ordinary unsafe one in nested simplifications for
conditional rules or congruences. Note that in this way the overall
tactic is not totally safe:  it may instantiate unknowns that appear also 
in other subgoals.

\begin{ttdescription}
\item[\ttindexbold{mk_solver} $s$ $tacf$] converts $tacf$ into a new solver;
  the string $s$ is only attached as a comment and has no other significance.

\item[$ss$ \ttindexbold{setSSolver} $tacf$] installs $tacf$ as the
  \emph{safe} solver of $ss$.
  
\item[$ss$ \ttindexbold{addSSolver} $tacf$] adds $tacf$ as an
  additional \emph{safe} solver; it will be tried after the solvers
  which had already been present in $ss$.
  
\item[$ss$ \ttindexbold{setSolver} $tacf$] installs $tacf$ as the
  unsafe solver of $ss$.
  
\item[$ss$ \ttindexbold{addSolver} $tacf$] adds $tacf$ as an
  additional unsafe solver; it will be tried after the solvers which
  had already been present in $ss$.

\end{ttdescription}

\medskip

\index{assumptions!in simplification} The solver tactic is invoked
with a list of theorems, namely assumptions that hold in the local
context.  This may be non-empty only if the simplifier has been told
to utilize local assumptions in the first place, e.g.\ if invoked via
\texttt{asm_simp_tac}.  The solver is also presented the full goal
including its assumptions in any case.  Thus it can use these (e.g.\ 
by calling \texttt{assume_tac}), even if the list of premises is not
passed.

\medskip

As explained in {\S}\ref{sec:simp-subgoaler}, the subgoaler is also used
to solve the premises of congruence rules.  These are usually of the
form $s = \Var{x}$, where $s$ needs to be simplified and $\Var{x}$
needs to be instantiated with the result.  Typically, the subgoaler
will invoke the simplifier at some point, which will eventually call
the solver.  For this reason, solver tactics must be prepared to solve
goals of the form $t = \Var{x}$, usually by reflexivity.  In
particular, reflexivity should be tried before any of the fancy
tactics like \texttt{blast_tac}.

It may even happen that due to simplification the subgoal is no longer
an equality.  For example $False \bimp \Var{Q}$ could be rewritten to
$\neg\Var{Q}$.  To cover this case, the solver could try resolving
with the theorem $\neg False$.

\medskip

\begin{warn}
  If a premise of a congruence rule cannot be proved, then the
  congruence is ignored.  This should only happen if the rule is
  \emph{conditional} --- that is, contains premises not of the form $t
  = \Var{x}$; otherwise it indicates that some congruence rule, or
  possibly the subgoaler or solver, is faulty.
\end{warn}


\subsection{*The looper}\label{sec:simp-looper}
\begin{ttbox}
setloop   : simpset *           (int -> tactic)  -> simpset \hfill{\bf infix 4}
addloop   : simpset * (string * (int -> tactic)) -> simpset \hfill{\bf infix 4}
delloop   : simpset *  string                    -> simpset \hfill{\bf infix 4}
addsplits : simpset * thm list -> simpset \hfill{\bf infix 4}
delsplits : simpset * thm list -> simpset \hfill{\bf infix 4}
\end{ttbox}

The looper is a list of tactics that are applied after simplification, in case
the solver failed to solve the simplified goal.  If the looper
succeeds, the simplification process is started all over again.  Each
of the subgoals generated by the looper is attacked in turn, in
reverse order.

A typical looper is \index{case splitting}: the expansion of a conditional.
Another possibility is to apply an elimination rule on the
assumptions.  More adventurous loopers could start an induction.

\begin{ttdescription}
  
\item[$ss$ \ttindexbold{setloop} $tacf$] installs $tacf$ as the only looper
  tactic of $ss$.
  
\item[$ss$ \ttindexbold{addloop} $(name,tacf)$] adds $tacf$ as an additional
  looper tactic with name $name$; it will be tried after the looper tactics
  that had already been present in $ss$.
  
\item[$ss$ \ttindexbold{delloop} $name$] deletes the looper tactic $name$
  from $ss$.
  
\item[$ss$ \ttindexbold{addsplits} $thms$] adds
  split tactics for $thms$ as additional looper tactics of $ss$.

\item[$ss$ \ttindexbold{addsplits} $thms$] deletes the
  split tactics for $thms$ from the looper tactics of $ss$.

\end{ttdescription}

The splitter replaces applications of a given function; the right-hand side
of the replacement can be anything.  For example, here is a splitting rule
for conditional expressions:
\[ \Var{P}(if(\Var{Q},\Var{x},\Var{y})) \bimp (\Var{Q} \imp \Var{P}(\Var{x}))
\conj (\neg\Var{Q} \imp \Var{P}(\Var{y})) 
\] 
Another example is the elimination operator for Cartesian products (which
happens to be called~$split$):  
\[ \Var{P}(split(\Var{f},\Var{p})) \bimp (\forall a~b. \Var{p} =
\langle a,b\rangle \imp \Var{P}(\Var{f}(a,b))) 
\] 

For technical reasons, there is a distinction between case splitting in the 
conclusion and in the premises of a subgoal. The former is done by
\texttt{split_tac} with rules like \texttt{split_if} or \texttt{option.split}, 
which do not split the subgoal, while the latter is done by 
\texttt{split_asm_tac} with rules like \texttt{split_if_asm} or 
\texttt{option.split_asm}, which split the subgoal.
The operator \texttt{addsplits} automatically takes care of which tactic to
call, analyzing the form of the rules given as argument.
\begin{warn}
Due to \texttt{split_asm_tac}, the simplifier may split subgoals!
\end{warn}

Case splits should be allowed only when necessary; they are expensive
and hard to control.  Here is an example of use, where \texttt{split_if}
is the first rule above:
\begin{ttbox}
by (simp_tac (simpset() 
                 addloop ("split if", split_tac [split_if])) 1);
\end{ttbox}
Users would usually prefer the following shortcut using \texttt{addsplits}:
\begin{ttbox}
by (simp_tac (simpset() addsplits [split_if]) 1);
\end{ttbox}
Case-splitting on conditional expressions is usually beneficial, so it is
enabled by default in the object-logics \texttt{HOL} and \texttt{FOL}.


\section{Permutative rewrite rules}
\index{rewrite rules!permutative|(}

A rewrite rule is {\bf permutative} if the left-hand side and right-hand
side are the same up to renaming of variables.  The most common permutative
rule is commutativity: $x+y = y+x$.  Other examples include $(x-y)-z =
(x-z)-y$ in arithmetic and $insert(x,insert(y,A)) = insert(y,insert(x,A))$
for sets.  Such rules are common enough to merit special attention.

Because ordinary rewriting loops given such rules, the simplifier
employs a special strategy, called {\bf ordered
  rewriting}\index{rewriting!ordered}.  There is a standard
lexicographic ordering on terms.  This should be perfectly OK in most
cases, but can be changed for special applications.

\begin{ttbox}
settermless : simpset * (term * term -> bool) -> simpset \hfill{\bf infix 4}
\end{ttbox}
\begin{ttdescription}
  
\item[$ss$ \ttindexbold{settermless} $rel$] installs relation $rel$ as
  term order in simpset $ss$.

\end{ttdescription}

\medskip

A permutative rewrite rule is applied only if it decreases the given
term with respect to this ordering.  For example, commutativity
rewrites~$b+a$ to $a+b$, but then stops because $a+b$ is strictly less
than $b+a$.  The Boyer-Moore theorem prover~\cite{bm88book} also
employs ordered rewriting.

Permutative rewrite rules are added to simpsets just like other
rewrite rules; the simplifier recognizes their special status
automatically.  They are most effective in the case of
associative-commutative operators.  (Associativity by itself is not
permutative.)  When dealing with an AC-operator~$f$, keep the
following points in mind:
\begin{itemize}\index{associative-commutative operators}
  
\item The associative law must always be oriented from left to right,
  namely $f(f(x,y),z) = f(x,f(y,z))$.  The opposite orientation, if
  used with commutativity, leads to looping in conjunction with the
  standard term order.

\item To complete your set of rewrite rules, you must add not just
  associativity~(A) and commutativity~(C) but also a derived rule, {\bf
    left-com\-mut\-ativ\-ity} (LC): $f(x,f(y,z)) = f(y,f(x,z))$.
\end{itemize}
Ordered rewriting with the combination of A, C, and~LC sorts a term
lexicographically:
\[\def\maps#1{\stackrel{#1}{\longmapsto}}
 (b+c)+a \maps{A} b+(c+a) \maps{C} b+(a+c) \maps{LC} a+(b+c) \]
Martin and Nipkow~\cite{martin-nipkow} discuss the theory and give many
examples; other algebraic structures are amenable to ordered rewriting,
such as boolean rings.

\subsection{Example: sums of natural numbers}

This example is again set in HOL (see \texttt{HOL/ex/NatSum}).  Theory
\thydx{Arith} contains natural numbers arithmetic.  Its associated simpset
contains many arithmetic laws including distributivity of~$\times$ over~$+$,
while \texttt{add_ac} is a list consisting of the A, C and LC laws for~$+$ on
type \texttt{nat}.  Let us prove the theorem
\[ \sum@{i=1}^n i = n\times(n+1)/2. \]
%
A functional~\texttt{sum} represents the summation operator under the
interpretation $\texttt{sum} \, f \, (n + 1) = \sum@{i=0}^n f\,i$.  We
extend \texttt{Arith} as follows:
\begin{ttbox}
NatSum = Arith +
consts sum     :: [nat=>nat, nat] => nat
primrec 
  "sum f 0 = 0"
  "sum f (Suc n) = f(n) + sum f n"
end
\end{ttbox}
The \texttt{primrec} declaration automatically adds rewrite rules for
\texttt{sum} to the default simpset.  We now remove the
\texttt{nat_cancel} simplification procedures (in order not to spoil
the example) and insert the AC-rules for~$+$:
\begin{ttbox}
Delsimprocs nat_cancel;
Addsimps add_ac;
\end{ttbox}
Our desired theorem now reads $\texttt{sum} \, (\lambda i.i) \, (n+1) =
n\times(n+1)/2$.  The Isabelle goal has both sides multiplied by~$2$:
\begin{ttbox}
Goal "2 * sum (\%i.i) (Suc n) = n * Suc n";
{\out Level 0}
{\out 2 * sum (\%i. i) (Suc n) = n * Suc n}
{\out  1. 2 * sum (\%i. i) (Suc n) = n * Suc n}
\end{ttbox}
Induction should not be applied until the goal is in the simplest
form:
\begin{ttbox}
by (Simp_tac 1);
{\out Level 1}
{\out 2 * sum (\%i. i) (Suc n) = n * Suc n}
{\out  1. n + (sum (\%i. i) n + sum (\%i. i) n) = n * n}
\end{ttbox}
Ordered rewriting has sorted the terms in the left-hand side.  The
subgoal is now ready for induction:
\begin{ttbox}
by (induct_tac "n" 1);
{\out Level 2}
{\out 2 * sum (\%i. i) (Suc n) = n * Suc n}
{\out  1. 0 + (sum (\%i. i) 0 + sum (\%i. i) 0) = 0 * 0}
\ttbreak
{\out  2. !!n. n + (sum (\%i. i) n + sum (\%i. i) n) = n * n}
{\out           ==> Suc n + (sum (\%i. i) (Suc n) + sum (\%i.\,i) (Suc n)) =}
{\out               Suc n * Suc n}
\end{ttbox}
Simplification proves both subgoals immediately:\index{*ALLGOALS}
\begin{ttbox}
by (ALLGOALS Asm_simp_tac);
{\out Level 3}
{\out 2 * sum (\%i. i) (Suc n) = n * Suc n}
{\out No subgoals!}
\end{ttbox}
Simplification cannot prove the induction step if we omit \texttt{add_ac} from
the simpset.  Observe that like terms have not been collected:
\begin{ttbox}
{\out Level 3}
{\out 2 * sum (\%i. i) (Suc n) = n * Suc n}
{\out  1. !!n. n + sum (\%i. i) n + (n + sum (\%i. i) n) = n + n * n}
{\out           ==> n + (n + sum (\%i. i) n) + (n + (n + sum (\%i.\,i) n)) =}
{\out               n + (n + (n + n * n))}
\end{ttbox}
Ordered rewriting proves this by sorting the left-hand side.  Proving
arithmetic theorems without ordered rewriting requires explicit use of
commutativity.  This is tedious; try it and see!

Ordered rewriting is equally successful in proving
$\sum@{i=1}^n i^3 = n^2\times(n+1)^2/4$.


\subsection{Re-orienting equalities}
Ordered rewriting with the derived rule \texttt{symmetry} can reverse
equations:
\begin{ttbox}
val symmetry = prove_goal HOL.thy "(x=y) = (y=x)"
                 (fn _ => [Blast_tac 1]);
\end{ttbox}
This is frequently useful.  Assumptions of the form $s=t$, where $t$ occurs
in the conclusion but not~$s$, can often be brought into the right form.
For example, ordered rewriting with \texttt{symmetry} can prove the goal
\[ f(a)=b \conj f(a)=c \imp b=c. \]
Here \texttt{symmetry} reverses both $f(a)=b$ and $f(a)=c$
because $f(a)$ is lexicographically greater than $b$ and~$c$.  These
re-oriented equations, as rewrite rules, replace $b$ and~$c$ in the
conclusion by~$f(a)$. 

Another example is the goal $\neg(t=u) \imp \neg(u=t)$.
The differing orientations make this appear difficult to prove.  Ordered
rewriting with \texttt{symmetry} makes the equalities agree.  (Without
knowing more about~$t$ and~$u$ we cannot say whether they both go to $t=u$
or~$u=t$.)  Then the simplifier can prove the goal outright.

\index{rewrite rules!permutative|)}

%%% Local Variables: 
%%% mode: latex
%%% TeX-master: "ref"
%%% End: 
