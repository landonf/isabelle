\documentclass[12pt,a4paper,fleqn]{article}
\usepackage{latexsym}
\usepackage{graphicx}
\usepackage{iman}
\usepackage{extra}
\usepackage{isar}
\usepackage{isabelle}
\usepackage{isabellesym}
\usepackage{style}
\usepackage{pdfsetup}

\newcommand{\cmd}[1]{\isacommand{#1}}

\hyphenation{isa-belle}

\isadroptag{theory}

\title{%\includegraphics[scale=0.5]{isabelle_hol} \\[4ex]
Defining (Co)datatypes in Isabelle/HOL}
\author{\hbox{} \\
Jasmin Christian Blanchette, Andrei Popescu, and Dmitriy Traytel \\
{\normalsize Institut f\"ur Informatik, Technische Universit\"at M\"unchen} \\
\hbox{}}
\begin{document}

\maketitle

\begin{abstract}
\noindent
This tutorial describes how to use the new package for defining
datatypes and codatatypes in Isabelle/HOL. The package provides four
main user-level commands: \cmd{datatype\_new}, \cmd{codatatype}, \cmd{primrec\_new}, and
\cmd{primcorec}. The \cmd{\_new} commands are designed to subsume, and eventually
replace, the corresponding commands from the old datatype package.
\end{abstract}

\input{Datatypes.tex}

\let\em=\sl
\bibliography{manual}{}
\bibliographystyle{abbrv}

\end{document}
