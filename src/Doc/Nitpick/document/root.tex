\documentclass[a4paper,12pt]{article}
\usepackage[T1]{fontenc}
\usepackage{amsmath}
\usepackage{amssymb}
\usepackage{color}
\usepackage{footmisc}
\usepackage{graphicx}
%\usepackage{mathpazo}
\usepackage{multicol}
\usepackage{stmaryrd}
%\usepackage[scaled=.85]{beramono}
\usepackage{isabelle,iman,pdfsetup}

%\oddsidemargin=4.6mm
%\evensidemargin=4.6mm
%\textwidth=150mm
%\topmargin=4.6mm
%\headheight=0mm
%\headsep=0mm
%\textheight=234mm

\def\Colon{\mathord{:\mkern-1.5mu:}}
%\def\lbrakk{\mathopen{\lbrack\mkern-3.25mu\lbrack}}
%\def\rbrakk{\mathclose{\rbrack\mkern-3.255mu\rbrack}}
\def\lparr{\mathopen{(\mkern-4mu\mid}}
\def\rparr{\mathclose{\mid\mkern-4mu)}}

%\def\unk{{?}}
\def\unk{{\_}}
\def\unkef{(\lambda x.\; \unk)}
\def\undef{(\lambda x.\; \_)}
%\def\unr{\textit{others}}
\def\unr{\ldots}
\def\Abs#1{\hbox{\rm{\guillemetleft}}{\,#1\,}\hbox{\rm{\guillemetright}}}
\def\Q{{\smash{\lower.2ex\hbox{$\scriptstyle?$}}}}

\hyphenation{Mini-Sat size-change First-Steps grand-parent nit-pick
counter-example counter-examples data-type data-types co-data-type
co-data-types in-duc-tive co-in-duc-tive}

\urlstyle{tt}

\renewcommand\_{\hbox{\textunderscore\kern-.05ex}}

\begin{document}

%%% TYPESETTING
%\renewcommand\labelitemi{$\bullet$}
\renewcommand\labelitemi{\raise.065ex\hbox{\small\textbullet}}

\title{\includegraphics[scale=0.5]{isabelle_logo} \\[4ex]
Picking Nits \\[\smallskipamount]
\Large A User's Guide to Nitpick for Isabelle/HOL}
\author{\hbox{} \\
Jasmin Blanchette \\
{\normalsize Institut f\"ur Informatik, Technische Universit\"at M\"unchen} \\
\hbox{}}

\maketitle

\tableofcontents

\setlength{\parskip}{.7em plus .2em minus .1em}
\setlength{\parindent}{0pt}
\setlength{\abovedisplayskip}{\parskip}
\setlength{\abovedisplayshortskip}{.9\parskip}
\setlength{\belowdisplayskip}{\parskip}
\setlength{\belowdisplayshortskip}{.9\parskip}

% General-purpose enum environment with correct spacing
\newenvironment{enum}%
    {\begin{list}{}{%
        \setlength{\topsep}{.1\parskip}%
        \setlength{\partopsep}{.1\parskip}%
        \setlength{\itemsep}{\parskip}%
        \advance\itemsep by-\parsep}}
    {\end{list}}

\def\pre{\begingroup\vskip0pt plus1ex\advance\leftskip by\leftmargin
\advance\rightskip by\leftmargin}
\def\post{\vskip0pt plus1ex\endgroup}

\def\prew{\pre\advance\rightskip by-\leftmargin}
\def\postw{\post}

\section{Introduction}
\label{introduction}

Nitpick \cite{blanchette-nipkow-2010} is a counterexample generator for
Isabelle/HOL \cite{isa-tutorial} that is designed to handle formulas
combining (co)in\-duc\-tive datatypes, (co)in\-duc\-tively defined predicates, and
quantifiers. It builds on Kodkod \cite{torlak-jackson-2007}, a highly optimized
first-order relational model finder developed by the Software Design Group at
MIT. It is conceptually similar to Refute \cite{weber-2008}, from which it
borrows many ideas and code fragments, but it benefits from Kodkod's
optimizations and a new encoding scheme. The name Nitpick is shamelessly
appropriated from a now retired Alloy precursor.

Nitpick is easy to use---you simply enter \textbf{nitpick} after a putative
theorem and wait a few seconds. Nonetheless, there are situations where knowing
how it works under the hood and how it reacts to various options helps
increase the test coverage. This manual also explains how to install the tool on
your workstation. Should the motivation fail you, think of the many hours of
hard work Nitpick will save you. Proving non-theorems is \textsl{hard work}.

Another common use of Nitpick is to find out whether the axioms of a locale are
satisfiable, while the locale is being developed. To check this, it suffices to
write

\prew
\textbf{lemma}~``$\textit{False\/}$'' \\
\textbf{nitpick}~[\textit{show\_all}]
\postw

after the locale's \textbf{begin} keyword. To falsify \textit{False}, Nitpick
must find a model for the axioms. If it finds no model, we have an indication
that the axioms might be unsatisfiable.

Nitpick provides an automatic mode that can be enabled via the ``Auto Nitpick''
option under ``Plugins > Plugin Options > Isabelle > General'' in
Isabelle/jEdit. In this mode, Nitpick is run on every newly entered theorem.

\newbox\boxA
\setbox\boxA=\hbox{\texttt{nospam}}

\newcommand\authoremail{\texttt{jasmin.blan{\color{white}nospam}\kern-\wd\boxA{}chette@\allowbreak
inria\allowbreak .\allowbreak fr}}

To run Nitpick, you must also make sure that the theory \textit{Nitpick} is
imported---this is rarely a problem in practice since it is part of
\textit{Main}. The examples presented in this manual can be found
in Isabelle's \texttt{src/HOL/\allowbreak Nitpick\_\allowbreak Examples/\allowbreak Manual\_Nits.thy} theory.
The known bugs and limitations at the time of writing are listed in
\S\ref{known-bugs-and-limitations}. Comments and bug reports concerning
the tool or the manual should be directed to the author at \authoremail.

\vskip2.5\smallskipamount

\textbf{Acknowledgment.} The author would like to thank Mark Summerfield for
suggesting several textual improvements.
% and Perry James for reporting a typo.

\section{Installation}
\label{installation}

Nitpick is part of Isabelle, so you do not need to install it. It relies on a
third-party Kodkod front-end called Kodkodi, which in turn requires a Java
virtual machine. Both are provided as official Isabelle components.

%There are two main ways of installing Kodkodi:
%
%\begin{enum}
%\item[\labelitemi] If you installed an official Isabelle package,
%it should already include a properly setup version of Kodkodi.
%
%\item[\labelitemi] If you use a repository or snapshot version of Isabelle, you
%an official Isabelle package, you can download the Isabelle-aware Kodkodi package
%from \url{http://www21.in.tum.de/~blanchet/\#software}. Extract the archive, then add a
%line to your \texttt{\$ISABELLE\_HOME\_USER\slash etc\slash components}%
%\footnote{The variable \texttt{\$ISABELLE\_HOME\_USER} is set by Isabelle at
%startup. Its value can be retrieved by executing \texttt{isabelle}
%\texttt{getenv} \texttt{ISABELLE\_HOME\_USER} on the command line.}
%file with the absolute path to Kodkodi. For example, if the
%\texttt{components} file does not exist yet and you extracted Kodkodi to
%\texttt{/usr/local/kodkodi-1.5.2}, create it with the single line
%
%\prew
%\texttt{/usr/local/kodkodi-1.5.2}
%\postw
%
%(including an invisible newline character) in it.
%\end{enum}

To check whether Kodkodi is successfully installed, you can try out the example
in \S\ref{propositional-logic}.

\section{First Steps}
\label{first-steps}

This section introduces Nitpick by presenting small examples. If possible, you
should try out the examples on your workstation. Your theory file should start
as follows:

\prew
\textbf{theory}~\textit{Scratch} \\
\textbf{imports}~\textit{Main~Quotient\_Product~RealDef} \\
\textbf{begin}
\postw

The results presented here were obtained using the JNI (Java Native Interface)
version of MiniSat and with multithreading disabled to reduce nondeterminism.
This was done by adding the line

\prew
\textbf{nitpick\_params} [\textit{sat\_solver}~= \textit{MiniSat\_JNI}, \,\textit{max\_threads}~= 1]
\postw

after the \textbf{begin} keyword. The JNI version of MiniSat is bundled with
Kodkodi and is precompiled for Linux, Mac~OS~X, and Windows (Cygwin). Other SAT
solvers can also be used, as explained in \S\ref{optimizations}. If you
have already configured SAT solvers in Isabelle (e.g., for Refute), these will
also be available to Nitpick.

\subsection{Propositional Logic}
\label{propositional-logic}

Let's start with a trivial example from propositional logic:

\prew
\textbf{lemma}~``$P \longleftrightarrow Q$'' \\
\textbf{nitpick}
\postw

You should get the following output:

\prew
\slshape
Nitpick found a counterexample: \\[2\smallskipamount]
\hbox{}\qquad Free variables: \nopagebreak \\
\hbox{}\qquad\qquad $P = \textit{True}$ \\
\hbox{}\qquad\qquad $Q = \textit{False}$
\postw

Nitpick can also be invoked on individual subgoals, as in the example below:

\prew
\textbf{apply}~\textit{auto} \\[2\smallskipamount]
{\slshape goal (2 subgoals): \\
\phantom{0}1. $P\,\Longrightarrow\, Q$ \\
\phantom{0}2. $Q\,\Longrightarrow\, P$} \\[2\smallskipamount]
\textbf{nitpick}~1 \\[2\smallskipamount]
{\slshape Nitpick found a counterexample: \\[2\smallskipamount]
\hbox{}\qquad Free variables: \nopagebreak \\
\hbox{}\qquad\qquad $P = \textit{True}$ \\
\hbox{}\qquad\qquad $Q = \textit{False}$} \\[2\smallskipamount]
\textbf{nitpick}~2 \\[2\smallskipamount]
{\slshape Nitpick found a counterexample: \\[2\smallskipamount]
\hbox{}\qquad Free variables: \nopagebreak \\
\hbox{}\qquad\qquad $P = \textit{False}$ \\
\hbox{}\qquad\qquad $Q = \textit{True}$} \\[2\smallskipamount]
\textbf{oops}
\postw

\subsection{Type Variables}
\label{type-variables}

If you are left unimpressed by the previous example, don't worry. The next
one is more mind- and computer-boggling:

\prew
\textbf{lemma} ``$x \in A\,\Longrightarrow\, (\textrm{THE}~y.\;y \in A) \in A$''
\postw
\pagebreak[2] %% TYPESETTING

The putative lemma involves the definite description operator, {THE}, presented
in section 5.10.1 of the Isabelle tutorial \cite{isa-tutorial}. The
operator is defined by the axiom $(\textrm{THE}~x.\; x = a) = a$. The putative
lemma is merely asserting the indefinite description operator axiom with {THE}
substituted for {SOME}.

The free variable $x$ and the bound variable $y$ have type $'a$. For formulas
containing type variables, Nitpick enumerates the possible domains for each type
variable, up to a given cardinality (10 by default), looking for a finite
countermodel:

\prew
\textbf{nitpick} [\textit{verbose}] \\[2\smallskipamount]
\slshape
Trying 10 scopes: \nopagebreak \\
\hbox{}\qquad \textit{card}~$'a$~= 1; \\
\hbox{}\qquad \textit{card}~$'a$~= 2; \\
\hbox{}\qquad $\qquad\vdots$ \\[.5\smallskipamount]
\hbox{}\qquad \textit{card}~$'a$~= 10 \\[2\smallskipamount]
Nitpick found a counterexample for \textit{card} $'a$~= 3: \\[2\smallskipamount]
\hbox{}\qquad Free variables: \nopagebreak \\
\hbox{}\qquad\qquad $A = \{a_2,\, a_3\}$ \\
\hbox{}\qquad\qquad $x = a_3$ \\[2\smallskipamount]
Total time: 963 ms
\postw

Nitpick found a counterexample in which $'a$ has cardinality 3. (For
cardinalities 1 and 2, the formula holds.) In the counterexample, the three
values of type $'a$ are written $a_1$, $a_2$, and $a_3$.

The message ``Trying $n$ scopes: {\ldots}''\ is shown only if the option
\textit{verbose} is enabled. You can specify \textit{verbose} each time you
invoke \textbf{nitpick}, or you can set it globally using the command

\prew
\textbf{nitpick\_params} [\textit{verbose}]
\postw

This command also displays the current default values for all of the options
supported by Nitpick. The options are listed in \S\ref{option-reference}.

\subsection{Constants}
\label{constants}

By just looking at Nitpick's output, it might not be clear why the
counterexample in \S\ref{type-variables} is genuine. Let's invoke Nitpick again,
this time telling it to show the values of the constants that occur in the
formula:

\prew
\textbf{lemma} ``$x \in A\,\Longrightarrow\, (\textrm{THE}~y.\;y \in A) \in A$'' \\
\textbf{nitpick}~[\textit{show\_consts}] \\[2\smallskipamount]
\slshape
Nitpick found a counterexample for \textit{card} $'a$~= 3: \\[2\smallskipamount]
\hbox{}\qquad Free variables: \nopagebreak \\
\hbox{}\qquad\qquad $A = \{a_2,\, a_3\}$ \\
\hbox{}\qquad\qquad $x = a_3$ \\
\hbox{}\qquad Constant: \nopagebreak \\
\hbox{}\qquad\qquad $\hbox{\slshape THE}~y.\;y \in A = a_1$
\postw

As the result of an optimization, Nitpick directly assigned a value to the
subterm $\textrm{THE}~y.\;y \in A$, rather than to the \textit{The} constant. We
can disable this optimization by using the command

\prew
\textbf{nitpick}~[\textit{dont\_specialize},\, \textit{show\_consts}]
\postw

Our misadventures with THE suggest adding `$\exists!x{.}$' (``there exists a
unique $x$ such that'') at the front of our putative lemma's assumption:

\prew
\textbf{lemma} ``$\exists {!}x.\; x \in A\,\Longrightarrow\, (\textrm{THE}~y.\;y \in A) \in A$''
\postw

The fix appears to work:

\prew
\textbf{nitpick} \\[2\smallskipamount]
\slshape Nitpick found no counterexample
\postw

We can further increase our confidence in the formula by exhausting all
cardinalities up to 50:

\prew
\textbf{nitpick} [\textit{card} $'a$~= 1--50]\footnote{The symbol `--'
is entered as \texttt{-} (hyphen).} \\[2\smallskipamount]
\slshape Nitpick found no counterexample.
\postw

Let's see if Sledgehammer can find a proof:

\prew
\textbf{sledgehammer} \\[2\smallskipamount]
{\slshape Sledgehammer: ``$e$'' on goal \\
Try this: \textbf{by}~(\textit{metis~theI}) (42 ms)} \\
\hbox{}\qquad\vdots \\[2\smallskipamount]
\textbf{by}~(\textit{metis~theI\/})
\postw

This must be our lucky day.

\subsection{Skolemization}
\label{skolemization}

Are all invertible functions onto? Let's find out:

\prew
\textbf{lemma} ``$\exists g.\; \forall x.~g~(f~x) = x
 \,\Longrightarrow\, \forall y.\; \exists x.~y = f~x$'' \\
\textbf{nitpick} \\[2\smallskipamount]
\slshape
Nitpick found a counterexample for \textit{card} $'a$~= 2 and \textit{card} $'b$~=~1: \\[2\smallskipamount]
\hbox{}\qquad Free variable: \nopagebreak \\
\hbox{}\qquad\qquad $f = \undef{}(b_1 := a_1)$ \\
\hbox{}\qquad Skolem constants: \nopagebreak \\
\hbox{}\qquad\qquad $g = \undef{}(a_1 := b_1,\> a_2 := b_1)$ \\
\hbox{}\qquad\qquad $y = a_2$
\postw

(The Isabelle/HOL notation $f(x := y)$ denotes the function that maps $x$ to $y$
and that otherwise behaves like $f$.)
Although $f$ is the only free variable occurring in the formula, Nitpick also
displays values for the bound variables $g$ and $y$. These values are available
to Nitpick because it performs skolemization as a preprocessing step.

In the previous example, skolemization only affected the outermost quantifiers.
This is not always the case, as illustrated below:

\prew
\textbf{lemma} ``$\exists x.\; \forall f.\; f~x = x$'' \\
\textbf{nitpick} \\[2\smallskipamount]
\slshape
Nitpick found a counterexample for \textit{card} $'a$~= 2: \\[2\smallskipamount]
\hbox{}\qquad Skolem constant: \nopagebreak \\
\hbox{}\qquad\qquad $\lambda x.\; f =
    \undef{}(\!\begin{aligned}[t]
    & a_1 := \undef{}(a_1 := a_2,\> a_2 := a_1), \\[-2pt]
    & a_2 := \undef{}(a_1 := a_1,\> a_2 := a_1))\end{aligned}$
\postw

The variable $f$ is bound within the scope of $x$; therefore, $f$ depends on
$x$, as suggested by the notation $\lambda x.\,f$. If $x = a_1$, then $f$ is the
function that maps $a_1$ to $a_2$ and vice versa; otherwise, $x = a_2$ and $f$
maps both $a_1$ and $a_2$ to $a_1$. In both cases, $f~x \not= x$.

The source of the Skolem constants is sometimes more obscure:

\prew
\textbf{lemma} ``$\mathit{refl}~r\,\Longrightarrow\, \mathit{sym}~r$'' \\
\textbf{nitpick} \\[2\smallskipamount]
\slshape
Nitpick found a counterexample for \textit{card} $'a$~= 2: \\[2\smallskipamount]
\hbox{}\qquad Free variable: \nopagebreak \\
\hbox{}\qquad\qquad $r = \{(a_1, a_1),\, (a_2, a_1),\, (a_2, a_2)\}$ \\
\hbox{}\qquad Skolem constants: \nopagebreak \\
\hbox{}\qquad\qquad $\mathit{sym}.x = a_2$ \\
\hbox{}\qquad\qquad $\mathit{sym}.y = a_1$
\postw

What happened here is that Nitpick expanded \textit{sym} to its definition:

\prew
$\mathit{sym}~r \,\equiv\,
 \forall x\> y.\,\> (x, y) \in r \longrightarrow (y, x) \in r.$
\postw

As their names suggest, the Skolem constants $\mathit{sym}.x$ and
$\mathit{sym}.y$ are simply the bound variables $x$ and $y$
from \textit{sym}'s definition.

\subsection{Natural Numbers and Integers}
\label{natural-numbers-and-integers}

Because of the axiom of infinity, the type \textit{nat} does not admit any
finite models. To deal with this, Nitpick's approach is to consider finite
subsets $N$ of \textit{nat} and maps all numbers $\notin N$ to the undefined
value (displayed as `$\unk$'). The type \textit{int} is handled similarly.
Internally, undefined values lead to a three-valued logic.

Here is an example involving \textit{int\/}:

\prew
\textbf{lemma} ``$\lbrakk i \le j;\> n \le (m{\Colon}\mathit{int})\rbrakk \,\Longrightarrow\, i * n + j * m \le i * m + j * n$'' \\
\textbf{nitpick} \\[2\smallskipamount]
\slshape Nitpick found a counterexample: \\[2\smallskipamount]
\hbox{}\qquad Free variables: \nopagebreak \\
\hbox{}\qquad\qquad $i = 0$ \\
\hbox{}\qquad\qquad $j = 1$ \\
\hbox{}\qquad\qquad $m = 1$ \\
\hbox{}\qquad\qquad $n = 0$
\postw

Internally, Nitpick uses either a unary or a binary representation of numbers.
The unary representation is more efficient but only suitable for numbers very
close to zero. By default, Nitpick attempts to choose the more appropriate
encoding by inspecting the formula at hand. This behavior can be overridden by
passing either \textit{unary\_ints} or \textit{binary\_ints} as option. For
binary notation, the number of bits to use can be specified using
the \textit{bits} option. For example:

\prew
\textbf{nitpick} [\textit{binary\_ints}, \textit{bits}${} = 16$]
\postw

With infinite types, we don't always have the luxury of a genuine counterexample
and must often content ourselves with a potentially spurious one.
For example:

\prew
\textbf{lemma} ``$\forall n.\; \textit{Suc}~n \mathbin{\not=} n \,\Longrightarrow\, P$'' \\
\textbf{nitpick} [\textit{card~nat}~= 50] \\[2\smallskipamount]
\slshape Warning: The conjecture either trivially holds for the given scopes or lies outside Nitpick's supported
fragment; only potentially spurious counterexamples may be found \\[2\smallskipamount]
Nitpick found a potentially spurious counterexample: \\[2\smallskipamount]
\hbox{}\qquad Free variable: \nopagebreak \\
\hbox{}\qquad\qquad $P = \textit{False}$
\postw

The issue is that the bound variable in $\forall n.\;
\textit{Suc}~n \mathbin{\not=} n$ ranges over an infinite type. If Nitpick finds
an $n$ such that $\textit{Suc}~n \mathbin{=} n$, it evaluates the assumption to
\textit{False}; but otherwise, it does not know anything about values of $n \ge
\textit{card~nat}$ and must therefore evaluate the assumption to~$\unk$, not
\textit{True}. Since the assumption can never be fully satisfied by Nitpick,
the putative lemma can never be falsified.

Some conjectures involving elementary number theory make Nitpick look like a
giant with feet of clay:

\prew
\textbf{lemma} ``$P~\textit{Suc\/}$'' \\
\textbf{nitpick} \\[2\smallskipamount]
\slshape
Nitpick found no counterexample
\postw

On any finite set $N$, \textit{Suc} is a partial function; for example, if $N =
\{0, 1, \ldots, k\}$, then \textit{Suc} is $\{0 \mapsto 1,\, 1 \mapsto 2,\,
\ldots,\, k \mapsto \unk\}$, which evaluates to $\unk$ when passed as
argument to $P$. As a result, $P~\textit{Suc}$ is always $\unk$. The next
example is similar:

\prew
\textbf{lemma} ``$P~(\textit{op}~{+}\Colon
\textit{nat}\mathbin{\Rightarrow}\textit{nat}\mathbin{\Rightarrow}\textit{nat})$'' \\
\textbf{nitpick} [\textit{card nat} = 1] \\[2\smallskipamount]
{\slshape Nitpick found a counterexample:} \\[2\smallskipamount]
\hbox{}\qquad Free variable: \nopagebreak \\
\hbox{}\qquad\qquad $P = \unkef(\unkef(0 := \unkef(0 := 0)) := \mathit{False})$ \\[2\smallskipamount]
\textbf{nitpick} [\textit{card nat} = 2] \\[2\smallskipamount]
{\slshape Nitpick found no counterexample.}
\postw

The problem here is that \textit{op}~+ is total when \textit{nat} is taken to be
$\{0\}$ but becomes partial as soon as we add $1$, because
$1 + 1 \notin \{0, 1\}$.

Because numbers are infinite and are approximated using a three-valued logic,
there is usually no need to systematically enumerate domain sizes. If Nitpick
cannot find a genuine counterexample for \textit{card~nat}~= $k$, it is very
unlikely that one could be found for smaller domains. (The $P~(\textit{op}~{+})$
example above is an exception to this principle.) Nitpick nonetheless enumerates
all cardinalities from 1 to 10 for \textit{nat}, mainly because smaller
cardinalities are fast to handle and give rise to simpler counterexamples. This
is explained in more detail in \S\ref{scope-monotonicity}.

\subsection{Inductive Datatypes}
\label{inductive-datatypes}

Like natural numbers and integers, inductive datatypes with recursive
constructors admit no finite models and must be approximated by a subterm-closed
subset. For example, using a cardinality of 10 for ${'}a~\textit{list}$,
Nitpick looks for all counterexamples that can be built using at most 10
different lists.

Let's see with an example involving \textit{hd} (which returns the first element
of a list) and $@$ (which concatenates two lists):

\prew
\textbf{lemma} ``$\textit{hd}~(\textit{xs} \mathbin{@} [y, y]) = \textit{hd}~\textit{xs\/}$'' \\
\textbf{nitpick} \\[2\smallskipamount]
\slshape Nitpick found a counterexample for \textit{card} $'a$~= 3: \\[2\smallskipamount]
\hbox{}\qquad Free variables: \nopagebreak \\
\hbox{}\qquad\qquad $\textit{xs} = []$ \\
\hbox{}\qquad\qquad $\textit{y} = a_1$
\postw

To see why the counterexample is genuine, we enable \textit{show\_consts}
and \textit{show\_\allowbreak datatypes}:

\prew
{\slshape Type:} \\
\hbox{}\qquad $'a$~\textit{list}~= $\{[],\, [a_1],\, [a_1, a_1],\, \unr\}$ \\
{\slshape Constants:} \\
\hbox{}\qquad $\lambda x_1.\; x_1 \mathbin{@} [y, y] = \unkef([] := [a_1, a_1])$ \\
\hbox{}\qquad $\textit{hd} = \unkef([] := a_2,\> [a_1] := a_1,\> [a_1, a_1] := a_1)$
\postw

Since $\mathit{hd}~[]$ is undefined in the logic, it may be given any value,
including $a_2$.

The second constant, $\lambda x_1.\; x_1 \mathbin{@} [y, y]$, is simply the
append operator whose second argument is fixed to be $[y, y]$. Appending $[a_1,
a_1]$ to $[a_1]$ would normally give $[a_1, a_1, a_1]$, but this value is not
representable in the subset of $'a$~\textit{list} considered by Nitpick, which
is shown under the ``Type'' heading; hence the result is $\unk$. Similarly,
appending $[a_1, a_1]$ to itself gives $\unk$.

Given \textit{card}~$'a = 3$ and \textit{card}~$'a~\textit{list} = 3$, Nitpick
considers the following subsets:

\kern-.5\smallskipamount %% TYPESETTING

\prew
\begin{multicols}{3}
$\{[],\, [a_1],\, [a_2]\}$; \\
$\{[],\, [a_1],\, [a_3]\}$; \\
$\{[],\, [a_2],\, [a_3]\}$; \\
$\{[],\, [a_1],\, [a_1, a_1]\}$; \\
$\{[],\, [a_1],\, [a_2, a_1]\}$; \\
$\{[],\, [a_1],\, [a_3, a_1]\}$; \\
$\{[],\, [a_2],\, [a_1, a_2]\}$; \\
$\{[],\, [a_2],\, [a_2, a_2]\}$; \\
$\{[],\, [a_2],\, [a_3, a_2]\}$; \\
$\{[],\, [a_3],\, [a_1, a_3]\}$; \\
$\{[],\, [a_3],\, [a_2, a_3]\}$; \\
$\{[],\, [a_3],\, [a_3, a_3]\}$.
\end{multicols}
\postw

\kern-2\smallskipamount %% TYPESETTING

All subterm-closed subsets of $'a~\textit{list}$ consisting of three values
are listed and only those. As an example of a non-subterm-closed subset,
consider $\mathcal{S} = \{[],\, [a_1],\,\allowbreak [a_1, a_2]\}$, and observe
that $[a_1, a_2]$ (i.e., $a_1 \mathbin{\#} [a_2]$) has $[a_2] \notin
\mathcal{S}$ as a subterm.

Here's another m\"ochtegern-lemma that Nitpick can refute without a blink:

\prew
\textbf{lemma} ``$\lbrakk \textit{length}~\textit{xs} = 1;\> \textit{length}~\textit{ys} = 1
\rbrakk \,\Longrightarrow\, \textit{xs} = \textit{ys\/}$''
\\
\textbf{nitpick} [\textit{show\_types}] \\[2\smallskipamount]
\slshape Nitpick found a counterexample for \textit{card} $'a$~= 3: \\[2\smallskipamount]
\hbox{}\qquad Free variables: \nopagebreak \\
\hbox{}\qquad\qquad $\textit{xs} = [a_2]$ \\
\hbox{}\qquad\qquad $\textit{ys} = [a_1]$ \\
\hbox{}\qquad Types: \\
\hbox{}\qquad\qquad $\textit{nat} = \{0,\, 1,\, 2,\, \unr\}$ \\
\hbox{}\qquad\qquad $'a$~\textit{list} = $\{[],\, [a_1],\, [a_2],\, \unr\}$
\postw

Because datatypes are approximated using a three-valued logic, there is usually
no need to systematically enumerate cardinalities: If Nitpick cannot find a
genuine counterexample for \textit{card}~$'a~\textit{list}$~= 10, it is very
unlikely that one could be found for smaller cardinalities.

\subsection{Typedefs, Quotient Types, Records, Rationals, and Reals}
\label{typedefs-quotient-types-records-rationals-and-reals}

Nitpick generally treats types declared using \textbf{typedef} as datatypes
whose single constructor is the corresponding \textit{Abs\_\kern.1ex} function.
For example:

\prew
\textbf{typedef}~\textit{three} = ``$\{0\Colon\textit{nat},\, 1,\, 2\}$'' \\
\textbf{by}~\textit{blast} \\[2\smallskipamount]
\textbf{definition}~$A \mathbin{\Colon} \textit{three}$ \textbf{where} ``\kern-.1em$A \,\equiv\, \textit{Abs\_\allowbreak three}~0$'' \\
\textbf{definition}~$B \mathbin{\Colon} \textit{three}$ \textbf{where} ``$B \,\equiv\, \textit{Abs\_three}~1$'' \\
\textbf{definition}~$C \mathbin{\Colon} \textit{three}$ \textbf{where} ``$C \,\equiv\, \textit{Abs\_three}~2$'' \\[2\smallskipamount]
\textbf{lemma} ``$\lbrakk A \in X;\> B \in X\rbrakk \,\Longrightarrow\, c \in X$'' \\
\textbf{nitpick} [\textit{show\_types}] \\[2\smallskipamount]
\slshape Nitpick found a counterexample: \\[2\smallskipamount]
\hbox{}\qquad Free variables: \nopagebreak \\
\hbox{}\qquad\qquad $X = \{\Abs{0},\, \Abs{1}\}$ \\
\hbox{}\qquad\qquad $c = \Abs{2}$ \\
\hbox{}\qquad Types: \\
\hbox{}\qquad\qquad $\textit{nat} = \{0,\, 1,\, 2,\, \unr\}$ \\
\hbox{}\qquad\qquad $\textit{three} = \{\Abs{0},\, \Abs{1},\, \Abs{2},\, \unr\}$
\postw

In the output above, $\Abs{n}$ abbreviates $\textit{Abs\_three}~n$.

Quotient types are handled in much the same way. The following fragment defines
the integer type \textit{my\_int} by encoding the integer $x$ by a pair of
natural numbers $(m, n)$ such that $x + n = m$:

\prew
\textbf{fun} \textit{my\_int\_rel} \textbf{where} \\
``$\textit{my\_int\_rel}~(x,\, y)~(u,\, v) = (x + v = u + y)$'' \\[2\smallskipamount]
%
\textbf{quotient\_type}~\textit{my\_int} = ``$\textit{nat} \times \textit{nat\/}$''$\;{/}\;$\textit{my\_int\_rel} \\
\textbf{by}~(\textit{auto simp add\/}:\ \textit{equivp\_def fun\_eq\_iff}) \\[2\smallskipamount]
%
\textbf{definition}~\textit{add\_raw}~\textbf{where} \\
``$\textit{add\_raw} \,\equiv\, \lambda(x,\, y)~(u,\, v).\; (x + (u\Colon\textit{nat}), y + (v\Colon\textit{nat}))$'' \\[2\smallskipamount]
%
\textbf{quotient\_definition} ``$\textit{add\/}\Colon\textit{my\_int} \Rightarrow \textit{my\_int} \Rightarrow \textit{my\_int\/}$'' \textbf{is} \textit{add\_raw} \\[2\smallskipamount]
%
\textbf{lemma} ``$\textit{add}~x~y = \textit{add}~x~x$'' \\
\textbf{nitpick} [\textit{show\_types}] \\[2\smallskipamount]
\slshape Nitpick found a counterexample: \\[2\smallskipamount]
\hbox{}\qquad Free variables: \nopagebreak \\
\hbox{}\qquad\qquad $x = \Abs{(0,\, 0)}$ \\
\hbox{}\qquad\qquad $y = \Abs{(0,\, 1)}$ \\
\hbox{}\qquad Types: \\
\hbox{}\qquad\qquad $\textit{nat} = \{0,\, 1,\, \unr\}$ \\
\hbox{}\qquad\qquad $\textit{nat} \times \textit{nat}~[\textsl{boxed\/}] = \{(0,\, 0),\> (1,\, 0),\> \unr\}$ \\
\hbox{}\qquad\qquad $\textit{my\_int} = \{\Abs{(0,\, 0)},\> \Abs{(0,\, 1)},\> \unr\}$
\postw

The values $\Abs{(0,\, 0)}$ and $\Abs{(0,\, 1)}$ represent the
integers $0$ and $-1$, respectively. Other representants would have been
possible---e.g., $\Abs{(5,\, 5)}$ and $\Abs{(11,\, 12)}$. If we are going to
use \textit{my\_int} extensively, it pays off to install a term postprocessor
that converts the pair notation to the standard mathematical notation:

\prew
$\textbf{ML}~\,\{{*} \\
\!\begin{aligned}[t]
%& ({*}~\,\textit{Proof.context} \rightarrow \textit{string} \rightarrow (\textit{typ} \rightarrow \textit{term~list\/}) \rightarrow \textit{typ} \rightarrow \textit{term} \\[-2pt]
%& \phantom{(*}~\,{\rightarrow}\;\textit{term}~\,{*}) \\[-2pt]
& \textbf{fun}\,~\textit{my\_int\_postproc}~\_~\_~\_~T~(\textit{Const}~\_~\$~(\textit{Const}~\_~\$~\textit{t1}~\$~\textit{t2\/})) = {} \\[-2pt]
& \phantom{fun}\,~\textit{HOLogic.mk\_number}~T~(\textit{snd}~(\textit{HOLogic.dest\_number~t1}) \\[-2pt]
& \phantom{fun\,~\textit{HOLogic.mk\_number}~T~(}{-}~\textit{snd}~(\textit{HOLogic.dest\_number~t2\/})) \\[-2pt]
& \phantom{fun}\!{\mid}\,~\textit{my\_int\_postproc}~\_~\_~\_~\_~t = t \\[-2pt]
{*}\}\end{aligned}$ \\[2\smallskipamount]
$\textbf{declaration}~\,\{{*} \\
\!\begin{aligned}[t]
& \textit{Nitpick\_Model.register\_term\_postprocessor}~\!\begin{aligned}[t]
  & @\{\textrm{typ}~\textit{my\_int}\} \\[-2pt]
  & \textit{my\_int\_postproc}\end{aligned} \\[-2pt]
{*}\}\end{aligned}$
\postw

Records are handled as datatypes with a single constructor:

\prew
\textbf{record} \textit{point} = \\
\hbox{}\quad $\textit{Xcoord} \mathbin{\Colon} \textit{int}$ \\
\hbox{}\quad $\textit{Ycoord} \mathbin{\Colon} \textit{int}$ \\[2\smallskipamount]
\textbf{lemma} ``$\textit{Xcoord}~(p\Colon\textit{point}) = \textit{Xcoord}~(q\Colon\textit{point})$'' \\
\textbf{nitpick} [\textit{show\_types}] \\[2\smallskipamount]
\slshape Nitpick found a counterexample: \\[2\smallskipamount]
\hbox{}\qquad Free variables: \nopagebreak \\
\hbox{}\qquad\qquad $p = \lparr\textit{Xcoord} = 1,\> \textit{Ycoord} = 1\rparr$ \\
\hbox{}\qquad\qquad $q = \lparr\textit{Xcoord} = 0,\> \textit{Ycoord} = 0\rparr$ \\
\hbox{}\qquad Types: \\
\hbox{}\qquad\qquad $\textit{int} = \{0,\, 1,\, \unr\}$ \\
\hbox{}\qquad\qquad $\textit{point} = \{\!\begin{aligned}[t]
& \lparr\textit{Xcoord} = 0,\> \textit{Ycoord} = 0\rparr, \\[-2pt] %% TYPESETTING
& \lparr\textit{Xcoord} = 1,\> \textit{Ycoord} = 1\rparr,\, \unr\}\end{aligned}$
\postw

Finally, Nitpick provides rudimentary support for rationals and reals using a
similar approach:

\prew
\textbf{lemma} ``$4 * x + 3 * (y\Colon\textit{real}) \not= 1/2$'' \\
\textbf{nitpick} [\textit{show\_types}] \\[2\smallskipamount]
\slshape Nitpick found a counterexample: \\[2\smallskipamount]
\hbox{}\qquad Free variables: \nopagebreak \\
\hbox{}\qquad\qquad $x = 1/2$ \\
\hbox{}\qquad\qquad $y = -1/2$ \\
\hbox{}\qquad Types: \\
\hbox{}\qquad\qquad $\textit{nat} = \{0,\, 1,\, 2,\, 3,\, 4,\, 5,\, 6,\, 7,\, \unr\}$ \\
\hbox{}\qquad\qquad $\textit{int} = \{-3,\, -2,\, -1,\, 0,\, 1,\, 2,\, 3,\, 4,\, \unr\}$ \\
\hbox{}\qquad\qquad $\textit{real} = \{-3/2,\, -1/2,\, 0,\, 1/2,\, 1,\, 2,\, 3,\, 4,\, \unr\}$
\postw

\subsection{Inductive and Coinductive Predicates}
\label{inductive-and-coinductive-predicates}

Inductively defined predicates (and sets) are particularly problematic for
counterexample generators. They can make Quickcheck~\cite{berghofer-nipkow-2004}
loop forever and Refute~\cite{weber-2008} run out of resources. The crux of
the problem is that they are defined using a least fixed-point construction.

Nitpick's philosophy is that not all inductive predicates are equal. Consider
the \textit{even} predicate below:

\prew
\textbf{inductive}~\textit{even}~\textbf{where} \\
``\textit{even}~0'' $\,\mid$ \\
``\textit{even}~$n\,\Longrightarrow\, \textit{even}~(\textit{Suc}~(\textit{Suc}~n))$''
\postw

This predicate enjoys the desirable property of being well-founded, which means
that the introduction rules don't give rise to infinite chains of the form

\prew
$\cdots\,\Longrightarrow\, \textit{even}~k''
       \,\Longrightarrow\, \textit{even}~k'
       \,\Longrightarrow\, \textit{even}~k.$
\postw

For \textit{even}, this is obvious: Any chain ending at $k$ will be of length
$k/2 + 1$:

\prew
$\textit{even}~0\,\Longrightarrow\, \textit{even}~2\,\Longrightarrow\, \cdots
       \,\Longrightarrow\, \textit{even}~(k - 2)
       \,\Longrightarrow\, \textit{even}~k.$
\postw

Wellfoundedness is desirable because it enables Nitpick to use a very efficient
fixed-point computation.%
\footnote{If an inductive predicate is
well-founded, then it has exactly one fixed point, which is simultaneously the
least and the greatest fixed point. In these circumstances, the computation of
the least fixed point amounts to the computation of an arbitrary fixed point,
which can be performed using a straightforward recursive equation.}
Moreover, Nitpick can prove wellfoundedness of most well-founded predicates,
just as Isabelle's \textbf{function} package usually discharges termination
proof obligations automatically.

Let's try an example:

\prew
\textbf{lemma} ``$\exists n.\; \textit{even}~n \mathrel{\land} \textit{even}~(\textit{Suc}~n)$'' \\
\textbf{nitpick}~[\textit{card nat}~= 50, \textit{unary\_ints}, \textit{verbose}] \\[2\smallskipamount]
\slshape The inductive predicate ``\textit{even}'' was proved well-founded;
Nitpick can compute it efficiently \\[2\smallskipamount]
Trying 1 scope: \\
\hbox{}\qquad \textit{card nat}~= 50. \\[2\smallskipamount]
Warning: The conjecture either trivially holds for the given scopes or lies outside Nitpick's supported fragment; only
potentially spurious counterexamples may be found \\[2\smallskipamount]
Nitpick found a potentially spurious counterexample for \textit{card nat}~= 50: \\[2\smallskipamount]
\hbox{}\qquad Empty assignment \\[2\smallskipamount]
Nitpick could not find a better counterexample
It checked 1 of 1 scope \\[2\smallskipamount]
Total time: 1.62 s.
\postw

No genuine counterexample is possible because Nitpick cannot rule out the
existence of a natural number $n \ge 50$ such that both $\textit{even}~n$ and
$\textit{even}~(\textit{Suc}~n)$ are true. To help Nitpick, we can bound the
existential quantifier:

\prew
\textbf{lemma} ``$\exists n \mathbin{\le} 49.\; \textit{even}~n \mathrel{\land} \textit{even}~(\textit{Suc}~n)$'' \\
\textbf{nitpick}~[\textit{card nat}~= 50, \textit{unary\_ints}] \\[2\smallskipamount]
\slshape Nitpick found a counterexample: \\[2\smallskipamount]
\hbox{}\qquad Empty assignment
\postw

So far we were blessed by the wellfoundedness of \textit{even}. What happens if
we use the following definition instead?

\prew
\textbf{inductive} $\textit{even}'$ \textbf{where} \\
``$\textit{even}'~(0{\Colon}\textit{nat})$'' $\,\mid$ \\
``$\textit{even}'~2$'' $\,\mid$ \\
``$\lbrakk\textit{even}'~m;\> \textit{even}'~n\rbrakk \,\Longrightarrow\, \textit{even}'~(m + n)$''
\postw

This definition is not well-founded: From $\textit{even}'~0$ and
$\textit{even}'~0$, we can derive that $\textit{even}'~0$. Nonetheless, the
predicates $\textit{even}$ and $\textit{even}'$ are equivalent.

Let's check a property involving $\textit{even}'$. To make up for the
foreseeable computational hurdles entailed by non-wellfoundedness, we decrease
\textit{nat}'s cardinality to a mere 10:

\prew
\textbf{lemma}~``$\exists n \in \{0, 2, 4, 6, 8\}.\;
\lnot\;\textit{even}'~n$'' \\
\textbf{nitpick}~[\textit{card nat}~= 10,\, \textit{verbose},\, \textit{show\_consts}] \\[2\smallskipamount]
\slshape
The inductive predicate ``$\textit{even}'\!$'' could not be proved well-founded; Nitpick might need to unroll it \\[2\smallskipamount]
Trying 6 scopes: \\
\hbox{}\qquad \textit{card nat}~= 10 and \textit{iter} $\textit{even}'$~= 0; \\
\hbox{}\qquad \textit{card nat}~= 10 and \textit{iter} $\textit{even}'$~= 1; \\
\hbox{}\qquad \textit{card nat}~= 10 and \textit{iter} $\textit{even}'$~= 2; \\
\hbox{}\qquad \textit{card nat}~= 10 and \textit{iter} $\textit{even}'$~= 4; \\
\hbox{}\qquad \textit{card nat}~= 10 and \textit{iter} $\textit{even}'$~= 8; \\
\hbox{}\qquad \textit{card nat}~= 10 and \textit{iter} $\textit{even}'$~= 9 \\[2\smallskipamount]
Nitpick found a counterexample for \textit{card nat}~= 10 and \textit{iter} $\textit{even}'$~= 2: \\[2\smallskipamount]
\hbox{}\qquad Constant: \nopagebreak \\
\hbox{}\qquad\qquad $\lambda i.\; \textit{even}'$ = $\unkef(\!\begin{aligned}[t]
& 0 := \unkef(0 := \textit{True},\, 2 := \textit{True}),\, \\[-2pt]
& 1 := \unkef(0 := \textit{True},\, 2 := \textit{True},\, 4 := \textit{True}),\, \\[-2pt]
& 2 := \unkef(0 := \textit{True},\, 2 := \textit{True},\, 4 := \textit{True},\, \\[-2pt]
& \phantom{2 := \unkef(}6 := \textit{True},\, 8 := \textit{True}))\end{aligned}$ \\[2\smallskipamount]
Total time: 1.87 s.
\postw

Nitpick's output is very instructive. First, it tells us that the predicate is
unrolled, meaning that it is computed iteratively from the empty set. Then it
lists six scopes specifying different bounds on the numbers of iterations:\ 0,
1, 2, 4, 8, and~9.

The output also shows how each iteration contributes to $\textit{even}'$. The
notation $\lambda i.\; \textit{even}'$ indicates that the value of the
predicate depends on an iteration counter. Iteration 0 provides the basis
elements, $0$ and $2$. Iteration 1 contributes $4$ ($= 2 + 2$). Iteration 2
throws $6$ ($= 2 + 4 = 4 + 2$) and $8$ ($= 4 + 4$) into the mix. Further
iterations would not contribute any new elements.
The predicate $\textit{even}'$ evaluates to either \textit{True} or $\unk$,
never \textit{False}.

%Some values are marked with superscripted question
%marks~(`\lower.2ex\hbox{$^\Q$}'). These are the elements for which the
%predicate evaluates to $\unk$.

When unrolling a predicate, Nitpick tries 0, 1, 2, 4, 8, 12, 16, 20, 24, and 28
iterations. However, these numbers are bounded by the cardinality of the
predicate's domain. With \textit{card~nat}~= 10, no more than 9 iterations are
ever needed to compute the value of a \textit{nat} predicate. You can specify
the number of iterations using the \textit{iter} option, as explained in
\S\ref{scope-of-search}.

In the next formula, $\textit{even}'$ occurs both positively and negatively:

\prew
\textbf{lemma} ``$\textit{even}'~(n - 2) \,\Longrightarrow\, \textit{even}'~n$'' \\
\textbf{nitpick} [\textit{card nat} = 10, \textit{show\_consts}] \\[2\smallskipamount]
\slshape Nitpick found a counterexample: \\[2\smallskipamount]
\hbox{}\qquad Free variable: \nopagebreak \\
\hbox{}\qquad\qquad $n = 1$ \\
\hbox{}\qquad Constants: \nopagebreak \\
\hbox{}\qquad\qquad $\lambda i.\; \textit{even}'$ = $\unkef(\!\begin{aligned}[t]
& 0 := \unkef(0 := \mathit{True},\, 2 := \mathit{True}))\end{aligned}$  \\
\hbox{}\qquad\qquad $\textit{even}' \leq \unkef(\!\begin{aligned}[t]
& 0 := \mathit{True},\, 1 := \mathit{False},\, 2 := \mathit{True},\, \\[-2pt]
& 4 := \mathit{True},\, 6 := \mathit{True},\, 8 := \mathit{True})\end{aligned}$
\postw

Notice the special constraint $\textit{even}' \leq \ldots$ in the output, whose
right-hand side represents an arbitrary fixed point (not necessarily the least
one). It is used to falsify $\textit{even}'~n$. In contrast, the unrolled
predicate is used to satisfy $\textit{even}'~(n - 2)$.

Coinductive predicates are handled dually. For example:

\prew
\textbf{coinductive} \textit{nats} \textbf{where} \\
``$\textit{nats}~(x\Colon\textit{nat}) \,\Longrightarrow\, \textit{nats}~x$'' \\[2\smallskipamount]
\textbf{lemma} ``$\textit{nats} = (\lambda n.\; n \mathbin\in \{0, 1, 2, 3, 4\})$'' \\
\textbf{nitpick}~[\textit{card nat} = 10,\, \textit{show\_consts}] \\[2\smallskipamount]
\slshape Nitpick found a counterexample:
\\[2\smallskipamount]
\hbox{}\qquad Constants: \nopagebreak \\
\hbox{}\qquad\qquad $\lambda i.\; \textit{nats} = \unkef(0 := \unkef,\, 1 := \unkef,\, 2 := \unkef)$ \\
\hbox{}\qquad\qquad $\textit{nats} \geq \unkef(3 := \textit{True},\, 4 := \textit{False},\, 5 := \textit{True})$
\postw

As a special case, Nitpick uses Kodkod's transitive closure operator to encode
negative occurrences of non-well-founded ``linear inductive predicates,'' i.e.,
inductive predicates for which each the predicate occurs in at most one
assumption of each introduction rule. For example:

\prew
\textbf{inductive} \textit{odd} \textbf{where} \\
``$\textit{odd}~1$'' $\,\mid$ \\
``$\lbrakk \textit{odd}~m;\>\, \textit{even}~n\rbrakk \,\Longrightarrow\, \textit{odd}~(m + n)$'' \\[2\smallskipamount]
\textbf{lemma}~``$\textit{odd}~n \,\Longrightarrow\, \textit{odd}~(n - 2)$'' \\
\textbf{nitpick}~[\textit{card nat} = 4,\, \textit{show\_consts}] \\[2\smallskipamount]
\slshape Nitpick found a counterexample:
\\[2\smallskipamount]
\hbox{}\qquad Free variable: \nopagebreak \\
\hbox{}\qquad\qquad $n = 1$ \\
\hbox{}\qquad Constants: \nopagebreak \\
\hbox{}\qquad\qquad $\textit{even} = \unkef(0 := True, 1 := False, 2 := True, 3 := False)$ \\
\hbox{}\qquad\qquad $\textit{odd}_{\textsl{base}} = {}$ \\
\hbox{}\qquad\qquad\quad $\unkef(0 := \textit{False},\, 1 := \textit{True},\, 2 := \textit{False},\, 3 := \textit{False})$ \\
\hbox{}\qquad\qquad $\textit{odd}_{\textsl{step}} = \unkef$\\
\hbox{}\qquad\qquad\quad $(
\!\begin{aligned}[t]
& 0 := \unkef(0 := \textit{True},\, 1 := \textit{False},\, 2 := \textit{True},\, 3 := \textit{False}), \\[-2pt]
& 1 := \unkef(0 := \textit{False},\, 1 := \textit{True},\, 2 := \textit{False},\, 3 := \textit{True}), \\[-2pt]
& 2 := \unkef(0 := \textit{False},\, 1 := \textit{False},\, 2 := \textit{True},\, 3 := \textit{False}), \\[-2pt]
& 3 := \unkef(0 := \textit{False},\, 1 := \textit{False},\, 2 := \textit{False},\, 3 := \textit{True}))
\end{aligned}$ \\
\hbox{}\qquad\qquad $\textit{odd} \leq \unkef(0 := \textit{False},\, 1 := \textit{True},\, 2 := \textit{False},\, 3 := \textit{True})$
\postw

\noindent
In the output, $\textit{odd}_{\textrm{base}}$ represents the base elements and
$\textit{odd}_{\textrm{step}}$ is a transition relation that computes new
elements from known ones. The set $\textit{odd}$ consists of all the values
reachable through the reflexive transitive closure of
$\textit{odd}_{\textrm{step}}$ starting with any element from
$\textit{odd}_{\textrm{base}}$, namely 1 and 3. Using Kodkod's
transitive closure to encode linear predicates is normally either more thorough
or more efficient than unrolling (depending on the value of \textit{iter}), but
you can disable it by passing the \textit{dont\_star\_linear\_preds} option.

\subsection{Coinductive Datatypes}
\label{coinductive-datatypes}

A coinductive datatype is similar to an inductive datatype but
allows infinite objects. Thus, the infinite lists $\textit{ps}$ $=$ $[a, a, a,
\ldots]$, $\textit{qs}$ $=$ $[a, b, a, b, \ldots]$, and $\textit{rs}$ $=$ $[0,
1, 2, 3, \ldots]$ can be defined as coinductive lists, or ``lazy lists,'' using the
$\textit{LNil}\mathbin{\Colon}{'}a~\textit{llist}$ and
$\textit{LCons}\mathbin{\Colon}{'}a \mathbin{\Rightarrow} {'}a~\textit{llist}
\mathbin{\Rightarrow} {'}a~\textit{llist}$ constructors.

Although it is otherwise no friend of infinity, Nitpick can find counterexamples
involving cyclic lists such as \textit{ps} and \textit{qs} above as well as
finite lists:

\prew
\textbf{codatatype} $'a$ \textit{llist} = \textit{LNil}~$\mid$~\textit{LCons}~$'a$~``$'a\;\textit{llist}$'' \\[2\smallskipamount]
\textbf{lemma} ``$\textit{xs} \not= \textit{LCons}~a~\textit{xs\/}$'' \\
\textbf{nitpick} \\[2\smallskipamount]
\slshape Nitpick found a counterexample for {\itshape card}~$'a$ = 1: \\[2\smallskipamount]
\hbox{}\qquad Free variables: \nopagebreak \\
\hbox{}\qquad\qquad $\textit{a} = a_1$ \\
\hbox{}\qquad\qquad $\textit{xs} = \textsl{THE}~\omega.\; \omega = \textit{LCons}~a_1~\omega$
\postw

The notation $\textrm{THE}~\omega.\; \omega = t(\omega)$ stands
for the infinite term $t(t(t(\ldots)))$. Hence, \textit{xs} is simply the
infinite list $[a_1, a_1, a_1, \ldots]$.

The next example is more interesting:

\prew
\textbf{primcorec}~$\textit{iterates}$~\textbf{where} \\
``$\textit{iterates}~f\>a = \textit{LCons}~a~(\textit{iterates}~f\>(f\>a))$'' \\[2\smallskipamount]
\textbf{lemma}~``$\lbrakk\textit{xs} = \textit{LCons}~a~\textit{xs};\>\,
\textit{ys} = \textit{iterates}~(\lambda b.\> a)~b\rbrakk \,\Longrightarrow\, \textit{xs} = \textit{ys\/}$'' \\
\textbf{nitpick} [\textit{verbose}] \\[2\smallskipamount]
\slshape The type $'a$ passed the monotonicity test; Nitpick might be able to skip
some scopes \\[2\smallskipamount]
Trying 10 scopes: \\
\hbox{}\qquad \textit{card} $'a$~= 1, \textit{card} ``\kern1pt$'a~\textit{list\/}$''~= 1,
and \textit{bisim\_depth}~= 0; \\
\hbox{}\qquad $\qquad\vdots$ \\[.5\smallskipamount]
\hbox{}\qquad \textit{card} $'a$~= 10, \textit{card} ``\kern1pt$'a~\textit{list\/}$''~= 10,
and \textit{bisim\_depth}~= 9 \\[2\smallskipamount]
Nitpick found a counterexample for {\itshape card}~$'a$ = 2,
\textit{card}~``\kern1pt$'a~\textit{llist\/}$''~= 2, and \textit{bisim\_\allowbreak
depth}~= 1:
\\[2\smallskipamount]
\hbox{}\qquad Free variables: \nopagebreak \\
\hbox{}\qquad\qquad $\textit{a} = a_1$ \\
\hbox{}\qquad\qquad $\textit{b} = a_2$ \\
\hbox{}\qquad\qquad $\textit{xs} = \textsl{THE}~\omega.\; \omega = \textit{LCons}~a_1~\omega$ \\
\hbox{}\qquad\qquad $\textit{ys} = \textit{LCons}~a_2~(\textsl{THE}~\omega.\; \omega = \textit{LCons}~a_1~\omega)$ \\[2\smallskipamount]
Total time: 1.11 s
\postw

The lazy list $\textit{xs}$ is simply $[a_1, a_1, a_1, \ldots]$, whereas
$\textit{ys}$ is $[a_2, a_1, a_1, a_1, \ldots]$, i.e., a lasso-shaped list with
$[a_2]$ as its stem and $[a_1]$ as its cycle. In general, the list segment
within the scope of the {THE} binder corresponds to the lasso's cycle, whereas
the segment leading to the binder is the stem.

A salient property of coinductive datatypes is that two objects are considered
equal if and only if they lead to the same observations. For example, the two
lazy lists
%
\begin{gather*}
\textrm{THE}~\omega.\; \omega = \textit{LCons}~a~(\textit{LCons}~b~\omega) \\
\textit{LCons}~a~(\textrm{THE}~\omega.\; \omega = \textit{LCons}~b~(\textit{LCons}~a~\omega))
\end{gather*}
%
are identical, because both lead
to the sequence of observations $a$, $b$, $a$, $b$, \hbox{\ldots} (or,
equivalently, both encode the infinite list $[a, b, a, b, \ldots]$). This
concept of equality for coinductive datatypes is called bisimulation and is
defined coinductively.

Internally, Nitpick encodes the coinductive bisimilarity predicate as part of
the Kodkod problem to ensure that distinct objects lead to different
observations. This precaution is somewhat expensive and often unnecessary, so it
can be disabled by setting the \textit{bisim\_depth} option to $-1$. The
bisimilarity check is then performed \textsl{after} the counterexample has been
found to ensure correctness. If this after-the-fact check fails, the
counterexample is tagged as ``quasi genuine'' and Nitpick recommends to try
again with \textit{bisim\_depth} set to a nonnegative integer.

The next formula illustrates the need for bisimilarity (either as a Kodkod
predicate or as an after-the-fact check) to prevent spurious counterexamples:

\prew
\textbf{lemma} ``$\lbrakk xs = \textit{LCons}~a~\textit{xs};\>\, \textit{ys} = \textit{LCons}~a~\textit{ys}\rbrakk
\,\Longrightarrow\, \textit{xs} = \textit{ys\/}$'' \\
\textbf{nitpick} [\textit{bisim\_depth} = $-1$, \textit{show\_types}] \\[2\smallskipamount]
\slshape Nitpick found a quasi genuine counterexample for $\textit{card}~'a$ = 2: \\[2\smallskipamount]
\hbox{}\qquad Free variables: \nopagebreak \\
\hbox{}\qquad\qquad $a = a_1$ \\
\hbox{}\qquad\qquad $\textit{xs} = \textsl{THE}~\omega.\; \omega =
\textit{LCons}~a_1~\omega$ \\
\hbox{}\qquad\qquad $\textit{ys} = \textsl{THE}~\omega.\; \omega = \textit{LCons}~a_1~\omega$ \\
\hbox{}\qquad Type:\strut \nopagebreak \\
\hbox{}\qquad\qquad $'a~\textit{llist} =
\{\!\begin{aligned}[t]
  & \textsl{THE}~\omega.\; \omega = \textit{LCons}~a_1~\omega, \\[-2pt]
  & \textsl{THE}~\omega.\; \omega = \textit{LCons}~a_1~\omega,\> \unr\}\end{aligned}$
\\[2\smallskipamount]
Try again with ``\textit{bisim\_depth}'' set to a nonnegative value to confirm
that the counterexample is genuine \\[2\smallskipamount]
{\upshape\textbf{nitpick}} \\[2\smallskipamount]
\slshape Nitpick found no counterexample
\postw

In the first \textbf{nitpick} invocation, the after-the-fact check discovered
that the two known elements of type $'a~\textit{llist}$ are bisimilar, prompting
Nitpick to label the example as only ``quasi genuine.''

A compromise between leaving out the bisimilarity predicate from the Kodkod
problem and performing the after-the-fact check is to specify a low
nonnegative \textit{bisim\_depth} value. In general, a value of $K$ means that
Nitpick will require all lists to be distinguished from each other by their
prefixes of length $K$. However, setting $K$ to a too low value can
overconstrain Nitpick, preventing it from finding any counterexamples.

\subsection{Boxing}
\label{boxing}

Nitpick normally maps function and product types directly to the corresponding
Kodkod concepts. As a consequence, if $'a$ has cardinality 3 and $'b$ has
cardinality 4, then $'a \times {'}b$ has cardinality 12 ($= 4 \times 3$) and $'a
\Rightarrow {'}b$ has cardinality 64 ($= 4^3$). In some circumstances, it pays
off to treat these types in the same way as plain datatypes, by approximating
them by a subset of a given cardinality. This technique is called ``boxing'' and
is particularly useful for functions passed as arguments to other functions, for
high-arity functions, and for large tuples. Under the hood, boxing involves
wrapping occurrences of the types $'a \times {'}b$ and $'a \Rightarrow {'}b$ in
isomorphic datatypes, as can be seen by enabling the \textit{debug} option.

To illustrate boxing, we consider a formalization of $\lambda$-terms represented
using de Bruijn's notation:

\prew
\textbf{datatype} \textit{tm} = \textit{Var}~\textit{nat}~$\mid$~\textit{Lam}~\textit{tm} $\mid$ \textit{App~tm~tm}
\postw

The $\textit{lift}~t~k$ function increments all variables with indices greater
than or equal to $k$ by one:

\prew
\textbf{primrec} \textit{lift} \textbf{where} \\
``$\textit{lift}~(\textit{Var}~j)~k = \textit{Var}~(\textrm{if}~j < k~\textrm{then}~j~\textrm{else}~j + 1)$'' $\mid$ \\
``$\textit{lift}~(\textit{Lam}~t)~k = \textit{Lam}~(\textit{lift}~t~(k + 1))$'' $\mid$ \\
``$\textit{lift}~(\textit{App}~t~u)~k = \textit{App}~(\textit{lift}~t~k)~(\textit{lift}~u~k)$''
\postw

The $\textit{loose}~t~k$ predicate returns \textit{True} if and only if
term $t$ has a loose variable with index $k$ or more:

\prew
\textbf{primrec}~\textit{loose} \textbf{where} \\
``$\textit{loose}~(\textit{Var}~j)~k = (j \ge k)$'' $\mid$ \\
``$\textit{loose}~(\textit{Lam}~t)~k = \textit{loose}~t~(\textit{Suc}~k)$'' $\mid$ \\
``$\textit{loose}~(\textit{App}~t~u)~k = (\textit{loose}~t~k \mathrel{\lor} \textit{loose}~u~k)$''
\postw

Next, the $\textit{subst}~\sigma~t$ function applies the substitution $\sigma$
on $t$:

\prew
\textbf{primrec}~\textit{subst} \textbf{where} \\
``$\textit{subst}~\sigma~(\textit{Var}~j) = \sigma~j$'' $\mid$ \\
``$\textit{subst}~\sigma~(\textit{Lam}~t) = {}$\phantom{''} \\
\phantom{``}$\textit{Lam}~(\textit{subst}~(\lambda n.\> \textrm{case}~n~\textrm{of}~0 \Rightarrow \textit{Var}~0 \mid \textit{Suc}~m \Rightarrow \textit{lift}~(\sigma~m)~1)~t)$'' $\mid$ \\
``$\textit{subst}~\sigma~(\textit{App}~t~u) = \textit{App}~(\textit{subst}~\sigma~t)~(\textit{subst}~\sigma~u)$''
\postw

A substitution is a function that maps variable indices to terms. Observe that
$\sigma$ is a function passed as argument and that Nitpick can't optimize it
away, because the recursive call for the \textit{Lam} case involves an altered
version. Also notice the \textit{lift} call, which increments the variable
indices when moving under a \textit{Lam}.

A reasonable property to expect of substitution is that it should leave closed
terms unchanged. Alas, even this simple property does not hold:

\pre
\textbf{lemma}~``$\lnot\,\textit{loose}~t~0 \,\Longrightarrow\, \textit{subst}~\sigma~t = t$'' \\
\textbf{nitpick} [\textit{verbose}] \\[2\smallskipamount]
\slshape
Trying 10 scopes: \nopagebreak \\
\hbox{}\qquad \textit{card~nat}~= 1, \textit{card tm}~= 1, and \textit{card} ``$\textit{nat} \Rightarrow \textit{tm\/}$'' = 1; \\
\hbox{}\qquad \textit{card~nat}~= 2, \textit{card tm}~= 2, and \textit{card} ``$\textit{nat} \Rightarrow \textit{tm\/}$'' = 2; \\
\hbox{}\qquad $\qquad\vdots$ \\[.5\smallskipamount]
\hbox{}\qquad \textit{card~nat}~= 10, \textit{card tm}~= 10, and \textit{card} ``$\textit{nat} \Rightarrow \textit{tm\/}$'' = 10 \\[2\smallskipamount]
Nitpick found a counterexample for \textit{card~nat}~= 6, \textit{card~tm}~= 6,
and \textit{card}~``$\textit{nat} \Rightarrow \textit{tm\/}$''~= 6: \\[2\smallskipamount]
\hbox{}\qquad Free variables: \nopagebreak \\
\hbox{}\qquad\qquad $\sigma = \unkef(\!\begin{aligned}[t]
& 0 := \textit{Var}~0,\>
  1 := \textit{Var}~0,\>
  2 := \textit{Var}~0, \\[-2pt]
& 3 := \textit{Var}~0,\>
  4 := \textit{Var}~0,\>
  5 := \textit{Lam}~(\textit{Lam}~(\textit{Var}~0)))\end{aligned}$ \\
\hbox{}\qquad\qquad $t = \textit{Lam}~(\textit{Lam}~(\textit{Var}~1))$ \\[2\smallskipamount]
Total time: 3.08 s
\postw

Using \textit{eval}, we find out that $\textit{subst}~\sigma~t =
\textit{Lam}~(\textit{Lam}~(\textit{Var}~0))$. Using the traditional
$\lambda$-calculus notation, $t$ is
$\lambda x\, y.\> x$ whereas $\textit{subst}~\sigma~t$ is (wrongly) $\lambda x\, y.\> y$.
The bug is in \textit{subst\/}: The $\textit{lift}~(\sigma~m)~1$ call should be
replaced with $\textit{lift}~(\sigma~m)~0$.

An interesting aspect of Nitpick's verbose output is that it assigned inceasing
cardinalities from 1 to 10 to the type $\textit{nat} \Rightarrow \textit{tm}$
of the higher-order argument $\sigma$ of \textit{subst}.
For the formula of interest, knowing 6 values of that type was enough to find
the counterexample. Without boxing, $6^6 = 46\,656$ values must be
considered, a hopeless undertaking:

\prew
\textbf{nitpick} [\textit{dont\_box}] \\[2\smallskipamount]
{\slshape Nitpick ran out of time after checking 3 of 10 scopes}
\postw

Boxing can be enabled or disabled globally or on a per-type basis using the
\textit{box} option. Nitpick usually performs reasonable choices about which
types should be boxed, but option tweaking sometimes helps.

%A related optimization,
%``finitization,'' attempts to wrap functions that are constant at all but finitely
%many points (e.g., finite sets); see the documentation for the \textit{finitize}
%option in \S\ref{scope-of-search} for details.

\subsection{Scope Monotonicity}
\label{scope-monotonicity}

The \textit{card} option (together with \textit{iter}, \textit{bisim\_depth},
and \textit{max}) controls which scopes are actually tested. In general, to
exhaust all models below a certain cardinality bound, the number of scopes that
Nitpick must consider increases exponentially with the number of type variables
(and \textbf{typedecl}'d types) occurring in the formula. Given the default
cardinality specification of 1--10, no fewer than $10^4 = 10\,000$ scopes must be
considered for a formula involving $'a$, $'b$, $'c$, and $'d$.

Fortunately, many formulas exhibit a property called \textsl{scope
monotonicity}, meaning that if the formula is falsifiable for a given scope,
it is also falsifiable for all larger scopes \cite[p.~165]{jackson-2006}.

Consider the formula

\prew
\textbf{lemma}~``$\textit{length~xs} = \textit{length~ys} \,\Longrightarrow\, \textit{rev}~(\textit{zip~xs~ys}) = \textit{zip~xs}~(\textit{rev~ys})$''
\postw

where \textit{xs} is of type $'a~\textit{list}$ and \textit{ys} is of type
$'b~\textit{list}$. A priori, Nitpick would need to consider $1\,000$ scopes to
exhaust the specification \textit{card}~= 1--10 (10 cardinalies for $'a$
$\times$ 10 cardinalities for $'b$ $\times$ 10 cardinalities for the datatypes).
However, our intuition tells us that any counterexample found with a small scope
would still be a counterexample in a larger scope---by simply ignoring the fresh
$'a$ and $'b$ values provided by the larger scope. Nitpick comes to the same
conclusion after a careful inspection of the formula and the relevant
definitions:

\prew
\textbf{nitpick}~[\textit{verbose}] \\[2\smallskipamount]
\slshape
The types $'a$ and $'b$ passed the monotonicity test; Nitpick might be able to skip some scopes.
 \\[2\smallskipamount]
Trying 10 scopes: \\
\hbox{}\qquad \textit{card} $'a$~= 1, \textit{card} $'b$~= 1,
\textit{card} \textit{nat}~= 1, \textit{card} ``$('a \times {'}b)$
\textit{list\/}''~= 1, \\
\hbox{}\qquad\quad \textit{card} ``\kern1pt$'a$ \textit{list\/}''~= 1, and
\textit{card} ``\kern1pt$'b$ \textit{list\/}''~= 1; \\
\hbox{}\qquad \textit{card} $'a$~= 2, \textit{card} $'b$~= 2,
\textit{card} \textit{nat}~= 2, \textit{card} ``$('a \times {'}b)$
\textit{list\/}''~= 2, \\
\hbox{}\qquad\quad \textit{card} ``\kern1pt$'a$ \textit{list\/}''~= 2, and
\textit{card} ``\kern1pt$'b$ \textit{list\/}''~= 2; \\
\hbox{}\qquad $\qquad\vdots$ \\[.5\smallskipamount]
\hbox{}\qquad \textit{card} $'a$~= 10, \textit{card} $'b$~= 10,
\textit{card} \textit{nat}~= 10, \textit{card} ``$('a \times {'}b)$
\textit{list\/}''~= 10, \\
\hbox{}\qquad\quad \textit{card} ``\kern1pt$'a$ \textit{list\/}''~= 10, and
\textit{card} ``\kern1pt$'b$ \textit{list\/}''~= 10
\\[2\smallskipamount]
Nitpick found a counterexample for
\textit{card} $'a$~= 5, \textit{card} $'b$~= 5,
\textit{card} \textit{nat}~= 5, \textit{card} ``$('a \times {'}b)$
\textit{list\/}''~= 5, \textit{card} ``\kern1pt$'a$ \textit{list\/}''~= 5, and
\textit{card} ``\kern1pt$'b$ \textit{list\/}''~= 5:
\\[2\smallskipamount]
\hbox{}\qquad Free variables: \nopagebreak \\
\hbox{}\qquad\qquad $\textit{xs} = [a_1, a_2]$ \\
\hbox{}\qquad\qquad $\textit{ys} = [b_1, b_1]$ \\[2\smallskipamount]
Total time: 1.63 s.
\postw

In theory, it should be sufficient to test a single scope:

\prew
\textbf{nitpick}~[\textit{card}~= 10]
\postw

However, this is often less efficient in practice and may lead to overly complex
counterexamples.

If the monotonicity check fails but we believe that the formula is monotonic (or
we don't mind missing some counterexamples), we can pass the
\textit{mono} option. To convince yourself that this option is risky,
simply consider this example from \S\ref{skolemization}:

\prew
\textbf{lemma} ``$\exists g.\; \forall x\Colon 'b.~g~(f~x) = x
 \,\Longrightarrow\, \forall y\Colon {'}a.\; \exists x.~y = f~x$'' \\
\textbf{nitpick} [\textit{mono}] \\[2\smallskipamount]
{\slshape Nitpick found no counterexample} \\[2\smallskipamount]
\textbf{nitpick} \\[2\smallskipamount]
\slshape
Nitpick found a counterexample for \textit{card} $'a$~= 2 and \textit{card} $'b$~=~1: \\
\hbox{}\qquad $\vdots$
\postw

(It turns out the formula holds if and only if $\textit{card}~'a \le
\textit{card}~'b$.) Although this is rarely advisable, the automatic
monotonicity checks can be disabled by passing \textit{non\_mono}
(\S\ref{optimizations}).

As insinuated in \S\ref{natural-numbers-and-integers} and
\S\ref{inductive-datatypes}, \textit{nat}, \textit{int}, and inductive datatypes
are normally monotonic and treated as such. The same is true for record types,
\textit{rat}, and \textit{real}. Thus, given the
cardinality specification 1--10, a formula involving \textit{nat}, \textit{int},
\textit{int~list}, \textit{rat}, and \textit{rat~list} will lead Nitpick to
consider only 10~scopes instead of $10^4 = 10\,000$. On the other hand,
\textbf{typedef}s and quotient types are generally nonmonotonic.

\subsection{Inductive Properties}
\label{inductive-properties}

Inductive properties are a particular pain to prove, because the failure to
establish an induction step can mean several things:
%
\begin{enumerate}
\item The property is invalid.
\item The property is valid but is too weak to support the induction step.
\item The property is valid and strong enough; it's just that we haven't found
the proof yet.
\end{enumerate}
%
Depending on which scenario applies, we would take the appropriate course of
action:
%
\begin{enumerate}
\item Repair the statement of the property so that it becomes valid.
\item Generalize the property and/or prove auxiliary properties.
\item Work harder on a proof.
\end{enumerate}
%
How can we distinguish between the three scenarios? Nitpick's normal mode of
operation can often detect scenario 1, and Isabelle's automatic tactics help with
scenario 3. Using appropriate techniques, it is also often possible to use
Nitpick to identify scenario 2. Consider the following transition system,
in which natural numbers represent states:

\prew
\textbf{inductive\_set}~\textit{reach}~\textbf{where} \\
``$(4\Colon\textit{nat}) \in \textit{reach\/}$'' $\mid$ \\
``$\lbrakk n < 4;\> n \in \textit{reach\/}\rbrakk \,\Longrightarrow\, 3 * n + 1 \in \textit{reach\/}$'' $\mid$ \\
``$n \in \textit{reach} \,\Longrightarrow n + 2 \in \textit{reach\/}$''
\postw

We will try to prove that only even numbers are reachable:

\prew
\textbf{lemma}~``$n \in \textit{reach} \,\Longrightarrow\, 2~\textrm{dvd}~n$''
\postw

Does this property hold? Nitpick cannot find a counterexample within 30 seconds,
so let's attempt a proof by induction:

\prew
\textbf{apply}~(\textit{induct~set}{:}~\textit{reach\/}) \\
\textbf{apply}~\textit{auto}
\postw

This leaves us in the following proof state:

\prew
{\slshape goal (2 subgoals): \\
\phantom{0}1. ${\bigwedge}n.\;\, \lbrakk n \in \textit{reach\/};\, n < 4;\, 2~\textsl{dvd}~n\rbrakk \,\Longrightarrow\, 2~\textsl{dvd}~\textit{Suc}~(3 * n)$ \\
\phantom{0}2. ${\bigwedge}n.\;\, \lbrakk n \in \textit{reach\/};\, 2~\textsl{dvd}~n\rbrakk \,\Longrightarrow\, 2~\textsl{dvd}~\textit{Suc}~(\textit{Suc}~n)$
}
\postw

If we run Nitpick on the first subgoal, it still won't find any
counterexample; and yet, \textit{auto} fails to go further, and \textit{arith}
is helpless. However, notice the $n \in \textit{reach}$ assumption, which
strengthens the induction hypothesis but is not immediately usable in the proof.
If we remove it and invoke Nitpick, this time we get a counterexample:

\prew
\textbf{apply}~(\textit{thin\_tac}~``$n \in \textit{reach\/}$'') \\
\textbf{nitpick} \\[2\smallskipamount]
\slshape Nitpick found a counterexample: \\[2\smallskipamount]
\hbox{}\qquad Skolem constant: \nopagebreak \\
\hbox{}\qquad\qquad $n = 0$
\postw

Indeed, 0 < 4, 2 divides 0, but 2 does not divide 1. We can use this information
to strength the lemma:

\prew
\textbf{lemma}~``$n \in \textit{reach} \,\Longrightarrow\, 2~\textrm{dvd}~n \mathrel{\lor} n \not= 0$''
\postw

Unfortunately, the proof by induction still gets stuck, except that Nitpick now
finds the counterexample $n = 2$. We generalize the lemma further to

\prew
\textbf{lemma}~``$n \in \textit{reach} \,\Longrightarrow\, 2~\textrm{dvd}~n \mathrel{\lor} n \ge 4$''
\postw

and this time \textit{arith} can finish off the subgoals.

\section{Case Studies}
\label{case-studies}

As a didactic device, the previous section focused mostly on toy formulas whose
validity can easily be assessed just by looking at the formula. We will now
review two somewhat more realistic case studies that are within Nitpick's
reach:\ a context-free grammar modeled by mutually inductive sets and a
functional implementation of AA trees. The results presented in this
section were produced with the following settings:

\prew
\textbf{nitpick\_params} [\textit{max\_potential}~= 0]
\postw

\subsection{A Context-Free Grammar}
\label{a-context-free-grammar}

Our first case study is taken from section 7.4 in the Isabelle tutorial
\cite{isa-tutorial}. The following grammar, originally due to Hopcroft and
Ullman, produces all strings with an equal number of $a$'s and $b$'s:

\prew
\begin{tabular}{@{}r@{$\;\,$}c@{$\;\,$}l@{}}
$S$ & $::=$ & $\epsilon \mid bA \mid aB$ \\
$A$ & $::=$ & $aS \mid bAA$ \\
$B$ & $::=$ & $bS \mid aBB$
\end{tabular}
\postw

The intuition behind the grammar is that $A$ generates all strings with one more
$a$ than $b$'s and $B$ generates all strings with one more $b$ than $a$'s.

The alphabet consists exclusively of $a$'s and $b$'s:

\prew
\textbf{datatype} \textit{alphabet}~= $a$ $\mid$ $b$
\postw

Strings over the alphabet are represented by \textit{alphabet list}s.
Nonterminals in the grammar become sets of strings. The production rules
presented above can be expressed as a mutually inductive definition:

\prew
\textbf{inductive\_set} $S$ \textbf{and} $A$ \textbf{and} $B$ \textbf{where} \\
\textit{R1}:\kern.4em ``$[] \in S$'' $\,\mid$ \\
\textit{R2}:\kern.4em ``$w \in A\,\Longrightarrow\, b \mathbin{\#} w \in S$'' $\,\mid$ \\
\textit{R3}:\kern.4em ``$w \in B\,\Longrightarrow\, a \mathbin{\#} w \in S$'' $\,\mid$ \\
\textit{R4}:\kern.4em ``$w \in S\,\Longrightarrow\, a \mathbin{\#} w \in A$'' $\,\mid$ \\
\textit{R5}:\kern.4em ``$w \in S\,\Longrightarrow\, b \mathbin{\#} w \in S$'' $\,\mid$ \\
\textit{R6}:\kern.4em ``$\lbrakk v \in B;\> v \in B\rbrakk \,\Longrightarrow\, a \mathbin{\#} v \mathbin{@} w \in B$''
\postw

The conversion of the grammar into the inductive definition was done manually by
Joe Blow, an underpaid undergraduate student. As a result, some errors might
have sneaked in.

Debugging faulty specifications is at the heart of Nitpick's \textsl{raison
d'\^etre}. A good approach is to state desirable properties of the specification
(here, that $S$ is exactly the set of strings over $\{a, b\}$ with as many $a$'s
as $b$'s) and check them with Nitpick. If the properties are correctly stated,
counterexamples will point to bugs in the specification. For our grammar
example, we will proceed in two steps, separating the soundness and the
completeness of the set $S$. First, soundness:

\prew
\textbf{theorem}~\textit{S\_sound\/}: \\
``$w \in S \longrightarrow \textit{length}~[x\mathbin{\leftarrow} w.\; x = a] =
  \textit{length}~[x\mathbin{\leftarrow} w.\; x = b]$'' \\
\textbf{nitpick} \\[2\smallskipamount]
\slshape Nitpick found a counterexample: \\[2\smallskipamount]
\hbox{}\qquad Free variable: \nopagebreak \\
\hbox{}\qquad\qquad $w = [b]$
\postw

It would seem that $[b] \in S$. How could this be? An inspection of the
introduction rules reveals that the only rule with a right-hand side of the form
$b \mathbin{\#} {\ldots} \in S$ that could have introduced $[b]$ into $S$ is
\textit{R5}:

\prew
``$w \in S\,\Longrightarrow\, b \mathbin{\#} w \in S$''
\postw

On closer inspection, we can see that this rule is wrong. To match the
production $B ::= bS$, the second $S$ should be a $B$. We fix the typo and try
again:

\prew
\textbf{nitpick} \\[2\smallskipamount]
\slshape Nitpick found a counterexample: \\[2\smallskipamount]
\hbox{}\qquad Free variable: \nopagebreak \\
\hbox{}\qquad\qquad $w = [a, a, b]$
\postw

Some detective work is necessary to find out what went wrong here. To get $[a,
a, b] \in S$, we need $[a, b] \in B$ by \textit{R3}, which in turn can only come
from \textit{R6}:

\prew
``$\lbrakk v \in B;\> v \in B\rbrakk \,\Longrightarrow\, a \mathbin{\#} v \mathbin{@} w \in B$''
\postw

Now, this formula must be wrong: The same assumption occurs twice, and the
variable $w$ is unconstrained. Clearly, one of the two occurrences of $v$ in
the assumptions should have been a $w$.

With the correction made, we don't get any counterexample from Nitpick. Let's
move on and check completeness:

\prew
\textbf{theorem}~\textit{S\_complete}: \\
``$\textit{length}~[x\mathbin{\leftarrow} w.\; x = a] =
   \textit{length}~[x\mathbin{\leftarrow} w.\; x = b]
  \longrightarrow w \in S$'' \\
\textbf{nitpick} \\[2\smallskipamount]
\slshape Nitpick found a counterexample: \\[2\smallskipamount]
\hbox{}\qquad Free variable: \nopagebreak \\
\hbox{}\qquad\qquad $w = [b, b, a, a]$
\postw

Apparently, $[b, b, a, a] \notin S$, even though it has the same numbers of
$a$'s and $b$'s. But since our inductive definition passed the soundness check,
the introduction rules we have are probably correct. Perhaps we simply lack an
introduction rule. Comparing the grammar with the inductive definition, our
suspicion is confirmed: Joe Blow simply forgot the production $A ::= bAA$,
without which the grammar cannot generate two or more $b$'s in a row. So we add
the rule

\prew
``$\lbrakk v \in A;\> w \in A\rbrakk \,\Longrightarrow\, b \mathbin{\#} v \mathbin{@} w \in A$''
\postw

With this last change, we don't get any counterexamples from Nitpick for either
soundness or completeness. We can even generalize our result to cover $A$ and
$B$ as well:

\prew
\textbf{theorem} \textit{S\_A\_B\_sound\_and\_complete}: \\
``$w \in S \longleftrightarrow \textit{length}~[x \mathbin{\leftarrow} w.\; x = a] = \textit{length}~[x \mathbin{\leftarrow} w.\; x = b]$'' \\
``$w \in A \longleftrightarrow \textit{length}~[x \mathbin{\leftarrow} w.\; x = a] = \textit{length}~[x \mathbin{\leftarrow} w.\; x = b] + 1$'' \\
``$w \in B \longleftrightarrow \textit{length}~[x \mathbin{\leftarrow} w.\; x = b] = \textit{length}~[x \mathbin{\leftarrow} w.\; x = a] + 1$'' \\
\textbf{nitpick} \\[2\smallskipamount]
\slshape Nitpick found no counterexample
\postw

\subsection{AA Trees}
\label{aa-trees}

AA trees are a kind of balanced trees discovered by Arne Andersson that provide
similar performance to red-black trees, but with a simpler implementation
\cite{andersson-1993}. They can be used to store sets of elements equipped with
a total order $<$. We start by defining the datatype and some basic extractor
functions:

\prew
\textbf{datatype} $'a$~\textit{aa\_tree} = \\
\hbox{}\quad $\Lambda$ $\mid$ $N$ ``\kern1pt$'a\Colon \textit{linorder\/}$'' \textit{nat} ``\kern1pt$'a$ \textit{aa\_tree}'' ``\kern1pt$'a$ \textit{aa\_tree}''  \\[2\smallskipamount]
\textbf{primrec} \textit{data} \textbf{where} \\
``$\textit{data}~\Lambda = \unkef$'' $\,\mid$ \\
``$\textit{data}~(N~x~\_~\_~\_) = x$'' \\[2\smallskipamount]
\textbf{primrec} \textit{dataset} \textbf{where} \\
``$\textit{dataset}~\Lambda = \{\}$'' $\,\mid$ \\
``$\textit{dataset}~(N~x~\_~t~u) = \{x\} \cup \textit{dataset}~t \mathrel{\cup} \textit{dataset}~u$'' \\[2\smallskipamount]
\textbf{primrec} \textit{level} \textbf{where} \\
``$\textit{level}~\Lambda = 0$'' $\,\mid$ \\
``$\textit{level}~(N~\_~k~\_~\_) = k$'' \\[2\smallskipamount]
\textbf{primrec} \textit{left} \textbf{where} \\
``$\textit{left}~\Lambda = \Lambda$'' $\,\mid$ \\
``$\textit{left}~(N~\_~\_~t~\_) = t$'' \\[2\smallskipamount]
\textbf{primrec} \textit{right} \textbf{where} \\
``$\textit{right}~\Lambda = \Lambda$'' $\,\mid$ \\
``$\textit{right}~(N~\_~\_~\_~u) = u$''
\postw

The wellformedness criterion for AA trees is fairly complex. Wikipedia states it
as follows \cite{wikipedia-2009-aa-trees}:

\kern.2\parskip %% TYPESETTING

\pre
Each node has a level field, and the following invariants must remain true for
the tree to be valid:

\raggedright

\kern-.4\parskip %% TYPESETTING

\begin{enum}
\item[]
\begin{enum}
\item[1.] The level of a leaf node is one.
\item[2.] The level of a left child is strictly less than that of its parent.
\item[3.] The level of a right child is less than or equal to that of its parent.
\item[4.] The level of a right grandchild is strictly less than that of its grandparent.
\item[5.] Every node of level greater than one must have two children.
\end{enum}
\end{enum}
\post

\kern.4\parskip %% TYPESETTING

The \textit{wf} predicate formalizes this description:

\prew
\textbf{primrec} \textit{wf} \textbf{where} \\
``$\textit{wf}~\Lambda = \textit{True\/}$'' $\,\mid$ \\
``$\textit{wf}~(N~\_~k~t~u) =$ \\
\phantom{``}$(\textrm{if}~t = \Lambda~\textrm{then}$ \\
\phantom{``$(\quad$}$k = 1 \mathrel{\land} (u = \Lambda \mathrel{\lor} (\textit{level}~u = 1 \mathrel{\land} \textit{left}~u = \Lambda \mathrel{\land} \textit{right}~u = \Lambda))$ \\
\phantom{``$($}$\textrm{else}$ \\
\hbox{}\phantom{``$(\quad$}$\textit{wf}~t \mathrel{\land} \textit{wf}~u
\mathrel{\land} u \not= \Lambda \mathrel{\land} \textit{level}~t < k
\mathrel{\land} \textit{level}~u \le k$ \\
\hbox{}\phantom{``$(\quad$}${\land}\; \textit{level}~(\textit{right}~u) < k)$''
\postw

Rebalancing the tree upon insertion and removal of elements is performed by two
auxiliary functions called \textit{skew} and \textit{split}, defined below:

\prew
\textbf{primrec} \textit{skew} \textbf{where} \\
``$\textit{skew}~\Lambda = \Lambda$'' $\,\mid$ \\
``$\textit{skew}~(N~x~k~t~u) = {}$ \\
\phantom{``}$(\textrm{if}~t \not= \Lambda \mathrel{\land} k =
\textit{level}~t~\textrm{then}$ \\
\phantom{``(\quad}$N~(\textit{data}~t)~k~(\textit{left}~t)~(N~x~k~
(\textit{right}~t)~u)$ \\
\phantom{``(}$\textrm{else}$ \\
\phantom{``(\quad}$N~x~k~t~u)$''
\postw

\prew
\textbf{primrec} \textit{split} \textbf{where} \\
``$\textit{split}~\Lambda = \Lambda$'' $\,\mid$ \\
``$\textit{split}~(N~x~k~t~u) = {}$ \\
\phantom{``}$(\textrm{if}~u \not= \Lambda \mathrel{\land} k =
\textit{level}~(\textit{right}~u)~\textrm{then}$ \\
\phantom{``(\quad}$N~(\textit{data}~u)~(\textit{Suc}~k)~
(N~x~k~t~(\textit{left}~u))~(\textit{right}~u)$ \\
\phantom{``(}$\textrm{else}$ \\
\phantom{``(\quad}$N~x~k~t~u)$''
\postw

Performing a \textit{skew} or a \textit{split} should have no impact on the set
of elements stored in the tree:

\prew
\textbf{theorem}~\textit{dataset\_skew\_split\/}:\\
``$\textit{dataset}~(\textit{skew}~t) = \textit{dataset}~t$'' \\
``$\textit{dataset}~(\textit{split}~t) = \textit{dataset}~t$'' \\
\textbf{nitpick} \\[2\smallskipamount]
{\slshape Nitpick ran out of time after checking 9 of 10 scopes}
\postw

Furthermore, applying \textit{skew} or \textit{split} on a well-formed tree
should not alter the tree:

\prew
\textbf{theorem}~\textit{wf\_skew\_split\/}:\\
``$\textit{wf}~t\,\Longrightarrow\, \textit{skew}~t = t$'' \\
``$\textit{wf}~t\,\Longrightarrow\, \textit{split}~t = t$'' \\
\textbf{nitpick} \\[2\smallskipamount]
{\slshape Nitpick found no counterexample}
\postw

Insertion is implemented recursively. It preserves the sort order:

\prew
\textbf{primrec}~\textit{insort} \textbf{where} \\
``$\textit{insort}~\Lambda~x = N~x~1~\Lambda~\Lambda$'' $\,\mid$ \\
``$\textit{insort}~(N~y~k~t~u)~x =$ \\
\phantom{``}$({*}~(\textit{split} \circ \textit{skew})~{*})~(N~y~k~(\textrm{if}~x < y~\textrm{then}~\textit{insort}~t~x~\textrm{else}~t)$ \\
\phantom{``$({*}~(\textit{split} \circ \textit{skew})~{*})~(N~y~k~$}$(\textrm{if}~x > y~\textrm{then}~\textit{insort}~u~x~\textrm{else}~u))$''
\postw

Notice that we deliberately commented out the application of \textit{skew} and
\textit{split}. Let's see if this causes any problems:

\prew
\textbf{theorem}~\textit{wf\_insort\/}:\kern.4em ``$\textit{wf}~t\,\Longrightarrow\, \textit{wf}~(\textit{insort}~t~x)$'' \\
\textbf{nitpick} \\[2\smallskipamount]
\slshape Nitpick found a counterexample for \textit{card} $'a$ = 4: \\[2\smallskipamount]
\hbox{}\qquad Free variables: \nopagebreak \\
\hbox{}\qquad\qquad $t = N~a_1~1~\Lambda~\Lambda$ \\
\hbox{}\qquad\qquad $x = a_2$
\postw

It's hard to see why this is a counterexample. To improve readability, we will
restrict the theorem to \textit{nat}, so that we don't need to look up the value
of the $\textit{op}~{<}$ constant to find out which element is smaller than the
other. In addition, we will tell Nitpick to display the value of
$\textit{insort}~t~x$ using the \textit{eval} option. This gives

\prew
\textbf{theorem} \textit{wf\_insort\_nat\/}:\kern.4em ``$\textit{wf}~t\,\Longrightarrow\, \textit{wf}~(\textit{insort}~t~(x\Colon\textit{nat}))$'' \\
\textbf{nitpick} [\textit{eval} = ``$\textit{insort}~t~x$''] \\[2\smallskipamount]
\slshape Nitpick found a counterexample: \\[2\smallskipamount]
\hbox{}\qquad Free variables: \nopagebreak \\
\hbox{}\qquad\qquad $t = N~1~1~\Lambda~\Lambda$ \\
\hbox{}\qquad\qquad $x = 0$ \\
\hbox{}\qquad Evaluated term: \\
\hbox{}\qquad\qquad $\textit{insort}~t~x = N~1~1~(N~0~1~\Lambda~\Lambda)~\Lambda$
\postw

Nitpick's output reveals that the element $0$ was added as a left child of $1$,
where both nodes have a level of 1. This violates the second AA tree invariant,
which states that a left child's level must be less than its parent's. This
shouldn't come as a surprise, considering that we commented out the tree
rebalancing code. Reintroducing the code seems to solve the problem:

\prew
\textbf{theorem}~\textit{wf\_insort\/}:\kern.4em ``$\textit{wf}~t\,\Longrightarrow\, \textit{wf}~(\textit{insort}~t~x)$'' \\
\textbf{nitpick} \\[2\smallskipamount]
{\slshape Nitpick ran out of time after checking 8 of 10 scopes}
\postw

Insertion should transform the set of elements represented by the tree in the
obvious way:

\prew
\textbf{theorem} \textit{dataset\_insort\/}:\kern.4em
``$\textit{dataset}~(\textit{insort}~t~x) = \{x\} \cup \textit{dataset}~t$'' \\
\textbf{nitpick} \\[2\smallskipamount]
{\slshape Nitpick ran out of time after checking 7 of 10 scopes}
\postw

We could continue like this and sketch a full-blown theory of AA trees. Once the
definitions and main theorems are in place and have been thoroughly tested using
Nitpick, we could start working on the proofs. Developing theories this way
usually saves time, because faulty theorems and definitions are discovered much
earlier in the process.

\section{Option Reference}
\label{option-reference}

\def\defl{\{}
\def\defr{\}}

\def\flushitem#1{\item[]\noindent\kern-\leftmargin \textbf{#1}}
\def\qty#1{$\left<\textit{#1}\right>$}
\def\qtybf#1{$\mathbf{\left<\textbf{\textit{#1}}\right>}$}
\def\optrue#1#2{\flushitem{\textit{#1} $\bigl[$= \qtybf{bool}$\bigr]$\enskip \defl\textit{true}\defr\hfill (neg.: \textit{#2})}\nopagebreak\\[\parskip]}
\def\opfalse#1#2{\flushitem{\textit{#1} $\bigl[$= \qtybf{bool}$\bigr]$\enskip \defl\textit{false}\defr\hfill (neg.: \textit{#2})}\nopagebreak\\[\parskip]}
\def\opsmart#1#2{\flushitem{\textit{#1} $\bigl[$= \qtybf{smart\_bool}$\bigr]$\enskip \defl\textit{smart}\defr\hfill (neg.: \textit{#2})}\nopagebreak\\[\parskip]}
\def\opnodefault#1#2{\flushitem{\textit{#1} = \qtybf{#2}} \nopagebreak\\[\parskip]}
\def\opdefault#1#2#3{\flushitem{\textit{#1} = \qtybf{#2}\enskip \defl\textit{#3}\defr} \nopagebreak\\[\parskip]}
\def\oparg#1#2#3{\flushitem{\textit{#1} \qtybf{#2} = \qtybf{#3}} \nopagebreak\\[\parskip]}
\def\opargbool#1#2#3{\flushitem{\textit{#1} \qtybf{#2} $\bigl[$= \qtybf{bool}$\bigr]$\hfill (neg.: \textit{#3})}\nopagebreak\\[\parskip]}
\def\opargboolorsmart#1#2#3{\flushitem{\textit{#1} \qtybf{#2} $\bigl[$= \qtybf{smart\_bool}$\bigr]$\hfill (neg.: \textit{#3})}\nopagebreak\\[\parskip]}

Nitpick's behavior can be influenced by various options, which can be specified
in brackets after the \textbf{nitpick} command. Default values can be set
using \textbf{nitpick\_\allowbreak params}. For example:

\prew
\textbf{nitpick\_params} [\textit{verbose}, \,\textit{timeout} = 60]
\postw

The options are categorized as follows:\ mode of operation
(\S\ref{mode-of-operation}), scope of search (\S\ref{scope-of-search}), output
format (\S\ref{output-format}), regression testing (\S\ref{regression-testing}),
optimizations (\S\ref{optimizations}), and timeouts (\S\ref{timeouts}).

Nitpick also provides an automatic mode that can be enabled via the
``Auto Nitpick'' option under ``Plugins > Plugin Options > Isabelle > General''
in Isabelle/jEdit. For automatic runs,
\textit{user\_axioms} (\S\ref{mode-of-operation}),
\textit{assms} (\S\ref{mode-of-operation}), and \textit{mono}
(\S\ref{scope-of-search}) are implicitly enabled,
\textit{verbose} (\S\ref{output-format}) and
\textit{debug} (\S\ref{output-format}) are disabled, \textit{max\_potential}
(\S\ref{output-format}) is taken to be 0, \textit{max\_threads}
(\S\ref{optimizations}) is taken to be 1, and \textit{timeout}
(\S\ref{timeouts}) is superseded by the ``Auto Time Limit'' in
Isabelle/jEdit. Nitpick's output is also more concise.

The number of options can be overwhelming at first glance. Do not let that worry
you: Nitpick's defaults have been chosen so that it almost always does the right
thing, and the most important options have been covered in context in
\S\ref{first-steps}.

The descriptions below refer to the following syntactic quantities:

\begin{enum}
\item[\labelitemi] \qtybf{string}: A string.
\item[\labelitemi] \qtybf{string\_list\/}: A space-separated list of strings
(e.g., ``\textit{ichi ni san}'').
\item[\labelitemi] \qtybf{bool\/}: \textit{true} or \textit{false}.
\item[\labelitemi] \qtybf{smart\_bool\/}: \textit{true}, \textit{false}, or \textit{smart}.
\item[\labelitemi] \qtybf{int\/}: An integer. Negative integers are prefixed with a hyphen.
\item[\labelitemi] \qtybf{smart\_int\/}: An integer or \textit{smart}.
\item[\labelitemi] \qtybf{int\_range}: An integer (e.g., 3) or a range
of nonnegative integers (e.g., $1$--$4$). The range symbol `--' is entered as \texttt{-} (hyphen).
\item[\labelitemi] \qtybf{int\_seq}: A comma-separated sequence of ranges of integers (e.g.,~1{,}3{,}\allowbreak6--8).
\item[\labelitemi] \qtybf{float}: An floating-point number (e.g., 0.5 or 60)
expressing a number of seconds.
\item[\labelitemi] \qtybf{const\/}: The name of a HOL constant.
\item[\labelitemi] \qtybf{term}: A HOL term (e.g., ``$f~x$'').
\item[\labelitemi] \qtybf{term\_list\/}: A space-separated list of HOL terms (e.g.,
``$f~x$''~``$g~y$'').
\item[\labelitemi] \qtybf{type}: A HOL type.
\end{enum}

Default values are indicated in curly brackets (\textrm{\{\}}). Boolean options
have a negated counterpart (e.g., \textit{mono} vs.\
\textit{non\_mono}). When setting them, ``= \textit{true}'' may be omitted.

\subsection{Mode of Operation}
\label{mode-of-operation}

\begin{enum}
\optrue{falsify}{satisfy}
Specifies whether Nitpick should look for falsifying examples (countermodels) or
satisfying examples (models). This manual assumes throughout that
\textit{falsify} is enabled.

\opsmart{user\_axioms}{no\_user\_axioms}
Specifies whether the user-defined axioms (specified using
\textbf{axiomatization} and \textbf{axioms}) should be considered. If the option
is set to \textit{smart}, Nitpick performs an ad hoc axiom selection based on
the constants that occur in the formula to falsify. The option is implicitly set
to \textit{true} for automatic runs.

\textbf{Warning:} If the option is set to \textit{true}, Nitpick might
nonetheless ignore some polymorphic axioms. Counterexamples generated under
these conditions are tagged as ``quasi genuine.'' The \textit{debug}
(\S\ref{output-format}) option can be used to find out which axioms were
considered.

\nopagebreak
{\small See also \textit{assms} (\S\ref{mode-of-operation}) and \textit{debug}
(\S\ref{output-format}).}

\optrue{assms}{no\_assms}
Specifies whether the relevant assumptions in structured proofs should be
considered. The option is implicitly enabled for automatic runs.

\nopagebreak
{\small See also \textit{user\_axioms} (\S\ref{mode-of-operation}).}

\opfalse{spy}{dont\_spy}
Specifies whether Nitpick should record statistics in
\texttt{\$ISA\-BELLE\_\allowbreak HOME\_\allowbreak USER/\allowbreak spy\_\allowbreak nitpick}.
These statistics can be useful to the developer of Nitpick. If you are willing to have your
interactions recorded in the name of science, please enable this feature and send the statistics
file every now and then to the author of this manual (\authoremail).
To change the default value of this option globally, set the environment variable
\texttt{NITPICK\_SPY} to \texttt{yes}.

\nopagebreak
{\small See also \textit{debug} (\S\ref{output-format}).}

\opfalse{overlord}{no\_overlord}
Specifies whether Nitpick should put its temporary files in
\texttt{\$ISABELLE\_\allowbreak HOME\_\allowbreak USER}, which is useful for
debugging Nitpick but also unsafe if several instances of the tool are run
simultaneously. The files are identified by the extensions
\texttt{.kki}, \texttt{.cnf}, \texttt{.out}, and
\texttt{.err}; you may safely remove them after Nitpick has run.

\textbf{Warning:} This option is not thread-safe. Use at your own risks.

\nopagebreak
{\small See also \textit{debug} (\S\ref{output-format}).}
\end{enum}

\subsection{Scope of Search}
\label{scope-of-search}

\begin{enum}
\oparg{card}{type}{int\_seq}
Specifies the sequence of cardinalities to use for a given type.
For free types, and often also for \textbf{typedecl}'d types, it usually makes
sense to specify cardinalities as a range of the form \textit{$1$--$n$}.

\nopagebreak
{\small See also \textit{box} (\S\ref{scope-of-search}) and \textit{mono}
(\S\ref{scope-of-search}).}

\opdefault{card}{int\_seq}{\upshape 1--10}
Specifies the default sequence of cardinalities to use. This can be overridden
on a per-type basis using the \textit{card}~\qty{type} option described above.

\oparg{max}{const}{int\_seq}
Specifies the sequence of maximum multiplicities to use for a given
(co)in\-duc\-tive datatype constructor. A constructor's multiplicity is the
number of distinct values that it can construct. Nonsensical values (e.g.,
\textit{max}~[]~$=$~2) are silently repaired. This option is only available for
datatypes equipped with several constructors.

\opnodefault{max}{int\_seq}
Specifies the default sequence of maximum multiplicities to use for
(co)in\-duc\-tive datatype constructors. This can be overridden on a per-constructor
basis using the \textit{max}~\qty{const} option described above.

\opsmart{binary\_ints}{unary\_ints}
Specifies whether natural numbers and integers should be encoded using a unary
or binary notation. In unary mode, the cardinality fully specifies the subset
used to approximate the type. For example:
%
$$\hbox{\begin{tabular}{@{}rll@{}}%
\textit{card nat} = 4 & induces & $\{0,\, 1,\, 2,\, 3\}$ \\
\textit{card int} = 4 & induces & $\{-1,\, 0,\, +1,\, +2\}$ \\
\textit{card int} = 5 & induces & $\{-2,\, -1,\, 0,\, +1,\, +2\}.$%
\end{tabular}}$$
%
In general:
%
$$\hbox{\begin{tabular}{@{}rll@{}}%
\textit{card nat} = $K$ & induces & $\{0,\, \ldots,\, K - 1\}$ \\
\textit{card int} = $K$ & induces & $\{-\lceil K/2 \rceil + 1,\, \ldots,\, +\lfloor K/2 \rfloor\}.$%
\end{tabular}}$$
%
In binary mode, the cardinality specifies the number of distinct values that can
be constructed. Each of these value is represented by a bit pattern whose length
is specified by the \textit{bits} (\S\ref{scope-of-search}) option. By default,
Nitpick attempts to choose the more appropriate encoding by inspecting the
formula at hand, preferring the binary notation for problems involving
multiplicative operators or large constants.

\textbf{Warning:} For technical reasons, Nitpick always reverts to unary for
problems that refer to the types \textit{rat} or \textit{real} or the constants
\textit{Suc}, \textit{gcd}, or \textit{lcm}.

{\small See also \textit{bits} (\S\ref{scope-of-search}) and
\textit{show\_types} (\S\ref{output-format}).}

\opdefault{bits}{int\_seq}{\upshape 1--10}
Specifies the number of bits to use to represent natural numbers and integers in
binary, excluding the sign bit. The minimum is 1 and the maximum is 31.

{\small See also \textit{binary\_ints} (\S\ref{scope-of-search}).}

\opargboolorsmart{wf}{const}{non\_wf}
Specifies whether the specified (co)in\-duc\-tively defined predicate is
well-founded. The option can take the following values:

\begin{enum}
\item[\labelitemi] \textbf{\textit{true}:} Tentatively treat the (co)in\-duc\-tive
predicate as if it were well-founded. Since this is generally not sound when the
predicate is not well-founded, the counterexamples are tagged as ``quasi
genuine.''

\item[\labelitemi] \textbf{\textit{false}:} Treat the (co)in\-duc\-tive predicate
as if it were not well-founded. The predicate is then unrolled as prescribed by
the \textit{star\_linear\_preds}, \textit{iter}~\qty{const}, and \textit{iter}
options.

\item[\labelitemi] \textbf{\textit{smart}:} Try to prove that the inductive
predicate is well-founded using Isabelle's \textit{lexicographic\_order} and
\textit{size\_change} tactics. If this succeeds (or the predicate occurs with an
appropriate polarity in the formula to falsify), use an efficient fixed-point
equation as specification of the predicate; otherwise, unroll the predicates
according to the \textit{iter}~\qty{const} and \textit{iter} options.
\end{enum}

\nopagebreak
{\small See also \textit{iter} (\S\ref{scope-of-search}),
\textit{star\_linear\_preds} (\S\ref{optimizations}), and \textit{tac\_timeout}
(\S\ref{timeouts}).}

\opsmart{wf}{non\_wf}
Specifies the default wellfoundedness setting to use. This can be overridden on
a per-predicate basis using the \textit{wf}~\qty{const} option above.

\oparg{iter}{const}{int\_seq}
Specifies the sequence of iteration counts to use when unrolling a given
(co)in\-duc\-tive predicate. By default, unrolling is applied for inductive
predicates that occur negatively and coinductive predicates that occur
positively in the formula to falsify and that cannot be proved to be
well-founded, but this behavior is influenced by the \textit{wf} option. The
iteration counts are automatically bounded by the cardinality of the predicate's
domain.

{\small See also \textit{wf} (\S\ref{scope-of-search}) and
\textit{star\_linear\_preds} (\S\ref{optimizations}).}

\opdefault{iter}{int\_seq}{\upshape 0{,}1{,}2{,}4{,}8{,}12{,}16{,}20{,}24{,}28}
Specifies the sequence of iteration counts to use when unrolling (co)in\-duc\-tive
predicates. This can be overridden on a per-predicate basis using the
\textit{iter} \qty{const} option above.

\opdefault{bisim\_depth}{int\_seq}{\upshape 9}
Specifies the sequence of iteration counts to use when unrolling the
bisimilarity predicate generated by Nitpick for coinductive datatypes. A value
of $-1$ means that no predicate is generated, in which case Nitpick performs an
after-the-fact check to see if the known coinductive datatype values are
bidissimilar. If two values are found to be bisimilar, the counterexample is
tagged as ``quasi genuine.'' The iteration counts are automatically bounded by
the sum of the cardinalities of the coinductive datatypes occurring in the
formula to falsify.

\opargboolorsmart{box}{type}{dont\_box}
Specifies whether Nitpick should attempt to wrap (``box'') a given function or
product type in an isomorphic datatype internally. Boxing is an effective mean
to reduce the search space and speed up Nitpick, because the isomorphic datatype
is approximated by a subset of the possible function or pair values.
Like other drastic optimizations, it can also prevent the discovery of
counterexamples. The option can take the following values:

\begin{enum}
\item[\labelitemi] \textbf{\textit{true}:} Box the specified type whenever
practicable.
\item[\labelitemi] \textbf{\textit{false}:} Never box the type.
\item[\labelitemi] \textbf{\textit{smart}:} Box the type only in contexts where it
is likely to help. For example, $n$-tuples where $n > 2$ and arguments to
higher-order functions are good candidates for boxing.
\end{enum}

\nopagebreak
{\small See also \textit{finitize} (\S\ref{scope-of-search}), \textit{verbose}
(\S\ref{output-format}), and \textit{debug} (\S\ref{output-format}).}

\opsmart{box}{dont\_box}
Specifies the default boxing setting to use. This can be overridden on a
per-type basis using the \textit{box}~\qty{type} option described above.

\opargboolorsmart{finitize}{type}{dont\_finitize}
Specifies whether Nitpick should attempt to finitize an infinite datatype. The
option can then take the following values:

\begin{enum}
\item[\labelitemi] \textbf{\textit{true}:} Finitize the datatype. Since this is
unsound, counterexamples generated under these conditions are tagged as ``quasi
genuine.''
\item[\labelitemi] \textbf{\textit{false}:} Don't attempt to finitize the datatype.
\item[\labelitemi] \textbf{\textit{smart}:}
If the datatype's constructors don't appear in the problem, perform a
monotonicity analysis to detect whether the datatype can be soundly finitized;
otherwise, don't finitize it.
\end{enum}

\nopagebreak
{\small See also \textit{box} (\S\ref{scope-of-search}), \textit{mono}
(\S\ref{scope-of-search}), \textit{verbose} (\S\ref{output-format}), and
\textit{debug} (\S\ref{output-format}).}

\opsmart{finitize}{dont\_finitize}
Specifies the default finitization setting to use. This can be overridden on a
per-type basis using the \textit{finitize}~\qty{type} option described above.

\opargboolorsmart{mono}{type}{non\_mono}
Specifies whether the given type should be considered monotonic when enumerating
scopes and finitizing types. If the option is set to \textit{smart}, Nitpick
performs a monotonicity check on the type. Setting this option to \textit{true}
can reduce the number of scopes tried, but it can also diminish the chance of
finding a counterexample, as demonstrated in \S\ref{scope-monotonicity}. The
option is implicitly set to \textit{true} for automatic runs.

\nopagebreak
{\small See also \textit{card} (\S\ref{scope-of-search}),
\textit{finitize} (\S\ref{scope-of-search}),
\textit{merge\_type\_vars} (\S\ref{scope-of-search}), and \textit{verbose}
(\S\ref{output-format}).}

\opsmart{mono}{non\_mono}
Specifies the default monotonicity setting to use. This can be overridden on a
per-type basis using the \textit{mono}~\qty{type} option described above.

\opfalse{merge\_type\_vars}{dont\_merge\_type\_vars}
Specifies whether type variables with the same sort constraints should be
merged. Setting this option to \textit{true} can reduce the number of scopes
tried and the size of the generated Kodkod formulas, but it also diminishes the
theoretical chance of finding a counterexample.

{\small See also \textit{mono} (\S\ref{scope-of-search}).}
\end{enum}

\subsection{Output Format}
\label{output-format}

\begin{enum}
\opfalse{verbose}{quiet}
Specifies whether the \textbf{nitpick} command should explain what it does. This
option is useful to determine which scopes are tried or which SAT solver is
used. This option is implicitly disabled for automatic runs.

\opfalse{debug}{no\_debug}
Specifies whether Nitpick should display additional debugging information beyond
what \textit{verbose} already displays. Enabling \textit{debug} also enables
\textit{verbose} and \textit{show\_all} behind the scenes. The \textit{debug}
option is implicitly disabled for automatic runs.

\nopagebreak
{\small See also \textit{spy} (\S\ref{mode-of-operation}),
\textit{overlord} (\S\ref{mode-of-operation}), and
\textit{batch\_size} (\S\ref{optimizations}).}

\opfalse{show\_types}{hide\_types}
Specifies whether the subsets used to approximate (co)in\-duc\-tive data\-types should
be displayed as part of counterexamples. Such subsets are sometimes helpful when
investigating whether a potentially spurious counterexample is genuine, but
their potential for clutter is real.

\optrue{show\_skolems}{hide\_skolem}
Specifies whether the values of Skolem constants should be displayed as part of
counterexamples. Skolem constants correspond to bound variables in the original
formula and usually help us to understand why the counterexample falsifies the
formula.

\opfalse{show\_consts}{hide\_consts}
Specifies whether the values of constants occurring in the formula (including
its axioms) should be displayed along with any counterexample. These values are
sometimes helpful when investigating why a counterexample is
genuine, but they can clutter the output.

\opnodefault{show\_all}{bool}
Abbreviation for \textit{show\_types}, \textit{show\_skolems}, and
\textit{show\_consts}.

\opdefault{max\_potential}{int}{\upshape 1}
Specifies the maximum number of potentially spurious counterexamples to display.
Setting this option to 0 speeds up the search for a genuine counterexample. This
option is implicitly set to 0 for automatic runs. If you set this option to a
value greater than 1, you will need an incremental SAT solver, such as
\textit{MiniSat\_JNI} (recommended) and \textit{SAT4J}. Be aware that many of
the counterexamples may be identical.

\nopagebreak
{\small See also \textit{sat\_solver} (\S\ref{optimizations}).}

\opdefault{max\_genuine}{int}{\upshape 1}
Specifies the maximum number of genuine counterexamples to display. If you set
this option to a value greater than 1, you will need an incremental SAT solver,
such as \textit{MiniSat\_JNI} (recommended) and \textit{SAT4J}. Be aware that
many of the counterexamples may be identical.

\nopagebreak
{\small See also \textit{sat\_solver} (\S\ref{optimizations}).}

\opnodefault{eval}{term\_list}
Specifies the list of terms whose values should be displayed along with
counterexamples. This option suffers from an ``observer effect'': Nitpick might
find different counterexamples for different values of this option.

\oparg{atoms}{type}{string\_list}
Specifies the names to use to refer to the atoms of the given type. By default,
Nitpick generates names of the form $a_1, \ldots, a_n$, where $a$ is the first
letter of the type's name.

\opnodefault{atoms}{string\_list}
Specifies the default names to use to refer to atoms of any type. For example,
to call the three atoms of type ${'}a$ \textit{ichi}, \textit{ni}, and
\textit{san} instead of $a_1$, $a_2$, $a_3$, specify the option
``\textit{atoms}~${'}a$ = \textit{ichi~ni~san}''. The default names can be
overridden on a per-type basis using the \textit{atoms}~\qty{type} option
described above.

\oparg{format}{term}{int\_seq}
Specifies how to uncurry the value displayed for a variable or constant.
Uncurrying sometimes increases the readability of the output for high-arity
functions. For example, given the variable $y \mathbin{\Colon} {'a}\Rightarrow
{'b}\Rightarrow {'c}\Rightarrow {'d}\Rightarrow {'e}\Rightarrow {'f}\Rightarrow
{'g}$, setting \textit{format}~$y$ = 3 tells Nitpick to group the last three
arguments, as if the type had been ${'a}\Rightarrow {'b}\Rightarrow
{'c}\Rightarrow {'d}\times {'e}\times {'f}\Rightarrow {'g}$. In general, a list
of values $n_1,\ldots,n_k$ tells Nitpick to show the last $n_k$ arguments as an
$n_k$-tuple, the previous $n_{k-1}$ arguments as an $n_{k-1}$-tuple, and so on;
arguments that are not accounted for are left alone, as if the specification had
been $1,\ldots,1,n_1,\ldots,n_k$.

\opdefault{format}{int\_seq}{\upshape 1}
Specifies the default format to use. Irrespective of the default format, the
extra arguments to a Skolem constant corresponding to the outer bound variables
are kept separated from the remaining arguments, the \textbf{for} arguments of
an inductive definitions are kept separated from the remaining arguments, and
the iteration counter of an unrolled inductive definition is shown alone. The
default format can be overridden on a per-variable or per-constant basis using
the \textit{format}~\qty{term} option described above.
\end{enum}

\subsection{Regression Testing}
\label{regression-testing}

\begin{enum}
\opnodefault{expect}{string}
Specifies the expected outcome, which must be one of the following:

\begin{enum}
\item[\labelitemi] \textbf{\textit{genuine}:} Nitpick found a genuine counterexample.
\item[\labelitemi] \textbf{\textit{quasi\_genuine}:} Nitpick found a ``quasi
genuine'' counterexample (i.e., a counterexample that is genuine unless
it contradicts a missing axiom or a dangerous option was used inappropriately).
\item[\labelitemi] \textbf{\textit{potential}:} Nitpick found a potentially
spurious counterexample.
\item[\labelitemi] \textbf{\textit{none}:} Nitpick found no counterexample.
\item[\labelitemi] \textbf{\textit{unknown}:} Nitpick encountered some problem (e.g.,
Kodkod ran out of memory).
\end{enum}

Nitpick emits an error if the actual outcome differs from the expected outcome.
This option is useful for regression testing.
\end{enum}

\subsection{Optimizations}
\label{optimizations}

\def\cpp{C\nobreak\raisebox{.1ex}{+}\nobreak\raisebox{.1ex}{+}}

\sloppy

\begin{enum}
\opdefault{sat\_solver}{string}{smart}
Specifies which SAT solver to use. SAT solvers implemented in C or \cpp{} tend
to be faster than their Java counterparts, but they can be more difficult to
install. Also, if you set the \textit{max\_potential} (\S\ref{output-format}) or
\textit{max\_genuine} (\S\ref{output-format}) option to a value greater than 1,
you will need an incremental SAT solver, such as \textit{MiniSat\_JNI}
(recommended) or \textit{SAT4J}.

The supported solvers are listed below:

\begin{enum}

\item[\labelitemi] \textbf{\textit{Lingeling\_JNI}:}
Lingeling is an efficient solver written in C. The JNI (Java Native Interface)
version of Lingeling is bundled with Kodkodi and is precompiled for Linux and
Mac~OS~X. It is also available from the Kodkod web site \cite{kodkod-2009}.

\item[\labelitemi] \textbf{\textit{CryptoMiniSat}:} CryptoMiniSat is the winner of
the 2010 SAT Race. To use CryptoMiniSat, set the environment variable
\texttt{CRYPTO\-MINISAT\_}\discretionary{}{}{}\texttt{HOME} to the directory that contains the \texttt{crypto\-minisat}
executable.%
\footnote{Important note for Cygwin users: The path must be specified using
native Windows syntax. Make sure to escape backslashes properly.%
\label{cygwin-paths}}
The \cpp{} sources and executables for Crypto\-Mini\-Sat are available at
\url{http://planete.inrialpes.fr/~soos/}\allowbreak\url{CryptoMiniSat2/index.php}.
Nitpick has been tested with version 2.51.

\item[\labelitemi] \textbf{\textit{CryptoMiniSat\_JNI}:} The JNI (Java Native
Interface) version of CryptoMiniSat is bundled with Kodkodi and is precompiled
for Linux and Mac~OS~X. It is also available from the Kodkod web site
\cite{kodkod-2009}.

\item[\labelitemi] \textbf{\textit{MiniSat}:} MiniSat is an efficient solver
written in \cpp{}. To use MiniSat, set the environment variable
\texttt{MINISAT\_HOME} to the directory that contains the \texttt{minisat}
executable.%
\footref{cygwin-paths}
The \cpp{} sources and executables for MiniSat are available at
\url{http://minisat.se/MiniSat.html}. Nitpick has been tested with versions 1.14
and 2.2.

\item[\labelitemi] \textbf{\textit{MiniSat\_JNI}:} The JNI
version of MiniSat is bundled with Kodkodi and is precompiled for Linux,
Mac~OS~X, and Windows (Cygwin). It is also available from the Kodkod web site
\cite{kodkod-2009}. Unlike the standard version of MiniSat, the JNI version can
be used incrementally.

\item[\labelitemi] \textbf{\textit{Riss3g}:} Riss3g is an efficient solver written in
\cpp{}. To use Riss3g, set the environment variable \texttt{RISS3G\_HOME} to the
directory that contains the \texttt{riss3g} executable.%
\footref{cygwin-paths}
The \cpp{} sources for Riss3g are available at
\url{http://tools.computational-logic.org/content/riss3g.php}.
Nitpick has been tested with the SAT Competition 2013 version.

\item[\labelitemi] \textbf{\textit{zChaff}:} zChaff is an older solver written
in \cpp{}. To use zChaff, set the environment variable \texttt{ZCHAFF\_HOME} to
the directory that contains the \texttt{zchaff} executable.%
\footref{cygwin-paths}
The \cpp{} sources and executables for zChaff are available at
\url{http://www.princeton.edu/~chaff/zchaff.html}. Nitpick has been tested with
versions 2004-05-13, 2004-11-15, and 2007-03-12.

\item[\labelitemi] \textbf{\textit{RSat}:} RSat is an efficient solver written in
\cpp{}. To use RSat, set the environment variable \texttt{RSAT\_HOME} to the
directory that contains the \texttt{rsat} executable.%
\footref{cygwin-paths}
The \cpp{} sources for RSat are available at
\url{http://reasoning.cs.ucla.edu/rsat/}. Nitpick has been tested with version
2.01.

\item[\labelitemi] \textbf{\textit{BerkMin}:} BerkMin561 is an efficient solver
written in C. To use BerkMin, set the environment variable
\texttt{BERKMIN\_HOME} to the directory that contains the \texttt{BerkMin561}
executable.\footref{cygwin-paths}
The BerkMin executables are available at
\url{http://eigold.tripod.com/BerkMin.html}.

\item[\labelitemi] \textbf{\textit{BerkMin\_Alloy}:} Variant of BerkMin that is
included with Alloy 4 and calls itself ``sat56'' in its banner text. To use this
version of BerkMin, set the environment variable
\texttt{BERKMINALLOY\_HOME} to the directory that contains the \texttt{berkmin}
executable.%
\footref{cygwin-paths}

\item[\labelitemi] \textbf{\textit{SAT4J}:} SAT4J is a reasonably efficient solver
written in Java that can be used incrementally. It is bundled with Kodkodi and
requires no further installation or configuration steps. Do not attempt to
install the official SAT4J packages, because their API is incompatible with
Kodkod.

\item[\labelitemi] \textbf{\textit{SAT4J\_Light}:} Variant of SAT4J that is
optimized for small problems. It can also be used incrementally.

\item[\labelitemi] \textbf{\textit{smart}:} If \textit{sat\_solver} is set to
\textit{smart}, Nitpick selects the first solver among the above that is
recognized by Isabelle. If \textit{verbose} (\S\ref{output-format}) is enabled,
Nitpick displays which SAT solver was chosen.
\end{enum}
\fussy

\opdefault{batch\_size}{smart\_int}{smart}
Specifies the maximum number of Kodkod problems that should be lumped together
when invoking Kodkodi. Each problem corresponds to one scope. Lumping problems
together ensures that Kodkodi is launched less often, but it makes the verbose
output less readable and is sometimes detrimental to performance. If
\textit{batch\_size} is set to \textit{smart}, the actual value used is 1 if
\textit{debug} (\S\ref{output-format}) is set and 50 otherwise.

\optrue{destroy\_constrs}{dont\_destroy\_constrs}
Specifies whether formulas involving (co)in\-duc\-tive datatype constructors should
be rewritten to use (automatically generated) discriminators and destructors.
This optimization can drastically reduce the size of the Boolean formulas given
to the SAT solver.

\nopagebreak
{\small See also \textit{debug} (\S\ref{output-format}).}

\optrue{specialize}{dont\_specialize}
Specifies whether functions invoked with static arguments should be specialized.
This optimization can drastically reduce the search space, especially for
higher-order functions.

\nopagebreak
{\small See also \textit{debug} (\S\ref{output-format}) and
\textit{show\_consts} (\S\ref{output-format}).}

\optrue{star\_linear\_preds}{dont\_star\_linear\_preds}
Specifies whether Nitpick should use Kodkod's transitive closure operator to
encode non-well-founded ``linear inductive predicates,'' i.e., inductive
predicates for which each the predicate occurs in at most one assumption of each
introduction rule. Using the reflexive transitive closure is in principle
equivalent to setting \textit{iter} to the cardinality of the predicate's
domain, but it is usually more efficient.

{\small See also \textit{wf} (\S\ref{scope-of-search}), \textit{debug}
(\S\ref{output-format}), and \textit{iter} (\S\ref{scope-of-search}).}

\opnodefault{whack}{term\_list}
Specifies a list of atomic terms (usually constants, but also free and schematic
variables) that should be taken as being $\unk$ (unknown). This can be useful to
reduce the size of the Kodkod problem if you can guess in advance that a
constant might not be needed to find a countermodel.

{\small See also \textit{debug} (\S\ref{output-format}).}

\opnodefault{need}{term\_list}
Specifies a list of datatype values (normally ground constructor terms) that
should be part of the subterm-closed subsets used to approximate datatypes. If
you know that a value must necessarily belong to the subset of representable
values that approximates a datatype, specifying it can speed up the search,
especially for high cardinalities.
%By default, Nitpick inspects the conjecture to infer needed datatype values.

\opsmart{total\_consts}{partial\_consts}
Specifies whether constants occurring in the problem other than constructors can
be assumed to be considered total for the representable values that approximate
a datatype. This option is highly incomplete; it should be used only for
problems that do not construct datatype values explicitly. Since this option is
(in rare cases) unsound, counterexamples generated under these conditions are
tagged as ``quasi genuine.''

\opdefault{datatype\_sym\_break}{int}{\upshape 5}
Specifies an upper bound on the number of datatypes for which Nitpick generates
symmetry breaking predicates. Symmetry breaking can speed up the SAT solver
considerably, especially for unsatisfiable problems, but too much of it can slow
it down.

\opdefault{kodkod\_sym\_break}{int}{\upshape 15}
Specifies an upper bound on the number of relations for which Kodkod generates
symmetry breaking predicates. Symmetry breaking can speed up the SAT solver
considerably, especially for unsatisfiable problems, but too much of it can slow
it down.

\optrue{peephole\_optim}{no\_peephole\_optim}
Specifies whether Nitpick should simplify the generated Kodkod formulas using a
peephole optimizer. These optimizations can make a significant difference.
Unless you are tracking down a bug in Nitpick or distrust the peephole
optimizer, you should leave this option enabled.

\opdefault{max\_threads}{int}{\upshape 0}
Specifies the maximum number of threads to use in Kodkod. If this option is set
to 0, Kodkod will compute an appropriate value based on the number of processor
cores available. The option is implicitly set to 1 for automatic runs.

\nopagebreak
{\small See also \textit{batch\_size} (\S\ref{optimizations}) and
\textit{timeout} (\S\ref{timeouts}).}
\end{enum}

\subsection{Timeouts}
\label{timeouts}

\begin{enum}
\opdefault{timeout}{float}{\upshape 30}
Specifies the maximum number of seconds that the \textbf{nitpick} command should
spend looking for a counterexample. Nitpick tries to honor this constraint as
well as it can but offers no guarantees. For automatic runs, the ``Auto Time
Limit'' option under ``Plugins > Plugin Options > Isabelle > General'' is used
instead.

\nopagebreak
{\small See also \textit{max\_threads} (\S\ref{optimizations}).}

\opdefault{tac\_timeout}{float}{\upshape 0.5}
Specifies the maximum number of seconds that should be used by internal
tactics---\textit{lexicographic\_order} and \textit{size\_change} when checking
whether a (co)in\-duc\-tive predicate is well-founded or the monotonicity
inference. Nitpick tries to honor this constraint but offers no guarantees.

\nopagebreak
{\small See also \textit{wf} (\S\ref{scope-of-search}) and
\textit{mono} (\S\ref{scope-of-search}).}
\end{enum}

\section{Attribute Reference}
\label{attribute-reference}

Nitpick needs to consider the definitions of all constants occurring in a
formula in order to falsify it. For constants introduced using the
\textbf{definition} command, the definition is simply the associated
\textit{\_def} axiom. In contrast, instead of using the internal representation
of functions synthesized by Isabelle's \textbf{primrec}, \textbf{function}, and
\textbf{nominal\_primrec} packages, Nitpick relies on the more natural
equational specification entered by the user.

Behind the scenes, Isabelle's built-in packages and theories rely on the
following attributes to affect Nitpick's behavior:

\begin{enum}
\flushitem{\textit{nitpick\_unfold}}

\nopagebreak
This attribute specifies an equation that Nitpick should use to expand a
constant. The equation should be logically equivalent to the constant's actual
definition and should be of the form

\qquad $c~{?}x_1~\ldots~{?}x_n \,=\, t$,

or

\qquad $c~{?}x_1~\ldots~{?}x_n \,\equiv\, t$,

where ${?}x_1, \ldots, {?}x_n$ are distinct variables and $c$ does not occur in
$t$. Each occurrence of $c$ in the problem is expanded to $\lambda x_1\,\ldots
x_n.\; t$.

\flushitem{\textit{nitpick\_simp}}

\nopagebreak
This attribute specifies the equations that constitute the specification of a
constant. The \textbf{primrec}, \textbf{function}, and
\textbf{nominal\_\allowbreak primrec} packages automatically attach this
attribute to their \textit{simps} rules. The equations must be of the form

\qquad $c~t_1~\ldots\ t_n \;\bigl[{=}\; u\bigr]$

or

\qquad $c~t_1~\ldots\ t_n \,\equiv\, u.$

\flushitem{\textit{nitpick\_psimp}}

\nopagebreak
This attribute specifies the equations that constitute the partial specification
of a constant. The \textbf{function} package automatically attaches this
attribute to its \textit{psimps} rules. The conditional equations must be of the
form

\qquad $\lbrakk P_1;\> \ldots;\> P_m\rbrakk \,\Longrightarrow\, c\ t_1\ \ldots\ t_n \;\bigl[{=}\; u\bigr]$

or

\qquad $\lbrakk P_1;\> \ldots;\> P_m\rbrakk \,\Longrightarrow\, c\ t_1\ \ldots\ t_n \,\equiv\, u$.

\flushitem{\textit{nitpick\_choice\_spec}}

\nopagebreak
This attribute specifies the (free-form) specification of a constant defined
using the \textbf{specification} command.
\end{enum}

When faced with a constant, Nitpick proceeds as follows:

\begin{enum}
\item[1.] If the \textit{nitpick\_simp} set associated with the constant
is not empty, Nitpick uses these rules as the specification of the constant.

\item[2.] Otherwise, if the \textit{nitpick\_psimp} set associated with
the constant is not empty, it uses these rules as the specification of the
constant.

\item[3.] Otherwise, if the constant was defined using the
\allowbreak\textbf{specification} command and the
\textit{nitpick\_choice\_spec} set associated with the constant is not empty, it
uses these theorems as the specification of the constant.

\item[4.] Otherwise, it looks up the definition of the constant. If the
\textit{nitpick\_unfold} set associated with the constant is not empty, it uses
the latest rule added to the set as the definition of the constant; otherwise it
uses the actual definition axiom.

\begin{enum}
\item[1.] If the definition is of the form

\qquad $c~{?}x_1~\ldots~{?}x_m \,\equiv\, \lambda y_1~\ldots~y_n.\; \textit{lfp}~(\lambda f.\; t)$

or

\qquad $c~{?}x_1~\ldots~{?}x_m \,\equiv\, \lambda y_1~\ldots~y_n.\; \textit{gfp}~(\lambda f.\; t).$

Nitpick assumes that the definition was made using a (co)inductive package
based on the user-specified introduction rules registered in Isabelle's internal
\textit{Spec\_Rules} table. The tool uses the introduction rules to ascertain
whether the definition is well-founded and the definition to generate a
fixed-point equation or an unrolled equation.

\item[2.] If the definition is compact enough, the constant is \textsl{unfolded}
wherever it appears; otherwise, it is defined equationally, as with
the \textit{nitpick\_simp} attribute.
\end{enum}
\end{enum}

As an illustration, consider the inductive definition

\prew
\textbf{inductive}~\textit{odd}~\textbf{where} \\
``\textit{odd}~1'' $\,\mid$ \\
``\textit{odd}~$n\,\Longrightarrow\, \textit{odd}~(\textit{Suc}~(\textit{Suc}~n))$''
\postw

By default, Nitpick uses the \textit{lfp}-based definition in conjunction with
the introduction rules. To override this, you can specify an alternative
definition as follows:

\prew
\textbf{lemma} $\mathit{odd\_alt\_unfold}$ [\textit{nitpick\_unfold}]:\kern.4em ``$\textit{odd}~n \,\equiv\, n~\textrm{mod}~2 = 1$''
\postw

Nitpick then expands all occurrences of $\mathit{odd}~n$ to $n~\textrm{mod}~2
= 1$. Alternatively, you can specify an equational specification of the constant:

\prew
\textbf{lemma} $\mathit{odd\_simp}$ [\textit{nitpick\_simp}]:\kern.4em ``$\textit{odd}~n = (n~\textrm{mod}~2 = 1)$''
\postw

Such tweaks should be done with great care, because Nitpick will assume that the
constant is completely defined by its equational specification. For example, if
you make ``$\textit{odd}~(2 * k + 1)$'' a \textit{nitpick\_simp} rule and neglect to provide rules to handle the $2 * k$ case, Nitpick will define
$\textit{odd}~n$ arbitrarily for even values of $n$. The \textit{debug}
(\S\ref{output-format}) option is extremely useful to understand what is going
on when experimenting with \textit{nitpick\_} attributes.

Because of its internal three-valued logic, Nitpick tends to lose a
lot of precision in the presence of partially specified constants. For example,

\prew
\textbf{lemma} \textit{odd\_simp} [\textit{nitpick\_simp}]:\kern.4em ``$\textit{odd~x} = \lnot\, \textit{even}~x$''
\postw

is superior to

\prew
\textbf{lemma} \textit{odd\_psimps} [\textit{nitpick\_simp}]: \\
``$\textit{even~x} \,\Longrightarrow\, \textit{odd~x} = \textit{False\/}$'' \\
``$\lnot\, \textit{even~x} \,\Longrightarrow\, \textit{odd~x} = \textit{True\/}$''
\postw

Because Nitpick sometimes unfolds definitions but never simplification rules,
you can ensure that a constant is defined explicitly using the
\textit{nitpick\_simp}. For example:

\prew
\textbf{definition}~\textit{optimum} \textbf{where} [\textit{nitpick\_simp}]: \\
``$\textit{optimum}~t =
     (\forall u.\; \textit{consistent}~u \mathrel{\land} \textit{alphabet}~t = \textit{alphabet}~u$ \\
\phantom{``$\textit{optimum}~t = (\forall u.\;$}${\mathrel{\land}}\; \textit{freq}~t = \textit{freq}~u \longrightarrow
         \textit{cost}~t \le \textit{cost}~u)$''
\postw

In some rare occasions, you might want to provide an inductive or coinductive
view on top of an existing constant $c$. The easiest way to achieve this is to
define a new constant $c'$ (co)inductively. Then prove that $c$ equals $c'$
and let Nitpick know about it:

\prew
\textbf{lemma} \textit{c\_alt\_unfold} [\textit{nitpick\_unfold}]:\kern.4em ``$c \equiv c'$\kern2pt ''
\postw

This ensures that Nitpick will substitute $c'$ for $c$ and use the (co)inductive
definition.

\section{Standard ML Interface}
\label{standard-ml-interface}

Nitpick provides a rich Standard ML interface used mainly for internal purposes
and debugging. Among the most interesting functions exported by Nitpick are
those that let you invoke the tool programmatically and those that let you
register and unregister custom term postprocessors as well as coinductive
datatypes.

\subsection{Invoking Nitpick}
\label{invoking-nitpick}

The \textit{Nitpick} structure offers the following functions for invoking your
favorite counterexample generator:

\prew
$\textbf{val}\,~\textit{pick\_nits\_in\_term} : \\
\hbox{}\quad\textit{Proof.state} \rightarrow \textit{params} \rightarrow \textit{mode}
\rightarrow \textit{int} \rightarrow \textit{int} \rightarrow \textit{int}$ \\
$\hbox{}\quad{\rightarrow}\; (\textit{term} * \textit{term})~\textit{list}
\rightarrow \textit{term~list} \rightarrow \textit{term} \rightarrow \textit{string} * \textit{Proof.state}$ \\
$\textbf{val}\,~\textit{pick\_nits\_in\_subgoal} : \\
\hbox{}\quad\textit{Proof.state} \rightarrow \textit{params} \rightarrow \textit{mode} \rightarrow \textit{int} \rightarrow \textit{int} \rightarrow \textit{string} * \textit{Proof.state}$
\postw

The return value is a new proof state paired with an outcome string
(``genuine'', ``quasi\_genuine'', ``potential'', ``none'', or ``unknown''). The
\textit{params} type is a large record that lets you set Nitpick's options. The
current default options can be retrieved by calling the following function
defined in the \textit{Nitpick\_Isar} structure:

\prew
$\textbf{val}\,~\textit{default\_params} :\,
\textit{theory} \rightarrow (\textit{string} * \textit{string})~\textit{list} \rightarrow \textit{params}$
\postw

The second argument lets you override option values before they are parsed and
put into a \textit{params} record. Here is an example where Nitpick is invoked
on subgoal $i$ of $n$ with no time limit:

\prew
$\textbf{val}\,~\textit{params} = \textit{Nitpick\_Isar.default\_params}~\textit{thy}~[(\textrm{``}\textrm{timeout\/}\textrm{''},\, \textrm{``}\textrm{none}\textrm{''})]$ \\
$\textbf{val}\,~(\textit{outcome},\, \textit{state}') = {}$ \\
$\hbox{}\quad\textit{Nitpick.pick\_nits\_in\_subgoal}~\textit{state}~\textit{params}~\textit{Nitpick.Normal}~\textit{i}~\textit{n}$
\postw

\let\antiq=\textrm

\subsection{Registering Term Postprocessors}
\label{registering-term-postprocessors}

It is possible to change the output of any term that Nitpick considers a
datatype by registering a term postprocessor. The interface for registering and
unregistering postprocessors consists of the following pair of functions defined
in the \textit{Nitpick\_Model} structure:

\prew
$\textbf{type}\,~\textit{term\_postprocessor}\,~{=} {}$ \\
$\hbox{}\quad\textit{Proof.context} \rightarrow \textit{string} \rightarrow (\textit{typ} \rightarrow \textit{term~list\/}) \rightarrow \textit{typ} \rightarrow \textit{term} \rightarrow \textit{term}$ \\
$\textbf{val}\,~\textit{register\_term\_postprocessor} : {}$ \\
$\hbox{}\quad\textit{typ} \rightarrow \textit{term\_postprocessor} \rightarrow \textit{morphism} \rightarrow \textit{Context.generic}$ \\
$\hbox{}\quad{\rightarrow}\; \textit{Context.generic}$ \\
$\textbf{val}\,~\textit{unregister\_term\_postprocessor} : {}$ \\
$\hbox{}\quad\textit{typ} \rightarrow \textit{morphism} \rightarrow \textit{Context.generic} \rightarrow \textit{Context.generic}$
\postw

\S\ref{typedefs-quotient-types-records-rationals-and-reals} and
\texttt{src/HOL/Library/Multiset.thy} illustrate this feature in context.

\subsection{Registering Coinductive Datatypes}
\label{registering-coinductive-datatypes}

Coinductive datatypes defined using the \textbf{codatatype} command that do not
involve nested recursion through non-codatatypes are supported by Nitpick.
If you have defined a custom coinductive datatype, you can tell Nitpick about
it, so that it can use an efficient Kodkod axiomatization. The interface for
registering and unregistering coinductive datatypes consists of the following
pair of functions defined in the \textit{Nitpick\_HOL} structure:

\prew
$\textbf{val}\,~\textit{register\_codatatype\/} : {}$ \\
$\hbox{}\quad\textit{morphism} \rightarrow \textit{typ} \rightarrow \textit{string} \rightarrow (\textit{string} \times \textit{typ})\;\textit{list} \rightarrow \textit{Context.generic} {}$ \\
$\hbox{}\quad{\rightarrow}\; \textit{Context.generic}$ \\
$\textbf{val}\,~\textit{unregister\_codatatype\/} : {}$ \\
$\hbox{}\quad\textit{morphism} \rightarrow \textit{typ} \rightarrow \textit{Context.generic} \rightarrow \textit{Context.generic} {}$
\postw

The type $'a~\textit{llist}$ of lazy lists is already registered; had it
not been, you could have told Nitpick about it by adding the following line
to your theory file:

\prew
$\textbf{declaration}~\,\{{*}$ \\
$\hbox{}\quad\textit{Nitpick\_HOL.register\_codatatype}~@\{\antiq{typ}~``\kern1pt'a~\textit{llist\/}\textrm{''}\}$ \\
$\hbox{}\qquad\quad @\{\antiq{const\_name}~ \textit{llist\_case}\}$ \\
$\hbox{}\qquad\quad (\textit{map}~\textit{dest\_Const}~[@\{\antiq{term}~\textit{LNil}\},\, @\{\antiq{term}~\textit{LCons}\}])$ \\
${*}\}$
\postw

The \textit{register\_codatatype} function takes a coinductive datatype, its
case function, and the list of its constructors (in addition to the current
morphism and generic proof context). The case function must take its arguments
in the order that the constructors are listed. If no case function with the
correct signature is available, simply pass the empty string.

On the other hand, if your goal is to cripple Nitpick, add the following line to
your theory file and try to check a few conjectures about lazy lists:

\prew
$\textbf{declaration}~\,\{{*}$ \\
$\hbox{}\quad\textit{Nitpick\_HOL.unregister\_codatatype}~@\{\antiq{typ}~``\kern1pt'a~\textit{llist\/}\textrm{''}\}$ \\
${*}\}$
\postw

Inductive datatypes can be registered as coinductive datatypes, given
appropriate coinductive constructors. However, doing so precludes
the use of the inductive constructors---Nitpick will generate an error if they
are needed.

\section{Known Bugs and Limitations}
\label{known-bugs-and-limitations}

Here are the known bugs and limitations in Nitpick at the time of writing:

\begin{enum}
\item[\labelitemi] Underspecified functions defined using the \textbf{primrec},
\textbf{function}, or \textbf{nominal\_\allowbreak primrec} packages can lead
Nitpick to generate spurious counterexamples for theorems that refer to values
for which the function is not defined. For example:

\prew
\textbf{primrec} \textit{prec} \textbf{where} \\
``$\textit{prec}~(\textit{Suc}~n) = n$'' \\[2\smallskipamount]
\textbf{lemma} ``$\textit{prec}~0 = \textit{undefined\/}$'' \\
\textbf{nitpick} \\[2\smallskipamount]
\quad{\slshape Nitpick found a counterexample for \textit{card nat}~= 2:
\nopagebreak
\\[2\smallskipamount]
\hbox{}\qquad Empty assignment} \nopagebreak\\[2\smallskipamount]
\textbf{by}~(\textit{auto simp}:~\textit{prec\_def})
\postw

Such theorems are generally considered bad style because they rely on the
internal representation of functions synthesized by Isabelle, an implementation
detail.

\item[\labelitemi] Similarly, Nitpick might find spurious counterexamples for
theorems that rely on the use of the indefinite description operator internally
by \textbf{specification} and \textbf{quot\_type}.

\item[\labelitemi] Axioms or definitions that restrict the possible values of the
\textit{undefined} constant or other partially specified built-in Isabelle
constants (e.g., \textit{Abs\_} and \textit{Rep\_} constants) are in general
ignored. Again, such nonconservative extensions are generally considered bad
style.

\item[\labelitemi] Nitpick produces spurious counterexamples when invoked after a
\textbf{guess} command in a structured proof.

\item[\labelitemi] Datatypes defined using \textbf{datatype} and
codatatypes defined using \textbf{codatatype} that involve nested (co)recursion
through non-(co)datatypes are not properly supported and may result in spurious
counterexamples.

\item[\labelitemi] Types that are registered with several distinct sets of
constructors, including \textit{enat} if the \textit{Coinductive} entry of
the \textit{Archive of Formal Proofs} is loaded, can confuse Nitpick.

\item[\labelitemi] The \textit{nitpick\_xxx} attributes and the
\textit{Nitpick\_xxx.register\_yyy} functions can cause havoc if used
improperly.

\item[\labelitemi] Although this has never been observed, arbitrary theorem
morphisms could possibly confuse Nitpick, resulting in spurious counterexamples.

\item[\labelitemi] All constants, types, free variables, and schematic variables
whose names start with \textit{Nitpick}{.} are reserved for internal use.

\item[\labelitemi] Some users report technical issues with the default SAT
solver on Windows. Setting the \textit{sat\_solver} option
(\S\ref{optimizations}) to \textit{MiniSat\_JNI} should solve this.
\end{enum}

\let\em=\sl
\bibliography{manual}{}
\bibliographystyle{abbrv}

\end{document}
