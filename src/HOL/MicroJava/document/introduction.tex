\chapter{Preface}

\section{Introduction}
\label{sec:introduction}

This document contains the automatically generated listings of the
Isabelle sources for \mJava. \mJava{} is a reduced model of JavaCard,
dedicated to the study of the interaction of the source language, byte
code, the byte code verifier and the compiler. In order to make the
Isabelle sources more accessible, this introduction provides a brief
survey of the main concepts of \mJava.

The \mJava{} \textbf{source language} (see \charef{cha:j})
only comprises a part of the original JavaCard language. It models
features such as:
\begin{itemize}
\item The basic ``primitive types'' of Java 
\item Object orientation, in particular classes, and relevant
  relations on classes (subclass, widening)

\item Methods and method signatures
\item Inheritance and overriding of methods, dynamic binding

\item Representatives of ``relevant'' expressions and statements 
\item Generation and propagation of system exceptions
\end{itemize}

However, the following features are missing in \mJava{} wrt.{} JavaCard:
\begin{itemize}
\item Some primitive types (\texttt{byte, short})
\item Interfaces and related concepts, arrays
\item Most numeric operations, syntactic variants of statements
  (\texttt{do}-loop, \texttt{for}-loop)
\item Complex block structure, method bodies with multiple returns
\item Abrupt termination (\texttt{break, continue})
\item Class and method modifiers (such as \texttt{static} and
  \texttt{public/private} access modifiers)
\item User-defined exception classes and an explicit
  \texttt{throw}-statement. Exceptions cannot be caught.
\item A ``definite assignment'' check
\end{itemize}
In addition, features are missing that are not part of the JavaCard
language, such as multithreading and garbage collection. No attempt
has been made to model peculiarities of JavaCard such as the applet
firewall or the transaction mechanism.

For a more complete Isabelle model of JavaCard, the reader should
consult the Bali formalization
(\url{http://isabelle.in.tum.de/verificard/Bali/document.pdf}), which
models most of the source language features of JavaCard, however without
describing the bytecode level.

The central topics of the source language formalization are:
\begin{itemize}
\item Description of the structure of the ``runtime environment'', in
  particular structure of classes and the program state
\item Definition of syntax, typing rules and operational semantics of
  statements and expressions
\item Definition of ``conformity'' (characterizing type safety) and a
  type safety proof
\end{itemize}


The \mJava{} \textbf{virtual machine} (see \charef{cha:jvm})
corresponds rather directly to the source level, in the sense that the
same data types are supported and bytecode instructions required for
emulating the source level operations are provided. Again, only one
representative of different variants of instructions has been
selected; for example, there is only one comparison operator.  The
formalization of the bytecode level is purely descriptive (``no
theorems'') and rather brief as compared to the source level; all
questions related to type systems for and type correctness of bytecode
are dealt with in chapter on bytecode verification.

The problem of \textbf{bytecode verification} (see \charef{cha:bv}) is
dealt with in several stages:
\begin{itemize}
\item First, the notion of ``method type'' is introduced, which
  corresponds to the notion of ``type'' on the source level.
\item Well-typedness of instructions wrt. a method type is defined
  (see \secref{sec:BVSpec}). Roughly speaking, determining
  well-typedness is \emph{type checking}.
\item It is shown that bytecode that is well-typed in this sense can
  be safely executed -- a type soundness proof on the bytecode level
  (\secref{sec:BVSpecTypeSafe}). 
\item Given raw bytecode, one of the purposes of bytecode verification
  is to determine a method type that is well-typed according to the
  above definition. Roughly speaking, this is \emph{type inference}.
  The Isabelle formalization presents bytecode verification as an
  instance of an abstract dataflow algorithm (Kildall's algorithm, see
  \secrefs{sec:Kildall} to \ref{sec:JVM}).
%\item For \emph{lightweight bytecode verification}, type checking of
%  bytecode can be reduced to method types with small size. 
\end{itemize}

Bytecode verification in \mJava{} so far takes into account:
\begin{itemize}
\item Operations and branching instructions
\item Exceptions
\end{itemize}
Initialization during object creation is not accounted for in the
present document 
(see the formalization in
\url{http://isabelle.in.tum.de/verificard/obj-init/document.pdf}),
neither is the \texttt{jsr} instruction.


%%% Local Variables:
%%% mode: latex
%%% TeX-master: "root"
%%% End:
