
\documentclass[11pt,a4paper,twoside]{article}

\usepackage{comment}
\usepackage{latexsym,theorem}
\usepackage{isabelle,pdfsetup} %last one!


\renewcommand{\isamarkupheader}[1]{\section{#1}}
\newcommand{\isasymbollambda}{${\mathtt{\lambda}}$}

\parindent 0pt \parskip 0.5ex

\newcommand{\name}[1]{\textsf{#1}}

\newcommand{\idt}[1]{{\mathord{\mathit{#1}}}}
\newcommand{\var}[1]{{?\!#1}}
\DeclareMathSymbol{\dshsym}{\mathalpha}{letters}{"2D}
\newcommand{\dsh}{\dshsym}

\newenvironment{matharray}[1]{\[\begin{array}{#1}}{\end{array}\]}

\newcommand{\ty}{{\mathbin{:\,}}}
\newcommand{\To}{\to}
\newcommand{\dt}{{\mathpunct.}}
\newcommand{\all}[1]{\forall #1\dt\;}
\newcommand{\ex}[1]{\exists #1\dt\;}
\newcommand{\EX}[1]{\exists #1\dt\;}
\newcommand{\eps}[1]{\epsilon\; #1}
%\newcommand{\Forall}{\mathop\bigwedge}
\newcommand{\Forall}{\forall}
\newcommand{\All}[1]{\Forall #1\dt\;}
\newcommand{\ALL}[1]{\Forall #1\dt\;}
\newcommand{\Eps}[1]{\Epsilon #1\dt\;}
\newcommand{\Eq}{\mathbin{\,\equiv\,}}
\newcommand{\True}{\name{true}}
\newcommand{\False}{\name{false}}
\newcommand{\Impl}{\Rightarrow}
\newcommand{\And}{\;\land\;}
\newcommand{\Or}{\;\lor\;}
\newcommand{\Le}{\le}
\newcommand{\Lt}{\lt}
\newcommand{\lam}[1]{\mathop{\lambda} #1\dt\;}
\newcommand{\ap}{\mathbin{\!}}


\newcommand{\norm}[1]{\|\, #1\,\|}
\newcommand{\fnorm}[1]{\|\, #1\,\|}
\newcommand{\zero}{{\mathord{\mathbf {0}}}}
\newcommand{\plus}{{\mathbin{\;\mathtt {+}\;}}}
\newcommand{\minus}{{\mathbin{\;\mathtt {-}\;}}}
\newcommand{\mult}{{\mathbin{\;\mathbf {\odot}\;}}}
\newcommand{\1}{{\mathord{\mathrm{1}}}}
%\newcommand{\zero}{{\mathord{\small\sl\tt {<0>}}}}
%\newcommand{\plus}{{\mathbin{\;\small\sl\tt {[+]}\;}}}
%\newcommand{\minus}{{\mathbin{\;\small\sl\tt {[-]}\;}}}
%\newcommand{\mult}{{\mathbin{\;\small\sl\tt {[*]}\;}}}
%\newcommand{\1}{{\mathord{\mathrb{1}}}}
\newcommand{\fl}{{\mathord{\bf\underline{\phantom{i}}}}}
\renewcommand{\times}{\;{\mathbin{\cdot}}\;}
\newcommand{\qed}{\hfill~$\Box$}

\newcommand{\isasymbolprod}{$\mult$}
\newcommand{\isasymbolzero}{$\zero$}

%%% Local Variables: 
%%% mode: latex
%%% TeX-master: "root"
%%% End: 


\begin{document}

\pagestyle{headings}
\pagenumbering{arabic}

\title{The Hahn-Banach Theorem for Real Vectorspaces}
\author{Gertrud Bauer}
\maketitle

\begin{abstract}
The Hahn-Banach theorem is one of the most important theorems
of functional analysis. We present the fully formal proof of two versions of
the theorem, one for general linear spaces and one for normed spaces
as a corollary of the first. 

The first part contains the definition of basic notions of
linear algebra, such as vector spaces, subspaces, normed spaces,
continous linearforms, norm of functions and an order on
functions by domain extension.

The second part contains some lemmas about the supremum w.r.t. the
function order and the extension of a non-maximal function, 
which are needed for the proof of the main theorem.

The third part is the proof of the theorem in its two different versions.

\end{abstract}

\tableofcontents

\part {Basic notions}

\input{Bounds.tex}
\input{Aux.tex}
\input{VectorSpace.tex}
\input{Subspace.tex}
\input{NormedSpace.tex}
\input{Linearform.tex}
\input{FunctionOrder.tex}
\input{FunctionNorm.tex}
\input{ZornLemma.tex}

\part {Lemmas for the proof}

\input{HahnBanachSupLemmas.tex}
\input{HahnBanachExtLemmas.tex}

\part {The proof}

\input{HahnBanach.tex}
\bibliographystyle{abbrv}
\bibliography{bib}

\end{document}
