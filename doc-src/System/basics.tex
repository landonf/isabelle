
% $Id$

\chapter{Basic concepts}

The \emph{Isabelle System Manual} describes Isabelle together with
related tools and user interfaces as seen from an outside (operating
system oriented) view.  See the \emph{Isabelle Reference Manual} for
all internal {\ML} level user commands, on the other hand.

\medskip The Isabelle system level environment is based on a few
general concepts:
\begin{itemize}
\item The \emph{Isabelle settings mechanism}, which provides
  environment variables to all Isabelle programs (including tools and
  user interfaces).
\item \emph{Isabelle proper} (\ttindex{isabelle}), which runs logic
  sessions, both interactively or in batch mode. In particular,
  \texttt{isabelle} abstracts over the invocation of the actual {\ML}
  system to be used.
\item The \emph{Isabelle tools wrapper} (\ttindex{isatool}), which
  provides a generic startup platform for Isabelle related utilities.
  Thus tools automatically benefit from the settings mechanism.
  Furthermore, the shell's search path is kept clean from many small
  programs.
\item The \emph{Isabelle interface wrapper}
  (\ttindex{Isabelle}\footnote{Note the capital \texttt{I}!}), which
  provides some abstraction over the actual user interface to be used.
\end{itemize}

\medskip The beginning user would probably just run one of the
interfaces (by invoking the capital \texttt{Isabelle}), and maybe some
basic tools like \texttt{doc} (see \S\ref{sec:tool-doc}).  This
assumes that the system has already been installed properly, of
course.\footnote{In case you have to do this yourself, see the
  \ttindex{INSTALL} file in the top-level directory of the
  distribution. The information provided there should be sufficient as
  a start.}


\section{Isabelle settings} \label{sec:settings}

The Isabelle system heavily depends on the \emph{settings
  mechanism}\indexbold{settings}. Basically, this is just a collection
of environment variables, e.g.\ \texttt{ISABELLE_HOME},
\texttt{ML_SYSTEM}, \texttt{ML_HOME}.  These variables are \emph{not}
intended to be set directly from the shell!

Isabelle employs a somewhat more sophisticated scheme of
\emph{settings files} --- one for site-wide defaults, another for
optional user-specific modifications.  With all configuration
variables in just a few places, this is considered much more
maintainable and user-friendly than ordinary shell environment
variables.

In particular, we avoid the typical situation where prospective users
of a software package are told to put this and that in their shell
startup scripts, before being able to actually run it. Isabelle
requires none such administrative chores of its end-users --- the
executables can be invoked straight away. Usually, users would just
want to put the Isabelle \texttt{bin} directory into their shell's
search path, of course.


\subsection{Building the environment}

Whenever any of the Isabelle executables is run, their settings
environment is built as follows:

\begin{enumerate}
\item The special variable \settdx{ISABELLE_HOME} is determined
  automatically from the location of the binary that has been run.
  
  You should not try to set \texttt{ISABELLE_HOME} manually. Also note
  that the Isabelle executables have to be run from their original
  location in the distribution directory --- copying or linking them
  somewhere else just won't work!
  
\item The file \texttt{\$ISABELLE_HOME/etc/settings} ist run as a
  shell script with the auto-export option for variables enabled.
  
  This file typically contains a rather long list of shell variable
  assigments, thus providing the site-wide default settings.  The
  Isabelle distribution already contains a global settings file with
  sensible defaults for most variables. When installing the system,
  only a few of these have to be adapted (most likely
  \texttt{ML_SYSTEM} etc.).
  
\item The file \texttt{\$ISABELLE_HOME_USER/etc/settings} (if it
  exists) is run the same way as the site default settings. The
  variable \texttt{ISABELLE_HOME_USER} has already been set before ---
  usually to \texttt{\~\relax/isabelle}.
  
  Thus individual users may override the site-wide defaults. See also
  file \texttt{etc/user-settings.sample} in the distribution.
  Typically, a user settings file would contain only a few lines, just
  the assigments that are really needed.  One should definitely
  \emph{not} start with a full copy the central
  \texttt{\$ISABELLE_HOME/etc/settings}. This could cause very
  annoying maintainance problems later, when the Isabelle installation
  is updated or changed otherwise.

\end{enumerate}

Note that settings files are actually full GNU bash scripts. So one
may use complex shell commands, say \texttt{if} or \texttt{case}
statements to set variables depending on the system architecture or
other environment variables, for example. Such advanced features
should be added only with great care, though. In particular, external
environment references should be kept at a minimum.

\medskip A few variables are somewhat special:
\begin{itemize}
\item \settdx{ISABELLE} and \settdx{ISATOOL} are set automatically to
  the absolute path names of the \texttt{isabelle} and
  \texttt{isatool} executables, respectively.
  
\item \settdx{ISABELLE_PATH} and \settdx{ISABELLE_OUTPUT} will have
  the {\ML} system identifier (as specified by \texttt{ML_SYSTEM})
  automatically appended to their values.
\end{itemize}

\medskip The Isabelle settings scheme is basically quite simple, but
non-trivial.  For debugging purposes, the generated environment may be
inspected with the \texttt{getenv} utility, see
\S\ref{sec:tool-getenv}.


\subsection{Common variables}

Below is a reference of common Isabelle settings variables. Note that
the list is somewhat open-ended. Third-party utilities or interfaces
may add their own selection. Variables that are special in some sense
are marked with *.

\begin{description}
\item[\settdx{ISABELLE_HOME}*] is the location of the top-level
  Isabelle distribution directory. This is automatically determined
  from the Isabelle executable that has been invoked.  Do not try to
  set \texttt{ISABELLE_HOME} yourself from the shell.
  
\item[\settdx{ISABELLE_HOME_USER}] is the user-specific counterpart of
  \texttt{ISABELLE_HOME}. The default value is
  \texttt{\~\relax/isabelle}, under rare circumstances this may be
  changed in the global setting file.  Typically, the
  \texttt{ISABELLE_HOME_USER} directory mimics \texttt{ISABELLE_HOME}
  to some extend. In particular, site-wide defaults may be overridden
  by a private \texttt{etc/settings}.
  
\item[\settdx{ISABELLE}*, \settdx{ISATOOL}*] are automatically set to
  the full paths of the \texttt{isabelle} and \texttt{isatool}
  executables, respectively.  Thus other tools and scripts need not
  assume that the Isabelle \texttt{bin} directory is on the current
  search path of the shell.

\item[\settdx{ML_SYSTEM}, \settdx{ML_HOME}, \settdx{ML_OPTIONS}]
  specify the underlying {\ML} system to be used for Isabelle.  The
  choice of \texttt{ML_SYSTEM} identifiers is quite fixed, see the
  global \texttt{etc/settings} file for some examples. The actual
  compiler binary will be run from directory \texttt{ML_HOME}, with
  \texttt{ML_OPTIONS} as first arguments on the command line.
  
\item[\settdx{ISABELLE_PATH}*] is a list of directories (separated by
  colons) where Isabelle logic images may reside. Note that the
  \texttt{ML_SYSTEM} identifier is appended to each component
  automatically.
  
\item[\settdx{ISABELLE_OUTPUT}*] is a directory where output heap
  files should be stored by default. The \texttt{ML_SYSTEM} identifier
  is appended here, too.
  
\item[\settdx{ISABELLE_BROWSER_INFO}] is the directory where theory
  browser information (HTML and graph data) is stored (see also
  \S\ref{sec:info}).

\item[\settdx{ISABELLE_LOGIC}] specifies the default logic to load if
  none is given explicitely by the user --- e.g.\ when running
  \texttt{isabelle} directly, or some user interface.
  
\item[\settdx{ISABELLE_USEDIR_OPTIONS}] is implicitly prefixed to the
  command line of any \texttt{isatool usedir} invocation (see also
  \S\ref{sec:tool-usedir}). This typically contains compilation
  options for object-logics --- \texttt{usedir} is the basic utility
  that builds them (cf.\ the \texttt{IsaMakefile}s in the
  distribution).

\item[\settdx{ISABELLE_TOOLS}] is a colon separated list of
  directories that are scanned by \texttt{isatool} for utility
  programs (see also \S\ref{sec:isatool}).

\item[\settdx{ISABELLE_DOCS}] is a colon separated list of directories
  with documentation files.
  
\item[\settdx{DVI_VIEWER}] specifies the command to be used for
  displaying \texttt{dvi} files.
  
\item[\settdx{ISABELLE_INSTALL_FONTS}] determines the way that the
  Isabelle symbol fonts are installed into your currently running X11
  display server. X11 fonts are a non-trivial issue, see
  \S\ref{sec:tool-installfonts} for more information.
  
\item[\settdx{ISABELLE_TMP_PREFIX}] is the prefix from which any
  \texttt{isabelle} session derives an individual directory for
  temporary files.  The default value of \texttt{ISABELLE_TMP_PREFIX}
  is \texttt{/tmp/isabelle}; this should not need to be changed under
  normal circumstances.
  
\item[\settdx{ISABELLE_INTERFACE}] is an identifier that specifies the
  actual user interface that the capital \texttt{Isabelle} should
  invoke.  Currently available are \texttt{none}, \texttt{xterm} and
  \texttt{emacs}. See \S\ref{sec:interface} for more details.

\end{description}


\section{Isabelle proper --- \texttt{isabelle}}

The \ttindex{isabelle} executable runs logic sessions --- either
interactively or in batch mode. It provides an abstraction over the
underlying {\ML} system, and over the actual heap file locations. Its
usage is:
\begin{ttbox}
Usage: isabelle [OPTIONS] [INPUT] [OUTPUT]

  Options are:
    -e MLTEXT    pass MLTEXT to the ML session
    -m MODE      add print mode for output
    -q           non-interactive session
    -r           open heap file read-only
    -u           pass 'use"ROOT.ML";' to the ML session
    -w           reset write permissions on OUTPUT

  INPUT (default "\$ISABELLE_LOGIC") and OUTPUT specify in/out heaps.
  These are either names to be searched in the Isabelle path, or actual
  file names (containing at least one /).
  If INPUT is "RAW_ML_SYSTEM", just start the bare bones ML system.
\end{ttbox}
Input files without path specifications are looked up in the
\texttt{ISABELLE_PATH} setting, which may consist of multiple
components separated by colons --- these are tried in order.
Likewise, base names are relative to the directory specified by
\texttt{ISABELLE_OUTPUT}.  In any case, actual file locations may also
be given by including at least one \texttt{/} in the name (hint: use
\texttt{./} to refer to the current directory).


\subsection*{Options}

If the input heap file does not have write permission bits set, or the
\texttt{-r} option is given explicitely, then the session will be
read-only. That is, the {\ML} world cannot be committed back into the
logic image.  Otherwise, a writable session enables commits into
either the input file, or into an alternative output heap file (in
case this is given as the second argument).

The read-write state of sessions is determined at startup only, it
cannot be changed intermediately. Also note that heap images may
require considerable amounts of disk space. Users are responsible
themselves to dispose their heap files when they are no longer needed.

\medskip The \texttt{-w} option makes the output heap file read-only
after terminating the Isabelle session.  Thus subsequent invocations
cause logic image to be read-only automatically.

\medskip Using the \texttt{-e} option, arbitrary {\ML} code may be
passed to the Isabelle session from the command line. Multiple
\texttt{-e}'s are evaluated in order. Strange things may happen when
errorneous {\ML} code is given. Also make sure that commands are
terminated properly by semicolon.

\medskip The \texttt{-u} option is a shortcut for \texttt{-e}, passing
\texttt{use"ROOT.ML";} to the {\ML} session.

\medskip The \texttt{-m} option adds identifiers of print modes to be
made active for this session. Typically, this is used by some user
interface, for example to enable output of mathematical symbols from a
special screen font.

\medskip Isabelle normally enters an interactice {\ML} top-level loop
(after processing the \texttt{-e} texts). The \texttt{-q} option
inhibits this, thus providing a pure batch mode facility.


\subsection*{Examples}

Run an interactive session of the default object-logic (as specified
by the \texttt{ISABELLE_LOGIC} setting) like this:
\begin{ttbox}
isabelle
\end{ttbox}
Usually \texttt{ISABELLE_LOGIC} refers to one of the standard logic
images, which are read-only by default.  A writable session --- based
on \texttt{FOL}, but output to \texttt{Foo} (in the directory
specified by the \texttt{ISABELLE_OUTPUT} setting) --- may be invoked
as follows:
\begin{ttbox}
isabelle FOL Foo
\end{ttbox}
Ending this session normally (e.g.\ by typing control-D) dumps the
whole {\ML} system state into \texttt{Foo}. Be prepared for several
megabytes!

The \texttt{Foo} session may be continued later (still in writable
state) by:
\begin{ttbox}
isabelle Foo
\end{ttbox}
A read-only \texttt{Foo} session may be started by:
\begin{ttbox}
isabelle -r Foo
\end{ttbox}

\medskip The next example demonstrates batch execution of Isabelle. We
print a certain theorem of \texttt{FOL}:
\begin{ttbox}
isabelle -e "prth allE;" -q -r FOL
\end{ttbox}
Note that the output text will be usually interspersed with some
garbage produced by the {\ML} compiler.


\section{The Isabelle tools wrapper --- \texttt{isatool}} \label{sec:isatool}

All Isabelle related utilities are called via a common wrapper ---
\ttindex{isatool}:
\begin{ttbox}
Usage: isatool TOOL [ARGS ...]

  Start Isabelle utility program TOOL with ARGS. Pass "-?" to TOOL
  for more specific help.

  Available tools are:

    browser - Isabelle theory graph browser
    doc - view Isabelle documentation
    \dots
\end{ttbox}
Basically, Isabelle tools are ordinary executable scripts.  These are
run within the same settings environment as Isabelle proper, see
\S\ref{sec:settings}.  The set of available tools is collected by
isatool from the directories listed in the \texttt{ISABELLE_TOOLS}
setting.  Do not try to call the scripts directly. Neither should you
add the tools directories to your shell's search path.


\medskip See chapter~\ref{ch:tools} for descriptions of most utilities
that come with the Isabelle distributions. Third-parties may add their
own, of course.


\section{The Isabelle interface wrapper --- \texttt{Isabelle}} \label{sec:interface}

Isabelle is a generic theorem prover, even w.r.t.\ its user interface.
The \ttindex{Isabelle} command (note the capital \texttt{I}) provides
a uniform way for end-users to invoke a certain interface. Which one
to actually start is determined by the \settdx{ISABELLE_INTERFACE}
setting variable. Currently, the following are recognized:
\begin{ttdescription}
\item[none] is just a pass-through to \texttt{isabelle}. Thus
  \texttt{Isabelle} basically becomes an alias for \texttt{isabelle}.
  
\item[xterm] refers to a simple xterm based interface which is part of
  the Isabelle distribution.
  
\item[emacs] refers to David Aspinall's \emph{Isamode} for emacs.
  Isabelle just provides a wrapper for this, the actual Isamode
  distribution is available elsewhere.
\end{ttdescription}

The factory default for \texttt{ISABELLE_INTERFACE} is \texttt{xterm}.
This interface runs \texttt{isabelle} within its own xterm window.
Usually, display of mathematical symbols from the Isabelle font is
also enabled (see \S\ref{sec:tool-installfonts} for font configuration
issues).  Furthermore, different kinds of identifiers in logical terms
are highlighted appropriately, e.g.\ free variables in bold and bound
variables underlined.

There are some more options available.  Just pass \texttt{-?} to the
\texttt{xterm} interface to have its usage printed.
