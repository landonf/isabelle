
\chapter{Fonts and character encodings}

With the advent of print modes in Isabelle (see also The
\emph{Isabelle Reference Manual}, variant forms of output have become
very easy. As the canonical application of this feature, {\Pure} and
major object-logics (\FOL, \ZF, \HOL, \HOLCF) support optional output
(and input) of nice mathematical symbols out of the box.

Symbolic output is enabled by activating the \ttindex{symbols} print
mode. User interfaces (e.g.\ \texttt{isa-xterm}, see
\S\ref{sec:isa-xterm}) usually do this by default.

\medskip Of course, this requires special screen fonts (see
\S\ref{sec:tool-installfonts}). Furthermore, various {\ML} systems
cause some problems with non-\textsc{ascii} characters in literal
strings. These are avoided by the \texttt{symbolinput} filter (see
\S\ref{sec:tool-symbolinput}).

Both of these are invoked transparently in normal operation. So one
does not actually have to read the explanations below, unless
something fails to work.


\section{Telling X11 about Isabelle fonts --- \texttt{isatool installfonts}}
\label{sec:tool-installfonts}

The \tooldx{installfonts} utility ensures that your currently running
X11 display server (as determined by the \texttt{DISPLAY} environment
variable) knows about the Isabelle fonts. Its usage is:
\begin{ttbox}
Usage: isatool installfonts

  Install the isabelle fonts into your X11 server.
  (May be safely called repeatedly.)
\end{ttbox}
Note that this need not be called manually under normal circumstances
--- user interfaces depending on the Isabelle fonts will invoke
\texttt{installfonts} transparently.

\medskip As simple as this might appear to be, it is not! X11 fonts
are a surprisingly complicated matter. Depending on your network
structure, fonts might have to be installed differently. This has to
be specified via the \settdx{ISABELLE_INSTALLFONTS}.

\medskip In the simplest situation, X11 is used only locally, i.e.\ 
the client program (Isabelle) and the display server are run on the
same machine. Then most X11 display servers should be happy to being
just told about the directory where the fonts reside within the
Isabelle distribution:
\begin{ttbox}
ISABELLE_INSTALLFONTS="xset fp+ $ISABELLE_HOME/lib/fonts; xset fp rehash"
\end{ttbox}
The same also works for remote X11 sessions in a somewhat homogeneous
network, where the X11 display machine mounts the Isabelle
distribution under the same name as the client side.

\medskip Above method fails, if the display machine does have the font
files at the same location as the client. In case your server is a
full workstation with its own file system, you could in principle just
copy the fonts there and do an appropriate \texttt{xset~fp+} manually
before running the Isabelle interface. This is awkward, of course, but
is even \emph{impossible} for proper X terminals that do not have
their own file system.

A much better solution to this problem is to use an \emph{X11 font
  server}, offering the Isabelle font to the Net. In principle it is
relatively easy to setup one your own --- the program is called
\texttt{xfs} (or just \texttt{fs)}, see the \texttt{man} pages of your
system.

Your are very lucky, though, if you have a sensible connection to the
Internet. We already offer a world-wide X11 font service for the
Isabelle fonts. To have \texttt{installfonts} make use of this, just
set \texttt{ISABELLE_INSTALLFONTS} is follows:
\begin{ttbox}
ISABELLE_INSTALLFONTS="xset fp+ tcp/isafonts.informatik.tu-muenchen.de:7200"
\end{ttbox}



\section{Filtering non-ASCII characters --- \texttt{isatool symbolinput}}
\label{sec:tool-symbolinput}


%FIXME not yet
%\section{ --- \texttt{isatool showsymbols}}
%
%\begin{ttbox}
%\end{ttbox}
