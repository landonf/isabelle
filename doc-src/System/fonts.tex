
\chapter{Fonts and character encodings}

With the advent of print modes in Isabelle, variant forms of output
have become very easy. As the canonical application of this feature,
{\Pure} and major object-logics (\FOL, \ZF, \HOL, \HOLCF) support
optional input and output of nice mathematical symbols as an built-in
option.

Symbolic output is enabled by activating the \ttindex{symbols} print
mode. User interfaces (e.g.\ \texttt{isa-xterm}, see
\S\ref{sec:interface}) usually do this already by default.

\medskip Displaying non-standard characters requires special screen
fonts, of course. The \texttt{installfonts} utility takes care of
this, see \S\ref{sec:tool-installfonts}. Furthermore, some {\ML}
systems disallow non-\textsc{ascii} characters in literal strings.
This problem is avoided by the \texttt{symbolinput} filter (see
\S\ref{sec:tool-symbolinput}).

Both of these are invoked transparently in normal operation. So one
does not actually have to read the explanations below, unless
something fails to work.


\section{Telling X11 about the Isabelle fonts --- \texttt{isatool installfonts}}
\label{sec:tool-installfonts}

The \tooldx{installfonts} utility ensures that your currently running
X11 display server (as determined by the \texttt{DISPLAY} environment
variable) knows about the Isabelle fonts. Its usage is:
\begin{ttbox}
Usage: isatool installfonts

  Install the isabelle fonts into your X11 server.
  (May be safely called repeatedly.)
\end{ttbox}
Note that this need not be called manually under normal circumstances
--- user interfaces depending on the Isabelle fonts usually invoke
\texttt{installfonts} automatically.

\medskip As simple as this might appear to be, it is not! X11 fonts
are a surprisingly complicated matter. Depending on your network
structure, fonts might have to be installed differently. This has to
be specified via the \settdx{ISABELLE_INSTALLFONTS} variable in your
local settings.

\medskip In the simplest situation, X11 is used only locally, i.e.\ 
the client program (Isabelle) and the display server are run on the
same machine. In this case, most X11 display servers should be happy
by being told about the Isabelle fonts directory as follows:
\begin{ttbox}
ISABELLE_INSTALLFONTS="xset fp+{\thinspace}$ISABELLE_HOME/lib/fonts; xset fp rehash"
\end{ttbox}
The same also works for remote X11 sessions in a somewhat homogeneous
network, where the X11 display machine mounts the Isabelle
distribution under the same name as the client side.

\medskip Above method fails, though, if the display machine does have
the font files at the same location as the client. In case your server
is a full workstation with its own file system, you could in principle
just copy the fonts there and do an appropriate \texttt{xset~fp+}
manually before running the Isabelle interface. This is very awkward,
of course. It is even \emph{impossible} for proper X terminals that do
not have their own file system.

A much better solution is to have a \emph{font server} offer the
Isabelle fonts to any X display on the network.  There is already a
suitable server running at Munich. So in case you have a sensible
Internet connection, you may just attach yourself as follows:
\begin{ttbox}
ISABELLE_INSTALLFONTS="xset fp+ tcp/isafonts.informatik.tu-muenchen.de:7200"
\end{ttbox}

\medskip In the unfortunate case that neither local fonts work, nor
accessing our world-wide font service is practical, it might be best
to start your own in-house font service. This is in principle easy to
setup. The program is called \texttt{xfs} (or just \texttt{fs)}, see
the \texttt{man} pages of your system. There is an example
configuration available in the \texttt{lib/fontserver} directory of
the Isabelle distribution.



\section{Filtering non-ASCII characters --- \texttt{isatool symbolinput}}
\label{sec:tool-symbolinput}

Processing non-\textsc{ascii} text is notoriously difficult.  In
particular, some {\ML} systems reject character codes outside the
range 32--127 as part of literal string constants. In order to
circumvent such restrictions, Isabelle employs a general notation
where glyphs are referred by some symbolic name instead of their
actual encoding: Its general form is \verb|\<|$charname$\verb|>|.

The \tooldx{symbolinput} utility converts a input stream encoded
according to the standard Isabelle font layout into pure
\textsc{ascii} text. There is no usage --- \texttt{symbolinput} just
works from standard input to standard output, without any options
available.

\medskip For example, the non-\textsc{ascii} input string \texttt{"A
  $\land$ B $\lor$ C"} will be replaced by \verb|"A \\<and> B \\<or> C"|.
Note that the \verb|\| are escaped, accomodating concrete {\ML} string
syntax.

\medskip In many cases, it might be wise not to rely on symbolic
characters and avoid non-\textsc{ascii} text in files altogether. Then
symbolic syntax would be really optional, with always suitable
\textsc{ascii} representations available: In theory definitions
symbols appear only in mixfix annotations --- using the
\verb|\<|$charname$\verb|>| form, proof scripts are just left in plain
\textsc{ascii}.

Thus users with \textsc{ascii}-only facilities will still be able to
read your files.


%FIXME not yet
%\section{ --- \texttt{isatool showsymbols}}
%
%\begin{ttbox}
%\end{ttbox}
