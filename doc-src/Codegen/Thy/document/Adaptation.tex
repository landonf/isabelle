%
\begin{isabellebody}%
\def\isabellecontext{Adaptation}%
%
\isadelimtheory
%
\endisadelimtheory
%
\isatagtheory
\isacommand{theory}\isamarkupfalse%
\ Adaptation\isanewline
\isakeyword{imports}\ Setup\isanewline
\isakeyword{begin}%
\endisatagtheory
{\isafoldtheory}%
%
\isadelimtheory
\isanewline
%
\endisadelimtheory
%
\isadeliminvisible
\isanewline
%
\endisadeliminvisible
%
\isataginvisible
\isacommand{setup}\isamarkupfalse%
\ {\isacharverbatimopen}\ Code{\isacharunderscore}Target{\isachardot}extend{\isacharunderscore}target\ {\isacharparenleft}{\isachardoublequote}{\isasymSML}{\isachardoublequote}{\isacharcomma}\ {\isacharparenleft}{\isachardoublequote}SML{\isachardoublequote}{\isacharcomma}\ K\ I{\isacharparenright}{\isacharparenright}\ {\isacharverbatimclose}%
\endisataginvisible
{\isafoldinvisible}%
%
\isadeliminvisible
%
\endisadeliminvisible
%
\isamarkupsection{Adaptation to target languages \label{sec:adaptation}%
}
\isamarkuptrue%
%
\isamarkupsubsection{Adapting code generation%
}
\isamarkuptrue%
%
\begin{isamarkuptext}%
The aspects of code generation introduced so far have two aspects
  in common:

  \begin{itemize}
    \item They act uniformly, without reference to a specific
       target language.
    \item They are \emph{safe} in the sense that as long as you trust
       the code generator meta theory and implementation, you cannot
       produce programs that yield results which are not derivable
       in the logic.
  \end{itemize}

  \noindent In this section we will introduce means to \emph{adapt} the serialiser
  to a specific target language, i.e.~to print program fragments
  in a way which accommodates \qt{already existing} ingredients of
  a target language environment, for three reasons:

  \begin{itemize}
    \item improving readability and aesthetics of generated code
    \item gaining efficiency
    \item interface with language parts which have no direct counterpart
      in \isa{HOL} (say, imperative data structures)
  \end{itemize}

  \noindent Generally, you should avoid using those features yourself
  \emph{at any cost}:

  \begin{itemize}
    \item The safe configuration methods act uniformly on every target language,
      whereas for adaptation you have to treat each target language separately.
    \item Application is extremely tedious since there is no abstraction
      which would allow for a static check, making it easy to produce garbage.
    \item Subtle errors can be introduced unconsciously.
  \end{itemize}

  \noindent However, even if you ought refrain from setting up adaptation
  yourself, already the \isa{HOL} comes with some reasonable default
  adaptations (say, using target language list syntax).  There also some
  common adaptation cases which you can setup by importing particular
  library theories.  In order to understand these, we provide some clues here;
  these however are not supposed to replace a careful study of the sources.%
\end{isamarkuptext}%
\isamarkuptrue%
%
\isamarkupsubsection{The adaptation principle%
}
\isamarkuptrue%
%
\begin{isamarkuptext}%
Figure \ref{fig:adaptation} illustrates what \qt{adaptation} is conceptually
  supposed to be:

  \begin{figure}[here]
    \includegraphics{adaptation}
    \caption{The adaptation principle}
    \label{fig:adaptation}
  \end{figure}

  \noindent In the tame view, code generation acts as broker between
  \isa{logic}, \isa{intermediate\ language} and
  \isa{target\ language} by means of \isa{translation} and
  \isa{serialisation};  for the latter, the serialiser has to observe
  the structure of the \isa{language} itself plus some \isa{reserved}
  keywords which have to be avoided for generated code.
  However, if you consider \isa{adaptation} mechanisms, the code generated
  by the serializer is just the tip of the iceberg:

  \begin{itemize}
    \item \isa{serialisation} can be \emph{parametrised} such that
      logical entities are mapped to target-specific ones
      (e.g. target-specific list syntax,
        see also \secref{sec:adaptation_mechanisms})
    \item Such parametrisations can involve references to a
      target-specific standard \isa{library} (e.g. using
      the \isa{Haskell} \verb|Maybe| type instead
      of the \isa{HOL} \isa{option} type);
      if such are used, the corresponding identifiers
      (in our example, \verb|Maybe|, \verb|Nothing|
      and \verb|Just|) also have to be considered \isa{reserved}.
    \item Even more, the user can enrich the library of the
      target-language by providing code snippets
      (\qt{\isa{includes}}) which are prepended to
      any generated code (see \secref{sec:include});  this typically
      also involves further \isa{reserved} identifiers.
  \end{itemize}

  \noindent As figure \ref{fig:adaptation} illustrates, all these adaptation mechanisms
  have to act consistently;  it is at the discretion of the user
  to take care for this.%
\end{isamarkuptext}%
\isamarkuptrue%
%
\isamarkupsubsection{Common adaptation patterns%
}
\isamarkuptrue%
%
\begin{isamarkuptext}%
The \hyperlink{theory.HOL}{\mbox{\isa{HOL}}} \hyperlink{theory.Main}{\mbox{\isa{Main}}} theory already provides a code
  generator setup
  which should be suitable for most applications.  Common extensions
  and modifications are available by certain theories of the \isa{HOL}
  library; beside being useful in applications, they may serve
  as a tutorial for customising the code generator setup (see below
  \secref{sec:adaptation_mechanisms}).

  \begin{description}

    \item[\hyperlink{theory.Code-Integer}{\mbox{\isa{Code{\isacharunderscore}Integer}}}] represents \isa{HOL} integers by big
       integer literals in target languages.
    \item[\hyperlink{theory.Code-Char}{\mbox{\isa{Code{\isacharunderscore}Char}}}] represents \isa{HOL} characters by 
       character literals in target languages.
    \item[\hyperlink{theory.Code-Char-chr}{\mbox{\isa{Code{\isacharunderscore}Char{\isacharunderscore}chr}}}] like \isa{Code{\isacharunderscore}Char},
       but also offers treatment of character codes; includes
       \hyperlink{theory.Code-Char}{\mbox{\isa{Code{\isacharunderscore}Char}}}.
    \item[\hyperlink{theory.Efficient-Nat}{\mbox{\isa{Efficient{\isacharunderscore}Nat}}}] \label{eff_nat} implements natural numbers by integers,
       which in general will result in higher efficiency; pattern
       matching with \isa{{\isadigit{0}}} / \isa{Suc}
       is eliminated;  includes \hyperlink{theory.Code-Integer}{\mbox{\isa{Code{\isacharunderscore}Integer}}}
       and \hyperlink{theory.Code-Numeral}{\mbox{\isa{Code{\isacharunderscore}Numeral}}}.
    \item[\hyperlink{theory.Code-Numeral}{\mbox{\isa{Code{\isacharunderscore}Numeral}}}] provides an additional datatype
       \isa{index} which is mapped to target-language built-in integers.
       Useful for code setups which involve e.g. indexing of
       target-language arrays.
    \item[\hyperlink{theory.String}{\mbox{\isa{String}}}] provides an additional datatype
       \isa{String{\isachardot}literal} which is isomorphic to strings;
       \isa{String{\isachardot}literal}s are mapped to target-language strings.
       Useful for code setups which involve e.g. printing (error) messages.

  \end{description}

  \begin{warn}
    When importing any of these theories, they should form the last
    items in an import list.  Since these theories adapt the
    code generator setup in a non-conservative fashion,
    strange effects may occur otherwise.
  \end{warn}%
\end{isamarkuptext}%
\isamarkuptrue%
%
\isamarkupsubsection{Parametrising serialisation \label{sec:adaptation_mechanisms}%
}
\isamarkuptrue%
%
\begin{isamarkuptext}%
Consider the following function and its corresponding
  SML code:%
\end{isamarkuptext}%
\isamarkuptrue%
%
\isadelimquote
%
\endisadelimquote
%
\isatagquote
\isacommand{primrec}\isamarkupfalse%
\ in{\isacharunderscore}interval\ {\isacharcolon}{\isacharcolon}\ {\isachardoublequoteopen}nat\ {\isasymtimes}\ nat\ {\isasymRightarrow}\ nat\ {\isasymRightarrow}\ bool{\isachardoublequoteclose}\ \isakeyword{where}\isanewline
\ \ {\isachardoublequoteopen}in{\isacharunderscore}interval\ {\isacharparenleft}k{\isacharcomma}\ l{\isacharparenright}\ n\ {\isasymlongleftrightarrow}\ k\ {\isasymle}\ n\ {\isasymand}\ n\ {\isasymle}\ l{\isachardoublequoteclose}%
\endisatagquote
{\isafoldquote}%
%
\isadelimquote
%
\endisadelimquote
%
\isadeliminvisible
%
\endisadeliminvisible
%
\isataginvisible
%
\endisataginvisible
{\isafoldinvisible}%
%
\isadeliminvisible
%
\endisadeliminvisible
%
\isadelimquote
%
\endisadelimquote
%
\isatagquote
%
\begin{isamarkuptext}%
\isatypewriter%
\noindent%
\hspace*{0pt}structure Example :~sig\\
\hspace*{0pt} ~datatype nat = Zero{\char95}nat | Suc of nat\\
\hspace*{0pt} ~datatype boola = True | False\\
\hspace*{0pt} ~val anda :~boola -> boola -> boola\\
\hspace*{0pt} ~val less{\char95}nat :~nat -> nat -> boola\\
\hspace*{0pt} ~val less{\char95}eq{\char95}nat :~nat -> nat -> boola\\
\hspace*{0pt} ~val in{\char95}interval :~nat * nat -> nat -> boola\\
\hspace*{0pt}end = struct\\
\hspace*{0pt}\\
\hspace*{0pt}datatype nat = Zero{\char95}nat | Suc of nat;\\
\hspace*{0pt}\\
\hspace*{0pt}datatype boola = True | False;\\
\hspace*{0pt}\\
\hspace*{0pt}fun anda p True = p\\
\hspace*{0pt} ~| anda p False = False\\
\hspace*{0pt} ~| anda True p = p\\
\hspace*{0pt} ~| anda False p = False;\\
\hspace*{0pt}\\
\hspace*{0pt}fun less{\char95}nat m (Suc n) = less{\char95}eq{\char95}nat m n\\
\hspace*{0pt} ~| less{\char95}nat n Zero{\char95}nat = False\\
\hspace*{0pt}and less{\char95}eq{\char95}nat (Suc m) n = less{\char95}nat m n\\
\hspace*{0pt} ~| less{\char95}eq{\char95}nat Zero{\char95}nat n = True;\\
\hspace*{0pt}\\
\hspace*{0pt}fun in{\char95}interval (k,~l) n = anda (less{\char95}eq{\char95}nat k n) (less{\char95}eq{\char95}nat n l);\\
\hspace*{0pt}\\
\hspace*{0pt}end;~(*struct Example*)%
\end{isamarkuptext}%
\isamarkuptrue%
%
\endisatagquote
{\isafoldquote}%
%
\isadelimquote
%
\endisadelimquote
%
\begin{isamarkuptext}%
\noindent Though this is correct code, it is a little bit unsatisfactory:
  boolean values and operators are materialised as distinguished
  entities with have nothing to do with the SML-built-in notion
  of \qt{bool}.  This results in less readable code;
  additionally, eager evaluation may cause programs to
  loop or break which would perfectly terminate when
  the existing SML \verb|bool| would be used.  To map
  the HOL \isa{bool} on SML \verb|bool|, we may use
  \qn{custom serialisations}:%
\end{isamarkuptext}%
\isamarkuptrue%
%
\isadelimquotett
%
\endisadelimquotett
%
\isatagquotett
\isacommand{code{\isacharunderscore}type}\isamarkupfalse%
\ bool\isanewline
\ \ {\isacharparenleft}SML\ {\isachardoublequoteopen}bool{\isachardoublequoteclose}{\isacharparenright}\isanewline
\isacommand{code{\isacharunderscore}const}\isamarkupfalse%
\ True\ \isakeyword{and}\ False\ \isakeyword{and}\ {\isachardoublequoteopen}op\ {\isasymand}{\isachardoublequoteclose}\isanewline
\ \ {\isacharparenleft}SML\ {\isachardoublequoteopen}true{\isachardoublequoteclose}\ \isakeyword{and}\ {\isachardoublequoteopen}false{\isachardoublequoteclose}\ \isakeyword{and}\ {\isachardoublequoteopen}{\isacharunderscore}\ andalso\ {\isacharunderscore}{\isachardoublequoteclose}{\isacharparenright}%
\endisatagquotett
{\isafoldquotett}%
%
\isadelimquotett
%
\endisadelimquotett
%
\begin{isamarkuptext}%
\noindent The \hyperlink{command.code-type}{\mbox{\isa{\isacommand{code{\isacharunderscore}type}}}} command takes a type constructor
  as arguments together with a list of custom serialisations.
  Each custom serialisation starts with a target language
  identifier followed by an expression, which during
  code serialisation is inserted whenever the type constructor
  would occur.  For constants, \hyperlink{command.code-const}{\mbox{\isa{\isacommand{code{\isacharunderscore}const}}}} implements
  the corresponding mechanism.  Each ``\verb|_|'' in
  a serialisation expression is treated as a placeholder
  for the type constructor's (the constant's) arguments.%
\end{isamarkuptext}%
\isamarkuptrue%
%
\isadelimquote
%
\endisadelimquote
%
\isatagquote
%
\begin{isamarkuptext}%
\isatypewriter%
\noindent%
\hspace*{0pt}structure Example :~sig\\
\hspace*{0pt} ~datatype nat = Zero{\char95}nat | Suc of nat\\
\hspace*{0pt} ~val less{\char95}nat :~nat -> nat -> bool\\
\hspace*{0pt} ~val less{\char95}eq{\char95}nat :~nat -> nat -> bool\\
\hspace*{0pt} ~val in{\char95}interval :~nat * nat -> nat -> bool\\
\hspace*{0pt}end = struct\\
\hspace*{0pt}\\
\hspace*{0pt}datatype nat = Zero{\char95}nat | Suc of nat;\\
\hspace*{0pt}\\
\hspace*{0pt}fun less{\char95}nat m (Suc n) = less{\char95}eq{\char95}nat m n\\
\hspace*{0pt} ~| less{\char95}nat n Zero{\char95}nat = false\\
\hspace*{0pt}and less{\char95}eq{\char95}nat (Suc m) n = less{\char95}nat m n\\
\hspace*{0pt} ~| less{\char95}eq{\char95}nat Zero{\char95}nat n = true;\\
\hspace*{0pt}\\
\hspace*{0pt}fun in{\char95}interval (k,~l) n = (less{\char95}eq{\char95}nat k n) andalso (less{\char95}eq{\char95}nat n l);\\
\hspace*{0pt}\\
\hspace*{0pt}end;~(*struct Example*)%
\end{isamarkuptext}%
\isamarkuptrue%
%
\endisatagquote
{\isafoldquote}%
%
\isadelimquote
%
\endisadelimquote
%
\begin{isamarkuptext}%
\noindent This still is not perfect: the parentheses
  around the \qt{andalso} expression are superfluous.
  Though the serialiser
  by no means attempts to imitate the rich Isabelle syntax
  framework, it provides some common idioms, notably
  associative infixes with precedences which may be used here:%
\end{isamarkuptext}%
\isamarkuptrue%
%
\isadelimquotett
%
\endisadelimquotett
%
\isatagquotett
\isacommand{code{\isacharunderscore}const}\isamarkupfalse%
\ {\isachardoublequoteopen}op\ {\isasymand}{\isachardoublequoteclose}\isanewline
\ \ {\isacharparenleft}SML\ \isakeyword{infixl}\ {\isadigit{1}}\ {\isachardoublequoteopen}andalso{\isachardoublequoteclose}{\isacharparenright}%
\endisatagquotett
{\isafoldquotett}%
%
\isadelimquotett
%
\endisadelimquotett
%
\isadelimquote
%
\endisadelimquote
%
\isatagquote
%
\begin{isamarkuptext}%
\isatypewriter%
\noindent%
\hspace*{0pt}structure Example :~sig\\
\hspace*{0pt} ~datatype nat = Zero{\char95}nat | Suc of nat\\
\hspace*{0pt} ~val less{\char95}nat :~nat -> nat -> bool\\
\hspace*{0pt} ~val less{\char95}eq{\char95}nat :~nat -> nat -> bool\\
\hspace*{0pt} ~val in{\char95}interval :~nat * nat -> nat -> bool\\
\hspace*{0pt}end = struct\\
\hspace*{0pt}\\
\hspace*{0pt}datatype nat = Zero{\char95}nat | Suc of nat;\\
\hspace*{0pt}\\
\hspace*{0pt}fun less{\char95}nat m (Suc n) = less{\char95}eq{\char95}nat m n\\
\hspace*{0pt} ~| less{\char95}nat n Zero{\char95}nat = false\\
\hspace*{0pt}and less{\char95}eq{\char95}nat (Suc m) n = less{\char95}nat m n\\
\hspace*{0pt} ~| less{\char95}eq{\char95}nat Zero{\char95}nat n = true;\\
\hspace*{0pt}\\
\hspace*{0pt}fun in{\char95}interval (k,~l) n = less{\char95}eq{\char95}nat k n andalso less{\char95}eq{\char95}nat n l;\\
\hspace*{0pt}\\
\hspace*{0pt}end;~(*struct Example*)%
\end{isamarkuptext}%
\isamarkuptrue%
%
\endisatagquote
{\isafoldquote}%
%
\isadelimquote
%
\endisadelimquote
%
\begin{isamarkuptext}%
\noindent The attentive reader may ask how we assert that no generated
  code will accidentally overwrite.  For this reason the serialiser has
  an internal table of identifiers which have to be avoided to be used
  for new declarations.  Initially, this table typically contains the
  keywords of the target language.  It can be extended manually, thus avoiding
  accidental overwrites, using the \hyperlink{command.code-reserved}{\mbox{\isa{\isacommand{code{\isacharunderscore}reserved}}}} command:%
\end{isamarkuptext}%
\isamarkuptrue%
%
\isadelimquote
%
\endisadelimquote
%
\isatagquote
\isacommand{code{\isacharunderscore}reserved}\isamarkupfalse%
\ {\isachardoublequoteopen}{\isasymSML}{\isachardoublequoteclose}\ bool\ true\ false\ andalso%
\endisatagquote
{\isafoldquote}%
%
\isadelimquote
%
\endisadelimquote
%
\begin{isamarkuptext}%
\noindent Next, we try to map HOL pairs to SML pairs, using the
  infix ``\verb|*|'' type constructor and parentheses:%
\end{isamarkuptext}%
\isamarkuptrue%
%
\isadeliminvisible
%
\endisadeliminvisible
%
\isataginvisible
%
\endisataginvisible
{\isafoldinvisible}%
%
\isadeliminvisible
%
\endisadeliminvisible
%
\isadelimquotett
%
\endisadelimquotett
%
\isatagquotett
\isacommand{code{\isacharunderscore}type}\isamarkupfalse%
\ {\isacharasterisk}\isanewline
\ \ {\isacharparenleft}SML\ \isakeyword{infix}\ {\isadigit{2}}\ {\isachardoublequoteopen}{\isacharasterisk}{\isachardoublequoteclose}{\isacharparenright}\isanewline
\isacommand{code{\isacharunderscore}const}\isamarkupfalse%
\ Pair\isanewline
\ \ {\isacharparenleft}SML\ {\isachardoublequoteopen}{\isacharbang}{\isacharparenleft}{\isacharparenleft}{\isacharunderscore}{\isacharparenright}{\isacharcomma}{\isacharslash}\ {\isacharparenleft}{\isacharunderscore}{\isacharparenright}{\isacharparenright}{\isachardoublequoteclose}{\isacharparenright}%
\endisatagquotett
{\isafoldquotett}%
%
\isadelimquotett
%
\endisadelimquotett
%
\begin{isamarkuptext}%
\noindent The initial bang ``\verb|!|'' tells the serialiser
  never to put
  parentheses around the whole expression (they are already present),
  while the parentheses around argument place holders
  tell not to put parentheses around the arguments.
  The slash ``\verb|/|'' (followed by arbitrary white space)
  inserts a space which may be used as a break if necessary
  during pretty printing.

  These examples give a glimpse what mechanisms
  custom serialisations provide; however their usage
  requires careful thinking in order not to introduce
  inconsistencies -- or, in other words:
  custom serialisations are completely axiomatic.

  A further noteworthy details is that any special
  character in a custom serialisation may be quoted
  using ``\verb|'|''; thus, in
  ``\verb|fn '_ => _|'' the first
  ``\verb|_|'' is a proper underscore while the
  second ``\verb|_|'' is a placeholder.%
\end{isamarkuptext}%
\isamarkuptrue%
%
\isamarkupsubsection{\isa{Haskell} serialisation%
}
\isamarkuptrue%
%
\begin{isamarkuptext}%
For convenience, the default
  \isa{HOL} setup for \isa{Haskell} maps the \isa{eq} class to
  its counterpart in \isa{Haskell}, giving custom serialisations
  for the class \isa{eq} (by command \hyperlink{command.code-class}{\mbox{\isa{\isacommand{code{\isacharunderscore}class}}}}) and its operation
  \isa{eq{\isacharunderscore}class{\isachardot}eq}%
\end{isamarkuptext}%
\isamarkuptrue%
%
\isadelimquotett
%
\endisadelimquotett
%
\isatagquotett
\isacommand{code{\isacharunderscore}class}\isamarkupfalse%
\ eq\isanewline
\ \ {\isacharparenleft}Haskell\ {\isachardoublequoteopen}Eq{\isachardoublequoteclose}{\isacharparenright}\isanewline
\isanewline
\isacommand{code{\isacharunderscore}const}\isamarkupfalse%
\ {\isachardoublequoteopen}op\ {\isacharequal}{\isachardoublequoteclose}\isanewline
\ \ {\isacharparenleft}Haskell\ \isakeyword{infixl}\ {\isadigit{4}}\ {\isachardoublequoteopen}{\isacharequal}{\isacharequal}{\isachardoublequoteclose}{\isacharparenright}%
\endisatagquotett
{\isafoldquotett}%
%
\isadelimquotett
%
\endisadelimquotett
%
\begin{isamarkuptext}%
\noindent A problem now occurs whenever a type which
  is an instance of \isa{eq} in \isa{HOL} is mapped
  on a \isa{Haskell}-built-in type which is also an instance
  of \isa{Haskell} \isa{Eq}:%
\end{isamarkuptext}%
\isamarkuptrue%
%
\isadelimquote
%
\endisadelimquote
%
\isatagquote
\isacommand{typedecl}\isamarkupfalse%
\ bar\isanewline
\isanewline
\isacommand{instantiation}\isamarkupfalse%
\ bar\ {\isacharcolon}{\isacharcolon}\ eq\isanewline
\isakeyword{begin}\isanewline
\isanewline
\isacommand{definition}\isamarkupfalse%
\ {\isachardoublequoteopen}eq{\isacharunderscore}class{\isachardot}eq\ {\isacharparenleft}x{\isasymColon}bar{\isacharparenright}\ y\ {\isasymlongleftrightarrow}\ x\ {\isacharequal}\ y{\isachardoublequoteclose}\isanewline
\isanewline
\isacommand{instance}\isamarkupfalse%
\ \isacommand{by}\isamarkupfalse%
\ default\ {\isacharparenleft}simp\ add{\isacharcolon}\ eq{\isacharunderscore}bar{\isacharunderscore}def{\isacharparenright}\isanewline
\isanewline
\isacommand{end}\isamarkupfalse%
%
\endisatagquote
{\isafoldquote}%
%
\isadelimquote
%
\endisadelimquote
%
\isadelimquotett
\ %
\endisadelimquotett
%
\isatagquotett
\isacommand{code{\isacharunderscore}type}\isamarkupfalse%
\ bar\isanewline
\ \ {\isacharparenleft}Haskell\ {\isachardoublequoteopen}Integer{\isachardoublequoteclose}{\isacharparenright}%
\endisatagquotett
{\isafoldquotett}%
%
\isadelimquotett
%
\endisadelimquotett
%
\begin{isamarkuptext}%
\noindent The code generator would produce
  an additional instance, which of course is rejected by the \isa{Haskell}
  compiler.
  To suppress this additional instance, use
  \isa{code{\isacharunderscore}instance}:%
\end{isamarkuptext}%
\isamarkuptrue%
%
\isadelimquotett
%
\endisadelimquotett
%
\isatagquotett
\isacommand{code{\isacharunderscore}instance}\isamarkupfalse%
\ bar\ {\isacharcolon}{\isacharcolon}\ eq\isanewline
\ \ {\isacharparenleft}Haskell\ {\isacharminus}{\isacharparenright}%
\endisatagquotett
{\isafoldquotett}%
%
\isadelimquotett
%
\endisadelimquotett
%
\isamarkupsubsection{Enhancing the target language context \label{sec:include}%
}
\isamarkuptrue%
%
\begin{isamarkuptext}%
In rare cases it is necessary to \emph{enrich} the context of a
  target language;  this is accomplished using the \hyperlink{command.code-include}{\mbox{\isa{\isacommand{code{\isacharunderscore}include}}}}
  command:%
\end{isamarkuptext}%
\isamarkuptrue%
%
\isadelimquotett
%
\endisadelimquotett
%
\isatagquotett
\isacommand{code{\isacharunderscore}include}\isamarkupfalse%
\ Haskell\ {\isachardoublequoteopen}Errno{\isachardoublequoteclose}\isanewline
{\isacharverbatimopen}errno\ i\ {\isacharequal}\ error\ {\isacharparenleft}{\isachardoublequote}Error\ number{\isacharcolon}\ {\isachardoublequote}\ {\isacharplus}{\isacharplus}\ show\ i{\isacharparenright}{\isacharverbatimclose}\isanewline
\isanewline
\isacommand{code{\isacharunderscore}reserved}\isamarkupfalse%
\ Haskell\ Errno%
\endisatagquotett
{\isafoldquotett}%
%
\isadelimquotett
%
\endisadelimquotett
%
\begin{isamarkuptext}%
\noindent Such named \isa{include}s are then prepended to every generated code.
  Inspect such code in order to find out how \hyperlink{command.code-include}{\mbox{\isa{\isacommand{code{\isacharunderscore}include}}}} behaves
  with respect to a particular target language.%
\end{isamarkuptext}%
\isamarkuptrue%
%
\isadelimtheory
%
\endisadelimtheory
%
\isatagtheory
\isacommand{end}\isamarkupfalse%
%
\endisatagtheory
{\isafoldtheory}%
%
\isadelimtheory
%
\endisadelimtheory
\isanewline
\end{isabellebody}%
%%% Local Variables:
%%% mode: latex
%%% TeX-master: "root"
%%% End:
