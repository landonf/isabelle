%
\begin{isabellebody}%
\def\isabellecontext{Program}%
%
\isadelimtheory
%
\endisadelimtheory
%
\isatagtheory
\isacommand{theory}\isamarkupfalse%
\ Program\isanewline
\isakeyword{imports}\ Introduction\isanewline
\isakeyword{begin}%
\endisatagtheory
{\isafoldtheory}%
%
\isadelimtheory
%
\endisadelimtheory
%
\isamarkupsection{Turning Theories into Programs \label{sec:program}%
}
\isamarkuptrue%
%
\isamarkupsubsection{The \isa{Isabelle{\isacharslash}HOL} default setup%
}
\isamarkuptrue%
%
\begin{isamarkuptext}%
We have already seen how by default equations stemming from
  \hyperlink{command.definition}{\mbox{\isa{\isacommand{definition}}}}/\hyperlink{command.primrec}{\mbox{\isa{\isacommand{primrec}}}}/\hyperlink{command.fun}{\mbox{\isa{\isacommand{fun}}}}
  statements are used for code generation.  This default behaviour
  can be changed, e.g. by providing different code equations.
  All kinds of customisation shown in this section is \emph{safe}
  in the sense that the user does not have to worry about
  correctness -- all programs generatable that way are partially
  correct.%
\end{isamarkuptext}%
\isamarkuptrue%
%
\isamarkupsubsection{Selecting code equations%
}
\isamarkuptrue%
%
\begin{isamarkuptext}%
Coming back to our introductory example, we
  could provide an alternative code equations for \isa{dequeue}
  explicitly:%
\end{isamarkuptext}%
\isamarkuptrue%
%
\isadelimquote
%
\endisadelimquote
%
\isatagquote
\isacommand{lemma}\isamarkupfalse%
\ {\isacharbrackleft}code{\isacharbrackright}{\isacharcolon}\isanewline
\ \ {\isachardoublequoteopen}dequeue\ {\isacharparenleft}AQueue\ xs\ {\isacharbrackleft}{\isacharbrackright}{\isacharparenright}\ {\isacharequal}\isanewline
\ \ \ \ \ {\isacharparenleft}if\ xs\ {\isacharequal}\ {\isacharbrackleft}{\isacharbrackright}\ then\ {\isacharparenleft}None{\isacharcomma}\ AQueue\ {\isacharbrackleft}{\isacharbrackright}\ {\isacharbrackleft}{\isacharbrackright}{\isacharparenright}\isanewline
\ \ \ \ \ \ \ else\ dequeue\ {\isacharparenleft}AQueue\ {\isacharbrackleft}{\isacharbrackright}\ {\isacharparenleft}rev\ xs{\isacharparenright}{\isacharparenright}{\isacharparenright}{\isachardoublequoteclose}\isanewline
\ \ {\isachardoublequoteopen}dequeue\ {\isacharparenleft}AQueue\ xs\ {\isacharparenleft}y\ {\isacharhash}\ ys{\isacharparenright}{\isacharparenright}\ {\isacharequal}\isanewline
\ \ \ \ \ {\isacharparenleft}Some\ y{\isacharcomma}\ AQueue\ xs\ ys{\isacharparenright}{\isachardoublequoteclose}\isanewline
\ \ \isacommand{by}\isamarkupfalse%
\ {\isacharparenleft}cases\ xs{\isacharcomma}\ simp{\isacharunderscore}all{\isacharparenright}\ {\isacharparenleft}cases\ {\isachardoublequoteopen}rev\ xs{\isachardoublequoteclose}{\isacharcomma}\ simp{\isacharunderscore}all{\isacharparenright}%
\endisatagquote
{\isafoldquote}%
%
\isadelimquote
%
\endisadelimquote
%
\begin{isamarkuptext}%
\noindent The annotation \isa{{\isacharbrackleft}code{\isacharbrackright}} is an \isa{Isar}
  \isa{attribute} which states that the given theorems should be
  considered as code equations for a \isa{fun} statement --
  the corresponding constant is determined syntactically.  The resulting code:%
\end{isamarkuptext}%
\isamarkuptrue%
%
\isadelimquote
%
\endisadelimquote
%
\isatagquote
%
\begin{isamarkuptext}%
\isatypewriter%
\noindent%
\hspace*{0pt}dequeue ::~forall a.~Queue a -> (Maybe a,~Queue a);\\
\hspace*{0pt}dequeue (AQueue xs (y :~ys)) = (Just y,~AQueue xs ys);\\
\hspace*{0pt}dequeue (AQueue xs []) =\\
\hspace*{0pt} ~(if nulla xs then (Nothing,~AQueue [] [])\\
\hspace*{0pt} ~~~else dequeue (AQueue [] (rev xs)));%
\end{isamarkuptext}%
\isamarkuptrue%
%
\endisatagquote
{\isafoldquote}%
%
\isadelimquote
%
\endisadelimquote
%
\begin{isamarkuptext}%
\noindent You may note that the equality test \isa{xs\ {\isacharequal}\ {\isacharbrackleft}{\isacharbrackright}} has been
  replaced by the predicate \isa{null\ xs}.  This is due to the default
  setup in the \qn{preprocessor} to be discussed further below (\secref{sec:preproc}).

  Changing the default constructor set of datatypes is also
  possible.  See \secref{sec:datatypes} for an example.

  As told in \secref{sec:concept}, code generation is based
  on a structured collection of code theorems.
  For explorative purpose, this collection
  may be inspected using the \hyperlink{command.code-thms}{\mbox{\isa{\isacommand{code{\isacharunderscore}thms}}}} command:%
\end{isamarkuptext}%
\isamarkuptrue%
%
\isadelimquote
%
\endisadelimquote
%
\isatagquote
\isacommand{code{\isacharunderscore}thms}\isamarkupfalse%
\ dequeue%
\endisatagquote
{\isafoldquote}%
%
\isadelimquote
%
\endisadelimquote
%
\begin{isamarkuptext}%
\noindent prints a table with \emph{all} code equations
  for \isa{dequeue}, including
  \emph{all} code equations those equations depend
  on recursively.
  
  Similarly, the \hyperlink{command.code-deps}{\mbox{\isa{\isacommand{code{\isacharunderscore}deps}}}} command shows a graph
  visualising dependencies between code equations.%
\end{isamarkuptext}%
\isamarkuptrue%
%
\isamarkupsubsection{\isa{class} and \isa{instantiation}%
}
\isamarkuptrue%
%
\begin{isamarkuptext}%
Concerning type classes and code generation, let us examine an example
  from abstract algebra:%
\end{isamarkuptext}%
\isamarkuptrue%
%
\isadelimquote
%
\endisadelimquote
%
\isatagquote
\isacommand{class}\isamarkupfalse%
\ semigroup\ {\isacharequal}\isanewline
\ \ \isakeyword{fixes}\ mult\ {\isacharcolon}{\isacharcolon}\ {\isachardoublequoteopen}{\isacharprime}a\ {\isasymRightarrow}\ {\isacharprime}a\ {\isasymRightarrow}\ {\isacharprime}a{\isachardoublequoteclose}\ {\isacharparenleft}\isakeyword{infixl}\ {\isachardoublequoteopen}{\isasymotimes}{\isachardoublequoteclose}\ {\isadigit{7}}{\isadigit{0}}{\isacharparenright}\isanewline
\ \ \isakeyword{assumes}\ assoc{\isacharcolon}\ {\isachardoublequoteopen}{\isacharparenleft}x\ {\isasymotimes}\ y{\isacharparenright}\ {\isasymotimes}\ z\ {\isacharequal}\ x\ {\isasymotimes}\ {\isacharparenleft}y\ {\isasymotimes}\ z{\isacharparenright}{\isachardoublequoteclose}\isanewline
\isanewline
\isacommand{class}\isamarkupfalse%
\ monoid\ {\isacharequal}\ semigroup\ {\isacharplus}\isanewline
\ \ \isakeyword{fixes}\ neutral\ {\isacharcolon}{\isacharcolon}\ {\isacharprime}a\ {\isacharparenleft}{\isachardoublequoteopen}{\isasymone}{\isachardoublequoteclose}{\isacharparenright}\isanewline
\ \ \isakeyword{assumes}\ neutl{\isacharcolon}\ {\isachardoublequoteopen}{\isasymone}\ {\isasymotimes}\ x\ {\isacharequal}\ x{\isachardoublequoteclose}\isanewline
\ \ \ \ \isakeyword{and}\ neutr{\isacharcolon}\ {\isachardoublequoteopen}x\ {\isasymotimes}\ {\isasymone}\ {\isacharequal}\ x{\isachardoublequoteclose}\isanewline
\isanewline
\isacommand{instantiation}\isamarkupfalse%
\ nat\ {\isacharcolon}{\isacharcolon}\ monoid\isanewline
\isakeyword{begin}\isanewline
\isanewline
\isacommand{primrec}\isamarkupfalse%
\ mult{\isacharunderscore}nat\ \isakeyword{where}\isanewline
\ \ \ \ {\isachardoublequoteopen}{\isadigit{0}}\ {\isasymotimes}\ n\ {\isacharequal}\ {\isacharparenleft}{\isadigit{0}}{\isasymColon}nat{\isacharparenright}{\isachardoublequoteclose}\isanewline
\ \ {\isacharbar}\ {\isachardoublequoteopen}Suc\ m\ {\isasymotimes}\ n\ {\isacharequal}\ n\ {\isacharplus}\ m\ {\isasymotimes}\ n{\isachardoublequoteclose}\isanewline
\isanewline
\isacommand{definition}\isamarkupfalse%
\ neutral{\isacharunderscore}nat\ \isakeyword{where}\isanewline
\ \ {\isachardoublequoteopen}{\isasymone}\ {\isacharequal}\ Suc\ {\isadigit{0}}{\isachardoublequoteclose}\isanewline
\isanewline
\isacommand{lemma}\isamarkupfalse%
\ add{\isacharunderscore}mult{\isacharunderscore}distrib{\isacharcolon}\isanewline
\ \ \isakeyword{fixes}\ n\ m\ q\ {\isacharcolon}{\isacharcolon}\ nat\isanewline
\ \ \isakeyword{shows}\ {\isachardoublequoteopen}{\isacharparenleft}n\ {\isacharplus}\ m{\isacharparenright}\ {\isasymotimes}\ q\ {\isacharequal}\ n\ {\isasymotimes}\ q\ {\isacharplus}\ m\ {\isasymotimes}\ q{\isachardoublequoteclose}\isanewline
\ \ \isacommand{by}\isamarkupfalse%
\ {\isacharparenleft}induct\ n{\isacharparenright}\ simp{\isacharunderscore}all\isanewline
\isanewline
\isacommand{instance}\isamarkupfalse%
\ \isacommand{proof}\isamarkupfalse%
\isanewline
\ \ \isacommand{fix}\isamarkupfalse%
\ m\ n\ q\ {\isacharcolon}{\isacharcolon}\ nat\isanewline
\ \ \isacommand{show}\isamarkupfalse%
\ {\isachardoublequoteopen}m\ {\isasymotimes}\ n\ {\isasymotimes}\ q\ {\isacharequal}\ m\ {\isasymotimes}\ {\isacharparenleft}n\ {\isasymotimes}\ q{\isacharparenright}{\isachardoublequoteclose}\isanewline
\ \ \ \ \isacommand{by}\isamarkupfalse%
\ {\isacharparenleft}induct\ m{\isacharparenright}\ {\isacharparenleft}simp{\isacharunderscore}all\ add{\isacharcolon}\ add{\isacharunderscore}mult{\isacharunderscore}distrib{\isacharparenright}\isanewline
\ \ \isacommand{show}\isamarkupfalse%
\ {\isachardoublequoteopen}{\isasymone}\ {\isasymotimes}\ n\ {\isacharequal}\ n{\isachardoublequoteclose}\isanewline
\ \ \ \ \isacommand{by}\isamarkupfalse%
\ {\isacharparenleft}simp\ add{\isacharcolon}\ neutral{\isacharunderscore}nat{\isacharunderscore}def{\isacharparenright}\isanewline
\ \ \isacommand{show}\isamarkupfalse%
\ {\isachardoublequoteopen}m\ {\isasymotimes}\ {\isasymone}\ {\isacharequal}\ m{\isachardoublequoteclose}\isanewline
\ \ \ \ \isacommand{by}\isamarkupfalse%
\ {\isacharparenleft}induct\ m{\isacharparenright}\ {\isacharparenleft}simp{\isacharunderscore}all\ add{\isacharcolon}\ neutral{\isacharunderscore}nat{\isacharunderscore}def{\isacharparenright}\isanewline
\isacommand{qed}\isamarkupfalse%
\isanewline
\isanewline
\isacommand{end}\isamarkupfalse%
%
\endisatagquote
{\isafoldquote}%
%
\isadelimquote
%
\endisadelimquote
%
\begin{isamarkuptext}%
\noindent We define the natural operation of the natural numbers
  on monoids:%
\end{isamarkuptext}%
\isamarkuptrue%
%
\isadelimquote
%
\endisadelimquote
%
\isatagquote
\isacommand{primrec}\isamarkupfalse%
\ {\isacharparenleft}\isakeyword{in}\ monoid{\isacharparenright}\ pow\ {\isacharcolon}{\isacharcolon}\ {\isachardoublequoteopen}nat\ {\isasymRightarrow}\ {\isacharprime}a\ {\isasymRightarrow}\ {\isacharprime}a{\isachardoublequoteclose}\ \isakeyword{where}\isanewline
\ \ \ \ {\isachardoublequoteopen}pow\ {\isadigit{0}}\ a\ {\isacharequal}\ {\isasymone}{\isachardoublequoteclose}\isanewline
\ \ {\isacharbar}\ {\isachardoublequoteopen}pow\ {\isacharparenleft}Suc\ n{\isacharparenright}\ a\ {\isacharequal}\ a\ {\isasymotimes}\ pow\ n\ a{\isachardoublequoteclose}%
\endisatagquote
{\isafoldquote}%
%
\isadelimquote
%
\endisadelimquote
%
\begin{isamarkuptext}%
\noindent This we use to define the discrete exponentiation function:%
\end{isamarkuptext}%
\isamarkuptrue%
%
\isadelimquote
%
\endisadelimquote
%
\isatagquote
\isacommand{definition}\isamarkupfalse%
\ bexp\ {\isacharcolon}{\isacharcolon}\ {\isachardoublequoteopen}nat\ {\isasymRightarrow}\ nat{\isachardoublequoteclose}\ \isakeyword{where}\isanewline
\ \ {\isachardoublequoteopen}bexp\ n\ {\isacharequal}\ pow\ n\ {\isacharparenleft}Suc\ {\isacharparenleft}Suc\ {\isadigit{0}}{\isacharparenright}{\isacharparenright}{\isachardoublequoteclose}%
\endisatagquote
{\isafoldquote}%
%
\isadelimquote
%
\endisadelimquote
%
\begin{isamarkuptext}%
\noindent The corresponding code:%
\end{isamarkuptext}%
\isamarkuptrue%
%
\isadelimquote
%
\endisadelimquote
%
\isatagquote
%
\begin{isamarkuptext}%
\isatypewriter%
\noindent%
\hspace*{0pt}module Example where {\char123}\\
\hspace*{0pt}\\
\hspace*{0pt}\\
\hspace*{0pt}data Nat = Zero{\char95}nat | Suc Nat;\\
\hspace*{0pt}\\
\hspace*{0pt}class Semigroup a where {\char123}\\
\hspace*{0pt} ~mult ::~a -> a -> a;\\
\hspace*{0pt}{\char125};\\
\hspace*{0pt}\\
\hspace*{0pt}class (Semigroup a) => Monoid a where {\char123}\\
\hspace*{0pt} ~neutral ::~a;\\
\hspace*{0pt}{\char125};\\
\hspace*{0pt}\\
\hspace*{0pt}pow ::~forall a.~(Monoid a) => Nat -> a -> a;\\
\hspace*{0pt}pow Zero{\char95}nat a = neutral;\\
\hspace*{0pt}pow (Suc n) a = mult a (pow n a);\\
\hspace*{0pt}\\
\hspace*{0pt}plus{\char95}nat ::~Nat -> Nat -> Nat;\\
\hspace*{0pt}plus{\char95}nat (Suc m) n = plus{\char95}nat m (Suc n);\\
\hspace*{0pt}plus{\char95}nat Zero{\char95}nat n = n;\\
\hspace*{0pt}\\
\hspace*{0pt}neutral{\char95}nat ::~Nat;\\
\hspace*{0pt}neutral{\char95}nat = Suc Zero{\char95}nat;\\
\hspace*{0pt}\\
\hspace*{0pt}mult{\char95}nat ::~Nat -> Nat -> Nat;\\
\hspace*{0pt}mult{\char95}nat Zero{\char95}nat n = Zero{\char95}nat;\\
\hspace*{0pt}mult{\char95}nat (Suc m) n = plus{\char95}nat n (mult{\char95}nat m n);\\
\hspace*{0pt}\\
\hspace*{0pt}instance Semigroup Nat where {\char123}\\
\hspace*{0pt} ~mult = mult{\char95}nat;\\
\hspace*{0pt}{\char125};\\
\hspace*{0pt}\\
\hspace*{0pt}instance Monoid Nat where {\char123}\\
\hspace*{0pt} ~neutral = neutral{\char95}nat;\\
\hspace*{0pt}{\char125};\\
\hspace*{0pt}\\
\hspace*{0pt}bexp ::~Nat -> Nat;\\
\hspace*{0pt}bexp n = pow n (Suc (Suc Zero{\char95}nat));\\
\hspace*{0pt}\\
\hspace*{0pt}{\char125}%
\end{isamarkuptext}%
\isamarkuptrue%
%
\endisatagquote
{\isafoldquote}%
%
\isadelimquote
%
\endisadelimquote
%
\begin{isamarkuptext}%
\noindent This is a convenient place to show how explicit dictionary construction
  manifests in generated code (here, the same example in \isa{SML}):%
\end{isamarkuptext}%
\isamarkuptrue%
%
\isadelimquote
%
\endisadelimquote
%
\isatagquote
%
\begin{isamarkuptext}%
\isatypewriter%
\noindent%
\hspace*{0pt}structure Example = \\
\hspace*{0pt}struct\\
\hspace*{0pt}\\
\hspace*{0pt}datatype nat = Zero{\char95}nat | Suc of nat;\\
\hspace*{0pt}\\
\hspace*{0pt}type 'a semigroup = {\char123}mult :~'a -> 'a -> 'a{\char125};\\
\hspace*{0pt}fun mult (A{\char95}:'a semigroup) = {\char35}mult A{\char95};\\
\hspace*{0pt}\\
\hspace*{0pt}type 'a monoid = {\char123}Program{\char95}{\char95}semigroup{\char95}monoid :~'a semigroup,~neutral :~'a{\char125};\\
\hspace*{0pt}fun semigroup{\char95}monoid (A{\char95}:'a monoid) = {\char35}Program{\char95}{\char95}semigroup{\char95}monoid A{\char95};\\
\hspace*{0pt}fun neutral (A{\char95}:'a monoid) = {\char35}neutral A{\char95};\\
\hspace*{0pt}\\
\hspace*{0pt}fun pow A{\char95}~Zero{\char95}nat a = neutral A{\char95}\\
\hspace*{0pt} ~| pow A{\char95}~(Suc n) a = mult (semigroup{\char95}monoid A{\char95}) a (pow A{\char95}~n a);\\
\hspace*{0pt}\\
\hspace*{0pt}fun plus{\char95}nat (Suc m) n = plus{\char95}nat m (Suc n)\\
\hspace*{0pt} ~| plus{\char95}nat Zero{\char95}nat n = n;\\
\hspace*{0pt}\\
\hspace*{0pt}val neutral{\char95}nat :~nat = Suc Zero{\char95}nat\\
\hspace*{0pt}\\
\hspace*{0pt}fun mult{\char95}nat Zero{\char95}nat n = Zero{\char95}nat\\
\hspace*{0pt} ~| mult{\char95}nat (Suc m) n = plus{\char95}nat n (mult{\char95}nat m n);\\
\hspace*{0pt}\\
\hspace*{0pt}val semigroup{\char95}nat = {\char123}mult = mult{\char95}nat{\char125}~:~nat semigroup;\\
\hspace*{0pt}\\
\hspace*{0pt}val monoid{\char95}nat =\\
\hspace*{0pt} ~{\char123}Program{\char95}{\char95}semigroup{\char95}monoid = semigroup{\char95}nat,~neutral = neutral{\char95}nat{\char125}~:\\
\hspace*{0pt} ~nat monoid;\\
\hspace*{0pt}\\
\hspace*{0pt}fun bexp n = pow monoid{\char95}nat n (Suc (Suc Zero{\char95}nat));\\
\hspace*{0pt}\\
\hspace*{0pt}end;~(*struct Example*)%
\end{isamarkuptext}%
\isamarkuptrue%
%
\endisatagquote
{\isafoldquote}%
%
\isadelimquote
%
\endisadelimquote
%
\begin{isamarkuptext}%
\noindent Note the parameters with trailing underscore (\verb|A_|)
    which are the dictionary parameters.%
\end{isamarkuptext}%
\isamarkuptrue%
%
\isamarkupsubsection{The preprocessor \label{sec:preproc}%
}
\isamarkuptrue%
%
\begin{isamarkuptext}%
Before selected function theorems are turned into abstract
  code, a chain of definitional transformation steps is carried
  out: \emph{preprocessing}.  In essence, the preprocessor
  consists of two components: a \emph{simpset} and \emph{function transformers}.

  The \emph{simpset} allows to employ the full generality of the Isabelle
  simplifier.  Due to the interpretation of theorems
  as code equations, rewrites are applied to the right
  hand side and the arguments of the left hand side of an
  equation, but never to the constant heading the left hand side.
  An important special case are \emph{inline theorems} which may be
  declared and undeclared using the
  \emph{code inline} or \emph{code inline del} attribute respectively.

  Some common applications:%
\end{isamarkuptext}%
\isamarkuptrue%
%
\begin{itemize}
%
\begin{isamarkuptext}%
\item replacing non-executable constructs by executable ones:%
\end{isamarkuptext}%
\isamarkuptrue%
%
\isadelimquote
%
\endisadelimquote
%
\isatagquote
\isacommand{lemma}\isamarkupfalse%
\ {\isacharbrackleft}code\ inline{\isacharbrackright}{\isacharcolon}\isanewline
\ \ {\isachardoublequoteopen}x\ {\isasymin}\ set\ xs\ {\isasymlongleftrightarrow}\ x\ mem\ xs{\isachardoublequoteclose}\ \isacommand{by}\isamarkupfalse%
\ {\isacharparenleft}induct\ xs{\isacharparenright}\ simp{\isacharunderscore}all%
\endisatagquote
{\isafoldquote}%
%
\isadelimquote
%
\endisadelimquote
%
\begin{isamarkuptext}%
\item eliminating superfluous constants:%
\end{isamarkuptext}%
\isamarkuptrue%
%
\isadelimquote
%
\endisadelimquote
%
\isatagquote
\isacommand{lemma}\isamarkupfalse%
\ {\isacharbrackleft}code\ inline{\isacharbrackright}{\isacharcolon}\isanewline
\ \ {\isachardoublequoteopen}{\isadigit{1}}\ {\isacharequal}\ Suc\ {\isadigit{0}}{\isachardoublequoteclose}\ \isacommand{by}\isamarkupfalse%
\ simp%
\endisatagquote
{\isafoldquote}%
%
\isadelimquote
%
\endisadelimquote
%
\begin{isamarkuptext}%
\item replacing executable but inconvenient constructs:%
\end{isamarkuptext}%
\isamarkuptrue%
%
\isadelimquote
%
\endisadelimquote
%
\isatagquote
\isacommand{lemma}\isamarkupfalse%
\ {\isacharbrackleft}code\ inline{\isacharbrackright}{\isacharcolon}\isanewline
\ \ {\isachardoublequoteopen}xs\ {\isacharequal}\ {\isacharbrackleft}{\isacharbrackright}\ {\isasymlongleftrightarrow}\ List{\isachardot}null\ xs{\isachardoublequoteclose}\ \isacommand{by}\isamarkupfalse%
\ {\isacharparenleft}induct\ xs{\isacharparenright}\ simp{\isacharunderscore}all%
\endisatagquote
{\isafoldquote}%
%
\isadelimquote
%
\endisadelimquote
%
\end{itemize}
%
\begin{isamarkuptext}%
\noindent \emph{Function transformers} provide a very general interface,
  transforming a list of function theorems to another
  list of function theorems, provided that neither the heading
  constant nor its type change.  The \isa{{\isadigit{0}}} / \isa{Suc}
  pattern elimination implemented in
  theory \isa{Efficient{\isacharunderscore}Nat} (see \secref{eff_nat}) uses this
  interface.

  \noindent The current setup of the preprocessor may be inspected using
  the \hyperlink{command.print-codesetup}{\mbox{\isa{\isacommand{print{\isacharunderscore}codesetup}}}} command.
  \hyperlink{command.code-thms}{\mbox{\isa{\isacommand{code{\isacharunderscore}thms}}}} provides a convenient
  mechanism to inspect the impact of a preprocessor setup
  on code equations.

  \begin{warn}
    The attribute \emph{code unfold}
    associated with the \isa{SML\ code\ generator} also applies to
    the \isa{generic\ code\ generator}:
    \emph{code unfold} implies \emph{code inline}.
  \end{warn}%
\end{isamarkuptext}%
\isamarkuptrue%
%
\isamarkupsubsection{Datatypes \label{sec:datatypes}%
}
\isamarkuptrue%
%
\begin{isamarkuptext}%
Conceptually, any datatype is spanned by a set of
  \emph{constructors} of type \isa{{\isasymtau}\ {\isacharequal}\ {\isasymdots}\ {\isasymRightarrow}\ {\isasymkappa}\ {\isasymalpha}\isactrlisub {\isadigit{1}}\ {\isasymdots}\ {\isasymalpha}\isactrlisub n} where \isa{{\isacharbraceleft}{\isasymalpha}\isactrlisub {\isadigit{1}}{\isacharcomma}\ {\isasymdots}{\isacharcomma}\ {\isasymalpha}\isactrlisub n{\isacharbraceright}} is exactly the set of \emph{all} type variables in
  \isa{{\isasymtau}}.  The HOL datatype package by default registers any new
  datatype in the table of datatypes, which may be inspected using the
  \hyperlink{command.print-codesetup}{\mbox{\isa{\isacommand{print{\isacharunderscore}codesetup}}}} command.

  In some cases, it is appropriate to alter or extend this table.  As
  an example, we will develop an alternative representation of the
  queue example given in \secref{sec:intro}.  The amortised
  representation is convenient for generating code but exposes its
  \qt{implementation} details, which may be cumbersome when proving
  theorems about it.  Therefore, here a simple, straightforward
  representation of queues:%
\end{isamarkuptext}%
\isamarkuptrue%
%
\isadelimquote
%
\endisadelimquote
%
\isatagquote
\isacommand{datatype}\isamarkupfalse%
\ {\isacharprime}a\ queue\ {\isacharequal}\ Queue\ {\isachardoublequoteopen}{\isacharprime}a\ list{\isachardoublequoteclose}\isanewline
\isanewline
\isacommand{definition}\isamarkupfalse%
\ empty\ {\isacharcolon}{\isacharcolon}\ {\isachardoublequoteopen}{\isacharprime}a\ queue{\isachardoublequoteclose}\ \isakeyword{where}\isanewline
\ \ {\isachardoublequoteopen}empty\ {\isacharequal}\ Queue\ {\isacharbrackleft}{\isacharbrackright}{\isachardoublequoteclose}\isanewline
\isanewline
\isacommand{primrec}\isamarkupfalse%
\ enqueue\ {\isacharcolon}{\isacharcolon}\ {\isachardoublequoteopen}{\isacharprime}a\ {\isasymRightarrow}\ {\isacharprime}a\ queue\ {\isasymRightarrow}\ {\isacharprime}a\ queue{\isachardoublequoteclose}\ \isakeyword{where}\isanewline
\ \ {\isachardoublequoteopen}enqueue\ x\ {\isacharparenleft}Queue\ xs{\isacharparenright}\ {\isacharequal}\ Queue\ {\isacharparenleft}xs\ {\isacharat}\ {\isacharbrackleft}x{\isacharbrackright}{\isacharparenright}{\isachardoublequoteclose}\isanewline
\isanewline
\isacommand{fun}\isamarkupfalse%
\ dequeue\ {\isacharcolon}{\isacharcolon}\ {\isachardoublequoteopen}{\isacharprime}a\ queue\ {\isasymRightarrow}\ {\isacharprime}a\ option\ {\isasymtimes}\ {\isacharprime}a\ queue{\isachardoublequoteclose}\ \isakeyword{where}\isanewline
\ \ \ \ {\isachardoublequoteopen}dequeue\ {\isacharparenleft}Queue\ {\isacharbrackleft}{\isacharbrackright}{\isacharparenright}\ {\isacharequal}\ {\isacharparenleft}None{\isacharcomma}\ Queue\ {\isacharbrackleft}{\isacharbrackright}{\isacharparenright}{\isachardoublequoteclose}\isanewline
\ \ {\isacharbar}\ {\isachardoublequoteopen}dequeue\ {\isacharparenleft}Queue\ {\isacharparenleft}x\ {\isacharhash}\ xs{\isacharparenright}{\isacharparenright}\ {\isacharequal}\ {\isacharparenleft}Some\ x{\isacharcomma}\ Queue\ xs{\isacharparenright}{\isachardoublequoteclose}%
\endisatagquote
{\isafoldquote}%
%
\isadelimquote
%
\endisadelimquote
%
\begin{isamarkuptext}%
\noindent This we can use directly for proving;  for executing,
  we provide an alternative characterisation:%
\end{isamarkuptext}%
\isamarkuptrue%
%
\isadelimquote
%
\endisadelimquote
%
\isatagquote
\isacommand{definition}\isamarkupfalse%
\ AQueue\ {\isacharcolon}{\isacharcolon}\ {\isachardoublequoteopen}{\isacharprime}a\ list\ {\isasymRightarrow}\ {\isacharprime}a\ list\ {\isasymRightarrow}\ {\isacharprime}a\ queue{\isachardoublequoteclose}\ \isakeyword{where}\isanewline
\ \ {\isachardoublequoteopen}AQueue\ xs\ ys\ {\isacharequal}\ Queue\ {\isacharparenleft}ys\ {\isacharat}\ rev\ xs{\isacharparenright}{\isachardoublequoteclose}\isanewline
\isanewline
\isacommand{code{\isacharunderscore}datatype}\isamarkupfalse%
\ AQueue%
\endisatagquote
{\isafoldquote}%
%
\isadelimquote
%
\endisadelimquote
%
\begin{isamarkuptext}%
\noindent Here we define a \qt{constructor} \isa{AQueue} which
  is defined in terms of \isa{Queue} and interprets its arguments
  according to what the \emph{content} of an amortised queue is supposed
  to be.  Equipped with this, we are able to prove the following equations
  for our primitive queue operations which \qt{implement} the simple
  queues in an amortised fashion:%
\end{isamarkuptext}%
\isamarkuptrue%
%
\isadelimquote
%
\endisadelimquote
%
\isatagquote
\isacommand{lemma}\isamarkupfalse%
\ empty{\isacharunderscore}AQueue\ {\isacharbrackleft}code{\isacharbrackright}{\isacharcolon}\isanewline
\ \ {\isachardoublequoteopen}empty\ {\isacharequal}\ AQueue\ {\isacharbrackleft}{\isacharbrackright}\ {\isacharbrackleft}{\isacharbrackright}{\isachardoublequoteclose}\isanewline
\ \ \isacommand{unfolding}\isamarkupfalse%
\ AQueue{\isacharunderscore}def\ empty{\isacharunderscore}def\ \isacommand{by}\isamarkupfalse%
\ simp\isanewline
\isanewline
\isacommand{lemma}\isamarkupfalse%
\ enqueue{\isacharunderscore}AQueue\ {\isacharbrackleft}code{\isacharbrackright}{\isacharcolon}\isanewline
\ \ {\isachardoublequoteopen}enqueue\ x\ {\isacharparenleft}AQueue\ xs\ ys{\isacharparenright}\ {\isacharequal}\ AQueue\ {\isacharparenleft}x\ {\isacharhash}\ xs{\isacharparenright}\ ys{\isachardoublequoteclose}\isanewline
\ \ \isacommand{unfolding}\isamarkupfalse%
\ AQueue{\isacharunderscore}def\ \isacommand{by}\isamarkupfalse%
\ simp\isanewline
\isanewline
\isacommand{lemma}\isamarkupfalse%
\ dequeue{\isacharunderscore}AQueue\ {\isacharbrackleft}code{\isacharbrackright}{\isacharcolon}\isanewline
\ \ {\isachardoublequoteopen}dequeue\ {\isacharparenleft}AQueue\ xs\ {\isacharbrackleft}{\isacharbrackright}{\isacharparenright}\ {\isacharequal}\isanewline
\ \ \ \ {\isacharparenleft}if\ xs\ {\isacharequal}\ {\isacharbrackleft}{\isacharbrackright}\ then\ {\isacharparenleft}None{\isacharcomma}\ AQueue\ {\isacharbrackleft}{\isacharbrackright}\ {\isacharbrackleft}{\isacharbrackright}{\isacharparenright}\isanewline
\ \ \ \ else\ dequeue\ {\isacharparenleft}AQueue\ {\isacharbrackleft}{\isacharbrackright}\ {\isacharparenleft}rev\ xs{\isacharparenright}{\isacharparenright}{\isacharparenright}{\isachardoublequoteclose}\isanewline
\ \ {\isachardoublequoteopen}dequeue\ {\isacharparenleft}AQueue\ xs\ {\isacharparenleft}y\ {\isacharhash}\ ys{\isacharparenright}{\isacharparenright}\ {\isacharequal}\ {\isacharparenleft}Some\ y{\isacharcomma}\ AQueue\ xs\ ys{\isacharparenright}{\isachardoublequoteclose}\isanewline
\ \ \isacommand{unfolding}\isamarkupfalse%
\ AQueue{\isacharunderscore}def\ \isacommand{by}\isamarkupfalse%
\ simp{\isacharunderscore}all%
\endisatagquote
{\isafoldquote}%
%
\isadelimquote
%
\endisadelimquote
%
\begin{isamarkuptext}%
\noindent For completeness, we provide a substitute for the
  \isa{case} combinator on queues:%
\end{isamarkuptext}%
\isamarkuptrue%
%
\isadelimquote
%
\endisadelimquote
%
\isatagquote
\isacommand{lemma}\isamarkupfalse%
\ queue{\isacharunderscore}case{\isacharunderscore}AQueue\ {\isacharbrackleft}code{\isacharbrackright}{\isacharcolon}\isanewline
\ \ {\isachardoublequoteopen}queue{\isacharunderscore}case\ f\ {\isacharparenleft}AQueue\ xs\ ys{\isacharparenright}\ {\isacharequal}\ f\ {\isacharparenleft}ys\ {\isacharat}\ rev\ xs{\isacharparenright}{\isachardoublequoteclose}\isanewline
\ \ \isacommand{unfolding}\isamarkupfalse%
\ AQueue{\isacharunderscore}def\ \isacommand{by}\isamarkupfalse%
\ simp%
\endisatagquote
{\isafoldquote}%
%
\isadelimquote
%
\endisadelimquote
%
\begin{isamarkuptext}%
\noindent The resulting code looks as expected:%
\end{isamarkuptext}%
\isamarkuptrue%
%
\isadelimquote
%
\endisadelimquote
%
\isatagquote
%
\begin{isamarkuptext}%
\isatypewriter%
\noindent%
\hspace*{0pt}structure Example = \\
\hspace*{0pt}struct\\
\hspace*{0pt}\\
\hspace*{0pt}fun foldl f a [] = a\\
\hspace*{0pt} ~| foldl f a (x ::~xs) = foldl f (f a x) xs;\\
\hspace*{0pt}\\
\hspace*{0pt}fun rev xs = foldl (fn xsa => fn x => x ::~xsa) [] xs;\\
\hspace*{0pt}\\
\hspace*{0pt}fun null [] = true\\
\hspace*{0pt} ~| null (x ::~xs) = false;\\
\hspace*{0pt}\\
\hspace*{0pt}datatype 'a queue = AQueue of 'a list * 'a list;\\
\hspace*{0pt}\\
\hspace*{0pt}val empty :~'a queue = AQueue ([],~[])\\
\hspace*{0pt}\\
\hspace*{0pt}fun dequeue (AQueue (xs,~y ::~ys)) = (SOME y,~AQueue (xs,~ys))\\
\hspace*{0pt} ~| dequeue (AQueue (xs,~[])) =\\
\hspace*{0pt} ~~~(if null xs then (NONE,~AQueue ([],~[]))\\
\hspace*{0pt} ~~~~~else dequeue (AQueue ([],~rev xs)));\\
\hspace*{0pt}\\
\hspace*{0pt}fun enqueue x (AQueue (xs,~ys)) = AQueue (x ::~xs,~ys);\\
\hspace*{0pt}\\
\hspace*{0pt}end;~(*struct Example*)%
\end{isamarkuptext}%
\isamarkuptrue%
%
\endisatagquote
{\isafoldquote}%
%
\isadelimquote
%
\endisadelimquote
%
\begin{isamarkuptext}%
\noindent From this example, it can be glimpsed that using own
  constructor sets is a little delicate since it changes the set of
  valid patterns for values of that type.  Without going into much
  detail, here some practical hints:

  \begin{itemize}

    \item When changing the constructor set for datatypes, take care
      to provide alternative equations for the \isa{case} combinator.

    \item Values in the target language need not to be normalised --
      different values in the target language may represent the same
      value in the logic.

    \item Usually, a good methodology to deal with the subtleties of
      pattern matching is to see the type as an abstract type: provide
      a set of operations which operate on the concrete representation
      of the type, and derive further operations by combinations of
      these primitive ones, without relying on a particular
      representation.

  \end{itemize}%
\end{isamarkuptext}%
\isamarkuptrue%
%
\isamarkupsubsection{Equality%
}
\isamarkuptrue%
%
\begin{isamarkuptext}%
Surely you have already noticed how equality is treated
  by the code generator:%
\end{isamarkuptext}%
\isamarkuptrue%
%
\isadelimquote
%
\endisadelimquote
%
\isatagquote
\isacommand{primrec}\isamarkupfalse%
\ collect{\isacharunderscore}duplicates\ {\isacharcolon}{\isacharcolon}\ {\isachardoublequoteopen}{\isacharprime}a\ list\ {\isasymRightarrow}\ {\isacharprime}a\ list\ {\isasymRightarrow}\ {\isacharprime}a\ list\ {\isasymRightarrow}\ {\isacharprime}a\ list{\isachardoublequoteclose}\ \isakeyword{where}\isanewline
\ \ {\isachardoublequoteopen}collect{\isacharunderscore}duplicates\ xs\ ys\ {\isacharbrackleft}{\isacharbrackright}\ {\isacharequal}\ xs{\isachardoublequoteclose}\isanewline
\ \ {\isacharbar}\ {\isachardoublequoteopen}collect{\isacharunderscore}duplicates\ xs\ ys\ {\isacharparenleft}z{\isacharhash}zs{\isacharparenright}\ {\isacharequal}\ {\isacharparenleft}if\ z\ {\isasymin}\ set\ xs\isanewline
\ \ \ \ \ \ then\ if\ z\ {\isasymin}\ set\ ys\isanewline
\ \ \ \ \ \ \ \ then\ collect{\isacharunderscore}duplicates\ xs\ ys\ zs\isanewline
\ \ \ \ \ \ \ \ else\ collect{\isacharunderscore}duplicates\ xs\ {\isacharparenleft}z{\isacharhash}ys{\isacharparenright}\ zs\isanewline
\ \ \ \ \ \ else\ collect{\isacharunderscore}duplicates\ {\isacharparenleft}z{\isacharhash}xs{\isacharparenright}\ {\isacharparenleft}z{\isacharhash}ys{\isacharparenright}\ zs{\isacharparenright}{\isachardoublequoteclose}%
\endisatagquote
{\isafoldquote}%
%
\isadelimquote
%
\endisadelimquote
%
\begin{isamarkuptext}%
\noindent The membership test during preprocessing is rewritten,
  resulting in \isa{op\ mem}, which itself
  performs an explicit equality check.%
\end{isamarkuptext}%
\isamarkuptrue%
%
\isadelimquote
%
\endisadelimquote
%
\isatagquote
%
\begin{isamarkuptext}%
\isatypewriter%
\noindent%
\hspace*{0pt}structure Example = \\
\hspace*{0pt}struct\\
\hspace*{0pt}\\
\hspace*{0pt}type 'a eq = {\char123}eq :~'a -> 'a -> bool{\char125};\\
\hspace*{0pt}fun eq (A{\char95}:'a eq) = {\char35}eq A{\char95};\\
\hspace*{0pt}\\
\hspace*{0pt}fun eqa A{\char95}~a b = eq A{\char95}~a b;\\
\hspace*{0pt}\\
\hspace*{0pt}fun member A{\char95}~x [] = false\\
\hspace*{0pt} ~| member A{\char95}~x (y ::~ys) = eqa A{\char95}~x y orelse member A{\char95}~x ys;\\
\hspace*{0pt}\\
\hspace*{0pt}fun collect{\char95}duplicates A{\char95}~xs ys [] = xs\\
\hspace*{0pt} ~| collect{\char95}duplicates A{\char95}~xs ys (z ::~zs) =\\
\hspace*{0pt} ~~~(if member A{\char95}~z xs\\
\hspace*{0pt} ~~~~~then (if member A{\char95}~z ys then collect{\char95}duplicates A{\char95}~xs ys zs\\
\hspace*{0pt} ~~~~~~~~~~~~else collect{\char95}duplicates A{\char95}~xs (z ::~ys) zs)\\
\hspace*{0pt} ~~~~~else collect{\char95}duplicates A{\char95}~(z ::~xs) (z ::~ys) zs);\\
\hspace*{0pt}\\
\hspace*{0pt}end;~(*struct Example*)%
\end{isamarkuptext}%
\isamarkuptrue%
%
\endisatagquote
{\isafoldquote}%
%
\isadelimquote
%
\endisadelimquote
%
\begin{isamarkuptext}%
\noindent Obviously, polymorphic equality is implemented the Haskell
  way using a type class.  How is this achieved?  HOL introduces
  an explicit class \isa{eq} with a corresponding operation
  \isa{eq{\isacharunderscore}class{\isachardot}eq} such that \isa{eq{\isacharunderscore}class{\isachardot}eq\ {\isacharequal}\ op\ {\isacharequal}}.
  The preprocessing framework does the rest by propagating the
  \isa{eq} constraints through all dependent code equations.
  For datatypes, instances of \isa{eq} are implicitly derived
  when possible.  For other types, you may instantiate \isa{eq}
  manually like any other type class.

  Though this \isa{eq} class is designed to get rarely in
  the way, in some cases the automatically derived code equations
  for equality on a particular type may not be appropriate.
  As example, watch the following datatype representing
  monomorphic parametric types (where type constructors
  are referred to by natural numbers):%
\end{isamarkuptext}%
\isamarkuptrue%
%
\isadelimquote
%
\endisadelimquote
%
\isatagquote
\isacommand{datatype}\isamarkupfalse%
\ monotype\ {\isacharequal}\ Mono\ nat\ {\isachardoublequoteopen}monotype\ list{\isachardoublequoteclose}%
\endisatagquote
{\isafoldquote}%
%
\isadelimquote
%
\endisadelimquote
%
\isadelimproof
%
\endisadelimproof
%
\isatagproof
%
\endisatagproof
{\isafoldproof}%
%
\isadelimproof
%
\endisadelimproof
%
\begin{isamarkuptext}%
\noindent Then code generation for SML would fail with a message
  that the generated code contains illegal mutual dependencies:
  the theorem \isa{eq{\isacharunderscore}class{\isachardot}eq\ {\isacharparenleft}Mono\ tyco{\isadigit{1}}\ typargs{\isadigit{1}}{\isacharparenright}\ {\isacharparenleft}Mono\ tyco{\isadigit{2}}\ typargs{\isadigit{2}}{\isacharparenright}\ {\isasymequiv}\ eq{\isacharunderscore}class{\isachardot}eq\ tyco{\isadigit{1}}\ tyco{\isadigit{2}}\ {\isasymand}\ eq{\isacharunderscore}class{\isachardot}eq\ typargs{\isadigit{1}}\ typargs{\isadigit{2}}} already requires the
  instance \isa{monotype\ {\isasymColon}\ eq}, which itself requires
  \isa{eq{\isacharunderscore}class{\isachardot}eq\ {\isacharparenleft}Mono\ tyco{\isadigit{1}}\ typargs{\isadigit{1}}{\isacharparenright}\ {\isacharparenleft}Mono\ tyco{\isadigit{2}}\ typargs{\isadigit{2}}{\isacharparenright}\ {\isasymequiv}\ eq{\isacharunderscore}class{\isachardot}eq\ tyco{\isadigit{1}}\ tyco{\isadigit{2}}\ {\isasymand}\ eq{\isacharunderscore}class{\isachardot}eq\ typargs{\isadigit{1}}\ typargs{\isadigit{2}}};  Haskell has no problem with mutually
  recursive \isa{instance} and \isa{function} definitions,
  but the SML serialiser does not support this.

  In such cases, you have to provide your own equality equations
  involving auxiliary constants.  In our case,
  \isa{list{\isacharunderscore}all{\isadigit{2}}} can do the job:%
\end{isamarkuptext}%
\isamarkuptrue%
%
\isadelimquote
%
\endisadelimquote
%
\isatagquote
\isacommand{lemma}\isamarkupfalse%
\ monotype{\isacharunderscore}eq{\isacharunderscore}list{\isacharunderscore}all{\isadigit{2}}\ {\isacharbrackleft}code{\isacharbrackright}{\isacharcolon}\isanewline
\ \ {\isachardoublequoteopen}eq{\isacharunderscore}class{\isachardot}eq\ {\isacharparenleft}Mono\ tyco{\isadigit{1}}\ typargs{\isadigit{1}}{\isacharparenright}\ {\isacharparenleft}Mono\ tyco{\isadigit{2}}\ typargs{\isadigit{2}}{\isacharparenright}\ {\isasymlongleftrightarrow}\isanewline
\ \ \ \ \ eq{\isacharunderscore}class{\isachardot}eq\ tyco{\isadigit{1}}\ tyco{\isadigit{2}}\ {\isasymand}\ list{\isacharunderscore}all{\isadigit{2}}\ eq{\isacharunderscore}class{\isachardot}eq\ typargs{\isadigit{1}}\ typargs{\isadigit{2}}{\isachardoublequoteclose}\isanewline
\ \ \isacommand{by}\isamarkupfalse%
\ {\isacharparenleft}simp\ add{\isacharcolon}\ eq\ list{\isacharunderscore}all{\isadigit{2}}{\isacharunderscore}eq\ {\isacharbrackleft}symmetric{\isacharbrackright}{\isacharparenright}%
\endisatagquote
{\isafoldquote}%
%
\isadelimquote
%
\endisadelimquote
%
\begin{isamarkuptext}%
\noindent does not depend on instance \isa{monotype\ {\isasymColon}\ eq}:%
\end{isamarkuptext}%
\isamarkuptrue%
%
\isadelimquote
%
\endisadelimquote
%
\isatagquote
%
\begin{isamarkuptext}%
\isatypewriter%
\noindent%
\hspace*{0pt}structure Example = \\
\hspace*{0pt}struct\\
\hspace*{0pt}\\
\hspace*{0pt}datatype nat = Zero{\char95}nat | Suc of nat;\\
\hspace*{0pt}\\
\hspace*{0pt}fun null [] = true\\
\hspace*{0pt} ~| null (x ::~xs) = false;\\
\hspace*{0pt}\\
\hspace*{0pt}fun eq{\char95}nat (Suc a) Zero{\char95}nat = false\\
\hspace*{0pt} ~| eq{\char95}nat Zero{\char95}nat (Suc a) = false\\
\hspace*{0pt} ~| eq{\char95}nat (Suc nat) (Suc nat') = eq{\char95}nat nat nat'\\
\hspace*{0pt} ~| eq{\char95}nat Zero{\char95}nat Zero{\char95}nat = true;\\
\hspace*{0pt}\\
\hspace*{0pt}datatype monotype = Mono of nat * monotype list;\\
\hspace*{0pt}\\
\hspace*{0pt}fun list{\char95}all2 p (x ::~xs) (y ::~ys) = p x y andalso list{\char95}all2 p xs ys\\
\hspace*{0pt} ~| list{\char95}all2 p xs [] = null xs\\
\hspace*{0pt} ~| list{\char95}all2 p [] ys = null ys;\\
\hspace*{0pt}\\
\hspace*{0pt}fun eq{\char95}monotype (Mono (tyco1,~typargs1)) (Mono (tyco2,~typargs2)) =\\
\hspace*{0pt} ~eq{\char95}nat tyco1 tyco2 andalso list{\char95}all2 eq{\char95}monotype typargs1 typargs2;\\
\hspace*{0pt}\\
\hspace*{0pt}end;~(*struct Example*)%
\end{isamarkuptext}%
\isamarkuptrue%
%
\endisatagquote
{\isafoldquote}%
%
\isadelimquote
%
\endisadelimquote
%
\isamarkupsubsection{Explicit partiality%
}
\isamarkuptrue%
%
\begin{isamarkuptext}%
Partiality usually enters the game by partial patterns, as
  in the following example, again for amortised queues:%
\end{isamarkuptext}%
\isamarkuptrue%
%
\isadelimquote
%
\endisadelimquote
%
\isatagquote
\isacommand{definition}\isamarkupfalse%
\ strict{\isacharunderscore}dequeue\ {\isacharcolon}{\isacharcolon}\ {\isachardoublequoteopen}{\isacharprime}a\ queue\ {\isasymRightarrow}\ {\isacharprime}a\ {\isasymtimes}\ {\isacharprime}a\ queue{\isachardoublequoteclose}\ \isakeyword{where}\isanewline
\ \ {\isachardoublequoteopen}strict{\isacharunderscore}dequeue\ q\ {\isacharequal}\ {\isacharparenleft}case\ dequeue\ q\isanewline
\ \ \ \ of\ {\isacharparenleft}Some\ x{\isacharcomma}\ q{\isacharprime}{\isacharparenright}\ {\isasymRightarrow}\ {\isacharparenleft}x{\isacharcomma}\ q{\isacharprime}{\isacharparenright}{\isacharparenright}{\isachardoublequoteclose}\isanewline
\isanewline
\isacommand{lemma}\isamarkupfalse%
\ strict{\isacharunderscore}dequeue{\isacharunderscore}AQueue\ {\isacharbrackleft}code{\isacharbrackright}{\isacharcolon}\isanewline
\ \ {\isachardoublequoteopen}strict{\isacharunderscore}dequeue\ {\isacharparenleft}AQueue\ xs\ {\isacharparenleft}y\ {\isacharhash}\ ys{\isacharparenright}{\isacharparenright}\ {\isacharequal}\ {\isacharparenleft}y{\isacharcomma}\ AQueue\ xs\ ys{\isacharparenright}{\isachardoublequoteclose}\isanewline
\ \ {\isachardoublequoteopen}strict{\isacharunderscore}dequeue\ {\isacharparenleft}AQueue\ xs\ {\isacharbrackleft}{\isacharbrackright}{\isacharparenright}\ {\isacharequal}\isanewline
\ \ \ \ {\isacharparenleft}case\ rev\ xs\ of\ y\ {\isacharhash}\ ys\ {\isasymRightarrow}\ {\isacharparenleft}y{\isacharcomma}\ AQueue\ {\isacharbrackleft}{\isacharbrackright}\ ys{\isacharparenright}{\isacharparenright}{\isachardoublequoteclose}\isanewline
\ \ \isacommand{by}\isamarkupfalse%
\ {\isacharparenleft}simp{\isacharunderscore}all\ add{\isacharcolon}\ strict{\isacharunderscore}dequeue{\isacharunderscore}def\ dequeue{\isacharunderscore}AQueue\ split{\isacharcolon}\ list{\isachardot}splits{\isacharparenright}%
\endisatagquote
{\isafoldquote}%
%
\isadelimquote
%
\endisadelimquote
%
\begin{isamarkuptext}%
\noindent In the corresponding code, there is no equation
  for the pattern \isa{AQueue\ {\isacharbrackleft}{\isacharbrackright}\ {\isacharbrackleft}{\isacharbrackright}}:%
\end{isamarkuptext}%
\isamarkuptrue%
%
\isadelimquote
%
\endisadelimquote
%
\isatagquote
%
\begin{isamarkuptext}%
\isatypewriter%
\noindent%
\hspace*{0pt}strict{\char95}dequeue ::~forall a.~Queue a -> (a,~Queue a);\\
\hspace*{0pt}strict{\char95}dequeue (AQueue xs []) =\\
\hspace*{0pt} ~let {\char123}\\
\hspace*{0pt} ~~~(y :~ys) = rev xs;\\
\hspace*{0pt} ~{\char125}~in (y,~AQueue [] ys);\\
\hspace*{0pt}strict{\char95}dequeue (AQueue xs (y :~ys)) = (y,~AQueue xs ys);%
\end{isamarkuptext}%
\isamarkuptrue%
%
\endisatagquote
{\isafoldquote}%
%
\isadelimquote
%
\endisadelimquote
%
\begin{isamarkuptext}%
\noindent In some cases it is desirable to have this
  pseudo-\qt{partiality} more explicitly, e.g.~as follows:%
\end{isamarkuptext}%
\isamarkuptrue%
%
\isadelimquote
%
\endisadelimquote
%
\isatagquote
\isacommand{axiomatization}\isamarkupfalse%
\ empty{\isacharunderscore}queue\ {\isacharcolon}{\isacharcolon}\ {\isacharprime}a\isanewline
\isanewline
\isacommand{definition}\isamarkupfalse%
\ strict{\isacharunderscore}dequeue{\isacharprime}\ {\isacharcolon}{\isacharcolon}\ {\isachardoublequoteopen}{\isacharprime}a\ queue\ {\isasymRightarrow}\ {\isacharprime}a\ {\isasymtimes}\ {\isacharprime}a\ queue{\isachardoublequoteclose}\ \isakeyword{where}\isanewline
\ \ {\isachardoublequoteopen}strict{\isacharunderscore}dequeue{\isacharprime}\ q\ {\isacharequal}\ {\isacharparenleft}case\ dequeue\ q\ of\ {\isacharparenleft}Some\ x{\isacharcomma}\ q{\isacharprime}{\isacharparenright}\ {\isasymRightarrow}\ {\isacharparenleft}x{\isacharcomma}\ q{\isacharprime}{\isacharparenright}\ {\isacharbar}\ {\isacharunderscore}\ {\isasymRightarrow}\ empty{\isacharunderscore}queue{\isacharparenright}{\isachardoublequoteclose}\isanewline
\isanewline
\isacommand{lemma}\isamarkupfalse%
\ strict{\isacharunderscore}dequeue{\isacharprime}{\isacharunderscore}AQueue\ {\isacharbrackleft}code{\isacharbrackright}{\isacharcolon}\isanewline
\ \ {\isachardoublequoteopen}strict{\isacharunderscore}dequeue{\isacharprime}\ {\isacharparenleft}AQueue\ xs\ {\isacharbrackleft}{\isacharbrackright}{\isacharparenright}\ {\isacharequal}\ {\isacharparenleft}if\ xs\ {\isacharequal}\ {\isacharbrackleft}{\isacharbrackright}\ then\ empty{\isacharunderscore}queue\isanewline
\ \ \ \ \ else\ strict{\isacharunderscore}dequeue{\isacharprime}\ {\isacharparenleft}AQueue\ {\isacharbrackleft}{\isacharbrackright}\ {\isacharparenleft}rev\ xs{\isacharparenright}{\isacharparenright}{\isacharparenright}{\isachardoublequoteclose}\isanewline
\ \ {\isachardoublequoteopen}strict{\isacharunderscore}dequeue{\isacharprime}\ {\isacharparenleft}AQueue\ xs\ {\isacharparenleft}y\ {\isacharhash}\ ys{\isacharparenright}{\isacharparenright}\ {\isacharequal}\isanewline
\ \ \ \ \ {\isacharparenleft}y{\isacharcomma}\ AQueue\ xs\ ys{\isacharparenright}{\isachardoublequoteclose}\isanewline
\ \ \isacommand{by}\isamarkupfalse%
\ {\isacharparenleft}simp{\isacharunderscore}all\ add{\isacharcolon}\ strict{\isacharunderscore}dequeue{\isacharprime}{\isacharunderscore}def\ dequeue{\isacharunderscore}AQueue\ split{\isacharcolon}\ list{\isachardot}splits{\isacharparenright}%
\endisatagquote
{\isafoldquote}%
%
\isadelimquote
%
\endisadelimquote
%
\begin{isamarkuptext}%
Observe that on the right hand side of the definition of \isa{strict{\isacharunderscore}dequeue{\isacharprime}} the constant \isa{empty{\isacharunderscore}queue} occurs
  which is unspecified.

  Normally, if constants without any code equations occur in a
  program, the code generator complains (since in most cases this is
  not what the user expects).  But such constants can also be thought
  of as function definitions with no equations which always fail,
  since there is never a successful pattern match on the left hand
  side.  In order to categorise a constant into that category
  explicitly, use \hyperlink{command.code-abort}{\mbox{\isa{\isacommand{code{\isacharunderscore}abort}}}}:%
\end{isamarkuptext}%
\isamarkuptrue%
%
\isadelimquote
%
\endisadelimquote
%
\isatagquote
\isacommand{code{\isacharunderscore}abort}\isamarkupfalse%
\ empty{\isacharunderscore}queue%
\endisatagquote
{\isafoldquote}%
%
\isadelimquote
%
\endisadelimquote
%
\begin{isamarkuptext}%
\noindent Then the code generator will just insert an error or
  exception at the appropriate position:%
\end{isamarkuptext}%
\isamarkuptrue%
%
\isadelimquote
%
\endisadelimquote
%
\isatagquote
%
\begin{isamarkuptext}%
\isatypewriter%
\noindent%
\hspace*{0pt}empty{\char95}queue ::~forall a.~a;\\
\hspace*{0pt}empty{\char95}queue = error {\char34}empty{\char95}queue{\char34};\\
\hspace*{0pt}\\
\hspace*{0pt}strict{\char95}dequeue' ::~forall a.~Queue a -> (a,~Queue a);\\
\hspace*{0pt}strict{\char95}dequeue' (AQueue xs (y :~ys)) = (y,~AQueue xs ys);\\
\hspace*{0pt}strict{\char95}dequeue' (AQueue xs []) =\\
\hspace*{0pt} ~(if nulla xs then empty{\char95}queue\\
\hspace*{0pt} ~~~else strict{\char95}dequeue' (AQueue [] (rev xs)));%
\end{isamarkuptext}%
\isamarkuptrue%
%
\endisatagquote
{\isafoldquote}%
%
\isadelimquote
%
\endisadelimquote
%
\begin{isamarkuptext}%
\noindent This feature however is rarely needed in practice.
  Note also that the \isa{HOL} default setup already declares
  \isa{undefined} as \hyperlink{command.code-abort}{\mbox{\isa{\isacommand{code{\isacharunderscore}abort}}}}, which is most
  likely to be used in such situations.%
\end{isamarkuptext}%
\isamarkuptrue%
%
\isadelimtheory
%
\endisadelimtheory
%
\isatagtheory
\isacommand{end}\isamarkupfalse%
%
\endisatagtheory
{\isafoldtheory}%
%
\isadelimtheory
%
\endisadelimtheory
\isanewline
\ \end{isabellebody}%
%%% Local Variables:
%%% mode: latex
%%% TeX-master: "root"
%%% End:
