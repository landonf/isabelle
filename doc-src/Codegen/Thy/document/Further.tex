%
\begin{isabellebody}%
\def\isabellecontext{Further}%
%
\isadelimtheory
%
\endisadelimtheory
%
\isatagtheory
\isacommand{theory}\isamarkupfalse%
\ Further\isanewline
\isakeyword{imports}\ Setup\isanewline
\isakeyword{begin}%
\endisatagtheory
{\isafoldtheory}%
%
\isadelimtheory
%
\endisadelimtheory
%
\isamarkupsection{Further issues \label{sec:further}%
}
\isamarkuptrue%
%
\isamarkupsubsection{Modules namespace%
}
\isamarkuptrue%
%
\begin{isamarkuptext}%
When invoking the \hyperlink{command.export-code}{\mbox{\isa{\isacommand{export{\isacharunderscore}code}}}} command it is possible to leave
  out the \hyperlink{keyword.module-name}{\mbox{\isa{\isakeyword{module{\isacharunderscore}name}}}} part;  then code is distributed over
  different modules, where the module name space roughly is induced
  by the Isabelle theory name space.

  Then sometimes the awkward situation occurs that dependencies between
  definitions introduce cyclic dependencies between modules, which in the
  \isa{Haskell} world leaves you to the mercy of the \isa{Haskell} implementation
  you are using,  while for \isa{SML}/\isa{OCaml} code generation is not possible.

  A solution is to declare module names explicitly.
  Let use assume the three cyclically dependent
  modules are named \emph{A}, \emph{B} and \emph{C}.
  Then, by stating%
\end{isamarkuptext}%
\isamarkuptrue%
%
\isadelimquote
%
\endisadelimquote
%
\isatagquote
\isacommand{code{\isacharunderscore}modulename}\isamarkupfalse%
\ SML\isanewline
\ \ A\ ABC\isanewline
\ \ B\ ABC\isanewline
\ \ C\ ABC%
\endisatagquote
{\isafoldquote}%
%
\isadelimquote
%
\endisadelimquote
%
\begin{isamarkuptext}%
\noindent
  we explicitly map all those modules on \emph{ABC},
  resulting in an ad-hoc merge of this three modules
  at serialisation time.%
\end{isamarkuptext}%
\isamarkuptrue%
%
\isamarkupsubsection{Locales and interpretation%
}
\isamarkuptrue%
%
\begin{isamarkuptext}%
A technical issue comes to surface when generating code from
  specifications stemming from locale interpretation.

  Let us assume a locale specifying a power operation
  on arbitrary types:%
\end{isamarkuptext}%
\isamarkuptrue%
%
\isadelimquote
%
\endisadelimquote
%
\isatagquote
\isacommand{locale}\isamarkupfalse%
\ power\ {\isacharequal}\isanewline
\ \ \isakeyword{fixes}\ power\ {\isacharcolon}{\isacharcolon}\ {\isachardoublequoteopen}{\isacharprime}a\ {\isasymRightarrow}\ {\isacharprime}b\ {\isasymRightarrow}\ {\isacharprime}b{\isachardoublequoteclose}\isanewline
\ \ \isakeyword{assumes}\ power{\isacharunderscore}commute{\isacharcolon}\ {\isachardoublequoteopen}power\ x\ {\isasymcirc}\ power\ y\ {\isacharequal}\ power\ y\ {\isasymcirc}\ power\ x{\isachardoublequoteclose}\isanewline
\isakeyword{begin}%
\endisatagquote
{\isafoldquote}%
%
\isadelimquote
%
\endisadelimquote
%
\begin{isamarkuptext}%
\noindent Inside that locale we can lift \isa{power} to exponent lists
  by means of specification relative to that locale:%
\end{isamarkuptext}%
\isamarkuptrue%
%
\isadelimquote
%
\endisadelimquote
%
\isatagquote
\isacommand{primrec}\isamarkupfalse%
\ powers\ {\isacharcolon}{\isacharcolon}\ {\isachardoublequoteopen}{\isacharprime}a\ list\ {\isasymRightarrow}\ {\isacharprime}b\ {\isasymRightarrow}\ {\isacharprime}b{\isachardoublequoteclose}\ \isakeyword{where}\isanewline
\ \ {\isachardoublequoteopen}powers\ {\isacharbrackleft}{\isacharbrackright}\ {\isacharequal}\ id{\isachardoublequoteclose}\isanewline
{\isacharbar}\ {\isachardoublequoteopen}powers\ {\isacharparenleft}x\ {\isacharhash}\ xs{\isacharparenright}\ {\isacharequal}\ power\ x\ {\isasymcirc}\ powers\ xs{\isachardoublequoteclose}\isanewline
\isanewline
\isacommand{lemma}\isamarkupfalse%
\ powers{\isacharunderscore}append{\isacharcolon}\isanewline
\ \ {\isachardoublequoteopen}powers\ {\isacharparenleft}xs\ {\isacharat}\ ys{\isacharparenright}\ {\isacharequal}\ powers\ xs\ {\isasymcirc}\ powers\ ys{\isachardoublequoteclose}\isanewline
\ \ \isacommand{by}\isamarkupfalse%
\ {\isacharparenleft}induct\ xs{\isacharparenright}\ simp{\isacharunderscore}all\isanewline
\isanewline
\isacommand{lemma}\isamarkupfalse%
\ powers{\isacharunderscore}power{\isacharcolon}\isanewline
\ \ {\isachardoublequoteopen}powers\ xs\ {\isasymcirc}\ power\ x\ {\isacharequal}\ power\ x\ {\isasymcirc}\ powers\ xs{\isachardoublequoteclose}\isanewline
\ \ \isacommand{by}\isamarkupfalse%
\ {\isacharparenleft}induct\ xs{\isacharparenright}\isanewline
\ \ \ \ {\isacharparenleft}simp{\isacharunderscore}all\ del{\isacharcolon}\ o{\isacharunderscore}apply\ id{\isacharunderscore}apply\ add{\isacharcolon}\ o{\isacharunderscore}assoc\ {\isacharbrackleft}symmetric{\isacharbrackright}{\isacharcomma}\isanewline
\ \ \ \ \ \ simp\ del{\isacharcolon}\ o{\isacharunderscore}apply\ add{\isacharcolon}\ o{\isacharunderscore}assoc\ power{\isacharunderscore}commute{\isacharparenright}\isanewline
\isanewline
\isacommand{lemma}\isamarkupfalse%
\ powers{\isacharunderscore}rev{\isacharcolon}\isanewline
\ \ {\isachardoublequoteopen}powers\ {\isacharparenleft}rev\ xs{\isacharparenright}\ {\isacharequal}\ powers\ xs{\isachardoublequoteclose}\isanewline
\ \ \ \ \isacommand{by}\isamarkupfalse%
\ {\isacharparenleft}induct\ xs{\isacharparenright}\ {\isacharparenleft}simp{\isacharunderscore}all\ add{\isacharcolon}\ powers{\isacharunderscore}append\ powers{\isacharunderscore}power{\isacharparenright}\isanewline
\isanewline
\isacommand{end}\isamarkupfalse%
%
\endisatagquote
{\isafoldquote}%
%
\isadelimquote
%
\endisadelimquote
%
\begin{isamarkuptext}%
After an interpretation of this locale (say, \indexdef{}{command}{interpretation}\hypertarget{command.interpretation}{\hyperlink{command.interpretation}{\mbox{\isa{\isacommand{interpretation}}}}} \isa{fun{\isacharunderscore}power{\isacharcolon}} \isa{{\isachardoublequote}power\ {\isacharparenleft}{\isasymlambda}n\ {\isacharparenleft}f\ {\isacharcolon}{\isacharcolon}\ {\isacharprime}a\ {\isasymRightarrow}\ {\isacharprime}a{\isacharparenright}{\isachardot}\ f\ {\isacharcircum}{\isacharcircum}\ n{\isacharparenright}{\isachardoublequote}}), one would expect to have a constant \isa{fun{\isacharunderscore}power{\isachardot}powers\ {\isacharcolon}{\isacharcolon}\ nat\ list\ {\isasymRightarrow}\ {\isacharparenleft}{\isacharprime}a\ {\isasymRightarrow}\ {\isacharprime}a{\isacharparenright}\ {\isasymRightarrow}\ {\isacharprime}a\ {\isasymRightarrow}\ {\isacharprime}a} for which code
  can be generated.  But this not the case: internally, the term
  \isa{fun{\isacharunderscore}power{\isachardot}powers} is an abbreviation for the foundational
  term \isa{{\isachardoublequote}power{\isachardot}powers\ {\isacharparenleft}{\isasymlambda}n\ {\isacharparenleft}f\ {\isacharcolon}{\isacharcolon}\ {\isacharprime}a\ {\isasymRightarrow}\ {\isacharprime}a{\isacharparenright}{\isachardot}\ f\ {\isacharcircum}{\isacharcircum}\ n{\isacharparenright}{\isachardoublequote}}
  (see \cite{isabelle-locale} for the details behind).

  Fortunately, with minor effort the desired behaviour can be achieved.
  First, a dedicated definition of the constant on which the local \isa{powers}
  after interpretation is supposed to be mapped on:%
\end{isamarkuptext}%
\isamarkuptrue%
%
\isadelimquote
%
\endisadelimquote
%
\isatagquote
\isacommand{definition}\isamarkupfalse%
\ funpows\ {\isacharcolon}{\isacharcolon}\ {\isachardoublequoteopen}nat\ list\ {\isasymRightarrow}\ {\isacharparenleft}{\isacharprime}a\ {\isasymRightarrow}\ {\isacharprime}a{\isacharparenright}\ {\isasymRightarrow}\ {\isacharprime}a\ {\isasymRightarrow}\ {\isacharprime}a{\isachardoublequoteclose}\ \isakeyword{where}\isanewline
\ \ {\isacharbrackleft}code\ del{\isacharbrackright}{\isacharcolon}\ {\isachardoublequoteopen}funpows\ {\isacharequal}\ power{\isachardot}powers\ {\isacharparenleft}{\isasymlambda}n\ f{\isachardot}\ f\ {\isacharcircum}{\isacharcircum}\ n{\isacharparenright}{\isachardoublequoteclose}%
\endisatagquote
{\isafoldquote}%
%
\isadelimquote
%
\endisadelimquote
%
\begin{isamarkuptext}%
\noindent In general, the pattern is \isa{c\ {\isacharequal}\ t} where \isa{c} is
  the name of the future constant and \isa{t} the foundational term
  corresponding to the local constant after interpretation.

  The interpretation itself is enriched with an equation \isa{t\ {\isacharequal}\ c}:%
\end{isamarkuptext}%
\isamarkuptrue%
%
\isadelimquote
%
\endisadelimquote
%
\isatagquote
\isacommand{interpretation}\isamarkupfalse%
\ fun{\isacharunderscore}power{\isacharcolon}\ power\ {\isachardoublequoteopen}{\isasymlambda}n\ {\isacharparenleft}f\ {\isacharcolon}{\isacharcolon}\ {\isacharprime}a\ {\isasymRightarrow}\ {\isacharprime}a{\isacharparenright}{\isachardot}\ f\ {\isacharcircum}{\isacharcircum}\ n{\isachardoublequoteclose}\ \isakeyword{where}\isanewline
\ \ {\isachardoublequoteopen}power{\isachardot}powers\ {\isacharparenleft}{\isasymlambda}n\ f{\isachardot}\ f\ {\isacharcircum}{\isacharcircum}\ n{\isacharparenright}\ {\isacharequal}\ funpows{\isachardoublequoteclose}\isanewline
\ \ \isacommand{by}\isamarkupfalse%
\ unfold{\isacharunderscore}locales\isanewline
\ \ \ \ {\isacharparenleft}simp{\isacharunderscore}all\ add{\isacharcolon}\ fun{\isacharunderscore}eq{\isacharunderscore}iff\ funpow{\isacharunderscore}mult\ mult{\isacharunderscore}commute\ funpows{\isacharunderscore}def{\isacharparenright}%
\endisatagquote
{\isafoldquote}%
%
\isadelimquote
%
\endisadelimquote
%
\begin{isamarkuptext}%
\noindent This additional equation is trivially proved by the definition
  itself.

  After this setup procedure, code generation can continue as usual:%
\end{isamarkuptext}%
\isamarkuptrue%
%
\isadelimquote
%
\endisadelimquote
%
\isatagquote
%
\begin{isamarkuptext}%
\begin{typewriter}
    funpow\ {\isacharcolon}{\isacharcolon}\ forall\ a{\isachardot}\ Nat\ {\isacharminus}{\isachargreater}\ {\isacharparenleft}a\ {\isacharminus}{\isachargreater}\ a{\isacharparenright}\ {\isacharminus}{\isachargreater}\ a\ {\isacharminus}{\isachargreater}\ a{\isacharsemicolon}\isanewline
funpow\ Zero{\isacharunderscore}nat\ f\ {\isacharequal}\ id{\isacharsemicolon}\isanewline
funpow\ {\isacharparenleft}Suc\ n{\isacharparenright}\ f\ {\isacharequal}\ f\ {\isachardot}\ funpow\ n\ f{\isacharsemicolon}\isanewline
\isanewline
funpows\ {\isacharcolon}{\isacharcolon}\ forall\ a{\isachardot}\ {\isacharbrackleft}Nat{\isacharbrackright}\ {\isacharminus}{\isachargreater}\ {\isacharparenleft}a\ {\isacharminus}{\isachargreater}\ a{\isacharparenright}\ {\isacharminus}{\isachargreater}\ a\ {\isacharminus}{\isachargreater}\ a{\isacharsemicolon}\isanewline
funpows\ {\isacharbrackleft}{\isacharbrackright}\ {\isacharequal}\ id{\isacharsemicolon}\isanewline
funpows\ {\isacharparenleft}x\ {\isacharcolon}\ xs{\isacharparenright}\ {\isacharequal}\ funpow\ x\ {\isachardot}\ funpows\ xs{\isacharsemicolon}
  \end{typewriter}%
\end{isamarkuptext}%
\isamarkuptrue%
%
\endisatagquote
{\isafoldquote}%
%
\isadelimquote
%
\endisadelimquote
%
\isamarkupsubsection{Imperative data structures%
}
\isamarkuptrue%
%
\begin{isamarkuptext}%
If you consider imperative data structures as inevitable for a
  specific application, you should consider \emph{Imperative
  Functional Programming with Isabelle/HOL}
  \cite{bulwahn-et-al:2008:imperative}; the framework described there
  is available in session \isa{Imperative{\isacharunderscore}HOL}.%
\end{isamarkuptext}%
\isamarkuptrue%
%
\isamarkupsubsection{ML system interfaces \label{sec:ml}%
}
\isamarkuptrue%
%
\begin{isamarkuptext}%
Since the code generator framework not only aims to provide a nice
  Isar interface but also to form a base for code-generation-based
  applications, here a short description of the most fundamental ML
  interfaces.%
\end{isamarkuptext}%
\isamarkuptrue%
%
\isamarkupsubsubsection{Managing executable content%
}
\isamarkuptrue%
%
\isadelimmlref
%
\endisadelimmlref
%
\isatagmlref
%
\begin{isamarkuptext}%
\begin{mldecls}
  \indexdef{}{ML}{Code.read\_const}\verb|Code.read_const: theory -> string -> string| \\
  \indexdef{}{ML}{Code.add\_eqn}\verb|Code.add_eqn: thm -> theory -> theory| \\
  \indexdef{}{ML}{Code.del\_eqn}\verb|Code.del_eqn: thm -> theory -> theory| \\
  \indexdef{}{ML}{Code\_Preproc.map\_pre}\verb|Code_Preproc.map_pre: (simpset -> simpset) -> theory -> theory| \\
  \indexdef{}{ML}{Code\_Preproc.map\_post}\verb|Code_Preproc.map_post: (simpset -> simpset) -> theory -> theory| \\
  \indexdef{}{ML}{Code\_Preproc.add\_functrans}\verb|Code_Preproc.add_functrans: |\isasep\isanewline%
\verb|    string * (theory -> (thm * bool) list -> (thm * bool) list option)|\isasep\isanewline%
\verb|      -> theory -> theory| \\
  \indexdef{}{ML}{Code\_Preproc.del\_functrans}\verb|Code_Preproc.del_functrans: string -> theory -> theory| \\
  \indexdef{}{ML}{Code.add\_datatype}\verb|Code.add_datatype: (string * typ) list -> theory -> theory| \\
  \indexdef{}{ML}{Code.get\_type}\verb|Code.get_type: theory -> string|\isasep\isanewline%
\verb|    -> (string * sort) list * ((string * string list) * typ list) list| \\
  \indexdef{}{ML}{Code.get\_type\_of\_constr\_or\_abstr}\verb|Code.get_type_of_constr_or_abstr: theory -> string -> (string * bool) option|
  \end{mldecls}

  \begin{description}

  \item \verb|Code.read_const|~\isa{thy}~\isa{s}
     reads a constant as a concrete term expression \isa{s}.

  \item \verb|Code.add_eqn|~\isa{thm}~\isa{thy} adds function
     theorem \isa{thm} to executable content.

  \item \verb|Code.del_eqn|~\isa{thm}~\isa{thy} removes function
     theorem \isa{thm} from executable content, if present.

  \item \verb|Code_Preproc.map_pre|~\isa{f}~\isa{thy} changes
     the preprocessor simpset.

  \item \verb|Code_Preproc.add_functrans|~\isa{{\isacharparenleft}name{\isacharcomma}\ f{\isacharparenright}}~\isa{thy} adds
     function transformer \isa{f} (named \isa{name}) to executable content;
     \isa{f} is a transformer of the code equations belonging
     to a certain function definition, depending on the
     current theory context.  Returning \isa{NONE} indicates that no
     transformation took place;  otherwise, the whole process will be iterated
     with the new code equations.

  \item \verb|Code_Preproc.del_functrans|~\isa{name}~\isa{thy} removes
     function transformer named \isa{name} from executable content.

  \item \verb|Code.add_datatype|~\isa{cs}~\isa{thy} adds
     a datatype to executable content, with generation
     set \isa{cs}.

  \item \verb|Code.get_type_of_constr_or_abstr|~\isa{thy}~\isa{const}
     returns type constructor corresponding to
     constructor \isa{const}; returns \isa{NONE}
     if \isa{const} is no constructor.

  \end{description}%
\end{isamarkuptext}%
\isamarkuptrue%
%
\endisatagmlref
{\isafoldmlref}%
%
\isadelimmlref
%
\endisadelimmlref
%
\isamarkupsubsubsection{Data depending on the theory's executable content%
}
\isamarkuptrue%
%
\begin{isamarkuptext}%
Implementing code generator applications on top
  of the framework set out so far usually not only
  involves using those primitive interfaces
  but also storing code-dependent data and various
  other things.

  Due to incrementality of code generation, changes in the
  theory's executable content have to be propagated in a
  certain fashion.  Additionally, such changes may occur
  not only during theory extension but also during theory
  merge, which is a little bit nasty from an implementation
  point of view.  The framework provides a solution
  to this technical challenge by providing a functorial
  data slot \verb|Code_Data|; on instantiation
  of this functor, the following types and operations
  are required:

  \medskip
  \begin{tabular}{l}
  \isa{type\ T} \\
  \isa{val\ empty{\isacharcolon}\ T} \\
  \end{tabular}

  \begin{description}

  \item \isa{T} the type of data to store.

  \item \isa{empty} initial (empty) data.

  \end{description}

  \noindent An instance of \verb|Code_Data| provides the following
  interface:

  \medskip
  \begin{tabular}{l}
  \isa{change{\isacharcolon}\ theory\ {\isasymrightarrow}\ {\isacharparenleft}T\ {\isasymrightarrow}\ T{\isacharparenright}\ {\isasymrightarrow}\ T} \\
  \isa{change{\isacharunderscore}yield{\isacharcolon}\ theory\ {\isasymrightarrow}\ {\isacharparenleft}T\ {\isasymrightarrow}\ {\isacharprime}a\ {\isacharasterisk}\ T{\isacharparenright}\ {\isasymrightarrow}\ {\isacharprime}a\ {\isacharasterisk}\ T}
  \end{tabular}

  \begin{description}

  \item \isa{change} update of current data (cached!)
    by giving a continuation.

  \item \isa{change{\isacharunderscore}yield} update with side result.

  \end{description}%
\end{isamarkuptext}%
\isamarkuptrue%
%
\isadelimtheory
%
\endisadelimtheory
%
\isatagtheory
\isacommand{end}\isamarkupfalse%
%
\endisatagtheory
{\isafoldtheory}%
%
\isadelimtheory
%
\endisadelimtheory
\isanewline
\end{isabellebody}%
%%% Local Variables:
%%% mode: latex
%%% TeX-master: "root"
%%% End:
