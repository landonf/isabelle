%
\begin{isabellebody}%
\def\isabellecontext{Further}%
%
\isadelimtheory
%
\endisadelimtheory
%
\isatagtheory
\isacommand{theory}\isamarkupfalse%
\ Further\isanewline
\isakeyword{imports}\ Setup\isanewline
\isakeyword{begin}%
\endisatagtheory
{\isafoldtheory}%
%
\isadelimtheory
%
\endisadelimtheory
%
\isamarkupsection{Further issues \label{sec:further}%
}
\isamarkuptrue%
%
\isamarkupsubsection{Specialities of the \isa{Scala} target language \label{sec:scala}%
}
\isamarkuptrue%
%
\begin{isamarkuptext}%
\isa{Scala} deviates from languages of the ML family in a couple
  of aspects; those which affect code generation mainly have to do with
  \isa{Scala}'s type system:

  \begin{itemize}

    \item \isa{Scala} prefers tupled syntax over curried syntax.

    \item \isa{Scala} sacrifices Hindely-Milner type inference for a
      much more rich type system with subtyping etc.  For this reason
      type arguments sometimes have to be given explicitly in square
      brackets (mimicing System F syntax).

    \item In contrast to \isa{Haskell} where most specialities of
      the type system are implemented using \emph{type classes},
      \isa{Scala} provides a sophisticated system of \emph{implicit
      arguments}.

  \end{itemize}

  \noindent Concerning currying, the \isa{Scala} serializer counts
  arguments in code equations to determine how many arguments
  shall be tupled; remaining arguments and abstractions in terms
  rather than function definitions are always curried.

  The second aspect affects user-defined adaptations with \hyperlink{command.code-const}{\mbox{\isa{\isacommand{code{\isaliteral{5F}{\isacharunderscore}}const}}}}.  For regular terms, the \isa{Scala} serializer prints
  all type arguments explicitly.  For user-defined term adaptations
  this is only possible for adaptations which take no arguments: here
  the type arguments are just appended.  Otherwise they are ignored;
  hence user-defined adaptations for polymorphic constants have to be
  designed very carefully to avoid ambiguity.

  Isabelle's type classes are mapped onto \isa{Scala} implicits; in
  cases with diamonds in the subclass hierarchy this can lead to
  ambiguities in the generated code:%
\end{isamarkuptext}%
\isamarkuptrue%
%
\isadelimquote
%
\endisadelimquote
%
\isatagquote
\isacommand{class}\isamarkupfalse%
\ class{\isadigit{1}}\ {\isaliteral{3D}{\isacharequal}}\isanewline
\ \ \isakeyword{fixes}\ foo\ {\isaliteral{3A}{\isacharcolon}}{\isaliteral{3A}{\isacharcolon}}\ {\isaliteral{22}{\isachardoublequoteopen}}{\isaliteral{27}{\isacharprime}}a\ {\isaliteral{5C3C52696768746172726F773E}{\isasymRightarrow}}\ {\isaliteral{27}{\isacharprime}}a{\isaliteral{22}{\isachardoublequoteclose}}\isanewline
\isanewline
\isacommand{class}\isamarkupfalse%
\ class{\isadigit{2}}\ {\isaliteral{3D}{\isacharequal}}\ class{\isadigit{1}}\isanewline
\isanewline
\isacommand{class}\isamarkupfalse%
\ class{\isadigit{3}}\ {\isaliteral{3D}{\isacharequal}}\ class{\isadigit{1}}%
\endisatagquote
{\isafoldquote}%
%
\isadelimquote
%
\endisadelimquote
%
\begin{isamarkuptext}%
\noindent Here both \isa{class{\isadigit{2}}} and \isa{class{\isadigit{3}}} inherit from \isa{class{\isadigit{1}}},
  forming the upper part of a diamond.%
\end{isamarkuptext}%
\isamarkuptrue%
%
\isadelimquote
%
\endisadelimquote
%
\isatagquote
\isacommand{definition}\isamarkupfalse%
\ bar\ {\isaliteral{3A}{\isacharcolon}}{\isaliteral{3A}{\isacharcolon}}\ {\isaliteral{22}{\isachardoublequoteopen}}{\isaliteral{27}{\isacharprime}}a\ {\isaliteral{3A}{\isacharcolon}}{\isaliteral{3A}{\isacharcolon}}\ {\isaliteral{7B}{\isacharbraceleft}}class{\isadigit{2}}{\isaliteral{2C}{\isacharcomma}}\ class{\isadigit{3}}{\isaliteral{7D}{\isacharbraceright}}\ {\isaliteral{5C3C52696768746172726F773E}{\isasymRightarrow}}\ {\isaliteral{27}{\isacharprime}}a{\isaliteral{22}{\isachardoublequoteclose}}\ \isakeyword{where}\isanewline
\ \ {\isaliteral{22}{\isachardoublequoteopen}}bar\ {\isaliteral{3D}{\isacharequal}}\ foo{\isaliteral{22}{\isachardoublequoteclose}}%
\endisatagquote
{\isafoldquote}%
%
\isadelimquote
%
\endisadelimquote
%
\begin{isamarkuptext}%
\noindent This yields the following code:%
\end{isamarkuptext}%
\isamarkuptrue%
%
\isadelimquotetypewriter
%
\endisadelimquotetypewriter
%
\isatagquotetypewriter
%
\begin{isamarkuptext}%
object\ Example\ {\isaliteral{7B}{\isacharbraceleft}}\isanewline
\isanewline
trait\ class{\isadigit{1}}{\isaliteral{5B}{\isacharbrackleft}}A{\isaliteral{5D}{\isacharbrackright}}\ {\isaliteral{7B}{\isacharbraceleft}}\isanewline
\ \ val\ {\isaliteral{60}{\isacharbackquote}}Example{\isaliteral{2E}{\isachardot}}foo{\isaliteral{60}{\isacharbackquote}}{\isaliteral{3A}{\isacharcolon}}\ A\ {\isaliteral{3D}{\isacharequal}}{\isaliteral{3E}{\isachargreater}}\ A\isanewline
{\isaliteral{7D}{\isacharbraceright}}\isanewline
def\ foo{\isaliteral{5B}{\isacharbrackleft}}A{\isaliteral{5D}{\isacharbrackright}}{\isaliteral{28}{\isacharparenleft}}a{\isaliteral{3A}{\isacharcolon}}\ A{\isaliteral{29}{\isacharparenright}}{\isaliteral{28}{\isacharparenleft}}implicit\ A{\isaliteral{3A}{\isacharcolon}}\ class{\isadigit{1}}{\isaliteral{5B}{\isacharbrackleft}}A{\isaliteral{5D}{\isacharbrackright}}{\isaliteral{29}{\isacharparenright}}{\isaliteral{3A}{\isacharcolon}}\ A\ {\isaliteral{3D}{\isacharequal}}\ A{\isaliteral{2E}{\isachardot}}{\isaliteral{60}{\isacharbackquote}}Example{\isaliteral{2E}{\isachardot}}foo{\isaliteral{60}{\isacharbackquote}}{\isaliteral{28}{\isacharparenleft}}a{\isaliteral{29}{\isacharparenright}}\isanewline
\isanewline
trait\ class{\isadigit{2}}{\isaliteral{5B}{\isacharbrackleft}}A{\isaliteral{5D}{\isacharbrackright}}\ extends\ class{\isadigit{1}}{\isaliteral{5B}{\isacharbrackleft}}A{\isaliteral{5D}{\isacharbrackright}}\ {\isaliteral{7B}{\isacharbraceleft}}\isanewline
{\isaliteral{7D}{\isacharbraceright}}\isanewline
\isanewline
trait\ class{\isadigit{3}}{\isaliteral{5B}{\isacharbrackleft}}A{\isaliteral{5D}{\isacharbrackright}}\ extends\ class{\isadigit{1}}{\isaliteral{5B}{\isacharbrackleft}}A{\isaliteral{5D}{\isacharbrackright}}\ {\isaliteral{7B}{\isacharbraceleft}}\isanewline
{\isaliteral{7D}{\isacharbraceright}}\isanewline
\isanewline
def\ bar{\isaliteral{5B}{\isacharbrackleft}}A\ {\isaliteral{3A}{\isacharcolon}}\ class{\isadigit{2}}\ {\isaliteral{3A}{\isacharcolon}}\ class{\isadigit{3}}{\isaliteral{5D}{\isacharbrackright}}{\isaliteral{3A}{\isacharcolon}}\ A\ {\isaliteral{3D}{\isacharequal}}{\isaliteral{3E}{\isachargreater}}\ A\ {\isaliteral{3D}{\isacharequal}}\ {\isaliteral{28}{\isacharparenleft}}a{\isaliteral{3A}{\isacharcolon}}\ A{\isaliteral{29}{\isacharparenright}}\ {\isaliteral{3D}{\isacharequal}}{\isaliteral{3E}{\isachargreater}}\ foo{\isaliteral{5B}{\isacharbrackleft}}A{\isaliteral{5D}{\isacharbrackright}}{\isaliteral{28}{\isacharparenleft}}a{\isaliteral{29}{\isacharparenright}}\isanewline
\isanewline
{\isaliteral{7D}{\isacharbraceright}}\ {\isaliteral{2F}{\isacharslash}}{\isaliteral{2A}{\isacharasterisk}}\ object\ Example\ {\isaliteral{2A}{\isacharasterisk}}{\isaliteral{2F}{\isacharslash}}\isanewline%
\end{isamarkuptext}%
\isamarkuptrue%
%
\endisatagquotetypewriter
{\isafoldquotetypewriter}%
%
\isadelimquotetypewriter
%
\endisadelimquotetypewriter
%
\begin{isamarkuptext}%
\noindent This code is rejected by the \isa{Scala} compiler: in
  the definition of \isa{bar}, it is not clear from where to derive
  the implicit argument for \isa{foo}.

  The solution to the problem is to close the diamond by a further
  class with inherits from both \isa{class{\isadigit{2}}} and \isa{class{\isadigit{3}}}:%
\end{isamarkuptext}%
\isamarkuptrue%
%
\isadelimquote
%
\endisadelimquote
%
\isatagquote
\isacommand{class}\isamarkupfalse%
\ class{\isadigit{4}}\ {\isaliteral{3D}{\isacharequal}}\ class{\isadigit{2}}\ {\isaliteral{2B}{\isacharplus}}\ class{\isadigit{3}}%
\endisatagquote
{\isafoldquote}%
%
\isadelimquote
%
\endisadelimquote
%
\begin{isamarkuptext}%
\noindent Then the offending code equation can be restricted to
  \isa{class{\isadigit{4}}}:%
\end{isamarkuptext}%
\isamarkuptrue%
%
\isadelimquote
%
\endisadelimquote
%
\isatagquote
\isacommand{lemma}\isamarkupfalse%
\ {\isaliteral{5B}{\isacharbrackleft}}code{\isaliteral{5D}{\isacharbrackright}}{\isaliteral{3A}{\isacharcolon}}\isanewline
\ \ {\isaliteral{22}{\isachardoublequoteopen}}{\isaliteral{28}{\isacharparenleft}}bar\ {\isaliteral{3A}{\isacharcolon}}{\isaliteral{3A}{\isacharcolon}}\ {\isaliteral{27}{\isacharprime}}a{\isaliteral{3A}{\isacharcolon}}{\isaliteral{3A}{\isacharcolon}}class{\isadigit{4}}\ {\isaliteral{5C3C52696768746172726F773E}{\isasymRightarrow}}\ {\isaliteral{27}{\isacharprime}}a{\isaliteral{29}{\isacharparenright}}\ {\isaliteral{3D}{\isacharequal}}\ foo{\isaliteral{22}{\isachardoublequoteclose}}\isanewline
\ \ \isacommand{by}\isamarkupfalse%
\ {\isaliteral{28}{\isacharparenleft}}simp\ only{\isaliteral{3A}{\isacharcolon}}\ bar{\isaliteral{5F}{\isacharunderscore}}def{\isaliteral{29}{\isacharparenright}}%
\endisatagquote
{\isafoldquote}%
%
\isadelimquote
%
\endisadelimquote
%
\begin{isamarkuptext}%
\noindent with the following code:%
\end{isamarkuptext}%
\isamarkuptrue%
%
\isadelimquotetypewriter
%
\endisadelimquotetypewriter
%
\isatagquotetypewriter
%
\begin{isamarkuptext}%
object\ Example\ {\isaliteral{7B}{\isacharbraceleft}}\isanewline
\isanewline
trait\ class{\isadigit{1}}{\isaliteral{5B}{\isacharbrackleft}}A{\isaliteral{5D}{\isacharbrackright}}\ {\isaliteral{7B}{\isacharbraceleft}}\isanewline
\ \ val\ {\isaliteral{60}{\isacharbackquote}}Example{\isaliteral{2E}{\isachardot}}foo{\isaliteral{60}{\isacharbackquote}}{\isaliteral{3A}{\isacharcolon}}\ A\ {\isaliteral{3D}{\isacharequal}}{\isaliteral{3E}{\isachargreater}}\ A\isanewline
{\isaliteral{7D}{\isacharbraceright}}\isanewline
def\ foo{\isaliteral{5B}{\isacharbrackleft}}A{\isaliteral{5D}{\isacharbrackright}}{\isaliteral{28}{\isacharparenleft}}a{\isaliteral{3A}{\isacharcolon}}\ A{\isaliteral{29}{\isacharparenright}}{\isaliteral{28}{\isacharparenleft}}implicit\ A{\isaliteral{3A}{\isacharcolon}}\ class{\isadigit{1}}{\isaliteral{5B}{\isacharbrackleft}}A{\isaliteral{5D}{\isacharbrackright}}{\isaliteral{29}{\isacharparenright}}{\isaliteral{3A}{\isacharcolon}}\ A\ {\isaliteral{3D}{\isacharequal}}\ A{\isaliteral{2E}{\isachardot}}{\isaliteral{60}{\isacharbackquote}}Example{\isaliteral{2E}{\isachardot}}foo{\isaliteral{60}{\isacharbackquote}}{\isaliteral{28}{\isacharparenleft}}a{\isaliteral{29}{\isacharparenright}}\isanewline
\isanewline
trait\ class{\isadigit{2}}{\isaliteral{5B}{\isacharbrackleft}}A{\isaliteral{5D}{\isacharbrackright}}\ extends\ class{\isadigit{1}}{\isaliteral{5B}{\isacharbrackleft}}A{\isaliteral{5D}{\isacharbrackright}}\ {\isaliteral{7B}{\isacharbraceleft}}\isanewline
{\isaliteral{7D}{\isacharbraceright}}\isanewline
\isanewline
trait\ class{\isadigit{3}}{\isaliteral{5B}{\isacharbrackleft}}A{\isaliteral{5D}{\isacharbrackright}}\ extends\ class{\isadigit{1}}{\isaliteral{5B}{\isacharbrackleft}}A{\isaliteral{5D}{\isacharbrackright}}\ {\isaliteral{7B}{\isacharbraceleft}}\isanewline
{\isaliteral{7D}{\isacharbraceright}}\isanewline
\isanewline
trait\ class{\isadigit{4}}{\isaliteral{5B}{\isacharbrackleft}}A{\isaliteral{5D}{\isacharbrackright}}\ extends\ class{\isadigit{2}}{\isaliteral{5B}{\isacharbrackleft}}A{\isaliteral{5D}{\isacharbrackright}}\ with\ class{\isadigit{3}}{\isaliteral{5B}{\isacharbrackleft}}A{\isaliteral{5D}{\isacharbrackright}}\ {\isaliteral{7B}{\isacharbraceleft}}\isanewline
{\isaliteral{7D}{\isacharbraceright}}\isanewline
\isanewline
def\ bar{\isaliteral{5B}{\isacharbrackleft}}A\ {\isaliteral{3A}{\isacharcolon}}\ class{\isadigit{4}}{\isaliteral{5D}{\isacharbrackright}}{\isaliteral{3A}{\isacharcolon}}\ A\ {\isaliteral{3D}{\isacharequal}}{\isaliteral{3E}{\isachargreater}}\ A\ {\isaliteral{3D}{\isacharequal}}\ {\isaliteral{28}{\isacharparenleft}}a{\isaliteral{3A}{\isacharcolon}}\ A{\isaliteral{29}{\isacharparenright}}\ {\isaliteral{3D}{\isacharequal}}{\isaliteral{3E}{\isachargreater}}\ foo{\isaliteral{5B}{\isacharbrackleft}}A{\isaliteral{5D}{\isacharbrackright}}{\isaliteral{28}{\isacharparenleft}}a{\isaliteral{29}{\isacharparenright}}\isanewline
\isanewline
{\isaliteral{7D}{\isacharbraceright}}\ {\isaliteral{2F}{\isacharslash}}{\isaliteral{2A}{\isacharasterisk}}\ object\ Example\ {\isaliteral{2A}{\isacharasterisk}}{\isaliteral{2F}{\isacharslash}}\isanewline%
\end{isamarkuptext}%
\isamarkuptrue%
%
\endisatagquotetypewriter
{\isafoldquotetypewriter}%
%
\isadelimquotetypewriter
%
\endisadelimquotetypewriter
%
\begin{isamarkuptext}%
\noindent which exposes no ambiguity.

  Since the preprocessor (cf.~\secref{sec:preproc}) propagates sort
  constraints through a system of code equations, it is usually not
  very difficult to identify the set of code equations which actually
  needs more restricted sort constraints.%
\end{isamarkuptext}%
\isamarkuptrue%
%
\isamarkupsubsection{Modules namespace%
}
\isamarkuptrue%
%
\begin{isamarkuptext}%
When invoking the \hyperlink{command.export-code}{\mbox{\isa{\isacommand{export{\isaliteral{5F}{\isacharunderscore}}code}}}} command it is possible to
  leave out the \hyperlink{keyword.module-name}{\mbox{\isa{\isakeyword{module{\isaliteral{5F}{\isacharunderscore}}name}}}} part; then code is
  distributed over different modules, where the module name space
  roughly is induced by the Isabelle theory name space.

  Then sometimes the awkward situation occurs that dependencies
  between definitions introduce cyclic dependencies between modules,
  which in the \isa{Haskell} world leaves you to the mercy of the
  \isa{Haskell} implementation you are using, while for \isa{SML}/\isa{OCaml} code generation is not possible.

  A solution is to declare module names explicitly.  Let use assume
  the three cyclically dependent modules are named \emph{A}, \emph{B}
  and \emph{C}.  Then, by stating%
\end{isamarkuptext}%
\isamarkuptrue%
%
\isadelimquote
%
\endisadelimquote
%
\isatagquote
\isacommand{code{\isaliteral{5F}{\isacharunderscore}}modulename}\isamarkupfalse%
\ SML\isanewline
\ \ A\ ABC\isanewline
\ \ B\ ABC\isanewline
\ \ C\ ABC%
\endisatagquote
{\isafoldquote}%
%
\isadelimquote
%
\endisadelimquote
%
\begin{isamarkuptext}%
\noindent we explicitly map all those modules on \emph{ABC},
  resulting in an ad-hoc merge of this three modules at serialisation
  time.%
\end{isamarkuptext}%
\isamarkuptrue%
%
\isamarkupsubsection{Locales and interpretation%
}
\isamarkuptrue%
%
\begin{isamarkuptext}%
A technical issue comes to surface when generating code from
  specifications stemming from locale interpretation.

  Let us assume a locale specifying a power operation on arbitrary
  types:%
\end{isamarkuptext}%
\isamarkuptrue%
%
\isadelimquote
%
\endisadelimquote
%
\isatagquote
\isacommand{locale}\isamarkupfalse%
\ power\ {\isaliteral{3D}{\isacharequal}}\isanewline
\ \ \isakeyword{fixes}\ power\ {\isaliteral{3A}{\isacharcolon}}{\isaliteral{3A}{\isacharcolon}}\ {\isaliteral{22}{\isachardoublequoteopen}}{\isaliteral{27}{\isacharprime}}a\ {\isaliteral{5C3C52696768746172726F773E}{\isasymRightarrow}}\ {\isaliteral{27}{\isacharprime}}b\ {\isaliteral{5C3C52696768746172726F773E}{\isasymRightarrow}}\ {\isaliteral{27}{\isacharprime}}b{\isaliteral{22}{\isachardoublequoteclose}}\isanewline
\ \ \isakeyword{assumes}\ power{\isaliteral{5F}{\isacharunderscore}}commute{\isaliteral{3A}{\isacharcolon}}\ {\isaliteral{22}{\isachardoublequoteopen}}power\ x\ {\isaliteral{5C3C636972633E}{\isasymcirc}}\ power\ y\ {\isaliteral{3D}{\isacharequal}}\ power\ y\ {\isaliteral{5C3C636972633E}{\isasymcirc}}\ power\ x{\isaliteral{22}{\isachardoublequoteclose}}\isanewline
\isakeyword{begin}%
\endisatagquote
{\isafoldquote}%
%
\isadelimquote
%
\endisadelimquote
%
\begin{isamarkuptext}%
\noindent Inside that locale we can lift \isa{power} to exponent
  lists by means of specification relative to that locale:%
\end{isamarkuptext}%
\isamarkuptrue%
%
\isadelimquote
%
\endisadelimquote
%
\isatagquote
\isacommand{primrec}\isamarkupfalse%
\ powers\ {\isaliteral{3A}{\isacharcolon}}{\isaliteral{3A}{\isacharcolon}}\ {\isaliteral{22}{\isachardoublequoteopen}}{\isaliteral{27}{\isacharprime}}a\ list\ {\isaliteral{5C3C52696768746172726F773E}{\isasymRightarrow}}\ {\isaliteral{27}{\isacharprime}}b\ {\isaliteral{5C3C52696768746172726F773E}{\isasymRightarrow}}\ {\isaliteral{27}{\isacharprime}}b{\isaliteral{22}{\isachardoublequoteclose}}\ \isakeyword{where}\isanewline
\ \ {\isaliteral{22}{\isachardoublequoteopen}}powers\ {\isaliteral{5B}{\isacharbrackleft}}{\isaliteral{5D}{\isacharbrackright}}\ {\isaliteral{3D}{\isacharequal}}\ id{\isaliteral{22}{\isachardoublequoteclose}}\isanewline
{\isaliteral{7C}{\isacharbar}}\ {\isaliteral{22}{\isachardoublequoteopen}}powers\ {\isaliteral{28}{\isacharparenleft}}x\ {\isaliteral{23}{\isacharhash}}\ xs{\isaliteral{29}{\isacharparenright}}\ {\isaliteral{3D}{\isacharequal}}\ power\ x\ {\isaliteral{5C3C636972633E}{\isasymcirc}}\ powers\ xs{\isaliteral{22}{\isachardoublequoteclose}}\isanewline
\isanewline
\isacommand{lemma}\isamarkupfalse%
\ powers{\isaliteral{5F}{\isacharunderscore}}append{\isaliteral{3A}{\isacharcolon}}\isanewline
\ \ {\isaliteral{22}{\isachardoublequoteopen}}powers\ {\isaliteral{28}{\isacharparenleft}}xs\ {\isaliteral{40}{\isacharat}}\ ys{\isaliteral{29}{\isacharparenright}}\ {\isaliteral{3D}{\isacharequal}}\ powers\ xs\ {\isaliteral{5C3C636972633E}{\isasymcirc}}\ powers\ ys{\isaliteral{22}{\isachardoublequoteclose}}\isanewline
\ \ \isacommand{by}\isamarkupfalse%
\ {\isaliteral{28}{\isacharparenleft}}induct\ xs{\isaliteral{29}{\isacharparenright}}\ simp{\isaliteral{5F}{\isacharunderscore}}all\isanewline
\isanewline
\isacommand{lemma}\isamarkupfalse%
\ powers{\isaliteral{5F}{\isacharunderscore}}power{\isaliteral{3A}{\isacharcolon}}\isanewline
\ \ {\isaliteral{22}{\isachardoublequoteopen}}powers\ xs\ {\isaliteral{5C3C636972633E}{\isasymcirc}}\ power\ x\ {\isaliteral{3D}{\isacharequal}}\ power\ x\ {\isaliteral{5C3C636972633E}{\isasymcirc}}\ powers\ xs{\isaliteral{22}{\isachardoublequoteclose}}\isanewline
\ \ \isacommand{by}\isamarkupfalse%
\ {\isaliteral{28}{\isacharparenleft}}induct\ xs{\isaliteral{29}{\isacharparenright}}\isanewline
\ \ \ \ {\isaliteral{28}{\isacharparenleft}}simp{\isaliteral{5F}{\isacharunderscore}}all\ del{\isaliteral{3A}{\isacharcolon}}\ o{\isaliteral{5F}{\isacharunderscore}}apply\ id{\isaliteral{5F}{\isacharunderscore}}apply\ add{\isaliteral{3A}{\isacharcolon}}\ o{\isaliteral{5F}{\isacharunderscore}}assoc\ {\isaliteral{5B}{\isacharbrackleft}}symmetric{\isaliteral{5D}{\isacharbrackright}}{\isaliteral{2C}{\isacharcomma}}\isanewline
\ \ \ \ \ \ simp\ del{\isaliteral{3A}{\isacharcolon}}\ o{\isaliteral{5F}{\isacharunderscore}}apply\ add{\isaliteral{3A}{\isacharcolon}}\ o{\isaliteral{5F}{\isacharunderscore}}assoc\ power{\isaliteral{5F}{\isacharunderscore}}commute{\isaliteral{29}{\isacharparenright}}\isanewline
\isanewline
\isacommand{lemma}\isamarkupfalse%
\ powers{\isaliteral{5F}{\isacharunderscore}}rev{\isaliteral{3A}{\isacharcolon}}\isanewline
\ \ {\isaliteral{22}{\isachardoublequoteopen}}powers\ {\isaliteral{28}{\isacharparenleft}}rev\ xs{\isaliteral{29}{\isacharparenright}}\ {\isaliteral{3D}{\isacharequal}}\ powers\ xs{\isaliteral{22}{\isachardoublequoteclose}}\isanewline
\ \ \ \ \isacommand{by}\isamarkupfalse%
\ {\isaliteral{28}{\isacharparenleft}}induct\ xs{\isaliteral{29}{\isacharparenright}}\ {\isaliteral{28}{\isacharparenleft}}simp{\isaliteral{5F}{\isacharunderscore}}all\ add{\isaliteral{3A}{\isacharcolon}}\ powers{\isaliteral{5F}{\isacharunderscore}}append\ powers{\isaliteral{5F}{\isacharunderscore}}power{\isaliteral{29}{\isacharparenright}}\isanewline
\isanewline
\isacommand{end}\isamarkupfalse%
%
\endisatagquote
{\isafoldquote}%
%
\isadelimquote
%
\endisadelimquote
%
\begin{isamarkuptext}%
After an interpretation of this locale (say, \indexdef{}{command}{interpretation}\hypertarget{command.interpretation}{\hyperlink{command.interpretation}{\mbox{\isa{\isacommand{interpretation}}}}} \isa{fun{\isaliteral{5F}{\isacharunderscore}}power{\isaliteral{3A}{\isacharcolon}}} \isa{{\isaliteral{22}{\isachardoublequote}}power\ {\isaliteral{28}{\isacharparenleft}}{\isaliteral{5C3C6C616D6264613E}{\isasymlambda}}n\ {\isaliteral{28}{\isacharparenleft}}f\ {\isaliteral{3A}{\isacharcolon}}{\isaliteral{3A}{\isacharcolon}}\ {\isaliteral{27}{\isacharprime}}a\ {\isaliteral{5C3C52696768746172726F773E}{\isasymRightarrow}}\ {\isaliteral{27}{\isacharprime}}a{\isaliteral{29}{\isacharparenright}}{\isaliteral{2E}{\isachardot}}\ f\ {\isaliteral{5E}{\isacharcircum}}{\isaliteral{5E}{\isacharcircum}}\ n{\isaliteral{29}{\isacharparenright}}{\isaliteral{22}{\isachardoublequote}}}), one would expect to have a constant \isa{fun{\isaliteral{5F}{\isacharunderscore}}power{\isaliteral{2E}{\isachardot}}powers\ {\isaliteral{3A}{\isacharcolon}}{\isaliteral{3A}{\isacharcolon}}\ nat\ list\ {\isaliteral{5C3C52696768746172726F773E}{\isasymRightarrow}}\ {\isaliteral{28}{\isacharparenleft}}{\isaliteral{27}{\isacharprime}}a\ {\isaliteral{5C3C52696768746172726F773E}{\isasymRightarrow}}\ {\isaliteral{27}{\isacharprime}}a{\isaliteral{29}{\isacharparenright}}\ {\isaliteral{5C3C52696768746172726F773E}{\isasymRightarrow}}\ {\isaliteral{27}{\isacharprime}}a\ {\isaliteral{5C3C52696768746172726F773E}{\isasymRightarrow}}\ {\isaliteral{27}{\isacharprime}}a} for which code
  can be generated.  But this not the case: internally, the term
  \isa{fun{\isaliteral{5F}{\isacharunderscore}}power{\isaliteral{2E}{\isachardot}}powers} is an abbreviation for the foundational
  term \isa{{\isaliteral{22}{\isachardoublequote}}power{\isaliteral{2E}{\isachardot}}powers\ {\isaliteral{28}{\isacharparenleft}}{\isaliteral{5C3C6C616D6264613E}{\isasymlambda}}n\ {\isaliteral{28}{\isacharparenleft}}f\ {\isaliteral{3A}{\isacharcolon}}{\isaliteral{3A}{\isacharcolon}}\ {\isaliteral{27}{\isacharprime}}a\ {\isaliteral{5C3C52696768746172726F773E}{\isasymRightarrow}}\ {\isaliteral{27}{\isacharprime}}a{\isaliteral{29}{\isacharparenright}}{\isaliteral{2E}{\isachardot}}\ f\ {\isaliteral{5E}{\isacharcircum}}{\isaliteral{5E}{\isacharcircum}}\ n{\isaliteral{29}{\isacharparenright}}{\isaliteral{22}{\isachardoublequote}}}
  (see \cite{isabelle-locale} for the details behind).

  Fortunately, with minor effort the desired behaviour can be
  achieved.  First, a dedicated definition of the constant on which
  the local \isa{powers} after interpretation is supposed to be
  mapped on:%
\end{isamarkuptext}%
\isamarkuptrue%
%
\isadelimquote
%
\endisadelimquote
%
\isatagquote
\isacommand{definition}\isamarkupfalse%
\ funpows\ {\isaliteral{3A}{\isacharcolon}}{\isaliteral{3A}{\isacharcolon}}\ {\isaliteral{22}{\isachardoublequoteopen}}nat\ list\ {\isaliteral{5C3C52696768746172726F773E}{\isasymRightarrow}}\ {\isaliteral{28}{\isacharparenleft}}{\isaliteral{27}{\isacharprime}}a\ {\isaliteral{5C3C52696768746172726F773E}{\isasymRightarrow}}\ {\isaliteral{27}{\isacharprime}}a{\isaliteral{29}{\isacharparenright}}\ {\isaliteral{5C3C52696768746172726F773E}{\isasymRightarrow}}\ {\isaliteral{27}{\isacharprime}}a\ {\isaliteral{5C3C52696768746172726F773E}{\isasymRightarrow}}\ {\isaliteral{27}{\isacharprime}}a{\isaliteral{22}{\isachardoublequoteclose}}\ \isakeyword{where}\isanewline
\ \ {\isaliteral{5B}{\isacharbrackleft}}code\ del{\isaliteral{5D}{\isacharbrackright}}{\isaliteral{3A}{\isacharcolon}}\ {\isaliteral{22}{\isachardoublequoteopen}}funpows\ {\isaliteral{3D}{\isacharequal}}\ power{\isaliteral{2E}{\isachardot}}powers\ {\isaliteral{28}{\isacharparenleft}}{\isaliteral{5C3C6C616D6264613E}{\isasymlambda}}n\ f{\isaliteral{2E}{\isachardot}}\ f\ {\isaliteral{5E}{\isacharcircum}}{\isaliteral{5E}{\isacharcircum}}\ n{\isaliteral{29}{\isacharparenright}}{\isaliteral{22}{\isachardoublequoteclose}}%
\endisatagquote
{\isafoldquote}%
%
\isadelimquote
%
\endisadelimquote
%
\begin{isamarkuptext}%
\noindent In general, the pattern is \isa{c\ {\isaliteral{3D}{\isacharequal}}\ t} where \isa{c}
  is the name of the future constant and \isa{t} the foundational
  term corresponding to the local constant after interpretation.

  The interpretation itself is enriched with an equation \isa{t\ {\isaliteral{3D}{\isacharequal}}\ c}:%
\end{isamarkuptext}%
\isamarkuptrue%
%
\isadelimquote
%
\endisadelimquote
%
\isatagquote
\isacommand{interpretation}\isamarkupfalse%
\ fun{\isaliteral{5F}{\isacharunderscore}}power{\isaliteral{3A}{\isacharcolon}}\ power\ {\isaliteral{22}{\isachardoublequoteopen}}{\isaliteral{5C3C6C616D6264613E}{\isasymlambda}}n\ {\isaliteral{28}{\isacharparenleft}}f\ {\isaliteral{3A}{\isacharcolon}}{\isaliteral{3A}{\isacharcolon}}\ {\isaliteral{27}{\isacharprime}}a\ {\isaliteral{5C3C52696768746172726F773E}{\isasymRightarrow}}\ {\isaliteral{27}{\isacharprime}}a{\isaliteral{29}{\isacharparenright}}{\isaliteral{2E}{\isachardot}}\ f\ {\isaliteral{5E}{\isacharcircum}}{\isaliteral{5E}{\isacharcircum}}\ n{\isaliteral{22}{\isachardoublequoteclose}}\ \isakeyword{where}\isanewline
\ \ {\isaliteral{22}{\isachardoublequoteopen}}power{\isaliteral{2E}{\isachardot}}powers\ {\isaliteral{28}{\isacharparenleft}}{\isaliteral{5C3C6C616D6264613E}{\isasymlambda}}n\ f{\isaliteral{2E}{\isachardot}}\ f\ {\isaliteral{5E}{\isacharcircum}}{\isaliteral{5E}{\isacharcircum}}\ n{\isaliteral{29}{\isacharparenright}}\ {\isaliteral{3D}{\isacharequal}}\ funpows{\isaliteral{22}{\isachardoublequoteclose}}\isanewline
\ \ \isacommand{by}\isamarkupfalse%
\ unfold{\isaliteral{5F}{\isacharunderscore}}locales\isanewline
\ \ \ \ {\isaliteral{28}{\isacharparenleft}}simp{\isaliteral{5F}{\isacharunderscore}}all\ add{\isaliteral{3A}{\isacharcolon}}\ fun{\isaliteral{5F}{\isacharunderscore}}eq{\isaliteral{5F}{\isacharunderscore}}iff\ funpow{\isaliteral{5F}{\isacharunderscore}}mult\ mult{\isaliteral{5F}{\isacharunderscore}}commute\ funpows{\isaliteral{5F}{\isacharunderscore}}def{\isaliteral{29}{\isacharparenright}}%
\endisatagquote
{\isafoldquote}%
%
\isadelimquote
%
\endisadelimquote
%
\begin{isamarkuptext}%
\noindent This additional equation is trivially proved by the
  definition itself.

  After this setup procedure, code generation can continue as usual:%
\end{isamarkuptext}%
\isamarkuptrue%
%
\isadelimquotetypewriter
%
\endisadelimquotetypewriter
%
\isatagquotetypewriter
%
\begin{isamarkuptext}%
funpow\ {\isaliteral{3A}{\isacharcolon}}{\isaliteral{3A}{\isacharcolon}}\ forall\ a{\isaliteral{2E}{\isachardot}}\ Nat\ {\isaliteral{2D}{\isacharminus}}{\isaliteral{3E}{\isachargreater}}\ {\isaliteral{28}{\isacharparenleft}}a\ {\isaliteral{2D}{\isacharminus}}{\isaliteral{3E}{\isachargreater}}\ a{\isaliteral{29}{\isacharparenright}}\ {\isaliteral{2D}{\isacharminus}}{\isaliteral{3E}{\isachargreater}}\ a\ {\isaliteral{2D}{\isacharminus}}{\isaliteral{3E}{\isachargreater}}\ a{\isaliteral{3B}{\isacharsemicolon}}\isanewline
funpow\ Zero{\isaliteral{5F}{\isacharunderscore}}nat\ f\ {\isaliteral{3D}{\isacharequal}}\ id{\isaliteral{3B}{\isacharsemicolon}}\isanewline
funpow\ {\isaliteral{28}{\isacharparenleft}}Suc\ n{\isaliteral{29}{\isacharparenright}}\ f\ {\isaliteral{3D}{\isacharequal}}\ f\ {\isaliteral{2E}{\isachardot}}\ funpow\ n\ f{\isaliteral{3B}{\isacharsemicolon}}\isanewline
\isanewline
funpows\ {\isaliteral{3A}{\isacharcolon}}{\isaliteral{3A}{\isacharcolon}}\ forall\ a{\isaliteral{2E}{\isachardot}}\ {\isaliteral{5B}{\isacharbrackleft}}Nat{\isaliteral{5D}{\isacharbrackright}}\ {\isaliteral{2D}{\isacharminus}}{\isaliteral{3E}{\isachargreater}}\ {\isaliteral{28}{\isacharparenleft}}a\ {\isaliteral{2D}{\isacharminus}}{\isaliteral{3E}{\isachargreater}}\ a{\isaliteral{29}{\isacharparenright}}\ {\isaliteral{2D}{\isacharminus}}{\isaliteral{3E}{\isachargreater}}\ a\ {\isaliteral{2D}{\isacharminus}}{\isaliteral{3E}{\isachargreater}}\ a{\isaliteral{3B}{\isacharsemicolon}}\isanewline
funpows\ {\isaliteral{5B}{\isacharbrackleft}}{\isaliteral{5D}{\isacharbrackright}}\ {\isaliteral{3D}{\isacharequal}}\ id{\isaliteral{3B}{\isacharsemicolon}}\isanewline
funpows\ {\isaliteral{28}{\isacharparenleft}}x\ {\isaliteral{3A}{\isacharcolon}}\ xs{\isaliteral{29}{\isacharparenright}}\ {\isaliteral{3D}{\isacharequal}}\ funpow\ x\ {\isaliteral{2E}{\isachardot}}\ funpows\ xs{\isaliteral{3B}{\isacharsemicolon}}\isanewline%
\end{isamarkuptext}%
\isamarkuptrue%
%
\endisatagquotetypewriter
{\isafoldquotetypewriter}%
%
\isadelimquotetypewriter
%
\endisadelimquotetypewriter
%
\isamarkupsubsection{Imperative data structures%
}
\isamarkuptrue%
%
\begin{isamarkuptext}%
If you consider imperative data structures as inevitable for a
  specific application, you should consider \emph{Imperative
  Functional Programming with Isabelle/HOL}
  \cite{bulwahn-et-al:2008:imperative}; the framework described there
  is available in session \isa{Imperative{\isaliteral{5F}{\isacharunderscore}}HOL}, together with a
  short primer document.%
\end{isamarkuptext}%
\isamarkuptrue%
%
\isamarkupsubsection{ML system interfaces \label{sec:ml}%
}
\isamarkuptrue%
%
\begin{isamarkuptext}%
Since the code generator framework not only aims to provide a nice
  Isar interface but also to form a base for code-generation-based
  applications, here a short description of the most fundamental ML
  interfaces.%
\end{isamarkuptext}%
\isamarkuptrue%
%
\isamarkupsubsubsection{Managing executable content%
}
\isamarkuptrue%
%
\isadelimmlref
%
\endisadelimmlref
%
\isatagmlref
%
\begin{isamarkuptext}%
\begin{mldecls}
  \indexdef{}{ML}{Code.read\_const}\verb|Code.read_const: theory -> string -> string| \\
  \indexdef{}{ML}{Code.add\_eqn}\verb|Code.add_eqn: thm -> theory -> theory| \\
  \indexdef{}{ML}{Code.del\_eqn}\verb|Code.del_eqn: thm -> theory -> theory| \\
  \indexdef{}{ML}{Code\_Preproc.map\_pre}\verb|Code_Preproc.map_pre: (simpset -> simpset) -> theory -> theory| \\
  \indexdef{}{ML}{Code\_Preproc.map\_post}\verb|Code_Preproc.map_post: (simpset -> simpset) -> theory -> theory| \\
  \indexdef{}{ML}{Code\_Preproc.add\_functrans}\verb|Code_Preproc.add_functrans: |\isasep\isanewline%
\verb|    string * (theory -> (thm * bool) list -> (thm * bool) list option)|\isasep\isanewline%
\verb|      -> theory -> theory| \\
  \indexdef{}{ML}{Code\_Preproc.del\_functrans}\verb|Code_Preproc.del_functrans: string -> theory -> theory| \\
  \indexdef{}{ML}{Code.add\_datatype}\verb|Code.add_datatype: (string * typ) list -> theory -> theory| \\
  \indexdef{}{ML}{Code.get\_type}\verb|Code.get_type: theory -> string|\isasep\isanewline%
\verb|    -> ((string * sort) list * (string * ((string * sort) list * typ list)) list) * bool| \\
  \indexdef{}{ML}{Code.get\_type\_of\_constr\_or\_abstr}\verb|Code.get_type_of_constr_or_abstr: theory -> string -> (string * bool) option|
  \end{mldecls}

  \begin{description}

  \item \verb|Code.read_const|~\isa{thy}~\isa{s}
     reads a constant as a concrete term expression \isa{s}.

  \item \verb|Code.add_eqn|~\isa{thm}~\isa{thy} adds function
     theorem \isa{thm} to executable content.

  \item \verb|Code.del_eqn|~\isa{thm}~\isa{thy} removes function
     theorem \isa{thm} from executable content, if present.

  \item \verb|Code_Preproc.map_pre|~\isa{f}~\isa{thy} changes
     the preprocessor simpset.

  \item \verb|Code_Preproc.add_functrans|~\isa{{\isaliteral{28}{\isacharparenleft}}name{\isaliteral{2C}{\isacharcomma}}\ f{\isaliteral{29}{\isacharparenright}}}~\isa{thy} adds
     function transformer \isa{f} (named \isa{name}) to executable content;
     \isa{f} is a transformer of the code equations belonging
     to a certain function definition, depending on the
     current theory context.  Returning \isa{NONE} indicates that no
     transformation took place;  otherwise, the whole process will be iterated
     with the new code equations.

  \item \verb|Code_Preproc.del_functrans|~\isa{name}~\isa{thy} removes
     function transformer named \isa{name} from executable content.

  \item \verb|Code.add_datatype|~\isa{cs}~\isa{thy} adds
     a datatype to executable content, with generation
     set \isa{cs}.

  \item \verb|Code.get_type_of_constr_or_abstr|~\isa{thy}~\isa{const}
     returns type constructor corresponding to
     constructor \isa{const}; returns \isa{NONE}
     if \isa{const} is no constructor.

  \end{description}%
\end{isamarkuptext}%
\isamarkuptrue%
%
\endisatagmlref
{\isafoldmlref}%
%
\isadelimmlref
%
\endisadelimmlref
%
\isamarkupsubsubsection{Data depending on the theory's executable content%
}
\isamarkuptrue%
%
\begin{isamarkuptext}%
Implementing code generator applications on top of the framework set
  out so far usually not only involves using those primitive
  interfaces but also storing code-dependent data and various other
  things.

  Due to incrementality of code generation, changes in the theory's
  executable content have to be propagated in a certain fashion.
  Additionally, such changes may occur not only during theory
  extension but also during theory merge, which is a little bit nasty
  from an implementation point of view.  The framework provides a
  solution to this technical challenge by providing a functorial data
  slot \verb|Code_Data|; on instantiation of this functor, the
  following types and operations are required:

  \medskip
  \begin{tabular}{l}
  \isa{type\ T} \\
  \isa{val\ empty{\isaliteral{3A}{\isacharcolon}}\ T} \\
  \end{tabular}

  \begin{description}

  \item \isa{T} the type of data to store.

  \item \isa{empty} initial (empty) data.

  \end{description}

  \noindent An instance of \verb|Code_Data| provides the
  following interface:

  \medskip
  \begin{tabular}{l}
  \isa{change{\isaliteral{3A}{\isacharcolon}}\ theory\ {\isaliteral{5C3C72696768746172726F773E}{\isasymrightarrow}}\ {\isaliteral{28}{\isacharparenleft}}T\ {\isaliteral{5C3C72696768746172726F773E}{\isasymrightarrow}}\ T{\isaliteral{29}{\isacharparenright}}\ {\isaliteral{5C3C72696768746172726F773E}{\isasymrightarrow}}\ T} \\
  \isa{change{\isaliteral{5F}{\isacharunderscore}}yield{\isaliteral{3A}{\isacharcolon}}\ theory\ {\isaliteral{5C3C72696768746172726F773E}{\isasymrightarrow}}\ {\isaliteral{28}{\isacharparenleft}}T\ {\isaliteral{5C3C72696768746172726F773E}{\isasymrightarrow}}\ {\isaliteral{27}{\isacharprime}}a\ {\isaliteral{2A}{\isacharasterisk}}\ T{\isaliteral{29}{\isacharparenright}}\ {\isaliteral{5C3C72696768746172726F773E}{\isasymrightarrow}}\ {\isaliteral{27}{\isacharprime}}a\ {\isaliteral{2A}{\isacharasterisk}}\ T}
  \end{tabular}

  \begin{description}

  \item \isa{change} update of current data (cached!) by giving a
    continuation.

  \item \isa{change{\isaliteral{5F}{\isacharunderscore}}yield} update with side result.

  \end{description}%
\end{isamarkuptext}%
\isamarkuptrue%
%
\isadelimtheory
%
\endisadelimtheory
%
\isatagtheory
\isacommand{end}\isamarkupfalse%
%
\endisatagtheory
{\isafoldtheory}%
%
\isadelimtheory
%
\endisadelimtheory
\isanewline
\end{isabellebody}%
%%% Local Variables:
%%% mode: latex
%%% TeX-master: "root"
%%% End:
