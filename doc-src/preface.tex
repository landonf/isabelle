\chapter*{Preface}
\markboth{Preface}{Preface}   %or Preface ?
\addcontentsline{toc}{chapter}{Preface} 

\index{Isabelle!object-logics supported}

Most theorem provers support a fixed logic, such as first-order or
equational logic.  They bring sophisticated proof procedures to bear upon
the conjectured formula.  An impressive example is the resolution prover
Otter, which Quaife~\cite{quaife-book} has used to formalize a body of
mathematics.

ALF~\cite{alf}, Coq~\cite{coq} and Nuprl~\cite{constable86} each support a
fixed logic too, but one far removed from first-order logic.  They are
explicitly concerned with computation.  A diverse collection of logics ---
type theories, process calculi, $\lambda$-calculi --- may be found in the
Computer Science literature.  Such logics require proof support.  Few proof
procedures exist, but the theorem prover can at least check that each
inference is valid.

A {\bf generic} theorem prover is one that can support many different
logics.  Most of these \cite{dawson90,mural,sawamura92} work by
implementing a syntactic framework that can express the features of typical
inference rules.  Isabelle's distinctive feature is its representation of
logics using a meta-logic.  This meta-logic is just a fragment of
higher-order logic; known proof theory may be used to demonstrate that the
representation is correct~\cite{paulson89}.  The representation has much in
common with the Edinburgh Logical Framework~\cite{harper-jacm} and with 
Felty's~\cite{felty93} use of $\lambda$Prolog to implement logics.

An inference rule in Isabelle is a generalized Horn clause.  Rules are
joined to make proofs by resolving such clauses.  Logical variables in
goals can be instantiated incrementally.  But Isabelle is not a resolution
theorem prover like Otter.  Isabelle's clauses are drawn from a richer,
higher-order language and a fully automatic search would be impractical.
Isabelle does not join clauses automatically, but under strict user
control.  You can conduct single-step proofs, use Isabelle's built-in proof
procedures, or develop new proof procedures using tactics and tacticals.

Isabelle's meta-logic is higher-order, based on the typed
$\lambda$-calculus.  So resolution cannot use ordinary unification, but
higher-order unification~\cite{huet75}.  This complicated procedure gives
Isabelle strong support for many logical formalisms involving variable
binding.

The diagram below illustrates some of the logics distributed with Isabelle.
These include first-order logic (intuitionistic and classical), the sequent
calculus, higher-order logic, Zermelo-Fraenkel set theory~\cite{suppes72},
a version of Constructive Type Theory~\cite{nordstrom90}, several modal
logics, and a Logic for Computable Functions.  Several experimental
logics are also available, such a term assignment calculus for linear
logic.  

\centerline{\epsfbox{Isa-logics.eps}}


\section*{How to read this book}
Isabelle is a large system, but beginners can get by with a few commands
and a basic knowledge of how Isabelle works.  Some knowledge of
Standard~\ML{} is essential because \ML{} is Isabelle's user interface.
Advanced Isabelle theorem proving can involve writing \ML{} code, possibly
with Isabelle's sources at hand.  My book on~\ML{}~\cite{paulson91} covers
much material connected with Isabelle, including a simple theorem prover.

The Isabelle documentation is divided into three parts, which serve
distinct purposes:
\begin{itemize}
\item {\em Introduction to Isabelle\/} describes the basic features of
  Isabelle.  This part is intended to be read through.  If you are
  impatient to get started, you might skip the first chapter, which
  describes Isabelle's meta-logic in some detail.  The other chapters
  present on-line sessions of increasing difficulty.  It also explains how
  to derive rules define theories, and concludes with an extended example:
  a Prolog interpreter.

\item {\em The Isabelle Reference Manual\/} contains information about most
  of the facilities of Isabelle, apart from particular object-logics.  This
  part would make boring reading, though browsing might be useful.  Mostly
  you should use it to locate facts quickly.

\item {\em Isabelle's Object-Logics\/} describes the various logics
  distributed with Isabelle.  Its final chapter explains how to define new
  logics.  The other chapters are intended for reference only.
\end{itemize}
This book should not be read from start to finish.  Instead you might read
a couple of chapters from {\em Introduction to Isabelle}, then try some
examples referring to the other parts, return to the {\em Introduction},
and so forth.  Starred sections discuss obscure matters and may be skipped
on a first reading.



\section*{Releases of Isabelle}\index{Isabelle!release history}
Isabelle was first distributed in 1986.  The 1987 version introduced a
higher-order meta-logic with an improved treatment of quantifiers.  The
1988 version added limited polymorphism and support for natural deduction.
The 1989 version included a parser and pretty printer generator.  The 1992
version introduced type classes, to support many-sorted and higher-order
logics.  The 1993 version provides greater support for theories and is
much faster.  

Isabelle is still under development.  Projects under consideration include
better support for inductive definitions, some means of recording proofs, a
graphical user interface, and developments in the standard object-logics.
I hope but cannot promise to maintain upwards compatibility.

Isabelle is available by anonymous ftp:
\begin{itemize}
\item University of Cambridge\\
        host {\tt ftp.cl.cam.ac.uk}\\
        directory {\tt ml}

\item Technical University of Munich\\
        host {\tt ftp.informatik.tu-muenchen.de}\\
        directory {\tt local/lehrstuhl/nipkow}
\end{itemize}
My electronic mail address is {\tt lcp\at cl.cam.ac.uk}.  Please report any
errors you find in this book and your problems or successes with Isabelle.


\subsection*{Acknowledgements} 
Tobias Nipkow has made immense contributions to Isabelle, including the
parser generator, type classes, the simplifier, and several object-logics.
He also arranged for several of his students to help.  Carsten Clasohm
implemented the theory database; Markus Wenzel implemented macros; Sonia
Mahjoub and Karin Nimmermann also contributed.  

Nipkow and his students wrote much of the documentation underlying this
book.  Nipkow wrote the first versions of \S\ref{sec:defining-theories},
Chap.\ts\ref{simp-chap}, Chap.\ts\ref{Defining-Logics} and part of
Chap.\ts\ref{theories}, and App.\ts\ref{app:TheorySyntax}.  Carsten Clasohm
contributed to Chap.\ts\ref{theories}.  Markus Wenzel contributed to
Chap.\ts\ref{Defining-Logics}.

David Aspinall, Sara Kalvala, Ina Kraan, Zhenyu Qian, Norbert Voelker and
Markus Wenzel suggested changes and corrections to the documentation.

Martin Coen, Rajeev Gor\'e, Philippe de Groote and Philippe No\"el helped
to develop Isabelle's standard object-logics.  David Aspinall performed
some useful research into theories and implemented an Isabelle Emacs mode.
Isabelle was developed using Dave Matthews's Standard~{\sc ml} compiler,
Poly/{\sc ml}.  

The research has been funded by numerous SERC grants dating from the Alvey
programme (grants GR/E0355.7, GR/G53279, GR/H40570) and by ESPRIT (projects
3245: Logical Frameworks and 6453: Types).


\index{ML}
