\chapter*{Preface}
\markboth{Preface}{Preface}   %or Preface ?
%%\addcontentsline{toc}{chapter}{Preface} 

Most theorem provers support a fixed logic, such as first-order or
equational logic.  They bring sophisticated proof procedures to bear upon
the conjectured formula.  The resolution prover Otter~\cite{wos-bledsoe} is
an impressive example.

{\sc alf}~\cite{alf}, Coq~\cite{coq} and Nuprl~\cite{constable86} each
support a fixed logic too.  These are higher-order type theories,
explicitly concerned with computation and capable of expressing
developments in constructive mathematics.  They are far removed from
classical first-order logic.

A diverse collection of logics --- type theories, process calculi,
$\lambda$-calculi --- may be found in the Computer Science literature.
Such logics require proof support.  Few proof procedures are known for
them, but the theorem prover can at least automate routine steps.

A {\bf generic} theorem prover is one that supports a variety of logics.
Some generic provers are noteworthy for their user interfaces
\cite{dawson90,mural,sawamura92}.  Most of them work by implementing a
syntactic framework that can express typical inference rules.  Isabelle's
distinctive feature is its representation of logics within a fragment of
higher-order logic, called the meta-logic.  The proof theory of
higher-order logic may be used to demonstrate that the representation is
correct~\cite{paulson89}.  The approach has much in common with the
Edinburgh Logical Framework~\cite{harper-jacm} and with
Felty's~\cite{felty93} use of $\lambda$Prolog to implement logics.

An inference rule in Isabelle is a generalized Horn clause.  Rules are
joined to make proofs by resolving such clauses.  Logical variables in
goals can be instantiated incrementally.  But Isabelle is not a resolution
theorem prover like Otter.  Isabelle's clauses are drawn from a richer
language and a fully automatic search would be impractical.  Isabelle does
not resolve clauses automatically, but under user direction.  You can
conduct single-step proofs, use Isabelle's built-in proof procedures, or
develop new proof procedures using tactics and tacticals.

Isabelle's meta-logic is higher-order, based on the simply typed
$\lambda$-calculus.  So resolution cannot use ordinary unification, but
higher-order unification~\cite{huet75}.  This complicated procedure gives
Isabelle strong support for many logical formalisms involving variable
binding.

The diagram below illustrates some of the logics distributed with Isabelle.
These include first-order logic (intuitionistic and classical), the sequent
calculus, higher-order logic, Zermelo-Fraenkel set theory~\cite{suppes72},
a version of Constructive Type Theory~\cite{nordstrom90}, several modal
logics, and a Logic for Computable Functions~\cite{paulson87}.  Several
experimental logics are being developed, such as linear logic.

\centerline{\epsfbox{gfx/Isa-logics.eps}}


\section*{How to read this book}
Isabelle is a complex system, but beginners can get by with a few commands
and a basic knowledge of how Isabelle works.  Some knowledge of
Standard~\ML{} is essential because \ML{} is Isabelle's user interface.
Advanced Isabelle theorem proving can involve writing \ML{} code, possibly
with Isabelle's sources at hand.  My book on~\ML{}~\cite{paulson91} covers
much material connected with Isabelle, including a simple theorem prover.

The Isabelle documentation is divided into three parts, which serve
distinct purposes:
\begin{itemize}
\item {\em Introduction to Isabelle\/} describes the basic features of
  Isabelle.  This part is intended to be read through.  If you are
  impatient to get started, you might skip the first chapter, which
  describes Isabelle's meta-logic in some detail.  The other chapters
  present on-line sessions of increasing difficulty.  It also explains how
  to derive rules define theories, and concludes with an extended example:
  a Prolog interpreter.

\item {\em The Isabelle Reference Manual\/} provides detailed information
  about Isabelle's facilities, excluding the object-logics.  This part
  would make boring reading, though browsing might be useful.  Mostly you
  should use it to locate facts quickly.

\item {\em Isabelle's Object-Logics\/} describes the various logics
  distributed with Isabelle.  The chapters are intended for reference only;
  they overlap somewhat so that each chapter can be read in isolation.
\end{itemize}
This book should not be read from start to finish.  Instead you might read
a couple of chapters from {\em Introduction to Isabelle}, then try some
examples referring to the other parts, return to the {\em Introduction},
and so forth.  Starred sections discuss obscure matters and may be skipped
on a first reading.



\section*{Releases of Isabelle}
Isabelle was first distributed in 1986.  The 1987 version introduced a
higher-order meta-logic with an improved treatment of quantifiers.  The
1988 version added limited polymorphism and support for natural deduction.
The 1989 version included a parser and pretty printer generator.  The 1992
version introduced type classes, to support many-sorted and higher-order
logics.  The 1993 version provides greater support for theories and is
much faster.  

Isabelle is still under development.  Projects under consideration include
better support for inductive definitions, some means of recording proofs, a
graphical user interface, and developments in the standard object-logics.
I hope but cannot promise to maintain upwards compatibility.

Isabelle is available by anonymous ftp:
\begin{itemize}
\item University of Cambridge\\
        host {\tt ftp.cl.cam.ac.uk}\\
        directory {\tt ml}

\item Technical University of Munich\\
        host {\tt ftp.informatik.tu-muenchen.de}\\
        directory {\tt local/lehrstuhl/nipkow}
\end{itemize}
The electronic distribution list {\tt isabelle-users\at cl.cam.ac.uk}
provides a forum for discussing problems and applications involving
Isabelle.  To join, send me a message via {\tt lcp\at cl.cam.ac.uk}.
Please notify me of any errors you find in this book.

\section*{Acknowledgements} 
Tobias Nipkow has made immense contributions to Isabelle, including the
parser generator, type classes, the simplifier, and several object-logics.
He also arranged for several of his students to help.  Carsten Clasohm
implemented the theory database; Markus Wenzel implemented macros; Sonia
Mahjoub and Karin Nimmermann also contributed.  

Nipkow and his students wrote much of the documentation underlying this
book.  Nipkow wrote the first versions of \S\ref{sec:defining-theories},
\S\ref{sec:ref-defining-theories}, Chap.\ts\ref{Defining-Logics},
Chap.\ts\ref{simp-chap} and App.\ts\ref{app:TheorySyntax}\@.  Carsten
Clasohm contributed to Chap.\ts\ref{theories}.  Markus Wenzel contributed
to Chap.\ts\ref{chap:syntax}.  Nipkow also provided the quotation at
the front.

David Aspinall, Sara Kalvala, Ina Kraan, Chris Owens, Zhenyu Qian, Norbert
V{\"o}lker and Markus Wenzel suggested changes and corrections to the
documentation.

Martin Coen, Rajeev Gor\'e, Philippe de Groote and Philippe No\"el helped
to develop Isabelle's standard object-logics.  David Aspinall performed
some useful research into theories and implemented an Isabelle Emacs mode.
Isabelle was developed using Dave Matthews's Standard~{\sc ml} compiler,
Poly/{\sc ml}.  

The research has been funded by numerous SERC grants dating from the Alvey
programme (grants GR/E0355.7, GR/G53279, GR/H40570) and by ESPRIT (projects
3245: Logical Frameworks and 6453: Types).
