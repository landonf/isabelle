%
\begin{isabellebody}%
\def\isabellecontext{Documents}%
\isamarkupfalse%
%
\isamarkupsection{Concrete Syntax \label{sec:concrete-syntax}%
}
\isamarkuptrue%
%
\begin{isamarkuptext}%
The core concept of Isabelle's framework for concrete syntax is that
  of \bfindex{mixfix annotations}.  Associated with any kind of
  constant declaration, mixfixes affect both the grammar productions
  for the parser and output templates for the pretty printer.

  In full generality, parser and pretty printer configuration is a
  subtle affair \cite{isabelle-ref}.  Your syntax specifications need
  to interact properly with the existing setup of Isabelle/Pure and
  Isabelle/HOL\@.  To avoid creating ambiguities with existing
  elements, it is particularly important to give new syntactic
  constructs the right precedence.

  \medskip Subsequently we introduce a few simple syntax declaration
  forms that already cover many common situations fairly well.%
\end{isamarkuptext}%
\isamarkuptrue%
%
\isamarkupsubsection{Infix Annotations%
}
\isamarkuptrue%
%
\begin{isamarkuptext}%
Syntax annotations may be included wherever constants are declared,
  such as \isacommand{consts} and \isacommand{constdefs} --- and also
  \isacommand{datatype}, which declares constructor operations.
  Type-constructors may be annotated as well, although this is less
  frequently encountered in practice (the infix type \isa{{\isasymtimes}} comes
  to mind).

  Infix declarations\index{infix annotations} provide a useful special
  case of mixfixes.  The following example of the exclusive-or
  operation on boolean values illustrates typical infix declarations.%
\end{isamarkuptext}%
\isamarkuptrue%
\isacommand{constdefs}\isanewline
\ \ xor\ {\isacharcolon}{\isacharcolon}\ {\isachardoublequote}bool\ {\isasymRightarrow}\ bool\ {\isasymRightarrow}\ bool{\isachardoublequote}\ \ \ \ {\isacharparenleft}\isakeyword{infixl}\ {\isachardoublequote}{\isacharbrackleft}{\isacharplus}{\isacharbrackright}{\isachardoublequote}\ {\isadigit{6}}{\isadigit{0}}{\isacharparenright}\isanewline
\ \ {\isachardoublequote}A\ {\isacharbrackleft}{\isacharplus}{\isacharbrackright}\ B\ {\isasymequiv}\ {\isacharparenleft}A\ {\isasymand}\ {\isasymnot}\ B{\isacharparenright}\ {\isasymor}\ {\isacharparenleft}{\isasymnot}\ A\ {\isasymand}\ B{\isacharparenright}{\isachardoublequote}\isamarkupfalse%
%
\begin{isamarkuptext}%
\noindent Now \isa{xor\ A\ B} and \isa{A\ {\isacharbrackleft}{\isacharplus}{\isacharbrackright}\ B} refer to the
  same expression internally.  Any curried function with at least two
  arguments may be given infix syntax.  For partial applications with
  fewer than two operands, there is a notation using the prefix~\isa{op}.  For instance, \isa{xor} without arguments is represented as
  \isa{op\ {\isacharbrackleft}{\isacharplus}{\isacharbrackright}}; together with ordinary function application, this
  turns \isa{xor\ A} into \isa{op\ {\isacharbrackleft}{\isacharplus}{\isacharbrackright}\ A}.

  \medskip The keyword \isakeyword{infixl} seen above specifies an
  infix operator that is nested to the \emph{left}: in iterated
  applications the more complex expression appears on the left-hand
  side, and \isa{A\ {\isacharbrackleft}{\isacharplus}{\isacharbrackright}\ B\ {\isacharbrackleft}{\isacharplus}{\isacharbrackright}\ C} stands for \isa{{\isacharparenleft}A\ {\isacharbrackleft}{\isacharplus}{\isacharbrackright}\ B{\isacharparenright}\ {\isacharbrackleft}{\isacharplus}{\isacharbrackright}\ C}.  Similarly, \isakeyword{infixr} means nesting to the
  \emph{right}, reading \isa{A\ {\isacharbrackleft}{\isacharplus}{\isacharbrackright}\ B\ {\isacharbrackleft}{\isacharplus}{\isacharbrackright}\ C} as \isa{A\ {\isacharbrackleft}{\isacharplus}{\isacharbrackright}\ {\isacharparenleft}B\ {\isacharbrackleft}{\isacharplus}{\isacharbrackright}\ C{\isacharparenright}}.  A \emph{non-oriented} declaration via \isakeyword{infix}
  would render \isa{A\ {\isacharbrackleft}{\isacharplus}{\isacharbrackright}\ B\ {\isacharbrackleft}{\isacharplus}{\isacharbrackright}\ C} illegal, but demand explicit
  parentheses to indicate the intended grouping.

  The string \isa{{\isachardoublequote}{\isacharbrackleft}{\isacharplus}{\isacharbrackright}{\isachardoublequote}} in our annotation refers to the
  concrete syntax to represent the operator (a literal token), while
  the number \isa{{\isadigit{6}}{\isadigit{0}}} determines the precedence of the construct:
  the syntactic priorities of the arguments and result.  Isabelle/HOL
  already uses up many popular combinations of ASCII symbols for its
  own use, including both \isa{{\isacharplus}} and \isa{{\isacharplus}{\isacharplus}}.  Longer
  character combinations are more likely to be still available for
  user extensions, such as our~\isa{{\isacharbrackleft}{\isacharplus}{\isacharbrackright}}.

  Operator precedences have a range of 0--1000.  Very low or high
  priorities are reserved for the meta-logic.  HOL syntax mainly uses
  the range of 10--100: the equality infix \isa{{\isacharequal}} is centered at
  50; logical connectives (like \isa{{\isasymor}} and \isa{{\isasymand}}) are
  below 50; algebraic ones (like \isa{{\isacharplus}} and \isa{{\isacharasterisk}}) are
  above 50.  User syntax should strive to coexist with common HOL
  forms, or use the mostly unused range 100--900.%
\end{isamarkuptext}%
\isamarkuptrue%
%
\isamarkupsubsection{Mathematical Symbols \label{sec:syntax-symbols}%
}
\isamarkuptrue%
%
\begin{isamarkuptext}%
Concrete syntax based on ASCII characters has inherent limitations.
  Mathematical notation demands a larger repertoire of glyphs.
  Several standards of extended character sets have been proposed over
  decades, but none has become universally available so far.  Isabelle
  has its own notion of \bfindex{symbols} as the smallest entities of
  source text, without referring to internal encodings.  There are
  three kinds of such ``generalized characters'':

  \begin{enumerate}

  \item 7-bit ASCII characters

  \item named symbols: \verb,\,\verb,<,$ident$\verb,>,

  \item named control symbols: \verb,\,\verb,<^,$ident$\verb,>,

  \end{enumerate}

  Here $ident$ is any sequence of letters. 
  This results in an infinite store of symbols, whose
  interpretation is left to further front-end tools.  For example, the
  user-interface of Proof~General + X-Symbol and the Isabelle document
  processor (see \S\ref{sec:document-preparation}) display the
  \verb,\,\verb,<forall>, symbol as~\isa{{\isasymforall}}.

  A list of standard Isabelle symbols is given in
  \cite[appendix~A]{isabelle-sys}.  You may introduce your own
  interpretation of further symbols by configuring the appropriate
  front-end tool accordingly, e.g.\ by defining certain {\LaTeX}
  macros (see also \S\ref{sec:doc-prep-symbols}).  There are also a
  few predefined control symbols, such as \verb,\,\verb,<^sub>, and
  \verb,\,\verb,<^sup>, for sub- and superscript of the subsequent
  printable symbol, respectively.  For example, \verb,A\<^sup>\<star>, is
  output as \isa{A\isactrlsup {\isasymstar}}.

  A number of symbols are considered letters by the Isabelle lexer 
  and can be used as part of identifiers. These are the greek letters
  \isa{{\isasymalpha}} (\verb+\+\verb+<alpha>+), \isa{{\isasymbeta}}, etc apart from
  \isa{{\isasymlambda}}, caligraphic letters like \isa{{\isasymA}}
  (\verb+\+\verb+<A>+) and \isa{{\isasymAA}} (\verb+\+\verb+<AA>+), 
  and the special control symbols \verb+\+\verb+<^isub>+ and 
  \verb+\+\verb+<^isup>+ for single letter sub and super scripts. This
  means that the input 

  \medskip
  {\small\noindent \verb,\,\verb,<forall>\,\verb,<alpha>\<^isub>1.\,\verb,<alpha>\<^isub>1=\,\verb,<Pi>\<^isup>\<A>,}

  \medskip
  \noindent is recognized as the term \isa{{\isasymforall}{\isasymalpha}\isactrlisub {\isadigit{1}}{\isachardot}\ {\isasymalpha}\isactrlisub {\isadigit{1}}\ {\isacharequal}\ {\isasymPi}\isactrlisup {\isasymA}} 
  by Isabelle. Note that \isa{{\isasymPi}\isactrlisup {\isasymA}} is a single entity like 
  \isa{PA}, not an exponentiation.


  \medskip Replacing our definition of \isa{xor} by the following
  specifies an Isabelle symbol for the new operator:%
\end{isamarkuptext}%
\isamarkuptrue%
\isamarkupfalse%
\isamarkupfalse%
\isacommand{constdefs}\isanewline
\ \ xor\ {\isacharcolon}{\isacharcolon}\ {\isachardoublequote}bool\ {\isasymRightarrow}\ bool\ {\isasymRightarrow}\ bool{\isachardoublequote}\ \ \ \ {\isacharparenleft}\isakeyword{infixl}\ {\isachardoublequote}{\isasymoplus}{\isachardoublequote}\ {\isadigit{6}}{\isadigit{0}}{\isacharparenright}\isanewline
\ \ {\isachardoublequote}A\ {\isasymoplus}\ B\ {\isasymequiv}\ {\isacharparenleft}A\ {\isasymand}\ {\isasymnot}\ B{\isacharparenright}\ {\isasymor}\ {\isacharparenleft}{\isasymnot}\ A\ {\isasymand}\ B{\isacharparenright}{\isachardoublequote}\isamarkupfalse%
\isamarkupfalse%
%
\begin{isamarkuptext}%
\noindent The X-Symbol package within Proof~General provides several
  input methods to enter \isa{{\isasymoplus}} in the text.  If all fails one may
  just type a named entity \verb,\,\verb,<oplus>, by hand; the
  corresponding symbol will be displayed after further input.

  \medskip More flexible is to provide alternative syntax forms
  through the \bfindex{print mode} concept~\cite{isabelle-ref}.  By
  convention, the mode of ``$xsymbols$'' is enabled whenever
  Proof~General's X-Symbol mode or {\LaTeX} output is active.  Now
  consider the following hybrid declaration of \isa{xor}:%
\end{isamarkuptext}%
\isamarkuptrue%
\isamarkupfalse%
\isamarkupfalse%
\isacommand{constdefs}\isanewline
\ \ xor\ {\isacharcolon}{\isacharcolon}\ {\isachardoublequote}bool\ {\isasymRightarrow}\ bool\ {\isasymRightarrow}\ bool{\isachardoublequote}\ \ \ \ {\isacharparenleft}\isakeyword{infixl}\ {\isachardoublequote}{\isacharbrackleft}{\isacharplus}{\isacharbrackright}{\isasymignore}{\isachardoublequote}\ {\isadigit{6}}{\isadigit{0}}{\isacharparenright}\isanewline
\ \ {\isachardoublequote}A\ {\isacharbrackleft}{\isacharplus}{\isacharbrackright}{\isasymignore}\ B\ {\isasymequiv}\ {\isacharparenleft}A\ {\isasymand}\ {\isasymnot}\ B{\isacharparenright}\ {\isasymor}\ {\isacharparenleft}{\isasymnot}\ A\ {\isasymand}\ B{\isacharparenright}{\isachardoublequote}\isanewline
\isanewline
\isamarkupfalse%
\isacommand{syntax}\ {\isacharparenleft}xsymbols{\isacharparenright}\isanewline
\ \ xor\ {\isacharcolon}{\isacharcolon}\ {\isachardoublequote}bool\ {\isasymRightarrow}\ bool\ {\isasymRightarrow}\ bool{\isachardoublequote}\ \ \ \ {\isacharparenleft}\isakeyword{infixl}\ {\isachardoublequote}{\isasymoplus}{\isasymignore}{\isachardoublequote}\ {\isadigit{6}}{\isadigit{0}}{\isacharparenright}\isamarkupfalse%
\isamarkupfalse%
%
\begin{isamarkuptext}%
The \commdx{syntax} command introduced here acts like
  \isakeyword{consts}, but without declaring a logical constant.  The
  print mode specification of \isakeyword{syntax}, here \isa{{\isacharparenleft}xsymbols{\isacharparenright}}, is optional.  Also note that its type merely serves
  for syntactic purposes, and is \emph{not} checked for consistency
  with the real constant.

  \medskip We may now write \isa{A\ {\isacharbrackleft}{\isacharplus}{\isacharbrackright}\ B} or \isa{A\ {\isasymoplus}\ B} in
  input, while output uses the nicer syntax of $xsymbols$ whenever
  that print mode is active.  Such an arrangement is particularly
  useful for interactive development, where users may type ASCII text
  and see mathematical symbols displayed during proofs.%
\end{isamarkuptext}%
\isamarkuptrue%
%
\isamarkupsubsection{Prefix Annotations%
}
\isamarkuptrue%
%
\begin{isamarkuptext}%
Prefix syntax annotations\index{prefix annotation} are another form
  of mixfixes \cite{isabelle-ref}, without any template arguments or
  priorities --- just some literal syntax.  The following example
  associates common symbols with the constructors of a datatype.%
\end{isamarkuptext}%
\isamarkuptrue%
\isacommand{datatype}\ currency\ {\isacharequal}\isanewline
\ \ \ \ Euro\ nat\ \ \ \ {\isacharparenleft}{\isachardoublequote}{\isasymeuro}{\isachardoublequote}{\isacharparenright}\isanewline
\ \ {\isacharbar}\ Pounds\ nat\ \ {\isacharparenleft}{\isachardoublequote}{\isasympounds}{\isachardoublequote}{\isacharparenright}\isanewline
\ \ {\isacharbar}\ Yen\ nat\ \ \ \ \ {\isacharparenleft}{\isachardoublequote}{\isasymyen}{\isachardoublequote}{\isacharparenright}\isanewline
\ \ {\isacharbar}\ Dollar\ nat\ \ {\isacharparenleft}{\isachardoublequote}{\isachardollar}{\isachardoublequote}{\isacharparenright}\isamarkupfalse%
%
\begin{isamarkuptext}%
\noindent Here the mixfix annotations on the rightmost column happen
  to consist of a single Isabelle symbol each: \verb,\,\verb,<euro>,,
  \verb,\,\verb,<pounds>,, \verb,\,\verb,<yen>,, and \verb,$,.  Recall
  that a constructor like \isa{Euro} actually is a function \isa{nat\ {\isasymRightarrow}\ currency}.  The expression \isa{Euro\ {\isadigit{1}}{\isadigit{0}}} will be
  printed as \isa{{\isasymeuro}\ {\isadigit{1}}{\isadigit{0}}}; only the head of the application is
  subject to our concrete syntax.  This rather simple form already
  achieves conformance with notational standards of the European
  Commission.

  Prefix syntax works the same way for \isakeyword{consts} or
  \isakeyword{constdefs}.%
\end{isamarkuptext}%
\isamarkuptrue%
%
\isamarkupsubsection{Syntax Translations \label{sec:syntax-translations}%
}
\isamarkuptrue%
%
\begin{isamarkuptext}%
Mixfix syntax annotations merely decorate particular constant
  application forms with concrete syntax, for instance replacing \
  \isa{xor\ A\ B} by \isa{A\ {\isasymoplus}\ B}.  Occasionally, the
  relationship between some piece of notation and its internal form is
  more complicated.  Here we need \bfindex{syntax translations}.

  Using the \isakeyword{syntax}\index{syntax (command)}, command we
  introduce uninterpreted notational elements.  Then
  \commdx{translations} relate input forms to complex logical
  expressions.  This provides a simple mechanism for syntactic macros;
  even heavier transformations may be written in ML
  \cite{isabelle-ref}.

  \medskip A typical use of syntax translations is to introduce
  relational notation for membership in a set of pair, replacing \
  \isa{{\isacharparenleft}x{\isacharcomma}\ y{\isacharparenright}\ {\isasymin}\ sim} by \isa{x\ {\isasymapprox}\ y}.%
\end{isamarkuptext}%
\isamarkuptrue%
\isacommand{consts}\isanewline
\ \ sim\ {\isacharcolon}{\isacharcolon}\ {\isachardoublequote}{\isacharparenleft}{\isacharprime}a\ {\isasymtimes}\ {\isacharprime}a{\isacharparenright}\ set{\isachardoublequote}\isanewline
\isanewline
\isamarkupfalse%
\isacommand{syntax}\isanewline
\ \ {\isachardoublequote}{\isacharunderscore}sim{\isachardoublequote}\ {\isacharcolon}{\isacharcolon}\ {\isachardoublequote}{\isacharprime}a\ {\isasymRightarrow}\ {\isacharprime}a\ {\isasymRightarrow}\ bool{\isachardoublequote}\ \ \ \ {\isacharparenleft}\isakeyword{infix}\ {\isachardoublequote}{\isasymapprox}{\isachardoublequote}\ {\isadigit{5}}{\isadigit{0}}{\isacharparenright}\isanewline
\isamarkupfalse%
\isacommand{translations}\isanewline
\ \ {\isachardoublequote}x\ {\isasymapprox}\ y{\isachardoublequote}\ {\isasymrightleftharpoons}\ {\isachardoublequote}{\isacharparenleft}x{\isacharcomma}\ y{\isacharparenright}\ {\isasymin}\ sim{\isachardoublequote}\isamarkupfalse%
%
\begin{isamarkuptext}%
\noindent Here the name of the dummy constant \isa{{\isacharunderscore}sim} does
  not matter, as long as it is not used elsewhere.  Prefixing an
  underscore is a common convention.  The \isakeyword{translations}
  declaration already uses concrete syntax on the left-hand side;
  internally we relate a raw application \isa{{\isacharunderscore}sim\ x\ y} with
  \isa{{\isacharparenleft}x{\isacharcomma}\ y{\isacharparenright}\ {\isasymin}\ sim}.

  \medskip Another common application of syntax translations is to
  provide variant versions of fundamental relational expressions, such
  as \isa{{\isasymnoteq}} for negated equalities.  The following declaration
  stems from Isabelle/HOL itself:%
\end{isamarkuptext}%
\isamarkuptrue%
\isacommand{syntax}\ {\isachardoublequote}{\isacharunderscore}not{\isacharunderscore}equal{\isachardoublequote}\ {\isacharcolon}{\isacharcolon}\ {\isachardoublequote}{\isacharprime}a\ {\isasymRightarrow}\ {\isacharprime}a\ {\isasymRightarrow}\ bool{\isachardoublequote}\ \ \ \ {\isacharparenleft}\isakeyword{infixl}\ {\isachardoublequote}{\isasymnoteq}{\isasymignore}{\isachardoublequote}\ {\isadigit{5}}{\isadigit{0}}{\isacharparenright}\isanewline
\isamarkupfalse%
\isacommand{translations}\ {\isachardoublequote}x\ {\isasymnoteq}{\isasymignore}\ y{\isachardoublequote}\ {\isasymrightleftharpoons}\ {\isachardoublequote}{\isasymnot}\ {\isacharparenleft}x\ {\isacharequal}\ y{\isacharparenright}{\isachardoublequote}\isamarkupfalse%
%
\begin{isamarkuptext}%
\noindent Normally one would introduce derived concepts like this
  within the logic, using \isakeyword{consts} + \isakeyword{defs}
  instead of \isakeyword{syntax} + \isakeyword{translations}.  The
  present formulation has the virtue that expressions are immediately
  replaced by the ``definition'' upon parsing; the effect is reversed
  upon printing.

  This sort of translation is appropriate when the defined concept is
  a trivial variation on an existing one.  On the other hand, syntax
  translations do not scale up well to large hierarchies of concepts.
  Translations do not replace definitions!%
\end{isamarkuptext}%
\isamarkuptrue%
%
\isamarkupsection{Document Preparation \label{sec:document-preparation}%
}
\isamarkuptrue%
%
\begin{isamarkuptext}%
Isabelle/Isar is centered around the concept of \bfindex{formal
  proof documents}\index{documents|bold}.  The outcome of a formal
  development effort is meant to be a human-readable record, presented
  as browsable PDF file or printed on paper.  The overall document
  structure follows traditional mathematical articles, with sections,
  intermediate explanations, definitions, theorems and proofs.

  \medskip The Isabelle document preparation system essentially acts
  as a front-end to {\LaTeX}.  After checking specifications and
  proofs formally, the theory sources are turned into typesetting
  instructions in a schematic manner.  This lets you write authentic
  reports on theory developments with little effort: many technical
  consistency checks are handled by the system.

  Here is an example to illustrate the idea of Isabelle document
  preparation.%
\end{isamarkuptext}%
\isamarkuptrue%
%
\begin{quotation}
%
\begin{isamarkuptext}%
The following datatype definition of \isa{{\isacharprime}a\ bintree} models
  binary trees with nodes being decorated by elements of type \isa{{\isacharprime}a}.%
\end{isamarkuptext}%
\isamarkuptrue%
\isacommand{datatype}\ {\isacharprime}a\ bintree\ {\isacharequal}\isanewline
\ \ \ \ \ Leaf\ {\isacharbar}\ Branch\ {\isacharprime}a\ \ {\isachardoublequote}{\isacharprime}a\ bintree{\isachardoublequote}\ \ {\isachardoublequote}{\isacharprime}a\ bintree{\isachardoublequote}\isamarkupfalse%
%
\begin{isamarkuptext}%
\noindent The datatype induction rule generated here is of the form
  \begin{isabelle}%
\ {\isasymlbrakk}P\ Leaf{\isacharsemicolon}\isanewline
\isaindent{\ \ }{\isasymAnd}a\ bintree{\isadigit{1}}\ bintree{\isadigit{2}}{\isachardot}\isanewline
\isaindent{\ \ \ \ \ }{\isasymlbrakk}P\ bintree{\isadigit{1}}{\isacharsemicolon}\ P\ bintree{\isadigit{2}}{\isasymrbrakk}\ {\isasymLongrightarrow}\ P\ {\isacharparenleft}Branch\ a\ bintree{\isadigit{1}}\ bintree{\isadigit{2}}{\isacharparenright}{\isasymrbrakk}\isanewline
\isaindent{\ }{\isasymLongrightarrow}\ P\ bintree%
\end{isabelle}%
\end{isamarkuptext}%
\isamarkuptrue%
%
\end{quotation}
%
\begin{isamarkuptext}%
\noindent The above document output has been produced as follows:

  \begin{ttbox}
  text {\ttlbrace}*
    The following datatype definition of {\at}{\ttlbrace}text "'a bintree"{\ttrbrace}
    models binary trees with nodes being decorated by elements
    of type {\at}{\ttlbrace}typ 'a{\ttrbrace}.
  *{\ttrbrace}

  datatype 'a bintree =
    Leaf | Branch 'a  "'a bintree"  "'a bintree"
  \end{ttbox}
  \begin{ttbox}
  text {\ttlbrace}*
    {\ttback}noindent The datatype induction rule generated here is
    of the form {\at}{\ttlbrace}thm [display] bintree.induct [no_vars]{\ttrbrace}
  *{\ttrbrace}
  \end{ttbox}\vspace{-\medskipamount}

  \noindent Here we have augmented the theory by formal comments
  (using \isakeyword{text} blocks), the informal parts may again refer
  to formal entities by means of ``antiquotations'' (such as
  \texttt{\at}\verb,{text "'a bintree"}, or
  \texttt{\at}\verb,{typ 'a},), see also \S\ref{sec:doc-prep-text}.%
\end{isamarkuptext}%
\isamarkuptrue%
%
\isamarkupsubsection{Isabelle Sessions%
}
\isamarkuptrue%
%
\begin{isamarkuptext}%
In contrast to the highly interactive mode of Isabelle/Isar theory
  development, the document preparation stage essentially works in
  batch-mode.  An Isabelle \bfindex{session} consists of a collection
  of source files that may contribute to an output document.  Each
  session is derived from a single parent, usually an object-logic
  image like \texttt{HOL}.  This results in an overall tree structure,
  which is reflected by the output location in the file system
  (usually rooted at \verb,~/isabelle/browser_info,).

  \medskip The easiest way to manage Isabelle sessions is via
  \texttt{isatool mkdir} (generates an initial session source setup)
  and \texttt{isatool make} (run sessions controlled by
  \texttt{IsaMakefile}).  For example, a new session
  \texttt{MySession} derived from \texttt{HOL} may be produced as
  follows:

\begin{verbatim}
  isatool mkdir HOL MySession
  isatool make
\end{verbatim}

  The \texttt{isatool make} job also informs about the file-system
  location of the ultimate results.  The above dry run should be able
  to produce some \texttt{document.pdf} (with dummy title, empty table
  of contents etc.).  Any failure at this stage usually indicates
  technical problems of the {\LaTeX} installation.\footnote{Especially
  make sure that \texttt{pdflatex} is present; if in doubt one may
  fall back on DVI output by changing \texttt{usedir} options in
  \texttt{IsaMakefile} \cite{isabelle-sys}.}

  \medskip The detailed arrangement of the session sources is as
  follows.

  \begin{itemize}

  \item Directory \texttt{MySession} holds the required theory files
  $T@1$\texttt{.thy}, \dots, $T@n$\texttt{.thy}.

  \item File \texttt{MySession/ROOT.ML} holds appropriate ML commands
  for loading all wanted theories, usually just
  ``\texttt{use_thy"$T@i$";}'' for any $T@i$ in leaf position of the
  dependency graph.

  \item Directory \texttt{MySession/document} contains everything
  required for the {\LaTeX} stage; only \texttt{root.tex} needs to be
  provided initially.

  The latter file holds appropriate {\LaTeX} code to commence a
  document (\verb,\documentclass, etc.), and to include the generated
  files $T@i$\texttt{.tex} for each theory.  Isabelle will generate a
  file \texttt{session.tex} holding {\LaTeX} commands to include all
  generated theory output files in topologically sorted order, so
  \verb,%
\begin{isabellebody}%
\def\isabellecontext{Locales}%
%
\isadelimtheory
\isanewline
%
\endisadelimtheory
%
\isatagtheory
%
\endisatagtheory
{\isafoldtheory}%
%
\isadelimtheory
%
\endisadelimtheory
%
\isadelimML
%
\endisadelimML
%
\isatagML
%
\endisatagML
{\isafoldML}%
%
\isadelimML
%
\endisadelimML
%
\isamarkupsection{Overview%
}
\isamarkuptrue%
%
\begin{isamarkuptext}%
The present text is based on~\cite{Ballarin2004a}.  It was updated
  for for Isabelle2005, but does not cover locale interpretation.

  Locales are an extension of the Isabelle proof assistant.  They
  provide support for modular reasoning. Locales were initially
  developed by Kamm\"uller~\cite{Kammuller2000} to support reasoning
  in abstract algebra, but are applied also in other domains --- for
  example, bytecode verification~\cite{Klein2003}.

  Kamm\"uller's original design, implemented in Isabelle99, provides, in
  addition to
  means for declaring locales, a set of ML functions that were used
  along with ML tactics in a proof.  In the meantime, the input format
  for proof in Isabelle has changed and users write proof
  scripts in ML only rarely if at all.  Two new proof styles are
  available, and can
  be used interchangeably: linear proof scripts that closely resemble ML
  tactics, and the structured Isar proof language by
  Wenzel~\cite{Wenzel2002a}.  Subsequently, Wenzel re-implemented
  locales for
  the new proof format.  The implementation, available with
  Isabelle2003, constitutes a complete re-design and exploits that
  both Isar and locales are based on the notion of context,
  and thus locales are seen as a natural extension of Isar.
  Nevertheless, locales can also be used with proof scripts:
  their use does not require a deep understanding of the structured
  Isar proof style.

  At the same time, Wenzel considerably extended locales.  The most
  important addition are locale expressions, which allow to combine
  locales more freely.  Previously only
  linear inheritance was possible.  Now locales support multiple
  inheritance through a normalisation algorithm.  New are also
  structures, which provide special syntax for locale parameters that
  represent algebraic structures.

  Unfortunately, Wenzel provided only an implementation but hardly any
  documentation.  Besides providing documentation, the present paper
  is a high-level description of locales, and in particular locale
  expressions.  It is meant as a first step towards the semantics of
  locales, and also as a base for comparing locales with module concepts
  in other provers.  It also constitutes the base for future
  extensions of locales in Isabelle.
  The description was derived mainly by experimenting
  with locales and partially also by inspecting the code.

  The main contribution of the author of the present paper is the
  abstract description of Wenzel's version of locales, and in
  particular of the normalisation algorithm for locale expressions (see
  Section~\ref{sec-normal-forms}).  Contributions to the
  implementation are confined to bug fixes and to provisions that
  enable the use of locales with linear proof scripts.

  Concepts are introduced along with examples, so that the text can be
  used as tutorial.  It is assumed that the reader is somewhat
  familiar with Isabelle proof scripts.  Examples have been phrased as
  structured
  Isar proofs.  However, in order to understand the key concepts,
  including locales expressions and their normalisation, detailed
  knowledge of Isabelle is not necessary. 

\nocite{Nipkow2003,Wenzel2002b,Wenzel2003}%
\end{isamarkuptext}%
\isamarkuptrue%
%
\isamarkupsection{Locales: Beyond Proof Contexts%
}
\isamarkuptrue%
%
\begin{isamarkuptext}%
In tactic-based provers the application of a sequence of proof
  tactics leads to a proof state.  This state is usually hard to
  predict from looking at the tactic script, unless one replays the
  proof step-by-step.  The structured proof language Isar is
  different.  It is additionally based on \emph{proof contexts},
  which are directly visible in Isar scripts, and since tactic
  sequences tend to be short, this commonly leads to clearer proof
  scripts.

  Goals are stated with the \textbf{theorem}
  command.  This is followed by a proof.  When discharging a goal
  requires an elaborate argument
  (rather than the application of a single tactic) a new context
  may be entered (\textbf{proof}).  Inside the context, variables may
  be fixed (\textbf{fix}), assumptions made (\textbf{assume}) and
  intermediate goals stated (\textbf{have}) and proved.  The
  assumptions must be dischargeable by premises of the surrounding
  goal, and once this goal has been proved (\textbf{show}) the proof context
  can be closed (\textbf{qed}). Contexts inherit from surrounding
  contexts, but it is not possible
  to export from them (with exception of the proved goal);
  they ``disappear'' after the closing \textbf{qed}.
  Facts may have attributes --- for example, identifying them as
  default to the simplifier or classical reasoner.

  Locales extend proof contexts in various ways:
  \begin{itemize}
  \item
    Locales are usually \emph{named}.  This makes them persistent.
  \item
    Fixed variables may have \emph{syntax}.
  \item
    It is possible to \emph{add} and \emph{export} facts.
  \item
    Locales can be combined and modified with \emph{locale
    expressions}.
  \end{itemize}
  The Locales facility extends the Isar language: it provides new ways
  of stating and managing facts, but it does not modify the language
  for proofs.  Its purpose is to support writing modular proofs.%
\end{isamarkuptext}%
\isamarkuptrue%
%
\isamarkupsection{Simple Locales%
}
\isamarkuptrue%
%
\isamarkupsubsection{Syntax and Terminology%
}
\isamarkuptrue%
%
\begin{isamarkuptext}%
The grammar of Isar is extended by commands for locales as shown in
  Figure~\ref{fig-grammar}.
  A key concept, introduced by Wenzel, is that
  locales are (internally) lists
  of \emph{context elements}.  There are five kinds, identified
  by the keywords \textbf{fixes}, \textbf{constrains},
  \textbf{assumes}, \textbf{defines} and \textbf{notes}.

  \begin{figure}
  \hrule
  \vspace{2ex}
  \begin{small}
  \begin{tabular}{l>$c<$l}
  \textit{attr-name} & ::=
  & \textit{name} $|$ \textit{attribute} $|$
    \textit{name} \textit{attribute} \\

  \textit{locale-expr}  & ::= 
  & \textit{locale-expr1} ( ``\textbf{+}'' \textit{locale-expr1} )$^*$ \\
  \textit{locale-expr1} & ::=
  & ( \textit{qualified-name} $|$
    ``\textbf{(}'' \textit{locale-expr} ``\textbf{)}'' ) \\
  & & ( \textit{name} [ \textit{mixfix} ] $|$ ``\textbf{\_}'' )$^*$ \\

  \textit{fixes} & ::=
  & \textit{name} [ ``\textbf{::}'' \textit{type} ]
    [ ``\textbf{(}'' \textbf{structure} ``\textbf{)}'' $|$
    \textit{mixfix} ] \\
  \textit{constrains} & ::=
  & \textit{name} ``\textbf{::}'' \textit{type} \\
  \textit{assumes} & ::=
  & [ \textit{attr-name} ``\textbf{:}'' ] \textit{proposition} \\
  \textit{defines} & ::=
  & [ \textit{attr-name} ``\textbf{:}'' ] \textit{proposition} \\
  \textit{notes} & ::=
  & [ \textit{attr-name} ``\textbf{=}'' ]
    ( \textit{qualified-name} [ \textit{attribute} ] )$^+$ \\

  \textit{element} & ::=
  & \textbf{fixes} \textit{fixes} ( \textbf{and} \textit{fixes} )$^*$ \\
  & |
  & \textbf{constrains} \textit{constrains}
    ( \textbf{and} \textit{constrains} )$^*$ \\
  & |
  & \textbf{assumes} \textit{assumes} ( \textbf{and} \textit{assumes} )$^*$ \\
  & |
  & \textbf{defines} \textit{defines} ( \textbf{and} \textit{defines} )$^*$ \\
  & |
  & \textbf{notes} \textit{notes} ( \textbf{and} \textit{notes} )$^*$ \\
  \textit{element1} & ::=
  & \textit{element} \\
  & | & \textbf{includes} \textit{locale-expr} \\

  \textit{locale} & ::=
  & \textit{element}$^+$ \\
  & | & \textit{locale-expr} [ ``\textbf{+}'' \textit{element}$^+$ ] \\

  \textit{in-target} & ::=
  & ``\textbf{(}'' \textbf{in} \textit{qualified-name} ``\textbf{)}'' \\

  \textit{theorem} & ::= & ( \textbf{theorem} $|$ \textbf{lemma} $|$
    \textbf{corollary} ) [ \textit{in-target} ] [ \textit{attr-name} ] \\

  \textit{theory-level} & ::= & \ldots \\
  & | & \textbf{locale} \textit{name} [ ``\textbf{=}''
    \textit{locale} ] \\
  % note: legacy "locale (open)" omitted.
  & | & ( \textbf{theorems} $|$ \textbf{lemmas} ) \\
  & & [ \textit{in-target} ] [ \textit{attr-name} ``\textbf{=}'' ]
    ( \textit{qualified-name} [ \textit{attribute} ] )$^+$ \\
  & | & \textbf{declare} [ \textit{in-target} ] ( \textit{qualified-name}
    [ \textit{attribute} ] )$^+$ \\
  & | & \textit{theorem} \textit{proposition} \textit{proof} \\
  & | & \textit{theorem} \textit{element1}$^*$
    \textbf{shows} \textit{proposition} \textit{proof} \\
  & | & \textbf{print\_locale} \textit{locale} \\
  & | & \textbf{print\_locales}
  \end{tabular}
  \end{small}
  \vspace{2ex}
  \hrule
  \caption{Locales extend the grammar of Isar.}
  \label{fig-grammar}
  \end{figure}

  At the theory level --- that is, at the outer syntactic level of an
  Isabelle input file --- \textbf{locale} declares a named
  locale.  Other kinds of locales,
  locale expressions and unnamed locales, will be introduced later.  When
  declaring a named locale, it is possible to \emph{import} another
  named locale, or indeed several ones by importing a locale
  expression.  The second part of the declaration, also optional,
  consists of a number of context element declarations.

  A number of Isar commands have an additional, optional \emph{target}
  argument, which always refers to a named locale.  These commands
  are \textbf{theorem} (together with \textbf{lemma} and
  \textbf{corollary}),  \textbf{theorems} (and
  \textbf{lemmas}), and \textbf{declare}.  The effect of specifying a target is
  that these commands focus on the specified locale, not the
  surrounding theory.  Commands that are used to
  prove new theorems will add them not to the theory, but to the
  locale.  Similarly, \textbf{declare} modifies attributes of theorems
  that belong to the specified target.  Additionally, for
  \textbf{theorem} (and related commands), theorems stored in the target
  can be used in the associated proof scripts.

  The Locales package provides a \emph{long goals format} for
  propositions stated with \textbf{theorem} (and friends).  While
  normally a goal is just a formula, a long goal is a list of context
  elements, followed by the keyword \textbf{shows}, followed by the
  formula.  Roughly speaking, the context elements are
  (additional) premises.  For an example, see
  Section~\ref{sec-includes}.  The list of context elements in a long goal
  is also called \emph{unnamed locale}.

  Finally, there are two commands to inspect locales when working in
  interactive mode: \textbf{print\_locales} prints the names of all
  targets
  visible in the current theory, \textbf{print\_locale} outputs the
  elements of a named locale or locale expression.

  The following presentation will use notation of
  Isabelle's meta logic, hence a few sentences to explain this.
  The logical
  primitives are universal quantification (\isa{{\isasymAnd}}), entailment
  (\isa{{\isasymLongrightarrow}}) and equality (\isa{{\isasymequiv}}).  Variables (not bound
  variables) are sometimes preceded by a question mark.  The logic is
  typed.  Type variables are denoted by \isa{{\isacharprime}a}, \isa{{\isacharprime}b}
  etc., and \isa{{\isasymRightarrow}} is the function type.  Double brackets \isa{{\isasymlbrakk}} and \isa{{\isasymrbrakk}} are used to abbreviate nested entailment.%
\end{isamarkuptext}%
\isamarkuptrue%
%
\isamarkupsubsection{Parameters, Assumptions and Facts%
}
\isamarkuptrue%
%
\begin{isamarkuptext}%
From a logical point of view a \emph{context} is a formula schema of
  the form
\[
  \isa{{\isasymAnd}x\isactrlsub {\isadigit{1}}{\isasymdots}x\isactrlsub n{\isachardot}\ {\isasymlbrakk}\ C\isactrlsub {\isadigit{1}}{\isacharsemicolon}\ {\isasymdots}\ {\isacharsemicolon}C\isactrlsub m\ {\isasymrbrakk}\ {\isasymLongrightarrow}\ {\isasymdots}}
\]
  The variables $\isa{x\isactrlsub {\isadigit{1}}}, \ldots, \isa{x\isactrlsub n}$ are
  called \emph{parameters}, the premises $\isa{C\isactrlsub {\isadigit{1}}}, \ldots,
  \isa{C\isactrlsub n}$ \emph{assumptions}.  A formula \isa{F}
  holds in this context if
\begin{equation}
\label{eq-fact-in-context}
  \isa{{\isasymAnd}x\isactrlsub {\isadigit{1}}{\isasymdots}x\isactrlsub n{\isachardot}\ {\isasymlbrakk}\ C\isactrlsub {\isadigit{1}}{\isacharsemicolon}\ {\isasymdots}\ {\isacharsemicolon}C\isactrlsub m\ {\isasymrbrakk}\ {\isasymLongrightarrow}\ F}
\end{equation}
  is valid.  The formula is called a \emph{fact} of the context.

  A locale allows fixing the parameters \isa{x\isactrlsub {\isadigit{1}}{\isacharcomma}\ {\isasymdots}{\isacharcomma}\ x\isactrlsub n} and making the assumptions \isa{C\isactrlsub {\isadigit{1}}{\isacharcomma}\ {\isasymdots}{\isacharcomma}\ C\isactrlsub m}.  This implicitly builds the context in
  which the formula \isa{F} can be established.
  Parameters of a locale correspond to the context element
  \textbf{fixes}, and assumptions may be declared with
  \textbf{assumes}.  Using these context elements one can define
  the specification of semigroups.%
\end{isamarkuptext}%
\isamarkuptrue%
\isacommand{locale}\isamarkupfalse%
\ semi\ {\isacharequal}\isanewline
\ \ \isakeyword{fixes}\ prod\ {\isacharcolon}{\isacharcolon}\ {\isachardoublequoteopen}{\isacharbrackleft}{\isacharprime}a{\isacharcomma}\ {\isacharprime}a{\isacharbrackright}\ {\isasymRightarrow}\ {\isacharprime}a{\isachardoublequoteclose}\ {\isacharparenleft}\isakeyword{infixl}\ {\isachardoublequoteopen}{\isasymcdot}{\isachardoublequoteclose}\ {\isadigit{7}}{\isadigit{0}}{\isacharparenright}\isanewline
\ \ \isakeyword{assumes}\ assoc{\isacharcolon}\ {\isachardoublequoteopen}{\isacharparenleft}x\ {\isasymcdot}\ y{\isacharparenright}\ {\isasymcdot}\ z\ {\isacharequal}\ x\ {\isasymcdot}\ {\isacharparenleft}y\ {\isasymcdot}\ z{\isacharparenright}{\isachardoublequoteclose}%
\begin{isamarkuptext}%
The parameter \isa{prod} has a
  syntax annotation enabling the infix ``\isa{{\isasymcdot}}'' in the
  assumption of associativity.  Parameters may have arbitrary mixfix
  syntax, like constants.  In the example, the type of \isa{prod} is
  specified explicitly.  This is not necessary.  If no type is
  specified, a most general type is inferred simultaneously for all
  parameters, taking into account all assumptions (and type
  specifications of parameters, if present).%
\footnote{Type inference also takes into account type constraints,
  definitions and import, as introduced later.}

  Free variables in assumptions are implicitly universally quantified,
  unless they are parameters.  Hence the context defined by the locale
  \isa{semi} is
\[
  \isa{{\isasymAnd}prod{\isachardot}\ {\isasymlbrakk}\ {\isasymAnd}x\ y\ z{\isachardot}\ prod\ {\isacharparenleft}prod\ x\ y{\isacharparenright}\ z\ {\isacharequal}\ prod\ x\ {\isacharparenleft}prod\ y\ z{\isacharparenright}\ {\isasymrbrakk}\ {\isasymLongrightarrow}\ {\isasymdots}}
\]
  The locale can be extended to commutative semigroups.%
\end{isamarkuptext}%
\isamarkuptrue%
\isacommand{locale}\isamarkupfalse%
\ comm{\isacharunderscore}semi\ {\isacharequal}\ semi\ {\isacharplus}\isanewline
\ \ \isakeyword{assumes}\ comm{\isacharcolon}\ {\isachardoublequoteopen}x\ {\isasymcdot}\ y\ {\isacharequal}\ y\ {\isasymcdot}\ x{\isachardoublequoteclose}%
\begin{isamarkuptext}%
This locale \emph{imports} all elements of \isa{semi}.  The
  latter locale is called the import of \isa{comm{\isacharunderscore}semi}. The
  definition adds commutativity, hence its context is
\begin{align*%
}
  \isa{{\isasymAnd}prod{\isachardot}\ {\isasymlbrakk}} & 
  \isa{{\isasymAnd}x\ y\ z{\isachardot}\ prod\ {\isacharparenleft}prod\ x\ y{\isacharparenright}\ z\ {\isacharequal}\ prod\ x\ {\isacharparenleft}prod\ y\ z{\isacharparenright}{\isacharsemicolon}} \\
  & \isa{{\isasymAnd}x\ y{\isachardot}\ prod\ x\ y\ {\isacharequal}\ prod\ y\ x\ {\isasymrbrakk}\ {\isasymLongrightarrow}\ {\isasymdots}}
\end{align*%
}
  One may now derive facts --- for example, left-commutativity --- in
  the context of \isa{comm{\isacharunderscore}semi} by specifying this locale as
  target, and by referring to the names of the assumptions \isa{assoc} and \isa{comm} in the proof.%
\end{isamarkuptext}%
\isamarkuptrue%
\isacommand{theorem}\isamarkupfalse%
\ {\isacharparenleft}\isakeyword{in}\ comm{\isacharunderscore}semi{\isacharparenright}\ lcomm{\isacharcolon}\isanewline
\ \ {\isachardoublequoteopen}x\ {\isasymcdot}\ {\isacharparenleft}y\ {\isasymcdot}\ z{\isacharparenright}\ {\isacharequal}\ y\ {\isasymcdot}\ {\isacharparenleft}x\ {\isasymcdot}\ z{\isacharparenright}{\isachardoublequoteclose}\isanewline
%
\isadelimproof
%
\endisadelimproof
%
\isatagproof
\isacommand{proof}\isamarkupfalse%
\ {\isacharminus}\isanewline
\ \ \isacommand{have}\isamarkupfalse%
\ {\isachardoublequoteopen}x\ {\isasymcdot}\ {\isacharparenleft}y\ {\isasymcdot}\ z{\isacharparenright}\ {\isacharequal}\ {\isacharparenleft}x\ {\isasymcdot}\ y{\isacharparenright}\ {\isasymcdot}\ z{\isachardoublequoteclose}\ \isacommand{by}\isamarkupfalse%
\ {\isacharparenleft}simp\ add{\isacharcolon}\ assoc{\isacharparenright}\isanewline
\ \ \isacommand{also}\isamarkupfalse%
\ \isacommand{have}\isamarkupfalse%
\ {\isachardoublequoteopen}{\isasymdots}\ {\isacharequal}\ {\isacharparenleft}y\ {\isasymcdot}\ x{\isacharparenright}\ {\isasymcdot}\ z{\isachardoublequoteclose}\ \isacommand{by}\isamarkupfalse%
\ {\isacharparenleft}simp\ add{\isacharcolon}\ comm{\isacharparenright}\isanewline
\ \ \isacommand{also}\isamarkupfalse%
\ \isacommand{have}\isamarkupfalse%
\ {\isachardoublequoteopen}{\isasymdots}\ {\isacharequal}\ y\ {\isasymcdot}\ {\isacharparenleft}x\ {\isasymcdot}\ z{\isacharparenright}{\isachardoublequoteclose}\ \isacommand{by}\isamarkupfalse%
\ {\isacharparenleft}simp\ add{\isacharcolon}\ assoc{\isacharparenright}\isanewline
\ \ \isacommand{finally}\isamarkupfalse%
\ \isacommand{show}\isamarkupfalse%
\ {\isacharquery}thesis\ \isacommand{{\isachardot}}\isamarkupfalse%
\isanewline
\isacommand{qed}\isamarkupfalse%
%
\endisatagproof
{\isafoldproof}%
%
\isadelimproof
%
\endisadelimproof
%
\begin{isamarkuptext}%
In this equational Isar proof, ``\isa{{\isasymdots}}'' refers to the
  right hand side of the preceding equation.
  After the proof is finished, the fact \isa{lcomm} is added to
  the locale \isa{comm{\isacharunderscore}semi}.  This is done by adding a
  \textbf{notes} element to the internal representation of the locale,
  as explained the next section.%
\end{isamarkuptext}%
\isamarkuptrue%
%
\isamarkupsubsection{Locale Predicates and the Internal Representation of
  Locales%
}
\isamarkuptrue%
%
\begin{isamarkuptext}%
\label{sec-locale-predicates}
  In mathematical texts, often arbitrary but fixed objects with
  certain properties are considered --- for instance, an arbitrary but
  fixed group $G$ --- with the purpose of establishing facts valid for
  any group.  These facts are subsequently used on other objects that
  also have these properties.

  Locales permit the same style of reasoning.  Exporting a fact $F$
  generalises the fixed parameters and leads to a (valid) formula of the
  form of equation~(\ref{eq-fact-in-context}).  If a locale has many
  assumptions
  (possibly accumulated through a number of imports) this formula can
  become large and cumbersome.  Therefore, Wenzel introduced 
  predicates that abbreviate the assumptions of locales.  These
  predicates are not confined to the locale but are visible in the
  surrounding theory.

  The definition of the locale \isa{semi} generates the \emph{locale
  predicate} \isa{semi} over the type of the parameter \isa{prod},
  hence the predicate's type is \isa{{\isacharparenleft}{\isacharbrackleft}{\isacharprime}a{\isacharcomma}\ {\isacharprime}a{\isacharbrackright}\ {\isasymRightarrow}\ {\isacharprime}a{\isacharparenright}\ {\isasymRightarrow}\ bool}.  Its
  definition is
\begin{equation}
  \tag*{\isa{semi{\isacharunderscore}def}:} \isa{semi\ {\isacharquery}prod\ {\isasymequiv}\ {\isasymforall}x\ y\ z{\isachardot}\ {\isacharquery}prod\ {\isacharparenleft}{\isacharquery}prod\ x\ y{\isacharparenright}\ z\ {\isacharequal}\ {\isacharquery}prod\ x\ {\isacharparenleft}{\isacharquery}prod\ y\ z{\isacharparenright}}.
\end{equation}
  In the case where the locale has no import, the generated
  predicate abbreviates all assumptions and is over the parameters
  that occur in these assumptions.

  The situation is more complicated when a locale extends
  another locale, as is the case for \isa{comm{\isacharunderscore}semi}.  Two
  predicates are defined.  The predicate
  \isa{comm{\isacharunderscore}semi{\isacharunderscore}axioms} corresponds to the new assumptions and is
  called \emph{delta predicate}, the locale
  predicate \isa{comm{\isacharunderscore}semi} captures the content of all the locale,
  including the import.
  If a locale has neither assumptions nor import, no predicate is
  defined.  If a locale has import but no assumptions, only the locale
  predicate is defined.%
\end{isamarkuptext}%
\isamarkuptrue%
%
\isadelimML
%
\endisadelimML
%
\isatagML
%
\endisatagML
{\isafoldML}%
%
\isadelimML
%
\endisadelimML
%
\begin{isamarkuptext}%
The Locales package generates a number of theorems for locale and
  delta predicates.  All predicates have a definition and an
  introduction rule.  Locale predicates that are defined in terms of
  other predicates (which is the case if and only if the locale has
  import) also have a number of elimination rules (called
  \emph{axioms}).  All generated theorems for the predicates of the
  locales \isa{semi} and \isa{comm{\isacharunderscore}semi} are shown in
  Figures~\ref{fig-theorems-semi} and~\ref{fig-theorems-comm-semi},
  respectively.
  \begin{figure}
  \hrule
  \vspace{2ex}
  Theorems generated for the predicate \isa{semi}.
  \begin{gather}
    \tag*{\isa{semi{\isacharunderscore}def}:} \isa{semi\ {\isacharquery}prod\ {\isasymequiv}\ {\isasymforall}x\ y\ z{\isachardot}\ {\isacharquery}prod\ {\isacharparenleft}{\isacharquery}prod\ x\ y{\isacharparenright}\ z\ {\isacharequal}\ {\isacharquery}prod\ x\ {\isacharparenleft}{\isacharquery}prod\ y\ z{\isacharparenright}} \\
    \tag*{\isa{semi{\isachardot}intro}:} \isa{{\isacharparenleft}{\isasymAnd}x\ y\ z{\isachardot}\ {\isacharquery}prod\ {\isacharparenleft}{\isacharquery}prod\ x\ y{\isacharparenright}\ z\ {\isacharequal}\ {\isacharquery}prod\ x\ {\isacharparenleft}{\isacharquery}prod\ y\ z{\isacharparenright}{\isacharparenright}\ {\isasymLongrightarrow}\ semi\ {\isacharquery}prod}
  \end{gather}
  \hrule
  \caption{Theorems for the locale predicate \isa{semi}.}
  \label{fig-theorems-semi}
  \end{figure}

  \begin{figure}[h]
  \hrule
  \vspace{2ex}
  Theorems generated for the predicate \isa{comm{\isacharunderscore}semi{\isacharunderscore}axioms}.
  \begin{gather}
    \tag*{\isa{comm{\isacharunderscore}semi{\isacharunderscore}axioms{\isacharunderscore}def}:} \isa{comm{\isacharunderscore}semi{\isacharunderscore}axioms\ {\isacharquery}prod\ {\isasymequiv}\ {\isasymforall}x\ y{\isachardot}\ {\isacharquery}prod\ x\ y\ {\isacharequal}\ {\isacharquery}prod\ y\ x} \\                        
    \tag*{\isa{comm{\isacharunderscore}semi{\isacharunderscore}axioms{\isachardot}intro}:} \isa{{\isacharparenleft}{\isasymAnd}x\ y{\isachardot}\ {\isacharquery}prod\ x\ y\ {\isacharequal}\ {\isacharquery}prod\ y\ x{\isacharparenright}\ {\isasymLongrightarrow}\ comm{\isacharunderscore}semi{\isacharunderscore}axioms\ {\isacharquery}prod}                       
  \end{gather}
  Theorems generated for the predicate \isa{comm{\isacharunderscore}semi}.
  \begin{gather}
    \tag*{\isa{comm{\isacharunderscore}semi{\isacharunderscore}def}:} \isa{comm{\isacharunderscore}semi\ {\isacharquery}prod\ {\isasymequiv}\ semi\ {\isacharquery}prod\ {\isasymand}\ comm{\isacharunderscore}semi{\isacharunderscore}axioms\ {\isacharquery}prod} \\                          
    \tag*{\isa{comm{\isacharunderscore}semi{\isachardot}intro}:} \isa{{\isasymlbrakk}semi\ {\isacharquery}prod{\isacharsemicolon}\ comm{\isacharunderscore}semi{\isacharunderscore}axioms\ {\isacharquery}prod{\isasymrbrakk}\ {\isasymLongrightarrow}\ comm{\isacharunderscore}semi\ {\isacharquery}prod} \\
    \tag*{\isa{comm{\isacharunderscore}semi{\isachardot}axioms}:} \mbox{} \\
    \notag \isa{comm{\isacharunderscore}semi\ {\isacharquery}prod\ {\isasymLongrightarrow}\ semi\ {\isacharquery}prod} \\
    \notag \isa{comm{\isacharunderscore}semi\ {\isacharquery}prod\ {\isasymLongrightarrow}\ comm{\isacharunderscore}semi{\isacharunderscore}axioms\ {\isacharquery}prod}               
  \end{gather} 
  \hrule
  \caption{Theorems for the predicates \isa{comm{\isacharunderscore}semi{\isacharunderscore}axioms} and
    \isa{comm{\isacharunderscore}semi}.}
  \label{fig-theorems-comm-semi}
  \end{figure}
  Note that the theorems generated by a locale
  definition may be inspected immediately after the definition in the
  Proof General interface \cite{Aspinall2000} of Isabelle through
  the menu item ``Isabelle/Isar$>$Show me $\ldots>$Theorems''.

  Locale and delta predicates are used also in the internal
  representation of locales as lists of context elements.  While all
  \textbf{fixes} in a
  declaration generate internal \textbf{fixes}, all assumptions
  of one locale declaration contribute to one internal \textbf{assumes}
  element.  The internal representation of \isa{semi} is
\[
\begin{array}{ll}
  \textbf{fixes} & \isa{prod} :: \isa{{\isachardoublequote}{\isacharbrackleft}{\isacharprime}a{\isacharcomma}\ {\isacharprime}a{\isacharbrackright}\ {\isasymRightarrow}\ {\isacharprime}a{\isachardoublequote}}
    (\textbf{infixl} \isa{{\isachardoublequote}{\isasymcdot}{\isachardoublequote}} 70) \\
  \textbf{assumes} & \isa{{\isachardoublequote}semi\ prod{\isachardoublequote}} \\
  \textbf{notes} & \isa{assoc}: \isa{{\isachardoublequote}{\isacharquery}x\ {\isasymcdot}\ {\isacharquery}y\ {\isasymcdot}\ {\isacharquery}z\ {\isacharequal}\ {\isacharquery}x\ {\isasymcdot}\ {\isacharparenleft}{\isacharquery}y\ {\isasymcdot}\ {\isacharquery}z{\isacharparenright}{\isachardoublequote}}
\end{array}
\]
  and the internal representation of \isa{{\isachardoublequote}comm{\isacharunderscore}semi{\isachardoublequote}} is
\begin{equation}
\label{eq-comm-semi}
\begin{array}{ll}
  \textbf{fixes} & \isa{prod} :: \isa{{\isachardoublequote}{\isacharbrackleft}{\isacharprime}a{\isacharcomma}\ {\isacharprime}a{\isacharbrackright}\ {\isasymRightarrow}\ {\isacharprime}a{\isachardoublequote}}
    ~(\textbf{infixl}~\isa{{\isachardoublequote}{\isasymcdot}{\isachardoublequote}}~70) \\
  \textbf{assumes} & \isa{{\isachardoublequote}semi\ prod{\isachardoublequote}} \\
  \textbf{notes} & \isa{assoc}: \isa{{\isachardoublequote}{\isacharquery}x\ {\isasymcdot}\ {\isacharquery}y\ {\isasymcdot}\ {\isacharquery}z\ {\isacharequal}\ {\isacharquery}x\ {\isasymcdot}\ {\isacharparenleft}{\isacharquery}y\ {\isasymcdot}\ {\isacharquery}z{\isacharparenright}{\isachardoublequote}} \\
  \textbf{assumes} & \isa{{\isachardoublequote}comm{\isacharunderscore}semi{\isacharunderscore}axioms\ prod{\isachardoublequote}} \\
  \textbf{notes} & \isa{comm}: \isa{{\isachardoublequote}{\isacharquery}x\ {\isasymcdot}\ {\isacharquery}y\ {\isacharequal}\ {\isacharquery}y\ {\isasymcdot}\ {\isacharquery}x{\isachardoublequote}} \\
  \textbf{notes} & \isa{lcomm}: \isa{{\isachardoublequote}{\isacharquery}x\ {\isasymcdot}\ {\isacharparenleft}{\isacharquery}y\ {\isasymcdot}\ {\isacharquery}z{\isacharparenright}\ {\isacharequal}\ {\isacharquery}y\ {\isasymcdot}\ {\isacharparenleft}{\isacharquery}x\ {\isasymcdot}\ {\isacharquery}z{\isacharparenright}{\isachardoublequote}}
\end{array}
\end{equation}
  The \textbf{notes} elements store facts of the
  locales.  The facts \isa{assoc} and \isa{comm} were added
  during the declaration of the locales.  They stem from assumptions,
  which are trivially facts.  The fact \isa{lcomm} was
  added later, after finishing the proof in the respective
  \textbf{theorem} command above.

  By using \textbf{notes} in a declaration, facts can be added
  to a locale directly.  Of course, these must be theorems.
  Typical use of this feature includes adding theorems that are not
  usually used as a default rewrite rules by the simplifier to the
  simpset of the locale by a \textbf{notes} element with the attribute
  \isa{{\isacharbrackleft}simp{\isacharbrackright}}.  This way it is also possible to add specialised
  versions of
  theorems to a locale by instantiating locale parameters for unknowns
  or locale assumptions for premises.%
\end{isamarkuptext}%
\isamarkuptrue%
%
\isamarkupsubsection{Definitions%
}
\isamarkuptrue%
%
\begin{isamarkuptext}%
Definitions were available in Kamm\"uller's version of Locales, and
  they are in Wenzel's.  
  The context element \textbf{defines} adds a definition of the form
  \isa{p\ x\isactrlsub {\isadigit{1}}\ {\isasymdots}\ x\isactrlsub n\ {\isasymequiv}\ t} as an assumption, where \isa{p} is a
  parameter of the locale (possibly an imported parameter), and \isa{t} a term that may contain the \isa{x\isactrlsub i}.  The parameter may
  neither occur in a previous \textbf{assumes} or \textbf{defines}
  element, nor on the right hand side of the definition.  Hence
  recursion is not allowed.
  The parameter may, however, occur in subsequent \textbf{assumes} and
  on the right hand side of subsequent \textbf{defines}.  We call
  \isa{p} \emph{defined parameter}.%
\end{isamarkuptext}%
\isamarkuptrue%
\isacommand{locale}\isamarkupfalse%
\ semi{\isadigit{2}}\ {\isacharequal}\ semi\ {\isacharplus}\isanewline
\ \ \isakeyword{fixes}\ rprod\ {\isacharparenleft}\isakeyword{infixl}\ {\isachardoublequoteopen}{\isasymodot}{\isachardoublequoteclose}\ {\isadigit{7}}{\isadigit{0}}{\isacharparenright}\isanewline
\ \ \isakeyword{defines}\ rprod{\isacharunderscore}def{\isacharcolon}\ {\isachardoublequoteopen}rprod\ x\ y\ {\isasymequiv}\ y\ {\isasymcdot}\ x\ {\isachardoublequoteclose}%
\begin{isamarkuptext}%
This locale extends \isa{semi} by a second binary operation \isa{{\isachardoublequote}{\isasymodot}{\isachardoublequote}} that is like \isa{{\isachardoublequote}{\isasymcdot}{\isachardoublequote}} but with
  reversed arguments.  The
  definition of the locale generates the predicate \isa{semi{\isadigit{2}}},
  which is equivalent to \isa{semi}, but no \isa{semi{\isadigit{2}}{\isacharunderscore}axioms}.
  The difference between \textbf{assumes} and \textbf{defines} lies in
  the way parameters are treated on export.%
\end{isamarkuptext}%
\isamarkuptrue%
%
\isamarkupsubsection{Export%
}
\isamarkuptrue%
%
\begin{isamarkuptext}%
A fact is exported out
  of a locale by generalising over the parameters and adding
  assumptions as premises.  For brevity of the exported theorems,
  locale predicates are used.  Exported facts are referenced by
  writing qualified names consisting of the locale name and the fact name ---
  for example,
\begin{equation}
  \tag*{\isa{semi{\isachardot}assoc}:} \isa{semi\ {\isacharquery}prod\ {\isasymLongrightarrow}\ {\isacharquery}prod\ {\isacharparenleft}{\isacharquery}prod\ {\isacharquery}x\ {\isacharquery}y{\isacharparenright}\ {\isacharquery}z\ {\isacharequal}\ {\isacharquery}prod\ {\isacharquery}x\ {\isacharparenleft}{\isacharquery}prod\ {\isacharquery}y\ {\isacharquery}z{\isacharparenright}}.
\end{equation}
  Defined parameters receive special treatment.  Instead of adding a
  premise for the definition, the definition is unfolded in the
  exported theorem.  In order to illustrate this we prove that the
  reverse operation \isa{{\isachardoublequote}{\isasymodot}{\isachardoublequote}} defined in the locale
  \isa{semi{\isadigit{2}}} is also associative.%
\end{isamarkuptext}%
\isamarkuptrue%
\isacommand{theorem}\isamarkupfalse%
\ {\isacharparenleft}\isakeyword{in}\ semi{\isadigit{2}}{\isacharparenright}\ r{\isacharunderscore}assoc{\isacharcolon}\ {\isachardoublequoteopen}{\isacharparenleft}x\ {\isasymodot}\ y{\isacharparenright}\ {\isasymodot}\ z\ {\isacharequal}\ x\ {\isasymodot}\ {\isacharparenleft}y\ {\isasymodot}\ z{\isacharparenright}{\isachardoublequoteclose}\isanewline
%
\isadelimproof
\ \ %
\endisadelimproof
%
\isatagproof
\isacommand{by}\isamarkupfalse%
\ {\isacharparenleft}simp\ only{\isacharcolon}\ rprod{\isacharunderscore}def\ assoc{\isacharparenright}%
\endisatagproof
{\isafoldproof}%
%
\isadelimproof
%
\endisadelimproof
%
\begin{isamarkuptext}%
The exported fact is
\begin{equation}
  \tag*{\isa{semi{\isadigit{2}}{\isachardot}r{\isacharunderscore}assoc}:} \isa{semi{\isadigit{2}}\ {\isacharquery}prod\ {\isasymLongrightarrow}\ {\isacharquery}prod\ {\isacharquery}z\ {\isacharparenleft}{\isacharquery}prod\ {\isacharquery}y\ {\isacharquery}x{\isacharparenright}\ {\isacharequal}\ {\isacharquery}prod\ {\isacharparenleft}{\isacharquery}prod\ {\isacharquery}z\ {\isacharquery}y{\isacharparenright}\ {\isacharquery}x}.
\end{equation}
  The defined parameter is not present but is replaced by its
  definition.  Note that the definition itself is not exported, hence
  there is no \isa{semi{\isadigit{2}}{\isachardot}rprod{\isacharunderscore}def}.%
\footnote{The definition could alternatively be exported using a
  let-construct if there was one in Isabelle's meta-logic.  Let is
  usually defined in object-logics.}%
\end{isamarkuptext}%
\isamarkuptrue%
%
\isamarkupsection{Locale Expressions%
}
\isamarkuptrue%
%
\begin{isamarkuptext}%
Locale expressions provide a simple language for combining
  locales.  They are an effective means of building complex
  specifications from simple ones.  Locale expressions are the main
  innovation of the version of Locales discussed here.  Locale
  expressions are also reason for introducing locale predicates.%
\end{isamarkuptext}%
\isamarkuptrue%
%
\isamarkupsubsection{Rename and Merge%
}
\isamarkuptrue%
%
\begin{isamarkuptext}%
The grammar of locale expressions is part of the grammar in
  Figure~\ref{fig-grammar}.  Locale names are locale
  expressions, and
  further expressions are obtained by \emph{rename} and \emph{merge}.
\begin{description}
\item[Rename.]
  The locale expression $e\: q_1 \ldots q_n$ denotes
  the locale of $e$ where pa\-ra\-me\-ters, in the order in
  which they are fixed, are renamed to $q_1$ to $q_n$.
  The expression is only well-formed if $n$ does not
  exceed the number of parameters of $e$.  Underscores denote
  parameters that are not renamed.
  Renaming by default removes mixfix syntax, but new syntax may be
  specified.
\item[Merge.]
  The locale expression $e_1 + e_2$ denotes
  the locale obtained by merging the locales of $e_1$
  and $e_2$.  This locale contains the context elements
  of $e_1$, followed by the context elements of $e_2$.

  In actual fact, the semantics of the merge operation
  is more complicated.  If $e_1$ and $e_2$ are expressions containing
  the same name, followed by
  identical parameter lists, then the merge of both will contain
  the elements of those locales only once.  Details are explained in
  Section~\ref{sec-normal-forms} below.

  The merge operation is associative but not commutative.  The latter
  is because parameters of $e_1$ appear
  before parameters of $e_2$ in the composite expression.
\end{description}

  Rename can be used if a different parameter name seems more
  appropriate --- for example, when moving from groups to rings, a
  parameter \isa{G} representing the group could be changed to
  \isa{R}.  Besides of this stylistic use, renaming is important in
  combination with merge.  Both operations are used in the following
  specification of semigroup homomorphisms.%
\end{isamarkuptext}%
\isamarkuptrue%
\isacommand{locale}\isamarkupfalse%
\ semi{\isacharunderscore}hom\ {\isacharequal}\ comm{\isacharunderscore}semi\ sum\ {\isacharparenleft}\isakeyword{infixl}\ {\isachardoublequoteopen}{\isasymoplus}{\isachardoublequoteclose}\ {\isadigit{6}}{\isadigit{5}}{\isacharparenright}\ {\isacharplus}\ comm{\isacharunderscore}semi\ {\isacharplus}\isanewline
\ \ \isakeyword{fixes}\ hom\isanewline
\ \ \isakeyword{assumes}\ hom{\isacharcolon}\ {\isachardoublequoteopen}hom\ {\isacharparenleft}x\ {\isasymoplus}\ y{\isacharparenright}\ {\isacharequal}\ hom\ x\ {\isasymcdot}\ hom\ y{\isachardoublequoteclose}%
\begin{isamarkuptext}%
This locale defines a context with three parameters \isa{sum},
  \isa{prod} and \isa{hom}.  The first two parameters have
  mixfix syntax.  They are associative operations,
  the first of type \isa{{\isacharbrackleft}{\isacharprime}a{\isacharcomma}\ {\isacharprime}a{\isacharbrackright}\ {\isasymRightarrow}\ {\isacharprime}a}, the second of
  type \isa{{\isacharbrackleft}{\isacharprime}b{\isacharcomma}\ {\isacharprime}b{\isacharbrackright}\ {\isasymRightarrow}\ {\isacharprime}b}.  

  How are facts that are imported via a locale expression identified?
  Facts are always introduced in a named locale (either in the
  locale's declaration, or by using the locale as target in
  \textbf{theorem}), and their names are
  qualified by the parameter names of this locale.
  Hence the full name of associativity in \isa{semi} is
  \isa{prod{\isachardot}assoc}.  Renaming parameters of a target also renames
  the qualifier of facts.  Hence, associativity of \isa{sum} is
  \isa{sum{\isachardot}assoc}.  Several parameters are separated by
  underscores in qualifiers.  For example, the full name of the fact
  \isa{hom} in the locale \isa{semi{\isacharunderscore}hom} is \isa{sum{\isacharunderscore}prod{\isacharunderscore}hom{\isachardot}hom}.

  The following example is quite artificial, it illustrates the use of
  facts, though.%
\end{isamarkuptext}%
\isamarkuptrue%
\isacommand{theorem}\isamarkupfalse%
\ {\isacharparenleft}\isakeyword{in}\ semi{\isacharunderscore}hom{\isacharparenright}\ {\isachardoublequoteopen}hom\ x\ {\isasymcdot}\ {\isacharparenleft}hom\ y\ {\isasymcdot}\ hom\ z{\isacharparenright}\ {\isacharequal}\ hom\ {\isacharparenleft}x\ {\isasymoplus}\ {\isacharparenleft}y\ {\isasymoplus}\ z{\isacharparenright}{\isacharparenright}{\isachardoublequoteclose}\isanewline
%
\isadelimproof
%
\endisadelimproof
%
\isatagproof
\isacommand{proof}\isamarkupfalse%
\ {\isacharminus}\isanewline
\ \ \isacommand{have}\isamarkupfalse%
\ {\isachardoublequoteopen}hom\ x\ {\isasymcdot}\ {\isacharparenleft}hom\ y\ {\isasymcdot}\ hom\ z{\isacharparenright}\ {\isacharequal}\ hom\ y\ {\isasymcdot}\ {\isacharparenleft}hom\ x\ {\isasymcdot}\ hom\ z{\isacharparenright}{\isachardoublequoteclose}\isanewline
\ \ \ \ \isacommand{by}\isamarkupfalse%
\ {\isacharparenleft}simp\ add{\isacharcolon}\ prod{\isachardot}lcomm{\isacharparenright}\isanewline
\ \ \isacommand{also}\isamarkupfalse%
\ \isacommand{have}\isamarkupfalse%
\ {\isachardoublequoteopen}{\isasymdots}\ {\isacharequal}\ hom\ {\isacharparenleft}y\ {\isasymoplus}\ {\isacharparenleft}x\ {\isasymoplus}\ z{\isacharparenright}{\isacharparenright}{\isachardoublequoteclose}\ \isacommand{by}\isamarkupfalse%
\ {\isacharparenleft}simp\ add{\isacharcolon}\ hom{\isacharparenright}\isanewline
\ \ \isacommand{also}\isamarkupfalse%
\ \isacommand{have}\isamarkupfalse%
\ {\isachardoublequoteopen}{\isasymdots}\ {\isacharequal}\ hom\ {\isacharparenleft}x\ {\isasymoplus}\ {\isacharparenleft}y\ {\isasymoplus}\ z{\isacharparenright}{\isacharparenright}{\isachardoublequoteclose}\ \isacommand{by}\isamarkupfalse%
\ {\isacharparenleft}simp\ add{\isacharcolon}\ sum{\isachardot}lcomm{\isacharparenright}\isanewline
\ \ \isacommand{finally}\isamarkupfalse%
\ \isacommand{show}\isamarkupfalse%
\ {\isacharquery}thesis\ \isacommand{{\isachardot}}\isamarkupfalse%
\isanewline
\isacommand{qed}\isamarkupfalse%
%
\endisatagproof
{\isafoldproof}%
%
\isadelimproof
%
\endisadelimproof
%
\begin{isamarkuptext}%
Importing via a locale expression imports all facts of
  the imported locales, hence both \isa{sum{\isachardot}lcomm} and \isa{prod{\isachardot}lcomm} are
  available in \isa{hom{\isacharunderscore}semi}.  The import is dynamic --- that is,
  whenever facts are added to a locale, they automatically
  become available in subsequent \textbf{theorem} commands that use
  the locale as a target, or a locale importing the locale.%
\end{isamarkuptext}%
\isamarkuptrue%
%
\isamarkupsubsection{Normal Forms%
}
\isamarkuptrue%
%
\label{sec-normal-forms}
\newcommand{\I}{\mathcal{I}}
\newcommand{\F}{\mathcal{F}}
\newcommand{\N}{\mathcal{N}}
\newcommand{\C}{\mathcal{C}}
\newcommand{\App}{\mathbin{\overline{@}}}
%
\begin{isamarkuptext}%
Locale expressions are interpreted in a two-step process.  First, an
  expression is normalised, then it is converted to a list of context
  elements.

  Normal forms are based on \textbf{locale} declarations.  These
  consist of an import section followed by a list of context
  elements.  Let $\I(l)$ denote the locale expression imported by
  locale $l$.  If $l$ has no import then $\I(l) = \varepsilon$.
  Likewise, let $\F(l)$ denote the list of context elements, also
  called the \emph{context fragment} of $l$.  Note that $\F(l)$
  contains only those context elements that are stated in the
  declaration of $l$, not imported ones.

\paragraph{Example 1.}  Consider the locales \isa{semi} and \isa{comm{\isacharunderscore}semi}.  We have $\I(\isa{semi}) = \varepsilon$ and
  $\I(\isa{comm{\isacharunderscore}semi}) = \isa{semi}$, and the context fragments
  are
\begin{align*%
}
  \F(\isa{semi}) & = \left[
\begin{array}{ll}
  \textbf{fixes} & \isa{prod} :: \isa{{\isachardoublequote}{\isacharbrackleft}{\isacharprime}a{\isacharcomma}\ {\isacharprime}a{\isacharbrackright}\ {\isasymRightarrow}\ {\isacharprime}a{\isachardoublequote}}
    ~(\textbf{infixl}~\isa{{\isachardoublequote}{\isasymcdot}{\isachardoublequote}}~70) \\
  \textbf{assumes} & \isa{{\isachardoublequote}semi\ prod{\isachardoublequote}} \\
  \textbf{notes} & \isa{assoc}: \isa{{\isachardoublequote}{\isacharquery}x\ {\isasymcdot}\ {\isacharquery}y\ {\isasymcdot}\ {\isacharquery}z\ {\isacharequal}\ {\isacharquery}x\ {\isasymcdot}\ {\isacharparenleft}{\isacharquery}y\ {\isasymcdot}\ {\isacharquery}z{\isacharparenright}{\isachardoublequote}}
\end{array} \right], \\
  \F(\isa{comm{\isacharunderscore}semi}) & = \left[
\begin{array}{ll}
  \textbf{assumes} & \isa{{\isachardoublequote}comm{\isacharunderscore}semi{\isacharunderscore}axioms\ prod{\isachardoublequote}} \\
  \textbf{notes} & \isa{comm}: \isa{{\isachardoublequote}{\isacharquery}x\ {\isasymcdot}\ {\isacharquery}y\ {\isacharequal}\ {\isacharquery}y\ {\isasymcdot}\ {\isacharquery}x{\isachardoublequote}} \\
  \textbf{notes} & \isa{lcomm}: \isa{{\isachardoublequote}{\isacharquery}x\ {\isasymcdot}\ {\isacharparenleft}{\isacharquery}y\ {\isasymcdot}\ {\isacharquery}z{\isacharparenright}\ {\isacharequal}\ {\isacharquery}y\ {\isasymcdot}\ {\isacharparenleft}{\isacharquery}x\ {\isasymcdot}\ {\isacharquery}z{\isacharparenright}{\isachardoublequote}}
\end{array} \right].
\end{align*%
}
  Let $\pi_0(\F(l))$ denote the list of parameters defined in the
  \textbf{fixes} elements of $\F(l)$ in the order of their occurrence.
  The list of parameters of a locale expression $\pi(e)$ is defined as
  follows:
\begin{align*%
}
  \pi(l) & = \pi(\I(l)) \App \pi_0(\F(l)) \text{, for named locale $l$.} \\
  \pi(e\: q_1 \ldots q_n) & = \text{$[q_1, \ldots, q_n, p_{n+1}, \ldots,
    p_{m}]$, where $\pi(e) = [p_1, \ldots, p_m]$.} \\
  \pi(e_1 + e_2) & = \pi(e_1) \App \pi(e_2)
\end{align*%
}
  The operation $\App$ concatenates two lists but omits elements from
  the second list that are also present in the first list.
  It is not possible to rename more parameters than there
  are present in an expression --- that is, $n \le m$ --- otherwise
  the renaming is illegal.  If $q_i
  = \_$ then the $i$th entry of the resulting list is $p_i$.

  In the normalisation phase, imports of named locales are unfolded, and
  renames and merges are recursively propagated to the imported locale
  expressions.  The result is a list of locale names, each with a full
  list of parameters, where locale names occurring with the same parameter
  list twice are removed.  Let $\N$ denote normalisation.  It is defined
  by these equations:
\begin{align*%
}
  \N(l) & = \N(\I(l)) \App [l\:\pi(l)] \text{, for named locale $l$.} \\
  \N(e\: q_1 \ldots q_n) & = \N(e)\:[q_1 \ldots q_n/\pi(e)] \\
  \N(e_1 + e_2) & = \N(e_1) \App \N(e_2)
\end{align*%
}
  Normalisation yields a list of \emph{identifiers}.  An identifier
  consists of a locale name and a (possibly empty) list of parameters.

  In the second phase, the list of identifiers $\N(e)$ is converted to
  a list of context elements $\C(e)$ by converting each identifier to
  a list of context elements, and flattening the obtained list.
  Conversion of the identifier $l\:q_1 \ldots q_n$ yields the list of
  context elements $\F(l)$, but with the following modifications:
\begin{itemize}
\item
  Rename the parameter in the $i$th \textbf{fixes} element of $\F(l)$
  to $q_i$, $i = 1, \ldots, n$.  If the parameter name is actually
  changed then delete the syntax annotation.  Renaming a parameter may
  also change its type.
\item
  Perform the same renamings on all occurrences of parameters (fixed
  variables) in \textbf{assumes}, \textbf{defines} and \textbf{notes}
  elements.
\item
  Qualify names of facts by $q_1\_\ldots\_q_n$.
\end{itemize}
  The locale expression is \emph{well-formed} if it contains no
  illegal renamings and the following conditions on $\C(e)$ hold,
  otherwise the expression is rejected:
\begin{itemize}
\item Parameters in \textbf{fixes} are distinct;
\item Free variables in \textbf{assumes} and
  \textbf{defines} occur in preceding \textbf{fixes};%
\footnote{This restriction is relaxed for contexts obtained with
  \textbf{includes}, see Section~\ref{sec-includes}.}
\item Parameters defined in \textbf{defines} must neither occur in
  preceding \textbf{assumes} nor \textbf{defines}.
\end{itemize}%
\end{isamarkuptext}%
\isamarkuptrue%
%
\isamarkupsubsection{Examples%
}
\isamarkuptrue%
%
\begin{isamarkuptext}%
\paragraph{Example 2.}
  We obtain the context fragment $\C(\isa{comm{\isacharunderscore}semi})$ of the
  locale \isa{comm{\isacharunderscore}semi}.  First, the parameters are computed.
\begin{align*%
}
  \pi(\isa{semi}) & = [\isa{prod}] \\
  \pi(\isa{comm{\isacharunderscore}semi}) & = \pi(\isa{semi}) \App []
     = [\isa{prod}]
\end{align*%
}
  Next, the normal form of the locale expression
  \isa{comm{\isacharunderscore}semi} is obtained.
\begin{align*%
}
  \N(\isa{semi}) & = [\isa{semi} \isa{prod}] \\
  \N(\isa{comm{\isacharunderscore}semi}) & = \N(\isa{semi}) \App
       [\isa{comm{\isacharunderscore}semi\ prod}]
   = [\isa{semi\ prod}, \isa{comm{\isacharunderscore}semi\ prod}]
\end{align*%
}
  Converting this to a list of context elements leads to the
  list~(\ref{eq-comm-semi}) shown in
  Section~\ref{sec-locale-predicates}, but with fact names qualified
  by \isa{prod} --- for example, \isa{prod{\isachardot}assoc}.
  Qualification was omitted to keep the presentation simple.
  Isabelle's scoping rules identify the most recent fact with
  qualified name $x.a$ when a fact with name $a$ is requested.

\paragraph{Example 3.}
  The locale expression \isa{comm{\isacharunderscore}semi\ sum} involves
  renaming.  Computing parameters yields $\pi(\isa{comm{\isacharunderscore}semi\ sum})
  = [\isa{sum}]$, the normal form is
\begin{align*%
}
  \N(\isa{comm{\isacharunderscore}semi\ sum}) & =
  \N(\isa{comm{\isacharunderscore}semi})[\isa{sum}/\isa{prod}] =
  [\isa{semi\ sum}, \isa{comm{\isacharunderscore}semi\ sum}]
\end{align*%
}
  and the list of context elements
\[
\begin{array}{ll}
  \textbf{fixes} & \isa{sum} :: \isa{{\isachardoublequote}{\isacharbrackleft}{\isacharprime}a{\isacharcomma}\ {\isacharprime}a{\isacharbrackright}\ {\isasymRightarrow}\ {\isacharprime}a{\isachardoublequote}}
    ~(\textbf{infixl}~\isa{{\isachardoublequote}{\isasymoplus}{\isachardoublequote}}~65) \\
  \textbf{assumes} & \isa{{\isachardoublequote}semi\ sum{\isachardoublequote}} \\
  \textbf{notes} & \isa{sum{\isachardot}assoc}: \isa{{\isachardoublequote}{\isacharparenleft}{\isacharquery}x\ {\isasymoplus}\ {\isacharquery}y{\isacharparenright}\ {\isasymoplus}\ {\isacharquery}z\ {\isacharequal}\ sum\ {\isacharquery}x\ {\isacharparenleft}sum\ {\isacharquery}y\ {\isacharquery}z{\isacharparenright}{\isachardoublequote}} \\
  \textbf{assumes} & \isa{{\isachardoublequote}comm{\isacharunderscore}semi{\isacharunderscore}axioms\ sum{\isachardoublequote}} \\
  \textbf{notes} & \isa{sum{\isachardot}comm}: \isa{{\isachardoublequote}{\isacharquery}x\ {\isasymoplus}\ {\isacharquery}y\ {\isacharequal}\ {\isacharquery}y\ {\isasymoplus}\ {\isacharquery}x{\isachardoublequote}} \\
  \textbf{notes} & \isa{sum{\isachardot}lcomm}: \isa{{\isachardoublequote}{\isacharquery}x\ {\isasymoplus}\ {\isacharparenleft}{\isacharquery}y\ {\isasymoplus}\ {\isacharquery}z{\isacharparenright}\ {\isacharequal}\ {\isacharquery}y\ {\isasymoplus}\ {\isacharparenleft}{\isacharquery}x\ {\isasymoplus}\ {\isacharquery}z{\isacharparenright}{\isachardoublequote}}
\end{array}
\]

\paragraph{Example 4.}
  The context defined by the locale \isa{semi{\isacharunderscore}hom} involves
  merging two copies of \isa{comm{\isacharunderscore}semi}.  We obtain the parameter
  list and normal form:
\begin{align*%
}
  \pi(\isa{semi{\isacharunderscore}hom}) & = \pi(\isa{comm{\isacharunderscore}semi\ sum} +
       \isa{comm{\isacharunderscore}semi}) \App [\isa{hom}] \\
     & = (\pi(\isa{comm{\isacharunderscore}semi\ sum}) \App \pi(\isa{comm{\isacharunderscore}semi}))
       \App [\isa{hom}] \\
     & = ([\isa{sum}] \App [\isa{prod}]) \App [\isa{hom}]
     = [\isa{sum}, \isa{prod}, \isa{hom}] \\
  \N(\isa{semi{\isacharunderscore}hom}) & =
       \N(\isa{comm{\isacharunderscore}semi\ sum} + \isa{comm{\isacharunderscore}semi}) \App \\
     & \phantom{==}
       [\isa{semi{\isacharunderscore}hom\ sum\ prod\ hom}] \\
     & = (\N(\isa{comm{\isacharunderscore}semi\ sum}) \App \N(\isa{comm{\isacharunderscore}semi})) \App \\
     & \phantom{==}
       [\isa{semi{\isacharunderscore}hom\ sum\ prod\ hom}] \\
     & = ([\isa{semi\ sum}, \isa{comm{\isacharunderscore}semi\ sum}] \App %\\
%     & \phantom{==}
       [\isa{semi\ prod}, \isa{comm{\isacharunderscore}semi\ prod}]) \App \\
     & \phantom{==}
       [\isa{semi{\isacharunderscore}hom\ sum\ prod\ hom}] \\
     & = [\isa{semi\ sum}, \isa{comm{\isacharunderscore}semi\ sum},
       \isa{semi\ prod}, \isa{comm{\isacharunderscore}semi\ prod}, \\
     & \phantom{==}
       \isa{semi{\isacharunderscore}hom\ sum\ prod\ hom}].
\end{align*%
}
  Hence $\C(\isa{semi{\isacharunderscore}hom})$, shown below, is again well-formed.
\[
\begin{array}{ll}
  \textbf{fixes} & \isa{sum} :: \isa{{\isachardoublequote}{\isacharbrackleft}{\isacharprime}a{\isacharcomma}\ {\isacharprime}a{\isacharbrackright}\ {\isasymRightarrow}\ {\isacharprime}a{\isachardoublequote}}
    ~(\textbf{infixl}~\isa{{\isachardoublequote}{\isasymoplus}{\isachardoublequote}}~65) \\
  \textbf{assumes} & \isa{{\isachardoublequote}semi\ sum{\isachardoublequote}} \\
  \textbf{notes} & \isa{sum{\isachardot}assoc}: \isa{{\isachardoublequote}{\isacharparenleft}{\isacharquery}x\ {\isasymoplus}\ {\isacharquery}y{\isacharparenright}\ {\isasymoplus}\ {\isacharquery}z\ {\isacharequal}\ {\isacharquery}x\ {\isasymoplus}\ {\isacharparenleft}{\isacharquery}y\ {\isasymoplus}\ {\isacharquery}z{\isacharparenright}{\isachardoublequote}} \\
  \textbf{assumes} & \isa{{\isachardoublequote}comm{\isacharunderscore}semi{\isacharunderscore}axioms\ sum{\isachardoublequote}} \\
  \textbf{notes} & \isa{sum{\isachardot}comm}: \isa{{\isachardoublequote}{\isacharquery}x\ {\isasymoplus}\ {\isacharquery}y\ {\isacharequal}\ {\isacharquery}y\ {\isasymoplus}\ {\isacharquery}x{\isachardoublequote}} \\
  \textbf{notes} & \isa{sum{\isachardot}lcomm}: \isa{{\isachardoublequote}{\isacharquery}x\ {\isasymoplus}\ {\isacharparenleft}{\isacharquery}y\ {\isasymoplus}\ {\isacharquery}z{\isacharparenright}\ {\isacharequal}\ {\isacharquery}y\ {\isasymoplus}\ {\isacharparenleft}{\isacharquery}x\ {\isasymoplus}\ {\isacharquery}z{\isacharparenright}{\isachardoublequote}} \\
  \textbf{fixes} & \isa{prod} :: \isa{{\isachardoublequote}{\isacharbrackleft}{\isacharprime}b{\isacharcomma}\ {\isacharprime}b{\isacharbrackright}\ {\isasymRightarrow}\ {\isacharprime}b{\isachardoublequote}}
    ~(\textbf{infixl}~\isa{{\isachardoublequote}{\isasymcdot}{\isachardoublequote}}~70) \\
  \textbf{assumes} & \isa{{\isachardoublequote}semi\ prod{\isachardoublequote}} \\
  \textbf{notes} & \isa{prod{\isachardot}assoc}: \isa{{\isachardoublequote}{\isacharquery}x\ {\isasymcdot}\ {\isacharquery}y\ {\isasymcdot}\ {\isacharquery}z\ {\isacharequal}\ {\isacharquery}x\ {\isasymcdot}\ {\isacharparenleft}{\isacharquery}y\ {\isasymcdot}\ {\isacharquery}z{\isacharparenright}{\isachardoublequote}} \\
  \textbf{assumes} & \isa{{\isachardoublequote}comm{\isacharunderscore}semi{\isacharunderscore}axioms\ prod{\isachardoublequote}} \\
  \textbf{notes} & \isa{prod{\isachardot}comm}: \isa{{\isachardoublequote}{\isacharquery}x\ {\isasymcdot}\ {\isacharquery}y\ {\isacharequal}\ {\isacharquery}y\ {\isasymcdot}\ {\isacharquery}x{\isachardoublequote}} \\
  \textbf{notes} & \isa{prod{\isachardot}lcomm}: \isa{{\isachardoublequote}{\isacharquery}x\ {\isasymcdot}\ {\isacharparenleft}{\isacharquery}y\ {\isasymcdot}\ {\isacharquery}z{\isacharparenright}\ {\isacharequal}\ {\isacharquery}y\ {\isasymcdot}\ {\isacharparenleft}{\isacharquery}x\ {\isasymcdot}\ {\isacharquery}z{\isacharparenright}{\isachardoublequote}} \\
  \textbf{fixes} & \isa{hom} :: \isa{{\isachardoublequote}{\isacharprime}a\ {\isasymRightarrow}\ {\isacharprime}b{\isachardoublequote}} \\
  \textbf{assumes} & \isa{{\isachardoublequote}semi{\isacharunderscore}hom{\isacharunderscore}axioms\ sum{\isachardoublequote}} \\
  \textbf{notes} & \isa{sum{\isacharunderscore}prod{\isacharunderscore}hom{\isachardot}hom}:
    \isa{hom\ {\isacharparenleft}x\ {\isasymoplus}\ y{\isacharparenright}\ {\isacharequal}\ hom\ x\ {\isasymcdot}\ hom\ y}
\end{array}
\]

\paragraph{Example 5.}
  In this example, a locale expression leading to a list of context
  elements that is not well-defined is encountered, and it is illustrated
  how normalisation deals with multiple inheritance.
  Consider the specification of monads (in the algebraic sense)
  and monoids.%
\end{isamarkuptext}%
\isamarkuptrue%
\isacommand{locale}\isamarkupfalse%
\ monad\ {\isacharequal}\isanewline
\ \ \isakeyword{fixes}\ prod\ {\isacharcolon}{\isacharcolon}\ {\isachardoublequoteopen}{\isacharbrackleft}{\isacharprime}a{\isacharcomma}\ {\isacharprime}a{\isacharbrackright}\ {\isasymRightarrow}\ {\isacharprime}a{\isachardoublequoteclose}\ {\isacharparenleft}\isakeyword{infixl}\ {\isachardoublequoteopen}{\isasymcdot}{\isachardoublequoteclose}\ {\isadigit{7}}{\isadigit{0}}{\isacharparenright}\ \isakeyword{and}\ one\ {\isacharcolon}{\isacharcolon}\ {\isacharprime}a\ {\isacharparenleft}{\isachardoublequoteopen}{\isasymone}{\isachardoublequoteclose}\ {\isadigit{1}}{\isadigit{0}}{\isadigit{0}}{\isacharparenright}\isanewline
\ \ \isakeyword{assumes}\ l{\isacharunderscore}one{\isacharcolon}\ {\isachardoublequoteopen}{\isasymone}\ {\isasymcdot}\ x\ {\isacharequal}\ x{\isachardoublequoteclose}\ \isakeyword{and}\ r{\isacharunderscore}one{\isacharcolon}\ {\isachardoublequoteopen}x\ {\isasymcdot}\ {\isasymone}\ {\isacharequal}\ x{\isachardoublequoteclose}%
\begin{isamarkuptext}%
Monoids are both semigroups and monads and one would want to
  specify them as locale expression \isa{semi\ {\isacharplus}\ monad}.
  Unfortunately, this expression is not well-formed.  Its normal form
\begin{align*%
}
  \N(\isa{monad}) & = [\isa{monad\ prod}] \\
  \N(\isa{semi}+\isa{monad}) & =
       \N(\isa{semi}) \App \N(\isa{monad})
     = [\isa{semi\ prod}, \isa{monad\ prod}]
\end{align*%
}
  leads to a list containing the context element
\[
  \textbf{fixes}~\isa{prod} :: \isa{{\isachardoublequote}{\isacharbrackleft}{\isacharprime}a{\isacharcomma}\ {\isacharprime}a{\isacharbrackright}\ {\isasymRightarrow}\ {\isacharprime}a{\isachardoublequote}}
    ~(\textbf{infixl}~\isa{{\isachardoublequote}{\isasymcdot}{\isachardoublequote}}~70)
\]
  twice and thus violating the first criterion of well-formedness.  To
  avoid this problem, one can
  introduce a new locale \isa{magma} with the sole purpose of fixing the
  parameter and defining its syntax.  The specifications of semigroup
  and monad are changed so that they import \isa{magma}.%
\end{isamarkuptext}%
\isamarkuptrue%
\isacommand{locale}\isamarkupfalse%
\ magma\ {\isacharequal}\ \isakeyword{fixes}\ prod\ {\isacharparenleft}\isakeyword{infixl}\ {\isachardoublequoteopen}{\isasymcdot}{\isachardoublequoteclose}\ {\isadigit{7}}{\isadigit{0}}{\isacharparenright}\isanewline
\isanewline
\isacommand{locale}\isamarkupfalse%
\ semi{\isacharprime}\ {\isacharequal}\ magma\ {\isacharplus}\ \isakeyword{assumes}\ assoc{\isacharcolon}\ {\isachardoublequoteopen}{\isacharparenleft}x\ {\isasymcdot}\ y{\isacharparenright}\ {\isasymcdot}\ z\ {\isacharequal}\ x\ {\isasymcdot}\ {\isacharparenleft}y\ {\isasymcdot}\ z{\isacharparenright}{\isachardoublequoteclose}\isanewline
\isacommand{locale}\isamarkupfalse%
\ monad{\isacharprime}\ {\isacharequal}\ magma\ {\isacharplus}\ \isakeyword{fixes}\ one\ {\isacharparenleft}{\isachardoublequoteopen}{\isasymone}{\isachardoublequoteclose}\ {\isadigit{1}}{\isadigit{0}}{\isadigit{0}}{\isacharparenright}\isanewline
\ \ \isakeyword{assumes}\ l{\isacharunderscore}one{\isacharcolon}\ {\isachardoublequoteopen}{\isasymone}\ {\isasymcdot}\ x\ {\isacharequal}\ x{\isachardoublequoteclose}\ \isakeyword{and}\ r{\isacharunderscore}one{\isacharcolon}\ {\isachardoublequoteopen}x\ {\isasymcdot}\ {\isasymone}\ {\isacharequal}\ x{\isachardoublequoteclose}%
\begin{isamarkuptext}%
Normalisation now yields
\begin{align*%
}
  \N(\isa{semi{\isacharprime}\ {\isacharplus}\ monad{\isacharprime}}) & =
       \N(\isa{semi{\isacharprime}}) \App \N(\isa{monad{\isacharprime}}) \\
     & = (\N(\isa{magma}) \App [\isa{semi{\isacharprime}\ prod}]) \App
         (\N(\isa{magma}) \App [\isa{monad{\isacharprime}\ prod}]) \\
     & = [\isa{magma\ prod}, \isa{semi{\isacharprime}\ prod}] \App
         [\isa{magma\ prod}, \isa{monad{\isacharprime}\ prod}]) \\
     & = [\isa{magma\ prod}, \isa{semi{\isacharprime}\ prod},
          \isa{monad{\isacharprime}\ prod}]
\end{align*%
}
  where the second occurrence of \isa{magma\ prod} is eliminated.
  The reader is encouraged to check, using the \textbf{print\_locale}
  command, that the list of context elements generated from this is
  indeed well-formed.

  It follows that importing
  parameters is more flexible than fixing them using a context element.
  The Locale package provides the predefined locale \isa{var} that
  can be used to import parameters if no
  particular mixfix syntax is required.
  Its definition is
\begin{center}
  \textbf{locale} \isa{var} = \textbf{fixes} \isa{x{\isacharunderscore}}
\end{center}
  The use of the internal variable \isa{x{\isacharunderscore}}
  enforces that the parameter is renamed before being used, because
  internal variables may not occur in the input syntax.  Using
  \isa{var}, the locale \isa{magma} could have been replaced by
  the locale expression
\begin{center}
  \isa{var} \isa{prod} (\textbf{infixl} \isa{{\isachardoublequote}{\isasymcdot}{\isachardoublequote}} 70)
\end{center}
  in the above locale declarations.%
\end{isamarkuptext}%
\isamarkuptrue%
%
\isamarkupsubsection{Includes%
}
\isamarkuptrue%
%
\begin{isamarkuptext}%
\label{sec-includes}
  The context element \textbf{includes} takes a locale expression $e$
  as argument.  It can only occur in long goals, where it
  adds, like a target, locale context to the proof context.  Unlike
  with targets, the proved theorem is not stored
  in the locale.  Instead, it is exported immediately.%
\end{isamarkuptext}%
\isamarkuptrue%
\isacommand{theorem}\isamarkupfalse%
\ lcomm{\isadigit{2}}{\isacharcolon}\isanewline
\ \ \isakeyword{includes}\ comm{\isacharunderscore}semi\ \isakeyword{shows}\ {\isachardoublequoteopen}x\ {\isasymcdot}\ {\isacharparenleft}y\ {\isasymcdot}\ z{\isacharparenright}\ {\isacharequal}\ y\ {\isasymcdot}\ {\isacharparenleft}x\ {\isasymcdot}\ z{\isacharparenright}{\isachardoublequoteclose}\isanewline
%
\isadelimproof
%
\endisadelimproof
%
\isatagproof
\isacommand{proof}\isamarkupfalse%
\ {\isacharminus}\isanewline
\ \ \isacommand{have}\isamarkupfalse%
\ {\isachardoublequoteopen}x\ {\isasymcdot}\ {\isacharparenleft}y\ {\isasymcdot}\ z{\isacharparenright}\ {\isacharequal}\ {\isacharparenleft}x\ {\isasymcdot}\ y{\isacharparenright}\ {\isasymcdot}\ z{\isachardoublequoteclose}\ \isacommand{by}\isamarkupfalse%
\ {\isacharparenleft}simp\ add{\isacharcolon}\ assoc{\isacharparenright}\isanewline
\ \ \isacommand{also}\isamarkupfalse%
\ \isacommand{have}\isamarkupfalse%
\ {\isachardoublequoteopen}{\isasymdots}\ {\isacharequal}\ {\isacharparenleft}y\ {\isasymcdot}\ x{\isacharparenright}\ {\isasymcdot}\ z{\isachardoublequoteclose}\ \isacommand{by}\isamarkupfalse%
\ {\isacharparenleft}simp\ add{\isacharcolon}\ comm{\isacharparenright}\isanewline
\ \ \isacommand{also}\isamarkupfalse%
\ \isacommand{have}\isamarkupfalse%
\ {\isachardoublequoteopen}{\isasymdots}\ {\isacharequal}\ y\ {\isasymcdot}\ {\isacharparenleft}x\ {\isasymcdot}\ z{\isacharparenright}{\isachardoublequoteclose}\ \isacommand{by}\isamarkupfalse%
\ {\isacharparenleft}simp\ add{\isacharcolon}\ assoc{\isacharparenright}\isanewline
\ \ \isacommand{finally}\isamarkupfalse%
\ \isacommand{show}\isamarkupfalse%
\ {\isacharquery}thesis\ \isacommand{{\isachardot}}\isamarkupfalse%
\isanewline
\isacommand{qed}\isamarkupfalse%
%
\endisatagproof
{\isafoldproof}%
%
\isadelimproof
%
\endisadelimproof
%
\begin{isamarkuptext}%
This proof is identical to the proof of \isa{lcomm}.  The use of
  \textbf{includes} provides the same context and facts as when using
  \isa{comm{\isacharunderscore}semi} as target.  On the other hand, \isa{lcomm{\isadigit{2}}} is not added as a fact to the locale \isa{comm{\isacharunderscore}semi}, but
  is directly visible in the theory.  The theorem is
\[
  \isa{comm{\isacharunderscore}semi\ {\isacharquery}prod\ {\isasymLongrightarrow}\ {\isacharquery}prod\ {\isacharquery}x\ {\isacharparenleft}{\isacharquery}prod\ {\isacharquery}y\ {\isacharquery}z{\isacharparenright}\ {\isacharequal}\ {\isacharquery}prod\ {\isacharquery}y\ {\isacharparenleft}{\isacharquery}prod\ {\isacharquery}x\ {\isacharquery}z{\isacharparenright}}.
\]
  Note that it is possible to
  combine a target and (several) \textbf{includes} in a goal statement, thus
  using contexts of several locales but storing the theorem in only
  one of them.%
\end{isamarkuptext}%
\isamarkuptrue%
%
\isamarkupsection{Structures%
}
\isamarkuptrue%
%
\begin{isamarkuptext}%
\label{sec-structures}
  The specifications of semigroups and monoids that served as examples
  in previous sections modelled each operation of an algebraic
  structure as a single parameter.  This is rather inconvenient for
  structures with many operations, and also unnatural.  In accordance
  to mathematical texts, one would rather fix two groups instead of
  two sets of operations.

  The approach taken in Isabelle is to encode algebraic structures
  with suitable types (in Isabelle/HOL usually records).  An issue to
  be addressed by
  locales is syntax for algebraic structures.  This is the purpose of
  the \textbf{(structure)} annotation in \textbf{fixes}, introduced by
  Wenzel.  We illustrate this, independently of record types, with a
  different formalisation of semigroups.

  Let \isa{{\isacharprime}a\ semi{\isacharunderscore}type} be a not further specified type that
  represents semigroups over the carrier type \isa{{\isacharprime}a}.  Let \isa{s{\isacharunderscore}op} be an operation that maps an object of \isa{{\isacharprime}a\ semi{\isacharunderscore}type} to
  a binary operation.%
\end{isamarkuptext}%
\isamarkuptrue%
\isacommand{typedecl}\isamarkupfalse%
\ {\isacharprime}a\ semi{\isacharunderscore}type\isanewline
\isacommand{consts}\isamarkupfalse%
\ s{\isacharunderscore}op\ {\isacharcolon}{\isacharcolon}\ {\isachardoublequoteopen}{\isacharbrackleft}{\isacharprime}a\ semi{\isacharunderscore}type{\isacharcomma}\ {\isacharprime}a{\isacharcomma}\ {\isacharprime}a{\isacharbrackright}\ {\isasymRightarrow}\ {\isacharprime}a{\isachardoublequoteclose}\ {\isacharparenleft}\isakeyword{infixl}\ {\isachardoublequoteopen}{\isasymstar}{\isasymindex}{\isachardoublequoteclose}\ {\isadigit{7}}{\isadigit{0}}{\isacharparenright}%
\begin{isamarkuptext}%
Although \isa{s{\isacharunderscore}op} is a ternary operation, it is declared
  infix.  The syntax annotation contains the token  \isa{{\isasymindex}}
  (\verb.\<index>.), which refers to the first argument.  This syntax is only
  effective in the context of a locale, and only if the first argument
  is a
  \emph{structural} parameter --- that is, a parameter with annotation
  \textbf{(structure)}.  The token has the effect of subscripting the
  parameter --- by bracketing it between \verb.\<^bsub>. and  \verb.\<^esub>..
  Additionally, the subscript of the first structural parameter may be
  omitted, as in this specification of semigroups with structures:%
\end{isamarkuptext}%
\isamarkuptrue%
\isacommand{locale}\isamarkupfalse%
\ comm{\isacharunderscore}semi{\isacharprime}\ {\isacharequal}\isanewline
\ \ \isakeyword{fixes}\ G\ {\isacharcolon}{\isacharcolon}\ {\isachardoublequoteopen}{\isacharprime}a\ semi{\isacharunderscore}type{\isachardoublequoteclose}\ {\isacharparenleft}\isakeyword{structure}{\isacharparenright}\isanewline
\ \ \isakeyword{assumes}\ assoc{\isacharcolon}\ {\isachardoublequoteopen}{\isacharparenleft}x\ {\isasymstar}\ y{\isacharparenright}\ {\isasymstar}\ z\ {\isacharequal}\ x\ {\isasymstar}\ {\isacharparenleft}y\ {\isasymstar}\ z{\isacharparenright}{\isachardoublequoteclose}\ \isakeyword{and}\ comm{\isacharcolon}\ {\isachardoublequoteopen}x\ {\isasymstar}\ y\ {\isacharequal}\ y\ {\isasymstar}\ x{\isachardoublequoteclose}%
\begin{isamarkuptext}%
Here \isa{x\ {\isasymstar}\ y} is equivalent to \isa{x\ {\isasymstar}\isactrlbsub G\isactrlesub \ y} and
  abbreviates \isa{s{\isacharunderscore}op\ G\ x\ y}.  A specification of homomorphisms
  requires a second structural parameter.%
\end{isamarkuptext}%
\isamarkuptrue%
\isacommand{locale}\isamarkupfalse%
\ semi{\isacharprime}{\isacharunderscore}hom\ {\isacharequal}\ comm{\isacharunderscore}semi{\isacharprime}\ {\isacharplus}\ comm{\isacharunderscore}semi{\isacharprime}\ H\ {\isacharplus}\isanewline
\ \ \isakeyword{fixes}\ hom\isanewline
\ \ \isakeyword{assumes}\ hom{\isacharcolon}\ {\isachardoublequoteopen}hom\ {\isacharparenleft}x\ {\isasymstar}\ y{\isacharparenright}\ {\isacharequal}\ hom\ x\ {\isasymstar}\isactrlbsub H\isactrlesub \ hom\ y{\isachardoublequoteclose}%
\begin{isamarkuptext}%
The parameter \isa{H} is defined in the second \textbf{fixes}
  element of $\C(\isa{semi{\isacharprime}{\isacharunderscore}comm})$. Hence \isa{{\isasymstar}\isactrlbsub H\isactrlesub }
  abbreviates \isa{s{\isacharunderscore}op\ H\ x\ y}.  The same construction can be done
  with records instead of an \textit{ad-hoc} type.%
\end{isamarkuptext}%
\isamarkuptrue%
\isacommand{record}\isamarkupfalse%
\ {\isacharprime}a\ semi\ {\isacharequal}\ prod\ {\isacharcolon}{\isacharcolon}\ {\isachardoublequoteopen}{\isacharbrackleft}{\isacharprime}a{\isacharcomma}\ {\isacharprime}a{\isacharbrackright}\ {\isasymRightarrow}\ {\isacharprime}a{\isachardoublequoteclose}\ {\isacharparenleft}\isakeyword{infixl}\ {\isachardoublequoteopen}{\isasymbullet}{\isasymindex}{\isachardoublequoteclose}\ {\isadigit{7}}{\isadigit{0}}{\isacharparenright}%
\begin{isamarkuptext}%
This declares the types \isa{{\isacharprime}a\ semi} and  \isa{{\isacharparenleft}{\isacharprime}a{\isacharcomma}\ {\isacharprime}b{\isacharparenright}\ semi{\isacharunderscore}scheme}.  The latter is an extensible record, where the second
  type argument is the type of the extension field.  For details on
  records, see \cite{NipkowEtAl2002} Chapter~8.3.%
\end{isamarkuptext}%
\isamarkuptrue%
\isacommand{locale}\isamarkupfalse%
\ semi{\isacharunderscore}w{\isacharunderscore}records\ {\isacharequal}\ struct\ G\ {\isacharplus}\isanewline
\ \ \isakeyword{assumes}\ assoc{\isacharcolon}\ {\isachardoublequoteopen}{\isacharparenleft}x\ {\isasymbullet}\ y{\isacharparenright}\ {\isasymbullet}\ z\ {\isacharequal}\ x\ {\isasymbullet}\ {\isacharparenleft}y\ {\isasymbullet}\ z{\isacharparenright}{\isachardoublequoteclose}%
\begin{isamarkuptext}%
The type \isa{{\isacharparenleft}{\isacharprime}a{\isacharcomma}\ {\isacharprime}b{\isacharparenright}\ semi{\isacharunderscore}scheme} is inferred for the parameter \isa{G}.  Using subtyping on records, the specification can be extended
  to groups easily.%
\end{isamarkuptext}%
\isamarkuptrue%
\isacommand{record}\isamarkupfalse%
\ {\isacharprime}a\ group\ {\isacharequal}\ {\isachardoublequoteopen}{\isacharprime}a\ semi{\isachardoublequoteclose}\ {\isacharplus}\isanewline
\ \ one\ {\isacharcolon}{\isacharcolon}\ {\isachardoublequoteopen}{\isacharprime}a{\isachardoublequoteclose}\ {\isacharparenleft}{\isachardoublequoteopen}l{\isasymindex}{\isachardoublequoteclose}\ {\isadigit{1}}{\isadigit{0}}{\isadigit{0}}{\isacharparenright}\isanewline
\ \ inv\ {\isacharcolon}{\isacharcolon}\ {\isachardoublequoteopen}{\isacharprime}a\ {\isasymRightarrow}\ {\isacharprime}a{\isachardoublequoteclose}\ {\isacharparenleft}{\isachardoublequoteopen}inv{\isasymindex}\ {\isacharunderscore}{\isachardoublequoteclose}\ {\isacharbrackleft}{\isadigit{8}}{\isadigit{1}}{\isacharbrackright}\ {\isadigit{8}}{\isadigit{0}}{\isacharparenright}\isanewline
\isacommand{locale}\isamarkupfalse%
\ group{\isacharunderscore}w{\isacharunderscore}records\ {\isacharequal}\ semi{\isacharunderscore}w{\isacharunderscore}records\ {\isacharplus}\isanewline
\ \ \isakeyword{assumes}\ l{\isacharunderscore}one{\isacharcolon}\ {\isachardoublequoteopen}l\ {\isasymbullet}\ x\ {\isacharequal}\ x{\isachardoublequoteclose}\ \isakeyword{and}\ l{\isacharunderscore}inv{\isacharcolon}\ {\isachardoublequoteopen}inv\ x\ {\isasymbullet}\ x\ {\isacharequal}\ l{\isachardoublequoteclose}%
\begin{isamarkuptext}%
Finally, the predefined locale
\begin{center}
  \textbf{locale} \textit{struct} = \textbf{fixes} \isa{S{\isacharunderscore}}
    \textbf{(structure)}.
\end{center}
  is analogous to \isa{var}.  
  More examples on the use of structures, including groups, rings and
  polynomials can be found in the Isabelle distribution in the
  session HOL-Algebra.%
\end{isamarkuptext}%
\isamarkuptrue%
%
\isamarkupsection{Conclusions and Outlook%
}
\isamarkuptrue%
%
\begin{isamarkuptext}%
Locales provide a simple means of modular reasoning.  They enable to
  abbreviate frequently occurring context statements and maintain facts
  valid in these contexts.  Importantly, using structures, they allow syntax to be
  effective only in certain contexts, and thus to mimic common
  practice in mathematics, where notation is chosen very flexibly.
  This is also known as literate formalisation \cite{Bailey1998}.
  Locale expressions allow to duplicate and merge
  specifications.  This is a necessity, for example, when reasoning about
  homomorphisms.  Normalisation makes it possible to deal with
  diamond-shaped inheritance structures, and generally with directed
  acyclic graphs.  The combination of locales with record
  types in higher-order logic provides an effective means for
  specifying algebraic structures: locale import and record subtyping
  provide independent hierarchies for specifications and structure
  elements.  Rich examples for this can be found in
  the Isabelle distribution in the session HOL-Algebra.

  The primary reason for writing this report was to provide a better
  understanding of locales in Isar.  Wenzel provided hardly any
  documentation, with the exception of \cite{Wenzel2002b}.  The
  present report should make it easier for users of Isabelle to take
  advantage of locales.

  The report is also a base for future extensions.  These include
  improved syntax for structures.  Identifying them by numbers seems
  unnatural and can be confusing if more than two structures are
  involved --- for example, when reasoning about universal
  properties --- and numbering them by order of occurrence seems
  arbitrary.  Another desirable feature is \emph{instantiation}.  One
  may, in the course of a theory development, construct objects that
  fulfil the specification of a locale.  These objects are possibly
  defined in the context of another locale.  Instantiation should make it
  simple to specialise abstract facts for the object under
  consideration and to use the specified facts.

  A detailed comparison of locales with module systems in type theory
  has not been undertaken yet, but could be beneficial.  For example,
  a module system for Coq has recently been presented by Chrzaszcz
  \cite{Chrzaszcz2003,Chrzaszcz2004}.  While the
  latter usually constitute extensions of the calculus, locales are
  a rather thin layer that does not change Isabelle's meta logic.
  Locales mainly manage specifications and facts.  Functors, like
  the constructor for polynomial rings, remain objects of the
  logic.

  \textbf{Acknowledgements.}  Lawrence C.\ Paulson and Norbert
  Schirmer made useful comments on a draft of this paper.%
\end{isamarkuptext}%
\isamarkuptrue%
%
\isadelimtheory
%
\endisadelimtheory
%
\isatagtheory
%
\endisatagtheory
{\isafoldtheory}%
%
\isadelimtheory
%
\endisadelimtheory
\end{isabellebody}%
%%% Local Variables:
%%% mode: latex
%%% TeX-master: "root"
%%% End:


%%% Local Variables:
%%% mode: latex
%%% TeX-master: "root"
%%% End:
, in the body of \texttt{root.tex} does the job
  in most situations.

  \item \texttt{IsaMakefile} holds appropriate dependencies and
  invocations of Isabelle tools to control the batch job.  In fact,
  several sessions may be managed by the same \texttt{IsaMakefile}.
  See the \emph{Isabelle System Manual} \cite{isabelle-sys} 
  for further details, especially on
  \texttt{isatool usedir} and \texttt{isatool make}.

  \end{itemize}

  One may now start to populate the directory \texttt{MySession}, and
  the file \texttt{MySession/ROOT.ML} accordingly.  The file
  \texttt{MySession/document/root.tex} should also be adapted at some
  point; the default version is mostly self-explanatory.  Note that
  \verb,\isabellestyle, enables fine-tuning of the general appearance
  of characters and mathematical symbols (see also
  \S\ref{sec:doc-prep-symbols}).

  Especially observe the included {\LaTeX} packages \texttt{isabelle}
  (mandatory), \texttt{isabellesym} (required for mathematical
  symbols), and the final \texttt{pdfsetup} (provides sane defaults
  for \texttt{hyperref}, including URL markup).  All three are
  distributed with Isabelle. Further packages may be required in
  particular applications, say for unusual mathematical symbols.

  \medskip Any additional files for the {\LaTeX} stage go into the
  \texttt{MySession/document} directory as well.  In particular,
  adding a file named \texttt{root.bib} causes an automatic run of
  \texttt{bibtex} to process a bibliographic database; see also
  \texttt{isatool document} \cite{isabelle-sys}.

  \medskip Any failure of the document preparation phase in an
  Isabelle batch session leaves the generated sources in their target
  location, identified by the accompanying error message.  This lets
  you trace {\LaTeX} problems with the generated files at hand.%
\end{isamarkuptext}%
\isamarkuptrue%
%
\isamarkupsubsection{Structure Markup%
}
\isamarkuptrue%
%
\begin{isamarkuptext}%
The large-scale structure of Isabelle documents follows existing
  {\LaTeX} conventions, with chapters, sections, subsubsections etc.
  The Isar language includes separate \bfindex{markup commands}, which
  do not affect the formal meaning of a theory (or proof), but result
  in corresponding {\LaTeX} elements.

  There are separate markup commands depending on the textual context:
  in header position (just before \isakeyword{theory}), within the
  theory body, or within a proof.  The header needs to be treated
  specially here, since ordinary theory and proof commands may only
  occur \emph{after} the initial \isakeyword{theory} specification.

  \medskip

  \begin{tabular}{llll}
  header & theory & proof & default meaning \\\hline
    & \commdx{chapter} & & \verb,\chapter, \\
  \commdx{header} & \commdx{section} & \commdx{sect} & \verb,\section, \\
    & \commdx{subsection} & \commdx{subsect} & \verb,\subsection, \\
    & \commdx{subsubsection} & \commdx{subsubsect} & \verb,\subsubsection, \\
  \end{tabular}

  \medskip

  From the Isabelle perspective, each markup command takes a single
  $text$ argument (delimited by \verb,",~\isa{{\isasymdots}}~\verb,", or
  \verb,{,\verb,*,~\isa{{\isasymdots}}~\verb,*,\verb,},).  After stripping any
  surrounding white space, the argument is passed to a {\LaTeX} macro
  \verb,\isamarkupXYZ, for command \isakeyword{XYZ}.  These macros are
  defined in \verb,isabelle.sty, according to the meaning given in the
  rightmost column above.

  \medskip The following source fragment illustrates structure markup
  of a theory.  Note that {\LaTeX} labels may be included inside of
  section headings as well.

  \begin{ttbox}
  header {\ttlbrace}* Some properties of Foo Bar elements *{\ttrbrace}

  theory Foo_Bar = Main:

  subsection {\ttlbrace}* Basic definitions *{\ttrbrace}

  consts
    foo :: \dots
    bar :: \dots

  defs \dots

  subsection {\ttlbrace}* Derived rules *{\ttrbrace}

  lemma fooI: \dots
  lemma fooE: \dots

  subsection {\ttlbrace}* Main theorem {\ttback}label{\ttlbrace}sec:main-theorem{\ttrbrace} *{\ttrbrace}

  theorem main: \dots

  end
  \end{ttbox}\vspace{-\medskipamount}

  You may occasionally want to change the meaning of markup commands,
  say via \verb,\renewcommand, in \texttt{root.tex}.  For example,
  \verb,\isamarkupheader, is a good candidate for some tuning.  We
  could move it up in the hierarchy to become \verb,\chapter,.

\begin{verbatim}
  \renewcommand{\isamarkupheader}[1]{\chapter{#1}}
\end{verbatim}

  \noindent Now we must change the document class given in
  \texttt{root.tex} to something that supports chapters.  A suitable
  command is \verb,\documentclass{report},.

  \medskip The {\LaTeX} macro \verb,\isabellecontext, is maintained to
  hold the name of the current theory context.  This is particularly
  useful for document headings:

\begin{verbatim}
  \renewcommand{\isamarkupheader}[1]
  {\chapter{#1}\markright{THEORY~\isabellecontext}}
\end{verbatim}

  \noindent Make sure to include something like
  \verb,\pagestyle{headings}, in \texttt{root.tex}; the document
  should have more than two pages to show the effect.%
\end{isamarkuptext}%
\isamarkuptrue%
%
\isamarkupsubsection{Formal Comments and Antiquotations \label{sec:doc-prep-text}%
}
\isamarkuptrue%
%
\begin{isamarkuptext}%
Isabelle \bfindex{source comments}, which are of the form
  \verb,(,\verb,*,~\isa{{\isasymdots}}~\verb,*,\verb,),, essentially act like
  white space and do not really contribute to the content.  They
  mainly serve technical purposes to mark certain oddities in the raw
  input text.  In contrast, \bfindex{formal comments} are portions of
  text that are associated with formal Isabelle/Isar commands
  (\bfindex{marginal comments}), or as standalone paragraphs within a
  theory or proof context (\bfindex{text blocks}).

  \medskip Marginal comments are part of each command's concrete
  syntax \cite{isabelle-ref}; the common form is ``\verb,--,~$text$''
  where $text$ is delimited by \verb,",\isa{{\isasymdots}}\verb,", or
  \verb,{,\verb,*,~\isa{{\isasymdots}}~\verb,*,\verb,}, as before.  Multiple
  marginal comments may be given at the same time.  Here is a simple
  example:%
\end{isamarkuptext}%
\isamarkuptrue%
\isacommand{lemma}\ {\isachardoublequote}A\ {\isacharminus}{\isacharminus}{\isachargreater}\ A{\isachardoublequote}\isanewline
\ \ %
\isamarkupcmt{a triviality of propositional logic%
}
\isanewline
\ \ %
\isamarkupcmt{(should not really bother)%
}
\isanewline
\ \ \isamarkupfalse%
\isacommand{by}\ {\isacharparenleft}rule\ impI{\isacharparenright}\ %
\isamarkupcmt{implicit assumption step involved here%
}
\isamarkupfalse%
%
\begin{isamarkuptext}%
\noindent The above output has been produced as follows:

\begin{verbatim}
  lemma "A --> A"
    -- "a triviality of propositional logic"
    -- "(should not really bother)"
    by (rule impI) -- "implicit assumption step involved here"
\end{verbatim}

  From the {\LaTeX} viewpoint, ``\verb,--,'' acts like a markup
  command, associated with the macro \verb,\isamarkupcmt, (taking a
  single argument).

  \medskip Text blocks are introduced by the commands \bfindex{text}
  and \bfindex{txt}, for theory and proof contexts, respectively.
  Each takes again a single $text$ argument, which is interpreted as a
  free-form paragraph in {\LaTeX} (surrounded by some additional
  vertical space).  This behavior may be changed by redefining the
  {\LaTeX} environments of \verb,isamarkuptext, or
  \verb,isamarkuptxt,, respectively (via \verb,\renewenvironment,) The
  text style of the body is determined by \verb,\isastyletext, and
  \verb,\isastyletxt,; the default setup uses a smaller font within
  proofs.  This may be changed as follows:

\begin{verbatim}
  \renewcommand{\isastyletxt}{\isastyletext}
\end{verbatim}

  \medskip The $text$ part of Isabelle markup commands essentially
  inserts \emph{quoted material} into a formal text, mainly for
  instruction of the reader.  An \bfindex{antiquotation} is again a
  formal object embedded into such an informal portion.  The
  interpretation of antiquotations is limited to some well-formedness
  checks, with the result being pretty printed to the resulting
  document.  Quoted text blocks together with antiquotations provide
  an attractive means of referring to formal entities, with good
  confidence in getting the technical details right (especially syntax
  and types).

  The general syntax of antiquotations is as follows:
  \texttt{{\at}{\ttlbrace}$name$ $arguments${\ttrbrace}}, or
  \texttt{{\at}{\ttlbrace}$name$ [$options$] $arguments${\ttrbrace}}
  for a comma-separated list of options consisting of a $name$ or
  \texttt{$name$=$value$} each.  The syntax of $arguments$ depends on
  the kind of antiquotation, it generally follows the same conventions
  for types, terms, or theorems as in the formal part of a theory.

  \medskip This sentence demonstrates quotations and antiquotations:
  \isa{{\isasymlambda}x\ y{\isachardot}\ x} is a well-typed term.

  \medskip\noindent The output above was produced as follows:
  \begin{ttbox}
text {\ttlbrace}*
  This sentence demonstrates quotations and antiquotations:
  {\at}{\ttlbrace}term "%x y. x"{\ttrbrace} is a well-typed term.
*{\ttrbrace}
  \end{ttbox}\vspace{-\medskipamount}

  The notational change from the ASCII character~\verb,%, to the
  symbol~\isa{{\isasymlambda}} reveals that Isabelle printed this term, after
  parsing and type-checking.  Document preparation enables symbolic
  output by default.

  \medskip The next example includes an option to modify Isabelle's
  \verb,show_types, flag.  The antiquotation
  \texttt{{\at}}\verb,{term [show_types] "%x y. x"}, produces the
  output \isa{{\isasymlambda}{\isacharparenleft}x{\isasymColon}{\isacharprime}a{\isacharparenright}\ y{\isasymColon}{\isacharprime}b{\isachardot}\ x}.  Type inference has figured
  out the most general typings in the present theory context.  Terms
  may acquire different typings due to constraints imposed by their
  environment; within a proof, for example, variables are given the
  same types as they have in the main goal statement.

  \medskip Several further kinds of antiquotations and options are
  available \cite{isabelle-sys}.  Here are a few commonly used
  combinations:

  \medskip

  \begin{tabular}{ll}
  \texttt{\at}\verb,{typ,~$\tau$\verb,}, & print type $\tau$ \\
  \texttt{\at}\verb,{term,~$t$\verb,}, & print term $t$ \\
  \texttt{\at}\verb,{prop,~$\phi$\verb,}, & print proposition $\phi$ \\
  \texttt{\at}\verb,{prop [display],~$\phi$\verb,}, & print large proposition $\phi$ (with linebreaks) \\
  \texttt{\at}\verb,{prop [source],~$\phi$\verb,}, & check proposition $\phi$, print its input \\
  \texttt{\at}\verb,{thm,~$a$\verb,}, & print fact $a$ \\
  \texttt{\at}\verb,{thm,~$a$~\verb,[no_vars]}, & print fact $a$, fixing schematic variables \\
  \texttt{\at}\verb,{thm [source],~$a$\verb,}, & check availability of fact $a$, print its name \\
  \texttt{\at}\verb,{text,~$s$\verb,}, & print uninterpreted text $s$ \\
  \end{tabular}

  \medskip

  Note that \attrdx{no_vars} given above is \emph{not} an
  antiquotation option, but an attribute of the theorem argument given
  here.  This might be useful with a diagnostic command like
  \isakeyword{thm}, too.

  \medskip The \texttt{\at}\verb,{text, $s$\verb,}, antiquotation is
  particularly interesting.  Embedding uninterpreted text within an
  informal body might appear useless at first sight.  Here the key
  virtue is that the string $s$ is processed as Isabelle output,
  interpreting Isabelle symbols appropriately.

  For example, \texttt{\at}\verb,{text "\<forall>\<exists>"}, produces \isa{{\isasymforall}{\isasymexists}}, according to the standard interpretation of these symbol
  (cf.\ \S\ref{sec:doc-prep-symbols}).  Thus we achieve consistent
  mathematical notation in both the formal and informal parts of the
  document very easily, independently of the term language of
  Isabelle.  Manual {\LaTeX} code would leave more control over the
  typesetting, but is also slightly more tedious.%
\end{isamarkuptext}%
\isamarkuptrue%
%
\isamarkupsubsection{Interpretation of Symbols \label{sec:doc-prep-symbols}%
}
\isamarkuptrue%
%
\begin{isamarkuptext}%
As has been pointed out before (\S\ref{sec:syntax-symbols}),
  Isabelle symbols are the smallest syntactic entities --- a
  straightforward generalization of ASCII characters.  While Isabelle
  does not impose any interpretation of the infinite collection of
  named symbols, {\LaTeX} documents use canonical glyphs for certain
  standard symbols \cite[appendix~A]{isabelle-sys}.

  The {\LaTeX} code produced from Isabelle text follows a simple
  scheme.  You can tune the final appearance by redefining certain
  macros, say in \texttt{root.tex} of the document.

  \begin{enumerate}

  \item 7-bit ASCII characters: letters \texttt{A\dots Z} and
  \texttt{a\dots z} are output directly, digits are passed as an
  argument to the \verb,\isadigit, macro, other characters are
  replaced by specifically named macros of the form
  \verb,\isacharXYZ,.

  \item Named symbols: \verb,\,\verb,<XYZ>, is turned into
  \verb,{\isasymXYZ},; note the additional braces.

  \item Named control symbols: \verb,\,\verb,<^XYZ>, is turned into
  \verb,\isactrlXYZ,; subsequent symbols may act as arguments if the
  control macro is defined accordingly.

  \end{enumerate}

  You may occasionally wish to give new {\LaTeX} interpretations of
  named symbols.  This merely requires an appropriate definition of
  \verb,\isasymXYZ,, for \verb,\,\verb,<XYZ>, (see
  \texttt{isabelle.sty} for working examples).  Control symbols are
  slightly more difficult to get right, though.

  \medskip The \verb,\isabellestyle, macro provides a high-level
  interface to tune the general appearance of individual symbols.  For
  example, \verb,\isabellestyle{it}, uses the italics text style to
  mimic the general appearance of the {\LaTeX} math mode; double
  quotes are not printed at all.  The resulting quality of typesetting
  is quite good, so this should be the default style for work that
  gets distributed to a broader audience.%
\end{isamarkuptext}%
\isamarkuptrue%
%
\isamarkupsubsection{Suppressing Output \label{sec:doc-prep-suppress}%
}
\isamarkuptrue%
%
\begin{isamarkuptext}%
By default, Isabelle's document system generates a {\LaTeX} file for
  each theory that gets loaded while running the session.  The
  generated \texttt{session.tex} will include all of these in order of
  appearance, which in turn gets included by the standard
  \texttt{root.tex}.  Certainly one may change the order or suppress
  unwanted theories by ignoring \texttt{session.tex} and load
  individual files directly in \texttt{root.tex}.  On the other hand,
  such an arrangement requires additional maintenance whenever the
  collection of theories changes.

  Alternatively, one may tune the theory loading process in
  \texttt{ROOT.ML} itself: traversal of the theory dependency graph
  may be fine-tuned by adding \verb,use_thy, invocations, although
  topological sorting still has to be observed.  Moreover, the ML
  operator \verb,no_document, temporarily disables document generation
  while executing a theory loader command.  Its usage is like this:

\begin{verbatim}
  no_document use_thy "T";
\end{verbatim}

  \medskip Theory output may be suppressed more selectively.  Research
  articles and slides usually do not include the formal content in
  full.  Delimiting \bfindex{ignored material} by the special source
  comments \verb,(,\verb,*,\verb,<,\verb,*,\verb,), and
  \verb,(,\verb,*,\verb,>,\verb,*,\verb,), tells the document
  preparation system to suppress these parts; the formal checking of
  the theory is unchanged, of course.

  In this example, we hide a theory's \isakeyword{theory} and
  \isakeyword{end} brackets:

  \medskip

  \begin{tabular}{l}
  \verb,(,\verb,*,\verb,<,\verb,*,\verb,), \\
  \texttt{theory T = Main:} \\
  \verb,(,\verb,*,\verb,>,\verb,*,\verb,), \\
  ~~$\vdots$ \\
  \verb,(,\verb,*,\verb,<,\verb,*,\verb,), \\
  \texttt{end} \\
  \verb,(,\verb,*,\verb,>,\verb,*,\verb,), \\
  \end{tabular}

  \medskip

  Text may be suppressed in a fine-grained manner.  We may even hide
  vital parts of a proof, pretending that things have been simpler
  than they really were.  For example, this ``fully automatic'' proof
  is actually a fake:%
\end{isamarkuptext}%
\isamarkuptrue%
\isacommand{lemma}\ {\isachardoublequote}x\ {\isasymnoteq}\ {\isacharparenleft}{\isadigit{0}}{\isacharcolon}{\isacharcolon}int{\isacharparenright}\ {\isasymLongrightarrow}\ {\isadigit{0}}\ {\isacharless}\ x\ {\isacharasterisk}\ x{\isachardoublequote}\isanewline
\ \ \isamarkupfalse%
\isacommand{by}\ {\isacharparenleft}auto{\isacharparenright}\isamarkupfalse%
%
\begin{isamarkuptext}%
\noindent Here the real source of the proof has been as follows:

\begin{verbatim}
  by (auto(*<*)simp add: zero_less_mult_iff(*>*))
\end{verbatim}
%(*

  \medskip Suppressing portions of printed text demands care.  You
  should not misrepresent the underlying theory development.  It is
  easy to invalidate the visible text by hiding references to
  questionable axioms.

  Authentic reports of Isabelle/Isar theories, say as part of a
  library, should suppress nothing.  Other users may need the full
  information for their own derivative work.  If a particular
  formalization appears inadequate for general public coverage, it is
  often more appropriate to think of a better way in the first place.

  \medskip Some technical subtleties of the
  \verb,(,\verb,*,\verb,<,\verb,*,\verb,),~\verb,(,\verb,*,\verb,>,\verb,*,\verb,),
  elements need to be kept in mind, too --- the system performs few
  sanity checks here.  Arguments of markup commands and formal
  comments must not be hidden, otherwise presentation fails.  Open and
  close parentheses need to be inserted carefully; it is easy to hide
  the wrong parts, especially after rearranging the theory text.%
\end{isamarkuptext}%
\isamarkuptrue%
\isamarkupfalse%
\end{isabellebody}%
%%% Local Variables:
%%% mode: latex
%%% TeX-master: "root"
%%% End:
