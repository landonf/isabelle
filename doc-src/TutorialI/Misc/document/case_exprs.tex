%
\begin{isabellebody}%
\def\isabellecontext{case{\isacharunderscore}exprs}%
\isamarkupfalse%
%
\isamarkupsubsection{Case Expressions%
}
\isamarkuptrue%
%
\begin{isamarkuptext}%
\label{sec:case-expressions}\index{*case expressions}%
HOL also features \isa{case}-expressions for analyzing
elements of a datatype. For example,
\begin{isabelle}%
\ \ \ \ \ case\ xs\ of\ {\isacharbrackleft}{\isacharbrackright}\ {\isasymRightarrow}\ {\isacharbrackleft}{\isacharbrackright}\ {\isacharbar}\ y\ {\isacharhash}\ ys\ {\isasymRightarrow}\ y%
\end{isabelle}
evaluates to \isa{{\isacharbrackleft}{\isacharbrackright}} if \isa{xs} is \isa{{\isacharbrackleft}{\isacharbrackright}} and to \isa{y} if 
\isa{xs} is \isa{y\ {\isacharhash}\ ys}. (Since the result in both branches must be of
the same type, it follows that \isa{y} is of type \isa{{\isacharprime}a\ list} and hence
that \isa{xs} is of type \isa{{\isacharprime}a\ list\ list}.)

In general, if $e$ is a term of the datatype $t$ defined in
\S\ref{sec:general-datatype} above, the corresponding
\isa{case}-expression analyzing $e$ is
\[
\begin{array}{rrcl}
\isa{case}~e~\isa{of} & C@1~x@ {11}~\dots~x@ {1k@1} & \To & e@1 \\
                           \vdots \\
                           \mid & C@m~x@ {m1}~\dots~x@ {mk@m} & \To & e@m
\end{array}
\]

\begin{warn}
\emph{All} constructors must be present, their order is fixed, and nested
patterns are not supported.  Violating these restrictions results in strange
error messages.
\end{warn}
\noindent
Nested patterns can be simulated by nested \isa{case}-expressions: instead
of
\begin{isabelle}%
\ \ \ \ \ case\ xs\ of\ {\isacharbrackleft}{\isacharbrackright}\ {\isacharequal}{\isachargreater}\ {\isacharbrackleft}{\isacharbrackright}\ {\isacharbar}\ {\isacharbrackleft}x{\isacharbrackright}\ {\isacharequal}{\isachargreater}\ x\ {\isacharbar}\ x\ {\isacharhash}\ {\isacharparenleft}y\ {\isacharhash}\ zs{\isacharparenright}\ {\isacharequal}{\isachargreater}\ y%
\end{isabelle}
write
\begin{isabelle}%
\ \ \ \ \ case\ xs\ of\ {\isacharbrackleft}{\isacharbrackright}\ {\isasymRightarrow}\ {\isacharbrackleft}{\isacharbrackright}\isanewline
\isaindent{\ \ \ \ \ }{\isacharbar}\ x\ {\isacharhash}\ ys\ {\isasymRightarrow}\ case\ ys\ of\ {\isacharbrackleft}{\isacharbrackright}\ {\isasymRightarrow}\ x\ {\isacharbar}\ y\ {\isacharhash}\ zs\ {\isasymRightarrow}\ y%
\end{isabelle}

Note that \isa{case}-expressions may need to be enclosed in parentheses to
indicate their scope%
\end{isamarkuptext}%
\isamarkuptrue%
%
\isamarkupsubsection{Structural Induction and Case Distinction%
}
\isamarkuptrue%
%
\begin{isamarkuptext}%
\label{sec:struct-ind-case}
\index{case distinctions}\index{induction!structural}%
Induction is invoked by \methdx{induct_tac}, as we have seen above; 
it works for any datatype.  In some cases, induction is overkill and a case
distinction over all constructors of the datatype suffices.  This is performed
by \methdx{case_tac}.  Here is a trivial example:%
\end{isamarkuptext}%
\isamarkuptrue%
\isacommand{lemma}\ {\isachardoublequote}{\isacharparenleft}case\ xs\ of\ {\isacharbrackleft}{\isacharbrackright}\ {\isasymRightarrow}\ {\isacharbrackleft}{\isacharbrackright}\ {\isacharbar}\ y{\isacharhash}ys\ {\isasymRightarrow}\ xs{\isacharparenright}\ {\isacharequal}\ xs{\isachardoublequote}\isanewline
\isamarkupfalse%
\isacommand{apply}{\isacharparenleft}case{\isacharunderscore}tac\ xs{\isacharparenright}\isamarkupfalse%
%
\begin{isamarkuptxt}%
\noindent
results in the proof state
\begin{isabelle}%
\ {\isadigit{1}}{\isachardot}\ xs\ {\isacharequal}\ {\isacharbrackleft}{\isacharbrackright}\ {\isasymLongrightarrow}\ {\isacharparenleft}case\ xs\ of\ {\isacharbrackleft}{\isacharbrackright}\ {\isasymRightarrow}\ {\isacharbrackleft}{\isacharbrackright}\ {\isacharbar}\ y\ {\isacharhash}\ ys\ {\isasymRightarrow}\ xs{\isacharparenright}\ {\isacharequal}\ xs\isanewline
\ {\isadigit{2}}{\isachardot}\ {\isasymAnd}a\ list{\isachardot}\isanewline
\isaindent{\ {\isadigit{2}}{\isachardot}\ \ \ \ }xs\ {\isacharequal}\ a\ {\isacharhash}\ list\ {\isasymLongrightarrow}\ {\isacharparenleft}case\ xs\ of\ {\isacharbrackleft}{\isacharbrackright}\ {\isasymRightarrow}\ {\isacharbrackleft}{\isacharbrackright}\ {\isacharbar}\ y\ {\isacharhash}\ ys\ {\isasymRightarrow}\ xs{\isacharparenright}\ {\isacharequal}\ xs%
\end{isabelle}
which is solved automatically:%
\end{isamarkuptxt}%
\isamarkuptrue%
\isacommand{apply}{\isacharparenleft}auto{\isacharparenright}\isamarkupfalse%
\isamarkupfalse%
%
\begin{isamarkuptext}%
Note that we do not need to give a lemma a name if we do not intend to refer
to it explicitly in the future.
Other basic laws about a datatype are applied automatically during
simplification, so no special methods are provided for them.

\begin{warn}
  Induction is only allowed on free (or \isasymAnd-bound) variables that
  should not occur among the assumptions of the subgoal; see
  \S\ref{sec:ind-var-in-prems} for details. Case distinction
  (\isa{case{\isacharunderscore}tac}) works for arbitrary terms, which need to be
  quoted if they are non-atomic. However, apart from \isa{{\isasymAnd}}-bound
  variables, the terms must not contain variables that are bound outside.
  For example, given the goal \isa{{\isasymforall}xs{\isachardot}\ xs\ {\isacharequal}\ {\isacharbrackleft}{\isacharbrackright}\ {\isasymor}\ {\isacharparenleft}{\isasymexists}y\ ys{\isachardot}\ xs\ {\isacharequal}\ y\ {\isacharhash}\ ys{\isacharparenright}},
  \isa{case{\isacharunderscore}tac\ xs} will not work as expected because Isabelle interprets
  the \isa{xs} as a new free variable distinct from the bound
  \isa{xs} in the goal.
\end{warn}%
\end{isamarkuptext}%
\isamarkuptrue%
\isamarkupfalse%
\end{isabellebody}%
%%% Local Variables:
%%% mode: latex
%%% TeX-master: "root"
%%% End:
