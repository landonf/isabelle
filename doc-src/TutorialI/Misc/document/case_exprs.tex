%
\begin{isabellebody}%
\def\isabellecontext{case{\isacharunderscore}exprs}%
%
\isadelimtheory
%
\endisadelimtheory
%
\isatagtheory
%
\endisatagtheory
{\isafoldtheory}%
%
\isadelimtheory
%
\endisadelimtheory
%
\begin{isamarkuptext}%
\subsection{Case Expressions}
\label{sec:case-expressions}\index{*case expressions}%
HOL also features \isa{case}-expressions for analyzing
elements of a datatype. For example,
\begin{isabelle}%
\ \ \ \ \ case\ xs\ of\ {\isacharbrackleft}{\isacharbrackright}\ {\isasymRightarrow}\ {\isacharbrackleft}{\isacharbrackright}\ {\isacharbar}\ y\ {\isacharhash}\ ys\ {\isasymRightarrow}\ y%
\end{isabelle}
evaluates to \isa{{\isacharbrackleft}{\isacharbrackright}} if \isa{xs} is \isa{{\isacharbrackleft}{\isacharbrackright}} and to \isa{y} if 
\isa{xs} is \isa{y\ {\isacharhash}\ ys}. (Since the result in both branches must be of
the same type, it follows that \isa{y} is of type \isa{{\isacharprime}a\ list} and hence
that \isa{xs} is of type \isa{{\isacharprime}a\ list\ list}.)

In general, case expressions are of the form
\[
\begin{array}{c}
\isa{case}~e~\isa{of}\ pattern@1~\isa{{\isasymRightarrow}}~e@1\ \isa{{\isacharbar}}\ \dots\
 \isa{{\isacharbar}}~pattern@m~\isa{{\isasymRightarrow}}~e@m
\end{array}
\]
Like in functional programming, patterns are expressions consisting of
datatype constructors (e.g. \isa{{\isacharbrackleft}{\isacharbrackright}} and \isa{{\isacharhash}})
and variables, including the wildcard ``\verb$_$''.
Not all cases need to be covered and the order of cases matters.
However, one is well-advised not to wallow in complex patterns because
complex case distinctions tend to induce complex proofs.

\begin{warn}
Internally Isabelle only knows about exhaustive case expressions with
non-nested patterns: $pattern@i$ must be of the form
$C@i~x@ {i1}~\dots~x@ {ik@i}$ and $C@1, \dots, C@m$ must be exactly the
constructors of the type of $e$.
%
More complex case expressions are automatically
translated into the simpler form upon parsing but are not translated
back for printing. This may lead to surprising output.
\end{warn}

\begin{warn}
Like \isa{if}, \isa{case}-expressions may need to be enclosed in
parentheses to indicate their scope.
\end{warn}

\subsection{Structural Induction and Case Distinction}
\label{sec:struct-ind-case}
\index{case distinctions}\index{induction!structural}%
Induction is invoked by \methdx{induct_tac}, as we have seen above; 
it works for any datatype.  In some cases, induction is overkill and a case
distinction over all constructors of the datatype suffices.  This is performed
by \methdx{case_tac}.  Here is a trivial example:%
\end{isamarkuptext}%
\isamarkuptrue%
\isacommand{lemma}\isamarkupfalse%
\ {\isachardoublequoteopen}{\isacharparenleft}case\ xs\ of\ {\isacharbrackleft}{\isacharbrackright}\ {\isasymRightarrow}\ {\isacharbrackleft}{\isacharbrackright}\ {\isacharbar}\ y{\isacharhash}ys\ {\isasymRightarrow}\ xs{\isacharparenright}\ {\isacharequal}\ xs{\isachardoublequoteclose}\isanewline
%
\isadelimproof
%
\endisadelimproof
%
\isatagproof
\isacommand{apply}\isamarkupfalse%
{\isacharparenleft}case{\isacharunderscore}tac\ xs{\isacharparenright}%
\begin{isamarkuptxt}%
\noindent
results in the proof state
\begin{isabelle}%
\ {\isadigit{1}}{\isachardot}\ xs\ {\isacharequal}\ {\isacharbrackleft}{\isacharbrackright}\ {\isasymLongrightarrow}\ {\isacharparenleft}case\ xs\ of\ {\isacharbrackleft}{\isacharbrackright}\ {\isasymRightarrow}\ {\isacharbrackleft}{\isacharbrackright}\ {\isacharbar}\ y\ {\isacharhash}\ ys\ {\isasymRightarrow}\ xs{\isacharparenright}\ {\isacharequal}\ xs\isanewline
\ {\isadigit{2}}{\isachardot}\ {\isasymAnd}a\ list{\isachardot}\isanewline
\isaindent{\ {\isadigit{2}}{\isachardot}\ \ \ \ }xs\ {\isacharequal}\ a\ {\isacharhash}\ list\ {\isasymLongrightarrow}\ {\isacharparenleft}case\ xs\ of\ {\isacharbrackleft}{\isacharbrackright}\ {\isasymRightarrow}\ {\isacharbrackleft}{\isacharbrackright}\ {\isacharbar}\ y\ {\isacharhash}\ ys\ {\isasymRightarrow}\ xs{\isacharparenright}\ {\isacharequal}\ xs%
\end{isabelle}
which is solved automatically:%
\end{isamarkuptxt}%
\isamarkuptrue%
\isacommand{apply}\isamarkupfalse%
{\isacharparenleft}auto{\isacharparenright}%
\endisatagproof
{\isafoldproof}%
%
\isadelimproof
%
\endisadelimproof
%
\begin{isamarkuptext}%
Note that we do not need to give a lemma a name if we do not intend to refer
to it explicitly in the future.
Other basic laws about a datatype are applied automatically during
simplification, so no special methods are provided for them.

\begin{warn}
  Induction is only allowed on free (or \isasymAnd-bound) variables that
  should not occur among the assumptions of the subgoal; see
  \S\ref{sec:ind-var-in-prems} for details. Case distinction
  (\isa{case{\isacharunderscore}tac}) works for arbitrary terms, which need to be
  quoted if they are non-atomic. However, apart from \isa{{\isasymAnd}}-bound
  variables, the terms must not contain variables that are bound outside.
  For example, given the goal \isa{{\isasymforall}xs{\isachardot}\ xs\ {\isacharequal}\ {\isacharbrackleft}{\isacharbrackright}\ {\isasymor}\ {\isacharparenleft}{\isasymexists}y\ ys{\isachardot}\ xs\ {\isacharequal}\ y\ {\isacharhash}\ ys{\isacharparenright}},
  \isa{case{\isacharunderscore}tac\ xs} will not work as expected because Isabelle interprets
  the \isa{xs} as a new free variable distinct from the bound
  \isa{xs} in the goal.
\end{warn}%
\end{isamarkuptext}%
\isamarkuptrue%
%
\isadelimtheory
%
\endisadelimtheory
%
\isatagtheory
%
\endisatagtheory
{\isafoldtheory}%
%
\isadelimtheory
%
\endisadelimtheory
\end{isabellebody}%
%%% Local Variables:
%%% mode: latex
%%% TeX-master: "root"
%%% End:
