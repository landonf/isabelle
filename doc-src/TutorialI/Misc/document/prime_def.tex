%
\begin{isabellebody}%
\def\isabellecontext{prime{\isaliteral{5F}{\isacharunderscore}}def}%
%
\isadelimtheory
%
\endisadelimtheory
%
\isatagtheory
%
\endisatagtheory
{\isafoldtheory}%
%
\isadelimtheory
%
\endisadelimtheory
%
\begin{isamarkuptext}%
\begin{warn}
A common mistake when writing definitions is to introduce extra free
variables on the right-hand side.  Consider the following, flawed definition
(where \isa{dvd} means ``divides''):
\begin{isabelle}%
\ \ \ \ \ {\isaliteral{22}{\isachardoublequote}}prime\ p\ {\isaliteral{5C3C65717569763E}{\isasymequiv}}\ {\isadigit{1}}\ {\isaliteral{3C}{\isacharless}}\ p\ {\isaliteral{5C3C616E643E}{\isasymand}}\ {\isaliteral{28}{\isacharparenleft}}m\ dvd\ p\ {\isaliteral{5C3C6C6F6E6772696768746172726F773E}{\isasymlongrightarrow}}\ m\ {\isaliteral{3D}{\isacharequal}}\ {\isadigit{1}}\ {\isaliteral{5C3C6F723E}{\isasymor}}\ m\ {\isaliteral{3D}{\isacharequal}}\ p{\isaliteral{29}{\isacharparenright}}{\isaliteral{22}{\isachardoublequote}}%
\end{isabelle}
\par\noindent\hangindent=0pt
Isabelle rejects this ``definition'' because of the extra \isa{m} on the
right-hand side, which would introduce an inconsistency (why?). 
The correct version is
\begin{isabelle}%
\ \ \ \ \ {\isaliteral{22}{\isachardoublequote}}prime\ p\ {\isaliteral{5C3C65717569763E}{\isasymequiv}}\ {\isadigit{1}}\ {\isaliteral{3C}{\isacharless}}\ p\ {\isaliteral{5C3C616E643E}{\isasymand}}\ {\isaliteral{28}{\isacharparenleft}}{\isaliteral{5C3C666F72616C6C3E}{\isasymforall}}m{\isaliteral{2E}{\isachardot}}\ m\ dvd\ p\ {\isaliteral{5C3C6C6F6E6772696768746172726F773E}{\isasymlongrightarrow}}\ m\ {\isaliteral{3D}{\isacharequal}}\ {\isadigit{1}}\ {\isaliteral{5C3C6F723E}{\isasymor}}\ m\ {\isaliteral{3D}{\isacharequal}}\ p{\isaliteral{29}{\isacharparenright}}{\isaliteral{22}{\isachardoublequote}}%
\end{isabelle}
\end{warn}%
\end{isamarkuptext}%
\isamarkuptrue%
%
\isadelimtheory
%
\endisadelimtheory
%
\isatagtheory
%
\endisatagtheory
{\isafoldtheory}%
%
\isadelimtheory
%
\endisadelimtheory
\end{isabellebody}%
%%% Local Variables:
%%% mode: latex
%%% TeX-master: "root"
%%% End:
