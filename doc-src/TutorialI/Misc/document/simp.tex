%
\begin{isabellebody}%
\def\isabellecontext{simp}%
%
\isamarkupsubsubsection{Simplification rules%
}
%
\begin{isamarkuptext}%
\indexbold{simplification rule}
To facilitate simplification, theorems can be declared to be simplification
rules (by the attribute \isa{{\isacharbrackleft}simp{\isacharbrackright}}\index{*simp
  (attribute)}), in which case proofs by simplification make use of these
rules automatically. In addition the constructs \isacommand{datatype} and
\isacommand{primrec} (and a few others) invisibly declare useful
simplification rules. Explicit definitions are \emph{not} declared
simplification rules automatically!

Not merely equations but pretty much any theorem can become a simplification
rule. The simplifier will try to make sense of it.  For example, a theorem
\isa{{\isasymnot}\ P} is automatically turned into \isa{P\ {\isacharequal}\ False}. The details
are explained in \S\ref{sec:SimpHow}.

The simplification attribute of theorems can be turned on and off as follows:
\begin{quote}
\isacommand{declare} \textit{theorem-name}\isa{{\isacharbrackleft}simp{\isacharbrackright}}\\
\isacommand{declare} \textit{theorem-name}\isa{{\isacharbrackleft}simp\ del{\isacharbrackright}}
\end{quote}
As a rule of thumb, equations that really simplify (like \isa{rev\ {\isacharparenleft}rev\ xs{\isacharparenright}\ {\isacharequal}\ xs} and \isa{xs\ {\isacharat}\ {\isacharbrackleft}{\isacharbrackright}\ {\isacharequal}\ xs}) should be made simplification
rules.  Those of a more specific nature (e.g.\ distributivity laws, which
alter the structure of terms considerably) should only be used selectively,
i.e.\ they should not be default simplification rules.  Conversely, it may
also happen that a simplification rule needs to be disabled in certain
proofs.  Frequent changes in the simplification status of a theorem may
indicate a badly designed theory.
\begin{warn}
  Simplification may run forever, for example if both $f(x) = g(x)$ and
  $g(x) = f(x)$ are simplification rules. It is the user's responsibility not
  to include simplification rules that can lead to nontermination, either on
  their own or in combination with other simplification rules.
\end{warn}%
\end{isamarkuptext}%
%
\isamarkupsubsubsection{The simplification method%
}
%
\begin{isamarkuptext}%
\index{*simp (method)|bold}
The general format of the simplification method is
\begin{quote}
\isa{simp} \textit{list of modifiers}
\end{quote}
where the list of \emph{modifiers} fine tunes the behaviour and may
be empty. Most if not all of the proofs seen so far could have been performed
with \isa{simp} instead of \isa{auto}, except that \isa{simp} attacks
only the first subgoal and may thus need to be repeated---use
\isaindex{simp_all} to simplify all subgoals.
Note that \isa{simp} fails if nothing changes.%
\end{isamarkuptext}%
%
\isamarkupsubsubsection{Adding and deleting simplification rules%
}
%
\begin{isamarkuptext}%
If a certain theorem is merely needed in a few proofs by simplification,
we do not need to make it a global simplification rule. Instead we can modify
the set of simplification rules used in a simplification step by adding rules
to it and/or deleting rules from it. The two modifiers for this are
\begin{quote}
\isa{add{\isacharcolon}} \textit{list of theorem names}\\
\isa{del{\isacharcolon}} \textit{list of theorem names}
\end{quote}
In case you want to use only a specific list of theorems and ignore all
others:
\begin{quote}
\isa{only{\isacharcolon}} \textit{list of theorem names}
\end{quote}%
\end{isamarkuptext}%
%
\isamarkupsubsubsection{Assumptions%
}
%
\begin{isamarkuptext}%
\index{simplification!with/of assumptions}
By default, assumptions are part of the simplification process: they are used
as simplification rules and are simplified themselves. For example:%
\end{isamarkuptext}%
\isacommand{lemma}\ {\isachardoublequote}{\isasymlbrakk}\ xs\ {\isacharat}\ zs\ {\isacharequal}\ ys\ {\isacharat}\ xs{\isacharsemicolon}\ {\isacharbrackleft}{\isacharbrackright}\ {\isacharat}\ xs\ {\isacharequal}\ {\isacharbrackleft}{\isacharbrackright}\ {\isacharat}\ {\isacharbrackleft}{\isacharbrackright}\ {\isasymrbrakk}\ {\isasymLongrightarrow}\ ys\ {\isacharequal}\ zs{\isachardoublequote}\isanewline
\isacommand{apply}\ simp\isanewline
\isacommand{done}%
\begin{isamarkuptext}%
\noindent
The second assumption simplifies to \isa{xs\ {\isacharequal}\ {\isacharbrackleft}{\isacharbrackright}}, which in turn
simplifies the first assumption to \isa{zs\ {\isacharequal}\ ys}, thus reducing the
conclusion to \isa{ys\ {\isacharequal}\ ys} and hence to \isa{True}.

In some cases this may be too much of a good thing and may lead to
nontermination:%
\end{isamarkuptext}%
\isacommand{lemma}\ {\isachardoublequote}{\isasymforall}x{\isachardot}\ f\ x\ {\isacharequal}\ g\ {\isacharparenleft}f\ {\isacharparenleft}g\ x{\isacharparenright}{\isacharparenright}\ {\isasymLongrightarrow}\ f\ {\isacharbrackleft}{\isacharbrackright}\ {\isacharequal}\ f\ {\isacharbrackleft}{\isacharbrackright}\ {\isacharat}\ {\isacharbrackleft}{\isacharbrackright}{\isachardoublequote}%
\begin{isamarkuptxt}%
\noindent
cannot be solved by an unmodified application of \isa{simp} because the
simplification rule \isa{f\ x\ {\isacharequal}\ g\ {\isacharparenleft}f\ {\isacharparenleft}g\ x{\isacharparenright}{\isacharparenright}} extracted from the assumption
does not terminate. Isabelle notices certain simple forms of
nontermination but not this one. The problem can be circumvented by
explicitly telling the simplifier to ignore the assumptions:%
\end{isamarkuptxt}%
\isacommand{apply}{\isacharparenleft}simp\ {\isacharparenleft}no{\isacharunderscore}asm{\isacharparenright}{\isacharparenright}\isanewline
\isacommand{done}%
\begin{isamarkuptext}%
\noindent
There are three options that influence the treatment of assumptions:
\begin{description}
\item[\isa{{\isacharparenleft}no{\isacharunderscore}asm{\isacharparenright}}]\indexbold{*no_asm}
 means that assumptions are completely ignored.
\item[\isa{{\isacharparenleft}no{\isacharunderscore}asm{\isacharunderscore}simp{\isacharparenright}}]\indexbold{*no_asm_simp}
 means that the assumptions are not simplified but
  are used in the simplification of the conclusion.
\item[\isa{{\isacharparenleft}no{\isacharunderscore}asm{\isacharunderscore}use{\isacharparenright}}]\indexbold{*no_asm_use}
 means that the assumptions are simplified but are not
  used in the simplification of each other or the conclusion.
\end{description}
Both \isa{{\isacharparenleft}no{\isacharunderscore}asm{\isacharunderscore}simp{\isacharparenright}} and \isa{{\isacharparenleft}no{\isacharunderscore}asm{\isacharunderscore}use{\isacharparenright}} run forever on
the problematic subgoal above.
Note that only one of the options is allowed, and it must precede all
other arguments.%
\end{isamarkuptext}%
%
\isamarkupsubsubsection{Rewriting with definitions%
}
%
\begin{isamarkuptext}%
\index{simplification!with definitions}
Constant definitions (\S\ref{sec:ConstDefinitions}) can
be used as simplification rules, but by default they are not.  Hence the
simplifier does not expand them automatically, just as it should be:
definitions are introduced for the purpose of abbreviating complex
concepts. Of course we need to expand the definitions initially to derive
enough lemmas that characterize the concept sufficiently for us to forget the
original definition. For example, given%
\end{isamarkuptext}%
\isacommand{constdefs}\ xor\ {\isacharcolon}{\isacharcolon}\ {\isachardoublequote}bool\ {\isasymRightarrow}\ bool\ {\isasymRightarrow}\ bool{\isachardoublequote}\isanewline
\ \ \ \ \ \ \ \ \ {\isachardoublequote}xor\ A\ B\ {\isasymequiv}\ {\isacharparenleft}A\ {\isasymand}\ {\isasymnot}B{\isacharparenright}\ {\isasymor}\ {\isacharparenleft}{\isasymnot}A\ {\isasymand}\ B{\isacharparenright}{\isachardoublequote}%
\begin{isamarkuptext}%
\noindent
we may want to prove%
\end{isamarkuptext}%
\isacommand{lemma}\ {\isachardoublequote}xor\ A\ {\isacharparenleft}{\isasymnot}A{\isacharparenright}{\isachardoublequote}%
\begin{isamarkuptxt}%
\noindent
Typically, the opening move consists in \emph{unfolding} the definition(s), which we need to
get started, but nothing else:\indexbold{*unfold}\indexbold{definition!unfolding}%
\end{isamarkuptxt}%
\isacommand{apply}{\isacharparenleft}simp\ only{\isacharcolon}xor{\isacharunderscore}def{\isacharparenright}%
\begin{isamarkuptxt}%
\noindent
In this particular case, the resulting goal
\begin{isabelle}%
\ {\isadigit{1}}{\isachardot}\ A\ {\isasymand}\ {\isasymnot}\ {\isasymnot}\ A\ {\isasymor}\ {\isasymnot}\ A\ {\isasymand}\ {\isasymnot}\ A%
\end{isabelle}
can be proved by simplification. Thus we could have proved the lemma outright by%
\end{isamarkuptxt}%
\isacommand{apply}{\isacharparenleft}simp\ add{\isacharcolon}\ xor{\isacharunderscore}def{\isacharparenright}%
\begin{isamarkuptext}%
\noindent
Of course we can also unfold definitions in the middle of a proof.

You should normally not turn a definition permanently into a simplification
rule because this defeats the whole purpose of an abbreviation.

\begin{warn}
  If you have defined $f\,x\,y~\isasymequiv~t$ then you can only expand
  occurrences of $f$ with at least two arguments. Thus it is safer to define
  $f$~\isasymequiv~\isasymlambda$x\,y.\;t$.
\end{warn}%
\end{isamarkuptext}%
%
\isamarkupsubsubsection{Simplifying let-expressions%
}
%
\begin{isamarkuptext}%
\index{simplification!of let-expressions}
Proving a goal containing \isaindex{let}-expressions almost invariably
requires the \isa{let}-con\-structs to be expanded at some point. Since
\isa{let}\ldots\isa{=}\ldots\isa{in}{\ldots} is just syntactic sugar for a predefined constant
(called \isa{Let}), expanding \isa{let}-constructs means rewriting with
\isa{Let{\isacharunderscore}def}:%
\end{isamarkuptext}%
\isacommand{lemma}\ {\isachardoublequote}{\isacharparenleft}let\ xs\ {\isacharequal}\ {\isacharbrackleft}{\isacharbrackright}\ in\ xs{\isacharat}ys{\isacharat}xs{\isacharparenright}\ {\isacharequal}\ ys{\isachardoublequote}\isanewline
\isacommand{apply}{\isacharparenleft}simp\ add{\isacharcolon}\ Let{\isacharunderscore}def{\isacharparenright}\isanewline
\isacommand{done}%
\begin{isamarkuptext}%
If, in a particular context, there is no danger of a combinatorial explosion
of nested \isa{let}s one could even simlify with \isa{Let{\isacharunderscore}def} by
default:%
\end{isamarkuptext}%
\isacommand{declare}\ Let{\isacharunderscore}def\ {\isacharbrackleft}simp{\isacharbrackright}%
\isamarkupsubsubsection{Conditional equations%
}
%
\begin{isamarkuptext}%
So far all examples of rewrite rules were equations. The simplifier also
accepts \emph{conditional} equations, for example%
\end{isamarkuptext}%
\isacommand{lemma}\ hd{\isacharunderscore}Cons{\isacharunderscore}tl{\isacharbrackleft}simp{\isacharbrackright}{\isacharcolon}\ {\isachardoublequote}xs\ {\isasymnoteq}\ {\isacharbrackleft}{\isacharbrackright}\ \ {\isasymLongrightarrow}\ \ hd\ xs\ {\isacharhash}\ tl\ xs\ {\isacharequal}\ xs{\isachardoublequote}\isanewline
\isacommand{apply}{\isacharparenleft}case{\isacharunderscore}tac\ xs{\isacharcomma}\ simp{\isacharcomma}\ simp{\isacharparenright}\isanewline
\isacommand{done}%
\begin{isamarkuptext}%
\noindent
Note the use of ``\ttindexboldpos{,}{$Isar}'' to string together a
sequence of methods. Assuming that the simplification rule
\isa{{\isacharparenleft}rev\ xs\ {\isacharequal}\ {\isacharbrackleft}{\isacharbrackright}{\isacharparenright}\ {\isacharequal}\ {\isacharparenleft}xs\ {\isacharequal}\ {\isacharbrackleft}{\isacharbrackright}{\isacharparenright}}
is present as well,
the lemma below is proved by plain simplification:%
\end{isamarkuptext}%
\isacommand{lemma}\ {\isachardoublequote}xs\ {\isasymnoteq}\ {\isacharbrackleft}{\isacharbrackright}\ {\isasymLongrightarrow}\ hd{\isacharparenleft}rev\ xs{\isacharparenright}\ {\isacharhash}\ tl{\isacharparenleft}rev\ xs{\isacharparenright}\ {\isacharequal}\ rev\ xs{\isachardoublequote}%
\begin{isamarkuptext}%
\noindent
The conditional equation \isa{hd{\isacharunderscore}Cons{\isacharunderscore}tl} above
can simplify \isa{hd\ {\isacharparenleft}rev\ xs{\isacharparenright}\ {\isacharhash}\ tl\ {\isacharparenleft}rev\ xs{\isacharparenright}} to \isa{rev\ xs}
because the corresponding precondition \isa{rev\ xs\ {\isasymnoteq}\ {\isacharbrackleft}{\isacharbrackright}}
simplifies to \isa{xs\ {\isasymnoteq}\ {\isacharbrackleft}{\isacharbrackright}}, which is exactly the local
assumption of the subgoal.%
\end{isamarkuptext}%
%
\isamarkupsubsubsection{Automatic case splits%
}
%
\begin{isamarkuptext}%
\indexbold{case splits}\index{*split (method, attr.)|(}
Goals containing \isa{if}-expressions are usually proved by case
distinction on the condition of the \isa{if}. For example the goal%
\end{isamarkuptext}%
\isacommand{lemma}\ {\isachardoublequote}{\isasymforall}xs{\isachardot}\ if\ xs\ {\isacharequal}\ {\isacharbrackleft}{\isacharbrackright}\ then\ rev\ xs\ {\isacharequal}\ {\isacharbrackleft}{\isacharbrackright}\ else\ rev\ xs\ {\isasymnoteq}\ {\isacharbrackleft}{\isacharbrackright}{\isachardoublequote}%
\begin{isamarkuptxt}%
\noindent
can be split by a special method \isa{split}:%
\end{isamarkuptxt}%
\isacommand{apply}{\isacharparenleft}split\ split{\isacharunderscore}if{\isacharparenright}%
\begin{isamarkuptxt}%
\noindent
\begin{isabelle}%
\ {\isadigit{1}}{\isachardot}\ {\isasymforall}xs{\isachardot}\ {\isacharparenleft}xs\ {\isacharequal}\ {\isacharbrackleft}{\isacharbrackright}\ {\isasymlongrightarrow}\ rev\ xs\ {\isacharequal}\ {\isacharbrackleft}{\isacharbrackright}{\isacharparenright}\ {\isasymand}\ {\isacharparenleft}xs\ {\isasymnoteq}\ {\isacharbrackleft}{\isacharbrackright}\ {\isasymlongrightarrow}\ rev\ xs\ {\isasymnoteq}\ {\isacharbrackleft}{\isacharbrackright}{\isacharparenright}%
\end{isabelle}
where \isaindexbold{split_if} is a theorem that expresses splitting of
\isa{if}s. Because
case-splitting on \isa{if}s is almost always the right proof strategy, the
simplifier performs it automatically. Try \isacommand{apply}\isa{{\isacharparenleft}simp{\isacharparenright}}
on the initial goal above.

This splitting idea generalizes from \isa{if} to \isaindex{case}:%
\end{isamarkuptxt}%
\isacommand{lemma}\ {\isachardoublequote}{\isacharparenleft}case\ xs\ of\ {\isacharbrackleft}{\isacharbrackright}\ {\isasymRightarrow}\ zs\ {\isacharbar}\ y{\isacharhash}ys\ {\isasymRightarrow}\ y{\isacharhash}{\isacharparenleft}ys{\isacharat}zs{\isacharparenright}{\isacharparenright}\ {\isacharequal}\ xs{\isacharat}zs{\isachardoublequote}\isanewline
\isacommand{apply}{\isacharparenleft}split\ list{\isachardot}split{\isacharparenright}%
\begin{isamarkuptxt}%
\begin{isabelle}%
\ {\isadigit{1}}{\isachardot}\ {\isacharparenleft}xs\ {\isacharequal}\ {\isacharbrackleft}{\isacharbrackright}\ {\isasymlongrightarrow}\ zs\ {\isacharequal}\ xs\ {\isacharat}\ zs{\isacharparenright}\ {\isasymand}\isanewline
\ \ \ \ {\isacharparenleft}{\isasymforall}a\ list{\isachardot}\ xs\ {\isacharequal}\ a\ {\isacharhash}\ list\ {\isasymlongrightarrow}\ a\ {\isacharhash}\ list\ {\isacharat}\ zs\ {\isacharequal}\ xs\ {\isacharat}\ zs{\isacharparenright}%
\end{isabelle}
In contrast to \isa{if}-expressions, the simplifier does not split
\isa{case}-expressions by default because this can lead to nontermination
in case of recursive datatypes. Therefore the simplifier has a modifier
\isa{split} for adding further splitting rules explicitly. This means the
above lemma can be proved in one step by%
\end{isamarkuptxt}%
\isacommand{apply}{\isacharparenleft}simp\ split{\isacharcolon}\ list{\isachardot}split{\isacharparenright}%
\begin{isamarkuptext}%
\noindent
whereas \isacommand{apply}\isa{{\isacharparenleft}simp{\isacharparenright}} alone will not succeed.

In general, every datatype $t$ comes with a theorem
$t$\isa{{\isachardot}split} which can be declared to be a \bfindex{split rule} either
locally as above, or by giving it the \isa{split} attribute globally:%
\end{isamarkuptext}%
\isacommand{declare}\ list{\isachardot}split\ {\isacharbrackleft}split{\isacharbrackright}%
\begin{isamarkuptext}%
\noindent
The \isa{split} attribute can be removed with the \isa{del} modifier,
either locally%
\end{isamarkuptext}%
\isacommand{apply}{\isacharparenleft}simp\ split\ del{\isacharcolon}\ split{\isacharunderscore}if{\isacharparenright}%
\begin{isamarkuptext}%
\noindent
or globally:%
\end{isamarkuptext}%
\isacommand{declare}\ list{\isachardot}split\ {\isacharbrackleft}split\ del{\isacharbrackright}%
\begin{isamarkuptext}%
In polished proofs the \isa{split} method is rarely used on its own
but always as part of the simplifier. However, if a goal contains
multiple splittable constructs, the \isa{split} method can be
helpful in selectively exploring the effects of splitting.

The above split rules intentionally only affect the conclusion of a
subgoal.  If you want to split an \isa{if} or \isa{case}-expression in
the assumptions, you have to apply \isa{split{\isacharunderscore}if{\isacharunderscore}asm} or
$t$\isa{{\isachardot}split{\isacharunderscore}asm}:%
\end{isamarkuptext}%
\isacommand{lemma}\ {\isachardoublequote}if\ xs\ {\isacharequal}\ {\isacharbrackleft}{\isacharbrackright}\ then\ ys\ {\isasymnoteq}\ {\isacharbrackleft}{\isacharbrackright}\ else\ ys\ {\isacharequal}\ {\isacharbrackleft}{\isacharbrackright}\ {\isasymLongrightarrow}\ xs\ {\isacharat}\ ys\ {\isasymnoteq}\ {\isacharbrackleft}{\isacharbrackright}{\isachardoublequote}\isanewline
\isacommand{apply}{\isacharparenleft}split\ split{\isacharunderscore}if{\isacharunderscore}asm{\isacharparenright}%
\begin{isamarkuptxt}%
\noindent
In contrast to splitting the conclusion, this actually creates two
separate subgoals (which are solved by \isa{simp{\isacharunderscore}all}):
\begin{isabelle}%
\ {\isadigit{1}}{\isachardot}\ {\isasymlbrakk}xs\ {\isacharequal}\ {\isacharbrackleft}{\isacharbrackright}{\isacharsemicolon}\ ys\ {\isasymnoteq}\ {\isacharbrackleft}{\isacharbrackright}{\isasymrbrakk}\ {\isasymLongrightarrow}\ xs\ {\isacharat}\ ys\ {\isasymnoteq}\ {\isacharbrackleft}{\isacharbrackright}\isanewline
\ {\isadigit{2}}{\isachardot}\ {\isasymlbrakk}xs\ {\isasymnoteq}\ {\isacharbrackleft}{\isacharbrackright}{\isacharsemicolon}\ ys\ {\isacharequal}\ {\isacharbrackleft}{\isacharbrackright}{\isasymrbrakk}\ {\isasymLongrightarrow}\ xs\ {\isacharat}\ ys\ {\isasymnoteq}\ {\isacharbrackleft}{\isacharbrackright}%
\end{isabelle}
If you need to split both in the assumptions and the conclusion,
use $t$\isa{{\isachardot}splits} which subsumes $t$\isa{{\isachardot}split} and
$t$\isa{{\isachardot}split{\isacharunderscore}asm}. Analogously, there is \isa{if{\isacharunderscore}splits}.

\begin{warn}
  The simplifier merely simplifies the condition of an \isa{if} but not the
  \isa{then} or \isa{else} parts. The latter are simplified only after the
  condition reduces to \isa{True} or \isa{False}, or after splitting. The
  same is true for \isaindex{case}-expressions: only the selector is
  simplified at first, until either the expression reduces to one of the
  cases or it is split.
\end{warn}\index{*split (method, attr.)|)}%
\end{isamarkuptxt}%
%
\isamarkupsubsubsection{Arithmetic%
}
%
\begin{isamarkuptext}%
\index{arithmetic}
The simplifier routinely solves a small class of linear arithmetic formulae
(over type \isa{nat} and other numeric types): it only takes into account
assumptions and conclusions that are relations
($=$, $\le$, $<$, possibly negated) and it only knows about addition. Thus%
\end{isamarkuptext}%
\isacommand{lemma}\ {\isachardoublequote}{\isasymlbrakk}\ {\isasymnot}\ m\ {\isacharless}\ n{\isacharsemicolon}\ m\ {\isacharless}\ n{\isacharplus}{\isadigit{1}}\ {\isasymrbrakk}\ {\isasymLongrightarrow}\ m\ {\isacharequal}\ n{\isachardoublequote}%
\begin{isamarkuptext}%
\noindent
is proved by simplification, whereas the only slightly more complex%
\end{isamarkuptext}%
\isacommand{lemma}\ {\isachardoublequote}{\isasymnot}\ m\ {\isacharless}\ n\ {\isasymand}\ m\ {\isacharless}\ n{\isacharplus}{\isadigit{1}}\ {\isasymLongrightarrow}\ m\ {\isacharequal}\ n{\isachardoublequote}%
\begin{isamarkuptext}%
\noindent
is not proved by simplification and requires \isa{arith}.%
\end{isamarkuptext}%
%
\isamarkupsubsubsection{Tracing%
}
%
\begin{isamarkuptext}%
\indexbold{tracing the simplifier}
Using the simplifier effectively may take a bit of experimentation.  Set the
\isaindexbold{trace_simp} \rmindex{flag} to get a better idea of what is going
on:%
\end{isamarkuptext}%
\isacommand{ML}\ {\isachardoublequote}set\ trace{\isacharunderscore}simp{\isachardoublequote}\isanewline
\isacommand{lemma}\ {\isachardoublequote}rev\ {\isacharbrackleft}a{\isacharbrackright}\ {\isacharequal}\ {\isacharbrackleft}{\isacharbrackright}{\isachardoublequote}\isanewline
\isacommand{apply}{\isacharparenleft}simp{\isacharparenright}%
\begin{isamarkuptext}%
\noindent
produces the trace

\begin{ttbox}\makeatother
Applying instance of rewrite rule:
rev (?x1 \# ?xs1) == rev ?xs1 @ [?x1]
Rewriting:
rev [x] == rev [] @ [x]
Applying instance of rewrite rule:
rev [] == []
Rewriting:
rev [] == []
Applying instance of rewrite rule:
[] @ ?y == ?y
Rewriting:
[] @ [x] == [x]
Applying instance of rewrite rule:
?x3 \# ?t3 = ?t3 == False
Rewriting:
[x] = [] == False
\end{ttbox}

In more complicated cases, the trace can be quite lenghty, especially since
invocations of the simplifier are often nested (e.g.\ when solving conditions
of rewrite rules). Thus it is advisable to reset it:%
\end{isamarkuptext}%
\isacommand{ML}\ {\isachardoublequote}reset\ trace{\isacharunderscore}simp{\isachardoublequote}\isanewline
\end{isabellebody}%
%%% Local Variables:
%%% mode: latex
%%% TeX-master: "root"
%%% End:
