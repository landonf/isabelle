%
\begin{isabellebody}%
\def\isabellecontext{natsum}%
%
\begin{isamarkuptext}%
\noindent
In particular, there are \isa{case}-expressions, for example
\begin{isabelle}%
\ \ \ \ \ case\ n\ of\ {\isadigit{0}}\ {\isasymRightarrow}\ {\isadigit{0}}\ {\isacharbar}\ Suc\ m\ {\isasymRightarrow}\ m%
\end{isabelle}
primitive recursion, for example%
\end{isamarkuptext}%
\isacommand{consts}\ sum\ {\isacharcolon}{\isacharcolon}\ {\isachardoublequote}nat\ {\isasymRightarrow}\ nat{\isachardoublequote}\isanewline
\isacommand{primrec}\ {\isachardoublequote}sum\ {\isadigit{0}}\ {\isacharequal}\ {\isadigit{0}}{\isachardoublequote}\isanewline
\ \ \ \ \ \ \ \ {\isachardoublequote}sum\ {\isacharparenleft}Suc\ n{\isacharparenright}\ {\isacharequal}\ Suc\ n\ {\isacharplus}\ sum\ n{\isachardoublequote}%
\begin{isamarkuptext}%
\noindent
and induction, for example%
\end{isamarkuptext}%
\isacommand{lemma}\ {\isachardoublequote}sum\ n\ {\isacharplus}\ sum\ n\ {\isacharequal}\ n{\isacharasterisk}{\isacharparenleft}Suc\ n{\isacharparenright}{\isachardoublequote}\isanewline
\isacommand{apply}{\isacharparenleft}induct{\isacharunderscore}tac\ n{\isacharparenright}\isanewline
\isacommand{apply}{\isacharparenleft}auto{\isacharparenright}\isanewline
\isacommand{done}%
\begin{isamarkuptext}%
\newcommand{\mystar}{*%
}
The usual arithmetic operations \ttindexboldpos{+}{$HOL2arithfun},
\ttindexboldpos{-}{$HOL2arithfun}, \ttindexboldpos{\mystar}{$HOL2arithfun},
\sdx{div}, \sdx{mod}, \cdx{min} and
\cdx{max} are predefined, as are the relations
\indexboldpos{\isasymle}{$HOL2arithrel} and
\ttindexboldpos{<}{$HOL2arithrel}. As usual, \isa{m\ {\isacharminus}\ n\ {\isacharequal}\ {\isadigit{0}}} if
\isa{m\ {\isacharless}\ n}. There is even a least number operation
\sdx{LEAST}\@.  For example, \isa{{\isacharparenleft}LEAST\ n{\isachardot}\ {\isadigit{1}}\ {\isacharless}\ n{\isacharparenright}\ {\isacharequal}\ {\isadigit{2}}}, although
Isabelle does not prove this automatically. Note that \isa{{\isadigit{1}}}
and \isa{{\isadigit{2}}} are available as abbreviations for the corresponding
\isa{Suc}-expressions. If you need the full set of numerals,
see~\S\ref{sec:numerals}.

\begin{warn}
  The constant \cdx{0} and the operations
  \ttindexboldpos{+}{$HOL2arithfun}, \ttindexboldpos{-}{$HOL2arithfun},
  \ttindexboldpos{\mystar}{$HOL2arithfun}, \cdx{min},
  \cdx{max}, \indexboldpos{\isasymle}{$HOL2arithrel} and
  \ttindexboldpos{<}{$HOL2arithrel} are overloaded, i.e.\ they are available
  not just for natural numbers but at other types as well.
  For example, given the goal \isa{x\ {\isacharplus}\ {\isadigit{0}}\ {\isacharequal}\ x},
  there is nothing to indicate that you are talking about natural numbers.
  Hence Isabelle can only infer that \isa{x} is of some arbitrary type where
  \isa{{\isadigit{0}}} and \isa{{\isacharplus}} are declared. As a consequence, you will be unable
  to prove the goal (although it may take you some time to realize what has
  happened if \isa{show{\isacharunderscore}types} is not set).  In this particular example,
  you need to include an explicit type constraint, for example
  \isa{x{\isacharplus}{\isadigit{0}}\ {\isacharequal}\ {\isacharparenleft}x{\isacharcolon}{\isacharcolon}nat{\isacharparenright}}. If there is enough contextual information this
  may not be necessary: \isa{Suc\ x\ {\isacharequal}\ x} automatically implies
  \isa{x{\isacharcolon}{\isacharcolon}nat} because \isa{Suc} is not overloaded.

  For details see \S\ref{sec:numbers} and \S\ref{sec:overloading};
  Table~\ref{tab:overloading} in the appendix shows the most important overloaded
  operations.
\end{warn}

Simple arithmetic goals are proved automatically by both \isa{auto} and the
simplification method introduced in \S\ref{sec:Simplification}.  For
example,%
\end{isamarkuptext}%
\isacommand{lemma}\ {\isachardoublequote}{\isasymlbrakk}\ {\isasymnot}\ m\ {\isacharless}\ n{\isacharsemicolon}\ m\ {\isacharless}\ n{\isacharplus}{\isadigit{1}}\ {\isasymrbrakk}\ {\isasymLongrightarrow}\ m\ {\isacharequal}\ n{\isachardoublequote}%
\begin{isamarkuptext}%
\noindent
is proved automatically. The main restriction is that only addition is taken
into account; other arithmetic operations and quantified formulae are ignored.

For more complex goals, there is the special method \methdx{arith}
(which attacks the first subgoal). It proves arithmetic goals involving the
usual logical connectives (\isa{{\isasymnot}}, \isa{{\isasymand}}, \isa{{\isasymor}},
\isa{{\isasymlongrightarrow}}), the relations \isa{{\isacharequal}}, \isa{{\isasymle}} and \isa{{\isacharless}},
and the operations
\isa{{\isacharplus}}, \isa{{\isacharminus}}, \isa{min} and \isa{max}. Technically, this is
known as the class of (quantifier free) \textbf{linear arithmetic} formulae.
For example,%
\end{isamarkuptext}%
\isacommand{lemma}\ {\isachardoublequote}min\ i\ {\isacharparenleft}max\ j\ {\isacharparenleft}k{\isacharasterisk}k{\isacharparenright}{\isacharparenright}\ {\isacharequal}\ max\ {\isacharparenleft}min\ {\isacharparenleft}k{\isacharasterisk}k{\isacharparenright}\ i{\isacharparenright}\ {\isacharparenleft}min\ i\ {\isacharparenleft}j{\isacharcolon}{\isacharcolon}nat{\isacharparenright}{\isacharparenright}{\isachardoublequote}\isanewline
\isacommand{apply}{\isacharparenleft}arith{\isacharparenright}%
\begin{isamarkuptext}%
\noindent
succeeds because \isa{k\ {\isacharasterisk}\ k} can be treated as atomic. In contrast,%
\end{isamarkuptext}%
\isacommand{lemma}\ {\isachardoublequote}n{\isacharasterisk}n\ {\isacharequal}\ n\ {\isasymLongrightarrow}\ n{\isacharequal}{\isadigit{0}}\ {\isasymor}\ n{\isacharequal}{\isadigit{1}}{\isachardoublequote}%
\begin{isamarkuptext}%
\noindent
is not even proved by \isa{arith} because the proof relies essentially
on properties of multiplication.

\begin{warn}
  The running time of \isa{arith} is exponential in the number of occurrences
  of \ttindexboldpos{-}{$HOL2arithfun}, \cdx{min} and
  \cdx{max} because they are first eliminated by case distinctions.

  \isa{arith} is incomplete even for the restricted class of
  linear arithmetic formulae. If divisibility plays a
  role, it may fail to prove a valid formula, for example
  \isa{m\ {\isacharplus}\ m\ {\isasymnoteq}\ n\ {\isacharplus}\ n\ {\isacharplus}\ {\isadigit{1}}}. Fortunately, such examples are rare in practice.
\end{warn}%
\end{isamarkuptext}%
\end{isabellebody}%
%%% Local Variables:
%%% mode: latex
%%% TeX-master: "root"
%%% End:
