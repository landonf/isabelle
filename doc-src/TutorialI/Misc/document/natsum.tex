%
\begin{isabellebody}%
%
\begin{isamarkuptext}%
\noindent
In particular, there are \isa{case}-expressions, for example
\begin{quote}

\begin{isabelle}%
case\ \mbox{n}\ of\ \isadigit{0}\ {\isasymRightarrow}\ \isadigit{0}\ {\isacharbar}\ Suc\ \mbox{m}\ {\isasymRightarrow}\ \mbox{m}
\end{isabelle}%

\end{quote}
primitive recursion, for example%
\end{isamarkuptext}%
\isacommand{consts}\ sum\ {\isacharcolon}{\isacharcolon}\ {\isachardoublequote}nat\ {\isasymRightarrow}\ nat{\isachardoublequote}\isanewline
\isacommand{primrec}\ {\isachardoublequote}sum\ \isadigit{0}\ {\isacharequal}\ \isadigit{0}{\isachardoublequote}\isanewline
\ \ \ \ \ \ \ \ {\isachardoublequote}sum\ {\isacharparenleft}Suc\ n{\isacharparenright}\ {\isacharequal}\ Suc\ n\ {\isacharplus}\ sum\ n{\isachardoublequote}%
\begin{isamarkuptext}%
\noindent
and induction, for example%
\end{isamarkuptext}%
\isacommand{lemma}\ {\isachardoublequote}sum\ n\ {\isacharplus}\ sum\ n\ {\isacharequal}\ n{\isacharasterisk}{\isacharparenleft}Suc\ n{\isacharparenright}{\isachardoublequote}\isanewline
\isacommand{apply}{\isacharparenleft}induct{\isacharunderscore}tac\ n{\isacharparenright}\isanewline
\isacommand{by}{\isacharparenleft}auto{\isacharparenright}\isanewline
\end{isabellebody}%
%%% Local Variables:
%%% mode: latex
%%% TeX-master: "root"
%%% End:
