\begin{isabelle}%
%
\begin{isamarkuptext}%
\noindent
In particular, there are \isa{case}-expressions, for example
\begin{quote}

\begin{isabelle}%
case\ \mbox{n}\ of\ 0\ {\isasymRightarrow}\ 0\ |\ Suc\ \mbox{m}\ {\isasymRightarrow}\ \mbox{m}
\end{isabelle}%

\end{quote}
primitive recursion, for example%
\end{isamarkuptext}%
\isacommand{consts}\ sum\ ::\ {"}nat\ {\isasymRightarrow}\ nat{"}\isanewline
\isacommand{primrec}\ {"}sum\ 0\ =\ 0{"}\isanewline
\ \ \ \ \ \ \ \ {"}sum\ (Suc\ n)\ =\ Suc\ n\ +\ sum\ n{"}%
\begin{isamarkuptext}%
\noindent
and induction, for example%
\end{isamarkuptext}%
\isacommand{lemma}\ {"}sum\ n\ +\ sum\ n\ =\ n*(Suc\ n){"}\isanewline
\isacommand{apply}(induct\_tac\ n)\isanewline
\isacommand{by}(auto)\isanewline
\end{isabelle}%
%%% Local Variables:
%%% mode: latex
%%% TeX-master: "root"
%%% End:
