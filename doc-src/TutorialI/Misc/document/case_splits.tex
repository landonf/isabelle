\begin{isabelle}%
%
\begin{isamarkuptext}%
Goals containing \isaindex{if}-expressions are usually proved by case
distinction on the condition of the \isa{if}. For example the goal%
\end{isamarkuptext}%
\isacommand{lemma}~{"}{\isasymforall}xs.~if~xs~=~[]~then~rev~xs~=~[]~else~rev~xs~{\isasymnoteq}~[]{"}%
\begin{isamarkuptxt}%
\noindent
can be split into
\begin{isabellepar}%
~1.~{\isasymforall}xs.~(xs~=~[]~{\isasymlongrightarrow}~rev~xs~=~[])~{\isasymand}~(xs~{\isasymnoteq}~[]~{\isasymlongrightarrow}~rev~xs~{\isasymnoteq}~[])%
\end{isabellepar}%
by a degenerate form of simplification%
\end{isamarkuptxt}%
\isacommand{apply}(simp~only:~split:~split\_if)%
\begin{isamarkuptext}%
\noindent
where no simplification rules are included (\isa{only:} is followed by the
empty list of theorems) but the rule \isaindexbold{split_if} for
splitting \isa{if}s is added (via the modifier \isa{split:}). Because
case-splitting on \isa{if}s is almost always the right proof strategy, the
simplifier performs it automatically. Try \isacommand{apply}\isa{(simp)}
on the initial goal above.

This splitting idea generalizes from \isa{if} to \isaindex{case}:%
\end{isamarkuptext}%
\isacommand{lemma}~{"}(case~xs~of~[]~{\isasymRightarrow}~zs~|~y\#ys~{\isasymRightarrow}~y\#(ys@zs))~=~xs@zs{"}%
\begin{isamarkuptxt}%
\noindent
becomes
\begin{isabellepar}%
~1.~(xs~=~[]~{\isasymlongrightarrow}~zs~=~xs~@~zs)~{\isasymand}\isanewline
~~~~({\isasymforall}a~list.~xs~=~a~\#~list~{\isasymlongrightarrow}~a~\#~list~@~zs~=~xs~@~zs)%
\end{isabellepar}%
by typing%
\end{isamarkuptxt}%
\isacommand{apply}(simp~only:~split:~list.split)%
\begin{isamarkuptext}%
\noindent
In contrast to \isa{if}-expressions, the simplifier does not split
\isa{case}-expressions by default because this can lead to nontermination
in case of recursive datatypes. Again, if the \isa{only:} modifier is
dropped, the above goal is solved,%
\end{isamarkuptext}%
\isacommand{apply}(simp~split:~list.split)\isacommand{.}%
\begin{isamarkuptext}%
\noindent%
which \isacommand{apply}\isa{(simp)} alone will not do.

In general, every datatype $t$ comes with a theorem
\isa{$t$.split} which can be declared to be a \bfindex{split rule} either
locally as above, or by giving it the \isa{split} attribute globally:%
\end{isamarkuptext}%
\isacommand{theorems}~[split]~=~list.split%
\begin{isamarkuptext}%
\noindent
The \isa{split} attribute can be removed with the \isa{del} modifier,
either locally%
\end{isamarkuptext}%
\isacommand{apply}(simp~split~del:~split\_if)%
\begin{isamarkuptext}%
\noindent
or globally:%
\end{isamarkuptext}%
\isacommand{theorems}~[split~del]~=~list.split\isanewline
\end{isabelle}%
