\begin{isabelle}%
%
\begin{isamarkuptext}%
Goals containing \isaindex{if}-expressions are usually proved by case
distinction on the condition of the \isa{if}. For example the goal%
\end{isamarkuptext}%
\isacommand{lemma}\ {\isachardoublequote}{\isasymforall}xs{\isachardot}\ if\ xs\ {\isacharequal}\ {\isacharbrackleft}{\isacharbrackright}\ then\ rev\ xs\ {\isacharequal}\ {\isacharbrackleft}{\isacharbrackright}\ else\ rev\ xs\ {\isasymnoteq}\ {\isacharbrackleft}{\isacharbrackright}{\isachardoublequote}%
\begin{isamarkuptxt}%
\noindent
can be split into
\begin{isabellepar}%
~1.~{\isasymforall}xs.~(xs~=~[]~{\isasymlongrightarrow}~rev~xs~=~[])~{\isasymand}~(xs~{\isasymnoteq}~[]~{\isasymlongrightarrow}~rev~xs~{\isasymnoteq}~[])%
\end{isabellepar}%
by a degenerate form of simplification%
\end{isamarkuptxt}%
\isacommand{apply}{\isacharparenleft}simp\ only{\isacharcolon}\ split{\isacharcolon}\ split{\isacharunderscore}if{\isacharparenright}%
\begin{isamarkuptext}%
\noindent
where no simplification rules are included (\isa{only:} is followed by the
empty list of theorems) but the rule \isaindexbold{split_if} for
splitting \isa{if}s is added (via the modifier \isa{split:}). Because
case-splitting on \isa{if}s is almost always the right proof strategy, the
simplifier performs it automatically. Try \isacommand{apply}\isa{(simp)}
on the initial goal above.

This splitting idea generalizes from \isa{if} to \isaindex{case}:%
\end{isamarkuptext}%
\isacommand{lemma}\ {\isachardoublequote}{\isacharparenleft}case\ xs\ of\ {\isacharbrackleft}{\isacharbrackright}\ {\isasymRightarrow}\ zs\ {\isacharbar}\ y{\isacharhash}ys\ {\isasymRightarrow}\ y{\isacharhash}{\isacharparenleft}ys{\isacharat}zs{\isacharparenright}{\isacharparenright}\ {\isacharequal}\ xs{\isacharat}zs{\isachardoublequote}%
\begin{isamarkuptxt}%
\noindent
becomes
\begin{isabellepar}%
~1.~(xs~=~[]~{\isasymlongrightarrow}~zs~=~xs~@~zs)~{\isasymand}\isanewline
~~~~({\isasymforall}a~list.~xs~=~a~\#~list~{\isasymlongrightarrow}~a~\#~list~@~zs~=~xs~@~zs)%
\end{isabellepar}%
by typing%
\end{isamarkuptxt}%
\isacommand{apply}{\isacharparenleft}simp\ only{\isacharcolon}\ split{\isacharcolon}\ list{\isachardot}split{\isacharparenright}%
\begin{isamarkuptext}%
\noindent
In contrast to \isa{if}-expressions, the simplifier does not split
\isa{case}-expressions by default because this can lead to nontermination
in case of recursive datatypes. Again, if the \isa{only:} modifier is
dropped, the above goal is solved,%
\end{isamarkuptext}%
\isacommand{by}{\isacharparenleft}simp\ split{\isacharcolon}\ list{\isachardot}split{\isacharparenright}%
\begin{isamarkuptext}%
\noindent%
which \isacommand{apply}\isa{(simp)} alone will not do.

In general, every datatype $t$ comes with a theorem
\isa{$t$.split} which can be declared to be a \bfindex{split rule} either
locally as above, or by giving it the \isa{split} attribute globally:%
\end{isamarkuptext}%
\isacommand{lemmas}\ {\isacharbrackleft}split{\isacharbrackright}\ {\isacharequal}\ list{\isachardot}split%
\begin{isamarkuptext}%
\noindent
The \isa{split} attribute can be removed with the \isa{del} modifier,
either locally%
\end{isamarkuptext}%
\isacommand{apply}{\isacharparenleft}simp\ split\ del{\isacharcolon}\ split{\isacharunderscore}if{\isacharparenright}%
\begin{isamarkuptext}%
\noindent
or globally:%
\end{isamarkuptext}%
\isacommand{lemmas}\ {\isacharbrackleft}split\ del{\isacharbrackright}\ {\isacharequal}\ list{\isachardot}split%
\begin{isamarkuptext}%
The above split rules intentionally only affect the conclusion of a
subgoal.  If you want to split an \isa{if} or \isa{case}-expression in
the assumptions, you have to apply \isa{split\_if\_asm} or $t$\isa{.split_asm}:%
\end{isamarkuptext}%
\isacommand{lemma}\ {\isachardoublequote}if\ xs\ {\isacharequal}\ {\isacharbrackleft}{\isacharbrackright}\ then\ ys\ {\isachartilde}{\isacharequal}\ {\isacharbrackleft}{\isacharbrackright}\ else\ ys\ {\isacharequal}\ {\isacharbrackleft}{\isacharbrackright}\ {\isacharequal}{\isacharequal}{\isachargreater}\ xs\ {\isacharat}\ ys\ {\isachartilde}{\isacharequal}\ {\isacharbrackleft}{\isacharbrackright}{\isachardoublequote}\isanewline
\isacommand{apply}{\isacharparenleft}simp\ only{\isacharcolon}\ split{\isacharcolon}\ split{\isacharunderscore}if{\isacharunderscore}asm{\isacharparenright}%
\begin{isamarkuptxt}%
\noindent
In contrast to splitting the conclusion, this actually creates two
separate subgoals (which are solved by \isa{simp\_all}):
\begin{isabelle}
\ \isadigit{1}{\isachardot}\ {\isasymlbrakk}\mbox{xs}\ {\isacharequal}\ {\isacharbrackleft}{\isacharbrackright}{\isacharsemicolon}\ \mbox{ys}\ {\isasymnoteq}\ {\isacharbrackleft}{\isacharbrackright}{\isasymrbrakk}\ {\isasymLongrightarrow}\ {\isacharbrackleft}{\isacharbrackright}\ {\isacharat}\ \mbox{ys}\ {\isasymnoteq}\ {\isacharbrackleft}{\isacharbrackright}\isanewline
\ \isadigit{2}{\isachardot}\ {\isasymlbrakk}\mbox{xs}\ {\isasymnoteq}\ {\isacharbrackleft}{\isacharbrackright}{\isacharsemicolon}\ \mbox{ys}\ {\isacharequal}\ {\isacharbrackleft}{\isacharbrackright}{\isasymrbrakk}\ {\isasymLongrightarrow}\ \mbox{xs}\ {\isacharat}\ {\isacharbrackleft}{\isacharbrackright}\ {\isasymnoteq}\ {\isacharbrackleft}{\isacharbrackright}
\end{isabelle}
If you need to split both in the assumptions and the conclusion,
use $t$\isa{.splits} which subsumes $t$\isa{.split} and $t$\isa{.split_asm}.%
\end{isamarkuptxt}%
\end{isabelle}%
%%% Local Variables:
%%% mode: latex
%%% TeX-master: "root"
%%% End:
