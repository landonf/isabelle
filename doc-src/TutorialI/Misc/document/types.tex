%
\begin{isabellebody}%
\def\isabellecontext{types}%
\isacommand{types}\ number\ \ \ \ \ \ \ {\isacharequal}\ nat\isanewline
\ \ \ \ \ \ gate\ \ \ \ \ \ \ \ \ {\isacharequal}\ {\isachardoublequote}bool\ {\isasymRightarrow}\ bool\ {\isasymRightarrow}\ bool{\isachardoublequote}\isanewline
\ \ \ \ \ \ {\isacharparenleft}{\isacharprime}a{\isacharcomma}{\isacharprime}b{\isacharparenright}alist\ {\isacharequal}\ {\isachardoublequote}{\isacharparenleft}{\isacharprime}a\ {\isasymtimes}\ {\isacharprime}b{\isacharparenright}list{\isachardoublequote}%
\begin{isamarkuptext}%
\noindent\indexbold{*types}%
Internally all synonyms are fully expanded.  As a consequence Isabelle's
output never contains synonyms.  Their main purpose is to improve the
readability of theories.  Synonyms can be used just like any other
type:%
\end{isamarkuptext}%
\isacommand{consts}\ nand\ {\isacharcolon}{\isacharcolon}\ gate\isanewline
\ \ \ \ \ \ \ exor\ {\isacharcolon}{\isacharcolon}\ gate%
\begin{isamarkuptext}%
\subsection{Constant definitions}
\label{sec:ConstDefinitions}
\indexbold{definition}

The above constants \isa{nand} and \isa{exor} are non-recursive and can
therefore be defined directly by%
\end{isamarkuptext}%
\isacommand{defs}\ nand{\isacharunderscore}def{\isacharcolon}\ {\isachardoublequote}nand\ A\ B\ {\isasymequiv}\ {\isasymnot}{\isacharparenleft}A\ {\isasymand}\ B{\isacharparenright}{\isachardoublequote}\isanewline
\ \ \ \ \ exor{\isacharunderscore}def{\isacharcolon}\ {\isachardoublequote}exor\ A\ B\ {\isasymequiv}\ A\ {\isasymand}\ {\isasymnot}B\ {\isasymor}\ {\isasymnot}A\ {\isasymand}\ B{\isachardoublequote}%
\begin{isamarkuptext}%
\noindent%
where \isacommand{defs}\indexbold{*defs} is a keyword and
\isa{nand{\isacharunderscore}def} and \isa{exor{\isacharunderscore}def} are user-supplied names.
The symbol \indexboldpos{\isasymequiv}{$IsaEq} is a special form of equality
that must be used in constant definitions.
Declarations and definitions can also be merged%
\end{isamarkuptext}%
\isacommand{constdefs}\ nor\ {\isacharcolon}{\isacharcolon}\ gate\isanewline
\ \ \ \ \ \ \ \ \ {\isachardoublequote}nor\ A\ B\ {\isasymequiv}\ {\isasymnot}{\isacharparenleft}A\ {\isasymor}\ B{\isacharparenright}{\isachardoublequote}\isanewline
\ \ \ \ \ \ \ \ \ \ exor\isadigit{2}\ {\isacharcolon}{\isacharcolon}\ gate\isanewline
\ \ \ \ \ \ \ \ \ {\isachardoublequote}exor\isadigit{2}\ A\ B\ {\isasymequiv}\ {\isacharparenleft}A\ {\isasymor}\ B{\isacharparenright}\ {\isasymand}\ {\isacharparenleft}{\isasymnot}A\ {\isasymor}\ {\isasymnot}B{\isacharparenright}{\isachardoublequote}%
\begin{isamarkuptext}%
\noindent\indexbold{*constdefs}%
in which case the default name of each definition is $f$\isa{{\isacharunderscore}def}, where
$f$ is the name of the defined constant.%
\end{isamarkuptext}%
\end{isabellebody}%
%%% Local Variables:
%%% mode: latex
%%% TeX-master: "root"
%%% End:
