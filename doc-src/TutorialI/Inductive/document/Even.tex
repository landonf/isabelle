%
\begin{isabellebody}%
\def\isabellecontext{Even}%
\isanewline
\isacommand{theory}\ Even\ {\isacharequal}\ Main{\isacharcolon}\isanewline
\isanewline
\isanewline
\isacommand{consts}\ even\ {\isacharcolon}{\isacharcolon}\ {\isachardoublequote}nat\ set{\isachardoublequote}\isanewline
\isacommand{inductive}\ even\isanewline
\isakeyword{intros}\isanewline
zero{\isacharbrackleft}intro{\isacharbang}{\isacharbrackright}{\isacharcolon}\ {\isachardoublequote}{\isadigit{0}}\ {\isasymin}\ even{\isachardoublequote}\isanewline
step{\isacharbrackleft}intro{\isacharbang}{\isacharbrackright}{\isacharcolon}\ {\isachardoublequote}n\ {\isasymin}\ even\ {\isasymLongrightarrow}\ {\isacharparenleft}Suc\ {\isacharparenleft}Suc\ n{\isacharparenright}{\isacharparenright}\ {\isasymin}\ even{\isachardoublequote}%
\begin{isamarkuptext}%
An inductive definition consists of introduction rules. 

\begin{isabelle}%
\ \ \ \ \ n\ {\isasymin}\ even\ {\isasymLongrightarrow}\ Suc\ {\isacharparenleft}Suc\ n{\isacharparenright}\ {\isasymin}\ even%
\end{isabelle}
\rulename{even.step}

\begin{isabelle}%
\ \ \ \ \ {\isasymlbrakk}xa\ {\isasymin}\ even{\isacharsemicolon}\ P\ {\isadigit{0}}{\isacharsemicolon}\ {\isasymAnd}n{\isachardot}\ {\isasymlbrakk}n\ {\isasymin}\ even{\isacharsemicolon}\ P\ n{\isasymrbrakk}\ {\isasymLongrightarrow}\ P\ {\isacharparenleft}Suc\ {\isacharparenleft}Suc\ n{\isacharparenright}{\isacharparenright}{\isasymrbrakk}\ {\isasymLongrightarrow}\ P\ xa%
\end{isabelle}
\rulename{even.induct}

Attributes can be given to the introduction rules.  Here both rules are
specified as \isa{intro!}

Our first lemma states that numbers of the form $2\times k$ are even.%
\end{isamarkuptext}%
\isacommand{lemma}\ two{\isacharunderscore}times{\isacharunderscore}even{\isacharbrackleft}intro{\isacharbang}{\isacharbrackright}{\isacharcolon}\ {\isachardoublequote}{\isadigit{2}}{\isacharasterisk}k\ {\isasymin}\ even{\isachardoublequote}\isanewline
\isacommand{apply}\ {\isacharparenleft}induct\ {\isachardoublequote}k{\isachardoublequote}{\isacharparenright}%
\begin{isamarkuptxt}%
The first step is induction on the natural number \isa{k}, which leaves
two subgoals:
\begin{isabelle}%
\ {\isadigit{1}}{\isachardot}\ {\isadigit{2}}\ {\isacharasterisk}\ {\isadigit{0}}\ {\isasymin}\ even\isanewline
\ {\isadigit{2}}{\isachardot}\ {\isasymAnd}n{\isachardot}\ {\isadigit{2}}\ {\isacharasterisk}\ n\ {\isasymin}\ even\ {\isasymLongrightarrow}\ {\isadigit{2}}\ {\isacharasterisk}\ Suc\ n\ {\isasymin}\ even%
\end{isabelle}
Here \isa{auto} simplifies both subgoals so that they match the introduction
rules, which then are applied automatically.%
\end{isamarkuptxt}%
\ \isacommand{apply}\ auto\isanewline
\isacommand{done}%
\begin{isamarkuptext}%
Our goal is to prove the equivalence between the traditional definition
of even (using the divides relation) and our inductive definition.  Half of
this equivalence is trivial using the lemma just proved, whose \isa{intro!}
attribute ensures it will be applied automatically.%
\end{isamarkuptext}%
\isacommand{lemma}\ dvd{\isacharunderscore}imp{\isacharunderscore}even{\isacharcolon}\ {\isachardoublequote}{\isadigit{2}}\ dvd\ n\ {\isasymLongrightarrow}\ n\ {\isasymin}\ even{\isachardoublequote}\isanewline
\isacommand{by}\ {\isacharparenleft}auto\ simp\ add{\isacharcolon}\ dvd{\isacharunderscore}def{\isacharparenright}%
\begin{isamarkuptext}%
our first rule induction!%
\end{isamarkuptext}%
\isacommand{lemma}\ even{\isacharunderscore}imp{\isacharunderscore}dvd{\isacharcolon}\ {\isachardoublequote}n\ {\isasymin}\ even\ {\isasymLongrightarrow}\ {\isadigit{2}}\ dvd\ n{\isachardoublequote}\isanewline
\isacommand{apply}\ {\isacharparenleft}erule\ even{\isachardot}induct{\isacharparenright}%
\begin{isamarkuptxt}%
\begin{isabelle}%
\ {\isadigit{1}}{\isachardot}\ {\isadigit{2}}\ dvd\ {\isadigit{0}}\isanewline
\ {\isadigit{2}}{\isachardot}\ {\isasymAnd}n{\isachardot}\ {\isasymlbrakk}n\ {\isasymin}\ even{\isacharsemicolon}\ {\isadigit{2}}\ dvd\ n{\isasymrbrakk}\ {\isasymLongrightarrow}\ {\isadigit{2}}\ dvd\ Suc\ {\isacharparenleft}Suc\ n{\isacharparenright}%
\end{isabelle}%
\end{isamarkuptxt}%
\isacommand{apply}\ {\isacharparenleft}simp{\isacharunderscore}all\ add{\isacharcolon}\ dvd{\isacharunderscore}def{\isacharparenright}%
\begin{isamarkuptxt}%
\begin{isabelle}%
\ {\isadigit{1}}{\isachardot}\ {\isasymAnd}n{\isachardot}\ {\isasymlbrakk}n\ {\isasymin}\ even{\isacharsemicolon}\ {\isasymexists}k{\isachardot}\ n\ {\isacharequal}\ {\isadigit{2}}\ {\isacharasterisk}\ k{\isasymrbrakk}\ {\isasymLongrightarrow}\ {\isasymexists}k{\isachardot}\ Suc\ {\isacharparenleft}Suc\ n{\isacharparenright}\ {\isacharequal}\ {\isadigit{2}}\ {\isacharasterisk}\ k%
\end{isabelle}%
\end{isamarkuptxt}%
\isacommand{apply}\ clarify%
\begin{isamarkuptxt}%
\begin{isabelle}%
\ {\isadigit{1}}{\isachardot}\ {\isasymAnd}n\ k{\isachardot}\ {\isadigit{2}}\ {\isacharasterisk}\ k\ {\isasymin}\ even\ {\isasymLongrightarrow}\ {\isasymexists}ka{\isachardot}\ Suc\ {\isacharparenleft}Suc\ {\isacharparenleft}{\isadigit{2}}\ {\isacharasterisk}\ k{\isacharparenright}{\isacharparenright}\ {\isacharequal}\ {\isadigit{2}}\ {\isacharasterisk}\ ka%
\end{isabelle}%
\end{isamarkuptxt}%
\isacommand{apply}\ {\isacharparenleft}rule{\isacharunderscore}tac\ x\ {\isacharequal}\ {\isachardoublequote}Suc\ k{\isachardoublequote}\ \isakeyword{in}\ exI{\isacharcomma}\ simp{\isacharparenright}\isanewline
\isacommand{done}%
\begin{isamarkuptext}%
no iff-attribute because we don't always want to use it%
\end{isamarkuptext}%
\isacommand{theorem}\ even{\isacharunderscore}iff{\isacharunderscore}dvd{\isacharcolon}\ {\isachardoublequote}{\isacharparenleft}n\ {\isasymin}\ even{\isacharparenright}\ {\isacharequal}\ {\isacharparenleft}{\isadigit{2}}\ dvd\ n{\isacharparenright}{\isachardoublequote}\isanewline
\isacommand{by}\ {\isacharparenleft}blast\ intro{\isacharcolon}\ dvd{\isacharunderscore}imp{\isacharunderscore}even\ even{\isacharunderscore}imp{\isacharunderscore}dvd{\isacharparenright}%
\begin{isamarkuptext}%
this result ISN'T inductive...%
\end{isamarkuptext}%
\isacommand{lemma}\ Suc{\isacharunderscore}Suc{\isacharunderscore}even{\isacharunderscore}imp{\isacharunderscore}even{\isacharcolon}\ {\isachardoublequote}Suc\ {\isacharparenleft}Suc\ n{\isacharparenright}\ {\isasymin}\ even\ {\isasymLongrightarrow}\ n\ {\isasymin}\ even{\isachardoublequote}\isanewline
\isacommand{apply}\ {\isacharparenleft}erule\ even{\isachardot}induct{\isacharparenright}%
\begin{isamarkuptxt}%
\begin{isabelle}%
\ {\isadigit{1}}{\isachardot}\ n\ {\isasymin}\ even\isanewline
\ {\isadigit{2}}{\isachardot}\ {\isasymAnd}na{\isachardot}\ {\isasymlbrakk}na\ {\isasymin}\ even{\isacharsemicolon}\ n\ {\isasymin}\ even{\isasymrbrakk}\ {\isasymLongrightarrow}\ n\ {\isasymin}\ even%
\end{isabelle}%
\end{isamarkuptxt}%
\isacommand{oops}%
\begin{isamarkuptext}%
...so we need an inductive lemma...%
\end{isamarkuptext}%
\isacommand{lemma}\ even{\isacharunderscore}imp{\isacharunderscore}even{\isacharunderscore}minus{\isacharunderscore}{\isadigit{2}}{\isacharcolon}\ {\isachardoublequote}n\ {\isasymin}\ even\ {\isasymLongrightarrow}\ n\ {\isacharminus}\ {\isadigit{2}}\ {\isasymin}\ even{\isachardoublequote}\isanewline
\isacommand{apply}\ {\isacharparenleft}erule\ even{\isachardot}induct{\isacharparenright}%
\begin{isamarkuptxt}%
\begin{isabelle}%
\ {\isadigit{1}}{\isachardot}\ {\isadigit{0}}\ {\isacharminus}\ {\isadigit{2}}\ {\isasymin}\ even\isanewline
\ {\isadigit{2}}{\isachardot}\ {\isasymAnd}n{\isachardot}\ {\isasymlbrakk}n\ {\isasymin}\ even{\isacharsemicolon}\ n\ {\isacharminus}\ {\isadigit{2}}\ {\isasymin}\ even{\isasymrbrakk}\ {\isasymLongrightarrow}\ Suc\ {\isacharparenleft}Suc\ n{\isacharparenright}\ {\isacharminus}\ {\isadigit{2}}\ {\isasymin}\ even%
\end{isabelle}%
\end{isamarkuptxt}%
\isacommand{apply}\ auto\isanewline
\isacommand{done}%
\begin{isamarkuptext}%
...and prove it in a separate step%
\end{isamarkuptext}%
\isacommand{lemma}\ Suc{\isacharunderscore}Suc{\isacharunderscore}even{\isacharunderscore}imp{\isacharunderscore}even{\isacharcolon}\ {\isachardoublequote}Suc\ {\isacharparenleft}Suc\ n{\isacharparenright}\ {\isasymin}\ even\ {\isasymLongrightarrow}\ n\ {\isasymin}\ even{\isachardoublequote}\isanewline
\isacommand{by}\ {\isacharparenleft}drule\ even{\isacharunderscore}imp{\isacharunderscore}even{\isacharunderscore}minus{\isacharunderscore}{\isadigit{2}}{\isacharcomma}\ simp{\isacharparenright}\isanewline
\isanewline
\isanewline
\isacommand{lemma}\ {\isacharbrackleft}iff{\isacharbrackright}{\isacharcolon}\ {\isachardoublequote}{\isacharparenleft}{\isacharparenleft}Suc\ {\isacharparenleft}Suc\ n{\isacharparenright}{\isacharparenright}\ {\isasymin}\ even{\isacharparenright}\ {\isacharequal}\ {\isacharparenleft}n\ {\isasymin}\ even{\isacharparenright}{\isachardoublequote}\isanewline
\isacommand{by}\ {\isacharparenleft}blast\ dest{\isacharcolon}\ Suc{\isacharunderscore}Suc{\isacharunderscore}even{\isacharunderscore}imp{\isacharunderscore}even{\isacharparenright}\isanewline
\isanewline
\isacommand{end}\isanewline
\isanewline
\end{isabellebody}%
%%% Local Variables:
%%% mode: latex
%%% TeX-master: "root"
%%% End:
