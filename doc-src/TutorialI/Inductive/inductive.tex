\chapter{Inductively Defined Sets} \label{chap:inductive}
\index{inductive definition|(}

This chapter is dedicated to the most important definition principle after
recursive functions and datatypes: inductively defined sets.

We start with a simple example: the set of even numbers.  A slightly more
complicated example, the reflexive transitive closure, is the subject of
{\S}\ref{sec:rtc}. In particular, some standard induction heuristics are
discussed. Advanced forms of inductive definitions are discussed in
{\S}\ref{sec:adv-ind-def}. To demonstrate the versatility of inductive
definitions, the chapter closes with a case study from the realm of
context-free grammars. The first two sections are required reading for anybody
interested in mathematical modelling.

% $Id$
\section{The Set of Even Numbers}

\index{even numbers!defining inductively|(}%
The set of even numbers can be inductively defined as the least set
containing 0 and closed under the operation $+2$.  Obviously,
\emph{even} can also be expressed using the divides relation (\isa{dvd}). 
We shall prove below that the two formulations coincide.  On the way we
shall examine the primary means of reasoning about inductively defined
sets: rule induction.

\subsection{Making an Inductive Definition}

Using \isacommand{consts}, we declare the constant \isa{even} to be a set
of natural numbers. The \commdx{inductive} declaration gives it the
desired properties.
\begin{isabelle}
\isacommand{consts}\ even\ ::\ "nat\ set"\isanewline
\isacommand{inductive}\ even\isanewline
\isakeyword{intros}\isanewline
zero[intro!]:\ "0\ \isasymin \ even"\isanewline
step[intro!]:\ "n\ \isasymin \ even\ \isasymLongrightarrow \ (Suc\ (Suc\
n))\ \isasymin \ even"
\end{isabelle}

An inductive definition consists of introduction rules.  The first one
above states that 0 is even; the second states that if $n$ is even, then so
is~$n+2$.  Given this declaration, Isabelle generates a fixed point
definition for \isa{even} and proves theorems about it,
thus following the definitional approach (see {\S}\ref{sec:definitional}).
These theorems
include the introduction rules specified in the declaration, an elimination
rule for case analysis and an induction rule.  We can refer to these
theorems by automatically-generated names.  Here are two examples:
%
\begin{isabelle}
0\ \isasymin \ even
\rulename{even.zero}
\par\smallskip
n\ \isasymin \ even\ \isasymLongrightarrow \ Suc\ (Suc\ n)\ \isasymin \
even%
\rulename{even.step}
\end{isabelle}

The introduction rules can be given attributes.  Here
both rules are specified as \isa{intro!},%
\index{intro"!@\isa {intro"!} (attribute)}
directing the classical reasoner to 
apply them aggressively. Obviously, regarding 0 as even is safe.  The
\isa{step} rule is also safe because $n+2$ is even if and only if $n$ is
even.  We prove this equivalence later.

\subsection{Using Introduction Rules}

Our first lemma states that numbers of the form $2\times k$ are even.
Introduction rules are used to show that specific values belong to the
inductive set.  Such proofs typically involve 
induction, perhaps over some other inductive set.
\begin{isabelle}
\isacommand{lemma}\ two_times_even[intro!]:\ "2*k\ \isasymin \ even"
\isanewline
\isacommand{apply}\ (induct_tac\ k)\isanewline
\ \isacommand{apply}\ auto\isanewline
\isacommand{done}
\end{isabelle}
%
The first step is induction on the natural number \isa{k}, which leaves
two subgoals:
\begin{isabelle}
\ 1.\ 2\ *\ 0\ \isasymin \ even\isanewline
\ 2.\ \isasymAnd n.\ 2\ *\ n\ \isasymin \ even\ \isasymLongrightarrow \ 2\ *\ Suc\ n\ \isasymin \ even
\end{isabelle}
%
Here \isa{auto} simplifies both subgoals so that they match the introduction
rules, which are then applied automatically.

Our ultimate goal is to prove the equivalence between the traditional
definition of \isa{even} (using the divides relation) and our inductive
definition.  One direction of this equivalence is immediate by the lemma
just proved, whose \isa{intro!} attribute ensures it is applied automatically.
\begin{isabelle}
\isacommand{lemma}\ dvd_imp_even:\ "2\ dvd\ n\ \isasymLongrightarrow \ n\ \isasymin \ even"\isanewline
\isacommand{by}\ (auto\ simp\ add:\ dvd_def)
\end{isabelle}

\subsection{Rule Induction}
\label{sec:rule-induction}

\index{rule induction|(}%
From the definition of the set
\isa{even}, Isabelle has
generated an induction rule:
\begin{isabelle}
\isasymlbrakk xa\ \isasymin \ even;\isanewline
\ P\ 0;\isanewline
\ \isasymAnd n.\ \isasymlbrakk n\ \isasymin \ even;\ P\ n\isasymrbrakk \
\isasymLongrightarrow \ P\ (Suc\ (Suc\ n))\isasymrbrakk\isanewline
\ \isasymLongrightarrow \ P\ xa%
\rulename{even.induct}
\end{isabelle}
A property \isa{P} holds for every even number provided it
holds for~\isa{0} and is closed under the operation
\isa{Suc(Suc \(\cdot\))}.  Then \isa{P} is closed under the introduction
rules for \isa{even}, which is the least set closed under those rules. 
This type of inductive argument is called \textbf{rule induction}. 

Apart from the double application of \isa{Suc}, the induction rule above
resembles the familiar mathematical induction, which indeed is an instance
of rule induction; the natural numbers can be defined inductively to be
the least set containing \isa{0} and closed under~\isa{Suc}.

Induction is the usual way of proving a property of the elements of an
inductively defined set.  Let us prove that all members of the set
\isa{even} are multiples of two.  
\begin{isabelle}
\isacommand{lemma}\ even_imp_dvd:\ "n\ \isasymin \ even\ \isasymLongrightarrow \ 2\ dvd\ n"
\end{isabelle}
%
We begin by applying induction.  Note that \isa{even.induct} has the form
of an elimination rule, so we use the method \isa{erule}.  We get two
subgoals:
\begin{isabelle}
\isacommand{apply}\ (erule\ even.induct)
\isanewline\isanewline
\ 1.\ 2\ dvd\ 0\isanewline
\ 2.\ \isasymAnd n.\ \isasymlbrakk n\ \isasymin \ even;\ 2\ dvd\ n\isasymrbrakk \ \isasymLongrightarrow \ 2\ dvd\ Suc\ (Suc\ n)
\end{isabelle}
%
We unfold the definition of \isa{dvd} in both subgoals, proving the first
one and simplifying the second:
\begin{isabelle}
\isacommand{apply}\ (simp_all\ add:\ dvd_def)
\isanewline\isanewline
\ 1.\ \isasymAnd n.\ \isasymlbrakk n\ \isasymin \ even;\ \isasymexists k.\
n\ =\ 2\ *\ k\isasymrbrakk \ \isasymLongrightarrow \ \isasymexists k.\
Suc\ (Suc\ n)\ =\ 2\ *\ k
\end{isabelle}
%
The next command eliminates the existential quantifier from the assumption
and replaces \isa{n} by \isa{2\ *\ k}.
\begin{isabelle}
\isacommand{apply}\ clarify
\isanewline\isanewline
\ 1.\ \isasymAnd n\ k.\ 2\ *\ k\ \isasymin \ even\ \isasymLongrightarrow \ \isasymexists ka.\ Suc\ (Suc\ (2\ *\ k))\ =\ 2\ *\ ka%
\end{isabelle}
%
To conclude, we tell Isabelle that the desired value is
\isa{Suc\ k}.  With this hint, the subgoal falls to \isa{simp}.
\begin{isabelle}
\isacommand{apply}\ (rule_tac\ x\ =\ "Suc\ k"\ \isakeyword{in}\ exI, simp)
\end{isabelle}


\medskip
Combining the previous two results yields our objective, the
equivalence relating \isa{even} and \isa{dvd}. 
%
%we don't want [iff]: discuss?
\begin{isabelle}
\isacommand{theorem}\ even_iff_dvd:\ "(n\ \isasymin \ even)\ =\ (2\ dvd\ n)"\isanewline
\isacommand{by}\ (blast\ intro:\ dvd_imp_even\ even_imp_dvd)
\end{isabelle}


\subsection{Generalization and Rule Induction}
\label{sec:gen-rule-induction}

\index{generalizing for induction}%
Before applying induction, we typically must generalize
the induction formula.  With rule induction, the required generalization
can be hard to find and sometimes requires a complete reformulation of the
problem.  In this  example, our first attempt uses the obvious statement of
the result.  It fails:
%
\begin{isabelle}
\isacommand{lemma}\ "Suc\ (Suc\ n)\ \isasymin \ even\
\isasymLongrightarrow \ n\ \isasymin \ even"\isanewline
\isacommand{apply}\ (erule\ even.induct)\isanewline
\isacommand{oops}
\end{isabelle}
%
Rule induction finds no occurrences of \isa{Suc(Suc\ n)} in the
conclusion, which it therefore leaves unchanged.  (Look at
\isa{even.induct} to see why this happens.)  We have these subgoals:
\begin{isabelle}
\ 1.\ n\ \isasymin \ even\isanewline
\ 2.\ \isasymAnd na.\ \isasymlbrakk na\ \isasymin \ even;\ n\ \isasymin \ even\isasymrbrakk \ \isasymLongrightarrow \ n\ \isasymin \ even%
\end{isabelle}
The first one is hopeless.  Rule induction on
a non-variable term discards information, and usually fails.
How to deal with such situations
in general is described in {\S}\ref{sec:ind-var-in-prems} below.
In the current case the solution is easy because
we have the necessary inverse, subtraction:
\begin{isabelle}
\isacommand{lemma}\ even_imp_even_minus_2:\ "n\ \isasymin \ even\ \isasymLongrightarrow \ n-2\ \isasymin \ even"\isanewline
\isacommand{apply}\ (erule\ even.induct)\isanewline
\ \isacommand{apply}\ auto\isanewline
\isacommand{done}
\end{isabelle}
%
This lemma is trivially inductive.  Here are the subgoals:
\begin{isabelle}
\ 1.\ 0\ -\ 2\ \isasymin \ even\isanewline
\ 2.\ \isasymAnd n.\ \isasymlbrakk n\ \isasymin \ even;\ n\ -\ 2\ \isasymin \ even\isasymrbrakk \ \isasymLongrightarrow \ Suc\ (Suc\ n)\ -\ 2\ \isasymin \ even%
\end{isabelle}
The first is trivial because \isa{0\ -\ 2} simplifies to \isa{0}, which is
even.  The second is trivial too: \isa{Suc\ (Suc\ n)\ -\ 2} simplifies to
\isa{n}, matching the assumption.%
\index{rule induction|)}  %the sequel isn't really about induction

\medskip
Using our lemma, we can easily prove the result we originally wanted:
\begin{isabelle}
\isacommand{lemma}\ Suc_Suc_even_imp_even:\ "Suc\ (Suc\ n)\ \isasymin \ even\ \isasymLongrightarrow \ n\ \isasymin \ even"\isanewline
\isacommand{by}\ (drule\ even_imp_even_minus_2, simp)
\end{isabelle}

We have just proved the converse of the introduction rule \isa{even.step}. 
This suggests proving the following equivalence.  We give it the
\attrdx{iff} attribute because of its obvious value for simplification.
\begin{isabelle}
\isacommand{lemma}\ [iff]:\ "((Suc\ (Suc\ n))\ \isasymin \ even)\ =\ (n\
\isasymin \ even)"\isanewline
\isacommand{by}\ (blast\ dest:\ Suc_Suc_even_imp_even)
\end{isabelle}


\subsection{Rule Inversion}\label{sec:rule-inversion}

\index{rule inversion|(}%
Case analysis on an inductive definition is called \textbf{rule
inversion}.  It is frequently used in proofs about operational
semantics.  It can be highly effective when it is applied
automatically.  Let us look at how rule inversion is done in
Isabelle/HOL\@.

Recall that \isa{even} is the minimal set closed under these two rules:
\begin{isabelle}
0\ \isasymin \ even\isanewline
n\ \isasymin \ even\ \isasymLongrightarrow \ Suc\ (Suc\ n)\ \isasymin
\ even
\end{isabelle}
Minimality means that \isa{even} contains only the elements that these
rules force it to contain.  If we are told that \isa{a}
belongs to
\isa{even} then there are only two possibilities.  Either \isa{a} is \isa{0}
or else \isa{a} has the form \isa{Suc(Suc~n)}, for some suitable \isa{n}
that belongs to
\isa{even}.  That is the gist of the \isa{cases} rule, which Isabelle proves
for us when it accepts an inductive definition:
\begin{isabelle}
\isasymlbrakk a\ \isasymin \ even;\isanewline
\ a\ =\ 0\ \isasymLongrightarrow \ P;\isanewline
\ \isasymAnd n.\ \isasymlbrakk a\ =\ Suc(Suc\ n);\ n\ \isasymin \
even\isasymrbrakk \ \isasymLongrightarrow \ P\isasymrbrakk \
\isasymLongrightarrow \ P
\rulename{even.cases}
\end{isabelle}

This general rule is less useful than instances of it for
specific patterns.  For example, if \isa{a} has the form
\isa{Suc(Suc~n)} then the first case becomes irrelevant, while the second
case tells us that \isa{n} belongs to \isa{even}.  Isabelle will generate
this instance for us:
\begin{isabelle}
\isacommand{inductive\_cases}\ Suc_Suc_cases\ [elim!]:
\ "Suc(Suc\ n)\ \isasymin \ even"
\end{isabelle}
The \commdx{inductive\protect\_cases} command generates an instance of
the
\isa{cases} rule for the supplied pattern and gives it the supplied name:
%
\begin{isabelle}
\isasymlbrakk Suc(Suc\ n)\ \isasymin \ even;\ n\ \isasymin \ even\
\isasymLongrightarrow \ P\isasymrbrakk \ \isasymLongrightarrow \ P%
\rulename{Suc_Suc_cases}
\end{isabelle}
%
Applying this as an elimination rule yields one case where \isa{even.cases}
would yield two.  Rule inversion works well when the conclusions of the
introduction rules involve datatype constructors like \isa{Suc} and \isa{\#}
(list ``cons''); freeness reasoning discards all but one or two cases.

In the \isacommand{inductive\_cases} command we supplied an
attribute, \isa{elim!},
\index{elim"!@\isa {elim"!} (attribute)}%
indicating that this elimination rule can be
applied aggressively.  The original
\isa{cases} rule would loop if used in that manner because the
pattern~\isa{a} matches everything.

The rule \isa{Suc_Suc_cases} is equivalent to the following implication:
\begin{isabelle}
Suc (Suc\ n)\ \isasymin \ even\ \isasymLongrightarrow \ n\ \isasymin \
even
\end{isabelle}
%
Just above we devoted some effort to reaching precisely
this result.  Yet we could have obtained it by a one-line declaration,
dispensing with the lemma \isa{even_imp_even_minus_2}. 
This example also justifies the terminology
\textbf{rule inversion}: the new rule inverts the introduction rule
\isa{even.step}.  In general, a rule can be inverted when the set of elements
it introduces is disjoint from those of the other introduction rules.

For one-off applications of rule inversion, use the \methdx{ind_cases} method. 
Here is an example:
\begin{isabelle}
\isacommand{apply}\ (ind_cases\ "Suc(Suc\ n)\ \isasymin \ even")
\end{isabelle}
The specified instance of the \isa{cases} rule is generated, then applied
as an elimination rule.

To summarize, every inductive definition produces a \isa{cases} rule.  The
\commdx{inductive\protect\_cases} command stores an instance of the
\isa{cases} rule for a given pattern.  Within a proof, the
\isa{ind_cases} method applies an instance of the \isa{cases}
rule.

The even numbers example has shown how inductive definitions can be
used.  Later examples will show that they are actually worth using.%
\index{rule inversion|)}%
\index{even numbers!defining inductively|)}

%
\begin{isabellebody}%
\def\isabellecontext{Mutual}%
\isamarkupfalse%
%
\isamarkupsubsection{Mutually Inductive Definitions%
}
\isamarkuptrue%
%
\begin{isamarkuptext}%
Just as there are datatypes defined by mutual recursion, there are sets defined
by mutual induction. As a trivial example we consider the even and odd
natural numbers:%
\end{isamarkuptext}%
\isamarkuptrue%
\isacommand{consts}\ even\ {\isacharcolon}{\isacharcolon}\ {\isachardoublequote}nat\ set{\isachardoublequote}\isanewline
\ \ \ \ \ \ \ odd\ \ {\isacharcolon}{\isacharcolon}\ {\isachardoublequote}nat\ set{\isachardoublequote}\isanewline
\isanewline
\isamarkupfalse%
\isacommand{inductive}\ even\ odd\isanewline
\isakeyword{intros}\isanewline
zero{\isacharcolon}\ \ {\isachardoublequote}{\isadigit{0}}\ {\isasymin}\ even{\isachardoublequote}\isanewline
evenI{\isacharcolon}\ {\isachardoublequote}n\ {\isasymin}\ odd\ {\isasymLongrightarrow}\ Suc\ n\ {\isasymin}\ even{\isachardoublequote}\isanewline
oddI{\isacharcolon}\ \ {\isachardoublequote}n\ {\isasymin}\ even\ {\isasymLongrightarrow}\ Suc\ n\ {\isasymin}\ odd{\isachardoublequote}\isamarkupfalse%
%
\begin{isamarkuptext}%
\noindent
The mutually inductive definition of multiple sets is no different from
that of a single set, except for induction: just as for mutually recursive
datatypes, induction needs to involve all the simultaneously defined sets. In
the above case, the induction rule is called \isa{even{\isacharunderscore}odd{\isachardot}induct}
(simply concatenate the names of the sets involved) and has the conclusion
\begin{isabelle}%
\ \ \ \ \ {\isacharparenleft}{\isacharquery}x\ {\isasymin}\ even\ {\isasymlongrightarrow}\ {\isacharquery}P\ {\isacharquery}x{\isacharparenright}\ {\isasymand}\ {\isacharparenleft}{\isacharquery}y\ {\isasymin}\ odd\ {\isasymlongrightarrow}\ {\isacharquery}Q\ {\isacharquery}y{\isacharparenright}%
\end{isabelle}

If we want to prove that all even numbers are divisible by two, we have to
generalize the statement as follows:%
\end{isamarkuptext}%
\isamarkuptrue%
\isacommand{lemma}\ {\isachardoublequote}{\isacharparenleft}m\ {\isasymin}\ even\ {\isasymlongrightarrow}\ {\isadigit{2}}\ dvd\ m{\isacharparenright}\ {\isasymand}\ {\isacharparenleft}n\ {\isasymin}\ odd\ {\isasymlongrightarrow}\ {\isadigit{2}}\ dvd\ {\isacharparenleft}Suc\ n{\isacharparenright}{\isacharparenright}{\isachardoublequote}\isamarkupfalse%
\isamarkuptrue%
\isamarkupfalse%
\isamarkuptrue%
\isamarkupfalse%
\isamarkupfalse%
\isamarkupfalse%
\isamarkupfalse%
\isamarkupfalse%
\isamarkupfalse%
\isamarkupfalse%
\isamarkupfalse%
\end{isabellebody}%
%%% Local Variables:
%%% mode: latex
%%% TeX-master: "root"
%%% End:

%
\begin{isabellebody}%
\def\isabellecontext{Star}%
%
\isamarkupsection{The Reflexive Transitive Closure%
}
%
\begin{isamarkuptext}%
\label{sec:rtc}
An inductive definition may accept parameters, so it can express 
functions that yield sets.
Relations too can be defined inductively, since they are just sets of pairs.
A perfect example is the function that maps a relation to its
reflexive transitive closure.  This concept was already
introduced in \S\ref{sec:Relations}, where the operator \isa{{\isacharcircum}{\isacharasterisk}} was
defined as a least fixed point because inductive definitions were not yet
available. But now they are:%
\end{isamarkuptext}%
\isacommand{consts}\ rtc\ {\isacharcolon}{\isacharcolon}\ {\isachardoublequote}{\isacharparenleft}{\isacharprime}a\ {\isasymtimes}\ {\isacharprime}a{\isacharparenright}set\ {\isasymRightarrow}\ {\isacharparenleft}{\isacharprime}a\ {\isasymtimes}\ {\isacharprime}a{\isacharparenright}set{\isachardoublequote}\ \ \ {\isacharparenleft}{\isachardoublequote}{\isacharunderscore}{\isacharasterisk}{\isachardoublequote}\ {\isacharbrackleft}{\isadigit{1}}{\isadigit{0}}{\isadigit{0}}{\isadigit{0}}{\isacharbrackright}\ {\isadigit{9}}{\isadigit{9}}{\isadigit{9}}{\isacharparenright}\isanewline
\isacommand{inductive}\ {\isachardoublequote}r{\isacharasterisk}{\isachardoublequote}\isanewline
\isakeyword{intros}\isanewline
rtc{\isacharunderscore}refl{\isacharbrackleft}iff{\isacharbrackright}{\isacharcolon}\ \ {\isachardoublequote}{\isacharparenleft}x{\isacharcomma}x{\isacharparenright}\ {\isasymin}\ r{\isacharasterisk}{\isachardoublequote}\isanewline
rtc{\isacharunderscore}step{\isacharcolon}\ \ \ \ \ \ \ {\isachardoublequote}{\isasymlbrakk}\ {\isacharparenleft}x{\isacharcomma}y{\isacharparenright}\ {\isasymin}\ r{\isacharsemicolon}\ {\isacharparenleft}y{\isacharcomma}z{\isacharparenright}\ {\isasymin}\ r{\isacharasterisk}\ {\isasymrbrakk}\ {\isasymLongrightarrow}\ {\isacharparenleft}x{\isacharcomma}z{\isacharparenright}\ {\isasymin}\ r{\isacharasterisk}{\isachardoublequote}%
\begin{isamarkuptext}%
\noindent
The function \isa{rtc} is annotated with concrete syntax: instead of
\isa{rtc\ r} we can read and write \isa{r{\isacharasterisk}}. The actual definition
consists of two rules. Reflexivity is obvious and is immediately given the
\isa{iff} attribute to increase automation. The
second rule, \isa{rtc{\isacharunderscore}step}, says that we can always add one more
\isa{r}-step to the left. Although we could make \isa{rtc{\isacharunderscore}step} an
introduction rule, this is dangerous: the recursion in the second premise
slows down and may even kill the automatic tactics.

The above definition of the concept of reflexive transitive closure may
be sufficiently intuitive but it is certainly not the only possible one:
for a start, it does not even mention transitivity.
The rest of this section is devoted to proving that it is equivalent to
the standard definition. We start with a simple lemma:%
\end{isamarkuptext}%
\isacommand{lemma}\ {\isacharbrackleft}intro{\isacharbrackright}{\isacharcolon}\ {\isachardoublequote}{\isacharparenleft}x{\isacharcomma}y{\isacharparenright}\ {\isacharcolon}\ r\ {\isasymLongrightarrow}\ {\isacharparenleft}x{\isacharcomma}y{\isacharparenright}\ {\isasymin}\ r{\isacharasterisk}{\isachardoublequote}\isanewline
\isacommand{by}{\isacharparenleft}blast\ intro{\isacharcolon}\ rtc{\isacharunderscore}step{\isacharparenright}%
\begin{isamarkuptext}%
\noindent
Although the lemma itself is an unremarkable consequence of the basic rules,
it has the advantage that it can be declared an introduction rule without the
danger of killing the automatic tactics because \isa{r{\isacharasterisk}} occurs only in
the conclusion and not in the premise. Thus some proofs that would otherwise
need \isa{rtc{\isacharunderscore}step} can now be found automatically. The proof also
shows that \isa{blast} is able to handle \isa{rtc{\isacharunderscore}step}. But
some of the other automatic tactics are more sensitive, and even \isa{blast} can be lead astray in the presence of large numbers of rules.

To prove transitivity, we need rule induction, i.e.\ theorem
\isa{rtc{\isachardot}induct}:
\begin{isabelle}%
\ \ \ \ \ {\isasymlbrakk}{\isacharparenleft}{\isacharquery}xb{\isacharcomma}\ {\isacharquery}xa{\isacharparenright}\ {\isasymin}\ {\isacharquery}r{\isacharasterisk}{\isacharsemicolon}\ {\isasymAnd}x{\isachardot}\ {\isacharquery}P\ x\ x{\isacharsemicolon}\isanewline
\isaindent{\ \ \ \ \ \ \ \ }{\isasymAnd}x\ y\ z{\isachardot}\ {\isasymlbrakk}{\isacharparenleft}x{\isacharcomma}\ y{\isacharparenright}\ {\isasymin}\ {\isacharquery}r{\isacharsemicolon}\ {\isacharparenleft}y{\isacharcomma}\ z{\isacharparenright}\ {\isasymin}\ {\isacharquery}r{\isacharasterisk}{\isacharsemicolon}\ {\isacharquery}P\ y\ z{\isasymrbrakk}\ {\isasymLongrightarrow}\ {\isacharquery}P\ x\ z{\isasymrbrakk}\isanewline
\isaindent{\ \ \ \ \ }{\isasymLongrightarrow}\ {\isacharquery}P\ {\isacharquery}xb\ {\isacharquery}xa%
\end{isabelle}
It says that \isa{{\isacharquery}P} holds for an arbitrary pair \isa{{\isacharparenleft}{\isacharquery}xb{\isacharcomma}{\isacharquery}xa{\isacharparenright}\ {\isasymin}\ {\isacharquery}r{\isacharasterisk}} if \isa{{\isacharquery}P} is preserved by all rules of the inductive definition,
i.e.\ if \isa{{\isacharquery}P} holds for the conclusion provided it holds for the
premises. In general, rule induction for an $n$-ary inductive relation $R$
expects a premise of the form $(x@1,\dots,x@n) \in R$.

Now we turn to the inductive proof of transitivity:%
\end{isamarkuptext}%
\isacommand{lemma}\ rtc{\isacharunderscore}trans{\isacharcolon}\ {\isachardoublequote}{\isasymlbrakk}\ {\isacharparenleft}x{\isacharcomma}y{\isacharparenright}\ {\isasymin}\ r{\isacharasterisk}{\isacharsemicolon}\ {\isacharparenleft}y{\isacharcomma}z{\isacharparenright}\ {\isasymin}\ r{\isacharasterisk}\ {\isasymrbrakk}\ {\isasymLongrightarrow}\ {\isacharparenleft}x{\isacharcomma}z{\isacharparenright}\ {\isasymin}\ r{\isacharasterisk}{\isachardoublequote}\isanewline
\isacommand{apply}{\isacharparenleft}erule\ rtc{\isachardot}induct{\isacharparenright}%
\begin{isamarkuptxt}%
\noindent
Unfortunately, even the resulting base case is a problem
\begin{isabelle}%
\ {\isadigit{1}}{\isachardot}\ {\isasymAnd}x{\isachardot}\ {\isacharparenleft}y{\isacharcomma}\ z{\isacharparenright}\ {\isasymin}\ r{\isacharasterisk}\ {\isasymLongrightarrow}\ {\isacharparenleft}x{\isacharcomma}\ z{\isacharparenright}\ {\isasymin}\ r{\isacharasterisk}%
\end{isabelle}
and maybe not what you had expected. We have to abandon this proof attempt.
To understand what is going on, let us look again at \isa{rtc{\isachardot}induct}.
In the above application of \isa{erule}, the first premise of
\isa{rtc{\isachardot}induct} is unified with the first suitable assumption, which
is \isa{{\isacharparenleft}x{\isacharcomma}\ y{\isacharparenright}\ {\isasymin}\ r{\isacharasterisk}} rather than \isa{{\isacharparenleft}y{\isacharcomma}\ z{\isacharparenright}\ {\isasymin}\ r{\isacharasterisk}}. Although that
is what we want, it is merely due to the order in which the assumptions occur
in the subgoal, which it is not good practice to rely on. As a result,
\isa{{\isacharquery}xb} becomes \isa{x}, \isa{{\isacharquery}xa} becomes
\isa{y} and \isa{{\isacharquery}P} becomes \isa{{\isasymlambda}u\ v{\isachardot}\ {\isacharparenleft}u{\isacharcomma}\ z{\isacharparenright}\ {\isasymin}\ r{\isacharasterisk}}, thus
yielding the above subgoal. So what went wrong?

When looking at the instantiation of \isa{{\isacharquery}P} we see that it does not
depend on its second parameter at all. The reason is that in our original
goal, of the pair \isa{{\isacharparenleft}x{\isacharcomma}\ y{\isacharparenright}} only \isa{x} appears also in the
conclusion, but not \isa{y}. Thus our induction statement is too
weak. Fortunately, it can easily be strengthened:
transfer the additional premise \isa{{\isacharparenleft}y{\isacharcomma}\ z{\isacharparenright}\ {\isasymin}\ r{\isacharasterisk}} into the conclusion:%
\end{isamarkuptxt}%
\isacommand{lemma}\ rtc{\isacharunderscore}trans{\isacharbrackleft}rule{\isacharunderscore}format{\isacharbrackright}{\isacharcolon}\isanewline
\ \ {\isachardoublequote}{\isacharparenleft}x{\isacharcomma}y{\isacharparenright}\ {\isasymin}\ r{\isacharasterisk}\ {\isasymLongrightarrow}\ {\isacharparenleft}y{\isacharcomma}z{\isacharparenright}\ {\isasymin}\ r{\isacharasterisk}\ {\isasymlongrightarrow}\ {\isacharparenleft}x{\isacharcomma}z{\isacharparenright}\ {\isasymin}\ r{\isacharasterisk}{\isachardoublequote}%
\begin{isamarkuptxt}%
\noindent
This is not an obscure trick but a generally applicable heuristic:
\begin{quote}\em
Whe proving a statement by rule induction on $(x@1,\dots,x@n) \in R$,
pull all other premises containing any of the $x@i$ into the conclusion
using $\longrightarrow$.
\end{quote}
A similar heuristic for other kinds of inductions is formulated in
\S\ref{sec:ind-var-in-prems}. The \isa{rule{\isacharunderscore}format} directive turns
\isa{{\isasymlongrightarrow}} back into \isa{{\isasymLongrightarrow}}. Thus in the end we obtain the original
statement of our lemma.%
\end{isamarkuptxt}%
\isacommand{apply}{\isacharparenleft}erule\ rtc{\isachardot}induct{\isacharparenright}%
\begin{isamarkuptxt}%
\noindent
Now induction produces two subgoals which are both proved automatically:
\begin{isabelle}%
\ {\isadigit{1}}{\isachardot}\ {\isasymAnd}x{\isachardot}\ {\isacharparenleft}x{\isacharcomma}\ z{\isacharparenright}\ {\isasymin}\ r{\isacharasterisk}\ {\isasymlongrightarrow}\ {\isacharparenleft}x{\isacharcomma}\ z{\isacharparenright}\ {\isasymin}\ r{\isacharasterisk}\isanewline
\ {\isadigit{2}}{\isachardot}\ {\isasymAnd}x\ y\ za{\isachardot}\isanewline
\isaindent{\ {\isadigit{2}}{\isachardot}\ \ \ \ }{\isasymlbrakk}{\isacharparenleft}x{\isacharcomma}\ y{\isacharparenright}\ {\isasymin}\ r{\isacharsemicolon}\ {\isacharparenleft}y{\isacharcomma}\ za{\isacharparenright}\ {\isasymin}\ r{\isacharasterisk}{\isacharsemicolon}\ {\isacharparenleft}za{\isacharcomma}\ z{\isacharparenright}\ {\isasymin}\ r{\isacharasterisk}\ {\isasymlongrightarrow}\ {\isacharparenleft}y{\isacharcomma}\ z{\isacharparenright}\ {\isasymin}\ r{\isacharasterisk}{\isasymrbrakk}\isanewline
\isaindent{\ {\isadigit{2}}{\isachardot}\ \ \ \ }{\isasymLongrightarrow}\ {\isacharparenleft}za{\isacharcomma}\ z{\isacharparenright}\ {\isasymin}\ r{\isacharasterisk}\ {\isasymlongrightarrow}\ {\isacharparenleft}x{\isacharcomma}\ z{\isacharparenright}\ {\isasymin}\ r{\isacharasterisk}%
\end{isabelle}%
\end{isamarkuptxt}%
\ \isacommand{apply}{\isacharparenleft}blast{\isacharparenright}\isanewline
\isacommand{apply}{\isacharparenleft}blast\ intro{\isacharcolon}\ rtc{\isacharunderscore}step{\isacharparenright}\isanewline
\isacommand{done}%
\begin{isamarkuptext}%
Let us now prove that \isa{r{\isacharasterisk}} is really the reflexive transitive closure
of \isa{r}, i.e.\ the least reflexive and transitive
relation containing \isa{r}. The latter is easily formalized%
\end{isamarkuptext}%
\isacommand{consts}\ rtc{\isadigit{2}}\ {\isacharcolon}{\isacharcolon}\ {\isachardoublequote}{\isacharparenleft}{\isacharprime}a\ {\isasymtimes}\ {\isacharprime}a{\isacharparenright}set\ {\isasymRightarrow}\ {\isacharparenleft}{\isacharprime}a\ {\isasymtimes}\ {\isacharprime}a{\isacharparenright}set{\isachardoublequote}\isanewline
\isacommand{inductive}\ {\isachardoublequote}rtc{\isadigit{2}}\ r{\isachardoublequote}\isanewline
\isakeyword{intros}\isanewline
{\isachardoublequote}{\isacharparenleft}x{\isacharcomma}y{\isacharparenright}\ {\isasymin}\ r\ {\isasymLongrightarrow}\ {\isacharparenleft}x{\isacharcomma}y{\isacharparenright}\ {\isasymin}\ rtc{\isadigit{2}}\ r{\isachardoublequote}\isanewline
{\isachardoublequote}{\isacharparenleft}x{\isacharcomma}x{\isacharparenright}\ {\isasymin}\ rtc{\isadigit{2}}\ r{\isachardoublequote}\isanewline
{\isachardoublequote}{\isasymlbrakk}\ {\isacharparenleft}x{\isacharcomma}y{\isacharparenright}\ {\isasymin}\ rtc{\isadigit{2}}\ r{\isacharsemicolon}\ {\isacharparenleft}y{\isacharcomma}z{\isacharparenright}\ {\isasymin}\ rtc{\isadigit{2}}\ r\ {\isasymrbrakk}\ {\isasymLongrightarrow}\ {\isacharparenleft}x{\isacharcomma}z{\isacharparenright}\ {\isasymin}\ rtc{\isadigit{2}}\ r{\isachardoublequote}%
\begin{isamarkuptext}%
\noindent
and the equivalence of the two definitions is easily shown by the obvious rule
inductions:%
\end{isamarkuptext}%
\isacommand{lemma}\ {\isachardoublequote}{\isacharparenleft}x{\isacharcomma}y{\isacharparenright}\ {\isasymin}\ rtc{\isadigit{2}}\ r\ {\isasymLongrightarrow}\ {\isacharparenleft}x{\isacharcomma}y{\isacharparenright}\ {\isasymin}\ r{\isacharasterisk}{\isachardoublequote}\isanewline
\isacommand{apply}{\isacharparenleft}erule\ rtc{\isadigit{2}}{\isachardot}induct{\isacharparenright}\isanewline
\ \ \isacommand{apply}{\isacharparenleft}blast{\isacharparenright}\isanewline
\ \isacommand{apply}{\isacharparenleft}blast{\isacharparenright}\isanewline
\isacommand{apply}{\isacharparenleft}blast\ intro{\isacharcolon}\ rtc{\isacharunderscore}trans{\isacharparenright}\isanewline
\isacommand{done}\isanewline
\isanewline
\isacommand{lemma}\ {\isachardoublequote}{\isacharparenleft}x{\isacharcomma}y{\isacharparenright}\ {\isasymin}\ r{\isacharasterisk}\ {\isasymLongrightarrow}\ {\isacharparenleft}x{\isacharcomma}y{\isacharparenright}\ {\isasymin}\ rtc{\isadigit{2}}\ r{\isachardoublequote}\isanewline
\isacommand{apply}{\isacharparenleft}erule\ rtc{\isachardot}induct{\isacharparenright}\isanewline
\ \isacommand{apply}{\isacharparenleft}blast\ intro{\isacharcolon}\ rtc{\isadigit{2}}{\isachardot}intros{\isacharparenright}\isanewline
\isacommand{apply}{\isacharparenleft}blast\ intro{\isacharcolon}\ rtc{\isadigit{2}}{\isachardot}intros{\isacharparenright}\isanewline
\isacommand{done}%
\begin{isamarkuptext}%
So why did we start with the first definition? Because it is simpler. It
contains only two rules, and the single step rule is simpler than
transitivity.  As a consequence, \isa{rtc{\isachardot}induct} is simpler than
\isa{rtc{\isadigit{2}}{\isachardot}induct}. Since inductive proofs are hard enough
anyway, we should
certainly pick the simplest induction schema available.
Hence \isa{rtc} is the definition of choice.

\begin{exercise}\label{ex:converse-rtc-step}
Show that the converse of \isa{rtc{\isacharunderscore}step} also holds:
\begin{isabelle}%
\ \ \ \ \ {\isasymlbrakk}{\isacharparenleft}x{\isacharcomma}\ y{\isacharparenright}\ {\isasymin}\ r{\isacharasterisk}{\isacharsemicolon}\ {\isacharparenleft}y{\isacharcomma}\ z{\isacharparenright}\ {\isasymin}\ r{\isasymrbrakk}\ {\isasymLongrightarrow}\ {\isacharparenleft}x{\isacharcomma}\ z{\isacharparenright}\ {\isasymin}\ r{\isacharasterisk}%
\end{isabelle}
\end{exercise}
\begin{exercise}
Repeat the development of this section, but starting with a definition of
\isa{rtc} where \isa{rtc{\isacharunderscore}step} is replaced by its converse as shown
in exercise~\ref{ex:converse-rtc-step}.
\end{exercise}%
\end{isamarkuptext}%
\end{isabellebody}%
%%% Local Variables:
%%% mode: latex
%%% TeX-master: "root"
%%% End:


\section{Advanced Inductive Definitions}
\label{sec:adv-ind-def}
% $Id$
This section describes advanced features of inductive definitions. 
The premises of introduction rules may contain universal quantifiers and
monotonic functions.  Theorems may be proved by rule inversion.

\subsection{Universal Quantifiers in Introduction Rules}
\label{sec:gterm-datatype}

As a running example, this section develops the theory of \textbf{ground
terms}: terms constructed from constant and function 
symbols but not variables. To simplify matters further, we regard a
constant as a function applied to the null argument  list.  Let us declare a
datatype \isa{gterm} for the type of ground  terms. It is a type constructor
whose argument is a type of  function symbols. 
\begin{isabelle}
\isacommand{datatype}\ 'f\ gterm\ =\ Apply\ 'f\ "'f\ gterm\ list"
\end{isabelle}
To try it out, we declare a datatype of some integer operations: 
integer constants, the unary minus operator and the addition 
operator. 
\begin{isabelle}
\isacommand{datatype}\ integer_op\ =\ Number\ int\ |\ UnaryMinus\ |\ Plus
\end{isabelle}
Now the type \isa{integer\_op gterm} denotes the ground 
terms built over those symbols.

The type constructor \texttt{gterm} can be generalized to a function 
over sets.  It returns 
the set of ground terms that can be formed over a set \isa{F} of function symbols. For
example,  we could consider the set of ground terms formed from the finite 
set \isa{\isacharbraceleft Number 2, UnaryMinus,
Plus\isacharbraceright}.

This concept is inductive. If we have a list \isa{args} of ground terms 
over~\isa{F} and a function symbol \isa{f} in \isa{F}, then we 
can apply \isa{f} to  \isa{args} to obtain another ground term. 
The only difficulty is that the argument list may be of any length. Hitherto, 
each rule in an inductive definition referred to the inductively 
defined set a fixed number of times, typically once or twice. 
A universal quantifier in the premise of the introduction rule 
expresses that every element of \isa{args} belongs
to our inductively defined set: is a ground term 
over~\isa{F}.  The function {\isa{set}} denotes the set of elements in a given 
list. 
\begin{isabelle}
\isacommand{consts}\ gterms\ ::\ "'f\ set\ \isasymRightarrow \ 'f\ gterm\ set"\isanewline
\isacommand{inductive}\ "gterms\ F"\isanewline
\isakeyword{intros}\isanewline
step[intro!]:\ "\isasymlbrakk \isasymforall t\ \isasymin \ set\ args.\ t\ \isasymin \ gterms\ F;\ \ f\ \isasymin \ F\isasymrbrakk \isanewline
\ \ \ \ \ \ \ \ \ \ \ \ \ \ \ \isasymLongrightarrow \ (Apply\ f\ args)\ \isasymin \ gterms\
F"
\end{isabelle}

To demonstrate a proof from this definition, let us 
show that the function \isa{gterms}
is \textbf{monotonic}.  We shall need this concept shortly.
\begin{isabelle}
\isacommand{lemma}\ gterms_mono:\ "F\isasymsubseteq G\ \isasymLongrightarrow \ gterms\ F\ \isasymsubseteq \ gterms\ G"\isanewline
\isacommand{apply}\ clarify\isanewline
\isacommand{apply}\ (erule\ gterms.induct)\isanewline
\isacommand{apply}\ blast\isanewline
\isacommand{done}
\end{isabelle}
Intuitively, this theorem says that
enlarging the set of function symbols enlarges the set of ground 
terms. The proof is a trivial rule induction.
First we use the \isa{clarify} method to assume the existence of an element of
\isa{gterms~F}.  (We could have used \isa{intro subsetI}.)  We then
apply rule induction. Here is the resulting subgoal: 
\begin{isabelle}
\ 1.\ \isasymAnd x\ args\ f.\isanewline
\ \ \ \ \ \ \ \isasymlbrakk F\ \isasymsubseteq \ G;\ \isasymforall t\isasymin set\ args.\ t\ \isasymin \ gterms\ F\ \isasymand \ t\ \isasymin \ gterms\ G;\ f\ \isasymin \ F\isasymrbrakk \isanewline
\ \ \ \ \ \ \ \isasymLongrightarrow \ Apply\ f\ args\ \isasymin \ gterms\ G%
\end{isabelle}
%
The assumptions state that \isa{f} belongs 
to~\isa{F}, which is included in~\isa{G}, and that every element of the list \isa{args} is
a ground term over~\isa{G}.  The \isa{blast} method finds this chain of reasoning easily.  

\begin{warn}
Why do we call this function \isa{gterms} instead 
of {\isa{gterm}}?  A constant may have the same name as a type.  However,
name  clashes could arise in the theorems that Isabelle generates. 
Our choice of names keeps {\isa{gterms.induct}} separate from 
{\isa{gterm.induct}}.
\end{warn}


\subsection{Rule Inversion}\label{sec:rule-inversion}

Case analysis on an inductive definition is called \textbf{rule inversion}. 
It is frequently used in proofs about operational semantics.  It can be
highly effective when it is applied automatically.  Let us look at how rule
inversion is done in Isabelle.

Recall that \isa{even} is the minimal set closed under these two rules:
\begin{isabelle}
0\ \isasymin \ even\isanewline
n\ \isasymin \ even\ \isasymLongrightarrow \ Suc\ (Suc\ n)\ \isasymin
\ even
\end{isabelle}
Minimality means that \isa{even} contains only the elements that these
rules force it to contain.  If we are told that \isa{a}
belongs to
\isa{even} then there are only two possibilities.  Either \isa{a} is \isa{0}
or else \isa{a} has the form \isa{Suc(Suc~n)}, for some suitable \isa{n}
that belongs to
\isa{even}.  That is the gist of the \isa{cases} rule, which Isabelle proves
for us when it accepts an inductive definition:
\begin{isabelle}
\isasymlbrakk a\ \isasymin \ even;\isanewline
\ a\ =\ 0\ \isasymLongrightarrow \ P;\isanewline
\ \isasymAnd n.\ \isasymlbrakk a\ =\ Suc(Suc\ n);\ n\ \isasymin \
even\isasymrbrakk \ \isasymLongrightarrow \ P\isasymrbrakk \
\isasymLongrightarrow \ P
\rulename{even.cases}
\end{isabelle}

This general rule is less useful than instances of it for
specific patterns.  For example, if \isa{a} has the form
\isa{Suc(Suc~n)} then the first case becomes irrelevant, while the second
case tells us that \isa{n} belongs to \isa{even}.  Isabelle will generate
this instance for us:
\begin{isabelle}
\isacommand{inductive\_cases}\ Suc_Suc_cases\ [elim!]:
\ "Suc(Suc\ n)\ \isasymin \ even"
\end{isabelle}
The \isacommand{inductive\_cases} command generates an instance of the
\isa{cases} rule for the supplied pattern and gives it the supplied name:
%
\begin{isabelle}
\isasymlbrakk Suc\ (Suc\ n)\ \isasymin \ even;\ n\ \isasymin \ even\
\isasymLongrightarrow \ P\isasymrbrakk \ \isasymLongrightarrow \ P%
\rulename{Suc_Suc_cases}
\end{isabelle}
%
Applying this as an elimination rule yields one case where \isa{even.cases}
would yield two.  Rule inversion works well when the conclusions of the
introduction rules involve datatype constructors like \isa{Suc} and \isa{\#}
(list ``cons''); freeness reasoning discards all but one or two cases.

In the \isacommand{inductive\_cases} command we supplied an
attribute, \isa{elim!}, indicating that this elimination rule can be applied
aggressively.  The original
\isa{cases} rule would loop if used in that manner because the
pattern~\isa{a} matches everything.

The rule \isa{Suc_Suc_cases} is equivalent to the following implication:
\begin{isabelle}
Suc (Suc\ n)\ \isasymin \ even\ \isasymLongrightarrow \ n\ \isasymin \
even
\end{isabelle}
%
In {\S}\ref{sec:gen-rule-induction} we devoted some effort to proving precisely
this result.  Yet we could have obtained it by a one-line declaration. 
This example also justifies the terminology \textbf{rule inversion}: the new
rule inverts the introduction rule \isa{even.step}.

For one-off applications of rule inversion, use the \isa{ind_cases} method. 
Here is an example:
\begin{isabelle}
\isacommand{apply}\ (ind_cases\ "Suc(Suc\ n)\ \isasymin \ even")
\end{isabelle}
The specified instance of the \isa{cases} rule is generated, applied, and
discarded.

\medskip
Let us try rule inversion on our ground terms example:
\begin{isabelle}
\isacommand{inductive\_cases}\ gterm_Apply_elim\ [elim!]:\ "Apply\ f\ args\
\isasymin\ gterms\ F"
\end{isabelle}
%
Here is the result.  No cases are discarded (there was only one to begin
with) but the rule applies specifically to the pattern \isa{Apply\ f\ args}.
It can be applied repeatedly as an elimination rule without looping, so we
have given the
\isa{elim!}\ attribute. 
\begin{isabelle}
\isasymlbrakk Apply\ f\ args\ \isasymin \ gterms\ F;\isanewline
\ \isasymlbrakk
\isasymforall t\isasymin set\ args.\ t\ \isasymin \ gterms\ F;\ f\ \isasymin
\ F\isasymrbrakk \ \isasymLongrightarrow \ P\isasymrbrakk\isanewline
\isasymLongrightarrow \ P%
\rulename{gterm_Apply_elim}
\end{isabelle}

This rule replaces an assumption about \isa{Apply\ f\ args} by 
assumptions about \isa{f} and~\isa{args}.  Here is a proof in which this
inference is essential.  We show that if \isa{t} is a ground term over both
of the sets
\isa{F} and~\isa{G} then it is also a ground term over their intersection,
\isa{F\isasyminter G}.
\begin{isabelle}
\isacommand{lemma}\ gterms_IntI\ [rule_format]:\isanewline
\ \ \ \ \ "t\ \isasymin \ gterms\ F\ \isasymLongrightarrow \ t\ \isasymin \ gterms\ G\ \isasymlongrightarrow \ t\ \isasymin \ gterms\ (F\isasyminter G)"\isanewline
\isacommand{apply}\ (erule\ gterms.induct)\isanewline
\isacommand{apply}\ blast\isanewline
\isacommand{done}
\end{isabelle}
%
The proof begins with rule induction over the definition of
\isa{gterms}, which leaves a single subgoal:  
\begin{isabelle}
1.\ \isasymAnd args\ f.\isanewline
\ \ \ \ \ \ \isasymlbrakk \isasymforall t\isasymin set\ args.\ t\ \isasymin \ gterms\ F\ \isasymand\isanewline
\ \ \ \ \ \ \ \ \ \ \ \ \ \ \ \ \ \ \ \ \ (t\ \isasymin \ gterms\ G\ \isasymlongrightarrow \ t\ \isasymin \ gterms\ (F\ \isasyminter \ G));\isanewline
\ \ \ \ \ \ \ f\ \isasymin \ F\isasymrbrakk \isanewline
\ \ \ \ \ \ \isasymLongrightarrow \ Apply\ f\ args\ \isasymin \ gterms\ G\ \isasymlongrightarrow \ Apply\ f\ args\ \isasymin \ gterms\ (F\ \isasyminter \ G)
\end{isabelle}
%
The induction hypothesis states that every element of \isa{args} belongs to 
\isa{gterms\ (F\ \isasyminter \ G)} --- provided it already belongs to
\isa{gterms\ G}.  How do we meet that condition?  

By assuming (as we may) the formula \isa{Apply\ f\ args\ \isasymin \ gterms\
G}.  Rule inversion, in the form of \isa{gterm_Apply_elim}, infers that every
element of \isa{args} belongs to 
\isa{gterms~G}.  It also yields \isa{f\ \isasymin \ G}, which will allow us
to conclude \isa{f\ \isasymin \ F\ \isasyminter \ G}.  All of this reasoning
is done by \isa{blast}.

\medskip

To summarize, every inductive definition produces a \isa{cases} rule.  The
\isacommand{inductive\_cases} command stores an instance of the \isa{cases}
rule for a given pattern.  Within a proof, the \isa{ind_cases} method
applies an instance of the \isa{cases}
rule.


\subsection{Continuing the Ground Terms Example}

Call a term \textbf{well-formed} if each symbol occurring in it has 
the correct number of arguments. To formalize this concept, we 
introduce a function mapping each symbol to its \textbf{arity}, a natural 
number. 

Let us define the set of well-formed ground terms. 
Suppose we are given a function called \isa{arity}, specifying the arities to be used.
In the inductive step, we have a list \isa{args} of such terms and a function 
symbol~\isa{f}. If the length of the list matches the function's arity 
then applying \isa{f} to \isa{args} yields a well-formed term. 
\begin{isabelle}
\isacommand{consts}\ well_formed_gterm\ ::\ "('f\ \isasymRightarrow \ nat)\ \isasymRightarrow \ 'f\ gterm\ set"\isanewline
\isacommand{inductive}\ "well_formed_gterm\ arity"\isanewline
\isakeyword{intros}\isanewline
step[intro!]:\ "\isasymlbrakk \isasymforall t\ \isasymin \ set\ args.\ t\ \isasymin \ well_formed_gterm\ arity;\ \ \isanewline
\ \ \ \ \ \ \ \ \ \ \ \ \ \ \ \ length\ args\ =\ arity\ f\isasymrbrakk \isanewline
\ \ \ \ \ \ \ \ \ \ \ \ \ \ \ \isasymLongrightarrow \ (Apply\ f\ args)\ \isasymin \ well_formed_gterm\
arity"
\end{isabelle}
%
The inductive definition neatly captures the reasoning above.
It is just an elaboration of the previous one, consisting of a single 
introduction rule. The universal quantification over the
\isa{set} of arguments expresses that all of them are well-formed.

\subsection{Alternative Definition Using a Monotonic Function}

An inductive definition may refer to the inductively defined 
set through an arbitrary monotonic function.  To demonstrate this
powerful feature, let us
change the  inductive definition above, replacing the
quantifier by a use of the function \isa{lists}. This
function, from the Isabelle theory of lists, is analogous to the
function \isa{gterms} declared above: if \isa{A} is a set then
{\isa{lists A}} is the set of lists whose elements belong to
\isa{A}.  

In the inductive definition of well-formed terms, examine the one
introduction rule.  The first premise states that \isa{args} belongs to
the \isa{lists} of well-formed terms.  This formulation is more
direct, if more obscure, than using a universal quantifier.
\begin{isabelle}
\isacommand{consts}\ well_formed_gterm'\ ::\ "('f\ \isasymRightarrow \ nat)\ \isasymRightarrow \ 'f\ gterm\ set"\isanewline
\isacommand{inductive}\ "well_formed_gterm'\ arity"\isanewline
\isakeyword{intros}\isanewline
step[intro!]:\ "\isasymlbrakk args\ \isasymin \ lists\ (well_formed_gterm'\ arity);\ \ \isanewline
\ \ \ \ \ \ \ \ \ \ \ \ \ \ \ \ length\ args\ =\ arity\ f\isasymrbrakk \isanewline
\ \ \ \ \ \ \ \ \ \ \ \ \ \ \ \isasymLongrightarrow \ (Apply\ f\ args)\ \isasymin \ well_formed_gterm'\ arity"\isanewline
\isakeyword{monos}\ lists_mono
\end{isabelle}

We must cite the theorem \isa{lists_mono} in order to justify 
using the function \isa{lists}. 
\begin{isabelle}
A\ \isasymsubseteq\ B\ \isasymLongrightarrow \ lists\ A\ \isasymsubseteq
\ lists\ B\rulename{lists_mono}
\end{isabelle}
%
Why must the function be monotonic?  An inductive definition describes
an iterative construction: each element of the set is constructed by a
finite number of introduction rule applications.  For example, the
elements of \isa{even} are constructed by finitely many applications of
the rules 
\begin{isabelle}
0\ \isasymin \ even\isanewline
n\ \isasymin \ even\ \isasymLongrightarrow \ (Suc\ (Suc\ n))\ \isasymin
\ even
\end{isabelle}
All references to a set in its
inductive definition must be positive.  Applications of an
introduction rule cannot invalidate previous applications, allowing the
construction process to converge.
The following pair of rules do not constitute an inductive definition:
\begin{isabelle}
0\ \isasymin \ even\isanewline
n\ \isasymnotin \ even\ \isasymLongrightarrow \ (Suc\ n)\ \isasymin
\ even
\end{isabelle}
%
Showing that 4 is even using these rules requires showing that 3 is not
even.  It is far from trivial to show that this set of rules
characterizes the even numbers.  

Even with its use of the function \isa{lists}, the premise of our
introduction rule is positive:
\begin{isabelle}
args\ \isasymin \ lists\ (well_formed_gterm'\ arity)
\end{isabelle}
To apply the rule we construct a list \isa{args} of previously
constructed well-formed terms.  We obtain a
new term, \isa{Apply\ f\ args}.  Because \isa{lists} is monotonic,
applications of the rule remain valid as new terms are constructed.
Further lists of well-formed
terms become available and none are taken away.


\subsection{A Proof of Equivalence}

We naturally hope that these two inductive definitions of ``well-formed'' 
coincide.  The equality can be proved by separate inclusions in 
each direction.  Each is a trivial rule induction. 
\begin{isabelle}
\isacommand{lemma}\ "well_formed_gterm\ arity\ \isasymsubseteq \ well_formed_gterm'\ arity"\isanewline
\isacommand{apply}\ clarify\isanewline
\isacommand{apply}\ (erule\ well_formed_gterm.induct)\isanewline
\isacommand{apply}\ auto\isanewline
\isacommand{done}
\end{isabelle}

The \isa{clarify} method gives
us an element of \isa{well_formed_gterm\ arity} on which to perform 
induction.  The resulting subgoal can be proved automatically:
\begin{isabelle}
{\isadigit{1}}{\isachardot}\ {\isasymAnd}x\ args\ f{\isachardot}\isanewline
\ \ \ \ \ \ {\isasymlbrakk}{\isasymforall}t{\isasymin}set\ args{\isachardot}\isanewline
\ \ \ \ \ \ \ \ \ \ t\ {\isasymin}\ well{\isacharunderscore}formed{\isacharunderscore}gterm\ arity\ {\isasymand}\ t\ {\isasymin}\ well{\isacharunderscore}formed{\isacharunderscore}gterm{\isacharprime}\ arity{\isacharsemicolon}\isanewline
\ \ \ \ \ \ \ length\ args\ {\isacharequal}\ arity\ f{\isasymrbrakk}\isanewline
\ \ \ \ \ \ {\isasymLongrightarrow}\ Apply\ f\ args\ {\isasymin}\ well{\isacharunderscore}formed{\isacharunderscore}gterm{\isacharprime}\ arity%
\end{isabelle}
%
This proof resembles the one given in
{\S}\ref{sec:gterm-datatype} above, especially in the form of the
induction hypothesis.  Next, we consider the opposite inclusion:
\begin{isabelle}
\isacommand{lemma}\ "well_formed_gterm'\ arity\ \isasymsubseteq \ well_formed_gterm\ arity"\isanewline
\isacommand{apply}\ clarify\isanewline
\isacommand{apply}\ (erule\ well_formed_gterm'.induct)\isanewline
\isacommand{apply}\ auto\isanewline
\isacommand{done}
\end{isabelle}
%
The proof script is identical, but the subgoal after applying induction may
be surprising:
\begin{isabelle}
1.\ \isasymAnd x\ args\ f.\isanewline
\ \ \ \ \ \ \isasymlbrakk args\ \isasymin \ lists\ (well_formed_gterm'\ arity\ \isasyminter\isanewline
\ \ \ \ \ \ \ \ \ \ \ \ \ \ \ \ \ \ \ \ \ \isacharbraceleft u.\ u\ \isasymin \ well_formed_gterm\ arity\isacharbraceright );\isanewline
\ \ \ \ \ \ \ length\ args\ =\ arity\ f\isasymrbrakk \isanewline
\ \ \ \ \ \ \isasymLongrightarrow \ Apply\ f\ args\ \isasymin \ well_formed_gterm\ arity%
\end{isabelle}
The induction hypothesis contains an application of \isa{lists}.  Using a
monotonic function in the inductive definition always has this effect.  The
subgoal may look uninviting, but fortunately a useful rewrite rule concerning
\isa{lists} is available:
\begin{isabelle}
lists\ (A\ \isasyminter \ B)\ =\ lists\ A\ \isasyminter \ lists\ B
\rulename{lists_Int_eq}
\end{isabelle}
%
Thanks to this default simplification rule, the induction hypothesis 
is quickly replaced by its two parts:
\begin{isabelle}
\ \ \ \ \ \ \ args\ \isasymin \ lists\ (well_formed_gterm'\ arity)\isanewline
\ \ \ \ \ \ \ args\ \isasymin \ lists\ (well_formed_gterm\ arity)
\end{isabelle}
Invoking the rule \isa{well_formed_gterm.step} completes the proof.  The
call to
\isa{auto} does all this work.

This example is typical of how monotonic functions can be used.  In
particular, a rewrite rule analogous to \isa{lists_Int_eq} holds in most
cases.  Monotonicity implies one direction of this set equality; we have
this theorem:
\begin{isabelle}
mono\ f\ \isasymLongrightarrow \ f\ (A\ \isasyminter \ B)\ \isasymsubseteq \
f\ A\ \isasyminter \ f\ B%
\rulename{mono_Int}
\end{isabelle}


To summarize: a universal quantifier in an introduction rule 
lets it refer to any number of instances of 
the inductively defined set.  A monotonic function in an introduction 
rule lets it refer to constructions over the inductively defined 
set.  Each element of an inductively defined set is created 
through finitely many applications of the introduction rules.  So each rule
must be positive, and never can it refer to \textit{infinitely} many
previous instances of the inductively defined set. 



\begin{exercise}
Prove this theorem, one direction of which was proved in
{\S}\ref{sec:rule-inversion} above.
\begin{isabelle}
\isacommand{lemma}\ gterms_Int_eq\ [simp]:\ "gterms\ (F\isasyminter G)\ =\
gterms\ F\ \isasyminter \ gterms\ G"\isanewline
\end{isabelle}
\end{exercise}


\begin{exercise}
A function mapping function symbols to their 
types is called a \textbf{signature}.  Given a type 
ranging over type symbols, we can represent a function's type by a
list of argument types paired with the result type. 
Complete this inductive definition:
\begin{isabelle}
\isacommand{consts}\ well_typed_gterm::\ "('f\ \isasymRightarrow \ 't\ list\ *\ 't)\ \isasymRightarrow \ ('f\ gterm\ *\ 't)set"\isanewline
\isacommand{inductive}\ "well_typed_gterm\ sig"\isanewline
\end{isabelle}
\end{exercise}


%
\begin{isabellebody}%
\def\isabellecontext{AB}%
%
\isamarkupsection{Case study: A context free grammar%
}
%
\begin{isamarkuptext}%
\label{sec:CFG}
Grammars are nothing but shorthands for inductive definitions of nonterminals
which represent sets of strings. For example, the production
$A \to B c$ is short for
\[ w \in B \Longrightarrow wc \in A \]
This section demonstrates this idea with a standard example
\cite[p.\ 81]{HopcroftUllman}, a grammar for generating all words with an
equal number of $a$'s and $b$'s:
\begin{eqnarray}
S &\to& \epsilon \mid b A \mid a B \nonumber\\
A &\to& a S \mid b A A \nonumber\\
B &\to& b S \mid a B B \nonumber
\end{eqnarray}
At the end we say a few words about the relationship of the formalization
and the text in the book~\cite[p.\ 81]{HopcroftUllman}.

We start by fixing the alphabet, which consists only of \isa{a}'s
and \isa{b}'s:%
\end{isamarkuptext}%
\isacommand{datatype}\ alfa\ {\isacharequal}\ a\ {\isacharbar}\ b%
\begin{isamarkuptext}%
\noindent
For convenience we include the following easy lemmas as simplification rules:%
\end{isamarkuptext}%
\isacommand{lemma}\ {\isacharbrackleft}simp{\isacharbrackright}{\isacharcolon}\ {\isachardoublequote}{\isacharparenleft}x\ {\isasymnoteq}\ a{\isacharparenright}\ {\isacharequal}\ {\isacharparenleft}x\ {\isacharequal}\ b{\isacharparenright}\ {\isasymand}\ {\isacharparenleft}x\ {\isasymnoteq}\ b{\isacharparenright}\ {\isacharequal}\ {\isacharparenleft}x\ {\isacharequal}\ a{\isacharparenright}{\isachardoublequote}\isanewline
\isacommand{apply}{\isacharparenleft}case{\isacharunderscore}tac\ x{\isacharparenright}\isanewline
\isacommand{by}{\isacharparenleft}auto{\isacharparenright}%
\begin{isamarkuptext}%
\noindent
Words over this alphabet are of type \isa{alfa\ list}, and
the three nonterminals are declare as sets of such words:%
\end{isamarkuptext}%
\isacommand{consts}\ S\ {\isacharcolon}{\isacharcolon}\ {\isachardoublequote}alfa\ list\ set{\isachardoublequote}\isanewline
\ \ \ \ \ \ \ A\ {\isacharcolon}{\isacharcolon}\ {\isachardoublequote}alfa\ list\ set{\isachardoublequote}\isanewline
\ \ \ \ \ \ \ B\ {\isacharcolon}{\isacharcolon}\ {\isachardoublequote}alfa\ list\ set{\isachardoublequote}%
\begin{isamarkuptext}%
\noindent
The above productions are recast as a \emph{simultaneous} inductive
definition\index{inductive definition!simultaneous}
of \isa{S}, \isa{A} and \isa{B}:%
\end{isamarkuptext}%
\isacommand{inductive}\ S\ A\ B\isanewline
\isakeyword{intros}\isanewline
\ \ {\isachardoublequote}{\isacharbrackleft}{\isacharbrackright}\ {\isasymin}\ S{\isachardoublequote}\isanewline
\ \ {\isachardoublequote}w\ {\isasymin}\ A\ {\isasymLongrightarrow}\ b{\isacharhash}w\ {\isasymin}\ S{\isachardoublequote}\isanewline
\ \ {\isachardoublequote}w\ {\isasymin}\ B\ {\isasymLongrightarrow}\ a{\isacharhash}w\ {\isasymin}\ S{\isachardoublequote}\isanewline
\isanewline
\ \ {\isachardoublequote}w\ {\isasymin}\ S\ \ \ \ \ \ \ \ {\isasymLongrightarrow}\ a{\isacharhash}w\ \ \ {\isasymin}\ A{\isachardoublequote}\isanewline
\ \ {\isachardoublequote}{\isasymlbrakk}\ v{\isasymin}A{\isacharsemicolon}\ w{\isasymin}A\ {\isasymrbrakk}\ {\isasymLongrightarrow}\ b{\isacharhash}v{\isacharat}w\ {\isasymin}\ A{\isachardoublequote}\isanewline
\isanewline
\ \ {\isachardoublequote}w\ {\isasymin}\ S\ \ \ \ \ \ \ \ \ \ \ \ {\isasymLongrightarrow}\ b{\isacharhash}w\ \ \ {\isasymin}\ B{\isachardoublequote}\isanewline
\ \ {\isachardoublequote}{\isasymlbrakk}\ v\ {\isasymin}\ B{\isacharsemicolon}\ w\ {\isasymin}\ B\ {\isasymrbrakk}\ {\isasymLongrightarrow}\ a{\isacharhash}v{\isacharat}w\ {\isasymin}\ B{\isachardoublequote}%
\begin{isamarkuptext}%
\noindent
First we show that all words in \isa{S} contain the same number of \isa{a}'s and \isa{b}'s. Since the definition of \isa{S} is by simultaneous
induction, so is this proof: we show at the same time that all words in
\isa{A} contain one more \isa{a} than \isa{b} and all words in \isa{B} contains one more \isa{b} than \isa{a}.%
\end{isamarkuptext}%
\isacommand{lemma}\ correctness{\isacharcolon}\isanewline
\ \ {\isachardoublequote}{\isacharparenleft}w\ {\isasymin}\ S\ {\isasymlongrightarrow}\ size{\isacharbrackleft}x{\isasymin}w{\isachardot}\ x{\isacharequal}a{\isacharbrackright}\ {\isacharequal}\ size{\isacharbrackleft}x{\isasymin}w{\isachardot}\ x{\isacharequal}b{\isacharbrackright}{\isacharparenright}\ \ \ \ \ {\isasymand}\isanewline
\ \ \ {\isacharparenleft}w\ {\isasymin}\ A\ {\isasymlongrightarrow}\ size{\isacharbrackleft}x{\isasymin}w{\isachardot}\ x{\isacharequal}a{\isacharbrackright}\ {\isacharequal}\ size{\isacharbrackleft}x{\isasymin}w{\isachardot}\ x{\isacharequal}b{\isacharbrackright}\ {\isacharplus}\ {\isadigit{1}}{\isacharparenright}\ {\isasymand}\isanewline
\ \ \ {\isacharparenleft}w\ {\isasymin}\ B\ {\isasymlongrightarrow}\ size{\isacharbrackleft}x{\isasymin}w{\isachardot}\ x{\isacharequal}b{\isacharbrackright}\ {\isacharequal}\ size{\isacharbrackleft}x{\isasymin}w{\isachardot}\ x{\isacharequal}a{\isacharbrackright}\ {\isacharplus}\ {\isadigit{1}}{\isacharparenright}{\isachardoublequote}%
\begin{isamarkuptxt}%
\noindent
These propositions are expressed with the help of the predefined \isa{filter} function on lists, which has the convenient syntax \isa{{\isacharbrackleft}x{\isasymin}xs{\isachardot}\ P\ x{\isacharbrackright}}, the list of all elements \isa{x} in \isa{xs} such that \isa{P\ x}
holds. Remember that on lists \isa{size} and \isa{size} are synonymous.

The proof itself is by rule induction and afterwards automatic:%
\end{isamarkuptxt}%
\isacommand{apply}{\isacharparenleft}rule\ S{\isacharunderscore}A{\isacharunderscore}B{\isachardot}induct{\isacharparenright}\isanewline
\isacommand{by}{\isacharparenleft}auto{\isacharparenright}%
\begin{isamarkuptext}%
\noindent
This may seem surprising at first, and is indeed an indication of the power
of inductive definitions. But it is also quite straightforward. For example,
consider the production $A \to b A A$: if $v,w \in A$ and the elements of $A$
contain one more $a$ than $b$'s, then $bvw$ must again contain one more $a$
than $b$'s.

As usual, the correctness of syntactic descriptions is easy, but completeness
is hard: does \isa{S} contain \emph{all} words with an equal number of
\isa{a}'s and \isa{b}'s? It turns out that this proof requires the
following little lemma: every string with two more \isa{a}'s than \isa{b}'s can be cut somehwere such that each half has one more \isa{a} than
\isa{b}. This is best seen by imagining counting the difference between the
number of \isa{a}'s and \isa{b}'s starting at the left end of the
word. We start with 0 and end (at the right end) with 2. Since each move to the
right increases or decreases the difference by 1, we must have passed through
1 on our way from 0 to 2. Formally, we appeal to the following discrete
intermediate value theorem \isa{nat{\isadigit{0}}{\isacharunderscore}intermed{\isacharunderscore}int{\isacharunderscore}val}
\begin{isabelle}%
\ \ \ \ \ {\isasymforall}i{\isachardot}\ i\ {\isacharless}\ n\ {\isasymlongrightarrow}\ {\isasymbar}f\ {\isacharparenleft}i\ {\isacharplus}\ {\isadigit{1}}{\isacharparenright}\ {\isacharminus}\ f\ i{\isasymbar}\ {\isasymle}\ {\isacharhash}{\isadigit{1}}\ {\isasymLongrightarrow}\isanewline
\ \ \ \ \ f\ {\isadigit{0}}\ {\isasymle}\ k\ {\isasymLongrightarrow}\ k\ {\isasymle}\ f\ n\ {\isasymLongrightarrow}\ {\isasymexists}i{\isachardot}\ i\ {\isasymle}\ n\ {\isasymand}\ f\ i\ {\isacharequal}\ k%
\end{isabelle}
where \isa{f} is of type \isa{nat\ {\isasymRightarrow}\ int}, \isa{int} are the integers,
\isa{abs} is the absolute value function, and \isa{{\isacharhash}{\isadigit{1}}} is the
integer 1 (see \S\ref{sec:numbers}).

First we show that the our specific function, the difference between the
numbers of \isa{a}'s and \isa{b}'s, does indeed only change by 1 in every
move to the right. At this point we also start generalizing from \isa{a}'s
and \isa{b}'s to an arbitrary property \isa{P}. Otherwise we would have
to prove the desired lemma twice, once as stated above and once with the
roles of \isa{a}'s and \isa{b}'s interchanged.%
\end{isamarkuptext}%
\isacommand{lemma}\ step{\isadigit{1}}{\isacharcolon}\ {\isachardoublequote}{\isasymforall}i\ {\isacharless}\ size\ w{\isachardot}\isanewline
\ \ abs{\isacharparenleft}{\isacharparenleft}int{\isacharparenleft}size{\isacharbrackleft}x{\isasymin}take\ {\isacharparenleft}i{\isacharplus}{\isadigit{1}}{\isacharparenright}\ w{\isachardot}\ \ P\ x{\isacharbrackright}{\isacharparenright}\ {\isacharminus}\isanewline
\ \ \ \ \ \ \ int{\isacharparenleft}size{\isacharbrackleft}x{\isasymin}take\ {\isacharparenleft}i{\isacharplus}{\isadigit{1}}{\isacharparenright}\ w{\isachardot}\ {\isasymnot}P\ x{\isacharbrackright}{\isacharparenright}{\isacharparenright}\isanewline
\ \ \ \ \ \ {\isacharminus}\isanewline
\ \ \ \ \ \ {\isacharparenleft}int{\isacharparenleft}size{\isacharbrackleft}x{\isasymin}take\ i\ w{\isachardot}\ \ P\ x{\isacharbrackright}{\isacharparenright}\ {\isacharminus}\isanewline
\ \ \ \ \ \ \ int{\isacharparenleft}size{\isacharbrackleft}x{\isasymin}take\ i\ w{\isachardot}\ {\isasymnot}P\ x{\isacharbrackright}{\isacharparenright}{\isacharparenright}{\isacharparenright}\ {\isasymle}\ {\isacharhash}{\isadigit{1}}{\isachardoublequote}%
\begin{isamarkuptxt}%
\noindent
The lemma is a bit hard to read because of the coercion function
\isa{{\isachardoublequote}int{\isacharcolon}{\isacharcolon}nat\ {\isasymRightarrow}\ int{\isachardoublequote}}. It is required because \isa{size} returns
a natural number, but \isa{{\isacharminus}} on \isa{nat} will do the wrong thing.
Function \isa{take} is predefined and \isa{take\ i\ xs} is the prefix of
length \isa{i} of \isa{xs}; below we als need \isa{drop\ i\ xs}, which
is what remains after that prefix has been dropped from \isa{xs}.

The proof is by induction on \isa{w}, with a trivial base case, and a not
so trivial induction step. Since it is essentially just arithmetic, we do not
discuss it.%
\end{isamarkuptxt}%
\isacommand{apply}{\isacharparenleft}induct\ w{\isacharparenright}\isanewline
\ \isacommand{apply}{\isacharparenleft}simp{\isacharparenright}\isanewline
\isacommand{by}{\isacharparenleft}force\ simp\ add{\isacharcolon}zabs{\isacharunderscore}def\ take{\isacharunderscore}Cons\ split{\isacharcolon}nat{\isachardot}split\ if{\isacharunderscore}splits{\isacharparenright}%
\begin{isamarkuptext}%
Finally we come to the above mentioned lemma about cutting a word with two
more elements of one sort than of the other sort into two halves:%
\end{isamarkuptext}%
\isacommand{lemma}\ part{\isadigit{1}}{\isacharcolon}\isanewline
\ {\isachardoublequote}size{\isacharbrackleft}x{\isasymin}w{\isachardot}\ P\ x{\isacharbrackright}\ {\isacharequal}\ size{\isacharbrackleft}x{\isasymin}w{\isachardot}\ {\isasymnot}P\ x{\isacharbrackright}{\isacharplus}{\isadigit{2}}\ {\isasymLongrightarrow}\isanewline
\ \ {\isasymexists}i{\isasymle}size\ w{\isachardot}\ size{\isacharbrackleft}x{\isasymin}take\ i\ w{\isachardot}\ P\ x{\isacharbrackright}\ {\isacharequal}\ size{\isacharbrackleft}x{\isasymin}take\ i\ w{\isachardot}\ {\isasymnot}P\ x{\isacharbrackright}{\isacharplus}{\isadigit{1}}{\isachardoublequote}%
\begin{isamarkuptxt}%
\noindent
This is proved with the help of the intermediate value theorem, instantiated
appropriately and with its first premise disposed of by lemma
\isa{step{\isadigit{1}}}.%
\end{isamarkuptxt}%
\isacommand{apply}{\isacharparenleft}insert\ nat{\isadigit{0}}{\isacharunderscore}intermed{\isacharunderscore}int{\isacharunderscore}val{\isacharbrackleft}OF\ step{\isadigit{1}}{\isacharcomma}\ of\ {\isachardoublequote}P{\isachardoublequote}\ {\isachardoublequote}w{\isachardoublequote}\ {\isachardoublequote}{\isacharhash}{\isadigit{1}}{\isachardoublequote}{\isacharbrackright}{\isacharparenright}\isanewline
\isacommand{apply}\ simp\isanewline
\isacommand{by}{\isacharparenleft}simp\ del{\isacharcolon}int{\isacharunderscore}Suc\ add{\isacharcolon}zdiff{\isacharunderscore}eq{\isacharunderscore}eq\ sym{\isacharbrackleft}OF\ int{\isacharunderscore}Suc{\isacharbrackright}{\isacharparenright}%
\begin{isamarkuptext}%
\noindent
The additional lemmas are needed to mediate between \isa{nat} and \isa{int}.

Lemma \isa{part{\isadigit{1}}} tells us only about the prefix \isa{take\ i\ w}.
The suffix \isa{drop\ i\ w} is dealt with in the following easy lemma:%
\end{isamarkuptext}%
\isacommand{lemma}\ part{\isadigit{2}}{\isacharcolon}\isanewline
\ \ {\isachardoublequote}{\isasymlbrakk}size{\isacharbrackleft}x{\isasymin}take\ i\ w\ {\isacharat}\ drop\ i\ w{\isachardot}\ P\ x{\isacharbrackright}\ {\isacharequal}\isanewline
\ \ \ \ size{\isacharbrackleft}x{\isasymin}take\ i\ w\ {\isacharat}\ drop\ i\ w{\isachardot}\ {\isasymnot}P\ x{\isacharbrackright}{\isacharplus}{\isadigit{2}}{\isacharsemicolon}\isanewline
\ \ \ \ size{\isacharbrackleft}x{\isasymin}take\ i\ w{\isachardot}\ P\ x{\isacharbrackright}\ {\isacharequal}\ size{\isacharbrackleft}x{\isasymin}take\ i\ w{\isachardot}\ {\isasymnot}P\ x{\isacharbrackright}{\isacharplus}{\isadigit{1}}{\isasymrbrakk}\isanewline
\ \ \ {\isasymLongrightarrow}\ size{\isacharbrackleft}x{\isasymin}drop\ i\ w{\isachardot}\ P\ x{\isacharbrackright}\ {\isacharequal}\ size{\isacharbrackleft}x{\isasymin}drop\ i\ w{\isachardot}\ {\isasymnot}P\ x{\isacharbrackright}{\isacharplus}{\isadigit{1}}{\isachardoublequote}\isanewline
\isacommand{by}{\isacharparenleft}simp\ del{\isacharcolon}append{\isacharunderscore}take{\isacharunderscore}drop{\isacharunderscore}id{\isacharparenright}%
\begin{isamarkuptext}%
\noindent
Lemma \isa{append{\isacharunderscore}take{\isacharunderscore}drop{\isacharunderscore}id}, \isa{take\ n\ xs\ {\isacharat}\ drop\ n\ xs\ {\isacharequal}\ xs},
which is generally useful, needs to be disabled for once.

To dispose of trivial cases automatically, the rules of the inductive
definition are declared simplification rules:%
\end{isamarkuptext}%
\isacommand{declare}\ S{\isacharunderscore}A{\isacharunderscore}B{\isachardot}intros{\isacharbrackleft}simp{\isacharbrackright}%
\begin{isamarkuptext}%
\noindent
This could have been done earlier but was not necessary so far.

The completeness theorem tells us that if a word has the same number of
\isa{a}'s and \isa{b}'s, then it is in \isa{S}, and similarly and
simultaneously for \isa{A} and \isa{B}:%
\end{isamarkuptext}%
\isacommand{theorem}\ completeness{\isacharcolon}\isanewline
\ \ {\isachardoublequote}{\isacharparenleft}size{\isacharbrackleft}x{\isasymin}w{\isachardot}\ x{\isacharequal}a{\isacharbrackright}\ {\isacharequal}\ size{\isacharbrackleft}x{\isasymin}w{\isachardot}\ x{\isacharequal}b{\isacharbrackright}\ \ \ \ \ {\isasymlongrightarrow}\ w\ {\isasymin}\ S{\isacharparenright}\ {\isasymand}\isanewline
\ \ \ {\isacharparenleft}size{\isacharbrackleft}x{\isasymin}w{\isachardot}\ x{\isacharequal}a{\isacharbrackright}\ {\isacharequal}\ size{\isacharbrackleft}x{\isasymin}w{\isachardot}\ x{\isacharequal}b{\isacharbrackright}\ {\isacharplus}\ {\isadigit{1}}\ {\isasymlongrightarrow}\ w\ {\isasymin}\ A{\isacharparenright}\ {\isasymand}\isanewline
\ \ \ {\isacharparenleft}size{\isacharbrackleft}x{\isasymin}w{\isachardot}\ x{\isacharequal}b{\isacharbrackright}\ {\isacharequal}\ size{\isacharbrackleft}x{\isasymin}w{\isachardot}\ x{\isacharequal}a{\isacharbrackright}\ {\isacharplus}\ {\isadigit{1}}\ {\isasymlongrightarrow}\ w\ {\isasymin}\ B{\isacharparenright}{\isachardoublequote}%
\begin{isamarkuptxt}%
\noindent
The proof is by induction on \isa{w}. Structural induction would fail here
because, as we can see from the grammar, we need to make bigger steps than
merely appending a single letter at the front. Hence we induct on the length
of \isa{w}, using the induction rule \isa{length{\isacharunderscore}induct}:%
\end{isamarkuptxt}%
\isacommand{apply}{\isacharparenleft}induct{\isacharunderscore}tac\ w\ rule{\isacharcolon}\ length{\isacharunderscore}induct{\isacharparenright}%
\begin{isamarkuptxt}%
\noindent
The \isa{rule} parameter tells \isa{induct{\isacharunderscore}tac} explicitly which induction
rule to use. For details see \S\ref{sec:complete-ind} below.
In this case the result is that we may assume the lemma already
holds for all words shorter than \isa{w}.

The proof continues with a case distinction on \isa{w},
i.e.\ if \isa{w} is empty or not.%
\end{isamarkuptxt}%
\isacommand{apply}{\isacharparenleft}case{\isacharunderscore}tac\ w{\isacharparenright}\isanewline
\ \isacommand{apply}{\isacharparenleft}simp{\isacharunderscore}all{\isacharparenright}%
\begin{isamarkuptxt}%
\noindent
Simplification disposes of the base case and leaves only two step
cases to be proved:
if \isa{w\ {\isacharequal}\ a\ {\isacharhash}\ v} and \isa{length\ {\isacharbrackleft}x{\isasymin}v\ {\isachardot}\ x\ {\isacharequal}\ a{\isacharbrackright}\ {\isacharequal}\ length\ {\isacharbrackleft}x{\isasymin}v\ {\isachardot}\ x\ {\isacharequal}\ b{\isacharbrackright}\ {\isacharplus}\ {\isadigit{2}}} then
\isa{b\ {\isacharhash}\ v\ {\isasymin}\ A}, and similarly for \isa{w\ {\isacharequal}\ b\ {\isacharhash}\ v}.
We only consider the first case in detail.

After breaking the conjuction up into two cases, we can apply
\isa{part{\isadigit{1}}} to the assumption that \isa{w} contains two more \isa{a}'s than \isa{b}'s.%
\end{isamarkuptxt}%
\isacommand{apply}{\isacharparenleft}rule\ conjI{\isacharparenright}\isanewline
\ \isacommand{apply}{\isacharparenleft}clarify{\isacharparenright}\isanewline
\ \isacommand{apply}{\isacharparenleft}frule\ part{\isadigit{1}}{\isacharbrackleft}of\ {\isachardoublequote}{\isasymlambda}x{\isachardot}\ x{\isacharequal}a{\isachardoublequote}{\isacharcomma}\ simplified{\isacharbrackright}{\isacharparenright}\isanewline
\ \isacommand{apply}{\isacharparenleft}erule\ exE{\isacharparenright}\isanewline
\ \isacommand{apply}{\isacharparenleft}erule\ conjE{\isacharparenright}%
\begin{isamarkuptxt}%
\noindent
This yields an index \isa{i\ {\isasymle}\ length\ v} such that
\isa{length\ {\isacharbrackleft}x{\isasymin}take\ i\ v\ {\isachardot}\ x\ {\isacharequal}\ a{\isacharbrackright}\ {\isacharequal}\ length\ {\isacharbrackleft}x{\isasymin}take\ i\ v\ {\isachardot}\ x\ {\isacharequal}\ b{\isacharbrackright}\ {\isacharplus}\ {\isadigit{1}}}.
With the help of \isa{part{\isadigit{1}}} it follows that
\isa{length\ {\isacharbrackleft}x{\isasymin}drop\ i\ v\ {\isachardot}\ x\ {\isacharequal}\ a{\isacharbrackright}\ {\isacharequal}\ length\ {\isacharbrackleft}x{\isasymin}drop\ i\ v\ {\isachardot}\ x\ {\isacharequal}\ b{\isacharbrackright}\ {\isacharplus}\ {\isadigit{1}}}.%
\end{isamarkuptxt}%
\ \isacommand{apply}{\isacharparenleft}drule\ part{\isadigit{2}}{\isacharbrackleft}of\ {\isachardoublequote}{\isasymlambda}x{\isachardot}\ x{\isacharequal}a{\isachardoublequote}{\isacharcomma}\ simplified{\isacharbrackright}{\isacharparenright}\isanewline
\ \ \isacommand{apply}{\isacharparenleft}assumption{\isacharparenright}%
\begin{isamarkuptxt}%
\noindent
Now it is time to decompose \isa{v} in the conclusion \isa{b\ {\isacharhash}\ v\ {\isasymin}\ A}
into \isa{take\ i\ v\ {\isacharat}\ drop\ i\ v},
after which the appropriate rule of the grammar reduces the goal
to the two subgoals \isa{take\ i\ v\ {\isasymin}\ A} and \isa{drop\ i\ v\ {\isasymin}\ A}:%
\end{isamarkuptxt}%
\ \isacommand{apply}{\isacharparenleft}rule{\isacharunderscore}tac\ n{\isadigit{1}}{\isacharequal}i\ \isakeyword{and}\ t{\isacharequal}v\ \isakeyword{in}\ subst{\isacharbrackleft}OF\ append{\isacharunderscore}take{\isacharunderscore}drop{\isacharunderscore}id{\isacharbrackright}{\isacharparenright}\isanewline
\ \isacommand{apply}{\isacharparenleft}rule\ S{\isacharunderscore}A{\isacharunderscore}B{\isachardot}intros{\isacharparenright}%
\begin{isamarkuptxt}%
\noindent
Both subgoals follow from the induction hypothesis because both \isa{take\ i\ v} and \isa{drop\ i\ v} are shorter than \isa{w}:%
\end{isamarkuptxt}%
\ \ \isacommand{apply}{\isacharparenleft}force\ simp\ add{\isacharcolon}\ min{\isacharunderscore}less{\isacharunderscore}iff{\isacharunderscore}disj{\isacharparenright}\isanewline
\ \isacommand{apply}{\isacharparenleft}force\ split\ add{\isacharcolon}\ nat{\isacharunderscore}diff{\isacharunderscore}split{\isacharparenright}%
\begin{isamarkuptxt}%
\noindent
Note that the variables \isa{n{\isadigit{1}}} and \isa{t} referred to in the
substitution step above come from the derived theorem \isa{subst{\isacharbrackleft}OF\ append{\isacharunderscore}take{\isacharunderscore}drop{\isacharunderscore}id{\isacharbrackright}}.

The case \isa{w\ {\isacharequal}\ b\ {\isacharhash}\ v} is proved completely analogously:%
\end{isamarkuptxt}%
\isacommand{apply}{\isacharparenleft}clarify{\isacharparenright}\isanewline
\isacommand{apply}{\isacharparenleft}frule\ part{\isadigit{1}}{\isacharbrackleft}of\ {\isachardoublequote}{\isasymlambda}x{\isachardot}\ x{\isacharequal}b{\isachardoublequote}{\isacharcomma}\ simplified{\isacharbrackright}{\isacharparenright}\isanewline
\isacommand{apply}{\isacharparenleft}erule\ exE{\isacharparenright}\isanewline
\isacommand{apply}{\isacharparenleft}erule\ conjE{\isacharparenright}\isanewline
\isacommand{apply}{\isacharparenleft}drule\ part{\isadigit{2}}{\isacharbrackleft}of\ {\isachardoublequote}{\isasymlambda}x{\isachardot}\ x{\isacharequal}b{\isachardoublequote}{\isacharcomma}\ simplified{\isacharbrackright}{\isacharparenright}\isanewline
\ \isacommand{apply}{\isacharparenleft}assumption{\isacharparenright}\isanewline
\isacommand{apply}{\isacharparenleft}rule{\isacharunderscore}tac\ n{\isadigit{1}}{\isacharequal}i\ \isakeyword{and}\ t{\isacharequal}v\ \isakeyword{in}\ subst{\isacharbrackleft}OF\ append{\isacharunderscore}take{\isacharunderscore}drop{\isacharunderscore}id{\isacharbrackright}{\isacharparenright}\isanewline
\isacommand{apply}{\isacharparenleft}rule\ S{\isacharunderscore}A{\isacharunderscore}B{\isachardot}intros{\isacharparenright}\isanewline
\ \isacommand{apply}{\isacharparenleft}force\ simp\ add{\isacharcolon}min{\isacharunderscore}less{\isacharunderscore}iff{\isacharunderscore}disj{\isacharparenright}\isanewline
\isacommand{by}{\isacharparenleft}force\ simp\ add{\isacharcolon}min{\isacharunderscore}less{\isacharunderscore}iff{\isacharunderscore}disj\ split\ add{\isacharcolon}\ nat{\isacharunderscore}diff{\isacharunderscore}split{\isacharparenright}%
\begin{isamarkuptext}%
We conclude this section with a comparison of the above proof and the one
in the textbook \cite[p.\ 81]{HopcroftUllman}. For a start, the texbook
grammar, for no good reason, excludes the empty word, which complicates
matters just a little bit because we now have 8 instead of our 7 productions.

More importantly, the proof itself is different: rather than separating the
two directions, they perform one induction on the length of a word. This
deprives them of the beauty of rule induction and in the easy direction
(correctness) their reasoning is more detailed than our \isa{auto}. For the
hard part (completeness), they consider just one of the cases that our \isa{simp{\isacharunderscore}all} disposes of automatically. Then they conclude the proof by saying
about the remaining cases: ``We do this in a manner similar to our method of
proof for part (1); this part is left to the reader''. But this is precisely
the part that requires the intermediate value theorem and thus is not at all
similar to the other cases (which are automatic in Isabelle). We conclude
that the authors are at least cavalier about this point and may even have
overlooked the slight difficulty lurking in the omitted cases. This is not
atypical for pen-and-paper proofs, once analysed in detail.%
\end{isamarkuptext}%
\end{isabellebody}%
%%% Local Variables:
%%% mode: latex
%%% TeX-master: "root"
%%% End:


\index{inductive definition|)}
