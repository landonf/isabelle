\chapter{Inductively Defined Sets} \label{chap:inductive}
\index{inductive definitions|(}

This chapter is dedicated to the most important definition principle after
recursive functions and datatypes: inductively defined sets.

We start with a simple example: the set of even numbers.  A slightly more
complicated example, the reflexive transitive closure, is the subject of
{\S}\ref{sec:rtc}. In particular, some standard induction heuristics are
discussed. Advanced forms of inductive definitions are discussed in
{\S}\ref{sec:adv-ind-def}. To demonstrate the versatility of inductive
definitions, the chapter closes with a case study from the realm of
context-free grammars. The first two sections are required reading for anybody
interested in mathematical modelling.

\begin{warn}
Predicates can also be defined inductively.
See {\S}\ref{sec:ind-predicates}.
\end{warn}

%
\begin{isabellebody}%
\def\isabellecontext{Even}%
\isanewline
\isacommand{theory}\ Even\ {\isacharequal}\ Main{\isacharcolon}\isanewline
\isanewline
\isanewline
\isamarkupfalse%
\isacommand{consts}\ even\ {\isacharcolon}{\isacharcolon}\ {\isachardoublequote}nat\ set{\isachardoublequote}\isanewline
\isamarkupfalse%
\isacommand{inductive}\ even\isanewline
\isakeyword{intros}\isanewline
zero{\isacharbrackleft}intro{\isacharbang}{\isacharbrackright}{\isacharcolon}\ {\isachardoublequote}{\isadigit{0}}\ {\isasymin}\ even{\isachardoublequote}\isanewline
step{\isacharbrackleft}intro{\isacharbang}{\isacharbrackright}{\isacharcolon}\ {\isachardoublequote}n\ {\isasymin}\ even\ {\isasymLongrightarrow}\ {\isacharparenleft}Suc\ {\isacharparenleft}Suc\ n{\isacharparenright}{\isacharparenright}\ {\isasymin}\ even{\isachardoublequote}\isamarkupfalse%
%
\begin{isamarkuptext}%
An inductive definition consists of introduction rules. 

\begin{isabelle}%
\ \ \ \ \ n\ {\isasymin}\ Even{\isachardot}even\ {\isasymLongrightarrow}\ Suc\ {\isacharparenleft}Suc\ n{\isacharparenright}\ {\isasymin}\ Even{\isachardot}even%
\end{isabelle}
\rulename{even.step}

\begin{isabelle}%
\ \ \ \ \ {\isasymlbrakk}xa\ {\isasymin}\ Even{\isachardot}even{\isacharsemicolon}\ P\ {\isadigit{0}}{\isacharsemicolon}\ {\isasymAnd}n{\isachardot}\ {\isasymlbrakk}n\ {\isasymin}\ Even{\isachardot}even{\isacharsemicolon}\ P\ n{\isasymrbrakk}\ {\isasymLongrightarrow}\ P\ {\isacharparenleft}Suc\ {\isacharparenleft}Suc\ n{\isacharparenright}{\isacharparenright}{\isasymrbrakk}\isanewline
\isaindent{\ \ \ \ \ }{\isasymLongrightarrow}\ P\ xa%
\end{isabelle}
\rulename{even.induct}

Attributes can be given to the introduction rules.  Here both rules are
specified as \isa{intro!}

Our first lemma states that numbers of the form $2\times k$ are even.%
\end{isamarkuptext}%
\isamarkuptrue%
\isacommand{lemma}\ two{\isacharunderscore}times{\isacharunderscore}even{\isacharbrackleft}intro{\isacharbang}{\isacharbrackright}{\isacharcolon}\ {\isachardoublequote}{\isadigit{2}}{\isacharasterisk}k\ {\isasymin}\ even{\isachardoublequote}\isanewline
\isamarkupfalse%
\isamarkupfalse%
\isamarkuptrue%
\isamarkupfalse%
\isamarkupfalse%
%
\begin{isamarkuptext}%
Our goal is to prove the equivalence between the traditional definition
of even (using the divides relation) and our inductive definition.  Half of
this equivalence is trivial using the lemma just proved, whose \isa{intro!}
attribute ensures it will be applied automatically.%
\end{isamarkuptext}%
\isamarkuptrue%
\isacommand{lemma}\ dvd{\isacharunderscore}imp{\isacharunderscore}even{\isacharcolon}\ {\isachardoublequote}{\isadigit{2}}\ dvd\ n\ {\isasymLongrightarrow}\ n\ {\isasymin}\ even{\isachardoublequote}\isanewline
\isamarkupfalse%
\isamarkupfalse%
%
\begin{isamarkuptext}%
our first rule induction!%
\end{isamarkuptext}%
\isamarkuptrue%
\isacommand{lemma}\ even{\isacharunderscore}imp{\isacharunderscore}dvd{\isacharcolon}\ {\isachardoublequote}n\ {\isasymin}\ even\ {\isasymLongrightarrow}\ {\isadigit{2}}\ dvd\ n{\isachardoublequote}\isanewline
\isamarkupfalse%
\isamarkupfalse%
\isamarkuptrue%
\isamarkupfalse%
\isamarkuptrue%
\isamarkupfalse%
\isamarkuptrue%
\isamarkupfalse%
\isamarkupfalse%
%
\begin{isamarkuptext}%
no iff-attribute because we don't always want to use it%
\end{isamarkuptext}%
\isamarkuptrue%
\isacommand{theorem}\ even{\isacharunderscore}iff{\isacharunderscore}dvd{\isacharcolon}\ {\isachardoublequote}{\isacharparenleft}n\ {\isasymin}\ even{\isacharparenright}\ {\isacharequal}\ {\isacharparenleft}{\isadigit{2}}\ dvd\ n{\isacharparenright}{\isachardoublequote}\isanewline
\isamarkupfalse%
\isamarkupfalse%
%
\begin{isamarkuptext}%
this result ISN'T inductive...%
\end{isamarkuptext}%
\isamarkuptrue%
\isacommand{lemma}\ Suc{\isacharunderscore}Suc{\isacharunderscore}even{\isacharunderscore}imp{\isacharunderscore}even{\isacharcolon}\ {\isachardoublequote}Suc\ {\isacharparenleft}Suc\ n{\isacharparenright}\ {\isasymin}\ even\ {\isasymLongrightarrow}\ n\ {\isasymin}\ even{\isachardoublequote}\isanewline
\isamarkupfalse%
\isamarkupfalse%
\isamarkuptrue%
\isamarkupfalse%
%
\begin{isamarkuptext}%
...so we need an inductive lemma...%
\end{isamarkuptext}%
\isamarkuptrue%
\isacommand{lemma}\ even{\isacharunderscore}imp{\isacharunderscore}even{\isacharunderscore}minus{\isacharunderscore}{\isadigit{2}}{\isacharcolon}\ {\isachardoublequote}n\ {\isasymin}\ even\ {\isasymLongrightarrow}\ n\ {\isacharminus}\ {\isadigit{2}}\ {\isasymin}\ even{\isachardoublequote}\isanewline
\isamarkupfalse%
\isamarkupfalse%
\isamarkuptrue%
\isamarkupfalse%
\isamarkupfalse%
%
\begin{isamarkuptext}%
...and prove it in a separate step%
\end{isamarkuptext}%
\isamarkuptrue%
\isacommand{lemma}\ Suc{\isacharunderscore}Suc{\isacharunderscore}even{\isacharunderscore}imp{\isacharunderscore}even{\isacharcolon}\ {\isachardoublequote}Suc\ {\isacharparenleft}Suc\ n{\isacharparenright}\ {\isasymin}\ even\ {\isasymLongrightarrow}\ n\ {\isasymin}\ even{\isachardoublequote}\isanewline
\isamarkupfalse%
\isanewline
\isanewline
\isamarkupfalse%
\isacommand{lemma}\ {\isacharbrackleft}iff{\isacharbrackright}{\isacharcolon}\ {\isachardoublequote}{\isacharparenleft}{\isacharparenleft}Suc\ {\isacharparenleft}Suc\ n{\isacharparenright}{\isacharparenright}\ {\isasymin}\ even{\isacharparenright}\ {\isacharequal}\ {\isacharparenleft}n\ {\isasymin}\ even{\isacharparenright}{\isachardoublequote}\isanewline
\isamarkupfalse%
\isanewline
\isamarkupfalse%
\isacommand{end}\isanewline
\isanewline
\isamarkupfalse%
\end{isabellebody}%
%%% Local Variables:
%%% mode: latex
%%% TeX-master: "root"
%%% End:

%
\begin{isabellebody}%
\def\isabellecontext{Mutual}%
\isamarkupfalse%
%
\isamarkupsubsection{Mutually Inductive Definitions%
}
\isamarkuptrue%
%
\begin{isamarkuptext}%
Just as there are datatypes defined by mutual recursion, there are sets defined
by mutual induction. As a trivial example we consider the even and odd
natural numbers:%
\end{isamarkuptext}%
\isamarkuptrue%
\isacommand{consts}\ even\ {\isacharcolon}{\isacharcolon}\ {\isachardoublequote}nat\ set{\isachardoublequote}\isanewline
\ \ \ \ \ \ \ odd\ \ {\isacharcolon}{\isacharcolon}\ {\isachardoublequote}nat\ set{\isachardoublequote}\isanewline
\isanewline
\isamarkupfalse%
\isacommand{inductive}\ even\ odd\isanewline
\isakeyword{intros}\isanewline
zero{\isacharcolon}\ \ {\isachardoublequote}{\isadigit{0}}\ {\isasymin}\ even{\isachardoublequote}\isanewline
evenI{\isacharcolon}\ {\isachardoublequote}n\ {\isasymin}\ odd\ {\isasymLongrightarrow}\ Suc\ n\ {\isasymin}\ even{\isachardoublequote}\isanewline
oddI{\isacharcolon}\ \ {\isachardoublequote}n\ {\isasymin}\ even\ {\isasymLongrightarrow}\ Suc\ n\ {\isasymin}\ odd{\isachardoublequote}\isamarkupfalse%
%
\begin{isamarkuptext}%
\noindent
The mutually inductive definition of multiple sets is no different from
that of a single set, except for induction: just as for mutually recursive
datatypes, induction needs to involve all the simultaneously defined sets. In
the above case, the induction rule is called \isa{even{\isacharunderscore}odd{\isachardot}induct}
(simply concatenate the names of the sets involved) and has the conclusion
\begin{isabelle}%
\ \ \ \ \ {\isacharparenleft}{\isacharquery}x\ {\isasymin}\ even\ {\isasymlongrightarrow}\ {\isacharquery}P\ {\isacharquery}x{\isacharparenright}\ {\isasymand}\ {\isacharparenleft}{\isacharquery}y\ {\isasymin}\ odd\ {\isasymlongrightarrow}\ {\isacharquery}Q\ {\isacharquery}y{\isacharparenright}%
\end{isabelle}

If we want to prove that all even numbers are divisible by two, we have to
generalize the statement as follows:%
\end{isamarkuptext}%
\isamarkuptrue%
\isacommand{lemma}\ {\isachardoublequote}{\isacharparenleft}m\ {\isasymin}\ even\ {\isasymlongrightarrow}\ {\isadigit{2}}\ dvd\ m{\isacharparenright}\ {\isasymand}\ {\isacharparenleft}n\ {\isasymin}\ odd\ {\isasymlongrightarrow}\ {\isadigit{2}}\ dvd\ {\isacharparenleft}Suc\ n{\isacharparenright}{\isacharparenright}{\isachardoublequote}\isamarkupfalse%
\isamarkuptrue%
\isamarkupfalse%
\isamarkuptrue%
\isamarkupfalse%
\isamarkupfalse%
\isamarkupfalse%
\isamarkupfalse%
\isamarkupfalse%
\isamarkupfalse%
\isamarkupfalse%
\isamarkupfalse%
\end{isabellebody}%
%%% Local Variables:
%%% mode: latex
%%% TeX-master: "root"
%%% End:

%
\begin{isabellebody}%
\def\isabellecontext{Star}%
%
\isamarkupsection{The Reflexive Transitive Closure%
}
%
\begin{isamarkuptext}%
\label{sec:rtc}
An inductive definition may accept parameters, so it can express 
functions that yield sets.
Relations too can be defined inductively, since they are just sets of pairs.
A perfect example is the function that maps a relation to its
reflexive transitive closure.  This concept was already
introduced in \S\ref{sec:Relations}, where the operator \isa{{\isacharcircum}{\isacharasterisk}} was
defined as a least fixed point because inductive definitions were not yet
available. But now they are:%
\end{isamarkuptext}%
\isacommand{consts}\ rtc\ {\isacharcolon}{\isacharcolon}\ {\isachardoublequote}{\isacharparenleft}{\isacharprime}a\ {\isasymtimes}\ {\isacharprime}a{\isacharparenright}set\ {\isasymRightarrow}\ {\isacharparenleft}{\isacharprime}a\ {\isasymtimes}\ {\isacharprime}a{\isacharparenright}set{\isachardoublequote}\ \ \ {\isacharparenleft}{\isachardoublequote}{\isacharunderscore}{\isacharasterisk}{\isachardoublequote}\ {\isacharbrackleft}{\isadigit{1}}{\isadigit{0}}{\isadigit{0}}{\isadigit{0}}{\isacharbrackright}\ {\isadigit{9}}{\isadigit{9}}{\isadigit{9}}{\isacharparenright}\isanewline
\isacommand{inductive}\ {\isachardoublequote}r{\isacharasterisk}{\isachardoublequote}\isanewline
\isakeyword{intros}\isanewline
rtc{\isacharunderscore}refl{\isacharbrackleft}iff{\isacharbrackright}{\isacharcolon}\ \ {\isachardoublequote}{\isacharparenleft}x{\isacharcomma}x{\isacharparenright}\ {\isasymin}\ r{\isacharasterisk}{\isachardoublequote}\isanewline
rtc{\isacharunderscore}step{\isacharcolon}\ \ \ \ \ \ \ {\isachardoublequote}{\isasymlbrakk}\ {\isacharparenleft}x{\isacharcomma}y{\isacharparenright}\ {\isasymin}\ r{\isacharsemicolon}\ {\isacharparenleft}y{\isacharcomma}z{\isacharparenright}\ {\isasymin}\ r{\isacharasterisk}\ {\isasymrbrakk}\ {\isasymLongrightarrow}\ {\isacharparenleft}x{\isacharcomma}z{\isacharparenright}\ {\isasymin}\ r{\isacharasterisk}{\isachardoublequote}%
\begin{isamarkuptext}%
\noindent
The function \isa{rtc} is annotated with concrete syntax: instead of
\isa{rtc\ r} we can read and write \isa{r{\isacharasterisk}}. The actual definition
consists of two rules. Reflexivity is obvious and is immediately given the
\isa{iff} attribute to increase automation. The
second rule, \isa{rtc{\isacharunderscore}step}, says that we can always add one more
\isa{r}-step to the left. Although we could make \isa{rtc{\isacharunderscore}step} an
introduction rule, this is dangerous: the recursion in the second premise
slows down and may even kill the automatic tactics.

The above definition of the concept of reflexive transitive closure may
be sufficiently intuitive but it is certainly not the only possible one:
for a start, it does not even mention transitivity.
The rest of this section is devoted to proving that it is equivalent to
the standard definition. We start with a simple lemma:%
\end{isamarkuptext}%
\isacommand{lemma}\ {\isacharbrackleft}intro{\isacharbrackright}{\isacharcolon}\ {\isachardoublequote}{\isacharparenleft}x{\isacharcomma}y{\isacharparenright}\ {\isacharcolon}\ r\ {\isasymLongrightarrow}\ {\isacharparenleft}x{\isacharcomma}y{\isacharparenright}\ {\isasymin}\ r{\isacharasterisk}{\isachardoublequote}\isanewline
\isacommand{by}{\isacharparenleft}blast\ intro{\isacharcolon}\ rtc{\isacharunderscore}step{\isacharparenright}%
\begin{isamarkuptext}%
\noindent
Although the lemma itself is an unremarkable consequence of the basic rules,
it has the advantage that it can be declared an introduction rule without the
danger of killing the automatic tactics because \isa{r{\isacharasterisk}} occurs only in
the conclusion and not in the premise. Thus some proofs that would otherwise
need \isa{rtc{\isacharunderscore}step} can now be found automatically. The proof also
shows that \isa{blast} is able to handle \isa{rtc{\isacharunderscore}step}. But
some of the other automatic tactics are more sensitive, and even \isa{blast} can be lead astray in the presence of large numbers of rules.

To prove transitivity, we need rule induction, i.e.\ theorem
\isa{rtc{\isachardot}induct}:
\begin{isabelle}%
\ \ \ \ \ {\isasymlbrakk}{\isacharparenleft}{\isacharquery}xb{\isacharcomma}\ {\isacharquery}xa{\isacharparenright}\ {\isasymin}\ {\isacharquery}r{\isacharasterisk}{\isacharsemicolon}\ {\isasymAnd}x{\isachardot}\ {\isacharquery}P\ x\ x{\isacharsemicolon}\isanewline
\isaindent{\ \ \ \ \ \ \ \ }{\isasymAnd}x\ y\ z{\isachardot}\ {\isasymlbrakk}{\isacharparenleft}x{\isacharcomma}\ y{\isacharparenright}\ {\isasymin}\ {\isacharquery}r{\isacharsemicolon}\ {\isacharparenleft}y{\isacharcomma}\ z{\isacharparenright}\ {\isasymin}\ {\isacharquery}r{\isacharasterisk}{\isacharsemicolon}\ {\isacharquery}P\ y\ z{\isasymrbrakk}\ {\isasymLongrightarrow}\ {\isacharquery}P\ x\ z{\isasymrbrakk}\isanewline
\isaindent{\ \ \ \ \ }{\isasymLongrightarrow}\ {\isacharquery}P\ {\isacharquery}xb\ {\isacharquery}xa%
\end{isabelle}
It says that \isa{{\isacharquery}P} holds for an arbitrary pair \isa{{\isacharparenleft}{\isacharquery}xb{\isacharcomma}{\isacharquery}xa{\isacharparenright}\ {\isasymin}\ {\isacharquery}r{\isacharasterisk}} if \isa{{\isacharquery}P} is preserved by all rules of the inductive definition,
i.e.\ if \isa{{\isacharquery}P} holds for the conclusion provided it holds for the
premises. In general, rule induction for an $n$-ary inductive relation $R$
expects a premise of the form $(x@1,\dots,x@n) \in R$.

Now we turn to the inductive proof of transitivity:%
\end{isamarkuptext}%
\isacommand{lemma}\ rtc{\isacharunderscore}trans{\isacharcolon}\ {\isachardoublequote}{\isasymlbrakk}\ {\isacharparenleft}x{\isacharcomma}y{\isacharparenright}\ {\isasymin}\ r{\isacharasterisk}{\isacharsemicolon}\ {\isacharparenleft}y{\isacharcomma}z{\isacharparenright}\ {\isasymin}\ r{\isacharasterisk}\ {\isasymrbrakk}\ {\isasymLongrightarrow}\ {\isacharparenleft}x{\isacharcomma}z{\isacharparenright}\ {\isasymin}\ r{\isacharasterisk}{\isachardoublequote}\isanewline
\isacommand{apply}{\isacharparenleft}erule\ rtc{\isachardot}induct{\isacharparenright}%
\begin{isamarkuptxt}%
\noindent
Unfortunately, even the resulting base case is a problem
\begin{isabelle}%
\ {\isadigit{1}}{\isachardot}\ {\isasymAnd}x{\isachardot}\ {\isacharparenleft}y{\isacharcomma}\ z{\isacharparenright}\ {\isasymin}\ r{\isacharasterisk}\ {\isasymLongrightarrow}\ {\isacharparenleft}x{\isacharcomma}\ z{\isacharparenright}\ {\isasymin}\ r{\isacharasterisk}%
\end{isabelle}
and maybe not what you had expected. We have to abandon this proof attempt.
To understand what is going on, let us look again at \isa{rtc{\isachardot}induct}.
In the above application of \isa{erule}, the first premise of
\isa{rtc{\isachardot}induct} is unified with the first suitable assumption, which
is \isa{{\isacharparenleft}x{\isacharcomma}\ y{\isacharparenright}\ {\isasymin}\ r{\isacharasterisk}} rather than \isa{{\isacharparenleft}y{\isacharcomma}\ z{\isacharparenright}\ {\isasymin}\ r{\isacharasterisk}}. Although that
is what we want, it is merely due to the order in which the assumptions occur
in the subgoal, which it is not good practice to rely on. As a result,
\isa{{\isacharquery}xb} becomes \isa{x}, \isa{{\isacharquery}xa} becomes
\isa{y} and \isa{{\isacharquery}P} becomes \isa{{\isasymlambda}u\ v{\isachardot}\ {\isacharparenleft}u{\isacharcomma}\ z{\isacharparenright}\ {\isasymin}\ r{\isacharasterisk}}, thus
yielding the above subgoal. So what went wrong?

When looking at the instantiation of \isa{{\isacharquery}P} we see that it does not
depend on its second parameter at all. The reason is that in our original
goal, of the pair \isa{{\isacharparenleft}x{\isacharcomma}\ y{\isacharparenright}} only \isa{x} appears also in the
conclusion, but not \isa{y}. Thus our induction statement is too
weak. Fortunately, it can easily be strengthened:
transfer the additional premise \isa{{\isacharparenleft}y{\isacharcomma}\ z{\isacharparenright}\ {\isasymin}\ r{\isacharasterisk}} into the conclusion:%
\end{isamarkuptxt}%
\isacommand{lemma}\ rtc{\isacharunderscore}trans{\isacharbrackleft}rule{\isacharunderscore}format{\isacharbrackright}{\isacharcolon}\isanewline
\ \ {\isachardoublequote}{\isacharparenleft}x{\isacharcomma}y{\isacharparenright}\ {\isasymin}\ r{\isacharasterisk}\ {\isasymLongrightarrow}\ {\isacharparenleft}y{\isacharcomma}z{\isacharparenright}\ {\isasymin}\ r{\isacharasterisk}\ {\isasymlongrightarrow}\ {\isacharparenleft}x{\isacharcomma}z{\isacharparenright}\ {\isasymin}\ r{\isacharasterisk}{\isachardoublequote}%
\begin{isamarkuptxt}%
\noindent
This is not an obscure trick but a generally applicable heuristic:
\begin{quote}\em
Whe proving a statement by rule induction on $(x@1,\dots,x@n) \in R$,
pull all other premises containing any of the $x@i$ into the conclusion
using $\longrightarrow$.
\end{quote}
A similar heuristic for other kinds of inductions is formulated in
\S\ref{sec:ind-var-in-prems}. The \isa{rule{\isacharunderscore}format} directive turns
\isa{{\isasymlongrightarrow}} back into \isa{{\isasymLongrightarrow}}. Thus in the end we obtain the original
statement of our lemma.%
\end{isamarkuptxt}%
\isacommand{apply}{\isacharparenleft}erule\ rtc{\isachardot}induct{\isacharparenright}%
\begin{isamarkuptxt}%
\noindent
Now induction produces two subgoals which are both proved automatically:
\begin{isabelle}%
\ {\isadigit{1}}{\isachardot}\ {\isasymAnd}x{\isachardot}\ {\isacharparenleft}x{\isacharcomma}\ z{\isacharparenright}\ {\isasymin}\ r{\isacharasterisk}\ {\isasymlongrightarrow}\ {\isacharparenleft}x{\isacharcomma}\ z{\isacharparenright}\ {\isasymin}\ r{\isacharasterisk}\isanewline
\ {\isadigit{2}}{\isachardot}\ {\isasymAnd}x\ y\ za{\isachardot}\isanewline
\isaindent{\ {\isadigit{2}}{\isachardot}\ \ \ \ }{\isasymlbrakk}{\isacharparenleft}x{\isacharcomma}\ y{\isacharparenright}\ {\isasymin}\ r{\isacharsemicolon}\ {\isacharparenleft}y{\isacharcomma}\ za{\isacharparenright}\ {\isasymin}\ r{\isacharasterisk}{\isacharsemicolon}\ {\isacharparenleft}za{\isacharcomma}\ z{\isacharparenright}\ {\isasymin}\ r{\isacharasterisk}\ {\isasymlongrightarrow}\ {\isacharparenleft}y{\isacharcomma}\ z{\isacharparenright}\ {\isasymin}\ r{\isacharasterisk}{\isasymrbrakk}\isanewline
\isaindent{\ {\isadigit{2}}{\isachardot}\ \ \ \ }{\isasymLongrightarrow}\ {\isacharparenleft}za{\isacharcomma}\ z{\isacharparenright}\ {\isasymin}\ r{\isacharasterisk}\ {\isasymlongrightarrow}\ {\isacharparenleft}x{\isacharcomma}\ z{\isacharparenright}\ {\isasymin}\ r{\isacharasterisk}%
\end{isabelle}%
\end{isamarkuptxt}%
\ \isacommand{apply}{\isacharparenleft}blast{\isacharparenright}\isanewline
\isacommand{apply}{\isacharparenleft}blast\ intro{\isacharcolon}\ rtc{\isacharunderscore}step{\isacharparenright}\isanewline
\isacommand{done}%
\begin{isamarkuptext}%
Let us now prove that \isa{r{\isacharasterisk}} is really the reflexive transitive closure
of \isa{r}, i.e.\ the least reflexive and transitive
relation containing \isa{r}. The latter is easily formalized%
\end{isamarkuptext}%
\isacommand{consts}\ rtc{\isadigit{2}}\ {\isacharcolon}{\isacharcolon}\ {\isachardoublequote}{\isacharparenleft}{\isacharprime}a\ {\isasymtimes}\ {\isacharprime}a{\isacharparenright}set\ {\isasymRightarrow}\ {\isacharparenleft}{\isacharprime}a\ {\isasymtimes}\ {\isacharprime}a{\isacharparenright}set{\isachardoublequote}\isanewline
\isacommand{inductive}\ {\isachardoublequote}rtc{\isadigit{2}}\ r{\isachardoublequote}\isanewline
\isakeyword{intros}\isanewline
{\isachardoublequote}{\isacharparenleft}x{\isacharcomma}y{\isacharparenright}\ {\isasymin}\ r\ {\isasymLongrightarrow}\ {\isacharparenleft}x{\isacharcomma}y{\isacharparenright}\ {\isasymin}\ rtc{\isadigit{2}}\ r{\isachardoublequote}\isanewline
{\isachardoublequote}{\isacharparenleft}x{\isacharcomma}x{\isacharparenright}\ {\isasymin}\ rtc{\isadigit{2}}\ r{\isachardoublequote}\isanewline
{\isachardoublequote}{\isasymlbrakk}\ {\isacharparenleft}x{\isacharcomma}y{\isacharparenright}\ {\isasymin}\ rtc{\isadigit{2}}\ r{\isacharsemicolon}\ {\isacharparenleft}y{\isacharcomma}z{\isacharparenright}\ {\isasymin}\ rtc{\isadigit{2}}\ r\ {\isasymrbrakk}\ {\isasymLongrightarrow}\ {\isacharparenleft}x{\isacharcomma}z{\isacharparenright}\ {\isasymin}\ rtc{\isadigit{2}}\ r{\isachardoublequote}%
\begin{isamarkuptext}%
\noindent
and the equivalence of the two definitions is easily shown by the obvious rule
inductions:%
\end{isamarkuptext}%
\isacommand{lemma}\ {\isachardoublequote}{\isacharparenleft}x{\isacharcomma}y{\isacharparenright}\ {\isasymin}\ rtc{\isadigit{2}}\ r\ {\isasymLongrightarrow}\ {\isacharparenleft}x{\isacharcomma}y{\isacharparenright}\ {\isasymin}\ r{\isacharasterisk}{\isachardoublequote}\isanewline
\isacommand{apply}{\isacharparenleft}erule\ rtc{\isadigit{2}}{\isachardot}induct{\isacharparenright}\isanewline
\ \ \isacommand{apply}{\isacharparenleft}blast{\isacharparenright}\isanewline
\ \isacommand{apply}{\isacharparenleft}blast{\isacharparenright}\isanewline
\isacommand{apply}{\isacharparenleft}blast\ intro{\isacharcolon}\ rtc{\isacharunderscore}trans{\isacharparenright}\isanewline
\isacommand{done}\isanewline
\isanewline
\isacommand{lemma}\ {\isachardoublequote}{\isacharparenleft}x{\isacharcomma}y{\isacharparenright}\ {\isasymin}\ r{\isacharasterisk}\ {\isasymLongrightarrow}\ {\isacharparenleft}x{\isacharcomma}y{\isacharparenright}\ {\isasymin}\ rtc{\isadigit{2}}\ r{\isachardoublequote}\isanewline
\isacommand{apply}{\isacharparenleft}erule\ rtc{\isachardot}induct{\isacharparenright}\isanewline
\ \isacommand{apply}{\isacharparenleft}blast\ intro{\isacharcolon}\ rtc{\isadigit{2}}{\isachardot}intros{\isacharparenright}\isanewline
\isacommand{apply}{\isacharparenleft}blast\ intro{\isacharcolon}\ rtc{\isadigit{2}}{\isachardot}intros{\isacharparenright}\isanewline
\isacommand{done}%
\begin{isamarkuptext}%
So why did we start with the first definition? Because it is simpler. It
contains only two rules, and the single step rule is simpler than
transitivity.  As a consequence, \isa{rtc{\isachardot}induct} is simpler than
\isa{rtc{\isadigit{2}}{\isachardot}induct}. Since inductive proofs are hard enough
anyway, we should
certainly pick the simplest induction schema available.
Hence \isa{rtc} is the definition of choice.

\begin{exercise}\label{ex:converse-rtc-step}
Show that the converse of \isa{rtc{\isacharunderscore}step} also holds:
\begin{isabelle}%
\ \ \ \ \ {\isasymlbrakk}{\isacharparenleft}x{\isacharcomma}\ y{\isacharparenright}\ {\isasymin}\ r{\isacharasterisk}{\isacharsemicolon}\ {\isacharparenleft}y{\isacharcomma}\ z{\isacharparenright}\ {\isasymin}\ r{\isasymrbrakk}\ {\isasymLongrightarrow}\ {\isacharparenleft}x{\isacharcomma}\ z{\isacharparenright}\ {\isasymin}\ r{\isacharasterisk}%
\end{isabelle}
\end{exercise}
\begin{exercise}
Repeat the development of this section, but starting with a definition of
\isa{rtc} where \isa{rtc{\isacharunderscore}step} is replaced by its converse as shown
in exercise~\ref{ex:converse-rtc-step}.
\end{exercise}%
\end{isamarkuptext}%
\end{isabellebody}%
%%% Local Variables:
%%% mode: latex
%%% TeX-master: "root"
%%% End:


\section{Advanced Inductive Definitions}
\label{sec:adv-ind-def}
%
\begin{isabellebody}%
\def\isabellecontext{Advanced}%
%
\isadelimtheory
\isanewline
%
\endisadelimtheory
%
\isatagtheory
\isacommand{theory}\isamarkupfalse%
\ Advanced\ \isakeyword{imports}\ Even\ \isakeyword{begin}%
\endisatagtheory
{\isafoldtheory}%
%
\isadelimtheory
\isanewline
%
\endisadelimtheory
\isanewline
\isanewline
\isacommand{datatype}\isamarkupfalse%
\ {\isacharprime}f\ gterm\ {\isacharequal}\ Apply\ {\isacharprime}f\ {\isachardoublequoteopen}{\isacharprime}f\ gterm\ list{\isachardoublequoteclose}\isanewline
\isanewline
\isacommand{datatype}\isamarkupfalse%
\ integer{\isacharunderscore}op\ {\isacharequal}\ Number\ int\ {\isacharbar}\ UnaryMinus\ {\isacharbar}\ Plus\isanewline
\isanewline
\isacommand{inductive{\isacharunderscore}set}\isamarkupfalse%
\isanewline
\ \ gterms\ {\isacharcolon}{\isacharcolon}\ {\isachardoublequoteopen}{\isacharprime}f\ set\ {\isasymRightarrow}\ {\isacharprime}f\ gterm\ set{\isachardoublequoteclose}\isanewline
\ \ \isakeyword{for}\ F\ {\isacharcolon}{\isacharcolon}\ {\isachardoublequoteopen}{\isacharprime}f\ set{\isachardoublequoteclose}\isanewline
\isakeyword{where}\isanewline
step{\isacharbrackleft}intro{\isacharbang}{\isacharbrackright}{\isacharcolon}\ {\isachardoublequoteopen}{\isasymlbrakk}{\isasymforall}t\ {\isasymin}\ set\ args{\isachardot}\ t\ {\isasymin}\ gterms\ F{\isacharsemicolon}\ \ f\ {\isasymin}\ F{\isasymrbrakk}\isanewline
\ \ \ \ \ \ \ \ \ \ \ \ \ \ \ {\isasymLongrightarrow}\ {\isacharparenleft}Apply\ f\ args{\isacharparenright}\ {\isasymin}\ gterms\ F{\isachardoublequoteclose}\isanewline
\isanewline
\isacommand{lemma}\isamarkupfalse%
\ gterms{\isacharunderscore}mono{\isacharcolon}\ {\isachardoublequoteopen}F{\isasymsubseteq}G\ {\isasymLongrightarrow}\ gterms\ F\ {\isasymsubseteq}\ gterms\ G{\isachardoublequoteclose}\isanewline
%
\isadelimproof
%
\endisadelimproof
%
\isatagproof
\isacommand{apply}\isamarkupfalse%
\ clarify\isanewline
\isacommand{apply}\isamarkupfalse%
\ {\isacharparenleft}erule\ gterms{\isachardot}induct{\isacharparenright}%
\begin{isamarkuptxt}%
\begin{isabelle}%
\ {\isadigit{1}}{\isachardot}\ {\isasymAnd}x\ args\ f{\isachardot}\isanewline
\isaindent{\ {\isadigit{1}}{\isachardot}\ \ \ \ }{\isasymlbrakk}F\ {\isasymsubseteq}\ G{\isacharsemicolon}\ {\isasymforall}t{\isasymin}set\ args{\isachardot}\ t\ {\isasymin}\ gterms\ F\ {\isasyminter}\ {\isacharbraceleft}x{\isachardot}\ x\ {\isasymin}\ gterms\ G{\isacharbraceright}{\isacharsemicolon}\isanewline
\isaindent{\ {\isadigit{1}}{\isachardot}\ \ \ \ \ }f\ {\isasymin}\ F{\isasymrbrakk}\isanewline
\isaindent{\ {\isadigit{1}}{\isachardot}\ \ \ \ }{\isasymLongrightarrow}\ Apply\ f\ args\ {\isasymin}\ gterms\ G%
\end{isabelle}%
\end{isamarkuptxt}%
\isamarkuptrue%
\isacommand{apply}\isamarkupfalse%
\ blast\isanewline
\isacommand{done}\isamarkupfalse%
%
\endisatagproof
{\isafoldproof}%
%
\isadelimproof
%
\endisadelimproof
%
\begin{isamarkuptext}%
\begin{isabelle}%
\ \ \ \ \ {\isasymlbrakk}a\ {\isasymin}\ even{\isacharsemicolon}\ a\ {\isacharequal}\ {\isadigit{0}}\ {\isasymLongrightarrow}\ P{\isacharsemicolon}\ {\isasymAnd}n{\isachardot}\ {\isasymlbrakk}a\ {\isacharequal}\ Suc\ {\isacharparenleft}Suc\ n{\isacharparenright}{\isacharsemicolon}\ n\ {\isasymin}\ even{\isasymrbrakk}\ {\isasymLongrightarrow}\ P{\isasymrbrakk}\ {\isasymLongrightarrow}\ P%
\end{isabelle}
\rulename{even.cases}

Just as a demo I include
the two forms that Markus has made available. First the one for binding the
result to a name%
\end{isamarkuptext}%
\isamarkuptrue%
\isacommand{inductive{\isacharunderscore}cases}\isamarkupfalse%
\ Suc{\isacharunderscore}Suc{\isacharunderscore}cases\ {\isacharbrackleft}elim{\isacharbang}{\isacharbrackright}{\isacharcolon}\isanewline
\ \ {\isachardoublequoteopen}Suc{\isacharparenleft}Suc\ n{\isacharparenright}\ {\isasymin}\ even{\isachardoublequoteclose}\isanewline
\isanewline
\isacommand{thm}\isamarkupfalse%
\ Suc{\isacharunderscore}Suc{\isacharunderscore}cases%
\begin{isamarkuptext}%
\begin{isabelle}%
\ \ \ \ \ {\isasymlbrakk}Suc\ {\isacharparenleft}Suc\ n{\isacharparenright}\ {\isasymin}\ even{\isacharsemicolon}\ n\ {\isasymin}\ even\ {\isasymLongrightarrow}\ P{\isasymrbrakk}\ {\isasymLongrightarrow}\ P%
\end{isabelle}
\rulename{Suc_Suc_cases}

and now the one for local usage:%
\end{isamarkuptext}%
\isamarkuptrue%
\isacommand{lemma}\isamarkupfalse%
\ {\isachardoublequoteopen}Suc{\isacharparenleft}Suc\ n{\isacharparenright}\ {\isasymin}\ even\ {\isasymLongrightarrow}\ P\ n{\isachardoublequoteclose}\isanewline
%
\isadelimproof
%
\endisadelimproof
%
\isatagproof
\isacommand{apply}\isamarkupfalse%
\ {\isacharparenleft}ind{\isacharunderscore}cases\ {\isachardoublequoteopen}Suc{\isacharparenleft}Suc\ n{\isacharparenright}\ {\isasymin}\ even{\isachardoublequoteclose}{\isacharparenright}\isanewline
\isacommand{oops}\isamarkupfalse%
%
\endisatagproof
{\isafoldproof}%
%
\isadelimproof
\isanewline
%
\endisadelimproof
\isanewline
\isacommand{inductive{\isacharunderscore}cases}\isamarkupfalse%
\ gterm{\isacharunderscore}Apply{\isacharunderscore}elim\ {\isacharbrackleft}elim{\isacharbang}{\isacharbrackright}{\isacharcolon}\ {\isachardoublequoteopen}Apply\ f\ args\ {\isasymin}\ gterms\ F{\isachardoublequoteclose}%
\begin{isamarkuptext}%
this is what we get:

\begin{isabelle}%
\ \ \ \ \ {\isasymlbrakk}Apply\ f\ args\ {\isasymin}\ gterms\ F{\isacharsemicolon}\ {\isasymlbrakk}{\isasymforall}t{\isasymin}set\ args{\isachardot}\ t\ {\isasymin}\ gterms\ F{\isacharsemicolon}\ f\ {\isasymin}\ F{\isasymrbrakk}\ {\isasymLongrightarrow}\ P{\isasymrbrakk}\ {\isasymLongrightarrow}\ P%
\end{isabelle}
\rulename{gterm_Apply_elim}%
\end{isamarkuptext}%
\isamarkuptrue%
\isacommand{lemma}\isamarkupfalse%
\ gterms{\isacharunderscore}IntI\ {\isacharbrackleft}rule{\isacharunderscore}format{\isacharcomma}\ intro{\isacharbang}{\isacharbrackright}{\isacharcolon}\isanewline
\ \ \ \ \ {\isachardoublequoteopen}t\ {\isasymin}\ gterms\ F\ {\isasymLongrightarrow}\ t\ {\isasymin}\ gterms\ G\ {\isasymlongrightarrow}\ t\ {\isasymin}\ gterms\ {\isacharparenleft}F{\isasyminter}G{\isacharparenright}{\isachardoublequoteclose}\isanewline
%
\isadelimproof
%
\endisadelimproof
%
\isatagproof
\isacommand{apply}\isamarkupfalse%
\ {\isacharparenleft}erule\ gterms{\isachardot}induct{\isacharparenright}%
\begin{isamarkuptxt}%
\begin{isabelle}%
\ {\isadigit{1}}{\isachardot}\ {\isasymAnd}args\ f{\isachardot}\isanewline
\isaindent{\ {\isadigit{1}}{\isachardot}\ \ \ \ }{\isasymlbrakk}{\isasymforall}t{\isasymin}set\ args{\isachardot}\isanewline
\isaindent{\ {\isadigit{1}}{\isachardot}\ \ \ \ {\isasymlbrakk}\ \ \ }t\ {\isasymin}\ gterms\ F\ {\isasyminter}\isanewline
\isaindent{\ {\isadigit{1}}{\isachardot}\ \ \ \ {\isasymlbrakk}\ \ \ t\ {\isasymin}\ }{\isacharbraceleft}x{\isachardot}\ x\ {\isasymin}\ gterms\ G\ {\isasymlongrightarrow}\ x\ {\isasymin}\ gterms\ {\isacharparenleft}F\ {\isasyminter}\ G{\isacharparenright}{\isacharbraceright}{\isacharsemicolon}\isanewline
\isaindent{\ {\isadigit{1}}{\isachardot}\ \ \ \ \ }f\ {\isasymin}\ F{\isasymrbrakk}\isanewline
\isaindent{\ {\isadigit{1}}{\isachardot}\ \ \ \ }{\isasymLongrightarrow}\ Apply\ f\ args\ {\isasymin}\ gterms\ G\ {\isasymlongrightarrow}\isanewline
\isaindent{\ {\isadigit{1}}{\isachardot}\ \ \ \ {\isasymLongrightarrow}\ }Apply\ f\ args\ {\isasymin}\ gterms\ {\isacharparenleft}F\ {\isasyminter}\ G{\isacharparenright}%
\end{isabelle}%
\end{isamarkuptxt}%
\isamarkuptrue%
\isacommand{apply}\isamarkupfalse%
\ blast\isanewline
\isacommand{done}\isamarkupfalse%
%
\endisatagproof
{\isafoldproof}%
%
\isadelimproof
%
\endisadelimproof
%
\begin{isamarkuptext}%
\begin{isabelle}%
\ \ \ \ \ mono\ f\ {\isasymLongrightarrow}\ f\ {\isacharparenleft}A\ {\isasyminter}\ B{\isacharparenright}\ {\isasymsubseteq}\ f\ A\ {\isasyminter}\ f\ B%
\end{isabelle}
\rulename{mono_Int}%
\end{isamarkuptext}%
\isamarkuptrue%
\isacommand{lemma}\isamarkupfalse%
\ gterms{\isacharunderscore}Int{\isacharunderscore}eq\ {\isacharbrackleft}simp{\isacharbrackright}{\isacharcolon}\isanewline
\ \ \ \ \ {\isachardoublequoteopen}gterms\ {\isacharparenleft}F{\isasyminter}G{\isacharparenright}\ {\isacharequal}\ gterms\ F\ {\isasyminter}\ gterms\ G{\isachardoublequoteclose}\isanewline
%
\isadelimproof
%
\endisadelimproof
%
\isatagproof
\isacommand{by}\isamarkupfalse%
\ {\isacharparenleft}blast\ intro{\isacharbang}{\isacharcolon}\ mono{\isacharunderscore}Int\ monoI\ gterms{\isacharunderscore}mono{\isacharparenright}%
\endisatagproof
{\isafoldproof}%
%
\isadelimproof
%
\endisadelimproof
%
\begin{isamarkuptext}%
the following declaration isn't actually used%
\end{isamarkuptext}%
\isamarkuptrue%
\isacommand{consts}\isamarkupfalse%
\ integer{\isacharunderscore}arity\ {\isacharcolon}{\isacharcolon}\ {\isachardoublequoteopen}integer{\isacharunderscore}op\ {\isasymRightarrow}\ nat{\isachardoublequoteclose}\isanewline
\isacommand{primrec}\isamarkupfalse%
\isanewline
{\isachardoublequoteopen}integer{\isacharunderscore}arity\ {\isacharparenleft}Number\ n{\isacharparenright}\ \ \ \ \ \ \ \ {\isacharequal}\ {\isadigit{0}}{\isachardoublequoteclose}\isanewline
{\isachardoublequoteopen}integer{\isacharunderscore}arity\ UnaryMinus\ \ \ \ \ \ \ \ {\isacharequal}\ {\isadigit{1}}{\isachardoublequoteclose}\isanewline
{\isachardoublequoteopen}integer{\isacharunderscore}arity\ Plus\ \ \ \ \ \ \ \ \ \ \ \ \ \ {\isacharequal}\ {\isadigit{2}}{\isachardoublequoteclose}\isanewline
\isanewline
\isacommand{inductive{\isacharunderscore}set}\isamarkupfalse%
\isanewline
\ \ well{\isacharunderscore}formed{\isacharunderscore}gterm\ {\isacharcolon}{\isacharcolon}\ {\isachardoublequoteopen}{\isacharparenleft}{\isacharprime}f\ {\isasymRightarrow}\ nat{\isacharparenright}\ {\isasymRightarrow}\ {\isacharprime}f\ gterm\ set{\isachardoublequoteclose}\isanewline
\ \ \isakeyword{for}\ arity\ {\isacharcolon}{\isacharcolon}\ {\isachardoublequoteopen}{\isacharprime}f\ {\isasymRightarrow}\ nat{\isachardoublequoteclose}\isanewline
\isakeyword{where}\isanewline
step{\isacharbrackleft}intro{\isacharbang}{\isacharbrackright}{\isacharcolon}\ {\isachardoublequoteopen}{\isasymlbrakk}{\isasymforall}t\ {\isasymin}\ set\ args{\isachardot}\ t\ {\isasymin}\ well{\isacharunderscore}formed{\isacharunderscore}gterm\ arity{\isacharsemicolon}\ \ \isanewline
\ \ \ \ \ \ \ \ \ \ \ \ \ \ \ \ length\ args\ {\isacharequal}\ arity\ f{\isasymrbrakk}\isanewline
\ \ \ \ \ \ \ \ \ \ \ \ \ \ \ {\isasymLongrightarrow}\ {\isacharparenleft}Apply\ f\ args{\isacharparenright}\ {\isasymin}\ well{\isacharunderscore}formed{\isacharunderscore}gterm\ arity{\isachardoublequoteclose}\isanewline
\isanewline
\isanewline
\isacommand{inductive{\isacharunderscore}set}\isamarkupfalse%
\isanewline
\ \ well{\isacharunderscore}formed{\isacharunderscore}gterm{\isacharprime}\ {\isacharcolon}{\isacharcolon}\ {\isachardoublequoteopen}{\isacharparenleft}{\isacharprime}f\ {\isasymRightarrow}\ nat{\isacharparenright}\ {\isasymRightarrow}\ {\isacharprime}f\ gterm\ set{\isachardoublequoteclose}\isanewline
\ \ \isakeyword{for}\ arity\ {\isacharcolon}{\isacharcolon}\ {\isachardoublequoteopen}{\isacharprime}f\ {\isasymRightarrow}\ nat{\isachardoublequoteclose}\isanewline
\isakeyword{where}\isanewline
step{\isacharbrackleft}intro{\isacharbang}{\isacharbrackright}{\isacharcolon}\ {\isachardoublequoteopen}{\isasymlbrakk}args\ {\isasymin}\ lists\ {\isacharparenleft}well{\isacharunderscore}formed{\isacharunderscore}gterm{\isacharprime}\ arity{\isacharparenright}{\isacharsemicolon}\ \ \isanewline
\ \ \ \ \ \ \ \ \ \ \ \ \ \ \ \ length\ args\ {\isacharequal}\ arity\ f{\isasymrbrakk}\isanewline
\ \ \ \ \ \ \ \ \ \ \ \ \ \ \ {\isasymLongrightarrow}\ {\isacharparenleft}Apply\ f\ args{\isacharparenright}\ {\isasymin}\ well{\isacharunderscore}formed{\isacharunderscore}gterm{\isacharprime}\ arity{\isachardoublequoteclose}\isanewline
\isakeyword{monos}\ lists{\isacharunderscore}mono\isanewline
\isanewline
\isacommand{lemma}\isamarkupfalse%
\ {\isachardoublequoteopen}well{\isacharunderscore}formed{\isacharunderscore}gterm\ arity\ {\isasymsubseteq}\ well{\isacharunderscore}formed{\isacharunderscore}gterm{\isacharprime}\ arity{\isachardoublequoteclose}\isanewline
%
\isadelimproof
%
\endisadelimproof
%
\isatagproof
\isacommand{apply}\isamarkupfalse%
\ clarify%
\begin{isamarkuptxt}%
The situation after clarify
\begin{isabelle}%
\ {\isadigit{1}}{\isachardot}\ {\isasymAnd}x{\isachardot}\ x\ {\isasymin}\ well{\isacharunderscore}formed{\isacharunderscore}gterm\ arity\ {\isasymLongrightarrow}\isanewline
\isaindent{\ {\isadigit{1}}{\isachardot}\ {\isasymAnd}x{\isachardot}\ }x\ {\isasymin}\ well{\isacharunderscore}formed{\isacharunderscore}gterm{\isacharprime}\ arity%
\end{isabelle}%
\end{isamarkuptxt}%
\isamarkuptrue%
\isacommand{apply}\isamarkupfalse%
\ {\isacharparenleft}erule\ well{\isacharunderscore}formed{\isacharunderscore}gterm{\isachardot}induct{\isacharparenright}%
\begin{isamarkuptxt}%
note the induction hypothesis!
\begin{isabelle}%
\ {\isadigit{1}}{\isachardot}\ {\isasymAnd}x\ args\ f{\isachardot}\isanewline
\isaindent{\ {\isadigit{1}}{\isachardot}\ \ \ \ }{\isasymlbrakk}{\isasymforall}t{\isasymin}set\ args{\isachardot}\isanewline
\isaindent{\ {\isadigit{1}}{\isachardot}\ \ \ \ {\isasymlbrakk}\ \ \ }t\ {\isasymin}\ well{\isacharunderscore}formed{\isacharunderscore}gterm\ arity\ {\isasyminter}\isanewline
\isaindent{\ {\isadigit{1}}{\isachardot}\ \ \ \ {\isasymlbrakk}\ \ \ t\ {\isasymin}\ }{\isacharbraceleft}x{\isachardot}\ x\ {\isasymin}\ well{\isacharunderscore}formed{\isacharunderscore}gterm{\isacharprime}\ arity{\isacharbraceright}{\isacharsemicolon}\isanewline
\isaindent{\ {\isadigit{1}}{\isachardot}\ \ \ \ \ }length\ args\ {\isacharequal}\ arity\ f{\isasymrbrakk}\isanewline
\isaindent{\ {\isadigit{1}}{\isachardot}\ \ \ \ }{\isasymLongrightarrow}\ Apply\ f\ args\ {\isasymin}\ well{\isacharunderscore}formed{\isacharunderscore}gterm{\isacharprime}\ arity%
\end{isabelle}%
\end{isamarkuptxt}%
\isamarkuptrue%
\isacommand{apply}\isamarkupfalse%
\ auto\isanewline
\isacommand{done}\isamarkupfalse%
%
\endisatagproof
{\isafoldproof}%
%
\isadelimproof
\isanewline
%
\endisadelimproof
\isanewline
\isanewline
\isanewline
\isacommand{lemma}\isamarkupfalse%
\ {\isachardoublequoteopen}well{\isacharunderscore}formed{\isacharunderscore}gterm{\isacharprime}\ arity\ {\isasymsubseteq}\ well{\isacharunderscore}formed{\isacharunderscore}gterm\ arity{\isachardoublequoteclose}\isanewline
%
\isadelimproof
%
\endisadelimproof
%
\isatagproof
\isacommand{apply}\isamarkupfalse%
\ clarify%
\begin{isamarkuptxt}%
The situation after clarify
\begin{isabelle}%
\ {\isadigit{1}}{\isachardot}\ {\isasymAnd}x{\isachardot}\ x\ {\isasymin}\ well{\isacharunderscore}formed{\isacharunderscore}gterm{\isacharprime}\ arity\ {\isasymLongrightarrow}\isanewline
\isaindent{\ {\isadigit{1}}{\isachardot}\ {\isasymAnd}x{\isachardot}\ }x\ {\isasymin}\ well{\isacharunderscore}formed{\isacharunderscore}gterm\ arity%
\end{isabelle}%
\end{isamarkuptxt}%
\isamarkuptrue%
\isacommand{apply}\isamarkupfalse%
\ {\isacharparenleft}erule\ well{\isacharunderscore}formed{\isacharunderscore}gterm{\isacharprime}{\isachardot}induct{\isacharparenright}%
\begin{isamarkuptxt}%
note the induction hypothesis!
\begin{isabelle}%
\ {\isadigit{1}}{\isachardot}\ {\isasymAnd}x\ args\ f{\isachardot}\isanewline
\isaindent{\ {\isadigit{1}}{\isachardot}\ \ \ \ }{\isasymlbrakk}args\isanewline
\isaindent{\ {\isadigit{1}}{\isachardot}\ \ \ \ {\isasymlbrakk}}{\isasymin}\ lists\isanewline
\isaindent{\ {\isadigit{1}}{\isachardot}\ \ \ \ {\isasymlbrakk}{\isasymin}\ \ }{\isacharparenleft}well{\isacharunderscore}formed{\isacharunderscore}gterm{\isacharprime}\ arity\ {\isasyminter}\isanewline
\isaindent{\ {\isadigit{1}}{\isachardot}\ \ \ \ {\isasymlbrakk}{\isasymin}\ \ {\isacharparenleft}}{\isacharbraceleft}a{\isachardot}\ a\ {\isasymin}\ well{\isacharunderscore}formed{\isacharunderscore}gterm\ arity{\isacharbraceright}{\isacharparenright}{\isacharsemicolon}\isanewline
\isaindent{\ {\isadigit{1}}{\isachardot}\ \ \ \ \ }length\ args\ {\isacharequal}\ arity\ f{\isasymrbrakk}\isanewline
\isaindent{\ {\isadigit{1}}{\isachardot}\ \ \ \ }{\isasymLongrightarrow}\ Apply\ f\ args\ {\isasymin}\ well{\isacharunderscore}formed{\isacharunderscore}gterm\ arity%
\end{isabelle}%
\end{isamarkuptxt}%
\isamarkuptrue%
\isacommand{apply}\isamarkupfalse%
\ auto\isanewline
\isacommand{done}\isamarkupfalse%
%
\endisatagproof
{\isafoldproof}%
%
\isadelimproof
%
\endisadelimproof
%
\begin{isamarkuptext}%
\begin{isabelle}%
\ \ \ \ \ lists\ {\isacharparenleft}A\ {\isasyminter}\ B{\isacharparenright}\ {\isacharequal}\ lists\ A\ {\isasyminter}\ lists\ B%
\end{isabelle}%
\end{isamarkuptext}%
\isamarkuptrue%
%
\begin{isamarkuptext}%
the rest isn't used: too complicated.  OK for an exercise though.%
\end{isamarkuptext}%
\isamarkuptrue%
\isacommand{inductive{\isacharunderscore}set}\isamarkupfalse%
\isanewline
\ \ integer{\isacharunderscore}signature\ {\isacharcolon}{\isacharcolon}\ {\isachardoublequoteopen}{\isacharparenleft}integer{\isacharunderscore}op\ {\isacharasterisk}\ {\isacharparenleft}unit\ list\ {\isacharasterisk}\ unit{\isacharparenright}{\isacharparenright}\ set{\isachardoublequoteclose}\isanewline
\isakeyword{where}\isanewline
\ \ Number{\isacharcolon}\ \ \ \ \ {\isachardoublequoteopen}{\isacharparenleft}Number\ n{\isacharcomma}\ \ \ {\isacharparenleft}{\isacharbrackleft}{\isacharbrackright}{\isacharcomma}\ {\isacharparenleft}{\isacharparenright}{\isacharparenright}{\isacharparenright}\ {\isasymin}\ integer{\isacharunderscore}signature{\isachardoublequoteclose}\isanewline
{\isacharbar}\ UnaryMinus{\isacharcolon}\ {\isachardoublequoteopen}{\isacharparenleft}UnaryMinus{\isacharcomma}\ {\isacharparenleft}{\isacharbrackleft}{\isacharparenleft}{\isacharparenright}{\isacharbrackright}{\isacharcomma}\ {\isacharparenleft}{\isacharparenright}{\isacharparenright}{\isacharparenright}\ {\isasymin}\ integer{\isacharunderscore}signature{\isachardoublequoteclose}\isanewline
{\isacharbar}\ Plus{\isacharcolon}\ \ \ \ \ \ \ {\isachardoublequoteopen}{\isacharparenleft}Plus{\isacharcomma}\ \ \ \ \ \ \ {\isacharparenleft}{\isacharbrackleft}{\isacharparenleft}{\isacharparenright}{\isacharcomma}{\isacharparenleft}{\isacharparenright}{\isacharbrackright}{\isacharcomma}\ {\isacharparenleft}{\isacharparenright}{\isacharparenright}{\isacharparenright}\ {\isasymin}\ integer{\isacharunderscore}signature{\isachardoublequoteclose}\isanewline
\isanewline
\isanewline
\isacommand{inductive{\isacharunderscore}set}\isamarkupfalse%
\isanewline
\ \ well{\isacharunderscore}typed{\isacharunderscore}gterm\ {\isacharcolon}{\isacharcolon}\ {\isachardoublequoteopen}{\isacharparenleft}{\isacharprime}f\ {\isasymRightarrow}\ {\isacharprime}t\ list\ {\isacharasterisk}\ {\isacharprime}t{\isacharparenright}\ {\isasymRightarrow}\ {\isacharparenleft}{\isacharprime}f\ gterm\ {\isacharasterisk}\ {\isacharprime}t{\isacharparenright}set{\isachardoublequoteclose}\isanewline
\ \ \isakeyword{for}\ sig\ {\isacharcolon}{\isacharcolon}\ {\isachardoublequoteopen}{\isacharprime}f\ {\isasymRightarrow}\ {\isacharprime}t\ list\ {\isacharasterisk}\ {\isacharprime}t{\isachardoublequoteclose}\isanewline
\isakeyword{where}\isanewline
step{\isacharbrackleft}intro{\isacharbang}{\isacharbrackright}{\isacharcolon}\ \isanewline
\ \ \ \ {\isachardoublequoteopen}{\isasymlbrakk}{\isasymforall}pair\ {\isasymin}\ set\ args{\isachardot}\ pair\ {\isasymin}\ well{\isacharunderscore}typed{\isacharunderscore}gterm\ sig{\isacharsemicolon}\ \isanewline
\ \ \ \ \ \ sig\ f\ {\isacharequal}\ {\isacharparenleft}map\ snd\ args{\isacharcomma}\ rtype{\isacharparenright}{\isasymrbrakk}\isanewline
\ \ \ \ \ {\isasymLongrightarrow}\ {\isacharparenleft}Apply\ f\ {\isacharparenleft}map\ fst\ args{\isacharparenright}{\isacharcomma}\ rtype{\isacharparenright}\ \isanewline
\ \ \ \ \ \ \ \ \ {\isasymin}\ well{\isacharunderscore}typed{\isacharunderscore}gterm\ sig{\isachardoublequoteclose}\isanewline
\isanewline
\isacommand{inductive{\isacharunderscore}set}\isamarkupfalse%
\isanewline
\ \ well{\isacharunderscore}typed{\isacharunderscore}gterm{\isacharprime}\ {\isacharcolon}{\isacharcolon}\ {\isachardoublequoteopen}{\isacharparenleft}{\isacharprime}f\ {\isasymRightarrow}\ {\isacharprime}t\ list\ {\isacharasterisk}\ {\isacharprime}t{\isacharparenright}\ {\isasymRightarrow}\ {\isacharparenleft}{\isacharprime}f\ gterm\ {\isacharasterisk}\ {\isacharprime}t{\isacharparenright}set{\isachardoublequoteclose}\isanewline
\ \ \isakeyword{for}\ sig\ {\isacharcolon}{\isacharcolon}\ {\isachardoublequoteopen}{\isacharprime}f\ {\isasymRightarrow}\ {\isacharprime}t\ list\ {\isacharasterisk}\ {\isacharprime}t{\isachardoublequoteclose}\isanewline
\isakeyword{where}\isanewline
step{\isacharbrackleft}intro{\isacharbang}{\isacharbrackright}{\isacharcolon}\ \isanewline
\ \ \ \ {\isachardoublequoteopen}{\isasymlbrakk}args\ {\isasymin}\ lists{\isacharparenleft}well{\isacharunderscore}typed{\isacharunderscore}gterm{\isacharprime}\ sig{\isacharparenright}{\isacharsemicolon}\ \isanewline
\ \ \ \ \ \ sig\ f\ {\isacharequal}\ {\isacharparenleft}map\ snd\ args{\isacharcomma}\ rtype{\isacharparenright}{\isasymrbrakk}\isanewline
\ \ \ \ \ {\isasymLongrightarrow}\ {\isacharparenleft}Apply\ f\ {\isacharparenleft}map\ fst\ args{\isacharparenright}{\isacharcomma}\ rtype{\isacharparenright}\ \isanewline
\ \ \ \ \ \ \ \ \ {\isasymin}\ well{\isacharunderscore}typed{\isacharunderscore}gterm{\isacharprime}\ sig{\isachardoublequoteclose}\isanewline
\isakeyword{monos}\ lists{\isacharunderscore}mono\isanewline
\isanewline
\isanewline
\isacommand{lemma}\isamarkupfalse%
\ {\isachardoublequoteopen}well{\isacharunderscore}typed{\isacharunderscore}gterm\ sig\ {\isasymsubseteq}\ well{\isacharunderscore}typed{\isacharunderscore}gterm{\isacharprime}\ sig{\isachardoublequoteclose}\isanewline
%
\isadelimproof
%
\endisadelimproof
%
\isatagproof
\isacommand{apply}\isamarkupfalse%
\ clarify\isanewline
\isacommand{apply}\isamarkupfalse%
\ {\isacharparenleft}erule\ well{\isacharunderscore}typed{\isacharunderscore}gterm{\isachardot}induct{\isacharparenright}\isanewline
\isacommand{apply}\isamarkupfalse%
\ auto\isanewline
\isacommand{done}\isamarkupfalse%
%
\endisatagproof
{\isafoldproof}%
%
\isadelimproof
\isanewline
%
\endisadelimproof
\isanewline
\isacommand{lemma}\isamarkupfalse%
\ {\isachardoublequoteopen}well{\isacharunderscore}typed{\isacharunderscore}gterm{\isacharprime}\ sig\ {\isasymsubseteq}\ well{\isacharunderscore}typed{\isacharunderscore}gterm\ sig{\isachardoublequoteclose}\isanewline
%
\isadelimproof
%
\endisadelimproof
%
\isatagproof
\isacommand{apply}\isamarkupfalse%
\ clarify\isanewline
\isacommand{apply}\isamarkupfalse%
\ {\isacharparenleft}erule\ well{\isacharunderscore}typed{\isacharunderscore}gterm{\isacharprime}{\isachardot}induct{\isacharparenright}\isanewline
\isacommand{apply}\isamarkupfalse%
\ auto\isanewline
\isacommand{done}\isamarkupfalse%
%
\endisatagproof
{\isafoldproof}%
%
\isadelimproof
\isanewline
%
\endisadelimproof
\isanewline
%
\isadelimtheory
\isanewline
%
\endisadelimtheory
%
\isatagtheory
\isacommand{end}\isamarkupfalse%
%
\endisatagtheory
{\isafoldtheory}%
%
\isadelimtheory
\isanewline
%
\endisadelimtheory
\isanewline
\end{isabellebody}%
%%% Local Variables:
%%% mode: latex
%%% TeX-master: "root"
%%% End:


%
\begin{isabellebody}%
\def\isabellecontext{AB}%
%
\isamarkupsection{Case study: A context free grammar%
}
%
\begin{isamarkuptext}%
\label{sec:CFG}
Grammars are nothing but shorthands for inductive definitions of nonterminals
which represent sets of strings. For example, the production
$A \to B c$ is short for
\[ w \in B \Longrightarrow wc \in A \]
This section demonstrates this idea with a standard example
\cite[p.\ 81]{HopcroftUllman}, a grammar for generating all words with an
equal number of $a$'s and $b$'s:
\begin{eqnarray}
S &\to& \epsilon \mid b A \mid a B \nonumber\\
A &\to& a S \mid b A A \nonumber\\
B &\to& b S \mid a B B \nonumber
\end{eqnarray}
At the end we say a few words about the relationship of the formalization
and the text in the book~\cite[p.\ 81]{HopcroftUllman}.

We start by fixing the alphabet, which consists only of \isa{a}'s
and \isa{b}'s:%
\end{isamarkuptext}%
\isacommand{datatype}\ alfa\ {\isacharequal}\ a\ {\isacharbar}\ b%
\begin{isamarkuptext}%
\noindent
For convenience we include the following easy lemmas as simplification rules:%
\end{isamarkuptext}%
\isacommand{lemma}\ {\isacharbrackleft}simp{\isacharbrackright}{\isacharcolon}\ {\isachardoublequote}{\isacharparenleft}x\ {\isasymnoteq}\ a{\isacharparenright}\ {\isacharequal}\ {\isacharparenleft}x\ {\isacharequal}\ b{\isacharparenright}\ {\isasymand}\ {\isacharparenleft}x\ {\isasymnoteq}\ b{\isacharparenright}\ {\isacharequal}\ {\isacharparenleft}x\ {\isacharequal}\ a{\isacharparenright}{\isachardoublequote}\isanewline
\isacommand{apply}{\isacharparenleft}case{\isacharunderscore}tac\ x{\isacharparenright}\isanewline
\isacommand{by}{\isacharparenleft}auto{\isacharparenright}%
\begin{isamarkuptext}%
\noindent
Words over this alphabet are of type \isa{alfa\ list}, and
the three nonterminals are declare as sets of such words:%
\end{isamarkuptext}%
\isacommand{consts}\ S\ {\isacharcolon}{\isacharcolon}\ {\isachardoublequote}alfa\ list\ set{\isachardoublequote}\isanewline
\ \ \ \ \ \ \ A\ {\isacharcolon}{\isacharcolon}\ {\isachardoublequote}alfa\ list\ set{\isachardoublequote}\isanewline
\ \ \ \ \ \ \ B\ {\isacharcolon}{\isacharcolon}\ {\isachardoublequote}alfa\ list\ set{\isachardoublequote}%
\begin{isamarkuptext}%
\noindent
The above productions are recast as a \emph{simultaneous} inductive
definition\index{inductive definition!simultaneous}
of \isa{S}, \isa{A} and \isa{B}:%
\end{isamarkuptext}%
\isacommand{inductive}\ S\ A\ B\isanewline
\isakeyword{intros}\isanewline
\ \ {\isachardoublequote}{\isacharbrackleft}{\isacharbrackright}\ {\isasymin}\ S{\isachardoublequote}\isanewline
\ \ {\isachardoublequote}w\ {\isasymin}\ A\ {\isasymLongrightarrow}\ b{\isacharhash}w\ {\isasymin}\ S{\isachardoublequote}\isanewline
\ \ {\isachardoublequote}w\ {\isasymin}\ B\ {\isasymLongrightarrow}\ a{\isacharhash}w\ {\isasymin}\ S{\isachardoublequote}\isanewline
\isanewline
\ \ {\isachardoublequote}w\ {\isasymin}\ S\ \ \ \ \ \ \ \ {\isasymLongrightarrow}\ a{\isacharhash}w\ \ \ {\isasymin}\ A{\isachardoublequote}\isanewline
\ \ {\isachardoublequote}{\isasymlbrakk}\ v{\isasymin}A{\isacharsemicolon}\ w{\isasymin}A\ {\isasymrbrakk}\ {\isasymLongrightarrow}\ b{\isacharhash}v{\isacharat}w\ {\isasymin}\ A{\isachardoublequote}\isanewline
\isanewline
\ \ {\isachardoublequote}w\ {\isasymin}\ S\ \ \ \ \ \ \ \ \ \ \ \ {\isasymLongrightarrow}\ b{\isacharhash}w\ \ \ {\isasymin}\ B{\isachardoublequote}\isanewline
\ \ {\isachardoublequote}{\isasymlbrakk}\ v\ {\isasymin}\ B{\isacharsemicolon}\ w\ {\isasymin}\ B\ {\isasymrbrakk}\ {\isasymLongrightarrow}\ a{\isacharhash}v{\isacharat}w\ {\isasymin}\ B{\isachardoublequote}%
\begin{isamarkuptext}%
\noindent
First we show that all words in \isa{S} contain the same number of \isa{a}'s and \isa{b}'s. Since the definition of \isa{S} is by simultaneous
induction, so is this proof: we show at the same time that all words in
\isa{A} contain one more \isa{a} than \isa{b} and all words in \isa{B} contains one more \isa{b} than \isa{a}.%
\end{isamarkuptext}%
\isacommand{lemma}\ correctness{\isacharcolon}\isanewline
\ \ {\isachardoublequote}{\isacharparenleft}w\ {\isasymin}\ S\ {\isasymlongrightarrow}\ size{\isacharbrackleft}x{\isasymin}w{\isachardot}\ x{\isacharequal}a{\isacharbrackright}\ {\isacharequal}\ size{\isacharbrackleft}x{\isasymin}w{\isachardot}\ x{\isacharequal}b{\isacharbrackright}{\isacharparenright}\ \ \ \ \ {\isasymand}\isanewline
\ \ \ {\isacharparenleft}w\ {\isasymin}\ A\ {\isasymlongrightarrow}\ size{\isacharbrackleft}x{\isasymin}w{\isachardot}\ x{\isacharequal}a{\isacharbrackright}\ {\isacharequal}\ size{\isacharbrackleft}x{\isasymin}w{\isachardot}\ x{\isacharequal}b{\isacharbrackright}\ {\isacharplus}\ {\isadigit{1}}{\isacharparenright}\ {\isasymand}\isanewline
\ \ \ {\isacharparenleft}w\ {\isasymin}\ B\ {\isasymlongrightarrow}\ size{\isacharbrackleft}x{\isasymin}w{\isachardot}\ x{\isacharequal}b{\isacharbrackright}\ {\isacharequal}\ size{\isacharbrackleft}x{\isasymin}w{\isachardot}\ x{\isacharequal}a{\isacharbrackright}\ {\isacharplus}\ {\isadigit{1}}{\isacharparenright}{\isachardoublequote}%
\begin{isamarkuptxt}%
\noindent
These propositions are expressed with the help of the predefined \isa{filter} function on lists, which has the convenient syntax \isa{{\isacharbrackleft}x{\isasymin}xs{\isachardot}\ P\ x{\isacharbrackright}}, the list of all elements \isa{x} in \isa{xs} such that \isa{P\ x}
holds. Remember that on lists \isa{size} and \isa{size} are synonymous.

The proof itself is by rule induction and afterwards automatic:%
\end{isamarkuptxt}%
\isacommand{apply}{\isacharparenleft}rule\ S{\isacharunderscore}A{\isacharunderscore}B{\isachardot}induct{\isacharparenright}\isanewline
\isacommand{by}{\isacharparenleft}auto{\isacharparenright}%
\begin{isamarkuptext}%
\noindent
This may seem surprising at first, and is indeed an indication of the power
of inductive definitions. But it is also quite straightforward. For example,
consider the production $A \to b A A$: if $v,w \in A$ and the elements of $A$
contain one more $a$ than $b$'s, then $bvw$ must again contain one more $a$
than $b$'s.

As usual, the correctness of syntactic descriptions is easy, but completeness
is hard: does \isa{S} contain \emph{all} words with an equal number of
\isa{a}'s and \isa{b}'s? It turns out that this proof requires the
following little lemma: every string with two more \isa{a}'s than \isa{b}'s can be cut somehwere such that each half has one more \isa{a} than
\isa{b}. This is best seen by imagining counting the difference between the
number of \isa{a}'s and \isa{b}'s starting at the left end of the
word. We start with 0 and end (at the right end) with 2. Since each move to the
right increases or decreases the difference by 1, we must have passed through
1 on our way from 0 to 2. Formally, we appeal to the following discrete
intermediate value theorem \isa{nat{\isadigit{0}}{\isacharunderscore}intermed{\isacharunderscore}int{\isacharunderscore}val}
\begin{isabelle}%
\ \ \ \ \ {\isasymforall}i{\isachardot}\ i\ {\isacharless}\ n\ {\isasymlongrightarrow}\ {\isasymbar}f\ {\isacharparenleft}i\ {\isacharplus}\ {\isadigit{1}}{\isacharparenright}\ {\isacharminus}\ f\ i{\isasymbar}\ {\isasymle}\ {\isacharhash}{\isadigit{1}}\ {\isasymLongrightarrow}\isanewline
\ \ \ \ \ f\ {\isadigit{0}}\ {\isasymle}\ k\ {\isasymLongrightarrow}\ k\ {\isasymle}\ f\ n\ {\isasymLongrightarrow}\ {\isasymexists}i{\isachardot}\ i\ {\isasymle}\ n\ {\isasymand}\ f\ i\ {\isacharequal}\ k%
\end{isabelle}
where \isa{f} is of type \isa{nat\ {\isasymRightarrow}\ int}, \isa{int} are the integers,
\isa{abs} is the absolute value function, and \isa{{\isacharhash}{\isadigit{1}}} is the
integer 1 (see \S\ref{sec:numbers}).

First we show that the our specific function, the difference between the
numbers of \isa{a}'s and \isa{b}'s, does indeed only change by 1 in every
move to the right. At this point we also start generalizing from \isa{a}'s
and \isa{b}'s to an arbitrary property \isa{P}. Otherwise we would have
to prove the desired lemma twice, once as stated above and once with the
roles of \isa{a}'s and \isa{b}'s interchanged.%
\end{isamarkuptext}%
\isacommand{lemma}\ step{\isadigit{1}}{\isacharcolon}\ {\isachardoublequote}{\isasymforall}i\ {\isacharless}\ size\ w{\isachardot}\isanewline
\ \ abs{\isacharparenleft}{\isacharparenleft}int{\isacharparenleft}size{\isacharbrackleft}x{\isasymin}take\ {\isacharparenleft}i{\isacharplus}{\isadigit{1}}{\isacharparenright}\ w{\isachardot}\ \ P\ x{\isacharbrackright}{\isacharparenright}\ {\isacharminus}\isanewline
\ \ \ \ \ \ \ int{\isacharparenleft}size{\isacharbrackleft}x{\isasymin}take\ {\isacharparenleft}i{\isacharplus}{\isadigit{1}}{\isacharparenright}\ w{\isachardot}\ {\isasymnot}P\ x{\isacharbrackright}{\isacharparenright}{\isacharparenright}\isanewline
\ \ \ \ \ \ {\isacharminus}\isanewline
\ \ \ \ \ \ {\isacharparenleft}int{\isacharparenleft}size{\isacharbrackleft}x{\isasymin}take\ i\ w{\isachardot}\ \ P\ x{\isacharbrackright}{\isacharparenright}\ {\isacharminus}\isanewline
\ \ \ \ \ \ \ int{\isacharparenleft}size{\isacharbrackleft}x{\isasymin}take\ i\ w{\isachardot}\ {\isasymnot}P\ x{\isacharbrackright}{\isacharparenright}{\isacharparenright}{\isacharparenright}\ {\isasymle}\ {\isacharhash}{\isadigit{1}}{\isachardoublequote}%
\begin{isamarkuptxt}%
\noindent
The lemma is a bit hard to read because of the coercion function
\isa{{\isachardoublequote}int{\isacharcolon}{\isacharcolon}nat\ {\isasymRightarrow}\ int{\isachardoublequote}}. It is required because \isa{size} returns
a natural number, but \isa{{\isacharminus}} on \isa{nat} will do the wrong thing.
Function \isa{take} is predefined and \isa{take\ i\ xs} is the prefix of
length \isa{i} of \isa{xs}; below we als need \isa{drop\ i\ xs}, which
is what remains after that prefix has been dropped from \isa{xs}.

The proof is by induction on \isa{w}, with a trivial base case, and a not
so trivial induction step. Since it is essentially just arithmetic, we do not
discuss it.%
\end{isamarkuptxt}%
\isacommand{apply}{\isacharparenleft}induct\ w{\isacharparenright}\isanewline
\ \isacommand{apply}{\isacharparenleft}simp{\isacharparenright}\isanewline
\isacommand{by}{\isacharparenleft}force\ simp\ add{\isacharcolon}zabs{\isacharunderscore}def\ take{\isacharunderscore}Cons\ split{\isacharcolon}nat{\isachardot}split\ if{\isacharunderscore}splits{\isacharparenright}%
\begin{isamarkuptext}%
Finally we come to the above mentioned lemma about cutting a word with two
more elements of one sort than of the other sort into two halves:%
\end{isamarkuptext}%
\isacommand{lemma}\ part{\isadigit{1}}{\isacharcolon}\isanewline
\ {\isachardoublequote}size{\isacharbrackleft}x{\isasymin}w{\isachardot}\ P\ x{\isacharbrackright}\ {\isacharequal}\ size{\isacharbrackleft}x{\isasymin}w{\isachardot}\ {\isasymnot}P\ x{\isacharbrackright}{\isacharplus}{\isadigit{2}}\ {\isasymLongrightarrow}\isanewline
\ \ {\isasymexists}i{\isasymle}size\ w{\isachardot}\ size{\isacharbrackleft}x{\isasymin}take\ i\ w{\isachardot}\ P\ x{\isacharbrackright}\ {\isacharequal}\ size{\isacharbrackleft}x{\isasymin}take\ i\ w{\isachardot}\ {\isasymnot}P\ x{\isacharbrackright}{\isacharplus}{\isadigit{1}}{\isachardoublequote}%
\begin{isamarkuptxt}%
\noindent
This is proved with the help of the intermediate value theorem, instantiated
appropriately and with its first premise disposed of by lemma
\isa{step{\isadigit{1}}}.%
\end{isamarkuptxt}%
\isacommand{apply}{\isacharparenleft}insert\ nat{\isadigit{0}}{\isacharunderscore}intermed{\isacharunderscore}int{\isacharunderscore}val{\isacharbrackleft}OF\ step{\isadigit{1}}{\isacharcomma}\ of\ {\isachardoublequote}P{\isachardoublequote}\ {\isachardoublequote}w{\isachardoublequote}\ {\isachardoublequote}{\isacharhash}{\isadigit{1}}{\isachardoublequote}{\isacharbrackright}{\isacharparenright}\isanewline
\isacommand{apply}\ simp\isanewline
\isacommand{by}{\isacharparenleft}simp\ del{\isacharcolon}int{\isacharunderscore}Suc\ add{\isacharcolon}zdiff{\isacharunderscore}eq{\isacharunderscore}eq\ sym{\isacharbrackleft}OF\ int{\isacharunderscore}Suc{\isacharbrackright}{\isacharparenright}%
\begin{isamarkuptext}%
\noindent
The additional lemmas are needed to mediate between \isa{nat} and \isa{int}.

Lemma \isa{part{\isadigit{1}}} tells us only about the prefix \isa{take\ i\ w}.
The suffix \isa{drop\ i\ w} is dealt with in the following easy lemma:%
\end{isamarkuptext}%
\isacommand{lemma}\ part{\isadigit{2}}{\isacharcolon}\isanewline
\ \ {\isachardoublequote}{\isasymlbrakk}size{\isacharbrackleft}x{\isasymin}take\ i\ w\ {\isacharat}\ drop\ i\ w{\isachardot}\ P\ x{\isacharbrackright}\ {\isacharequal}\isanewline
\ \ \ \ size{\isacharbrackleft}x{\isasymin}take\ i\ w\ {\isacharat}\ drop\ i\ w{\isachardot}\ {\isasymnot}P\ x{\isacharbrackright}{\isacharplus}{\isadigit{2}}{\isacharsemicolon}\isanewline
\ \ \ \ size{\isacharbrackleft}x{\isasymin}take\ i\ w{\isachardot}\ P\ x{\isacharbrackright}\ {\isacharequal}\ size{\isacharbrackleft}x{\isasymin}take\ i\ w{\isachardot}\ {\isasymnot}P\ x{\isacharbrackright}{\isacharplus}{\isadigit{1}}{\isasymrbrakk}\isanewline
\ \ \ {\isasymLongrightarrow}\ size{\isacharbrackleft}x{\isasymin}drop\ i\ w{\isachardot}\ P\ x{\isacharbrackright}\ {\isacharequal}\ size{\isacharbrackleft}x{\isasymin}drop\ i\ w{\isachardot}\ {\isasymnot}P\ x{\isacharbrackright}{\isacharplus}{\isadigit{1}}{\isachardoublequote}\isanewline
\isacommand{by}{\isacharparenleft}simp\ del{\isacharcolon}append{\isacharunderscore}take{\isacharunderscore}drop{\isacharunderscore}id{\isacharparenright}%
\begin{isamarkuptext}%
\noindent
Lemma \isa{append{\isacharunderscore}take{\isacharunderscore}drop{\isacharunderscore}id}, \isa{take\ n\ xs\ {\isacharat}\ drop\ n\ xs\ {\isacharequal}\ xs},
which is generally useful, needs to be disabled for once.

To dispose of trivial cases automatically, the rules of the inductive
definition are declared simplification rules:%
\end{isamarkuptext}%
\isacommand{declare}\ S{\isacharunderscore}A{\isacharunderscore}B{\isachardot}intros{\isacharbrackleft}simp{\isacharbrackright}%
\begin{isamarkuptext}%
\noindent
This could have been done earlier but was not necessary so far.

The completeness theorem tells us that if a word has the same number of
\isa{a}'s and \isa{b}'s, then it is in \isa{S}, and similarly and
simultaneously for \isa{A} and \isa{B}:%
\end{isamarkuptext}%
\isacommand{theorem}\ completeness{\isacharcolon}\isanewline
\ \ {\isachardoublequote}{\isacharparenleft}size{\isacharbrackleft}x{\isasymin}w{\isachardot}\ x{\isacharequal}a{\isacharbrackright}\ {\isacharequal}\ size{\isacharbrackleft}x{\isasymin}w{\isachardot}\ x{\isacharequal}b{\isacharbrackright}\ \ \ \ \ {\isasymlongrightarrow}\ w\ {\isasymin}\ S{\isacharparenright}\ {\isasymand}\isanewline
\ \ \ {\isacharparenleft}size{\isacharbrackleft}x{\isasymin}w{\isachardot}\ x{\isacharequal}a{\isacharbrackright}\ {\isacharequal}\ size{\isacharbrackleft}x{\isasymin}w{\isachardot}\ x{\isacharequal}b{\isacharbrackright}\ {\isacharplus}\ {\isadigit{1}}\ {\isasymlongrightarrow}\ w\ {\isasymin}\ A{\isacharparenright}\ {\isasymand}\isanewline
\ \ \ {\isacharparenleft}size{\isacharbrackleft}x{\isasymin}w{\isachardot}\ x{\isacharequal}b{\isacharbrackright}\ {\isacharequal}\ size{\isacharbrackleft}x{\isasymin}w{\isachardot}\ x{\isacharequal}a{\isacharbrackright}\ {\isacharplus}\ {\isadigit{1}}\ {\isasymlongrightarrow}\ w\ {\isasymin}\ B{\isacharparenright}{\isachardoublequote}%
\begin{isamarkuptxt}%
\noindent
The proof is by induction on \isa{w}. Structural induction would fail here
because, as we can see from the grammar, we need to make bigger steps than
merely appending a single letter at the front. Hence we induct on the length
of \isa{w}, using the induction rule \isa{length{\isacharunderscore}induct}:%
\end{isamarkuptxt}%
\isacommand{apply}{\isacharparenleft}induct{\isacharunderscore}tac\ w\ rule{\isacharcolon}\ length{\isacharunderscore}induct{\isacharparenright}%
\begin{isamarkuptxt}%
\noindent
The \isa{rule} parameter tells \isa{induct{\isacharunderscore}tac} explicitly which induction
rule to use. For details see \S\ref{sec:complete-ind} below.
In this case the result is that we may assume the lemma already
holds for all words shorter than \isa{w}.

The proof continues with a case distinction on \isa{w},
i.e.\ if \isa{w} is empty or not.%
\end{isamarkuptxt}%
\isacommand{apply}{\isacharparenleft}case{\isacharunderscore}tac\ w{\isacharparenright}\isanewline
\ \isacommand{apply}{\isacharparenleft}simp{\isacharunderscore}all{\isacharparenright}%
\begin{isamarkuptxt}%
\noindent
Simplification disposes of the base case and leaves only two step
cases to be proved:
if \isa{w\ {\isacharequal}\ a\ {\isacharhash}\ v} and \isa{length\ {\isacharbrackleft}x{\isasymin}v\ {\isachardot}\ x\ {\isacharequal}\ a{\isacharbrackright}\ {\isacharequal}\ length\ {\isacharbrackleft}x{\isasymin}v\ {\isachardot}\ x\ {\isacharequal}\ b{\isacharbrackright}\ {\isacharplus}\ {\isadigit{2}}} then
\isa{b\ {\isacharhash}\ v\ {\isasymin}\ A}, and similarly for \isa{w\ {\isacharequal}\ b\ {\isacharhash}\ v}.
We only consider the first case in detail.

After breaking the conjuction up into two cases, we can apply
\isa{part{\isadigit{1}}} to the assumption that \isa{w} contains two more \isa{a}'s than \isa{b}'s.%
\end{isamarkuptxt}%
\isacommand{apply}{\isacharparenleft}rule\ conjI{\isacharparenright}\isanewline
\ \isacommand{apply}{\isacharparenleft}clarify{\isacharparenright}\isanewline
\ \isacommand{apply}{\isacharparenleft}frule\ part{\isadigit{1}}{\isacharbrackleft}of\ {\isachardoublequote}{\isasymlambda}x{\isachardot}\ x{\isacharequal}a{\isachardoublequote}{\isacharcomma}\ simplified{\isacharbrackright}{\isacharparenright}\isanewline
\ \isacommand{apply}{\isacharparenleft}erule\ exE{\isacharparenright}\isanewline
\ \isacommand{apply}{\isacharparenleft}erule\ conjE{\isacharparenright}%
\begin{isamarkuptxt}%
\noindent
This yields an index \isa{i\ {\isasymle}\ length\ v} such that
\isa{length\ {\isacharbrackleft}x{\isasymin}take\ i\ v\ {\isachardot}\ x\ {\isacharequal}\ a{\isacharbrackright}\ {\isacharequal}\ length\ {\isacharbrackleft}x{\isasymin}take\ i\ v\ {\isachardot}\ x\ {\isacharequal}\ b{\isacharbrackright}\ {\isacharplus}\ {\isadigit{1}}}.
With the help of \isa{part{\isadigit{1}}} it follows that
\isa{length\ {\isacharbrackleft}x{\isasymin}drop\ i\ v\ {\isachardot}\ x\ {\isacharequal}\ a{\isacharbrackright}\ {\isacharequal}\ length\ {\isacharbrackleft}x{\isasymin}drop\ i\ v\ {\isachardot}\ x\ {\isacharequal}\ b{\isacharbrackright}\ {\isacharplus}\ {\isadigit{1}}}.%
\end{isamarkuptxt}%
\ \isacommand{apply}{\isacharparenleft}drule\ part{\isadigit{2}}{\isacharbrackleft}of\ {\isachardoublequote}{\isasymlambda}x{\isachardot}\ x{\isacharequal}a{\isachardoublequote}{\isacharcomma}\ simplified{\isacharbrackright}{\isacharparenright}\isanewline
\ \ \isacommand{apply}{\isacharparenleft}assumption{\isacharparenright}%
\begin{isamarkuptxt}%
\noindent
Now it is time to decompose \isa{v} in the conclusion \isa{b\ {\isacharhash}\ v\ {\isasymin}\ A}
into \isa{take\ i\ v\ {\isacharat}\ drop\ i\ v},
after which the appropriate rule of the grammar reduces the goal
to the two subgoals \isa{take\ i\ v\ {\isasymin}\ A} and \isa{drop\ i\ v\ {\isasymin}\ A}:%
\end{isamarkuptxt}%
\ \isacommand{apply}{\isacharparenleft}rule{\isacharunderscore}tac\ n{\isadigit{1}}{\isacharequal}i\ \isakeyword{and}\ t{\isacharequal}v\ \isakeyword{in}\ subst{\isacharbrackleft}OF\ append{\isacharunderscore}take{\isacharunderscore}drop{\isacharunderscore}id{\isacharbrackright}{\isacharparenright}\isanewline
\ \isacommand{apply}{\isacharparenleft}rule\ S{\isacharunderscore}A{\isacharunderscore}B{\isachardot}intros{\isacharparenright}%
\begin{isamarkuptxt}%
\noindent
Both subgoals follow from the induction hypothesis because both \isa{take\ i\ v} and \isa{drop\ i\ v} are shorter than \isa{w}:%
\end{isamarkuptxt}%
\ \ \isacommand{apply}{\isacharparenleft}force\ simp\ add{\isacharcolon}\ min{\isacharunderscore}less{\isacharunderscore}iff{\isacharunderscore}disj{\isacharparenright}\isanewline
\ \isacommand{apply}{\isacharparenleft}force\ split\ add{\isacharcolon}\ nat{\isacharunderscore}diff{\isacharunderscore}split{\isacharparenright}%
\begin{isamarkuptxt}%
\noindent
Note that the variables \isa{n{\isadigit{1}}} and \isa{t} referred to in the
substitution step above come from the derived theorem \isa{subst{\isacharbrackleft}OF\ append{\isacharunderscore}take{\isacharunderscore}drop{\isacharunderscore}id{\isacharbrackright}}.

The case \isa{w\ {\isacharequal}\ b\ {\isacharhash}\ v} is proved completely analogously:%
\end{isamarkuptxt}%
\isacommand{apply}{\isacharparenleft}clarify{\isacharparenright}\isanewline
\isacommand{apply}{\isacharparenleft}frule\ part{\isadigit{1}}{\isacharbrackleft}of\ {\isachardoublequote}{\isasymlambda}x{\isachardot}\ x{\isacharequal}b{\isachardoublequote}{\isacharcomma}\ simplified{\isacharbrackright}{\isacharparenright}\isanewline
\isacommand{apply}{\isacharparenleft}erule\ exE{\isacharparenright}\isanewline
\isacommand{apply}{\isacharparenleft}erule\ conjE{\isacharparenright}\isanewline
\isacommand{apply}{\isacharparenleft}drule\ part{\isadigit{2}}{\isacharbrackleft}of\ {\isachardoublequote}{\isasymlambda}x{\isachardot}\ x{\isacharequal}b{\isachardoublequote}{\isacharcomma}\ simplified{\isacharbrackright}{\isacharparenright}\isanewline
\ \isacommand{apply}{\isacharparenleft}assumption{\isacharparenright}\isanewline
\isacommand{apply}{\isacharparenleft}rule{\isacharunderscore}tac\ n{\isadigit{1}}{\isacharequal}i\ \isakeyword{and}\ t{\isacharequal}v\ \isakeyword{in}\ subst{\isacharbrackleft}OF\ append{\isacharunderscore}take{\isacharunderscore}drop{\isacharunderscore}id{\isacharbrackright}{\isacharparenright}\isanewline
\isacommand{apply}{\isacharparenleft}rule\ S{\isacharunderscore}A{\isacharunderscore}B{\isachardot}intros{\isacharparenright}\isanewline
\ \isacommand{apply}{\isacharparenleft}force\ simp\ add{\isacharcolon}min{\isacharunderscore}less{\isacharunderscore}iff{\isacharunderscore}disj{\isacharparenright}\isanewline
\isacommand{by}{\isacharparenleft}force\ simp\ add{\isacharcolon}min{\isacharunderscore}less{\isacharunderscore}iff{\isacharunderscore}disj\ split\ add{\isacharcolon}\ nat{\isacharunderscore}diff{\isacharunderscore}split{\isacharparenright}%
\begin{isamarkuptext}%
We conclude this section with a comparison of the above proof and the one
in the textbook \cite[p.\ 81]{HopcroftUllman}. For a start, the texbook
grammar, for no good reason, excludes the empty word, which complicates
matters just a little bit because we now have 8 instead of our 7 productions.

More importantly, the proof itself is different: rather than separating the
two directions, they perform one induction on the length of a word. This
deprives them of the beauty of rule induction and in the easy direction
(correctness) their reasoning is more detailed than our \isa{auto}. For the
hard part (completeness), they consider just one of the cases that our \isa{simp{\isacharunderscore}all} disposes of automatically. Then they conclude the proof by saying
about the remaining cases: ``We do this in a manner similar to our method of
proof for part (1); this part is left to the reader''. But this is precisely
the part that requires the intermediate value theorem and thus is not at all
similar to the other cases (which are automatic in Isabelle). We conclude
that the authors are at least cavalier about this point and may even have
overlooked the slight difficulty lurking in the omitted cases. This is not
atypical for pen-and-paper proofs, once analysed in detail.%
\end{isamarkuptext}%
\end{isabellebody}%
%%% Local Variables:
%%% mode: latex
%%% TeX-master: "root"
%%% End:


\index{inductive definitions|)}
