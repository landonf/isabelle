%
\begin{isabellebody}%
\def\isabellecontext{CTL}%
%
\isadelimtheory
%
\endisadelimtheory
%
\isatagtheory
%
\endisatagtheory
{\isafoldtheory}%
%
\isadelimtheory
%
\endisadelimtheory
%
\isamarkupsubsection{Computation Tree Logic --- CTL%
}
\isamarkuptrue%
%
\begin{isamarkuptext}%
\label{sec:CTL}
\index{CTL|(}%
The semantics of PDL only needs reflexive transitive closure.
Let us be adventurous and introduce a more expressive temporal operator.
We extend the datatype
\isa{formula} by a new constructor%
\end{isamarkuptext}%
\isamarkuptrue%
\ \ \ \ \ \ \ \ \ \ \ \ \ \ \ \ \ \ {\isacharbar}\ AF\ formula%
\begin{isamarkuptext}%
\noindent
which stands for ``\emph{A}lways in the \emph{F}uture'':
on all infinite paths, at some point the formula holds.
Formalizing the notion of an infinite path is easy
in HOL: it is simply a function from \isa{nat} to \isa{state}.%
\end{isamarkuptext}%
\isamarkuptrue%
\isacommand{definition}\isamarkupfalse%
\ Paths\ {\isacharcolon}{\isacharcolon}\ {\isachardoublequoteopen}state\ {\isasymRightarrow}\ {\isacharparenleft}nat\ {\isasymRightarrow}\ state{\isacharparenright}set{\isachardoublequoteclose}\ \isakeyword{where}\isanewline
{\isachardoublequoteopen}Paths\ s\ {\isasymequiv}\ {\isacharbraceleft}p{\isachardot}\ s\ {\isacharequal}\ p\ {\isadigit{0}}\ {\isasymand}\ {\isacharparenleft}{\isasymforall}i{\isachardot}\ {\isacharparenleft}p\ i{\isacharcomma}\ p{\isacharparenleft}i{\isacharplus}{\isadigit{1}}{\isacharparenright}{\isacharparenright}\ {\isasymin}\ M{\isacharparenright}{\isacharbraceright}{\isachardoublequoteclose}%
\begin{isamarkuptext}%
\noindent
This definition allows a succinct statement of the semantics of \isa{AF}:
\footnote{Do not be misled: neither datatypes nor recursive functions can be
extended by new constructors or equations. This is just a trick of the
presentation (see \S\ref{sec:doc-prep-suppress}). In reality one has to define
a new datatype and a new function.}%
\end{isamarkuptext}%
\isamarkuptrue%
{\isachardoublequoteopen}s\ {\isasymTurnstile}\ AF\ f\ \ \ \ {\isacharequal}\ {\isacharparenleft}{\isasymforall}p\ {\isasymin}\ Paths\ s{\isachardot}\ {\isasymexists}i{\isachardot}\ p\ i\ {\isasymTurnstile}\ f{\isacharparenright}{\isachardoublequoteclose}%
\begin{isamarkuptext}%
\noindent
Model checking \isa{AF} involves a function which
is just complicated enough to warrant a separate definition:%
\end{isamarkuptext}%
\isamarkuptrue%
\isacommand{definition}\isamarkupfalse%
\ af\ {\isacharcolon}{\isacharcolon}\ {\isachardoublequoteopen}state\ set\ {\isasymRightarrow}\ state\ set\ {\isasymRightarrow}\ state\ set{\isachardoublequoteclose}\ \isakeyword{where}\isanewline
{\isachardoublequoteopen}af\ A\ T\ {\isasymequiv}\ A\ {\isasymunion}\ {\isacharbraceleft}s{\isachardot}\ {\isasymforall}t{\isachardot}\ {\isacharparenleft}s{\isacharcomma}\ t{\isacharparenright}\ {\isasymin}\ M\ {\isasymlongrightarrow}\ t\ {\isasymin}\ T{\isacharbraceright}{\isachardoublequoteclose}%
\begin{isamarkuptext}%
\noindent
Now we define \isa{mc\ {\isacharparenleft}AF\ f{\isacharparenright}} as the least set \isa{T} that includes
\isa{mc\ f} and all states all of whose direct successors are in \isa{T}:%
\end{isamarkuptext}%
\isamarkuptrue%
{\isachardoublequoteopen}mc{\isacharparenleft}AF\ f{\isacharparenright}\ \ \ \ {\isacharequal}\ lfp{\isacharparenleft}af{\isacharparenleft}mc\ f{\isacharparenright}{\isacharparenright}{\isachardoublequoteclose}%
\begin{isamarkuptext}%
\noindent
Because \isa{af} is monotone in its second argument (and also its first, but
that is irrelevant), \isa{af\ A} has a least fixed point:%
\end{isamarkuptext}%
\isamarkuptrue%
\isacommand{lemma}\isamarkupfalse%
\ mono{\isacharunderscore}af{\isacharcolon}\ {\isachardoublequoteopen}mono{\isacharparenleft}af\ A{\isacharparenright}{\isachardoublequoteclose}\isanewline
%
\isadelimproof
%
\endisadelimproof
%
\isatagproof
\isacommand{apply}\isamarkupfalse%
{\isacharparenleft}simp\ add{\isacharcolon}\ mono{\isacharunderscore}def\ af{\isacharunderscore}def{\isacharparenright}\isanewline
\isacommand{apply}\isamarkupfalse%
\ blast\isanewline
\isacommand{done}\isamarkupfalse%
%
\endisatagproof
{\isafoldproof}%
%
\isadelimproof
%
\endisadelimproof
%
\isadelimproof
%
\endisadelimproof
%
\isatagproof
%
\endisatagproof
{\isafoldproof}%
%
\isadelimproof
%
\endisadelimproof
%
\isadelimproof
%
\endisadelimproof
%
\isatagproof
%
\endisatagproof
{\isafoldproof}%
%
\isadelimproof
%
\endisadelimproof
%
\begin{isamarkuptext}%
All we need to prove now is  \isa{mc\ {\isacharparenleft}AF\ f{\isacharparenright}\ {\isacharequal}\ {\isacharbraceleft}s{\isachardot}\ s\ {\isasymTurnstile}\ AF\ f{\isacharbraceright}}, which states
that \isa{mc} and \isa{{\isasymTurnstile}} agree for \isa{AF}\@.
This time we prove the two inclusions separately, starting
with the easy one:%
\end{isamarkuptext}%
\isamarkuptrue%
\isacommand{theorem}\isamarkupfalse%
\ AF{\isacharunderscore}lemma{\isadigit{1}}{\isacharcolon}\ {\isachardoublequoteopen}lfp{\isacharparenleft}af\ A{\isacharparenright}\ {\isasymsubseteq}\ {\isacharbraceleft}s{\isachardot}\ {\isasymforall}p\ {\isasymin}\ Paths\ s{\isachardot}\ {\isasymexists}i{\isachardot}\ p\ i\ {\isasymin}\ A{\isacharbraceright}{\isachardoublequoteclose}%
\isadelimproof
%
\endisadelimproof
%
\isatagproof
%
\begin{isamarkuptxt}%
\noindent
In contrast to the analogous proof for \isa{EF}, and just
for a change, we do not use fixed point induction.  Park-induction,
named after David Park, is weaker but sufficient for this proof:
\begin{center}
\isa{f\ S\ {\isasymle}\ S\ {\isasymLongrightarrow}\ lfp\ f\ {\isasymle}\ S} \hfill (\isa{lfp{\isacharunderscore}lowerbound})
\end{center}
The instance of the premise \isa{f\ S\ {\isasymsubseteq}\ S} is proved pointwise,
a decision that \isa{auto} takes for us:%
\end{isamarkuptxt}%
\isamarkuptrue%
\isacommand{apply}\isamarkupfalse%
{\isacharparenleft}rule\ lfp{\isacharunderscore}lowerbound{\isacharparenright}\isanewline
\isacommand{apply}\isamarkupfalse%
{\isacharparenleft}auto\ simp\ add{\isacharcolon}\ af{\isacharunderscore}def\ Paths{\isacharunderscore}def{\isacharparenright}%
\begin{isamarkuptxt}%
\begin{isabelle}%
\ {\isadigit{1}}{\isachardot}\ {\isasymAnd}p{\isachardot}\ {\isasymlbrakk}{\isasymforall}t{\isachardot}\ {\isacharparenleft}p\ {\isadigit{0}}{\isacharcomma}\ t{\isacharparenright}\ {\isasymin}\ M\ {\isasymlongrightarrow}\isanewline
\isaindent{\ {\isadigit{1}}{\isachardot}\ {\isasymAnd}p{\isachardot}\ {\isasymlbrakk}{\isasymforall}t{\isachardot}\ }{\isacharparenleft}{\isasymforall}p{\isachardot}\ t\ {\isacharequal}\ p\ {\isadigit{0}}\ {\isasymand}\ {\isacharparenleft}{\isasymforall}i{\isachardot}\ {\isacharparenleft}p\ i{\isacharcomma}\ p\ {\isacharparenleft}Suc\ i{\isacharparenright}{\isacharparenright}\ {\isasymin}\ M{\isacharparenright}\ {\isasymlongrightarrow}\isanewline
\isaindent{\ {\isadigit{1}}{\isachardot}\ {\isasymAnd}p{\isachardot}\ {\isasymlbrakk}{\isasymforall}t{\isachardot}\ {\isacharparenleft}{\isasymforall}p{\isachardot}\ }{\isacharparenleft}{\isasymexists}i{\isachardot}\ p\ i\ {\isasymin}\ A{\isacharparenright}{\isacharparenright}{\isacharsemicolon}\isanewline
\isaindent{\ {\isadigit{1}}{\isachardot}\ {\isasymAnd}p{\isachardot}\ \ }{\isasymforall}i{\isachardot}\ {\isacharparenleft}p\ i{\isacharcomma}\ p\ {\isacharparenleft}Suc\ i{\isacharparenright}{\isacharparenright}\ {\isasymin}\ M{\isasymrbrakk}\isanewline
\isaindent{\ {\isadigit{1}}{\isachardot}\ {\isasymAnd}p{\isachardot}\ }{\isasymLongrightarrow}\ {\isasymexists}i{\isachardot}\ p\ i\ {\isasymin}\ A%
\end{isabelle}
In this remaining case, we set \isa{t} to \isa{p\ {\isadigit{1}}}.
The rest is automatic, which is surprising because it involves
finding the instantiation \isa{{\isasymlambda}i{\isachardot}\ p\ {\isacharparenleft}i\ {\isacharplus}\ {\isadigit{1}}{\isacharparenright}}
for \isa{{\isasymforall}p}.%
\end{isamarkuptxt}%
\isamarkuptrue%
\isacommand{apply}\isamarkupfalse%
{\isacharparenleft}erule{\isacharunderscore}tac\ x\ {\isacharequal}\ {\isachardoublequoteopen}p\ {\isadigit{1}}{\isachardoublequoteclose}\ \isakeyword{in}\ allE{\isacharparenright}\isanewline
\isacommand{apply}\isamarkupfalse%
{\isacharparenleft}auto{\isacharparenright}\isanewline
\isacommand{done}\isamarkupfalse%
%
\endisatagproof
{\isafoldproof}%
%
\isadelimproof
%
\endisadelimproof
%
\begin{isamarkuptext}%
The opposite inclusion is proved by contradiction: if some state
\isa{s} is not in \isa{lfp\ {\isacharparenleft}af\ A{\isacharparenright}}, then we can construct an
infinite \isa{A}-avoiding path starting from~\isa{s}. The reason is
that by unfolding \isa{lfp} we find that if \isa{s} is not in
\isa{lfp\ {\isacharparenleft}af\ A{\isacharparenright}}, then \isa{s} is not in \isa{A} and there is a
direct successor of \isa{s} that is again not in \mbox{\isa{lfp\ {\isacharparenleft}af\ A{\isacharparenright}}}. Iterating this argument yields the promised infinite
\isa{A}-avoiding path. Let us formalize this sketch.

The one-step argument in the sketch above
is proved by a variant of contraposition:%
\end{isamarkuptext}%
\isamarkuptrue%
\isacommand{lemma}\isamarkupfalse%
\ not{\isacharunderscore}in{\isacharunderscore}lfp{\isacharunderscore}afD{\isacharcolon}\isanewline
\ {\isachardoublequoteopen}s\ {\isasymnotin}\ lfp{\isacharparenleft}af\ A{\isacharparenright}\ {\isasymLongrightarrow}\ s\ {\isasymnotin}\ A\ {\isasymand}\ {\isacharparenleft}{\isasymexists}\ t{\isachardot}\ {\isacharparenleft}s{\isacharcomma}t{\isacharparenright}\ {\isasymin}\ M\ {\isasymand}\ t\ {\isasymnotin}\ lfp{\isacharparenleft}af\ A{\isacharparenright}{\isacharparenright}{\isachardoublequoteclose}\isanewline
%
\isadelimproof
%
\endisadelimproof
%
\isatagproof
\isacommand{apply}\isamarkupfalse%
{\isacharparenleft}erule\ contrapos{\isacharunderscore}np{\isacharparenright}\isanewline
\isacommand{apply}\isamarkupfalse%
{\isacharparenleft}subst\ lfp{\isacharunderscore}unfold{\isacharbrackleft}OF\ mono{\isacharunderscore}af{\isacharbrackright}{\isacharparenright}\isanewline
\isacommand{apply}\isamarkupfalse%
{\isacharparenleft}simp\ add{\isacharcolon}\ af{\isacharunderscore}def{\isacharparenright}\isanewline
\isacommand{done}\isamarkupfalse%
%
\endisatagproof
{\isafoldproof}%
%
\isadelimproof
%
\endisadelimproof
%
\begin{isamarkuptext}%
\noindent
We assume the negation of the conclusion and prove \isa{s\ {\isasymin}\ lfp\ {\isacharparenleft}af\ A{\isacharparenright}}.
Unfolding \isa{lfp} once and
simplifying with the definition of \isa{af} finishes the proof.

Now we iterate this process. The following construction of the desired
path is parameterized by a predicate \isa{Q} that should hold along the path:%
\end{isamarkuptext}%
\isamarkuptrue%
\isacommand{primrec}\isamarkupfalse%
\ path\ {\isacharcolon}{\isacharcolon}\ {\isachardoublequoteopen}state\ {\isasymRightarrow}\ {\isacharparenleft}state\ {\isasymRightarrow}\ bool{\isacharparenright}\ {\isasymRightarrow}\ {\isacharparenleft}nat\ {\isasymRightarrow}\ state{\isacharparenright}{\isachardoublequoteclose}\ \isakeyword{where}\isanewline
{\isachardoublequoteopen}path\ s\ Q\ {\isadigit{0}}\ {\isacharequal}\ s{\isachardoublequoteclose}\ {\isacharbar}\isanewline
{\isachardoublequoteopen}path\ s\ Q\ {\isacharparenleft}Suc\ n{\isacharparenright}\ {\isacharequal}\ {\isacharparenleft}SOME\ t{\isachardot}\ {\isacharparenleft}path\ s\ Q\ n{\isacharcomma}t{\isacharparenright}\ {\isasymin}\ M\ {\isasymand}\ Q\ t{\isacharparenright}{\isachardoublequoteclose}%
\begin{isamarkuptext}%
\noindent
Element \isa{n\ {\isacharplus}\ {\isadigit{1}}} on this path is some arbitrary successor
\isa{t} of element \isa{n} such that \isa{Q\ t} holds.  Remember that \isa{SOME\ t{\isachardot}\ R\ t}
is some arbitrary but fixed \isa{t} such that \isa{R\ t} holds (see \S\ref{sec:SOME}). Of
course, such a \isa{t} need not exist, but that is of no
concern to us since we will only use \isa{path} when a
suitable \isa{t} does exist.

Let us show that if each state \isa{s} that satisfies \isa{Q}
has a successor that again satisfies \isa{Q}, then there exists an infinite \isa{Q}-path:%
\end{isamarkuptext}%
\isamarkuptrue%
\isacommand{lemma}\isamarkupfalse%
\ infinity{\isacharunderscore}lemma{\isacharcolon}\isanewline
\ \ {\isachardoublequoteopen}{\isasymlbrakk}\ Q\ s{\isacharsemicolon}\ {\isasymforall}s{\isachardot}\ Q\ s\ {\isasymlongrightarrow}\ {\isacharparenleft}{\isasymexists}\ t{\isachardot}\ {\isacharparenleft}s{\isacharcomma}t{\isacharparenright}\ {\isasymin}\ M\ {\isasymand}\ Q\ t{\isacharparenright}\ {\isasymrbrakk}\ {\isasymLongrightarrow}\isanewline
\ \ \ {\isasymexists}p{\isasymin}Paths\ s{\isachardot}\ {\isasymforall}i{\isachardot}\ Q{\isacharparenleft}p\ i{\isacharparenright}{\isachardoublequoteclose}%
\isadelimproof
%
\endisadelimproof
%
\isatagproof
%
\begin{isamarkuptxt}%
\noindent
First we rephrase the conclusion slightly because we need to prove simultaneously
both the path property and the fact that \isa{Q} holds:%
\end{isamarkuptxt}%
\isamarkuptrue%
\isacommand{apply}\isamarkupfalse%
{\isacharparenleft}subgoal{\isacharunderscore}tac\isanewline
\ \ {\isachardoublequoteopen}{\isasymexists}p{\isachardot}\ s\ {\isacharequal}\ p\ {\isadigit{0}}\ {\isasymand}\ {\isacharparenleft}{\isasymforall}i{\isacharcolon}{\isacharcolon}nat{\isachardot}\ {\isacharparenleft}p\ i{\isacharcomma}\ p{\isacharparenleft}i{\isacharplus}{\isadigit{1}}{\isacharparenright}{\isacharparenright}\ {\isasymin}\ M\ {\isasymand}\ Q{\isacharparenleft}p\ i{\isacharparenright}{\isacharparenright}{\isachardoublequoteclose}{\isacharparenright}%
\begin{isamarkuptxt}%
\noindent
From this proposition the original goal follows easily:%
\end{isamarkuptxt}%
\isamarkuptrue%
\ \isacommand{apply}\isamarkupfalse%
{\isacharparenleft}simp\ add{\isacharcolon}\ Paths{\isacharunderscore}def{\isacharcomma}\ blast{\isacharparenright}%
\begin{isamarkuptxt}%
\noindent
The new subgoal is proved by providing the witness \isa{path\ s\ Q} for \isa{p}:%
\end{isamarkuptxt}%
\isamarkuptrue%
\isacommand{apply}\isamarkupfalse%
{\isacharparenleft}rule{\isacharunderscore}tac\ x\ {\isacharequal}\ {\isachardoublequoteopen}path\ s\ Q{\isachardoublequoteclose}\ \isakeyword{in}\ exI{\isacharparenright}\isanewline
\isacommand{apply}\isamarkupfalse%
{\isacharparenleft}clarsimp{\isacharparenright}%
\begin{isamarkuptxt}%
\noindent
After simplification and clarification, the subgoal has the following form:
\begin{isabelle}%
\ {\isadigit{1}}{\isachardot}\ {\isasymAnd}i{\isachardot}\ {\isasymlbrakk}Q\ s{\isacharsemicolon}\ {\isasymforall}s{\isachardot}\ Q\ s\ {\isasymlongrightarrow}\ {\isacharparenleft}{\isasymexists}t{\isachardot}\ {\isacharparenleft}s{\isacharcomma}\ t{\isacharparenright}\ {\isasymin}\ M\ {\isasymand}\ Q\ t{\isacharparenright}{\isasymrbrakk}\isanewline
\isaindent{\ {\isadigit{1}}{\isachardot}\ {\isasymAnd}i{\isachardot}\ }{\isasymLongrightarrow}\ {\isacharparenleft}path\ s\ Q\ i{\isacharcomma}\ SOME\ t{\isachardot}\ {\isacharparenleft}path\ s\ Q\ i{\isacharcomma}\ t{\isacharparenright}\ {\isasymin}\ M\ {\isasymand}\ Q\ t{\isacharparenright}\ {\isasymin}\ M\ {\isasymand}\isanewline
\isaindent{\ {\isadigit{1}}{\isachardot}\ {\isasymAnd}i{\isachardot}\ {\isasymLongrightarrow}\ }Q\ {\isacharparenleft}path\ s\ Q\ i{\isacharparenright}%
\end{isabelle}
It invites a proof by induction on \isa{i}:%
\end{isamarkuptxt}%
\isamarkuptrue%
\isacommand{apply}\isamarkupfalse%
{\isacharparenleft}induct{\isacharunderscore}tac\ i{\isacharparenright}\isanewline
\ \isacommand{apply}\isamarkupfalse%
{\isacharparenleft}simp{\isacharparenright}%
\begin{isamarkuptxt}%
\noindent
After simplification, the base case boils down to
\begin{isabelle}%
\ {\isadigit{1}}{\isachardot}\ {\isasymlbrakk}Q\ s{\isacharsemicolon}\ {\isasymforall}s{\isachardot}\ Q\ s\ {\isasymlongrightarrow}\ {\isacharparenleft}{\isasymexists}t{\isachardot}\ {\isacharparenleft}s{\isacharcomma}\ t{\isacharparenright}\ {\isasymin}\ M\ {\isasymand}\ Q\ t{\isacharparenright}{\isasymrbrakk}\isanewline
\isaindent{\ {\isadigit{1}}{\isachardot}\ }{\isasymLongrightarrow}\ {\isacharparenleft}s{\isacharcomma}\ SOME\ t{\isachardot}\ {\isacharparenleft}s{\isacharcomma}\ t{\isacharparenright}\ {\isasymin}\ M\ {\isasymand}\ Q\ t{\isacharparenright}\ {\isasymin}\ M%
\end{isabelle}
The conclusion looks exceedingly trivial: after all, \isa{t} is chosen such that \isa{{\isacharparenleft}s{\isacharcomma}\ t{\isacharparenright}\ {\isasymin}\ M}
holds. However, we first have to show that such a \isa{t} actually exists! This reasoning
is embodied in the theorem \isa{someI{\isadigit{2}}{\isacharunderscore}ex}:
\begin{isabelle}%
\ \ \ \ \ {\isasymlbrakk}{\isasymexists}a{\isachardot}\ {\isacharquery}P\ a{\isacharsemicolon}\ {\isasymAnd}x{\isachardot}\ {\isacharquery}P\ x\ {\isasymLongrightarrow}\ {\isacharquery}Q\ x{\isasymrbrakk}\ {\isasymLongrightarrow}\ {\isacharquery}Q\ {\isacharparenleft}SOME\ x{\isachardot}\ {\isacharquery}P\ x{\isacharparenright}%
\end{isabelle}
When we apply this theorem as an introduction rule, \isa{{\isacharquery}P\ x} becomes
\isa{{\isacharparenleft}s{\isacharcomma}\ x{\isacharparenright}\ {\isasymin}\ M\ {\isasymand}\ Q\ x} and \isa{{\isacharquery}Q\ x} becomes \isa{{\isacharparenleft}s{\isacharcomma}\ x{\isacharparenright}\ {\isasymin}\ M} and we have to prove
two subgoals: \isa{{\isasymexists}a{\isachardot}\ {\isacharparenleft}s{\isacharcomma}\ a{\isacharparenright}\ {\isasymin}\ M\ {\isasymand}\ Q\ a}, which follows from the assumptions, and
\isa{{\isacharparenleft}s{\isacharcomma}\ x{\isacharparenright}\ {\isasymin}\ M\ {\isasymand}\ Q\ x\ {\isasymLongrightarrow}\ {\isacharparenleft}s{\isacharcomma}\ x{\isacharparenright}\ {\isasymin}\ M}, which is trivial. Thus it is not surprising that
\isa{fast} can prove the base case quickly:%
\end{isamarkuptxt}%
\isamarkuptrue%
\ \isacommand{apply}\isamarkupfalse%
{\isacharparenleft}fast\ intro{\isacharcolon}\ someI{\isadigit{2}}{\isacharunderscore}ex{\isacharparenright}%
\begin{isamarkuptxt}%
\noindent
What is worth noting here is that we have used \methdx{fast} rather than
\isa{blast}.  The reason is that \isa{blast} would fail because it cannot
cope with \isa{someI{\isadigit{2}}{\isacharunderscore}ex}: unifying its conclusion with the current
subgoal is non-trivial because of the nested schematic variables. For
efficiency reasons \isa{blast} does not even attempt such unifications.
Although \isa{fast} can in principle cope with complicated unification
problems, in practice the number of unifiers arising is often prohibitive and
the offending rule may need to be applied explicitly rather than
automatically. This is what happens in the step case.

The induction step is similar, but more involved, because now we face nested
occurrences of \isa{SOME}. As a result, \isa{fast} is no longer able to
solve the subgoal and we apply \isa{someI{\isadigit{2}}{\isacharunderscore}ex} by hand.  We merely
show the proof commands but do not describe the details:%
\end{isamarkuptxt}%
\isamarkuptrue%
\isacommand{apply}\isamarkupfalse%
{\isacharparenleft}simp{\isacharparenright}\isanewline
\isacommand{apply}\isamarkupfalse%
{\isacharparenleft}rule\ someI{\isadigit{2}}{\isacharunderscore}ex{\isacharparenright}\isanewline
\ \isacommand{apply}\isamarkupfalse%
{\isacharparenleft}blast{\isacharparenright}\isanewline
\isacommand{apply}\isamarkupfalse%
{\isacharparenleft}rule\ someI{\isadigit{2}}{\isacharunderscore}ex{\isacharparenright}\isanewline
\ \isacommand{apply}\isamarkupfalse%
{\isacharparenleft}blast{\isacharparenright}\isanewline
\isacommand{apply}\isamarkupfalse%
{\isacharparenleft}blast{\isacharparenright}\isanewline
\isacommand{done}\isamarkupfalse%
%
\endisatagproof
{\isafoldproof}%
%
\isadelimproof
%
\endisadelimproof
%
\begin{isamarkuptext}%
Function \isa{path} has fulfilled its purpose now and can be forgotten.
It was merely defined to provide the witness in the proof of the
\isa{infinity{\isacharunderscore}lemma}. Aficionados of minimal proofs might like to know
that we could have given the witness without having to define a new function:
the term
\begin{isabelle}%
\ \ \ \ \ nat{\isacharunderscore}rec\ s\ {\isacharparenleft}{\isasymlambda}n\ t{\isachardot}\ SOME\ u{\isachardot}\ {\isacharparenleft}t{\isacharcomma}\ u{\isacharparenright}\ {\isasymin}\ M\ {\isasymand}\ Q\ u{\isacharparenright}%
\end{isabelle}
is extensionally equal to \isa{path\ s\ Q},
where \isa{nat{\isacharunderscore}rec} is the predefined primitive recursor on \isa{nat}.%
\end{isamarkuptext}%
\isamarkuptrue%
%
\isadelimproof
%
\endisadelimproof
%
\isatagproof
%
\endisatagproof
{\isafoldproof}%
%
\isadelimproof
%
\endisadelimproof
%
\begin{isamarkuptext}%
At last we can prove the opposite direction of \isa{AF{\isacharunderscore}lemma{\isadigit{1}}}:%
\end{isamarkuptext}%
\isamarkuptrue%
\isacommand{theorem}\isamarkupfalse%
\ AF{\isacharunderscore}lemma{\isadigit{2}}{\isacharcolon}\ {\isachardoublequoteopen}{\isacharbraceleft}s{\isachardot}\ {\isasymforall}p\ {\isasymin}\ Paths\ s{\isachardot}\ {\isasymexists}i{\isachardot}\ p\ i\ {\isasymin}\ A{\isacharbraceright}\ {\isasymsubseteq}\ lfp{\isacharparenleft}af\ A{\isacharparenright}{\isachardoublequoteclose}%
\isadelimproof
%
\endisadelimproof
%
\isatagproof
%
\begin{isamarkuptxt}%
\noindent
The proof is again pointwise and then by contraposition:%
\end{isamarkuptxt}%
\isamarkuptrue%
\isacommand{apply}\isamarkupfalse%
{\isacharparenleft}rule\ subsetI{\isacharparenright}\isanewline
\isacommand{apply}\isamarkupfalse%
{\isacharparenleft}erule\ contrapos{\isacharunderscore}pp{\isacharparenright}\isanewline
\isacommand{apply}\isamarkupfalse%
\ simp%
\begin{isamarkuptxt}%
\begin{isabelle}%
\ {\isadigit{1}}{\isachardot}\ {\isasymAnd}x{\isachardot}\ x\ {\isasymnotin}\ lfp\ {\isacharparenleft}af\ A{\isacharparenright}\ {\isasymLongrightarrow}\ {\isasymexists}p{\isasymin}Paths\ x{\isachardot}\ {\isasymforall}i{\isachardot}\ p\ i\ {\isasymnotin}\ A%
\end{isabelle}
Applying the \isa{infinity{\isacharunderscore}lemma} as a destruction rule leaves two subgoals, the second
premise of \isa{infinity{\isacharunderscore}lemma} and the original subgoal:%
\end{isamarkuptxt}%
\isamarkuptrue%
\isacommand{apply}\isamarkupfalse%
{\isacharparenleft}drule\ infinity{\isacharunderscore}lemma{\isacharparenright}%
\begin{isamarkuptxt}%
\begin{isabelle}%
\ {\isadigit{1}}{\isachardot}\ {\isasymAnd}x{\isachardot}\ {\isasymforall}s{\isachardot}\ s\ {\isasymnotin}\ lfp\ {\isacharparenleft}af\ A{\isacharparenright}\ {\isasymlongrightarrow}\ {\isacharparenleft}{\isasymexists}t{\isachardot}\ {\isacharparenleft}s{\isacharcomma}\ t{\isacharparenright}\ {\isasymin}\ M\ {\isasymand}\ t\ {\isasymnotin}\ lfp\ {\isacharparenleft}af\ A{\isacharparenright}{\isacharparenright}\isanewline
\ {\isadigit{2}}{\isachardot}\ {\isasymAnd}x{\isachardot}\ {\isasymexists}p{\isasymin}Paths\ x{\isachardot}\ {\isasymforall}i{\isachardot}\ p\ i\ {\isasymnotin}\ lfp\ {\isacharparenleft}af\ A{\isacharparenright}\ {\isasymLongrightarrow}\isanewline
\isaindent{\ {\isadigit{2}}{\isachardot}\ {\isasymAnd}x{\isachardot}\ }{\isasymexists}p{\isasymin}Paths\ x{\isachardot}\ {\isasymforall}i{\isachardot}\ p\ i\ {\isasymnotin}\ A%
\end{isabelle}
Both are solved automatically:%
\end{isamarkuptxt}%
\isamarkuptrue%
\ \isacommand{apply}\isamarkupfalse%
{\isacharparenleft}auto\ dest{\isacharcolon}\ not{\isacharunderscore}in{\isacharunderscore}lfp{\isacharunderscore}afD{\isacharparenright}\isanewline
\isacommand{done}\isamarkupfalse%
%
\endisatagproof
{\isafoldproof}%
%
\isadelimproof
%
\endisadelimproof
%
\begin{isamarkuptext}%
If you find these proofs too complicated, we recommend that you read
\S\ref{sec:CTL-revisited}, where we show how inductive definitions lead to
simpler arguments.

The main theorem is proved as for PDL, except that we also derive the
necessary equality \isa{lfp{\isacharparenleft}af\ A{\isacharparenright}\ {\isacharequal}\ {\isachardot}{\isachardot}{\isachardot}} by combining
\isa{AF{\isacharunderscore}lemma{\isadigit{1}}} and \isa{AF{\isacharunderscore}lemma{\isadigit{2}}} on the spot:%
\end{isamarkuptext}%
\isamarkuptrue%
\isacommand{theorem}\isamarkupfalse%
\ {\isachardoublequoteopen}mc\ f\ {\isacharequal}\ {\isacharbraceleft}s{\isachardot}\ s\ {\isasymTurnstile}\ f{\isacharbraceright}{\isachardoublequoteclose}\isanewline
%
\isadelimproof
%
\endisadelimproof
%
\isatagproof
\isacommand{apply}\isamarkupfalse%
{\isacharparenleft}induct{\isacharunderscore}tac\ f{\isacharparenright}\isanewline
\isacommand{apply}\isamarkupfalse%
{\isacharparenleft}auto\ simp\ add{\isacharcolon}\ EF{\isacharunderscore}lemma\ equalityI{\isacharbrackleft}OF\ AF{\isacharunderscore}lemma{\isadigit{1}}\ AF{\isacharunderscore}lemma{\isadigit{2}}{\isacharbrackright}{\isacharparenright}\isanewline
\isacommand{done}\isamarkupfalse%
%
\endisatagproof
{\isafoldproof}%
%
\isadelimproof
%
\endisadelimproof
%
\begin{isamarkuptext}%
The language defined above is not quite CTL\@. The latter also includes an
until-operator \isa{EU\ f\ g} with semantics ``there \emph{E}xists a path
where \isa{f} is true \emph{U}ntil \isa{g} becomes true''.  We need
an auxiliary function:%
\end{isamarkuptext}%
\isamarkuptrue%
\isacommand{consts}\isamarkupfalse%
\ until{\isacharcolon}{\isacharcolon}\ {\isachardoublequoteopen}state\ set\ {\isasymRightarrow}\ state\ set\ {\isasymRightarrow}\ state\ {\isasymRightarrow}\ state\ list\ {\isasymRightarrow}\ bool{\isachardoublequoteclose}\isanewline
\isacommand{primrec}\isamarkupfalse%
\isanewline
{\isachardoublequoteopen}until\ A\ B\ s\ {\isacharbrackleft}{\isacharbrackright}\ \ \ \ {\isacharequal}\ {\isacharparenleft}s\ {\isasymin}\ B{\isacharparenright}{\isachardoublequoteclose}\isanewline
{\isachardoublequoteopen}until\ A\ B\ s\ {\isacharparenleft}t{\isacharhash}p{\isacharparenright}\ {\isacharequal}\ {\isacharparenleft}s\ {\isasymin}\ A\ {\isasymand}\ {\isacharparenleft}s{\isacharcomma}t{\isacharparenright}\ {\isasymin}\ M\ {\isasymand}\ until\ A\ B\ t\ p{\isacharparenright}{\isachardoublequoteclose}%
\begin{isamarkuptext}%
\noindent
Expressing the semantics of \isa{EU} is now straightforward:
\begin{isabelle}%
\ \ \ \ \ s\ {\isasymTurnstile}\ EU\ f\ g\ {\isacharequal}\ {\isacharparenleft}{\isasymexists}p{\isachardot}\ until\ {\isacharbraceleft}t{\isachardot}\ t\ {\isasymTurnstile}\ f{\isacharbraceright}\ {\isacharbraceleft}t{\isachardot}\ t\ {\isasymTurnstile}\ g{\isacharbraceright}\ s\ p{\isacharparenright}%
\end{isabelle}
Note that \isa{EU} is not definable in terms of the other operators!

Model checking \isa{EU} is again a least fixed point construction:
\begin{isabelle}%
\ \ \ \ \ mc{\isacharparenleft}EU\ f\ g{\isacharparenright}\ {\isacharequal}\ lfp{\isacharparenleft}{\isasymlambda}T{\isachardot}\ mc\ g\ {\isasymunion}\ mc\ f\ {\isasyminter}\ {\isacharparenleft}M{\isasyminverse}\ {\isacharbackquote}{\isacharbackquote}\ T{\isacharparenright}{\isacharparenright}%
\end{isabelle}

\begin{exercise}
Extend the datatype of formulae by the above until operator
and prove the equivalence between semantics and model checking, i.e.\ that
\begin{isabelle}%
\ \ \ \ \ mc\ {\isacharparenleft}EU\ f\ g{\isacharparenright}\ {\isacharequal}\ {\isacharbraceleft}s{\isachardot}\ s\ {\isasymTurnstile}\ EU\ f\ g{\isacharbraceright}%
\end{isabelle}
%For readability you may want to annotate {term EU} with its customary syntax
%{text[display]"| EU formula formula    E[_ U _]"}
%which enables you to read and write {text"E[f U g]"} instead of {term"EU f g"}.
\end{exercise}
For more CTL exercises see, for example, Huth and Ryan \cite{Huth-Ryan-book}.%
\end{isamarkuptext}%
\isamarkuptrue%
%
\isadelimproof
%
\endisadelimproof
%
\isatagproof
%
\endisatagproof
{\isafoldproof}%
%
\isadelimproof
%
\endisadelimproof
%
\isadelimproof
%
\endisadelimproof
%
\isatagproof
%
\endisatagproof
{\isafoldproof}%
%
\isadelimproof
%
\endisadelimproof
%
\isadelimproof
%
\endisadelimproof
%
\isatagproof
%
\endisatagproof
{\isafoldproof}%
%
\isadelimproof
%
\endisadelimproof
%
\begin{isamarkuptext}%
Let us close this section with a few words about the executability of
our model checkers.  It is clear that if all sets are finite, they can be
represented as lists and the usual set operations are easily
implemented. Only \isa{lfp} requires a little thought.  Fortunately, theory
\isa{While{\isacharunderscore}Combinator} in the Library~\cite{HOL-Library} provides a
theorem stating that in the case of finite sets and a monotone
function~\isa{F}, the value of \mbox{\isa{lfp\ F}} can be computed by
iterated application of \isa{F} to~\isa{{\isacharbraceleft}{\isacharbraceright}} until a fixed point is
reached. It is actually possible to generate executable functional programs
from HOL definitions, but that is beyond the scope of the tutorial.%
\index{CTL|)}%
\end{isamarkuptext}%
\isamarkuptrue%
%
\isadelimtheory
%
\endisadelimtheory
%
\isatagtheory
%
\endisatagtheory
{\isafoldtheory}%
%
\isadelimtheory
%
\endisadelimtheory
\end{isabellebody}%
%%% Local Variables:
%%% mode: latex
%%% TeX-master: "root"
%%% End:
