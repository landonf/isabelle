%
\begin{isabellebody}%
\def\isabellecontext{examples}%
\isamarkupfalse%
%
\isadelimtheory
%
\endisadelimtheory
%
\isatagtheory
%
\endisatagtheory
{\isafoldtheory}%
%
\isadelimtheory
%
\endisadelimtheory
%
\begin{isamarkuptext}%
Here is a simple example, the \rmindex{Fibonacci function}:%
\end{isamarkuptext}%
\isamarkuptrue%
\isacommand{consts}\isamarkupfalse%
\ fib\ {\isacharcolon}{\isacharcolon}\ {\isachardoublequoteopen}nat\ {\isasymRightarrow}\ nat{\isachardoublequoteclose}\isanewline
\isacommand{recdef}\isamarkupfalse%
\ fib\ {\isachardoublequoteopen}measure{\isacharparenleft}{\isasymlambda}n{\isachardot}\ n{\isacharparenright}{\isachardoublequoteclose}\isanewline
\ \ {\isachardoublequoteopen}fib\ {\isadigit{0}}\ {\isacharequal}\ {\isadigit{0}}{\isachardoublequoteclose}\isanewline
\ \ {\isachardoublequoteopen}fib\ {\isacharparenleft}Suc\ {\isadigit{0}}{\isacharparenright}\ {\isacharequal}\ {\isadigit{1}}{\isachardoublequoteclose}\isanewline
\ \ {\isachardoublequoteopen}fib\ {\isacharparenleft}Suc{\isacharparenleft}Suc\ x{\isacharparenright}{\isacharparenright}\ {\isacharequal}\ fib\ x\ {\isacharplus}\ fib\ {\isacharparenleft}Suc\ x{\isacharparenright}{\isachardoublequoteclose}%
\begin{isamarkuptext}%
\noindent
\index{measure functions}%
The definition of \isa{fib} is accompanied by a \textbf{measure function}
\isa{{\isasymlambda}n{\isachardot}\ n} which maps the argument of \isa{fib} to a
natural number. The requirement is that in each equation the measure of the
argument on the left-hand side is strictly greater than the measure of the
argument of each recursive call. In the case of \isa{fib} this is
obviously true because the measure function is the identity and
\isa{Suc\ {\isacharparenleft}Suc\ x{\isacharparenright}} is strictly greater than both \isa{x} and
\isa{Suc\ x}.

Slightly more interesting is the insertion of a fixed element
between any two elements of a list:%
\end{isamarkuptext}%
\isamarkuptrue%
\isacommand{consts}\isamarkupfalse%
\ sep\ {\isacharcolon}{\isacharcolon}\ {\isachardoublequoteopen}{\isacharprime}a\ {\isasymtimes}\ {\isacharprime}a\ list\ {\isasymRightarrow}\ {\isacharprime}a\ list{\isachardoublequoteclose}\isanewline
\isacommand{recdef}\isamarkupfalse%
\ sep\ {\isachardoublequoteopen}measure\ {\isacharparenleft}{\isasymlambda}{\isacharparenleft}a{\isacharcomma}xs{\isacharparenright}{\isachardot}\ length\ xs{\isacharparenright}{\isachardoublequoteclose}\isanewline
\ \ {\isachardoublequoteopen}sep{\isacharparenleft}a{\isacharcomma}\ {\isacharbrackleft}{\isacharbrackright}{\isacharparenright}\ \ \ \ \ {\isacharequal}\ {\isacharbrackleft}{\isacharbrackright}{\isachardoublequoteclose}\isanewline
\ \ {\isachardoublequoteopen}sep{\isacharparenleft}a{\isacharcomma}\ {\isacharbrackleft}x{\isacharbrackright}{\isacharparenright}\ \ \ \ {\isacharequal}\ {\isacharbrackleft}x{\isacharbrackright}{\isachardoublequoteclose}\isanewline
\ \ {\isachardoublequoteopen}sep{\isacharparenleft}a{\isacharcomma}\ x{\isacharhash}y{\isacharhash}zs{\isacharparenright}\ {\isacharequal}\ x\ {\isacharhash}\ a\ {\isacharhash}\ sep{\isacharparenleft}a{\isacharcomma}y{\isacharhash}zs{\isacharparenright}{\isachardoublequoteclose}%
\begin{isamarkuptext}%
\noindent
This time the measure is the length of the list, which decreases with the
recursive call; the first component of the argument tuple is irrelevant.
The details of tupled $\lambda$-abstractions \isa{{\isasymlambda}{\isacharparenleft}x\isactrlsub {\isadigit{1}}{\isacharcomma}{\isasymdots}{\isacharcomma}x\isactrlsub n{\isacharparenright}} are
explained in \S\ref{sec:products}, but for now your intuition is all you need.

Pattern matching\index{pattern matching!and \isacommand{recdef}}
need not be exhaustive:%
\end{isamarkuptext}%
\isamarkuptrue%
\isacommand{consts}\isamarkupfalse%
\ last\ {\isacharcolon}{\isacharcolon}\ {\isachardoublequoteopen}{\isacharprime}a\ list\ {\isasymRightarrow}\ {\isacharprime}a{\isachardoublequoteclose}\isanewline
\isacommand{recdef}\isamarkupfalse%
\ last\ {\isachardoublequoteopen}measure\ {\isacharparenleft}{\isasymlambda}xs{\isachardot}\ length\ xs{\isacharparenright}{\isachardoublequoteclose}\isanewline
\ \ {\isachardoublequoteopen}last\ {\isacharbrackleft}x{\isacharbrackright}\ \ \ \ \ \ {\isacharequal}\ x{\isachardoublequoteclose}\isanewline
\ \ {\isachardoublequoteopen}last\ {\isacharparenleft}x{\isacharhash}y{\isacharhash}zs{\isacharparenright}\ {\isacharequal}\ last\ {\isacharparenleft}y{\isacharhash}zs{\isacharparenright}{\isachardoublequoteclose}%
\begin{isamarkuptext}%
Overlapping patterns are disambiguated by taking the order of equations into
account, just as in functional programming:%
\end{isamarkuptext}%
\isamarkuptrue%
\isacommand{consts}\isamarkupfalse%
\ sep{\isadigit{1}}\ {\isacharcolon}{\isacharcolon}\ {\isachardoublequoteopen}{\isacharprime}a\ {\isasymtimes}\ {\isacharprime}a\ list\ {\isasymRightarrow}\ {\isacharprime}a\ list{\isachardoublequoteclose}\isanewline
\isacommand{recdef}\isamarkupfalse%
\ sep{\isadigit{1}}\ {\isachardoublequoteopen}measure\ {\isacharparenleft}{\isasymlambda}{\isacharparenleft}a{\isacharcomma}xs{\isacharparenright}{\isachardot}\ length\ xs{\isacharparenright}{\isachardoublequoteclose}\isanewline
\ \ {\isachardoublequoteopen}sep{\isadigit{1}}{\isacharparenleft}a{\isacharcomma}\ x{\isacharhash}y{\isacharhash}zs{\isacharparenright}\ {\isacharequal}\ x\ {\isacharhash}\ a\ {\isacharhash}\ sep{\isadigit{1}}{\isacharparenleft}a{\isacharcomma}y{\isacharhash}zs{\isacharparenright}{\isachardoublequoteclose}\isanewline
\ \ {\isachardoublequoteopen}sep{\isadigit{1}}{\isacharparenleft}a{\isacharcomma}\ xs{\isacharparenright}\ \ \ \ \ {\isacharequal}\ xs{\isachardoublequoteclose}%
\begin{isamarkuptext}%
\noindent
To guarantee that the second equation can only be applied if the first
one does not match, Isabelle internally replaces the second equation
by the two possibilities that are left: \isa{sep{\isadigit{1}}\ {\isacharparenleft}a{\isacharcomma}\ {\isacharbrackleft}{\isacharbrackright}{\isacharparenright}\ {\isacharequal}\ {\isacharbrackleft}{\isacharbrackright}} and
\isa{sep{\isadigit{1}}\ {\isacharparenleft}a{\isacharcomma}\ {\isacharbrackleft}x{\isacharbrackright}{\isacharparenright}\ {\isacharequal}\ {\isacharbrackleft}x{\isacharbrackright}}.  Thus the functions \isa{sep} and
\isa{sep{\isadigit{1}}} are identical.

\begin{warn}
  \isacommand{recdef} only takes the first argument of a (curried)
  recursive function into account. This means both the termination measure
  and pattern matching can only use that first argument. In general, you will
  therefore have to combine several arguments into a tuple. In case only one
  argument is relevant for termination, you can also rearrange the order of
  arguments as in the following definition:
\end{warn}%
\end{isamarkuptext}%
\isamarkuptrue%
\isacommand{consts}\isamarkupfalse%
\ sep{\isadigit{2}}\ {\isacharcolon}{\isacharcolon}\ {\isachardoublequoteopen}{\isacharprime}a\ list\ {\isasymRightarrow}\ {\isacharprime}a\ {\isasymRightarrow}\ {\isacharprime}a\ list{\isachardoublequoteclose}\isanewline
\isacommand{recdef}\isamarkupfalse%
\ sep{\isadigit{2}}\ {\isachardoublequoteopen}measure\ length{\isachardoublequoteclose}\isanewline
\ \ {\isachardoublequoteopen}sep{\isadigit{2}}\ {\isacharparenleft}x{\isacharhash}y{\isacharhash}zs{\isacharparenright}\ {\isacharequal}\ {\isacharparenleft}{\isasymlambda}a{\isachardot}\ x\ {\isacharhash}\ a\ {\isacharhash}\ sep{\isadigit{2}}\ {\isacharparenleft}y{\isacharhash}zs{\isacharparenright}\ a{\isacharparenright}{\isachardoublequoteclose}\isanewline
\ \ {\isachardoublequoteopen}sep{\isadigit{2}}\ xs\ \ \ \ \ \ \ {\isacharequal}\ {\isacharparenleft}{\isasymlambda}a{\isachardot}\ xs{\isacharparenright}{\isachardoublequoteclose}%
\begin{isamarkuptext}%
Because of its pattern-matching syntax, \isacommand{recdef} is also useful
for the definition of non-recursive functions, where the termination measure
degenerates to the empty set \isa{{\isacharbraceleft}{\isacharbraceright}}:%
\end{isamarkuptext}%
\isamarkuptrue%
\isacommand{consts}\isamarkupfalse%
\ swap{\isadigit{1}}{\isadigit{2}}\ {\isacharcolon}{\isacharcolon}\ {\isachardoublequoteopen}{\isacharprime}a\ list\ {\isasymRightarrow}\ {\isacharprime}a\ list{\isachardoublequoteclose}\isanewline
\isacommand{recdef}\isamarkupfalse%
\ swap{\isadigit{1}}{\isadigit{2}}\ {\isachardoublequoteopen}{\isacharbraceleft}{\isacharbraceright}{\isachardoublequoteclose}\isanewline
\ \ {\isachardoublequoteopen}swap{\isadigit{1}}{\isadigit{2}}\ {\isacharparenleft}x{\isacharhash}y{\isacharhash}zs{\isacharparenright}\ {\isacharequal}\ y{\isacharhash}x{\isacharhash}zs{\isachardoublequoteclose}\isanewline
\ \ {\isachardoublequoteopen}swap{\isadigit{1}}{\isadigit{2}}\ zs\ \ \ \ \ \ \ {\isacharequal}\ zs{\isachardoublequoteclose}\isanewline
%
\isadelimtheory
%
\endisadelimtheory
%
\isatagtheory
%
\endisatagtheory
{\isafoldtheory}%
%
\isadelimtheory
%
\endisadelimtheory
\end{isabellebody}%
%%% Local Variables:
%%% mode: latex
%%% TeX-master: "root"
%%% End:
