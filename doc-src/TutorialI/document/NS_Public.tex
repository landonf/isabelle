%
\begin{isabellebody}%
\def\isabellecontext{NS{\isaliteral{5F}{\isacharunderscore}}Public}%
%
\isadelimtheory
%
\endisadelimtheory
%
\isatagtheory
%
\endisatagtheory
{\isafoldtheory}%
%
\isadelimtheory
%
\endisadelimtheory
%
\isamarkupsection{Modelling the Protocol \label{sec:modelling}%
}
\isamarkuptrue%
%
\begin{figure}
\begin{isabelle}
\isacommand{inductive{\isaliteral{5F}{\isacharunderscore}}set}\isamarkupfalse%
\ ns{\isaliteral{5F}{\isacharunderscore}}public\ {\isaliteral{3A}{\isacharcolon}}{\isaliteral{3A}{\isacharcolon}}\ {\isaliteral{22}{\isachardoublequoteopen}}event\ list\ set{\isaliteral{22}{\isachardoublequoteclose}}\isanewline
\ \ \isakeyword{where}\isanewline
\isanewline
\ \ \ Nil{\isaliteral{3A}{\isacharcolon}}\ \ {\isaliteral{22}{\isachardoublequoteopen}}{\isaliteral{5B}{\isacharbrackleft}}{\isaliteral{5D}{\isacharbrackright}}\ {\isaliteral{5C3C696E3E}{\isasymin}}\ ns{\isaliteral{5F}{\isacharunderscore}}public{\isaliteral{22}{\isachardoublequoteclose}}\isanewline
\isanewline
\isanewline
\ {\isaliteral{7C}{\isacharbar}}\ Fake{\isaliteral{3A}{\isacharcolon}}\ {\isaliteral{22}{\isachardoublequoteopen}}{\isaliteral{5C3C6C6272616B6B3E}{\isasymlbrakk}}evsf\ {\isaliteral{5C3C696E3E}{\isasymin}}\ ns{\isaliteral{5F}{\isacharunderscore}}public{\isaliteral{3B}{\isacharsemicolon}}\ \ X\ {\isaliteral{5C3C696E3E}{\isasymin}}\ synth\ {\isaliteral{28}{\isacharparenleft}}analz\ {\isaliteral{28}{\isacharparenleft}}knows\ Spy\ evsf{\isaliteral{29}{\isacharparenright}}{\isaliteral{29}{\isacharparenright}}{\isaliteral{5C3C726272616B6B3E}{\isasymrbrakk}}\isanewline
\ \ \ \ \ \ \ \ \ \ {\isaliteral{5C3C4C6F6E6772696768746172726F773E}{\isasymLongrightarrow}}\ Says\ Spy\ B\ X\ \ {\isaliteral{23}{\isacharhash}}\ evsf\ {\isaliteral{5C3C696E3E}{\isasymin}}\ ns{\isaliteral{5F}{\isacharunderscore}}public{\isaliteral{22}{\isachardoublequoteclose}}\isanewline
\isanewline
\isanewline
\ {\isaliteral{7C}{\isacharbar}}\ NS{\isadigit{1}}{\isaliteral{3A}{\isacharcolon}}\ \ {\isaliteral{22}{\isachardoublequoteopen}}{\isaliteral{5C3C6C6272616B6B3E}{\isasymlbrakk}}evs{\isadigit{1}}\ {\isaliteral{5C3C696E3E}{\isasymin}}\ ns{\isaliteral{5F}{\isacharunderscore}}public{\isaliteral{3B}{\isacharsemicolon}}\ \ Nonce\ NA\ {\isaliteral{5C3C6E6F74696E3E}{\isasymnotin}}\ used\ evs{\isadigit{1}}{\isaliteral{5C3C726272616B6B3E}{\isasymrbrakk}}\isanewline
\ \ \ \ \ \ \ \ \ \ {\isaliteral{5C3C4C6F6E6772696768746172726F773E}{\isasymLongrightarrow}}\ Says\ A\ B\ {\isaliteral{28}{\isacharparenleft}}Crypt\ {\isaliteral{28}{\isacharparenleft}}pubK\ B{\isaliteral{29}{\isacharparenright}}\ {\isaliteral{5C3C6C62726163653E}{\isasymlbrace}}Nonce\ NA{\isaliteral{2C}{\isacharcomma}}\ Agent\ A{\isaliteral{5C3C7262726163653E}{\isasymrbrace}}{\isaliteral{29}{\isacharparenright}}\isanewline
\ \ \ \ \ \ \ \ \ \ \ \ \ \ \ \ \ {\isaliteral{23}{\isacharhash}}\ evs{\isadigit{1}}\ \ {\isaliteral{5C3C696E3E}{\isasymin}}\ \ ns{\isaliteral{5F}{\isacharunderscore}}public{\isaliteral{22}{\isachardoublequoteclose}}\isanewline
\isanewline
\isanewline
\ {\isaliteral{7C}{\isacharbar}}\ NS{\isadigit{2}}{\isaliteral{3A}{\isacharcolon}}\ \ {\isaliteral{22}{\isachardoublequoteopen}}{\isaliteral{5C3C6C6272616B6B3E}{\isasymlbrakk}}evs{\isadigit{2}}\ {\isaliteral{5C3C696E3E}{\isasymin}}\ ns{\isaliteral{5F}{\isacharunderscore}}public{\isaliteral{3B}{\isacharsemicolon}}\ \ Nonce\ NB\ {\isaliteral{5C3C6E6F74696E3E}{\isasymnotin}}\ used\ evs{\isadigit{2}}{\isaliteral{3B}{\isacharsemicolon}}\isanewline
\ \ \ \ \ \ \ \ \ \ \ Says\ A{\isaliteral{27}{\isacharprime}}\ B\ {\isaliteral{28}{\isacharparenleft}}Crypt\ {\isaliteral{28}{\isacharparenleft}}pubK\ B{\isaliteral{29}{\isacharparenright}}\ {\isaliteral{5C3C6C62726163653E}{\isasymlbrace}}Nonce\ NA{\isaliteral{2C}{\isacharcomma}}\ Agent\ A{\isaliteral{5C3C7262726163653E}{\isasymrbrace}}{\isaliteral{29}{\isacharparenright}}\ {\isaliteral{5C3C696E3E}{\isasymin}}\ set\ evs{\isadigit{2}}{\isaliteral{5C3C726272616B6B3E}{\isasymrbrakk}}\isanewline
\ \ \ \ \ \ \ \ \ \ {\isaliteral{5C3C4C6F6E6772696768746172726F773E}{\isasymLongrightarrow}}\ Says\ B\ A\ {\isaliteral{28}{\isacharparenleft}}Crypt\ {\isaliteral{28}{\isacharparenleft}}pubK\ A{\isaliteral{29}{\isacharparenright}}\ {\isaliteral{5C3C6C62726163653E}{\isasymlbrace}}Nonce\ NA{\isaliteral{2C}{\isacharcomma}}\ Nonce\ NB{\isaliteral{2C}{\isacharcomma}}\ Agent\ B{\isaliteral{5C3C7262726163653E}{\isasymrbrace}}{\isaliteral{29}{\isacharparenright}}\isanewline
\ \ \ \ \ \ \ \ \ \ \ \ \ \ \ \ {\isaliteral{23}{\isacharhash}}\ evs{\isadigit{2}}\ \ {\isaliteral{5C3C696E3E}{\isasymin}}\ \ ns{\isaliteral{5F}{\isacharunderscore}}public{\isaliteral{22}{\isachardoublequoteclose}}\isanewline
\isanewline
\isanewline
\ {\isaliteral{7C}{\isacharbar}}\ NS{\isadigit{3}}{\isaliteral{3A}{\isacharcolon}}\ \ {\isaliteral{22}{\isachardoublequoteopen}}{\isaliteral{5C3C6C6272616B6B3E}{\isasymlbrakk}}evs{\isadigit{3}}\ {\isaliteral{5C3C696E3E}{\isasymin}}\ ns{\isaliteral{5F}{\isacharunderscore}}public{\isaliteral{3B}{\isacharsemicolon}}\isanewline
\ \ \ \ \ \ \ \ \ \ \ Says\ A\ \ B\ {\isaliteral{28}{\isacharparenleft}}Crypt\ {\isaliteral{28}{\isacharparenleft}}pubK\ B{\isaliteral{29}{\isacharparenright}}\ {\isaliteral{5C3C6C62726163653E}{\isasymlbrace}}Nonce\ NA{\isaliteral{2C}{\isacharcomma}}\ Agent\ A{\isaliteral{5C3C7262726163653E}{\isasymrbrace}}{\isaliteral{29}{\isacharparenright}}\ {\isaliteral{5C3C696E3E}{\isasymin}}\ set\ evs{\isadigit{3}}{\isaliteral{3B}{\isacharsemicolon}}\isanewline
\ \ \ \ \ \ \ \ \ \ \ Says\ B{\isaliteral{27}{\isacharprime}}\ A\ {\isaliteral{28}{\isacharparenleft}}Crypt\ {\isaliteral{28}{\isacharparenleft}}pubK\ A{\isaliteral{29}{\isacharparenright}}\ {\isaliteral{5C3C6C62726163653E}{\isasymlbrace}}Nonce\ NA{\isaliteral{2C}{\isacharcomma}}\ Nonce\ NB{\isaliteral{2C}{\isacharcomma}}\ Agent\ B{\isaliteral{5C3C7262726163653E}{\isasymrbrace}}{\isaliteral{29}{\isacharparenright}}\isanewline
\ \ \ \ \ \ \ \ \ \ \ \ \ \ {\isaliteral{5C3C696E3E}{\isasymin}}\ set\ evs{\isadigit{3}}{\isaliteral{5C3C726272616B6B3E}{\isasymrbrakk}}\isanewline
\ \ \ \ \ \ \ \ \ \ {\isaliteral{5C3C4C6F6E6772696768746172726F773E}{\isasymLongrightarrow}}\ Says\ A\ B\ {\isaliteral{28}{\isacharparenleft}}Crypt\ {\isaliteral{28}{\isacharparenleft}}pubK\ B{\isaliteral{29}{\isacharparenright}}\ {\isaliteral{28}{\isacharparenleft}}Nonce\ NB{\isaliteral{29}{\isacharparenright}}{\isaliteral{29}{\isacharparenright}}\ {\isaliteral{23}{\isacharhash}}\ evs{\isadigit{3}}\ {\isaliteral{5C3C696E3E}{\isasymin}}\ ns{\isaliteral{5F}{\isacharunderscore}}public{\isaliteral{22}{\isachardoublequoteclose}}%
\end{isabelle}
\caption{An Inductive Protocol Definition}\label{fig:ns_public}
\end{figure}
%
\begin{isamarkuptext}%
Let us formalize the Needham-Schroeder public-key protocol, as corrected by
Lowe:
\begin{alignat*%
}{2}
  &1.&\quad  A\to B  &: \comp{Na,A}\sb{Kb} \\
  &2.&\quad  B\to A  &: \comp{Na,Nb,B}\sb{Ka} \\
  &3.&\quad  A\to B  &: \comp{Nb}\sb{Kb}
\end{alignat*%
}

Each protocol step is specified by a rule of an inductive definition.  An
event trace has type \isa{event\ list}, so we declare the constant
\isa{ns{\isaliteral{5F}{\isacharunderscore}}public} to be a set of such traces.

Figure~\ref{fig:ns_public} presents the inductive definition.  The
\isa{Nil} rule introduces the empty trace.  The \isa{Fake} rule models the
adversary's sending a message built from components taken from past
traffic, expressed using the functions \isa{synth} and
\isa{analz}. 
The next three rules model how honest agents would perform the three
protocol steps.  

Here is a detailed explanation of rule \isa{NS{\isadigit{2}}}.
A trace containing an event of the form
\begin{isabelle}%
\ \ \ \ \ Says\ A{\isaliteral{27}{\isacharprime}}\ B\ {\isaliteral{28}{\isacharparenleft}}Crypt\ {\isaliteral{28}{\isacharparenleft}}pubK\ B{\isaliteral{29}{\isacharparenright}}\ {\isaliteral{5C3C6C62726163653E}{\isasymlbrace}}Nonce\ NA{\isaliteral{2C}{\isacharcomma}}\ Agent\ A{\isaliteral{5C3C7262726163653E}{\isasymrbrace}}{\isaliteral{29}{\isacharparenright}}%
\end{isabelle}
may be extended by an event of the form
\begin{isabelle}%
\ \ \ \ \ Says\ B\ A\ {\isaliteral{28}{\isacharparenleft}}Crypt\ {\isaliteral{28}{\isacharparenleft}}pubK\ A{\isaliteral{29}{\isacharparenright}}\ {\isaliteral{5C3C6C62726163653E}{\isasymlbrace}}Nonce\ NA{\isaliteral{2C}{\isacharcomma}}\ Nonce\ NB{\isaliteral{2C}{\isacharcomma}}\ Agent\ B{\isaliteral{5C3C7262726163653E}{\isasymrbrace}}{\isaliteral{29}{\isacharparenright}}%
\end{isabelle}
where \isa{NB} is a fresh nonce: \isa{Nonce\ NB\ {\isaliteral{5C3C6E6F74696E3E}{\isasymnotin}}\ used\ evs{\isadigit{2}}}.
Writing the sender as \isa{A{\isaliteral{27}{\isacharprime}}} indicates that \isa{B} does not 
know who sent the message.  Calling the trace variable \isa{evs{\isadigit{2}}} rather
than simply \isa{evs} helps us know where we are in a proof after many
case-splits: every subgoal mentioning \isa{evs{\isadigit{2}}} involves message~2 of the
protocol.

Benefits of this approach are simplicity and clarity.  The semantic model
is set theory, proofs are by induction and the translation from the informal
notation to the inductive rules is straightforward.%
\end{isamarkuptext}%
\isamarkuptrue%
%
\isamarkupsection{Proving Elementary Properties \label{sec:regularity}%
}
\isamarkuptrue%
%
\isadelimproof
%
\endisadelimproof
%
\isatagproof
%
\endisatagproof
{\isafoldproof}%
%
\isadelimproof
%
\endisadelimproof
%
\begin{isamarkuptext}%
Secrecy properties can be hard to prove.  The conclusion of a typical
secrecy theorem is 
\isa{X\ {\isaliteral{5C3C6E6F74696E3E}{\isasymnotin}}\ analz\ {\isaliteral{28}{\isacharparenleft}}knows\ Spy\ evs{\isaliteral{29}{\isacharparenright}}}.  The difficulty arises from
having to reason about \isa{analz}, or less formally, showing that the spy
can never learn~\isa{X}.  Much easier is to prove that \isa{X} can never
occur at all.  Such \emph{regularity} properties are typically expressed
using \isa{parts} rather than \isa{analz}.

The following lemma states that \isa{A}'s private key is potentially
known to the spy if and only if \isa{A} belongs to the set \isa{bad} of
compromised agents.  The statement uses \isa{parts}: the very presence of
\isa{A}'s private key in a message, whether protected by encryption or
not, is enough to confirm that \isa{A} is compromised.  The proof, like
nearly all protocol proofs, is by induction over traces.%
\end{isamarkuptext}%
\isamarkuptrue%
\isacommand{lemma}\isamarkupfalse%
\ Spy{\isaliteral{5F}{\isacharunderscore}}see{\isaliteral{5F}{\isacharunderscore}}priK\ {\isaliteral{5B}{\isacharbrackleft}}simp{\isaliteral{5D}{\isacharbrackright}}{\isaliteral{3A}{\isacharcolon}}\isanewline
\ \ \ \ \ \ {\isaliteral{22}{\isachardoublequoteopen}}evs\ {\isaliteral{5C3C696E3E}{\isasymin}}\ ns{\isaliteral{5F}{\isacharunderscore}}public\isanewline
\ \ \ \ \ \ \ {\isaliteral{5C3C4C6F6E6772696768746172726F773E}{\isasymLongrightarrow}}\ {\isaliteral{28}{\isacharparenleft}}Key\ {\isaliteral{28}{\isacharparenleft}}priK\ A{\isaliteral{29}{\isacharparenright}}\ {\isaliteral{5C3C696E3E}{\isasymin}}\ parts\ {\isaliteral{28}{\isacharparenleft}}knows\ Spy\ evs{\isaliteral{29}{\isacharparenright}}{\isaliteral{29}{\isacharparenright}}\ {\isaliteral{3D}{\isacharequal}}\ {\isaliteral{28}{\isacharparenleft}}A\ {\isaliteral{5C3C696E3E}{\isasymin}}\ bad{\isaliteral{29}{\isacharparenright}}{\isaliteral{22}{\isachardoublequoteclose}}\isanewline
%
\isadelimproof
%
\endisadelimproof
%
\isatagproof
\isacommand{apply}\isamarkupfalse%
\ {\isaliteral{28}{\isacharparenleft}}erule\ ns{\isaliteral{5F}{\isacharunderscore}}public{\isaliteral{2E}{\isachardot}}induct{\isaliteral{2C}{\isacharcomma}}\ simp{\isaliteral{5F}{\isacharunderscore}}all{\isaliteral{29}{\isacharparenright}}%
\begin{isamarkuptxt}%
The induction yields five subgoals, one for each rule in the definition of
\isa{ns{\isaliteral{5F}{\isacharunderscore}}public}.  The idea is to prove that the protocol property holds initially
(rule \isa{Nil}), is preserved by each of the legitimate protocol steps (rules
\isa{NS{\isadigit{1}}}--\isa{{\isadigit{3}}}), and even is preserved in the face of anything the
spy can do (rule \isa{Fake}).  

The proof is trivial.  No legitimate protocol rule sends any keys
at all, so only \isa{Fake} is relevant. Indeed, simplification leaves
only the \isa{Fake} case, as indicated by the variable name \isa{evsf}:
\begin{isabelle}%
\ {\isadigit{1}}{\isaliteral{2E}{\isachardot}}\ {\isaliteral{5C3C416E643E}{\isasymAnd}}evsf\ X{\isaliteral{2E}{\isachardot}}\isanewline
\isaindent{\ {\isadigit{1}}{\isaliteral{2E}{\isachardot}}\ \ \ \ }{\isaliteral{5C3C6C6272616B6B3E}{\isasymlbrakk}}evsf\ {\isaliteral{5C3C696E3E}{\isasymin}}\ ns{\isaliteral{5F}{\isacharunderscore}}public{\isaliteral{3B}{\isacharsemicolon}}\isanewline
\isaindent{\ {\isadigit{1}}{\isaliteral{2E}{\isachardot}}\ \ \ \ \ }{\isaliteral{28}{\isacharparenleft}}Key\ {\isaliteral{28}{\isacharparenleft}}priK\ A{\isaliteral{29}{\isacharparenright}}\ {\isaliteral{5C3C696E3E}{\isasymin}}\ parts\ {\isaliteral{28}{\isacharparenleft}}knows\ Spy\ evsf{\isaliteral{29}{\isacharparenright}}{\isaliteral{29}{\isacharparenright}}\ {\isaliteral{3D}{\isacharequal}}\ {\isaliteral{28}{\isacharparenleft}}A\ {\isaliteral{5C3C696E3E}{\isasymin}}\ bad{\isaliteral{29}{\isacharparenright}}{\isaliteral{3B}{\isacharsemicolon}}\isanewline
\isaindent{\ {\isadigit{1}}{\isaliteral{2E}{\isachardot}}\ \ \ \ \ }X\ {\isaliteral{5C3C696E3E}{\isasymin}}\ synth\ {\isaliteral{28}{\isacharparenleft}}analz\ {\isaliteral{28}{\isacharparenleft}}knows\ Spy\ evsf{\isaliteral{29}{\isacharparenright}}{\isaliteral{29}{\isacharparenright}}{\isaliteral{5C3C726272616B6B3E}{\isasymrbrakk}}\isanewline
\isaindent{\ {\isadigit{1}}{\isaliteral{2E}{\isachardot}}\ \ \ \ }{\isaliteral{5C3C4C6F6E6772696768746172726F773E}{\isasymLongrightarrow}}\ {\isaliteral{28}{\isacharparenleft}}Key\ {\isaliteral{28}{\isacharparenleft}}priK\ A{\isaliteral{29}{\isacharparenright}}\ {\isaliteral{5C3C696E3E}{\isasymin}}\ parts\ {\isaliteral{28}{\isacharparenleft}}insert\ X\ {\isaliteral{28}{\isacharparenleft}}knows\ Spy\ evsf{\isaliteral{29}{\isacharparenright}}{\isaliteral{29}{\isacharparenright}}{\isaliteral{29}{\isacharparenright}}\ {\isaliteral{3D}{\isacharequal}}\isanewline
\isaindent{\ {\isadigit{1}}{\isaliteral{2E}{\isachardot}}\ \ \ \ {\isaliteral{5C3C4C6F6E6772696768746172726F773E}{\isasymLongrightarrow}}\ }{\isaliteral{28}{\isacharparenleft}}A\ {\isaliteral{5C3C696E3E}{\isasymin}}\ bad{\isaliteral{29}{\isacharparenright}}%
\end{isabelle}%
\end{isamarkuptxt}%
\isamarkuptrue%
\isacommand{by}\isamarkupfalse%
\ blast%
\endisatagproof
{\isafoldproof}%
%
\isadelimproof
%
\endisadelimproof
%
\isadelimproof
%
\endisadelimproof
%
\isatagproof
%
\endisatagproof
{\isafoldproof}%
%
\isadelimproof
%
\endisadelimproof
%
\begin{isamarkuptext}%
The \isa{Fake} case is proved automatically.  If
\isa{priK\ A} is in the extended trace then either (1) it was already in the
original trace or (2) it was
generated by the spy, who must have known this key already. 
Either way, the induction hypothesis applies.

\emph{Unicity} lemmas are regularity lemmas stating that specified items
can occur only once in a trace.  The following lemma states that a nonce
cannot be used both as $Na$ and as $Nb$ unless
it is known to the spy.  Intuitively, it holds because honest agents
always choose fresh values as nonces; only the spy might reuse a value,
and he doesn't know this particular value.  The proof script is short:
induction, simplification, \isa{blast}.  The first line uses the rule
\isa{rev{\isaliteral{5F}{\isacharunderscore}}mp} to prepare the induction by moving two assumptions into the 
induction formula.%
\end{isamarkuptext}%
\isamarkuptrue%
\isacommand{lemma}\isamarkupfalse%
\ no{\isaliteral{5F}{\isacharunderscore}}nonce{\isaliteral{5F}{\isacharunderscore}}NS{\isadigit{1}}{\isaliteral{5F}{\isacharunderscore}}NS{\isadigit{2}}{\isaliteral{3A}{\isacharcolon}}\isanewline
\ \ \ \ {\isaliteral{22}{\isachardoublequoteopen}}{\isaliteral{5C3C6C6272616B6B3E}{\isasymlbrakk}}Crypt\ {\isaliteral{28}{\isacharparenleft}}pubK\ C{\isaliteral{29}{\isacharparenright}}\ {\isaliteral{5C3C6C62726163653E}{\isasymlbrace}}NA{\isaliteral{27}{\isacharprime}}{\isaliteral{2C}{\isacharcomma}}\ Nonce\ NA{\isaliteral{2C}{\isacharcomma}}\ Agent\ D{\isaliteral{5C3C7262726163653E}{\isasymrbrace}}\ {\isaliteral{5C3C696E3E}{\isasymin}}\ parts\ {\isaliteral{28}{\isacharparenleft}}knows\ Spy\ evs{\isaliteral{29}{\isacharparenright}}{\isaliteral{3B}{\isacharsemicolon}}\isanewline
\ \ \ \ \ \ Crypt\ {\isaliteral{28}{\isacharparenleft}}pubK\ B{\isaliteral{29}{\isacharparenright}}\ {\isaliteral{5C3C6C62726163653E}{\isasymlbrace}}Nonce\ NA{\isaliteral{2C}{\isacharcomma}}\ Agent\ A{\isaliteral{5C3C7262726163653E}{\isasymrbrace}}\ {\isaliteral{5C3C696E3E}{\isasymin}}\ parts\ {\isaliteral{28}{\isacharparenleft}}knows\ Spy\ evs{\isaliteral{29}{\isacharparenright}}{\isaliteral{3B}{\isacharsemicolon}}\isanewline
\ \ \ \ \ \ evs\ {\isaliteral{5C3C696E3E}{\isasymin}}\ ns{\isaliteral{5F}{\isacharunderscore}}public{\isaliteral{5C3C726272616B6B3E}{\isasymrbrakk}}\isanewline
\ \ \ \ \ {\isaliteral{5C3C4C6F6E6772696768746172726F773E}{\isasymLongrightarrow}}\ Nonce\ NA\ {\isaliteral{5C3C696E3E}{\isasymin}}\ analz\ {\isaliteral{28}{\isacharparenleft}}knows\ Spy\ evs{\isaliteral{29}{\isacharparenright}}{\isaliteral{22}{\isachardoublequoteclose}}\isanewline
%
\isadelimproof
%
\endisadelimproof
%
\isatagproof
\isacommand{apply}\isamarkupfalse%
\ {\isaliteral{28}{\isacharparenleft}}erule\ rev{\isaliteral{5F}{\isacharunderscore}}mp{\isaliteral{2C}{\isacharcomma}}\ erule\ rev{\isaliteral{5F}{\isacharunderscore}}mp{\isaliteral{29}{\isacharparenright}}\isanewline
\isacommand{apply}\isamarkupfalse%
\ {\isaliteral{28}{\isacharparenleft}}erule\ ns{\isaliteral{5F}{\isacharunderscore}}public{\isaliteral{2E}{\isachardot}}induct{\isaliteral{2C}{\isacharcomma}}\ simp{\isaliteral{5F}{\isacharunderscore}}all{\isaliteral{29}{\isacharparenright}}\isanewline
\isacommand{apply}\isamarkupfalse%
\ {\isaliteral{28}{\isacharparenleft}}blast\ intro{\isaliteral{3A}{\isacharcolon}}\ analz{\isaliteral{5F}{\isacharunderscore}}insertI{\isaliteral{29}{\isacharparenright}}{\isaliteral{2B}{\isacharplus}}\isanewline
\isacommand{done}\isamarkupfalse%
%
\endisatagproof
{\isafoldproof}%
%
\isadelimproof
%
\endisadelimproof
%
\begin{isamarkuptext}%
The following unicity lemma states that, if \isa{NA} is secret, then its
appearance in any instance of message~1 determines the other components. 
The proof is similar to the previous one.%
\end{isamarkuptext}%
\isamarkuptrue%
\isacommand{lemma}\isamarkupfalse%
\ unique{\isaliteral{5F}{\isacharunderscore}}NA{\isaliteral{3A}{\isacharcolon}}\isanewline
\ \ \ \ \ {\isaliteral{22}{\isachardoublequoteopen}}{\isaliteral{5C3C6C6272616B6B3E}{\isasymlbrakk}}Crypt{\isaliteral{28}{\isacharparenleft}}pubK\ B{\isaliteral{29}{\isacharparenright}}\ \ {\isaliteral{5C3C6C62726163653E}{\isasymlbrace}}Nonce\ NA{\isaliteral{2C}{\isacharcomma}}\ Agent\ A\ {\isaliteral{5C3C7262726163653E}{\isasymrbrace}}\ {\isaliteral{5C3C696E3E}{\isasymin}}\ parts{\isaliteral{28}{\isacharparenleft}}knows\ Spy\ evs{\isaliteral{29}{\isacharparenright}}{\isaliteral{3B}{\isacharsemicolon}}\isanewline
\ \ \ \ \ \ \ Crypt{\isaliteral{28}{\isacharparenleft}}pubK\ B{\isaliteral{27}{\isacharprime}}{\isaliteral{29}{\isacharparenright}}\ {\isaliteral{5C3C6C62726163653E}{\isasymlbrace}}Nonce\ NA{\isaliteral{2C}{\isacharcomma}}\ Agent\ A{\isaliteral{27}{\isacharprime}}{\isaliteral{5C3C7262726163653E}{\isasymrbrace}}\ {\isaliteral{5C3C696E3E}{\isasymin}}\ parts{\isaliteral{28}{\isacharparenleft}}knows\ Spy\ evs{\isaliteral{29}{\isacharparenright}}{\isaliteral{3B}{\isacharsemicolon}}\isanewline
\ \ \ \ \ \ \ Nonce\ NA\ {\isaliteral{5C3C6E6F74696E3E}{\isasymnotin}}\ analz\ {\isaliteral{28}{\isacharparenleft}}knows\ Spy\ evs{\isaliteral{29}{\isacharparenright}}{\isaliteral{3B}{\isacharsemicolon}}\ evs\ {\isaliteral{5C3C696E3E}{\isasymin}}\ ns{\isaliteral{5F}{\isacharunderscore}}public{\isaliteral{5C3C726272616B6B3E}{\isasymrbrakk}}\isanewline
\ \ \ \ \ \ {\isaliteral{5C3C4C6F6E6772696768746172726F773E}{\isasymLongrightarrow}}\ A{\isaliteral{3D}{\isacharequal}}A{\isaliteral{27}{\isacharprime}}\ {\isaliteral{5C3C616E643E}{\isasymand}}\ B{\isaliteral{3D}{\isacharequal}}B{\isaliteral{27}{\isacharprime}}{\isaliteral{22}{\isachardoublequoteclose}}%
\isadelimproof
%
\endisadelimproof
%
\isatagproof
%
\endisatagproof
{\isafoldproof}%
%
\isadelimproof
%
\endisadelimproof
%
\isamarkupsection{Proving Secrecy Theorems \label{sec:secrecy}%
}
\isamarkuptrue%
%
\isadelimproof
%
\endisadelimproof
%
\isatagproof
%
\endisatagproof
{\isafoldproof}%
%
\isadelimproof
%
\endisadelimproof
%
\isadelimproof
%
\endisadelimproof
%
\isatagproof
%
\endisatagproof
{\isafoldproof}%
%
\isadelimproof
%
\endisadelimproof
%
\isadelimproof
%
\endisadelimproof
%
\isatagproof
%
\endisatagproof
{\isafoldproof}%
%
\isadelimproof
%
\endisadelimproof
%
\isadelimproof
%
\endisadelimproof
%
\isatagproof
%
\endisatagproof
{\isafoldproof}%
%
\isadelimproof
%
\endisadelimproof
%
\isadelimproof
%
\endisadelimproof
%
\isatagproof
%
\endisatagproof
{\isafoldproof}%
%
\isadelimproof
%
\endisadelimproof
%
\begin{isamarkuptext}%
The secrecy theorems for Bob (the second participant) are especially
important because they fail for the original protocol.  The following
theorem states that if Bob sends message~2 to Alice, and both agents are
uncompromised, then Bob's nonce will never reach the spy.%
\end{isamarkuptext}%
\isamarkuptrue%
\isacommand{theorem}\isamarkupfalse%
\ Spy{\isaliteral{5F}{\isacharunderscore}}not{\isaliteral{5F}{\isacharunderscore}}see{\isaliteral{5F}{\isacharunderscore}}NB\ {\isaliteral{5B}{\isacharbrackleft}}dest{\isaliteral{5D}{\isacharbrackright}}{\isaliteral{3A}{\isacharcolon}}\isanewline
\ {\isaliteral{22}{\isachardoublequoteopen}}{\isaliteral{5C3C6C6272616B6B3E}{\isasymlbrakk}}Says\ B\ A\ {\isaliteral{28}{\isacharparenleft}}Crypt\ {\isaliteral{28}{\isacharparenleft}}pubK\ A{\isaliteral{29}{\isacharparenright}}\ {\isaliteral{5C3C6C62726163653E}{\isasymlbrace}}Nonce\ NA{\isaliteral{2C}{\isacharcomma}}\ Nonce\ NB{\isaliteral{2C}{\isacharcomma}}\ Agent\ B{\isaliteral{5C3C7262726163653E}{\isasymrbrace}}{\isaliteral{29}{\isacharparenright}}\ {\isaliteral{5C3C696E3E}{\isasymin}}\ set\ evs{\isaliteral{3B}{\isacharsemicolon}}\isanewline
\ \ \ A\ {\isaliteral{5C3C6E6F74696E3E}{\isasymnotin}}\ bad{\isaliteral{3B}{\isacharsemicolon}}\ \ B\ {\isaliteral{5C3C6E6F74696E3E}{\isasymnotin}}\ bad{\isaliteral{3B}{\isacharsemicolon}}\ \ evs\ {\isaliteral{5C3C696E3E}{\isasymin}}\ ns{\isaliteral{5F}{\isacharunderscore}}public{\isaliteral{5C3C726272616B6B3E}{\isasymrbrakk}}\isanewline
\ \ {\isaliteral{5C3C4C6F6E6772696768746172726F773E}{\isasymLongrightarrow}}\ Nonce\ NB\ {\isaliteral{5C3C6E6F74696E3E}{\isasymnotin}}\ analz\ {\isaliteral{28}{\isacharparenleft}}knows\ Spy\ evs{\isaliteral{29}{\isacharparenright}}{\isaliteral{22}{\isachardoublequoteclose}}%
\isadelimproof
%
\endisadelimproof
%
\isatagproof
%
\begin{isamarkuptxt}%
To prove it, we must formulate the induction properly (one of the
assumptions mentions~\isa{evs}), apply induction, and simplify:%
\end{isamarkuptxt}%
\isamarkuptrue%
\isacommand{apply}\isamarkupfalse%
\ {\isaliteral{28}{\isacharparenleft}}erule\ rev{\isaliteral{5F}{\isacharunderscore}}mp{\isaliteral{2C}{\isacharcomma}}\ erule\ ns{\isaliteral{5F}{\isacharunderscore}}public{\isaliteral{2E}{\isachardot}}induct{\isaliteral{2C}{\isacharcomma}}\ simp{\isaliteral{5F}{\isacharunderscore}}all{\isaliteral{29}{\isacharparenright}}%
\begin{isamarkuptxt}%
The proof states are too complicated to present in full.  
Let's examine the simplest subgoal, that for message~1.  The following
event has just occurred:
\[ 1.\quad  A'\to B'  : \comp{Na',A'}\sb{Kb'} \]
The variables above have been primed because this step
belongs to a different run from that referred to in the theorem
statement --- the theorem
refers to a past instance of message~2, while this subgoal
concerns message~1 being sent just now.
In the Isabelle subgoal, instead of primed variables like $B'$ and $Na'$
we have \isa{Ba} and~\isa{NAa}:
\begin{isabelle}%
\ {\isadigit{1}}{\isaliteral{2E}{\isachardot}}\ {\isaliteral{5C3C416E643E}{\isasymAnd}}evs{\isadigit{1}}\ NAa\ Ba{\isaliteral{2E}{\isachardot}}\isanewline
\isaindent{\ {\isadigit{1}}{\isaliteral{2E}{\isachardot}}\ \ \ \ }{\isaliteral{5C3C6C6272616B6B3E}{\isasymlbrakk}}A\ {\isaliteral{5C3C6E6F74696E3E}{\isasymnotin}}\ bad{\isaliteral{3B}{\isacharsemicolon}}\ B\ {\isaliteral{5C3C6E6F74696E3E}{\isasymnotin}}\ bad{\isaliteral{3B}{\isacharsemicolon}}\ evs{\isadigit{1}}\ {\isaliteral{5C3C696E3E}{\isasymin}}\ ns{\isaliteral{5F}{\isacharunderscore}}public{\isaliteral{3B}{\isacharsemicolon}}\isanewline
\isaindent{\ {\isadigit{1}}{\isaliteral{2E}{\isachardot}}\ \ \ \ \ }Says\ B\ A\ {\isaliteral{28}{\isacharparenleft}}Crypt\ {\isaliteral{28}{\isacharparenleft}}pubK\ A{\isaliteral{29}{\isacharparenright}}\ {\isaliteral{5C3C6C62726163653E}{\isasymlbrace}}Nonce\ NA{\isaliteral{2C}{\isacharcomma}}\ Nonce\ NB{\isaliteral{2C}{\isacharcomma}}\ Agent\ B{\isaliteral{5C3C7262726163653E}{\isasymrbrace}}{\isaliteral{29}{\isacharparenright}}\isanewline
\isaindent{\ {\isadigit{1}}{\isaliteral{2E}{\isachardot}}\ \ \ \ \ }{\isaliteral{5C3C696E3E}{\isasymin}}\ set\ evs{\isadigit{1}}\ {\isaliteral{5C3C6C6F6E6772696768746172726F773E}{\isasymlongrightarrow}}\isanewline
\isaindent{\ {\isadigit{1}}{\isaliteral{2E}{\isachardot}}\ \ \ \ \ }Nonce\ NB\ {\isaliteral{5C3C6E6F74696E3E}{\isasymnotin}}\ analz\ {\isaliteral{28}{\isacharparenleft}}knows\ Spy\ evs{\isadigit{1}}{\isaliteral{29}{\isacharparenright}}{\isaliteral{3B}{\isacharsemicolon}}\isanewline
\isaindent{\ {\isadigit{1}}{\isaliteral{2E}{\isachardot}}\ \ \ \ \ }Nonce\ NAa\ {\isaliteral{5C3C6E6F74696E3E}{\isasymnotin}}\ used\ evs{\isadigit{1}}{\isaliteral{5C3C726272616B6B3E}{\isasymrbrakk}}\isanewline
\isaindent{\ {\isadigit{1}}{\isaliteral{2E}{\isachardot}}\ \ \ \ }{\isaliteral{5C3C4C6F6E6772696768746172726F773E}{\isasymLongrightarrow}}\ Ba\ {\isaliteral{5C3C696E3E}{\isasymin}}\ bad\ {\isaliteral{5C3C6C6F6E6772696768746172726F773E}{\isasymlongrightarrow}}\isanewline
\isaindent{\ {\isadigit{1}}{\isaliteral{2E}{\isachardot}}\ \ \ \ {\isaliteral{5C3C4C6F6E6772696768746172726F773E}{\isasymLongrightarrow}}\ }Says\ B\ A\ {\isaliteral{28}{\isacharparenleft}}Crypt\ {\isaliteral{28}{\isacharparenleft}}pubK\ A{\isaliteral{29}{\isacharparenright}}\ {\isaliteral{5C3C6C62726163653E}{\isasymlbrace}}Nonce\ NA{\isaliteral{2C}{\isacharcomma}}\ Nonce\ NB{\isaliteral{2C}{\isacharcomma}}\ Agent\ B{\isaliteral{5C3C7262726163653E}{\isasymrbrace}}{\isaliteral{29}{\isacharparenright}}\isanewline
\isaindent{\ {\isadigit{1}}{\isaliteral{2E}{\isachardot}}\ \ \ \ {\isaliteral{5C3C4C6F6E6772696768746172726F773E}{\isasymLongrightarrow}}\ }{\isaliteral{5C3C696E3E}{\isasymin}}\ set\ evs{\isadigit{1}}\ {\isaliteral{5C3C6C6F6E6772696768746172726F773E}{\isasymlongrightarrow}}\isanewline
\isaindent{\ {\isadigit{1}}{\isaliteral{2E}{\isachardot}}\ \ \ \ {\isaliteral{5C3C4C6F6E6772696768746172726F773E}{\isasymLongrightarrow}}\ }NB\ {\isaliteral{5C3C6E6F7465713E}{\isasymnoteq}}\ NAa%
\end{isabelle}
The simplifier has used a 
default simplification rule that does a case
analysis for each encrypted message on whether or not the decryption key
is compromised.
\begin{isabelle}%
analz\ {\isaliteral{28}{\isacharparenleft}}insert\ {\isaliteral{28}{\isacharparenleft}}Crypt\ K\ X{\isaliteral{29}{\isacharparenright}}\ H{\isaliteral{29}{\isacharparenright}}\ {\isaliteral{3D}{\isacharequal}}\isanewline
{\isaliteral{28}{\isacharparenleft}}if\ Key\ {\isaliteral{28}{\isacharparenleft}}invKey\ K{\isaliteral{29}{\isacharparenright}}\ {\isaliteral{5C3C696E3E}{\isasymin}}\ analz\ H\isanewline
\isaindent{{\isaliteral{28}{\isacharparenleft}}}then\ insert\ {\isaliteral{28}{\isacharparenleft}}Crypt\ K\ X{\isaliteral{29}{\isacharparenright}}\ {\isaliteral{28}{\isacharparenleft}}analz\ {\isaliteral{28}{\isacharparenleft}}insert\ X\ H{\isaliteral{29}{\isacharparenright}}{\isaliteral{29}{\isacharparenright}}\isanewline
\isaindent{{\isaliteral{28}{\isacharparenleft}}}else\ insert\ {\isaliteral{28}{\isacharparenleft}}Crypt\ K\ X{\isaliteral{29}{\isacharparenright}}\ {\isaliteral{28}{\isacharparenleft}}analz\ H{\isaliteral{29}{\isacharparenright}}{\isaliteral{29}{\isacharparenright}}\rulename{analz{\isaliteral{5F}{\isacharunderscore}}Crypt{\isaliteral{5F}{\isacharunderscore}}if}%
\end{isabelle}
The simplifier has also used \isa{Spy{\isaliteral{5F}{\isacharunderscore}}see{\isaliteral{5F}{\isacharunderscore}}priK}, proved in
{\S}\ref{sec:regularity} above, to yield \isa{Ba\ {\isaliteral{5C3C696E3E}{\isasymin}}\ bad}.

Recall that this subgoal concerns the case
where the last message to be sent was
\[ 1.\quad  A'\to B'  : \comp{Na',A'}\sb{Kb'}. \]
This message can compromise $Nb$ only if $Nb=Na'$ and $B'$ is compromised,
allowing the spy to decrypt the message.  The Isabelle subgoal says
precisely this, if we allow for its choice of variable names.
Proving \isa{NB\ {\isaliteral{5C3C6E6F7465713E}{\isasymnoteq}}\ NAa} is easy: \isa{NB} was
sent earlier, while \isa{NAa} is fresh; formally, we have
the assumption \isa{Nonce\ NAa\ {\isaliteral{5C3C6E6F74696E3E}{\isasymnotin}}\ used\ evs{\isadigit{1}}}. 

Note that our reasoning concerned \isa{B}'s participation in another
run.  Agents may engage in several runs concurrently, and some attacks work
by interleaving the messages of two runs.  With model checking, this
possibility can cause a state-space explosion, and for us it
certainly complicates proofs.  The biggest subgoal concerns message~2.  It
splits into several cases, such as whether or not the message just sent is
the very message mentioned in the theorem statement.
Some of the cases are proved by unicity, others by
the induction hypothesis.  For all those complications, the proofs are
automatic by \isa{blast} with the theorem \isa{no{\isaliteral{5F}{\isacharunderscore}}nonce{\isaliteral{5F}{\isacharunderscore}}NS{\isadigit{1}}{\isaliteral{5F}{\isacharunderscore}}NS{\isadigit{2}}}.

The remaining theorems about the protocol are not hard to prove.  The
following one asserts a form of \emph{authenticity}: if
\isa{B} has sent an instance of message~2 to~\isa{A} and has received the
expected reply, then that reply really originated with~\isa{A}.  The
proof is a simple induction.%
\end{isamarkuptxt}%
\isamarkuptrue%
%
\endisatagproof
{\isafoldproof}%
%
\isadelimproof
%
\endisadelimproof
%
\isadelimproof
%
\endisadelimproof
%
\isatagproof
%
\endisatagproof
{\isafoldproof}%
%
\isadelimproof
%
\endisadelimproof
\isacommand{theorem}\isamarkupfalse%
\ B{\isaliteral{5F}{\isacharunderscore}}trusts{\isaliteral{5F}{\isacharunderscore}}NS{\isadigit{3}}{\isaliteral{3A}{\isacharcolon}}\isanewline
\ {\isaliteral{22}{\isachardoublequoteopen}}{\isaliteral{5C3C6C6272616B6B3E}{\isasymlbrakk}}Says\ B\ A\ \ {\isaliteral{28}{\isacharparenleft}}Crypt\ {\isaliteral{28}{\isacharparenleft}}pubK\ A{\isaliteral{29}{\isacharparenright}}\ {\isaliteral{5C3C6C62726163653E}{\isasymlbrace}}Nonce\ NA{\isaliteral{2C}{\isacharcomma}}\ Nonce\ NB{\isaliteral{2C}{\isacharcomma}}\ Agent\ B{\isaliteral{5C3C7262726163653E}{\isasymrbrace}}{\isaliteral{29}{\isacharparenright}}\ {\isaliteral{5C3C696E3E}{\isasymin}}\ set\ evs{\isaliteral{3B}{\isacharsemicolon}}\isanewline
\ \ \ Says\ A{\isaliteral{27}{\isacharprime}}\ B\ {\isaliteral{28}{\isacharparenleft}}Crypt\ {\isaliteral{28}{\isacharparenleft}}pubK\ B{\isaliteral{29}{\isacharparenright}}\ {\isaliteral{28}{\isacharparenleft}}Nonce\ NB{\isaliteral{29}{\isacharparenright}}{\isaliteral{29}{\isacharparenright}}\ {\isaliteral{5C3C696E3E}{\isasymin}}\ set\ evs{\isaliteral{3B}{\isacharsemicolon}}\isanewline
\ \ \ A\ {\isaliteral{5C3C6E6F74696E3E}{\isasymnotin}}\ bad{\isaliteral{3B}{\isacharsemicolon}}\ \ B\ {\isaliteral{5C3C6E6F74696E3E}{\isasymnotin}}\ bad{\isaliteral{3B}{\isacharsemicolon}}\ \ evs\ {\isaliteral{5C3C696E3E}{\isasymin}}\ ns{\isaliteral{5F}{\isacharunderscore}}public{\isaliteral{5C3C726272616B6B3E}{\isasymrbrakk}}\isanewline
\ \ {\isaliteral{5C3C4C6F6E6772696768746172726F773E}{\isasymLongrightarrow}}\ Says\ A\ B\ {\isaliteral{28}{\isacharparenleft}}Crypt\ {\isaliteral{28}{\isacharparenleft}}pubK\ B{\isaliteral{29}{\isacharparenright}}\ {\isaliteral{28}{\isacharparenleft}}Nonce\ NB{\isaliteral{29}{\isacharparenright}}{\isaliteral{29}{\isacharparenright}}\ {\isaliteral{5C3C696E3E}{\isasymin}}\ set\ evs{\isaliteral{22}{\isachardoublequoteclose}}%
\isadelimproof
%
\endisadelimproof
%
\isatagproof
%
\endisatagproof
{\isafoldproof}%
%
\isadelimproof
%
\endisadelimproof
%
\isadelimproof
%
\endisadelimproof
%
\isatagproof
%
\endisatagproof
{\isafoldproof}%
%
\isadelimproof
%
\endisadelimproof
%
\begin{isamarkuptext}%
From similar assumptions, we can prove that \isa{A} started the protocol
run by sending an instance of message~1 involving the nonce~\isa{NA}\@. 
For this theorem, the conclusion is 
\begin{isabelle}%
Says\ A\ B\ {\isaliteral{28}{\isacharparenleft}}Crypt\ {\isaliteral{28}{\isacharparenleft}}pubK\ B{\isaliteral{29}{\isacharparenright}}\ {\isaliteral{5C3C6C62726163653E}{\isasymlbrace}}Nonce\ NA{\isaliteral{2C}{\isacharcomma}}\ Agent\ A{\isaliteral{5C3C7262726163653E}{\isasymrbrace}}{\isaliteral{29}{\isacharparenright}}\ {\isaliteral{5C3C696E3E}{\isasymin}}\ set\ evs%
\end{isabelle}
Analogous theorems can be proved for~\isa{A}, stating that nonce~\isa{NA}
remains secret and that message~2 really originates with~\isa{B}.  Even the
flawed protocol establishes these properties for~\isa{A};
the flaw only harms the second participant.

\medskip

Detailed information on this protocol verification technique can be found
elsewhere~\cite{paulson-jcs}, including proofs of an Internet
protocol~\cite{paulson-tls}.  We must stress that the protocol discussed
in this chapter is trivial.  There are only three messages; no keys are
exchanged; we merely have to prove that encrypted data remains secret. 
Real world protocols are much longer and distribute many secrets to their
participants.  To be realistic, the model has to include the possibility
of keys being lost dynamically due to carelessness.  If those keys have
been used to encrypt other sensitive information, there may be cascading
losses.  We may still be able to establish a bound on the losses and to
prove that other protocol runs function
correctly~\cite{paulson-yahalom}.  Proofs of real-world protocols follow
the strategy illustrated above, but the subgoals can
be much bigger and there are more of them.
\index{protocols!security|)}%
\end{isamarkuptext}%
\isamarkuptrue%
%
\isadelimtheory
%
\endisadelimtheory
%
\isatagtheory
%
\endisatagtheory
{\isafoldtheory}%
%
\isadelimtheory
%
\endisadelimtheory
\end{isabellebody}%
%%% Local Variables:
%%% mode: latex
%%% TeX-master: "root"
%%% End:
