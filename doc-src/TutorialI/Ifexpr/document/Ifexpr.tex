\begin{isabelle}%
%
\begin{isamarkuptext}%
\subsubsection{How can we model boolean expressions?}

We want to represent boolean expressions built up from variables and
constants by negation and conjunction. The following datatype serves exactly
that purpose:%
\end{isamarkuptext}%
\isacommand{datatype}~boolex~=~Const~bool~|~Var~nat~|~Neg~boolex\isanewline
~~~~~~~~~~~~~~~~|~And~boolex~boolex%
\begin{isamarkuptext}%
\noindent
The two constants are represented by \isa{Const~True} and
\isa{Const~False}. Variables are represented by terms of the form
\isa{Var~$n$}, where $n$ is a natural number (type \isa{nat}).
For example, the formula $P@0 \land \neg P@1$ is represented by the term
\isa{And~(Var~0)~(Neg(Var~1))}.

\subsubsection{What is the value of a boolean expression?}

The value of a boolean expression depends on the value of its variables.
Hence the function \isa{value} takes an additional parameter, an {\em
  environment} of type \isa{nat \isasymFun\ bool}, which maps variables to
their values:%
\end{isamarkuptext}%
\isacommand{consts}~value~::~{"}boolex~{\isasymRightarrow}~(nat~{\isasymRightarrow}~bool)~{\isasymRightarrow}~bool{"}\isanewline
\isacommand{primrec}\isanewline
{"}value~(Const~b)~env~=~b{"}\isanewline
{"}value~(Var~x)~~~env~=~env~x{"}\isanewline
{"}value~(Neg~b)~~~env~=~({\isasymnot}~value~b~env){"}\isanewline
{"}value~(And~b~c)~env~=~(value~b~env~{\isasymand}~value~c~env){"}%
\begin{isamarkuptext}%
\noindent
\subsubsection{If-expressions}

An alternative and often more efficient (because in a certain sense
canonical) representation are so-called \emph{If-expressions} built up
from constants (\isa{CIF}), variables (\isa{VIF}) and conditionals
(\isa{IF}):%
\end{isamarkuptext}%
\isacommand{datatype}~ifex~=~CIF~bool~|~VIF~nat~|~IF~ifex~ifex~ifex%
\begin{isamarkuptext}%
\noindent
The evaluation if If-expressions proceeds as for \isa{boolex}:%
\end{isamarkuptext}%
\isacommand{consts}~valif~::~{"}ifex~{\isasymRightarrow}~(nat~{\isasymRightarrow}~bool)~{\isasymRightarrow}~bool{"}\isanewline
\isacommand{primrec}\isanewline
{"}valif~(CIF~b)~~~~env~=~b{"}\isanewline
{"}valif~(VIF~x)~~~~env~=~env~x{"}\isanewline
{"}valif~(IF~b~t~e)~env~=~(if~valif~b~env~then~valif~t~env\isanewline
~~~~~~~~~~~~~~~~~~~~~~~~~~~~~~~~~~~~~~~~else~valif~e~env){"}%
\begin{isamarkuptext}%
\subsubsection{Transformation into and of If-expressions}

The type \isa{boolex} is close to the customary representation of logical
formulae, whereas \isa{ifex} is designed for efficiency. Thus we need to
translate from \isa{boolex} into \isa{ifex}:%
\end{isamarkuptext}%
\isacommand{consts}~bool2if~::~{"}boolex~{\isasymRightarrow}~ifex{"}\isanewline
\isacommand{primrec}\isanewline
{"}bool2if~(Const~b)~=~CIF~b{"}\isanewline
{"}bool2if~(Var~x)~~~=~VIF~x{"}\isanewline
{"}bool2if~(Neg~b)~~~=~IF~(bool2if~b)~(CIF~False)~(CIF~True){"}\isanewline
{"}bool2if~(And~b~c)~=~IF~(bool2if~b)~(bool2if~c)~(CIF~False){"}%
\begin{isamarkuptext}%
\noindent
At last, we have something we can verify: that \isa{bool2if} preserves the
value of its argument:%
\end{isamarkuptext}%
\isacommand{lemma}~{"}valif~(bool2if~b)~env~=~value~b~env{"}%
\begin{isamarkuptxt}%
\noindent
The proof is canonical:%
\end{isamarkuptxt}%
\isacommand{apply}(induct\_tac~b)\isanewline
\isacommand{apply}(auto)\isacommand{.}%
\begin{isamarkuptext}%
\noindent
In fact, all proofs in this case study look exactly like this. Hence we do
not show them below.

More interesting is the transformation of If-expressions into a normal form
where the first argument of \isa{IF} cannot be another \isa{IF} but
must be a constant or variable. Such a normal form can be computed by
repeatedly replacing a subterm of the form \isa{IF~(IF~b~x~y)~z~u} by
\isa{IF b (IF x z u) (IF y z u)}, which has the same value. The following
primitive recursive functions perform this task:%
\end{isamarkuptext}%
\isacommand{consts}~normif~::~{"}ifex~{\isasymRightarrow}~ifex~{\isasymRightarrow}~ifex~{\isasymRightarrow}~ifex{"}\isanewline
\isacommand{primrec}\isanewline
{"}normif~(CIF~b)~~~~t~e~=~IF~(CIF~b)~t~e{"}\isanewline
{"}normif~(VIF~x)~~~~t~e~=~IF~(VIF~x)~t~e{"}\isanewline
{"}normif~(IF~b~t~e)~u~f~=~normif~b~(normif~t~u~f)~(normif~e~u~f){"}\isanewline
\isanewline
\isacommand{consts}~norm~::~{"}ifex~{\isasymRightarrow}~ifex{"}\isanewline
\isacommand{primrec}\isanewline
{"}norm~(CIF~b)~~~~=~CIF~b{"}\isanewline
{"}norm~(VIF~x)~~~~=~VIF~x{"}\isanewline
{"}norm~(IF~b~t~e)~=~normif~b~(norm~t)~(norm~e){"}%
\begin{isamarkuptext}%
\noindent
Their interplay is a bit tricky, and we leave it to the reader to develop an
intuitive understanding. Fortunately, Isabelle can help us to verify that the
transformation preserves the value of the expression:%
\end{isamarkuptext}%
\isacommand{theorem}~{"}valif~(norm~b)~env~=~valif~b~env{"}%
\begin{isamarkuptext}%
\noindent
The proof is canonical, provided we first show the following simplification
lemma (which also helps to understand what \isa{normif} does):%
\end{isamarkuptext}%
\isacommand{lemma}~[simp]:\isanewline
~~{"}{\isasymforall}t~e.~valif~(normif~b~t~e)~env~=~valif~(IF~b~t~e)~env{"}%
\begin{isamarkuptext}%
\noindent
Note that the lemma does not have a name, but is implicitly used in the proof
of the theorem shown above because of the \isa{[simp]} attribute.

But how can we be sure that \isa{norm} really produces a normal form in
the above sense? We define a function that tests If-expressions for normality%
\end{isamarkuptext}%
\isacommand{consts}~normal~::~{"}ifex~{\isasymRightarrow}~bool{"}\isanewline
\isacommand{primrec}\isanewline
{"}normal(CIF~b)~=~True{"}\isanewline
{"}normal(VIF~x)~=~True{"}\isanewline
{"}normal(IF~b~t~e)~=~(normal~t~{\isasymand}~normal~e~{\isasymand}\isanewline
~~~~~(case~b~of~CIF~b~{\isasymRightarrow}~True~|~VIF~x~{\isasymRightarrow}~True~|~IF~x~y~z~{\isasymRightarrow}~False)){"}%
\begin{isamarkuptext}%
\noindent
and prove \isa{normal(norm b)}. Of course, this requires a lemma about
normality of \isa{normif}:%
\end{isamarkuptext}%
\isacommand{lemma}~[simp]:~{"}{\isasymforall}t~e.~normal(normif~b~t~e)~=~(normal~t~{\isasymand}~normal~e){"}\end{isabelle}%
