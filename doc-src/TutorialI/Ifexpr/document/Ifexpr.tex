%
\begin{isabellebody}%
\def\isabellecontext{Ifexpr}%
%
\isamarkupsubsection{Case study: boolean expressions%
}
%
\begin{isamarkuptext}%
\label{sec:boolex}
The aim of this case study is twofold: it shows how to model boolean
expressions and some algorithms for manipulating them, and it demonstrates
the constructs introduced above.%
\end{isamarkuptext}%
%
\isamarkupsubsubsection{How can we model boolean expressions?%
}
%
\begin{isamarkuptext}%
We want to represent boolean expressions built up from variables and
constants by negation and conjunction. The following datatype serves exactly
that purpose:%
\end{isamarkuptext}%
\isacommand{datatype}\ boolex\ {\isacharequal}\ Const\ bool\ {\isacharbar}\ Var\ nat\ {\isacharbar}\ Neg\ boolex\isanewline
\ \ \ \ \ \ \ \ \ \ \ \ \ \ \ \ {\isacharbar}\ And\ boolex\ boolex%
\begin{isamarkuptext}%
\noindent
The two constants are represented by \isa{Const\ True} and
\isa{Const\ False}. Variables are represented by terms of the form
\isa{Var\ n}, where \isa{n} is a natural number (type \isa{nat}).
For example, the formula $P@0 \land \neg P@1$ is represented by the term
\isa{And\ {\isacharparenleft}Var\ {\isadigit{0}}{\isacharparenright}\ {\isacharparenleft}Neg\ {\isacharparenleft}Var\ {\isadigit{1}}{\isacharparenright}{\isacharparenright}}.

\subsubsection{What is the value of a boolean expression?}

The value of a boolean expression depends on the value of its variables.
Hence the function \isa{value} takes an additional parameter, an
\emph{environment} of type \isa{nat\ {\isasymRightarrow}\ bool}, which maps variables to their
values:%
\end{isamarkuptext}%
\isacommand{consts}\ value\ {\isacharcolon}{\isacharcolon}\ {\isachardoublequote}boolex\ {\isasymRightarrow}\ {\isacharparenleft}nat\ {\isasymRightarrow}\ bool{\isacharparenright}\ {\isasymRightarrow}\ bool{\isachardoublequote}\isanewline
\isacommand{primrec}\isanewline
{\isachardoublequote}value\ {\isacharparenleft}Const\ b{\isacharparenright}\ env\ {\isacharequal}\ b{\isachardoublequote}\isanewline
{\isachardoublequote}value\ {\isacharparenleft}Var\ x{\isacharparenright}\ \ \ env\ {\isacharequal}\ env\ x{\isachardoublequote}\isanewline
{\isachardoublequote}value\ {\isacharparenleft}Neg\ b{\isacharparenright}\ \ \ env\ {\isacharequal}\ {\isacharparenleft}{\isasymnot}\ value\ b\ env{\isacharparenright}{\isachardoublequote}\isanewline
{\isachardoublequote}value\ {\isacharparenleft}And\ b\ c{\isacharparenright}\ env\ {\isacharequal}\ {\isacharparenleft}value\ b\ env\ {\isasymand}\ value\ c\ env{\isacharparenright}{\isachardoublequote}%
\begin{isamarkuptext}%
\noindent
\subsubsection{If-expressions}

An alternative and often more efficient (because in a certain sense
canonical) representation are so-called \emph{If-expressions} built up
from constants (\isa{CIF}), variables (\isa{VIF}) and conditionals
(\isa{IF}):%
\end{isamarkuptext}%
\isacommand{datatype}\ ifex\ {\isacharequal}\ CIF\ bool\ {\isacharbar}\ VIF\ nat\ {\isacharbar}\ IF\ ifex\ ifex\ ifex%
\begin{isamarkuptext}%
\noindent
The evaluation if If-expressions proceeds as for \isa{boolex}:%
\end{isamarkuptext}%
\isacommand{consts}\ valif\ {\isacharcolon}{\isacharcolon}\ {\isachardoublequote}ifex\ {\isasymRightarrow}\ {\isacharparenleft}nat\ {\isasymRightarrow}\ bool{\isacharparenright}\ {\isasymRightarrow}\ bool{\isachardoublequote}\isanewline
\isacommand{primrec}\isanewline
{\isachardoublequote}valif\ {\isacharparenleft}CIF\ b{\isacharparenright}\ \ \ \ env\ {\isacharequal}\ b{\isachardoublequote}\isanewline
{\isachardoublequote}valif\ {\isacharparenleft}VIF\ x{\isacharparenright}\ \ \ \ env\ {\isacharequal}\ env\ x{\isachardoublequote}\isanewline
{\isachardoublequote}valif\ {\isacharparenleft}IF\ b\ t\ e{\isacharparenright}\ env\ {\isacharequal}\ {\isacharparenleft}if\ valif\ b\ env\ then\ valif\ t\ env\isanewline
\ \ \ \ \ \ \ \ \ \ \ \ \ \ \ \ \ \ \ \ \ \ \ \ \ \ \ \ \ \ \ \ \ \ \ \ \ \ \ \ else\ valif\ e\ env{\isacharparenright}{\isachardoublequote}%
\begin{isamarkuptext}%
\subsubsection{Transformation into and of If-expressions}

The type \isa{boolex} is close to the customary representation of logical
formulae, whereas \isa{ifex} is designed for efficiency. It is easy to
translate from \isa{boolex} into \isa{ifex}:%
\end{isamarkuptext}%
\isacommand{consts}\ bool{\isadigit{2}}if\ {\isacharcolon}{\isacharcolon}\ {\isachardoublequote}boolex\ {\isasymRightarrow}\ ifex{\isachardoublequote}\isanewline
\isacommand{primrec}\isanewline
{\isachardoublequote}bool{\isadigit{2}}if\ {\isacharparenleft}Const\ b{\isacharparenright}\ {\isacharequal}\ CIF\ b{\isachardoublequote}\isanewline
{\isachardoublequote}bool{\isadigit{2}}if\ {\isacharparenleft}Var\ x{\isacharparenright}\ \ \ {\isacharequal}\ VIF\ x{\isachardoublequote}\isanewline
{\isachardoublequote}bool{\isadigit{2}}if\ {\isacharparenleft}Neg\ b{\isacharparenright}\ \ \ {\isacharequal}\ IF\ {\isacharparenleft}bool{\isadigit{2}}if\ b{\isacharparenright}\ {\isacharparenleft}CIF\ False{\isacharparenright}\ {\isacharparenleft}CIF\ True{\isacharparenright}{\isachardoublequote}\isanewline
{\isachardoublequote}bool{\isadigit{2}}if\ {\isacharparenleft}And\ b\ c{\isacharparenright}\ {\isacharequal}\ IF\ {\isacharparenleft}bool{\isadigit{2}}if\ b{\isacharparenright}\ {\isacharparenleft}bool{\isadigit{2}}if\ c{\isacharparenright}\ {\isacharparenleft}CIF\ False{\isacharparenright}{\isachardoublequote}%
\begin{isamarkuptext}%
\noindent
At last, we have something we can verify: that \isa{bool{\isadigit{2}}if} preserves the
value of its argument:%
\end{isamarkuptext}%
\isacommand{lemma}\ {\isachardoublequote}valif\ {\isacharparenleft}bool{\isadigit{2}}if\ b{\isacharparenright}\ env\ {\isacharequal}\ value\ b\ env{\isachardoublequote}%
\begin{isamarkuptxt}%
\noindent
The proof is canonical:%
\end{isamarkuptxt}%
\isacommand{apply}{\isacharparenleft}induct{\isacharunderscore}tac\ b{\isacharparenright}\isanewline
\isacommand{apply}{\isacharparenleft}auto{\isacharparenright}\isanewline
\isacommand{done}%
\begin{isamarkuptext}%
\noindent
In fact, all proofs in this case study look exactly like this. Hence we do
not show them below.

More interesting is the transformation of If-expressions into a normal form
where the first argument of \isa{IF} cannot be another \isa{IF} but
must be a constant or variable. Such a normal form can be computed by
repeatedly replacing a subterm of the form \isa{IF\ {\isacharparenleft}IF\ b\ x\ y{\isacharparenright}\ z\ u} by
\isa{IF\ b\ {\isacharparenleft}IF\ x\ z\ u{\isacharparenright}\ {\isacharparenleft}IF\ y\ z\ u{\isacharparenright}}, which has the same value. The following
primitive recursive functions perform this task:%
\end{isamarkuptext}%
\isacommand{consts}\ normif\ {\isacharcolon}{\isacharcolon}\ {\isachardoublequote}ifex\ {\isasymRightarrow}\ ifex\ {\isasymRightarrow}\ ifex\ {\isasymRightarrow}\ ifex{\isachardoublequote}\isanewline
\isacommand{primrec}\isanewline
{\isachardoublequote}normif\ {\isacharparenleft}CIF\ b{\isacharparenright}\ \ \ \ t\ e\ {\isacharequal}\ IF\ {\isacharparenleft}CIF\ b{\isacharparenright}\ t\ e{\isachardoublequote}\isanewline
{\isachardoublequote}normif\ {\isacharparenleft}VIF\ x{\isacharparenright}\ \ \ \ t\ e\ {\isacharequal}\ IF\ {\isacharparenleft}VIF\ x{\isacharparenright}\ t\ e{\isachardoublequote}\isanewline
{\isachardoublequote}normif\ {\isacharparenleft}IF\ b\ t\ e{\isacharparenright}\ u\ f\ {\isacharequal}\ normif\ b\ {\isacharparenleft}normif\ t\ u\ f{\isacharparenright}\ {\isacharparenleft}normif\ e\ u\ f{\isacharparenright}{\isachardoublequote}\isanewline
\isanewline
\isacommand{consts}\ norm\ {\isacharcolon}{\isacharcolon}\ {\isachardoublequote}ifex\ {\isasymRightarrow}\ ifex{\isachardoublequote}\isanewline
\isacommand{primrec}\isanewline
{\isachardoublequote}norm\ {\isacharparenleft}CIF\ b{\isacharparenright}\ \ \ \ {\isacharequal}\ CIF\ b{\isachardoublequote}\isanewline
{\isachardoublequote}norm\ {\isacharparenleft}VIF\ x{\isacharparenright}\ \ \ \ {\isacharequal}\ VIF\ x{\isachardoublequote}\isanewline
{\isachardoublequote}norm\ {\isacharparenleft}IF\ b\ t\ e{\isacharparenright}\ {\isacharequal}\ normif\ b\ {\isacharparenleft}norm\ t{\isacharparenright}\ {\isacharparenleft}norm\ e{\isacharparenright}{\isachardoublequote}%
\begin{isamarkuptext}%
\noindent
Their interplay is a bit tricky, and we leave it to the reader to develop an
intuitive understanding. Fortunately, Isabelle can help us to verify that the
transformation preserves the value of the expression:%
\end{isamarkuptext}%
\isacommand{theorem}\ {\isachardoublequote}valif\ {\isacharparenleft}norm\ b{\isacharparenright}\ env\ {\isacharequal}\ valif\ b\ env{\isachardoublequote}%
\begin{isamarkuptext}%
\noindent
The proof is canonical, provided we first show the following simplification
lemma (which also helps to understand what \isa{normif} does):%
\end{isamarkuptext}%
\isacommand{lemma}\ {\isacharbrackleft}simp{\isacharbrackright}{\isacharcolon}\isanewline
\ \ {\isachardoublequote}{\isasymforall}t\ e{\isachardot}\ valif\ {\isacharparenleft}normif\ b\ t\ e{\isacharparenright}\ env\ {\isacharequal}\ valif\ {\isacharparenleft}IF\ b\ t\ e{\isacharparenright}\ env{\isachardoublequote}%
\begin{isamarkuptext}%
\noindent
Note that the lemma does not have a name, but is implicitly used in the proof
of the theorem shown above because of the \isa{{\isacharbrackleft}simp{\isacharbrackright}} attribute.

But how can we be sure that \isa{norm} really produces a normal form in
the above sense? We define a function that tests If-expressions for normality%
\end{isamarkuptext}%
\isacommand{consts}\ normal\ {\isacharcolon}{\isacharcolon}\ {\isachardoublequote}ifex\ {\isasymRightarrow}\ bool{\isachardoublequote}\isanewline
\isacommand{primrec}\isanewline
{\isachardoublequote}normal{\isacharparenleft}CIF\ b{\isacharparenright}\ {\isacharequal}\ True{\isachardoublequote}\isanewline
{\isachardoublequote}normal{\isacharparenleft}VIF\ x{\isacharparenright}\ {\isacharequal}\ True{\isachardoublequote}\isanewline
{\isachardoublequote}normal{\isacharparenleft}IF\ b\ t\ e{\isacharparenright}\ {\isacharequal}\ {\isacharparenleft}normal\ t\ {\isasymand}\ normal\ e\ {\isasymand}\isanewline
\ \ \ \ \ {\isacharparenleft}case\ b\ of\ CIF\ b\ {\isasymRightarrow}\ True\ {\isacharbar}\ VIF\ x\ {\isasymRightarrow}\ True\ {\isacharbar}\ IF\ x\ y\ z\ {\isasymRightarrow}\ False{\isacharparenright}{\isacharparenright}{\isachardoublequote}%
\begin{isamarkuptext}%
\noindent
and prove \isa{normal\ {\isacharparenleft}norm\ b{\isacharparenright}}. Of course, this requires a lemma about
normality of \isa{normif}:%
\end{isamarkuptext}%
\isacommand{lemma}{\isacharbrackleft}simp{\isacharbrackright}{\isacharcolon}\ {\isachardoublequote}{\isasymforall}t\ e{\isachardot}\ normal{\isacharparenleft}normif\ b\ t\ e{\isacharparenright}\ {\isacharequal}\ {\isacharparenleft}normal\ t\ {\isasymand}\ normal\ e{\isacharparenright}{\isachardoublequote}%
\begin{isamarkuptext}%
\medskip
How does one come up with the required lemmas? Try to prove the main theorems
without them and study carefully what \isa{auto} leaves unproved. This has
to provide the clue.  The necessity of universal quantification
(\isa{{\isasymforall}t\ e}) in the two lemmas is explained in
\S\ref{sec:InductionHeuristics}

\begin{exercise}
  We strengthen the definition of a \isa{normal} If-expression as follows:
  the first argument of all \isa{IF}s must be a variable. Adapt the above
  development to this changed requirement. (Hint: you may need to formulate
  some of the goals as implications (\isa{{\isasymlongrightarrow}}) rather than
  equalities (\isa{{\isacharequal}}).)
\end{exercise}%
\end{isamarkuptext}%
\end{isabellebody}%
%%% Local Variables:
%%% mode: latex
%%% TeX-master: "root"
%%% End:
