\begin{isabelle}%
\isacommand{datatype}~('a,'i)bigtree~=~Tip~|~Branch~'a~{"}'i~{\isasymRightarrow}~('a,'i)bigtree{"}%
\begin{isamarkuptext}%
\noindent Parameter \isa{'a} is the type of values stored in
the \isa{Branch}es of the tree, whereas \isa{'i} is the index
type over which the tree branches. If \isa{'i} is instantiated to
\isa{bool}, the result is a binary tree; if it is instantiated to
\isa{nat}, we have an infinitely branching tree because each node
has as many subtrees as there are natural numbers. How can we possibly
write down such a tree? Using functional notation! For example, the%
\end{isamarkuptext}%
\isacommand{term}~{"}Branch~0~({\isasymlambda}i.~Branch~i~({\isasymlambda}n.~Tip)){"}%
\begin{isamarkuptext}%
\noindent of type \isa{(nat,nat)bigtree} is the tree whose
root is labeled with 0 and whose $n$th subtree is labeled with $n$ and
has merely \isa{Tip}s as further subtrees.

Function \isa{map_bt} applies a function to all labels in a \isa{bigtree}:%
\end{isamarkuptext}%
\isacommand{consts}~map\_bt~::~{"}('a~{\isasymRightarrow}~'b)~{\isasymRightarrow}~('a,'i)bigtree~{\isasymRightarrow}~('b,'i)bigtree{"}\isanewline
\isacommand{primrec}\isanewline
{"}map\_bt~f~Tip~~~~~~~~~~=~Tip{"}\isanewline
{"}map\_bt~f~(Branch~a~F)~=~Branch~(f~a)~({\isasymlambda}i.~map\_bt~f~(F~i)){"}%
\begin{isamarkuptext}%
\noindent This is a valid \isacommand{primrec} definition because the
recursive calls of \isa{map_bt} involve only subtrees obtained from
\isa{F}, i.e.\ the left-hand side. Thus termination is assured.  The
seasoned functional programmer might have written \isa{map_bt~f~o~F} instead
of \isa{\isasymlambda{}i.~map_bt~f~(F~i)}, but the former is not accepted by
Isabelle because the termination proof is not as obvious since
\isa{map_bt} is only partially applied.

The following lemma has a canonical proof%
\end{isamarkuptext}%
\isacommand{lemma}~{"}map\_bt~g~(map\_bt~f~T)~=~map\_bt~(g~o~f)~T{"}\isanewline
\isacommand{apply}(induct\_tac~{"}T{"},~auto)\isacommand{.}%
\begin{isamarkuptext}%
\noindent
but it is worth taking a look at the proof state after the induction step
to understand what the presence of the function type entails:
\begin{isabellepar}%
~1.~map\_bt~g~(map\_bt~f~Tip)~=~map\_bt~(g~{\isasymcirc}~f)~Tip\isanewline
~2.~{\isasymAnd}a~F.\isanewline
~~~~~~{\isasymforall}x.~map\_bt~g~(map\_bt~f~(F~x))~=~map\_bt~(g~{\isasymcirc}~f)~(F~x)~{\isasymLongrightarrow}\isanewline
~~~~~~map\_bt~g~(map\_bt~f~(Branch~a~F))~=~map\_bt~(g~{\isasymcirc}~f)~(Branch~a~F)%
\end{isabellepar}%%
\end{isamarkuptext}%
\end{isabelle}%
