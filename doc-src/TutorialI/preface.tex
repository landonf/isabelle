\chapter*{Preface}
\markboth{Preface}{Preface}

This volume is a self-contained introduction to interactive proof
in higher-order logic (HOL), using the proof assistant Isabelle. 
It is written for potential users rather
than for our colleagues in the research world.

The book has three parts.  
\begin{itemize}
\item 
The first part, \textbf{Elementary Techniques},
shows how to model functional programs in higher-order logic.  Early
examples involve lists and the natural numbers.  Most proofs
are two steps long, consisting of induction on a chosen variable
followed by the \isa{auto} tactic.  But even this elementary part
covers such advanced topics as nested and mutual recursion.
\item 
The second part, \textbf{Logic and Sets}, presents a collection of
lower-level tactics that you can use to apply rules selectively.  It
also describes Isabelle/HOL's treatment of sets, functions and
relations and explains how to define sets inductively.  One of the
examples concerns the theory of model checking, and another is drawn
from a classic textbook on formal languages.
\item 
The third part, \textbf{Advanced Material}, describes a variety of other
topics.  Among these are the real numbers, records and overloading.  Advanced
techniques for induction and recursion are described.  A whole chapter is
devoted to an extended example: the verification of a security protocol.
\end{itemize}

The typesetting relies on Wenzel's theory presentation tools.  An
annotated source file is run, typesetting the theory
in the form of a \LaTeX\ source file.  This book is derived almost entirely
from output generated in this way.  The final chapter of Part~I explains how
users may produce their own formal documents in a similar fashion.

Isabelle's \hfootref{http://isabelle.in.tum.de/}{web site} contains links to
the download area and to documentation and other information.  Most Isabelle
sessions are now run from within David Aspinall's\index{Aspinall, David}
wonderful user interface, \hfootref{http://proofgeneral.inf.ed.ac.uk/}{Proof
  General}, even together with the
\hfootref{http://x-symbol.sourceforge.net}{X-Symbol} package for XEmacs.  This
book says very little about Proof General, which has its own documentation.
In order to run Isabelle, you will need a Standard ML compiler.  We recommend
\hfootref{http://www.polyml.org/}{Poly/ML}, which is free and gives the best
performance.  The other fully supported compiler is
\hfootref{http://www.smlnj.org/index.html}{Standard ML of New Jersey}.

This tutorial owes a lot to the constant discussions with and the valuable
feedback from the Isabelle group at Munich: Stefan Berghofer, Olaf
M{\"u}ller, Wolfgang Naraschewski, David von Oheimb, Leonor Prensa Nieto,
Cornelia Pusch, Norbert Schirmer and Martin Strecker. Stephan
Merz was also kind enough to read and comment on a draft version.  We
received comments from Stefano Bistarelli, Gergely Buday, John Matthews
and Tanja Vos.

The research has been funded by many sources, including the {\sc dfg} grants
NI~491/2, NI~491/3, NI~491/4, NI~491/6, {\sc bmbf} project Verisoft, the {\sc
epsrc} grants GR/K57381, GR/K77051, GR/M75440, GR/R01156/01 GR/S57198/01 and
by the \textsc{esprit} working groups 21900 and IST-1999-29001 (the
\emph{Types} project).
