\chapter{Functional Programming in HOL}

This chapter describes how to write
functional programs in HOL and how to verify them.  However, 
most of the constructs and
proof procedures introduced are general and recur in any specification
or verification task.  We really should speak of functional
\emph{modelling} rather than functional \emph{programming}: 
our primary aim is not
to write programs but to design abstract models of systems.  HOL is
a specification language that goes well beyond what can be expressed as a
program. However, for the time being we concentrate on the computable.

If you are a purist functional programmer, please note that all functions
in HOL must be total:
they must terminate for all inputs.  Lazy data structures are not
directly available.

\section{An Introductory Theory}
\label{sec:intro-theory}

Functional programming needs datatypes and functions. Both of them can be
defined in a theory with a syntax reminiscent of languages like ML or
Haskell. As an example consider the theory in figure~\ref{fig:ToyList}.
We will now examine it line by line.

\begin{figure}[htbp]
\begin{ttbox}\makeatother
\input{ToyList2/ToyList1}\end{ttbox}
\caption{A Theory of Lists}
\label{fig:ToyList}
\end{figure}

\index{*ToyList example|(}
{\makeatother\medskip\begin{isabelle}%
\isacommand{theory}\ ToyList\ =\ PreList:%
\begin{isamarkuptext}%
\noindent
HOL already has a predefined theory of lists called \isa{List} ---
\isa{ToyList} is merely a small fragment of it chosen as an example. In
contrast to what is recommended in \S\ref{sec:Basic:Theories},
\isa{ToyList} is not based on \isa{Main} but on \isa{PreList}, a
theory that contains pretty much everything but lists, thus avoiding
ambiguities caused by defining lists twice.%
\end{isamarkuptext}%
\isacommand{datatype}\ 'a\ list\ =\ Nil\ \ \ \ \ \ \ \ \ \ \ \ \ \ \ \ \ \ \ \ \ \ \ \ \ \ ({"}[]{"})\isanewline
\ \ \ \ \ \ \ \ \ \ \ \ \ \ \ \ \ |\ Cons\ 'a\ {"}'a\ list{"}\ \ \ \ \ \ \ \ \ \ \ \ (\isakeyword{infixr}\ {"}\#{"}\ 65)%
\begin{isamarkuptext}%
\noindent
The datatype\index{*datatype} \isaindexbold{list} introduces two
constructors \isaindexbold{Nil} and \isaindexbold{Cons}, the
empty~list and the operator that adds an element to the front of a list. For
example, the term \isa{Cons True (Cons False Nil)} is a value of type
\isa{bool\ list}, namely the list with the elements \isa{True} and
\isa{False}. Because this notation becomes unwieldy very quickly, the
datatype declaration is annotated with an alternative syntax: instead of
\isa{Nil} and \isa{Cons x xs} we can write
\isa{[]}\index{$HOL2list@\texttt{[]}|bold} and
\isa{\mbox{x}\ \#\ \mbox{xs}}\index{$HOL2list@\texttt{\#}|bold}. In fact, this
alternative syntax is the standard syntax. Thus the list \isa{Cons True
(Cons False Nil)} becomes \isa{True\ \#\ False\ \#\ []}. The annotation
\isacommand{infixr}\indexbold{*infixr} means that \isa{\#} associates to
the right, i.e.\ the term \isa{\mbox{x}\ \#\ \mbox{y}\ \#\ \mbox{z}} is read as \isa{x
\# (y \# z)} and not as \isa{(x \# y) \# z}.

\begin{warn}
  Syntax annotations are a powerful but completely optional feature. You
  could drop them from theory \isa{ToyList} and go back to the identifiers
  \isa{Nil} and \isa{Cons}. However, lists are such a central datatype
  that their syntax is highly customized. We recommend that novices should
  not use syntax annotations in their own theories.
\end{warn}
Next, two functions \isa{app} and \isaindexbold{rev} are declared:%
\end{isamarkuptext}%
\isacommand{consts}\ app\ ::\ {"}'a\ list\ {\isasymRightarrow}\ 'a\ list\ {\isasymRightarrow}\ 'a\ list{"}\ \ \ (\isakeyword{infixr}\ {"}@{"}\ 65)\isanewline
\ \ \ \ \ \ \ rev\ ::\ {"}'a\ list\ {\isasymRightarrow}\ 'a\ list{"}%
\begin{isamarkuptext}%
\noindent
In contrast to ML, Isabelle insists on explicit declarations of all functions
(keyword \isacommand{consts}).  (Apart from the declaration-before-use
restriction, the order of items in a theory file is unconstrained.) Function
\isa{app} is annotated with concrete syntax too. Instead of the prefix
syntax \isa{app xs ys} the infix
\isa{\mbox{xs}\ @\ \mbox{ys}}\index{$HOL2list@\texttt{\at}|bold} becomes the preferred
form. Both functions are defined recursively:%
\end{isamarkuptext}%
\isacommand{primrec}\isanewline
{"}[]\ @\ ys\ \ \ \ \ \ \ =\ ys{"}\isanewline
{"}(x\ \#\ xs)\ @\ ys\ =\ x\ \#\ (xs\ @\ ys){"}\isanewline
\isanewline
\isacommand{primrec}\isanewline
{"}rev\ []\ \ \ \ \ \ \ \ =\ []{"}\isanewline
{"}rev\ (x\ \#\ xs)\ \ =\ (rev\ xs)\ @\ (x\ \#\ []){"}%
\begin{isamarkuptext}%
\noindent
The equations for \isa{app} and \isa{rev} hardly need comments:
\isa{app} appends two lists and \isa{rev} reverses a list.  The keyword
\isacommand{primrec}\index{*primrec} indicates that the recursion is of a
particularly primitive kind where each recursive call peels off a datatype
constructor from one of the arguments.  Thus the
recursion always terminates, i.e.\ the function is \bfindex{total}.

The termination requirement is absolutely essential in HOL, a logic of total
functions. If we were to drop it, inconsistencies would quickly arise: the
``definition'' $f(n) = f(n)+1$ immediately leads to $0 = 1$ by subtracting
$f(n)$ on both sides.
% However, this is a subtle issue that we cannot discuss here further.

\begin{warn}
  As we have indicated, the desire for total functions is not a gratuitously
  imposed restriction but an essential characteristic of HOL. It is only
  because of totality that reasoning in HOL is comparatively easy.  More
  generally, the philosophy in HOL is not to allow arbitrary axioms (such as
  function definitions whose totality has not been proved) because they
  quickly lead to inconsistencies. Instead, fixed constructs for introducing
  types and functions are offered (such as \isacommand{datatype} and
  \isacommand{primrec}) which are guaranteed to preserve consistency.
\end{warn}

A remark about syntax.  The textual definition of a theory follows a fixed
syntax with keywords like \isacommand{datatype} and \isacommand{end} (see
Fig.~\ref{fig:keywords} in Appendix~\ref{sec:Appendix} for a full list).
Embedded in this syntax are the types and formulae of HOL, whose syntax is
extensible, e.g.\ by new user-defined infix operators
(see~\ref{sec:infix-syntax}). To distinguish the two levels, everything
HOL-specific (terms and types) should be enclosed in
\texttt{"}\dots\texttt{"}. 
To lessen this burden, quotation marks around a single identifier can be
dropped, unless the identifier happens to be a keyword, as in%
\end{isamarkuptext}%
\isacommand{consts}\ {"}end{"}\ ::\ {"}'a\ list\ {\isasymRightarrow}\ 'a{"}%
\begin{isamarkuptext}%
\noindent
When Isabelle prints a syntax error message, it refers to the HOL syntax as
the \bfindex{inner syntax} and the enclosing theory language as the \bfindex{outer syntax}.


\section{An introductory proof}
\label{sec:intro-proof}

Assuming you have input the declarations and definitions of \texttt{ToyList}
presented so far, we are ready to prove a few simple theorems. This will
illustrate not just the basic proof commands but also the typical proof
process.

\subsubsection*{Main goal: \texttt{rev(rev xs) = xs}}

Our goal is to show that reversing a list twice produces the original
list. The input line%
\end{isamarkuptext}%
\isacommand{theorem}\ rev\_rev\ [simp]:\ {"}rev(rev\ xs)\ =\ xs{"}%
\begin{isamarkuptxt}%
\index{*theorem|bold}\index{*simp (attribute)|bold}
\begin{itemize}
\item
establishes a new theorem to be proved, namely \isa{rev(rev xs) = xs},
\item
gives that theorem the name \isa{rev_rev} by which it can be referred to,
\item
and tells Isabelle (via \isa{[simp]}) to use the theorem (once it has been
proved) as a simplification rule, i.e.\ all future proofs involving
simplification will replace occurrences of \isa{rev(rev xs)} by
\isa{xs}.

The name and the simplification attribute are optional.
\end{itemize}
Isabelle's response is to print
\begin{isabellepar}%
proof(prove):~step~0\isanewline
\isanewline
goal~(theorem~rev\_rev):\isanewline
rev~(rev~xs)~=~xs\isanewline
~1.~rev~(rev~xs)~=~xs
\end{isabellepar}%
The first three lines tell us that we are 0 steps into the proof of
theorem \isa{rev_rev}; for compactness reasons we rarely show these
initial lines in this tutorial. The remaining lines display the current
proof state.
Until we have finished a proof, the proof state always looks like this:
\begin{isabellepar}%
$G$\isanewline
~1.~$G\sb{1}$\isanewline
~~\vdots~~\isanewline
~$n$.~$G\sb{n}$
\end{isabellepar}%
where $G$
is the overall goal that we are trying to prove, and the numbered lines
contain the subgoals $G\sb{1}$, \dots, $G\sb{n}$ that we need to prove to
establish $G$. At \isa{step 0} there is only one subgoal, which is
identical with the overall goal.  Normally $G$ is constant and only serves as
a reminder. Hence we rarely show it in this tutorial.

Let us now get back to \isa{rev(rev xs) = xs}. Properties of recursively
defined functions are best established by induction. In this case there is
not much choice except to induct on \isa{xs}:%
\end{isamarkuptxt}%
\isacommand{apply}(induct\_tac\ xs)%
\begin{isamarkuptxt}%
\noindent\index{*induct_tac}%
This tells Isabelle to perform induction on variable \isa{xs}. The suffix
\isa{tac} stands for ``tactic'', a synonym for ``theorem proving function''.
By default, induction acts on the first subgoal. The new proof state contains
two subgoals, namely the base case (\isa{Nil}) and the induction step
(\isa{Cons}):
\begin{isabellepar}%
~1.~rev~(rev~[])~=~[]\isanewline
~2.~{\isasymAnd}a~list.~rev(rev~list)~=~list~{\isasymLongrightarrow}~rev(rev(a~\#~list))~=~a~\#~list%
\end{isabellepar}%

The induction step is an example of the general format of a subgoal:
\begin{isabellepar}%
~$i$.~{\indexboldpos{\isasymAnd}{$IsaAnd}}$x\sb{1}$~\dots~$x\sb{n}$.~{\it assumptions}~{\isasymLongrightarrow}~{\it conclusion}
\end{isabellepar}%
The prefix of bound variables \isasymAnd$x\sb{1}$~\dots~$x\sb{n}$ can be
ignored most of the time, or simply treated as a list of variables local to
this subgoal. Their deeper significance is explained in \S\ref{sec:PCproofs}.
The {\it assumptions} are the local assumptions for this subgoal and {\it
  conclusion} is the actual proposition to be proved. Typical proof steps
that add new assumptions are induction or case distinction. In our example
the only assumption is the induction hypothesis \isa{rev\ (rev\ \mbox{list})\ =\ \mbox{list}}, where \isa{list} is a variable name chosen by Isabelle. If there
are multiple assumptions, they are enclosed in the bracket pair
\indexboldpos{\isasymlbrakk}{$Isabrl} and
\indexboldpos{\isasymrbrakk}{$Isabrr} and separated by semicolons.

%FIXME indent!
Let us try to solve both goals automatically:%
\end{isamarkuptxt}%
\isacommand{apply}(auto)%
\begin{isamarkuptxt}%
\noindent
This command tells Isabelle to apply a proof strategy called
\isa{auto} to all subgoals. Essentially, \isa{auto} tries to
``simplify'' the subgoals.  In our case, subgoal~1 is solved completely (thanks
to the equation \isa{rev [] = []}) and disappears; the simplified version
of subgoal~2 becomes the new subgoal~1:
\begin{isabellepar}%
~1.~\dots~rev(rev~list)~=~list~{\isasymLongrightarrow}~rev(rev~list~@~a~\#~[])~=~a~\#~list
\end{isabellepar}%
In order to simplify this subgoal further, a lemma suggests itself.%
\end{isamarkuptxt}%
%
\begin{isamarkuptext}%
\subsubsection*{First lemma: \texttt{rev(xs \at~ys) = (rev ys) \at~(rev xs)}}

After abandoning the above proof attempt\indexbold{abandon
proof}\indexbold{proof!abandon} (at the shell level type
\isacommand{oops}\indexbold{*oops}) we start a new proof:%
\end{isamarkuptext}%
\isacommand{lemma}\ rev\_app\ [simp]:\ {"}rev(xs\ @\ ys)\ =\ (rev\ ys)\ @\ (rev\ xs){"}%
\begin{isamarkuptxt}%
\noindent The keywords \isacommand{theorem}\index{*theorem} and
\isacommand{lemma}\indexbold{*lemma} are interchangable and merely indicate
the importance we attach to a proposition. In general, we use the words
\emph{theorem}\index{theorem} and \emph{lemma}\index{lemma} pretty much
interchangeably.

There are two variables that we could induct on: \isa{xs} and
\isa{ys}. Because \isa{\at} is defined by recursion on
the first argument, \isa{xs} is the correct one:%
\end{isamarkuptxt}%
\isacommand{apply}(induct\_tac\ xs)%
\begin{isamarkuptxt}%
\noindent
This time not even the base case is solved automatically:%
\end{isamarkuptxt}%
\isacommand{apply}(auto)%
\begin{isamarkuptxt}%
\begin{isabellepar}%
~1.~rev~ys~=~rev~ys~@~[]\isanewline
~2. \dots
\end{isabellepar}%
Again, we need to abandon this proof attempt and prove another simple lemma first.
In the future the step of abandoning an incomplete proof before embarking on
the proof of a lemma usually remains implicit.%
\end{isamarkuptxt}%
%
\begin{isamarkuptext}%
\subsubsection*{Second lemma: \texttt{xs \at~[] = xs}}

This time the canonical proof procedure%
\end{isamarkuptext}%
\isacommand{lemma}\ app\_Nil2\ [simp]:\ {"}xs\ @\ []\ =\ xs{"}\isanewline
\isacommand{apply}(induct\_tac\ xs)\isanewline
\isacommand{apply}(auto)%
\begin{isamarkuptxt}%
\noindent
leads to the desired message \isa{No subgoals!}:
\begin{isabellepar}%
xs~@~[]~=~xs\isanewline
No~subgoals!
\end{isabellepar}%

We still need to confirm that the proof is now finished:%
\end{isamarkuptxt}%
\isacommand{.}%
\begin{isamarkuptext}%
\noindent\indexbold{$Isar@\texttt{.}}%
As a result of that final dot, Isabelle associates the lemma
just proved with its name. Notice that in the lemma \isa{app_Nil2} (as
printed out after the final dot) the free variable \isa{xs} has been
replaced by the unknown \isa{?xs}, just as explained in
\S\ref{sec:variables}. Note that instead of instead of \isacommand{apply}
followed by a dot, you can simply write \isacommand{by}\indexbold{by},
which we do most of the time.

Going back to the proof of the first lemma%
\end{isamarkuptext}%
\isacommand{lemma}\ rev\_app\ [simp]:\ {"}rev(xs\ @\ ys)\ =\ (rev\ ys)\ @\ (rev\ xs){"}\isanewline
\isacommand{apply}(induct\_tac\ xs)\isanewline
\isacommand{apply}(auto)%
\begin{isamarkuptxt}%
\noindent
we find that this time \isa{auto} solves the base case, but the
induction step merely simplifies to
\begin{isabellepar}
~1.~{\isasymAnd}a~list.\isanewline
~~~~~~~rev~(list~@~ys)~=~rev~ys~@~rev~list~{\isasymLongrightarrow}\isanewline
~~~~~~~(rev~ys~@~rev~list)~@~a~\#~[]~=~rev~ys~@~rev~list~@~a~\#~[]
\end{isabellepar}%
Now we need to remember that \isa{\at} associates to the right, and that
\isa{\#} and \isa{\at} have the same priority (namely the \isa{65}
in their \isacommand{infixr} annotation). Thus the conclusion really is
\begin{isabellepar}%
~~~~~(rev~ys~@~rev~list)~@~(a~\#~[])~=~rev~ys~@~(rev~list~@~(a~\#~[]))%
\end{isabellepar}%
and the missing lemma is associativity of \isa{\at}.

\subsubsection*{Third lemma: \texttt{(xs \at~ys) \at~zs = xs \at~(ys \at~zs)}}

Abandoning the previous proof, the canonical proof procedure%
\end{isamarkuptxt}%
%
\begin{comment}
\isacommand{oops}%
\end{comment}
\isacommand{lemma}\ app\_assoc\ [simp]:\ {"}(xs\ @\ ys)\ @\ zs\ =\ xs\ @\ (ys\ @\ zs){"}\isanewline
\isacommand{apply}(induct\_tac\ xs)\isanewline
\isacommand{by}(auto)%
\begin{isamarkuptext}%
\noindent
succeeds without further ado.

Now we can go back and prove the first lemma%
\end{isamarkuptext}%
\isacommand{lemma}\ rev\_app\ [simp]:\ {"}rev(xs\ @\ ys)\ =\ (rev\ ys)\ @\ (rev\ xs){"}\isanewline
\isacommand{apply}(induct\_tac\ xs)\isanewline
\isacommand{by}(auto)%
\begin{isamarkuptext}%
\noindent
and then solve our main theorem:%
\end{isamarkuptext}%
\isacommand{theorem}\ rev\_rev\ [simp]:\ {"}rev(rev\ xs)\ =\ xs{"}\isanewline
\isacommand{apply}(induct\_tac\ xs)\isanewline
\isacommand{by}(auto)%
\begin{isamarkuptext}%
\noindent
The final \isa{end} tells Isabelle to close the current theory because
we are finished with its development:%
\end{isamarkuptext}%
\isacommand{end}\isanewline
\end{isabelle}%
%%% Local Variables:
%%% mode: latex
%%% TeX-master: "root"
%%% End:
}

The complete proof script is shown in Fig.\ts\ref{fig:ToyList-proofs}. The
concatenation of Figs.\ts\ref{fig:ToyList} and~\ref{fig:ToyList-proofs}
constitutes the complete theory \texttt{ToyList} and should reside in file
\texttt{ToyList.thy}.
% It is good practice to present all declarations and
%definitions at the beginning of a theory to facilitate browsing.%
\index{*ToyList example|)}

\begin{figure}[htbp]
\begin{ttbox}\makeatother
\input{ToyList2/ToyList2}\end{ttbox}
\caption{Proofs about Lists}
\label{fig:ToyList-proofs}
\end{figure}

\subsubsection*{Review}

This is the end of our toy proof. It should have familiarized you with
\begin{itemize}
\item the standard theorem proving procedure:
state a goal (lemma or theorem); proceed with proof until a separate lemma is
required; prove that lemma; come back to the original goal.
\item a specific procedure that works well for functional programs:
induction followed by all-out simplification via \isa{auto}.
\item a basic repertoire of proof commands.
\end{itemize}

\begin{warn}
It is tempting to think that all lemmas should have the \isa{simp} attribute
just because this was the case in the example above. However, in that example
all lemmas were equations, and the right-hand side was simpler than the
left-hand side --- an ideal situation for simplification purposes. Unless
this is clearly the case, novices should refrain from awarding a lemma the
\isa{simp} attribute, which has a global effect. Instead, lemmas can be
applied locally where they are needed, which is discussed in the following
chapter.
\end{warn}

\section{Some Helpful Commands}
\label{sec:commands-and-hints}

This section discusses a few basic commands for manipulating the proof state
and can be skipped by casual readers.

There are two kinds of commands used during a proof: the actual proof
commands and auxiliary commands for examining the proof state and controlling
the display. Simple proof commands are of the form
\commdx{apply}(\textit{method}), where \textit{method} is typically 
\isa{induct_tac} or \isa{auto}.  All such theorem proving operations
are referred to as \bfindex{methods}, and further ones are
introduced throughout the tutorial.  Unless stated otherwise, you may
assume that a method attacks merely the first subgoal. An exception is
\isa{auto}, which tries to solve all subgoals.

The most useful auxiliary commands are as follows:
\begin{description}
\item[Modifying the order of subgoals:]
\commdx{defer} moves the first subgoal to the end and
\commdx{prefer}~$n$ moves subgoal $n$ to the front.
\item[Printing theorems:]
  \commdx{thm}~\textit{name}$@1$~\dots~\textit{name}$@n$
  prints the named theorems.
\item[Reading terms and types:] \commdx{term}
  \textit{string} reads, type-checks and prints the given string as a term in
  the current context; the inferred type is output as well.
  \commdx{typ} \textit{string} reads and prints the given
  string as a type in the current context.
\end{description}
Further commands are found in the Isabelle/Isar Reference
Manual~\cite{isabelle-isar-ref}.

\begin{pgnote}
Clicking on the \pgmenu{State} button redisplays the current proof state.
This is helpful in case commands like \isacommand{thm} have overwritten it.
\end{pgnote}

We now examine Isabelle's functional programming constructs systematically,
starting with inductive datatypes.


\section{Datatypes}
\label{sec:datatype}

\index{datatypes|(}%
Inductive datatypes are part of almost every non-trivial application of HOL.
First we take another look at an important example, the datatype of
lists, before we turn to datatypes in general. The section closes with a
case study.


\subsection{Lists}

\index{*list (type)}%
Lists are one of the essential datatypes in computing.  We expect that you
are already familiar with their basic operations.
Theory \isa{ToyList} is only a small fragment of HOL's predefined theory
\thydx{List}\footnote{\url{http://isabelle.in.tum.de/library/HOL/List.html}}.
The latter contains many further operations. For example, the functions
\cdx{hd} (``head'') and \cdx{tl} (``tail'') return the first
element and the remainder of a list. (However, pattern matching is usually
preferable to \isa{hd} and \isa{tl}.)  
Also available are higher-order functions like \isa{map} and \isa{filter}.
Theory \isa{List} also contains
more syntactic sugar: \isa{[}$x@1$\isa{,}\dots\isa{,}$x@n$\isa{]} abbreviates
$x@1$\isa{\#}\dots\isa{\#}$x@n$\isa{\#[]}.  In the rest of the tutorial we
always use HOL's predefined lists by building on theory \isa{Main}.


\subsection{The General Format}
\label{sec:general-datatype}

The general HOL \isacommand{datatype} definition is of the form
\[
\isacommand{datatype}~(\alpha@1, \dots, \alpha@n) \, t ~=~
C@1~\tau@{11}~\dots~\tau@{1k@1} ~\mid~ \dots ~\mid~
C@m~\tau@{m1}~\dots~\tau@{mk@m}
\]
where $\alpha@i$ are distinct type variables (the parameters), $C@i$ are distinct
constructor names and $\tau@{ij}$ are types; it is customary to capitalize
the first letter in constructor names. There are a number of
restrictions (such as that the type should not be empty) detailed
elsewhere~\cite{isabelle-HOL}. Isabelle notifies you if you violate them.

Laws about datatypes, such as \isa{[] \isasymnoteq~x\#xs} and
\isa{(x\#xs = y\#ys) = (x=y \isasymand~xs=ys)}, are used automatically
during proofs by simplification.  The same is true for the equations in
primitive recursive function definitions.

Every\footnote{Except for advanced datatypes where the recursion involves
``\isasymRightarrow'' as in {\S}\ref{sec:nested-fun-datatype}.} datatype $t$
comes equipped with a \isa{size} function from $t$ into the natural numbers
(see~{\S}\ref{sec:nat} below). For lists, \isa{size} is just the length, i.e.\
\isa{size [] = 0} and \isa{size(x \# xs) = size xs + 1}.  In general,
\cdx{size} returns
\begin{itemize}
\item zero for all constructors that do not have an argument of type $t$,
\item one plus the sum of the sizes of all arguments of type~$t$,
for all other constructors.
\end{itemize}
Note that because
\isa{size} is defined on every datatype, it is overloaded; on lists
\isa{size} is also called \sdx{length}, which is not overloaded.
Isabelle will always show \isa{size} on lists as \isa{length}.


\subsection{Primitive Recursion}

\index{recursion!primitive}%
Functions on datatypes are usually defined by recursion. In fact, most of the
time they are defined by what is called \textbf{primitive recursion} over some
datatype $t$. This means that the recursion equations must be of the form
\[ f \, x@1 \, \dots \, (C \, y@1 \, \dots \, y@k)\, \dots \, x@n = r \]
such that $C$ is a constructor of $t$ and all recursive calls of
$f$ in $r$ are of the form $f \, \dots \, y@i \, \dots$ for some $i$. Thus
Isabelle immediately sees that $f$ terminates because one (fixed!) argument
becomes smaller with every recursive call. There must be at most one equation
for each constructor.  Their order is immaterial.
A more general method for defining total recursive functions is introduced in
{\S}\ref{sec:fun}.

\begin{exercise}\label{ex:Tree}
%
\begin{isabellebody}%
\def\isabellecontext{Tree}%
\isamarkupfalse%
%
\begin{isamarkuptext}%
\noindent
Define the datatype of \rmindex{binary trees}:%
\end{isamarkuptext}%
\isamarkuptrue%
\isacommand{datatype}\ {\isacharprime}a\ tree\ {\isacharequal}\ Tip\ {\isacharbar}\ Node\ {\isachardoublequote}{\isacharprime}a\ tree{\isachardoublequote}\ {\isacharprime}a\ {\isachardoublequote}{\isacharprime}a\ tree{\isachardoublequote}\isamarkupfalse%
\isamarkupfalse%
\isamarkupfalse%
%
\begin{isamarkuptext}%
\noindent
Define a function \isa{mirror} that mirrors a binary tree
by swapping subtrees recursively. Prove%
\end{isamarkuptext}%
\isamarkuptrue%
\isacommand{lemma}\ mirror{\isacharunderscore}mirror{\isacharcolon}\ {\isachardoublequote}mirror{\isacharparenleft}mirror\ t{\isacharparenright}\ {\isacharequal}\ t{\isachardoublequote}\isamarkupfalse%
\isamarkupfalse%
\isanewline
\isamarkupfalse%
\isamarkupfalse%
\isamarkupfalse%
%
\begin{isamarkuptext}%
\noindent
Define a function \isa{flatten} that flattens a tree into a list
by traversing it in infix order. Prove%
\end{isamarkuptext}%
\isamarkuptrue%
\isacommand{lemma}\ {\isachardoublequote}flatten{\isacharparenleft}mirror\ t{\isacharparenright}\ {\isacharequal}\ rev{\isacharparenleft}flatten\ t{\isacharparenright}{\isachardoublequote}\isamarkupfalse%
\isamarkupfalse%
\isanewline
\isamarkupfalse%
\isamarkupfalse%
\end{isabellebody}%
%%% Local Variables:
%%% mode: latex
%%% TeX-master: "root"
%%% End:
%
\end{exercise}

%
\begin{isabellebody}%
\def\isabellecontext{case{\isacharunderscore}exprs}%
%
\isadelimtheory
%
\endisadelimtheory
%
\isatagtheory
%
\endisatagtheory
{\isafoldtheory}%
%
\isadelimtheory
%
\endisadelimtheory
%
\isamarkupsubsection{Case Expressions%
}
\isamarkuptrue%
%
\begin{isamarkuptext}%
\label{sec:case-expressions}\index{*case expressions}%
HOL also features \isa{case}-expressions for analyzing
elements of a datatype. For example,
\begin{isabelle}%
\ \ \ \ \ case\ xs\ of\ {\isacharbrackleft}{\isacharbrackright}\ {\isasymRightarrow}\ {\isacharbrackleft}{\isacharbrackright}\ {\isacharbar}\ y\ {\isacharhash}\ ys\ {\isasymRightarrow}\ y%
\end{isabelle}
evaluates to \isa{{\isacharbrackleft}{\isacharbrackright}} if \isa{xs} is \isa{{\isacharbrackleft}{\isacharbrackright}} and to \isa{y} if 
\isa{xs} is \isa{y\ {\isacharhash}\ ys}. (Since the result in both branches must be of
the same type, it follows that \isa{y} is of type \isa{{\isacharprime}a\ list} and hence
that \isa{xs} is of type \isa{{\isacharprime}a\ list\ list}.)

In general, if $e$ is a term of the datatype $t$ defined in
\S\ref{sec:general-datatype} above, the corresponding
\isa{case}-expression analyzing $e$ is
\[
\begin{array}{rrcl}
\isa{case}~e~\isa{of} & C@1~x@ {11}~\dots~x@ {1k@1} & \To & e@1 \\
                           \vdots \\
                           \mid & C@m~x@ {m1}~\dots~x@ {mk@m} & \To & e@m
\end{array}
\]

\begin{warn}
\emph{All} constructors must be present, their order is fixed, and nested
patterns are not supported.  Violating these restrictions results in strange
error messages.
\end{warn}
\noindent
Nested patterns can be simulated by nested \isa{case}-expressions: instead
of
\begin{isabelle}%
\ \ \ \ \ case\ xs\ of\ {\isacharbrackleft}{\isacharbrackright}\ {\isacharequal}{\isachargreater}\ {\isacharbrackleft}{\isacharbrackright}\ {\isacharbar}\ {\isacharbrackleft}x{\isacharbrackright}\ {\isacharequal}{\isachargreater}\ x\ {\isacharbar}\ x\ {\isacharhash}\ {\isacharparenleft}y\ {\isacharhash}\ zs{\isacharparenright}\ {\isacharequal}{\isachargreater}\ y%
\end{isabelle}
write
\begin{isabelle}%
\ \ \ \ \ case\ xs\ of\ {\isacharbrackleft}{\isacharbrackright}\ {\isasymRightarrow}\ {\isacharbrackleft}{\isacharbrackright}\ {\isacharbar}\ {\isacharbrackleft}x{\isacharbrackright}\ {\isasymRightarrow}\ x\ {\isacharbar}\ x\ {\isacharhash}\ y\ {\isacharhash}\ zs\ {\isasymRightarrow}\ y%
\end{isabelle}

Note that \isa{case}-expressions may need to be enclosed in parentheses to
indicate their scope%
\end{isamarkuptext}%
\isamarkuptrue%
%
\isamarkupsubsection{Structural Induction and Case Distinction%
}
\isamarkuptrue%
%
\begin{isamarkuptext}%
\label{sec:struct-ind-case}
\index{case distinctions}\index{induction!structural}%
Induction is invoked by \methdx{induct_tac}, as we have seen above; 
it works for any datatype.  In some cases, induction is overkill and a case
distinction over all constructors of the datatype suffices.  This is performed
by \methdx{case_tac}.  Here is a trivial example:%
\end{isamarkuptext}%
\isamarkuptrue%
\isacommand{lemma}\isamarkupfalse%
\ {\isachardoublequoteopen}{\isacharparenleft}case\ xs\ of\ {\isacharbrackleft}{\isacharbrackright}\ {\isasymRightarrow}\ {\isacharbrackleft}{\isacharbrackright}\ {\isacharbar}\ y{\isacharhash}ys\ {\isasymRightarrow}\ xs{\isacharparenright}\ {\isacharequal}\ xs{\isachardoublequoteclose}\isanewline
%
\isadelimproof
%
\endisadelimproof
%
\isatagproof
\isacommand{apply}\isamarkupfalse%
{\isacharparenleft}case{\isacharunderscore}tac\ xs{\isacharparenright}%
\begin{isamarkuptxt}%
\noindent
results in the proof state
\begin{isabelle}%
\ {\isadigit{1}}{\isachardot}\ xs\ {\isacharequal}\ {\isacharbrackleft}{\isacharbrackright}\ {\isasymLongrightarrow}\ {\isacharparenleft}case\ xs\ of\ {\isacharbrackleft}{\isacharbrackright}\ {\isasymRightarrow}\ {\isacharbrackleft}{\isacharbrackright}\ {\isacharbar}\ y\ {\isacharhash}\ ys\ {\isasymRightarrow}\ xs{\isacharparenright}\ {\isacharequal}\ xs\isanewline
\ {\isadigit{2}}{\isachardot}\ {\isasymAnd}a\ list{\isachardot}\isanewline
\isaindent{\ {\isadigit{2}}{\isachardot}\ \ \ \ }xs\ {\isacharequal}\ a\ {\isacharhash}\ list\ {\isasymLongrightarrow}\ {\isacharparenleft}case\ xs\ of\ {\isacharbrackleft}{\isacharbrackright}\ {\isasymRightarrow}\ {\isacharbrackleft}{\isacharbrackright}\ {\isacharbar}\ y\ {\isacharhash}\ ys\ {\isasymRightarrow}\ xs{\isacharparenright}\ {\isacharequal}\ xs%
\end{isabelle}
which is solved automatically:%
\end{isamarkuptxt}%
\isamarkuptrue%
\isacommand{apply}\isamarkupfalse%
{\isacharparenleft}auto{\isacharparenright}%
\endisatagproof
{\isafoldproof}%
%
\isadelimproof
%
\endisadelimproof
%
\begin{isamarkuptext}%
Note that we do not need to give a lemma a name if we do not intend to refer
to it explicitly in the future.
Other basic laws about a datatype are applied automatically during
simplification, so no special methods are provided for them.

\begin{warn}
  Induction is only allowed on free (or \isasymAnd-bound) variables that
  should not occur among the assumptions of the subgoal; see
  \S\ref{sec:ind-var-in-prems} for details. Case distinction
  (\isa{case{\isacharunderscore}tac}) works for arbitrary terms, which need to be
  quoted if they are non-atomic. However, apart from \isa{{\isasymAnd}}-bound
  variables, the terms must not contain variables that are bound outside.
  For example, given the goal \isa{{\isasymforall}xs{\isachardot}\ xs\ {\isacharequal}\ {\isacharbrackleft}{\isacharbrackright}\ {\isasymor}\ {\isacharparenleft}{\isasymexists}y\ ys{\isachardot}\ xs\ {\isacharequal}\ y\ {\isacharhash}\ ys{\isacharparenright}},
  \isa{case{\isacharunderscore}tac\ xs} will not work as expected because Isabelle interprets
  the \isa{xs} as a new free variable distinct from the bound
  \isa{xs} in the goal.
\end{warn}%
\end{isamarkuptext}%
\isamarkuptrue%
%
\isadelimtheory
%
\endisadelimtheory
%
\isatagtheory
%
\endisatagtheory
{\isafoldtheory}%
%
\isadelimtheory
%
\endisadelimtheory
\end{isabellebody}%
%%% Local Variables:
%%% mode: latex
%%% TeX-master: "root"
%%% End:


%
\begin{isabellebody}%
\def\isabellecontext{Ifexpr}%
%
\isamarkupsubsection{Case study: boolean expressions%
}
%
\begin{isamarkuptext}%
\label{sec:boolex}
The aim of this case study is twofold: it shows how to model boolean
expressions and some algorithms for manipulating them, and it demonstrates
the constructs introduced above.%
\end{isamarkuptext}%
%
\isamarkupsubsubsection{How can we model boolean expressions?%
}
%
\begin{isamarkuptext}%
We want to represent boolean expressions built up from variables and
constants by negation and conjunction. The following datatype serves exactly
that purpose:%
\end{isamarkuptext}%
\isacommand{datatype}\ boolex\ {\isacharequal}\ Const\ bool\ {\isacharbar}\ Var\ nat\ {\isacharbar}\ Neg\ boolex\isanewline
\ \ \ \ \ \ \ \ \ \ \ \ \ \ \ \ {\isacharbar}\ And\ boolex\ boolex%
\begin{isamarkuptext}%
\noindent
The two constants are represented by \isa{Const\ True} and
\isa{Const\ False}. Variables are represented by terms of the form
\isa{Var\ n}, where \isa{n} is a natural number (type \isa{nat}).
For example, the formula $P@0 \land \neg P@1$ is represented by the term
\isa{And\ {\isacharparenleft}Var\ {\isadigit{0}}{\isacharparenright}\ {\isacharparenleft}Neg\ {\isacharparenleft}Var\ {\isadigit{1}}{\isacharparenright}{\isacharparenright}}.

\subsubsection{What is the value of a boolean expression?}

The value of a boolean expression depends on the value of its variables.
Hence the function \isa{value} takes an additional parameter, an
\emph{environment} of type \isa{nat\ {\isasymRightarrow}\ bool}, which maps variables to their
values:%
\end{isamarkuptext}%
\isacommand{consts}\ value\ {\isacharcolon}{\isacharcolon}\ {\isachardoublequote}boolex\ {\isasymRightarrow}\ {\isacharparenleft}nat\ {\isasymRightarrow}\ bool{\isacharparenright}\ {\isasymRightarrow}\ bool{\isachardoublequote}\isanewline
\isacommand{primrec}\isanewline
{\isachardoublequote}value\ {\isacharparenleft}Const\ b{\isacharparenright}\ env\ {\isacharequal}\ b{\isachardoublequote}\isanewline
{\isachardoublequote}value\ {\isacharparenleft}Var\ x{\isacharparenright}\ \ \ env\ {\isacharequal}\ env\ x{\isachardoublequote}\isanewline
{\isachardoublequote}value\ {\isacharparenleft}Neg\ b{\isacharparenright}\ \ \ env\ {\isacharequal}\ {\isacharparenleft}{\isasymnot}\ value\ b\ env{\isacharparenright}{\isachardoublequote}\isanewline
{\isachardoublequote}value\ {\isacharparenleft}And\ b\ c{\isacharparenright}\ env\ {\isacharequal}\ {\isacharparenleft}value\ b\ env\ {\isasymand}\ value\ c\ env{\isacharparenright}{\isachardoublequote}%
\begin{isamarkuptext}%
\noindent
\subsubsection{If-expressions}

An alternative and often more efficient (because in a certain sense
canonical) representation are so-called \emph{If-expressions} built up
from constants (\isa{CIF}), variables (\isa{VIF}) and conditionals
(\isa{IF}):%
\end{isamarkuptext}%
\isacommand{datatype}\ ifex\ {\isacharequal}\ CIF\ bool\ {\isacharbar}\ VIF\ nat\ {\isacharbar}\ IF\ ifex\ ifex\ ifex%
\begin{isamarkuptext}%
\noindent
The evaluation if If-expressions proceeds as for \isa{boolex}:%
\end{isamarkuptext}%
\isacommand{consts}\ valif\ {\isacharcolon}{\isacharcolon}\ {\isachardoublequote}ifex\ {\isasymRightarrow}\ {\isacharparenleft}nat\ {\isasymRightarrow}\ bool{\isacharparenright}\ {\isasymRightarrow}\ bool{\isachardoublequote}\isanewline
\isacommand{primrec}\isanewline
{\isachardoublequote}valif\ {\isacharparenleft}CIF\ b{\isacharparenright}\ \ \ \ env\ {\isacharequal}\ b{\isachardoublequote}\isanewline
{\isachardoublequote}valif\ {\isacharparenleft}VIF\ x{\isacharparenright}\ \ \ \ env\ {\isacharequal}\ env\ x{\isachardoublequote}\isanewline
{\isachardoublequote}valif\ {\isacharparenleft}IF\ b\ t\ e{\isacharparenright}\ env\ {\isacharequal}\ {\isacharparenleft}if\ valif\ b\ env\ then\ valif\ t\ env\isanewline
\ \ \ \ \ \ \ \ \ \ \ \ \ \ \ \ \ \ \ \ \ \ \ \ \ \ \ \ \ \ \ \ \ \ \ \ \ \ \ \ else\ valif\ e\ env{\isacharparenright}{\isachardoublequote}%
\begin{isamarkuptext}%
\subsubsection{Transformation into and of If-expressions}

The type \isa{boolex} is close to the customary representation of logical
formulae, whereas \isa{ifex} is designed for efficiency. It is easy to
translate from \isa{boolex} into \isa{ifex}:%
\end{isamarkuptext}%
\isacommand{consts}\ bool{\isadigit{2}}if\ {\isacharcolon}{\isacharcolon}\ {\isachardoublequote}boolex\ {\isasymRightarrow}\ ifex{\isachardoublequote}\isanewline
\isacommand{primrec}\isanewline
{\isachardoublequote}bool{\isadigit{2}}if\ {\isacharparenleft}Const\ b{\isacharparenright}\ {\isacharequal}\ CIF\ b{\isachardoublequote}\isanewline
{\isachardoublequote}bool{\isadigit{2}}if\ {\isacharparenleft}Var\ x{\isacharparenright}\ \ \ {\isacharequal}\ VIF\ x{\isachardoublequote}\isanewline
{\isachardoublequote}bool{\isadigit{2}}if\ {\isacharparenleft}Neg\ b{\isacharparenright}\ \ \ {\isacharequal}\ IF\ {\isacharparenleft}bool{\isadigit{2}}if\ b{\isacharparenright}\ {\isacharparenleft}CIF\ False{\isacharparenright}\ {\isacharparenleft}CIF\ True{\isacharparenright}{\isachardoublequote}\isanewline
{\isachardoublequote}bool{\isadigit{2}}if\ {\isacharparenleft}And\ b\ c{\isacharparenright}\ {\isacharequal}\ IF\ {\isacharparenleft}bool{\isadigit{2}}if\ b{\isacharparenright}\ {\isacharparenleft}bool{\isadigit{2}}if\ c{\isacharparenright}\ {\isacharparenleft}CIF\ False{\isacharparenright}{\isachardoublequote}%
\begin{isamarkuptext}%
\noindent
At last, we have something we can verify: that \isa{bool{\isadigit{2}}if} preserves the
value of its argument:%
\end{isamarkuptext}%
\isacommand{lemma}\ {\isachardoublequote}valif\ {\isacharparenleft}bool{\isadigit{2}}if\ b{\isacharparenright}\ env\ {\isacharequal}\ value\ b\ env{\isachardoublequote}%
\begin{isamarkuptxt}%
\noindent
The proof is canonical:%
\end{isamarkuptxt}%
\isacommand{apply}{\isacharparenleft}induct{\isacharunderscore}tac\ b{\isacharparenright}\isanewline
\isacommand{apply}{\isacharparenleft}auto{\isacharparenright}\isanewline
\isacommand{done}%
\begin{isamarkuptext}%
\noindent
In fact, all proofs in this case study look exactly like this. Hence we do
not show them below.

More interesting is the transformation of If-expressions into a normal form
where the first argument of \isa{IF} cannot be another \isa{IF} but
must be a constant or variable. Such a normal form can be computed by
repeatedly replacing a subterm of the form \isa{IF\ {\isacharparenleft}IF\ b\ x\ y{\isacharparenright}\ z\ u} by
\isa{IF\ b\ {\isacharparenleft}IF\ x\ z\ u{\isacharparenright}\ {\isacharparenleft}IF\ y\ z\ u{\isacharparenright}}, which has the same value. The following
primitive recursive functions perform this task:%
\end{isamarkuptext}%
\isacommand{consts}\ normif\ {\isacharcolon}{\isacharcolon}\ {\isachardoublequote}ifex\ {\isasymRightarrow}\ ifex\ {\isasymRightarrow}\ ifex\ {\isasymRightarrow}\ ifex{\isachardoublequote}\isanewline
\isacommand{primrec}\isanewline
{\isachardoublequote}normif\ {\isacharparenleft}CIF\ b{\isacharparenright}\ \ \ \ t\ e\ {\isacharequal}\ IF\ {\isacharparenleft}CIF\ b{\isacharparenright}\ t\ e{\isachardoublequote}\isanewline
{\isachardoublequote}normif\ {\isacharparenleft}VIF\ x{\isacharparenright}\ \ \ \ t\ e\ {\isacharequal}\ IF\ {\isacharparenleft}VIF\ x{\isacharparenright}\ t\ e{\isachardoublequote}\isanewline
{\isachardoublequote}normif\ {\isacharparenleft}IF\ b\ t\ e{\isacharparenright}\ u\ f\ {\isacharequal}\ normif\ b\ {\isacharparenleft}normif\ t\ u\ f{\isacharparenright}\ {\isacharparenleft}normif\ e\ u\ f{\isacharparenright}{\isachardoublequote}\isanewline
\isanewline
\isacommand{consts}\ norm\ {\isacharcolon}{\isacharcolon}\ {\isachardoublequote}ifex\ {\isasymRightarrow}\ ifex{\isachardoublequote}\isanewline
\isacommand{primrec}\isanewline
{\isachardoublequote}norm\ {\isacharparenleft}CIF\ b{\isacharparenright}\ \ \ \ {\isacharequal}\ CIF\ b{\isachardoublequote}\isanewline
{\isachardoublequote}norm\ {\isacharparenleft}VIF\ x{\isacharparenright}\ \ \ \ {\isacharequal}\ VIF\ x{\isachardoublequote}\isanewline
{\isachardoublequote}norm\ {\isacharparenleft}IF\ b\ t\ e{\isacharparenright}\ {\isacharequal}\ normif\ b\ {\isacharparenleft}norm\ t{\isacharparenright}\ {\isacharparenleft}norm\ e{\isacharparenright}{\isachardoublequote}%
\begin{isamarkuptext}%
\noindent
Their interplay is a bit tricky, and we leave it to the reader to develop an
intuitive understanding. Fortunately, Isabelle can help us to verify that the
transformation preserves the value of the expression:%
\end{isamarkuptext}%
\isacommand{theorem}\ {\isachardoublequote}valif\ {\isacharparenleft}norm\ b{\isacharparenright}\ env\ {\isacharequal}\ valif\ b\ env{\isachardoublequote}%
\begin{isamarkuptext}%
\noindent
The proof is canonical, provided we first show the following simplification
lemma (which also helps to understand what \isa{normif} does):%
\end{isamarkuptext}%
\isacommand{lemma}\ {\isacharbrackleft}simp{\isacharbrackright}{\isacharcolon}\isanewline
\ \ {\isachardoublequote}{\isasymforall}t\ e{\isachardot}\ valif\ {\isacharparenleft}normif\ b\ t\ e{\isacharparenright}\ env\ {\isacharequal}\ valif\ {\isacharparenleft}IF\ b\ t\ e{\isacharparenright}\ env{\isachardoublequote}%
\begin{isamarkuptext}%
\noindent
Note that the lemma does not have a name, but is implicitly used in the proof
of the theorem shown above because of the \isa{{\isacharbrackleft}simp{\isacharbrackright}} attribute.

But how can we be sure that \isa{norm} really produces a normal form in
the above sense? We define a function that tests If-expressions for normality%
\end{isamarkuptext}%
\isacommand{consts}\ normal\ {\isacharcolon}{\isacharcolon}\ {\isachardoublequote}ifex\ {\isasymRightarrow}\ bool{\isachardoublequote}\isanewline
\isacommand{primrec}\isanewline
{\isachardoublequote}normal{\isacharparenleft}CIF\ b{\isacharparenright}\ {\isacharequal}\ True{\isachardoublequote}\isanewline
{\isachardoublequote}normal{\isacharparenleft}VIF\ x{\isacharparenright}\ {\isacharequal}\ True{\isachardoublequote}\isanewline
{\isachardoublequote}normal{\isacharparenleft}IF\ b\ t\ e{\isacharparenright}\ {\isacharequal}\ {\isacharparenleft}normal\ t\ {\isasymand}\ normal\ e\ {\isasymand}\isanewline
\ \ \ \ \ {\isacharparenleft}case\ b\ of\ CIF\ b\ {\isasymRightarrow}\ True\ {\isacharbar}\ VIF\ x\ {\isasymRightarrow}\ True\ {\isacharbar}\ IF\ x\ y\ z\ {\isasymRightarrow}\ False{\isacharparenright}{\isacharparenright}{\isachardoublequote}%
\begin{isamarkuptext}%
\noindent
and prove \isa{normal\ {\isacharparenleft}norm\ b{\isacharparenright}}. Of course, this requires a lemma about
normality of \isa{normif}:%
\end{isamarkuptext}%
\isacommand{lemma}{\isacharbrackleft}simp{\isacharbrackright}{\isacharcolon}\ {\isachardoublequote}{\isasymforall}t\ e{\isachardot}\ normal{\isacharparenleft}normif\ b\ t\ e{\isacharparenright}\ {\isacharequal}\ {\isacharparenleft}normal\ t\ {\isasymand}\ normal\ e{\isacharparenright}{\isachardoublequote}%
\begin{isamarkuptext}%
\medskip
How does one come up with the required lemmas? Try to prove the main theorems
without them and study carefully what \isa{auto} leaves unproved. This has
to provide the clue.  The necessity of universal quantification
(\isa{{\isasymforall}t\ e}) in the two lemmas is explained in
\S\ref{sec:InductionHeuristics}

\begin{exercise}
  We strengthen the definition of a \isa{normal} If-expression as follows:
  the first argument of all \isa{IF}s must be a variable. Adapt the above
  development to this changed requirement. (Hint: you may need to formulate
  some of the goals as implications (\isa{{\isasymlongrightarrow}}) rather than
  equalities (\isa{{\isacharequal}}).)
\end{exercise}%
\end{isamarkuptext}%
\end{isabellebody}%
%%% Local Variables:
%%% mode: latex
%%% TeX-master: "root"
%%% End:

\index{datatypes|)}


\section{Some Basic Types}

This section introduces the types of natural numbers and ordered pairs.  Also
described is type \isa{option}, which is useful for modelling exceptional
cases. 

\subsection{Natural Numbers}
\label{sec:nat}\index{natural numbers}%
\index{linear arithmetic|(}

\begin{isabelle}%
%
\begin{isamarkuptext}%
\noindent
The type \isaindexbold{nat}\index{*0|bold}\index{*Suc|bold} of natural
numbers is predefined and behaves like%
\end{isamarkuptext}%
\isacommand{datatype}~nat~=~{"}0{"}~|~Suc~nat\end{isabelle}%
%%% Local Variables:
%%% mode: latex
%%% TeX-master: "root"
%%% End:
\medskip
%
\begin{isabellebody}%
\def\isabellecontext{natsum}%
%
\isadelimtheory
%
\endisadelimtheory
%
\isatagtheory
\isamarkupfalse%
%
\endisatagtheory
{\isafoldtheory}%
%
\isadelimtheory
%
\endisadelimtheory
%
\begin{isamarkuptext}%
\noindent
In particular, there are \isa{case}-expressions, for example
\begin{isabelle}%
\ \ \ \ \ case\ n\ of\ {\isadigit{0}}\ {\isasymRightarrow}\ {\isadigit{0}}\ {\isacharbar}\ Suc\ m\ {\isasymRightarrow}\ m%
\end{isabelle}
primitive recursion, for example%
\end{isamarkuptext}%
\isamarkuptrue%
\isacommand{consts}\isamarkupfalse%
\ sum\ {\isacharcolon}{\isacharcolon}\ {\isachardoublequoteopen}nat\ {\isasymRightarrow}\ nat{\isachardoublequoteclose}\isanewline
\isacommand{primrec}\isamarkupfalse%
\ {\isachardoublequoteopen}sum\ {\isadigit{0}}\ {\isacharequal}\ {\isadigit{0}}{\isachardoublequoteclose}\isanewline
\ \ \ \ \ \ \ \ {\isachardoublequoteopen}sum\ {\isacharparenleft}Suc\ n{\isacharparenright}\ {\isacharequal}\ Suc\ n\ {\isacharplus}\ sum\ n{\isachardoublequoteclose}%
\begin{isamarkuptext}%
\noindent
and induction, for example%
\end{isamarkuptext}%
\isamarkuptrue%
\isacommand{lemma}\isamarkupfalse%
\ {\isachardoublequoteopen}sum\ n\ {\isacharplus}\ sum\ n\ {\isacharequal}\ n{\isacharasterisk}{\isacharparenleft}Suc\ n{\isacharparenright}{\isachardoublequoteclose}\isanewline
%
\isadelimproof
%
\endisadelimproof
%
\isatagproof
\isacommand{apply}\isamarkupfalse%
{\isacharparenleft}induct{\isacharunderscore}tac\ n{\isacharparenright}\isanewline
\isacommand{apply}\isamarkupfalse%
{\isacharparenleft}auto{\isacharparenright}\isanewline
\isacommand{done}\isamarkupfalse%
%
\endisatagproof
{\isafoldproof}%
%
\isadelimproof
%
\endisadelimproof
%
\begin{isamarkuptext}%
\newcommand{\mystar}{*%
}
\index{arithmetic operations!for \protect\isa{nat}}%
The arithmetic operations \isadxboldpos{+}{$HOL2arithfun},
\isadxboldpos{-}{$HOL2arithfun}, \isadxboldpos{\mystar}{$HOL2arithfun},
\sdx{div}, \sdx{mod}, \cdx{min} and
\cdx{max} are predefined, as are the relations
\isadxboldpos{\isasymle}{$HOL2arithrel} and
\isadxboldpos{<}{$HOL2arithrel}. As usual, \isa{m\ {\isacharminus}\ n\ {\isacharequal}\ {\isadigit{0}}} if
\isa{m\ {\isacharless}\ n}. There is even a least number operation
\sdx{LEAST}\@.  For example, \isa{{\isacharparenleft}LEAST\ n{\isachardot}\ {\isadigit{0}}\ {\isacharless}\ n{\isacharparenright}\ {\isacharequal}\ Suc\ {\isadigit{0}}}.
\begin{warn}\index{overloading}
  The constants \cdx{0} and \cdx{1} and the operations
  \isadxboldpos{+}{$HOL2arithfun}, \isadxboldpos{-}{$HOL2arithfun},
  \isadxboldpos{\mystar}{$HOL2arithfun}, \cdx{min},
  \cdx{max}, \isadxboldpos{\isasymle}{$HOL2arithrel} and
  \isadxboldpos{<}{$HOL2arithrel} are overloaded: they are available
  not just for natural numbers but for other types as well.
  For example, given the goal \isa{x\ {\isacharplus}\ {\isadigit{0}}\ {\isacharequal}\ x}, there is nothing to indicate
  that you are talking about natural numbers. Hence Isabelle can only infer
  that \isa{x} is of some arbitrary type where \isa{{\isadigit{0}}} and \isa{{\isacharplus}} are
  declared. As a consequence, you will be unable to prove the
  goal. To alert you to such pitfalls, Isabelle flags numerals without a
  fixed type in its output: \isa{x\ {\isacharplus}\ {\isacharparenleft}{\isadigit{0}}{\isasymColon}{\isacharprime}a{\isacharparenright}\ {\isacharequal}\ x}. (In the absence of a numeral,
  it may take you some time to realize what has happened if \pgmenu{Show
  Types} is not set).  In this particular example, you need to include
  an explicit type constraint, for example \isa{x{\isacharplus}{\isadigit{0}}\ {\isacharequal}\ {\isacharparenleft}x{\isacharcolon}{\isacharcolon}nat{\isacharparenright}}. If there
  is enough contextual information this may not be necessary: \isa{Suc\ x\ {\isacharequal}\ x} automatically implies \isa{x{\isacharcolon}{\isacharcolon}nat} because \isa{Suc} is not
  overloaded.

  For details on overloading see \S\ref{sec:overloading}.
  Table~\ref{tab:overloading} in the appendix shows the most important
  overloaded operations.
\end{warn}
\begin{warn}
  The symbols \isadxboldpos{>}{$HOL2arithrel} and
  \isadxboldpos{\isasymge}{$HOL2arithrel} are merely syntax: \isa{x\ {\isachargreater}\ y}
  stands for \isa{y\ {\isacharless}\ x} and similary for \isa{{\isasymge}} and
  \isa{{\isasymle}}.
\end{warn}
\begin{warn}
  Constant \isa{{\isadigit{1}}{\isacharcolon}{\isacharcolon}nat} is defined to equal \isa{Suc\ {\isadigit{0}}}. This definition
  (see \S\ref{sec:ConstDefinitions}) is unfolded automatically by some
  tactics (like \isa{auto}, \isa{simp} and \isa{arith}) but not by
  others (especially the single step tactics in Chapter~\ref{chap:rules}).
  If you need the full set of numerals, see~\S\ref{sec:numerals}.
  \emph{Novices are advised to stick to \isa{{\isadigit{0}}} and \isa{Suc}.}
\end{warn}

Both \isa{auto} and \isa{simp}
(a method introduced below, \S\ref{sec:Simplification}) prove 
simple arithmetic goals automatically:%
\end{isamarkuptext}%
\isamarkuptrue%
\isacommand{lemma}\isamarkupfalse%
\ {\isachardoublequoteopen}{\isasymlbrakk}\ {\isasymnot}\ m\ {\isacharless}\ n{\isacharsemicolon}\ m\ {\isacharless}\ n\ {\isacharplus}\ {\isacharparenleft}{\isadigit{1}}{\isacharcolon}{\isacharcolon}nat{\isacharparenright}\ {\isasymrbrakk}\ {\isasymLongrightarrow}\ m\ {\isacharequal}\ n{\isachardoublequoteclose}%
\isadelimproof
%
\endisadelimproof
%
\isatagproof
\isamarkupfalse%
%
\endisatagproof
{\isafoldproof}%
%
\isadelimproof
%
\endisadelimproof
%
\begin{isamarkuptext}%
\noindent
For efficiency's sake, this built-in prover ignores quantified formulae,
many logical connectives, and all arithmetic operations apart from addition.
In consequence, \isa{auto} and \isa{simp} cannot prove this slightly more complex goal:%
\end{isamarkuptext}%
\isamarkuptrue%
\isacommand{lemma}\isamarkupfalse%
\ {\isachardoublequoteopen}m\ {\isasymnoteq}\ {\isacharparenleft}n{\isacharcolon}{\isacharcolon}nat{\isacharparenright}\ {\isasymLongrightarrow}\ m\ {\isacharless}\ n\ {\isasymor}\ n\ {\isacharless}\ m{\isachardoublequoteclose}%
\isadelimproof
%
\endisadelimproof
%
\isatagproof
\isamarkupfalse%
%
\endisatagproof
{\isafoldproof}%
%
\isadelimproof
%
\endisadelimproof
%
\begin{isamarkuptext}%
\noindent The method \methdx{arith} is more general.  It attempts to
prove the first subgoal provided it is a \textbf{linear arithmetic} formula.
Such formulas may involve the usual logical connectives (\isa{{\isasymnot}},
\isa{{\isasymand}}, \isa{{\isasymor}}, \isa{{\isasymlongrightarrow}}, \isa{{\isacharequal}},
\isa{{\isasymforall}}, \isa{{\isasymexists}}), the relations \isa{{\isacharequal}},
\isa{{\isasymle}} and \isa{{\isacharless}}, and the operations \isa{{\isacharplus}}, \isa{{\isacharminus}},
\isa{min} and \isa{max}.  For example,%
\end{isamarkuptext}%
\isamarkuptrue%
\isacommand{lemma}\isamarkupfalse%
\ {\isachardoublequoteopen}min\ i\ {\isacharparenleft}max\ j\ {\isacharparenleft}k{\isacharasterisk}k{\isacharparenright}{\isacharparenright}\ {\isacharequal}\ max\ {\isacharparenleft}min\ {\isacharparenleft}k{\isacharasterisk}k{\isacharparenright}\ i{\isacharparenright}\ {\isacharparenleft}min\ i\ {\isacharparenleft}j{\isacharcolon}{\isacharcolon}nat{\isacharparenright}{\isacharparenright}{\isachardoublequoteclose}\isanewline
%
\isadelimproof
%
\endisadelimproof
%
\isatagproof
\isacommand{apply}\isamarkupfalse%
{\isacharparenleft}arith{\isacharparenright}\isamarkupfalse%
%
\endisatagproof
{\isafoldproof}%
%
\isadelimproof
%
\endisadelimproof
%
\begin{isamarkuptext}%
\noindent
succeeds because \isa{k\ {\isacharasterisk}\ k} can be treated as atomic. In contrast,%
\end{isamarkuptext}%
\isamarkuptrue%
\isacommand{lemma}\isamarkupfalse%
\ {\isachardoublequoteopen}n{\isacharasterisk}n\ {\isacharequal}\ n\ {\isasymLongrightarrow}\ n{\isacharequal}{\isadigit{0}}\ {\isasymor}\ n{\isacharequal}{\isadigit{1}}{\isachardoublequoteclose}%
\isadelimproof
%
\endisadelimproof
%
\isatagproof
\isamarkupfalse%
%
\endisatagproof
{\isafoldproof}%
%
\isadelimproof
%
\endisadelimproof
%
\begin{isamarkuptext}%
\noindent
is not proved even by \isa{arith} because the proof relies 
on properties of multiplication. Only multiplication by numerals (which is
the same as iterated addition) is allowed.

\begin{warn} The running time of \isa{arith} is exponential in the number
  of occurrences of \ttindexboldpos{-}{$HOL2arithfun}, \cdx{min} and
  \cdx{max} because they are first eliminated by case distinctions.

If \isa{k} is a numeral, \sdx{div}~\isa{k}, \sdx{mod}~\isa{k} and
\isa{k}~\sdx{dvd} are also supported, where the former two are eliminated
by case distinctions, again blowing up the running time.

If the formula involves quantifiers, \isa{arith} may take
super-exponential time and space.
\end{warn}%
\end{isamarkuptext}%
\isamarkuptrue%
%
\isadelimtheory
%
\endisadelimtheory
%
\isatagtheory
\isamarkupfalse%
%
\endisatagtheory
{\isafoldtheory}%
%
\isadelimtheory
%
\endisadelimtheory
\end{isabellebody}%
%%% Local Variables:
%%% mode: latex
%%% TeX-master: "root"
%%% End:


\index{linear arithmetic|)}


\subsection{Pairs}
%
\begin{isabellebody}%
\def\isabellecontext{pairs}%
%
\begin{isamarkuptext}%
\label{sec:pairs}\index{pairs and tuples}
HOL also has ordered pairs: \isa{($a@1$,$a@2$)} is of type $\tau@1$
\indexboldpos{\isasymtimes}{$Isatype} $\tau@2$ provided each $a@i$ is of type
$\tau@i$. The functions \cdx{fst} and
\cdx{snd} extract the components of a pair:
 \isa{fst($x$,$y$) = $x$} and \isa{snd($x$,$y$) = $y$}. Tuples
are simulated by pairs nested to the right: \isa{($a@1$,$a@2$,$a@3$)} stands
for \isa{($a@1$,($a@2$,$a@3$))} and $\tau@1 \times \tau@2 \times \tau@3$ for
$\tau@1 \times (\tau@2 \times \tau@3)$. Therefore we have
\isa{fst(snd($a@1$,$a@2$,$a@3$)) = $a@2$}.

Remarks:
\begin{itemize}
\item
There is also the type \tydx{unit}, which contains exactly one
element denoted by \ttindexboldpos{()}{$Isatype}. This type can be viewed
as a degenerate product with 0 components.
\item
Products, like type \isa{nat}, are datatypes, which means
in particular that \isa{induct{\isacharunderscore}tac} and \isa{case{\isacharunderscore}tac} are applicable to
terms of product type.
Both replace the term by a pair of variables.
\item
Tuples with more than two or three components become unwieldy;
records are preferable.
\end{itemize}
For more information on pairs and records see Chapter~\ref{ch:more-types}.%
\end{isamarkuptext}%
\end{isabellebody}%
%%% Local Variables:
%%% mode: latex
%%% TeX-master: "root"
%%% End:


\subsection{Datatype {\tt\slshape option}}
\label{sec:option}
%
\begin{isabellebody}%
\def\isabellecontext{Option{\isadigit{2}}}%
%
\isadelimtheory
%
\endisadelimtheory
%
\isatagtheory
%
\endisatagtheory
{\isafoldtheory}%
%
\isadelimtheory
%
\endisadelimtheory
%
\begin{isamarkuptext}%
\indexbold{*option (type)}\indexbold{*None (constant)}%
\indexbold{*Some (constant)}
Our final datatype is very simple but still eminently useful:%
\end{isamarkuptext}%
\isamarkuptrue%
\isacommand{datatype}\isamarkupfalse%
\ {\isaliteral{27}{\isacharprime}}a\ option\ {\isaliteral{3D}{\isacharequal}}\ None\ {\isaliteral{7C}{\isacharbar}}\ Some\ {\isaliteral{27}{\isacharprime}}a%
\begin{isamarkuptext}%
\noindent
Frequently one needs to add a distinguished element to some existing type.
For example, type \isa{t\ option} can model the result of a computation that
may either terminate with an error (represented by \isa{None}) or return
some value \isa{v} (represented by \isa{Some\ v}).
Similarly, \isa{nat} extended with $\infty$ can be modeled by type
\isa{nat\ option}. In both cases one could define a new datatype with
customized constructors like \isa{Error} and \isa{Infinity},
but it is often simpler to use \isa{option}. For an application see
\S\ref{sec:Trie}.%
\end{isamarkuptext}%
\isamarkuptrue%
%
\isadelimtheory
%
\endisadelimtheory
%
\isatagtheory
%
\endisatagtheory
{\isafoldtheory}%
%
\isadelimtheory
%
\endisadelimtheory
\end{isabellebody}%
%%% Local Variables:
%%% mode: latex
%%% TeX-master: "root"
%%% End:


\section{Definitions}
\label{sec:Definitions}

A definition is simply an abbreviation, i.e.\ a new name for an existing
construction. In particular, definitions cannot be recursive. Isabelle offers
definitions on the level of types and terms. Those on the type level are
called \textbf{type synonyms}; those on the term level are simply called 
definitions.


\subsection{Type Synonyms}

\index{type synonyms}%
Type synonyms are similar to those found in ML\@. They are created by a 
\commdx{types} command:

\medskip
%
\begin{isabellebody}%
\def\isabellecontext{types}%
\isacommand{types}\ number\ \ \ \ \ \ \ {\isacharequal}\ nat\isanewline
\ \ \ \ \ \ gate\ \ \ \ \ \ \ \ \ {\isacharequal}\ {\isachardoublequote}bool\ {\isasymRightarrow}\ bool\ {\isasymRightarrow}\ bool{\isachardoublequote}\isanewline
\ \ \ \ \ \ {\isacharparenleft}{\isacharprime}a{\isacharcomma}{\isacharprime}b{\isacharparenright}alist\ {\isacharequal}\ {\isachardoublequote}{\isacharparenleft}{\isacharprime}a\ {\isasymtimes}\ {\isacharprime}b{\isacharparenright}list{\isachardoublequote}%
\begin{isamarkuptext}%
\noindent\indexbold{*types}%
Internally all synonyms are fully expanded.  As a consequence Isabelle's
output never contains synonyms.  Their main purpose is to improve the
readability of theories.  Synonyms can be used just like any other
type:%
\end{isamarkuptext}%
\isacommand{consts}\ nand\ {\isacharcolon}{\isacharcolon}\ gate\isanewline
\ \ \ \ \ \ \ xor\ \ {\isacharcolon}{\isacharcolon}\ gate%
\isamarkupsubsection{Constant definitions%
}
%
\begin{isamarkuptext}%
\label{sec:ConstDefinitions}\indexbold{definition}%
The above constants \isa{nand} and \isa{xor} are non-recursive and can
therefore be defined directly by%
\end{isamarkuptext}%
\isacommand{defs}\ nand{\isacharunderscore}def{\isacharcolon}\ {\isachardoublequote}nand\ A\ B\ {\isasymequiv}\ {\isasymnot}{\isacharparenleft}A\ {\isasymand}\ B{\isacharparenright}{\isachardoublequote}\isanewline
\ \ \ \ \ xor{\isacharunderscore}def{\isacharcolon}\ \ {\isachardoublequote}xor\ A\ B\ \ {\isasymequiv}\ A\ {\isasymand}\ {\isasymnot}B\ {\isasymor}\ {\isasymnot}A\ {\isasymand}\ B{\isachardoublequote}%
\begin{isamarkuptext}%
\noindent%
where \isacommand{defs}\indexbold{*defs} is a keyword and
\isa{nand{\isacharunderscore}def} and \isa{xor{\isacharunderscore}def} are user-supplied names.
The symbol \indexboldpos{\isasymequiv}{$IsaEq} is a special form of equality
that must be used in constant definitions.
Declarations and definitions can also be merged%
\end{isamarkuptext}%
\isacommand{constdefs}\ nor\ {\isacharcolon}{\isacharcolon}\ gate\isanewline
\ \ \ \ \ \ \ \ \ {\isachardoublequote}nor\ A\ B\ {\isasymequiv}\ {\isasymnot}{\isacharparenleft}A\ {\isasymor}\ B{\isacharparenright}{\isachardoublequote}\isanewline
\ \ \ \ \ \ \ \ \ \ xor{\isadigit{2}}\ {\isacharcolon}{\isacharcolon}\ gate\isanewline
\ \ \ \ \ \ \ \ \ {\isachardoublequote}xor{\isadigit{2}}\ A\ B\ {\isasymequiv}\ {\isacharparenleft}A\ {\isasymor}\ B{\isacharparenright}\ {\isasymand}\ {\isacharparenleft}{\isasymnot}A\ {\isasymor}\ {\isasymnot}B{\isacharparenright}{\isachardoublequote}%
\begin{isamarkuptext}%
\noindent\indexbold{*constdefs}%
in which case the default name of each definition is $f$\isa{{\isacharunderscore}def}, where
$f$ is the name of the defined constant.%
\end{isamarkuptext}%
\end{isabellebody}%
%%% Local Variables:
%%% mode: latex
%%% TeX-master: "root"
%%% End:


%
\begin{isabellebody}%
\def\isabellecontext{prime{\isacharunderscore}def}%
%
\isadelimtheory
%
\endisadelimtheory
%
\isatagtheory
\isamarkupfalse%
%
\endisatagtheory
{\isafoldtheory}%
%
\isadelimtheory
%
\endisadelimtheory
\isamarkupfalse%
%
\begin{isamarkuptext}%
\begin{warn}
A common mistake when writing definitions is to introduce extra free
variables on the right-hand side.  Consider the following, flawed definition
(where \isa{dvd} means ``divides''):
\begin{isabelle}%
\ \ \ \ \ {\isachardoublequote}prime\ p\ {\isasymequiv}\ {\isadigit{1}}\ {\isacharless}\ p\ {\isasymand}\ {\isacharparenleft}m\ dvd\ p\ {\isasymlongrightarrow}\ m\ {\isacharequal}\ {\isadigit{1}}\ {\isasymor}\ m\ {\isacharequal}\ p{\isacharparenright}{\isachardoublequote}%
\end{isabelle}
\par\noindent\hangindent=0pt
Isabelle rejects this ``definition'' because of the extra \isa{m} on the
right-hand side, which would introduce an inconsistency (why?). 
The correct version is
\begin{isabelle}%
\ \ \ \ \ {\isachardoublequote}prime\ p\ {\isasymequiv}\ {\isadigit{1}}\ {\isacharless}\ p\ {\isasymand}\ {\isacharparenleft}{\isasymforall}m{\isachardot}\ m\ dvd\ p\ {\isasymlongrightarrow}\ m\ {\isacharequal}\ {\isadigit{1}}\ {\isasymor}\ m\ {\isacharequal}\ p{\isacharparenright}{\isachardoublequote}%
\end{isabelle}
\end{warn}%
\end{isamarkuptext}%
\isamarkuptrue%
%
\isadelimtheory
%
\endisadelimtheory
%
\isatagtheory
\isamarkupfalse%
%
\endisatagtheory
{\isafoldtheory}%
%
\isadelimtheory
%
\endisadelimtheory
\end{isabellebody}%
%%% Local Variables:
%%% mode: latex
%%% TeX-master: "root"
%%% End:



\section{The Definitional Approach}
\label{sec:definitional}

\index{Definitional Approach}%
As we pointed out at the beginning of the chapter, asserting arbitrary
axioms such as $f(n) = f(n) + 1$ can easily lead to contradictions. In order
to avoid this danger, we advocate the definitional rather than
the axiomatic approach: introduce new concepts by definitions. However,  Isabelle/HOL seems to
support many richer definitional constructs, such as
\isacommand{primrec}. The point is that Isabelle reduces such constructs to first principles. For example, each
\isacommand{primrec} function definition is turned into a proper
(nonrecursive!) definition from which the user-supplied recursion equations are
automatically proved.  This process is
hidden from the user, who does not have to understand the details.  Other commands described
later, like \isacommand{fun} and \isacommand{inductive}, work similarly.  
This strict adherence to the definitional approach reduces the risk of 
soundness errors.

\chapter{More Functional Programming}

The purpose of this chapter is to deepen your understanding of the
concepts encountered so far and to introduce advanced forms of datatypes and
recursive functions. The first two sections give a structured presentation of
theorem proving by simplification ({\S}\ref{sec:Simplification}) and discuss
important heuristics for induction ({\S}\ref{sec:InductionHeuristics}).  You can
skip them if you are not planning to perform proofs yourself.
We then present a case
study: a compiler for expressions ({\S}\ref{sec:ExprCompiler}). Advanced
datatypes, including those involving function spaces, are covered in
{\S}\ref{sec:advanced-datatypes}; it closes with another case study, search
trees (``tries'').  Finally we introduce \isacommand{fun}, a general
form of recursive function definition that goes well beyond 
\isacommand{primrec} ({\S}\ref{sec:fun}).


\section{Simplification}
\label{sec:Simplification}
\index{simplification|(}

So far we have proved our theorems by \isa{auto}, which simplifies
all subgoals. In fact, \isa{auto} can do much more than that. 
To go beyond toy examples, you
need to understand the ingredients of \isa{auto}.  This section covers the
method that \isa{auto} always applies first, simplification.

Simplification is one of the central theorem proving tools in Isabelle and
many other systems. The tool itself is called the \textbf{simplifier}. 
This section introduces the many features of the simplifier
and is required reading if you intend to perform proofs.  Later on,
{\S}\ref{sec:simplification-II} explains some more advanced features and a
little bit of how the simplifier works. The serious student should read that
section as well, in particular to understand why the simplifier did
something unexpected.

\subsection{What is Simplification?}

In its most basic form, simplification means repeated application of
equations from left to right. For example, taking the rules for \isa{\at}
and applying them to the term \isa{[0,1] \at\ []} results in a sequence of
simplification steps:
\begin{ttbox}\makeatother
(0#1#[]) @ []  \(\leadsto\)  0#((1#[]) @ [])  \(\leadsto\)  0#(1#([] @ []))  \(\leadsto\)  0#1#[]
\end{ttbox}
This is also known as \bfindex{term rewriting}\indexbold{rewriting} and the
equations are referred to as \bfindex{rewrite rules}.
``Rewriting'' is more honest than ``simplification'' because the terms do not
necessarily become simpler in the process.

The simplifier proves arithmetic goals as described in
{\S}\ref{sec:nat} above.  Arithmetic expressions are simplified using built-in
procedures that go beyond mere rewrite rules.  New simplification procedures
can be coded and installed, but they are definitely not a matter for this
tutorial. 

%
\begin{isabellebody}%
\def\isabellecontext{simp}%
%
\isamarkupsubsubsection{Simplification rules}
%
\begin{isamarkuptext}%
\indexbold{simplification rule}
To facilitate simplification, theorems can be declared to be simplification
rules (with the help of the attribute \isa{{\isacharbrackleft}simp{\isacharbrackright}}\index{*simp
  (attribute)}), in which case proofs by simplification make use of these
rules automatically. In addition the constructs \isacommand{datatype} and
\isacommand{primrec} (and a few others) invisibly declare useful
simplification rules. Explicit definitions are \emph{not} declared
simplification rules automatically!

Not merely equations but pretty much any theorem can become a simplification
rule. The simplifier will try to make sense of it.  For example, a theorem
\isa{{\isasymnot}\ P} is automatically turned into \isa{P\ {\isacharequal}\ False}. The details
are explained in \S\ref{sec:SimpHow}.

The simplification attribute of theorems can be turned on and off as follows:
\begin{quote}
\isacommand{declare} \textit{theorem-name}\isa{{\isacharbrackleft}simp{\isacharbrackright}}\\
\isacommand{declare} \textit{theorem-name}\isa{{\isacharbrackleft}simp\ del{\isacharbrackright}}
\end{quote}
As a rule of thumb, equations that really simplify (like \isa{rev\ {\isacharparenleft}rev\ xs{\isacharparenright}\ {\isacharequal}\ xs} and \isa{xs\ {\isacharat}\ {\isacharbrackleft}{\isacharbrackright}\ {\isacharequal}\ xs}) should be made simplification
rules.  Those of a more specific nature (e.g.\ distributivity laws, which
alter the structure of terms considerably) should only be used selectively,
i.e.\ they should not be default simplification rules.  Conversely, it may
also happen that a simplification rule needs to be disabled in certain
proofs.  Frequent changes in the simplification status of a theorem may
indicate a badly designed theory.
\begin{warn}
  Simplification may not terminate, for example if both $f(x) = g(x)$ and
  $g(x) = f(x)$ are simplification rules. It is the user's responsibility not
  to include simplification rules that can lead to nontermination, either on
  their own or in combination with other simplification rules.
\end{warn}%
\end{isamarkuptext}%
%
\isamarkupsubsubsection{The simplification method}
%
\begin{isamarkuptext}%
\index{*simp (method)|bold}
The general format of the simplification method is
\begin{quote}
\isa{simp} \textit{list of modifiers}
\end{quote}
where the list of \emph{modifiers} helps to fine tune the behaviour and may
be empty. Most if not all of the proofs seen so far could have been performed
with \isa{simp} instead of \isa{auto}, except that \isa{simp} attacks
only the first subgoal and may thus need to be repeated---use
\isaindex{simp_all} to simplify all subgoals.
Note that \isa{simp} fails if nothing changes.%
\end{isamarkuptext}%
%
\isamarkupsubsubsection{Adding and deleting simplification rules}
%
\begin{isamarkuptext}%
If a certain theorem is merely needed in a few proofs by simplification,
we do not need to make it a global simplification rule. Instead we can modify
the set of simplification rules used in a simplification step by adding rules
to it and/or deleting rules from it. The two modifiers for this are
\begin{quote}
\isa{add{\isacharcolon}} \textit{list of theorem names}\\
\isa{del{\isacharcolon}} \textit{list of theorem names}
\end{quote}
In case you want to use only a specific list of theorems and ignore all
others:
\begin{quote}
\isa{only{\isacharcolon}} \textit{list of theorem names}
\end{quote}%
\end{isamarkuptext}%
%
\isamarkupsubsubsection{Assumptions}
%
\begin{isamarkuptext}%
\index{simplification!with/of assumptions}
By default, assumptions are part of the simplification process: they are used
as simplification rules and are simplified themselves. For example:%
\end{isamarkuptext}%
\isacommand{lemma}\ {\isachardoublequote}{\isasymlbrakk}\ xs\ {\isacharat}\ zs\ {\isacharequal}\ ys\ {\isacharat}\ xs{\isacharsemicolon}\ {\isacharbrackleft}{\isacharbrackright}\ {\isacharat}\ xs\ {\isacharequal}\ {\isacharbrackleft}{\isacharbrackright}\ {\isacharat}\ {\isacharbrackleft}{\isacharbrackright}\ {\isasymrbrakk}\ {\isasymLongrightarrow}\ ys\ {\isacharequal}\ zs{\isachardoublequote}\isanewline
\isacommand{by}\ simp%
\begin{isamarkuptext}%
\noindent
The second assumption simplifies to \isa{xs\ {\isacharequal}\ {\isacharbrackleft}{\isacharbrackright}}, which in turn
simplifies the first assumption to \isa{zs\ {\isacharequal}\ ys}, thus reducing the
conclusion to \isa{ys\ {\isacharequal}\ ys} and hence to \isa{True}.

In some cases this may be too much of a good thing and may lead to
nontermination:%
\end{isamarkuptext}%
\isacommand{lemma}\ {\isachardoublequote}{\isasymforall}x{\isachardot}\ f\ x\ {\isacharequal}\ g\ {\isacharparenleft}f\ {\isacharparenleft}g\ x{\isacharparenright}{\isacharparenright}\ {\isasymLongrightarrow}\ f\ {\isacharbrackleft}{\isacharbrackright}\ {\isacharequal}\ f\ {\isacharbrackleft}{\isacharbrackright}\ {\isacharat}\ {\isacharbrackleft}{\isacharbrackright}{\isachardoublequote}%
\begin{isamarkuptxt}%
\noindent
cannot be solved by an unmodified application of \isa{simp} because the
simplification rule \isa{f\ x\ {\isacharequal}\ g\ {\isacharparenleft}f\ {\isacharparenleft}g\ x{\isacharparenright}{\isacharparenright}} extracted from the assumption
does not terminate. Isabelle notices certain simple forms of
nontermination but not this one. The problem can be circumvented by
explicitly telling the simplifier to ignore the assumptions:%
\end{isamarkuptxt}%
\isacommand{by}{\isacharparenleft}simp\ {\isacharparenleft}no{\isacharunderscore}asm{\isacharparenright}{\isacharparenright}%
\begin{isamarkuptext}%
\noindent
There are three options that influence the treatment of assumptions:
\begin{description}
\item[\isa{{\isacharparenleft}no{\isacharunderscore}asm{\isacharparenright}}]\indexbold{*no_asm}
 means that assumptions are completely ignored.
\item[\isa{{\isacharparenleft}no{\isacharunderscore}asm{\isacharunderscore}simp{\isacharparenright}}]\indexbold{*no_asm_simp}
 means that the assumptions are not simplified but
  are used in the simplification of the conclusion.
\item[\isa{{\isacharparenleft}no{\isacharunderscore}asm{\isacharunderscore}use{\isacharparenright}}]\indexbold{*no_asm_use}
 means that the assumptions are simplified but are not
  used in the simplification of each other or the conclusion.
\end{description}
Neither \isa{{\isacharparenleft}no{\isacharunderscore}asm{\isacharunderscore}simp{\isacharparenright}} nor \isa{{\isacharparenleft}no{\isacharunderscore}asm{\isacharunderscore}use{\isacharparenright}} allow to simplify
the above problematic subgoal.

Note that only one of the above options is allowed, and it must precede all
other arguments.%
\end{isamarkuptext}%
%
\isamarkupsubsubsection{Rewriting with definitions}
%
\begin{isamarkuptext}%
\index{simplification!with definitions}
Constant definitions (\S\ref{sec:ConstDefinitions}) can
be used as simplification rules, but by default they are not.  Hence the
simplifier does not expand them automatically, just as it should be:
definitions are introduced for the purpose of abbreviating complex
concepts. Of course we need to expand the definitions initially to derive
enough lemmas that characterize the concept sufficiently for us to forget the
original definition. For example, given%
\end{isamarkuptext}%
\isacommand{constdefs}\ exor\ {\isacharcolon}{\isacharcolon}\ {\isachardoublequote}bool\ {\isasymRightarrow}\ bool\ {\isasymRightarrow}\ bool{\isachardoublequote}\isanewline
\ \ \ \ \ \ \ \ \ {\isachardoublequote}exor\ A\ B\ {\isasymequiv}\ {\isacharparenleft}A\ {\isasymand}\ {\isasymnot}B{\isacharparenright}\ {\isasymor}\ {\isacharparenleft}{\isasymnot}A\ {\isasymand}\ B{\isacharparenright}{\isachardoublequote}%
\begin{isamarkuptext}%
\noindent
we may want to prove%
\end{isamarkuptext}%
\isacommand{lemma}\ {\isachardoublequote}exor\ A\ {\isacharparenleft}{\isasymnot}A{\isacharparenright}{\isachardoublequote}%
\begin{isamarkuptxt}%
\noindent
Typically, the opening move consists in \emph{unfolding} the definition(s), which we need to
get started, but nothing else:\indexbold{*unfold}\indexbold{definition!unfolding}%
\end{isamarkuptxt}%
\isacommand{apply}{\isacharparenleft}simp\ only{\isacharcolon}exor{\isacharunderscore}def{\isacharparenright}%
\begin{isamarkuptxt}%
\noindent
In this particular case, the resulting goal
\begin{isabelle}
~1.~A~{\isasymand}~{\isasymnot}~{\isasymnot}~A~{\isasymor}~{\isasymnot}~A~{\isasymand}~{\isasymnot}~A%
\end{isabelle}
can be proved by simplification. Thus we could have proved the lemma outright%
\end{isamarkuptxt}%
\isacommand{by}{\isacharparenleft}simp\ add{\isacharcolon}\ exor{\isacharunderscore}def{\isacharparenright}%
\begin{isamarkuptext}%
\noindent
Of course we can also unfold definitions in the middle of a proof.

You should normally not turn a definition permanently into a simplification
rule because this defeats the whole purpose of an abbreviation.

\begin{warn}
  If you have defined $f\,x\,y~\isasymequiv~t$ then you can only expand
  occurrences of $f$ with at least two arguments. Thus it is safer to define
  $f$~\isasymequiv~\isasymlambda$x\,y.\;t$.
\end{warn}%
\end{isamarkuptext}%
%
\isamarkupsubsubsection{Simplifying let-expressions}
%
\begin{isamarkuptext}%
\index{simplification!of let-expressions}
Proving a goal containing \isaindex{let}-expressions almost invariably
requires the \isa{let}-con\-structs to be expanded at some point. Since
\isa{let}-\isa{in} is just syntactic sugar for a predefined constant
(called \isa{Let}), expanding \isa{let}-constructs means rewriting with
\isa{Let{\isacharunderscore}def}:%
\end{isamarkuptext}%
\isacommand{lemma}\ {\isachardoublequote}{\isacharparenleft}let\ xs\ {\isacharequal}\ {\isacharbrackleft}{\isacharbrackright}\ in\ xs{\isacharat}ys{\isacharat}xs{\isacharparenright}\ {\isacharequal}\ ys{\isachardoublequote}\isanewline
\isacommand{by}{\isacharparenleft}simp\ add{\isacharcolon}\ Let{\isacharunderscore}def{\isacharparenright}%
\begin{isamarkuptext}%
If, in a particular context, there is no danger of a combinatorial explosion
of nested \isa{let}s one could even simlify with \isa{Let{\isacharunderscore}def} by
default:%
\end{isamarkuptext}%
\isacommand{declare}\ Let{\isacharunderscore}def\ {\isacharbrackleft}simp{\isacharbrackright}%
\isamarkupsubsubsection{Conditional equations}
%
\begin{isamarkuptext}%
So far all examples of rewrite rules were equations. The simplifier also
accepts \emph{conditional} equations, for example%
\end{isamarkuptext}%
\isacommand{lemma}\ hd{\isacharunderscore}Cons{\isacharunderscore}tl{\isacharbrackleft}simp{\isacharbrackright}{\isacharcolon}\ {\isachardoublequote}xs\ {\isasymnoteq}\ {\isacharbrackleft}{\isacharbrackright}\ \ {\isasymLongrightarrow}\ \ hd\ xs\ {\isacharhash}\ tl\ xs\ {\isacharequal}\ xs{\isachardoublequote}\isanewline
\isacommand{by}{\isacharparenleft}case{\isacharunderscore}tac\ xs{\isacharcomma}\ simp{\isacharcomma}\ simp{\isacharparenright}%
\begin{isamarkuptext}%
\noindent
Note the use of ``\ttindexboldpos{,}{$Isar}'' to string together a
sequence of methods. Assuming that the simplification rule
\isa{{\isacharparenleft}rev\ xs\ {\isacharequal}\ {\isacharbrackleft}{\isacharbrackright}{\isacharparenright}\ {\isacharequal}\ {\isacharparenleft}xs\ {\isacharequal}\ {\isacharbrackleft}{\isacharbrackright}{\isacharparenright}}
is present as well,%
\end{isamarkuptext}%
\isacommand{lemma}\ {\isachardoublequote}xs\ {\isasymnoteq}\ {\isacharbrackleft}{\isacharbrackright}\ {\isasymLongrightarrow}\ hd{\isacharparenleft}rev\ xs{\isacharparenright}\ {\isacharhash}\ tl{\isacharparenleft}rev\ xs{\isacharparenright}\ {\isacharequal}\ rev\ xs{\isachardoublequote}%
\begin{isamarkuptext}%
\noindent
is proved by plain simplification:
the conditional equation \isa{hd{\isacharunderscore}Cons{\isacharunderscore}tl} above
can simplify \isa{hd\ {\isacharparenleft}rev\ xs{\isacharparenright}\ {\isacharhash}\ tl\ {\isacharparenleft}rev\ xs{\isacharparenright}} to \isa{rev\ xs}
because the corresponding precondition \isa{rev\ xs\ {\isasymnoteq}\ {\isacharbrackleft}{\isacharbrackright}}
simplifies to \isa{xs\ {\isasymnoteq}\ {\isacharbrackleft}{\isacharbrackright}}, which is exactly the local
assumption of the subgoal.%
\end{isamarkuptext}%
%
\isamarkupsubsubsection{Automatic case splits}
%
\begin{isamarkuptext}%
\indexbold{case splits}\index{*split|(}
Goals containing \isa{if}-expressions are usually proved by case
distinction on the condition of the \isa{if}. For example the goal%
\end{isamarkuptext}%
\isacommand{lemma}\ {\isachardoublequote}{\isasymforall}xs{\isachardot}\ if\ xs\ {\isacharequal}\ {\isacharbrackleft}{\isacharbrackright}\ then\ rev\ xs\ {\isacharequal}\ {\isacharbrackleft}{\isacharbrackright}\ else\ rev\ xs\ {\isasymnoteq}\ {\isacharbrackleft}{\isacharbrackright}{\isachardoublequote}%
\begin{isamarkuptxt}%
\noindent
can be split into
\begin{isabelle}
~1.~{\isasymforall}xs.~(xs~=~[]~{\isasymlongrightarrow}~rev~xs~=~[])~{\isasymand}~(xs~{\isasymnoteq}~[]~{\isasymlongrightarrow}~rev~xs~{\isasymnoteq}~[])
\end{isabelle}
by a degenerate form of simplification%
\end{isamarkuptxt}%
\isacommand{apply}{\isacharparenleft}simp\ only{\isacharcolon}\ split{\isacharcolon}\ split{\isacharunderscore}if{\isacharparenright}%
\begin{isamarkuptext}%
\noindent
where no simplification rules are included (\isa{only{\isacharcolon}} is followed by the
empty list of theorems) but the rule \isaindexbold{split_if} for
splitting \isa{if}s is added (via the modifier \isa{split{\isacharcolon}}). Because
case-splitting on \isa{if}s is almost always the right proof strategy, the
simplifier performs it automatically. Try \isacommand{apply}\isa{{\isacharparenleft}simp{\isacharparenright}}
on the initial goal above.

This splitting idea generalizes from \isa{if} to \isaindex{case}:%
\end{isamarkuptext}%
\isacommand{lemma}\ {\isachardoublequote}{\isacharparenleft}case\ xs\ of\ {\isacharbrackleft}{\isacharbrackright}\ {\isasymRightarrow}\ zs\ {\isacharbar}\ y{\isacharhash}ys\ {\isasymRightarrow}\ y{\isacharhash}{\isacharparenleft}ys{\isacharat}zs{\isacharparenright}{\isacharparenright}\ {\isacharequal}\ xs{\isacharat}zs{\isachardoublequote}%
\begin{isamarkuptxt}%
\noindent
becomes
\begin{isabelle}\makeatother
~1.~(xs~=~[]~{\isasymlongrightarrow}~zs~=~xs~@~zs)~{\isasymand}\isanewline
~~~~({\isasymforall}a~list.~xs~=~a~\#~list~{\isasymlongrightarrow}~a~\#~list~@~zs~=~xs~@~zs)
\end{isabelle}
by typing%
\end{isamarkuptxt}%
\isacommand{apply}{\isacharparenleft}simp\ only{\isacharcolon}\ split{\isacharcolon}\ list{\isachardot}split{\isacharparenright}%
\begin{isamarkuptext}%
\noindent
In contrast to \isa{if}-expressions, the simplifier does not split
\isa{case}-expressions by default because this can lead to nontermination
in case of recursive datatypes. Again, if the \isa{only{\isacharcolon}} modifier is
dropped, the above goal is solved,%
\end{isamarkuptext}%
\isacommand{by}{\isacharparenleft}simp\ split{\isacharcolon}\ list{\isachardot}split{\isacharparenright}%
\begin{isamarkuptext}%
\noindent%
which \isacommand{apply}\isa{{\isacharparenleft}simp{\isacharparenright}} alone will not do.

In general, every datatype $t$ comes with a theorem
$t$\isa{{\isachardot}split} which can be declared to be a \bfindex{split rule} either
locally as above, or by giving it the \isa{split} attribute globally:%
\end{isamarkuptext}%
\isacommand{declare}\ list{\isachardot}split\ {\isacharbrackleft}split{\isacharbrackright}%
\begin{isamarkuptext}%
\noindent
The \isa{split} attribute can be removed with the \isa{del} modifier,
either locally%
\end{isamarkuptext}%
\isacommand{apply}{\isacharparenleft}simp\ split\ del{\isacharcolon}\ split{\isacharunderscore}if{\isacharparenright}%
\begin{isamarkuptext}%
\noindent
or globally:%
\end{isamarkuptext}%
\isacommand{declare}\ list{\isachardot}split\ {\isacharbrackleft}split\ del{\isacharbrackright}%
\begin{isamarkuptext}%
The above split rules intentionally only affect the conclusion of a
subgoal.  If you want to split an \isa{if} or \isa{case}-expression in
the assumptions, you have to apply \isa{split{\isacharunderscore}if{\isacharunderscore}asm} or
$t$\isa{{\isachardot}split{\isacharunderscore}asm}:%
\end{isamarkuptext}%
\isacommand{lemma}\ {\isachardoublequote}if\ xs\ {\isacharequal}\ {\isacharbrackleft}{\isacharbrackright}\ then\ ys\ {\isachartilde}{\isacharequal}\ {\isacharbrackleft}{\isacharbrackright}\ else\ ys\ {\isacharequal}\ {\isacharbrackleft}{\isacharbrackright}\ {\isacharequal}{\isacharequal}{\isachargreater}\ xs\ {\isacharat}\ ys\ {\isachartilde}{\isacharequal}\ {\isacharbrackleft}{\isacharbrackright}{\isachardoublequote}\isanewline
\isacommand{apply}{\isacharparenleft}simp\ only{\isacharcolon}\ split{\isacharcolon}\ split{\isacharunderscore}if{\isacharunderscore}asm{\isacharparenright}%
\begin{isamarkuptext}%
\noindent
In contrast to splitting the conclusion, this actually creates two
separate subgoals (which are solved by \isa{simp{\isacharunderscore}all}):
\begin{isabelle}
\ \isadigit{1}{\isachardot}\ {\isasymlbrakk}\mbox{xs}\ {\isacharequal}\ {\isacharbrackleft}{\isacharbrackright}{\isacharsemicolon}\ \mbox{ys}\ {\isasymnoteq}\ {\isacharbrackleft}{\isacharbrackright}{\isasymrbrakk}\ {\isasymLongrightarrow}\ {\isacharbrackleft}{\isacharbrackright}\ {\isacharat}\ \mbox{ys}\ {\isasymnoteq}\ {\isacharbrackleft}{\isacharbrackright}\isanewline
\ \isadigit{2}{\isachardot}\ {\isasymlbrakk}\mbox{xs}\ {\isasymnoteq}\ {\isacharbrackleft}{\isacharbrackright}{\isacharsemicolon}\ \mbox{ys}\ {\isacharequal}\ {\isacharbrackleft}{\isacharbrackright}{\isasymrbrakk}\ {\isasymLongrightarrow}\ \mbox{xs}\ {\isacharat}\ {\isacharbrackleft}{\isacharbrackright}\ {\isasymnoteq}\ {\isacharbrackleft}{\isacharbrackright}
\end{isabelle}
If you need to split both in the assumptions and the conclusion,
use $t$\isa{{\isachardot}splits} which subsumes $t$\isa{{\isachardot}split} and
$t$\isa{{\isachardot}split{\isacharunderscore}asm}. Analogously, there is \isa{if{\isacharunderscore}splits}.

\begin{warn}
  The simplifier merely simplifies the condition of an \isa{if} but not the
  \isa{then} or \isa{else} parts. The latter are simplified only after the
  condition reduces to \isa{True} or \isa{False}, or after splitting. The
  same is true for \isaindex{case}-expressions: only the selector is
  simplified at first, until either the expression reduces to one of the
  cases or it is split.
\end{warn}

\index{*split|)}%
\end{isamarkuptext}%
%
\isamarkupsubsubsection{Arithmetic}
%
\begin{isamarkuptext}%
\index{arithmetic}
The simplifier routinely solves a small class of linear arithmetic formulae
(over type \isa{nat} and other numeric types): it only takes into account
assumptions and conclusions that are (possibly negated) (in)equalities
(\isa{{\isacharequal}}, \isasymle, \isa{{\isacharless}}) and it only knows about addition. Thus%
\end{isamarkuptext}%
\isacommand{lemma}\ {\isachardoublequote}{\isasymlbrakk}\ {\isasymnot}\ m\ {\isacharless}\ n{\isacharsemicolon}\ m\ {\isacharless}\ n{\isacharplus}\isadigit{1}\ {\isasymrbrakk}\ {\isasymLongrightarrow}\ m\ {\isacharequal}\ n{\isachardoublequote}%
\begin{isamarkuptext}%
\noindent
is proved by simplification, whereas the only slightly more complex%
\end{isamarkuptext}%
\isacommand{lemma}\ {\isachardoublequote}{\isasymnot}\ m\ {\isacharless}\ n\ {\isasymand}\ m\ {\isacharless}\ n{\isacharplus}\isadigit{1}\ {\isasymLongrightarrow}\ m\ {\isacharequal}\ n{\isachardoublequote}%
\begin{isamarkuptext}%
\noindent
is not proved by simplification and requires \isa{arith}.%
\end{isamarkuptext}%
%
\isamarkupsubsubsection{Tracing}
%
\begin{isamarkuptext}%
\indexbold{tracing the simplifier}
Using the simplifier effectively may take a bit of experimentation.  Set the
\isaindexbold{trace_simp} \rmindex{flag} to get a better idea of what is going
on:%
\end{isamarkuptext}%
\isacommand{ML}\ {\isachardoublequote}set\ trace{\isacharunderscore}simp{\isachardoublequote}\isanewline
\isacommand{lemma}\ {\isachardoublequote}rev\ {\isacharbrackleft}a{\isacharbrackright}\ {\isacharequal}\ {\isacharbrackleft}{\isacharbrackright}{\isachardoublequote}\isanewline
\isacommand{apply}{\isacharparenleft}simp{\isacharparenright}%
\begin{isamarkuptext}%
\noindent
produces the trace

\begin{ttbox}\makeatother
Applying instance of rewrite rule:
rev (?x1 \# ?xs1) == rev ?xs1 @ [?x1]
Rewriting:
rev [x] == rev [] @ [x]
Applying instance of rewrite rule:
rev [] == []
Rewriting:
rev [] == []
Applying instance of rewrite rule:
[] @ ?y == ?y
Rewriting:
[] @ [x] == [x]
Applying instance of rewrite rule:
?x3 \# ?t3 = ?t3 == False
Rewriting:
[x] = [] == False
\end{ttbox}

In more complicated cases, the trace can be quite lenghty, especially since
invocations of the simplifier are often nested (e.g.\ when solving conditions
of rewrite rules). Thus it is advisable to reset it:%
\end{isamarkuptext}%
\isacommand{ML}\ {\isachardoublequote}reset\ trace{\isacharunderscore}simp{\isachardoublequote}\isanewline
\end{isabellebody}%
%%% Local Variables:
%%% mode: latex
%%% TeX-master: "root"
%%% End:


\index{simplification|)}

%
\begin{isabellebody}%
\def\isabellecontext{Itrev}%
\isamarkupfalse%
%
\isadelimtheory
%
\endisadelimtheory
%
\isatagtheory
%
\endisatagtheory
{\isafoldtheory}%
%
\isadelimtheory
%
\endisadelimtheory
%
\isamarkupsection{Induction Heuristics%
}
\isamarkuptrue%
%
\begin{isamarkuptext}%
\label{sec:InductionHeuristics}
\index{induction heuristics|(}%
The purpose of this section is to illustrate some simple heuristics for
inductive proofs. The first one we have already mentioned in our initial
example:
\begin{quote}
\emph{Theorems about recursive functions are proved by induction.}
\end{quote}
In case the function has more than one argument
\begin{quote}
\emph{Do induction on argument number $i$ if the function is defined by
recursion in argument number $i$.}
\end{quote}
When we look at the proof of \isa{{\isacharparenleft}xs{\isacharat}ys{\isacharparenright}\ {\isacharat}\ zs\ {\isacharequal}\ xs\ {\isacharat}\ {\isacharparenleft}ys{\isacharat}zs{\isacharparenright}}
in \S\ref{sec:intro-proof} we find
\begin{itemize}
\item \isa{{\isacharat}} is recursive in
the first argument
\item \isa{xs}  occurs only as the first argument of
\isa{{\isacharat}}
\item both \isa{ys} and \isa{zs} occur at least once as
the second argument of \isa{{\isacharat}}
\end{itemize}
Hence it is natural to perform induction on~\isa{xs}.

The key heuristic, and the main point of this section, is to
\emph{generalize the goal before induction}.
The reason is simple: if the goal is
too specific, the induction hypothesis is too weak to allow the induction
step to go through. Let us illustrate the idea with an example.

Function \cdx{rev} has quadratic worst-case running time
because it calls function \isa{{\isacharat}} for each element of the list and
\isa{{\isacharat}} is linear in its first argument.  A linear time version of
\isa{rev} reqires an extra argument where the result is accumulated
gradually, using only~\isa{{\isacharhash}}:%
\end{isamarkuptext}%
\isamarkuptrue%
\isacommand{consts}\isamarkupfalse%
\ itrev\ {\isacharcolon}{\isacharcolon}\ {\isachardoublequoteopen}{\isacharprime}a\ list\ {\isasymRightarrow}\ {\isacharprime}a\ list\ {\isasymRightarrow}\ {\isacharprime}a\ list{\isachardoublequoteclose}\isanewline
\isacommand{primrec}\isamarkupfalse%
\isanewline
{\isachardoublequoteopen}itrev\ {\isacharbrackleft}{\isacharbrackright}\ \ \ \ \ ys\ {\isacharequal}\ ys{\isachardoublequoteclose}\isanewline
{\isachardoublequoteopen}itrev\ {\isacharparenleft}x{\isacharhash}xs{\isacharparenright}\ ys\ {\isacharequal}\ itrev\ xs\ {\isacharparenleft}x{\isacharhash}ys{\isacharparenright}{\isachardoublequoteclose}%
\begin{isamarkuptext}%
\noindent
The behaviour of \cdx{itrev} is simple: it reverses
its first argument by stacking its elements onto the second argument,
and returning that second argument when the first one becomes
empty. Note that \isa{Itrev{\isachardot}itrev} is tail-recursive: it can be
compiled into a loop.

Naturally, we would like to show that \isa{Itrev{\isachardot}itrev} does indeed reverse
its first argument provided the second one is empty:%
\end{isamarkuptext}%
\isamarkuptrue%
\isacommand{lemma}\isamarkupfalse%
\ {\isachardoublequoteopen}itrev\ xs\ {\isacharbrackleft}{\isacharbrackright}\ {\isacharequal}\ rev\ xs{\isachardoublequoteclose}%
\isadelimproof
%
\endisadelimproof
%
\isatagproof
%
\begin{isamarkuptxt}%
\noindent
There is no choice as to the induction variable, and we immediately simplify:%
\end{isamarkuptxt}%
\isamarkuptrue%
\isacommand{apply}\isamarkupfalse%
{\isacharparenleft}induct{\isacharunderscore}tac\ xs{\isacharcomma}\ simp{\isacharunderscore}all{\isacharparenright}%
\begin{isamarkuptxt}%
\noindent
Unfortunately, this attempt does not prove
the induction step:
\begin{isabelle}%
\ {\isadigit{1}}{\isachardot}\ {\isasymAnd}a\ list{\isachardot}\isanewline
\isaindent{\ {\isadigit{1}}{\isachardot}\ \ \ \ }Itrev{\isachardot}itrev\ list\ {\isacharbrackleft}{\isacharbrackright}\ {\isacharequal}\ rev\ list\ {\isasymLongrightarrow}\isanewline
\isaindent{\ {\isadigit{1}}{\isachardot}\ \ \ \ }Itrev{\isachardot}itrev\ list\ {\isacharbrackleft}a{\isacharbrackright}\ {\isacharequal}\ rev\ list\ {\isacharat}\ {\isacharbrackleft}a{\isacharbrackright}%
\end{isabelle}
The induction hypothesis is too weak.  The fixed
argument,~\isa{{\isacharbrackleft}{\isacharbrackright}}, prevents it from rewriting the conclusion.  
This example suggests a heuristic:
\begin{quote}\index{generalizing induction formulae}%
\emph{Generalize goals for induction by replacing constants by variables.}
\end{quote}
Of course one cannot do this na\"{\i}vely: \isa{Itrev{\isachardot}itrev\ xs\ ys\ {\isacharequal}\ rev\ xs} is
just not true.  The correct generalization is%
\end{isamarkuptxt}%
\isamarkuptrue%
%
\endisatagproof
{\isafoldproof}%
%
\isadelimproof
%
\endisadelimproof
\isacommand{lemma}\isamarkupfalse%
\ {\isachardoublequoteopen}itrev\ xs\ ys\ {\isacharequal}\ rev\ xs\ {\isacharat}\ ys{\isachardoublequoteclose}%
\isadelimproof
%
\endisadelimproof
%
\isatagproof
%
\begin{isamarkuptxt}%
\noindent
If \isa{ys} is replaced by \isa{{\isacharbrackleft}{\isacharbrackright}}, the right-hand side simplifies to
\isa{rev\ xs}, as required.

In this instance it was easy to guess the right generalization.
Other situations can require a good deal of creativity.  

Although we now have two variables, only \isa{xs} is suitable for
induction, and we repeat our proof attempt. Unfortunately, we are still
not there:
\begin{isabelle}%
\ {\isadigit{1}}{\isachardot}\ {\isasymAnd}a\ list{\isachardot}\isanewline
\isaindent{\ {\isadigit{1}}{\isachardot}\ \ \ \ }Itrev{\isachardot}itrev\ list\ ys\ {\isacharequal}\ rev\ list\ {\isacharat}\ ys\ {\isasymLongrightarrow}\isanewline
\isaindent{\ {\isadigit{1}}{\isachardot}\ \ \ \ }Itrev{\isachardot}itrev\ list\ {\isacharparenleft}a\ {\isacharhash}\ ys{\isacharparenright}\ {\isacharequal}\ rev\ list\ {\isacharat}\ a\ {\isacharhash}\ ys%
\end{isabelle}
The induction hypothesis is still too weak, but this time it takes no
intuition to generalize: the problem is that \isa{ys} is fixed throughout
the subgoal, but the induction hypothesis needs to be applied with
\isa{a\ {\isacharhash}\ ys} instead of \isa{ys}. Hence we prove the theorem
for all \isa{ys} instead of a fixed one:%
\end{isamarkuptxt}%
\isamarkuptrue%
%
\endisatagproof
{\isafoldproof}%
%
\isadelimproof
%
\endisadelimproof
\isacommand{lemma}\isamarkupfalse%
\ {\isachardoublequoteopen}{\isasymforall}ys{\isachardot}\ itrev\ xs\ ys\ {\isacharequal}\ rev\ xs\ {\isacharat}\ ys{\isachardoublequoteclose}%
\isadelimproof
%
\endisadelimproof
%
\isatagproof
%
\endisatagproof
{\isafoldproof}%
%
\isadelimproof
%
\endisadelimproof
%
\begin{isamarkuptext}%
\noindent
This time induction on \isa{xs} followed by simplification succeeds. This
leads to another heuristic for generalization:
\begin{quote}
\emph{Generalize goals for induction by universally quantifying all free
variables {\em(except the induction variable itself!)}.}
\end{quote}
This prevents trivial failures like the one above and does not affect the
validity of the goal.  However, this heuristic should not be applied blindly.
It is not always required, and the additional quantifiers can complicate
matters in some cases. The variables that should be quantified are typically
those that change in recursive calls.

A final point worth mentioning is the orientation of the equation we just
proved: the more complex notion (\isa{Itrev{\isachardot}itrev}) is on the left-hand
side, the simpler one (\isa{rev}) on the right-hand side. This constitutes
another, albeit weak heuristic that is not restricted to induction:
\begin{quote}
  \emph{The right-hand side of an equation should (in some sense) be simpler
    than the left-hand side.}
\end{quote}
This heuristic is tricky to apply because it is not obvious that
\isa{rev\ xs\ {\isacharat}\ ys} is simpler than \isa{Itrev{\isachardot}itrev\ xs\ ys}. But see what
happens if you try to prove \isa{rev\ xs\ {\isacharat}\ ys\ {\isacharequal}\ Itrev{\isachardot}itrev\ xs\ ys}!

If you have tried these heuristics and still find your
induction does not go through, and no obvious lemma suggests itself, you may
need to generalize your proposition even further. This requires insight into
the problem at hand and is beyond simple rules of thumb.  
Additionally, you can read \S\ref{sec:advanced-ind}
to learn about some advanced techniques for inductive proofs.%
\index{induction heuristics|)}%
\end{isamarkuptext}%
\isamarkuptrue%
%
\isadelimtheory
%
\endisadelimtheory
%
\isatagtheory
%
\endisatagtheory
{\isafoldtheory}%
%
\isadelimtheory
%
\endisadelimtheory
\end{isabellebody}%
%%% Local Variables:
%%% mode: latex
%%% TeX-master: "root"
%%% End:

\begin{exercise}
%
\begin{isabellebody}%
\def\isabellecontext{Plus}%
%
\isadelimtheory
%
\endisadelimtheory
%
\isatagtheory
%
\endisatagtheory
{\isafoldtheory}%
%
\isadelimtheory
%
\endisadelimtheory
%
\begin{isamarkuptext}%
\noindent Define the following addition function%
\end{isamarkuptext}%
\isamarkuptrue%
\isacommand{consts}\isamarkupfalse%
\ plus\ {\isacharcolon}{\isacharcolon}\ {\isachardoublequoteopen}nat\ {\isasymRightarrow}\ nat\ {\isasymRightarrow}\ nat{\isachardoublequoteclose}\isanewline
\isacommand{primrec}\isamarkupfalse%
\isanewline
{\isachardoublequoteopen}plus\ m\ {\isadigit{0}}\ {\isacharequal}\ m{\isachardoublequoteclose}\isanewline
{\isachardoublequoteopen}plus\ m\ {\isacharparenleft}Suc\ n{\isacharparenright}\ {\isacharequal}\ plus\ {\isacharparenleft}Suc\ m{\isacharparenright}\ n{\isachardoublequoteclose}%
\begin{isamarkuptext}%
\noindent and prove%
\end{isamarkuptext}%
\isamarkuptrue%
%
\isadelimproof
%
\endisadelimproof
%
\isatagproof
%
\endisatagproof
{\isafoldproof}%
%
\isadelimproof
%
\endisadelimproof
\isacommand{lemma}\isamarkupfalse%
\ {\isachardoublequoteopen}plus\ m\ n\ {\isacharequal}\ m{\isacharplus}n{\isachardoublequoteclose}%
\isadelimproof
%
\endisadelimproof
%
\isatagproof
%
\endisatagproof
{\isafoldproof}%
%
\isadelimproof
%
\endisadelimproof
%
\isadelimtheory
%
\endisadelimtheory
%
\isatagtheory
%
\endisatagtheory
{\isafoldtheory}%
%
\isadelimtheory
%
\endisadelimtheory
\end{isabellebody}%
%%% Local Variables:
%%% mode: latex
%%% TeX-master: "root"
%%% End:
%
\end{exercise}
\begin{exercise}
%
\begin{isabellebody}%
\def\isabellecontext{Tree{\isadigit{2}}}%
\isamarkupfalse%
%
\begin{isamarkuptext}%
\noindent In Exercise~\ref{ex:Tree} we defined a function
\isa{flatten} from trees to lists. The straightforward version of
\isa{flatten} is based on \isa{{\isacharat}} and is thus, like \isa{rev},
quadratic. A linear time version of \isa{flatten} again reqires an extra
argument, the accumulator:%
\end{isamarkuptext}%
\isamarkuptrue%
\isacommand{consts}\ flatten{\isadigit{2}}\ {\isacharcolon}{\isacharcolon}\ {\isachardoublequote}{\isacharprime}a\ tree\ {\isasymRightarrow}\ {\isacharprime}a\ list\ {\isasymRightarrow}\ {\isacharprime}a\ list{\isachardoublequote}\isamarkupfalse%
\isamarkupfalse%
%
\begin{isamarkuptext}%
\noindent Define \isa{flatten{\isadigit{2}}} and prove%
\end{isamarkuptext}%
\isamarkuptrue%
\isamarkupfalse%
\isamarkupfalse%
\isamarkupfalse%
\isacommand{lemma}\ {\isachardoublequote}flatten{\isadigit{2}}\ t\ {\isacharbrackleft}{\isacharbrackright}\ {\isacharequal}\ flatten\ t{\isachardoublequote}\isamarkupfalse%
\isamarkupfalse%
\isamarkupfalse%
\end{isabellebody}%
%%% Local Variables:
%%% mode: latex
%%% TeX-master: "root"
%%% End:
%
\end{exercise}

%
\begin{isabellebody}%
\def\isabellecontext{CodeGen}%
\isamarkupfalse%
%
\isamarkupsection{Case Study: Compiling Expressions%
}
\isamarkuptrue%
%
\begin{isamarkuptext}%
\label{sec:ExprCompiler}
\index{compiling expressions example|(}%
The task is to develop a compiler from a generic type of expressions (built
from variables, constants and binary operations) to a stack machine.  This
generic type of expressions is a generalization of the boolean expressions in
\S\ref{sec:boolex}.  This time we do not commit ourselves to a particular
type of variables or values but make them type parameters.  Neither is there
a fixed set of binary operations: instead the expression contains the
appropriate function itself.%
\end{isamarkuptext}%
\isamarkuptrue%
\isacommand{types}\ {\isacharprime}v\ binop\ {\isacharequal}\ {\isachardoublequote}{\isacharprime}v\ {\isasymRightarrow}\ {\isacharprime}v\ {\isasymRightarrow}\ {\isacharprime}v{\isachardoublequote}\isanewline
\isamarkupfalse%
\isacommand{datatype}\ {\isacharparenleft}{\isacharprime}a{\isacharcomma}{\isacharprime}v{\isacharparenright}expr\ {\isacharequal}\ Cex\ {\isacharprime}v\isanewline
\ \ \ \ \ \ \ \ \ \ \ \ \ \ \ \ \ \ \ \ \ {\isacharbar}\ Vex\ {\isacharprime}a\isanewline
\ \ \ \ \ \ \ \ \ \ \ \ \ \ \ \ \ \ \ \ \ {\isacharbar}\ Bex\ {\isachardoublequote}{\isacharprime}v\ binop{\isachardoublequote}\ \ {\isachardoublequote}{\isacharparenleft}{\isacharprime}a{\isacharcomma}{\isacharprime}v{\isacharparenright}expr{\isachardoublequote}\ \ {\isachardoublequote}{\isacharparenleft}{\isacharprime}a{\isacharcomma}{\isacharprime}v{\isacharparenright}expr{\isachardoublequote}\isamarkupfalse%
%
\begin{isamarkuptext}%
\noindent
The three constructors represent constants, variables and the application of
a binary operation to two subexpressions.

The value of an expression with respect to an environment that maps variables to
values is easily defined:%
\end{isamarkuptext}%
\isamarkuptrue%
\isacommand{consts}\ value\ {\isacharcolon}{\isacharcolon}\ {\isachardoublequote}{\isacharparenleft}{\isacharprime}a{\isacharcomma}{\isacharprime}v{\isacharparenright}expr\ {\isasymRightarrow}\ {\isacharparenleft}{\isacharprime}a\ {\isasymRightarrow}\ {\isacharprime}v{\isacharparenright}\ {\isasymRightarrow}\ {\isacharprime}v{\isachardoublequote}\isanewline
\isamarkupfalse%
\isacommand{primrec}\isanewline
{\isachardoublequote}value\ {\isacharparenleft}Cex\ v{\isacharparenright}\ env\ {\isacharequal}\ v{\isachardoublequote}\isanewline
{\isachardoublequote}value\ {\isacharparenleft}Vex\ a{\isacharparenright}\ env\ {\isacharequal}\ env\ a{\isachardoublequote}\isanewline
{\isachardoublequote}value\ {\isacharparenleft}Bex\ f\ e{\isadigit{1}}\ e{\isadigit{2}}{\isacharparenright}\ env\ {\isacharequal}\ f\ {\isacharparenleft}value\ e{\isadigit{1}}\ env{\isacharparenright}\ {\isacharparenleft}value\ e{\isadigit{2}}\ env{\isacharparenright}{\isachardoublequote}\isamarkupfalse%
%
\begin{isamarkuptext}%
The stack machine has three instructions: load a constant value onto the
stack, load the contents of an address onto the stack, and apply a
binary operation to the two topmost elements of the stack, replacing them by
the result. As for \isa{expr}, addresses and values are type parameters:%
\end{isamarkuptext}%
\isamarkuptrue%
\isacommand{datatype}\ {\isacharparenleft}{\isacharprime}a{\isacharcomma}{\isacharprime}v{\isacharparenright}\ instr\ {\isacharequal}\ Const\ {\isacharprime}v\isanewline
\ \ \ \ \ \ \ \ \ \ \ \ \ \ \ \ \ \ \ \ \ \ \ {\isacharbar}\ Load\ {\isacharprime}a\isanewline
\ \ \ \ \ \ \ \ \ \ \ \ \ \ \ \ \ \ \ \ \ \ \ {\isacharbar}\ Apply\ {\isachardoublequote}{\isacharprime}v\ binop{\isachardoublequote}\isamarkupfalse%
%
\begin{isamarkuptext}%
The execution of the stack machine is modelled by a function
\isa{exec} that takes a list of instructions, a store (modelled as a
function from addresses to values, just like the environment for
evaluating expressions), and a stack (modelled as a list) of values,
and returns the stack at the end of the execution --- the store remains
unchanged:%
\end{isamarkuptext}%
\isamarkuptrue%
\isacommand{consts}\ exec\ {\isacharcolon}{\isacharcolon}\ {\isachardoublequote}{\isacharparenleft}{\isacharprime}a{\isacharcomma}{\isacharprime}v{\isacharparenright}instr\ list\ {\isasymRightarrow}\ {\isacharparenleft}{\isacharprime}a{\isasymRightarrow}{\isacharprime}v{\isacharparenright}\ {\isasymRightarrow}\ {\isacharprime}v\ list\ {\isasymRightarrow}\ {\isacharprime}v\ list{\isachardoublequote}\isanewline
\isamarkupfalse%
\isacommand{primrec}\isanewline
{\isachardoublequote}exec\ {\isacharbrackleft}{\isacharbrackright}\ s\ vs\ {\isacharequal}\ vs{\isachardoublequote}\isanewline
{\isachardoublequote}exec\ {\isacharparenleft}i{\isacharhash}is{\isacharparenright}\ s\ vs\ {\isacharequal}\ {\isacharparenleft}case\ i\ of\isanewline
\ \ \ \ Const\ v\ \ {\isasymRightarrow}\ exec\ is\ s\ {\isacharparenleft}v{\isacharhash}vs{\isacharparenright}\isanewline
\ \ {\isacharbar}\ Load\ a\ \ \ {\isasymRightarrow}\ exec\ is\ s\ {\isacharparenleft}{\isacharparenleft}s\ a{\isacharparenright}{\isacharhash}vs{\isacharparenright}\isanewline
\ \ {\isacharbar}\ Apply\ f\ \ {\isasymRightarrow}\ exec\ is\ s\ {\isacharparenleft}{\isacharparenleft}f\ {\isacharparenleft}hd\ vs{\isacharparenright}\ {\isacharparenleft}hd{\isacharparenleft}tl\ vs{\isacharparenright}{\isacharparenright}{\isacharparenright}{\isacharhash}{\isacharparenleft}tl{\isacharparenleft}tl\ vs{\isacharparenright}{\isacharparenright}{\isacharparenright}{\isacharparenright}{\isachardoublequote}\isamarkupfalse%
%
\begin{isamarkuptext}%
\noindent
Recall that \isa{hd} and \isa{tl}
return the first element and the remainder of a list.
Because all functions are total, \cdx{hd} is defined even for the empty
list, although we do not know what the result is. Thus our model of the
machine always terminates properly, although the definition above does not
tell us much about the result in situations where \isa{Apply} was executed
with fewer than two elements on the stack.

The compiler is a function from expressions to a list of instructions. Its
definition is obvious:%
\end{isamarkuptext}%
\isamarkuptrue%
\isacommand{consts}\ comp\ {\isacharcolon}{\isacharcolon}\ {\isachardoublequote}{\isacharparenleft}{\isacharprime}a{\isacharcomma}{\isacharprime}v{\isacharparenright}expr\ {\isasymRightarrow}\ {\isacharparenleft}{\isacharprime}a{\isacharcomma}{\isacharprime}v{\isacharparenright}instr\ list{\isachardoublequote}\isanewline
\isamarkupfalse%
\isacommand{primrec}\isanewline
{\isachardoublequote}comp\ {\isacharparenleft}Cex\ v{\isacharparenright}\ \ \ \ \ \ \ {\isacharequal}\ {\isacharbrackleft}Const\ v{\isacharbrackright}{\isachardoublequote}\isanewline
{\isachardoublequote}comp\ {\isacharparenleft}Vex\ a{\isacharparenright}\ \ \ \ \ \ \ {\isacharequal}\ {\isacharbrackleft}Load\ a{\isacharbrackright}{\isachardoublequote}\isanewline
{\isachardoublequote}comp\ {\isacharparenleft}Bex\ f\ e{\isadigit{1}}\ e{\isadigit{2}}{\isacharparenright}\ {\isacharequal}\ {\isacharparenleft}comp\ e{\isadigit{2}}{\isacharparenright}\ {\isacharat}\ {\isacharparenleft}comp\ e{\isadigit{1}}{\isacharparenright}\ {\isacharat}\ {\isacharbrackleft}Apply\ f{\isacharbrackright}{\isachardoublequote}\isamarkupfalse%
%
\begin{isamarkuptext}%
Now we have to prove the correctness of the compiler, i.e.\ that the
execution of a compiled expression results in the value of the expression:%
\end{isamarkuptext}%
\isamarkuptrue%
\isacommand{theorem}\ {\isachardoublequote}exec\ {\isacharparenleft}comp\ e{\isacharparenright}\ s\ {\isacharbrackleft}{\isacharbrackright}\ {\isacharequal}\ {\isacharbrackleft}value\ e\ s{\isacharbrackright}{\isachardoublequote}\isanewline
\isamarkupfalse%
\isamarkupfalse%
%
\begin{isamarkuptext}%
\noindent
This theorem needs to be generalized:%
\end{isamarkuptext}%
\isamarkuptrue%
\isacommand{theorem}\ {\isachardoublequote}{\isasymforall}vs{\isachardot}\ exec\ {\isacharparenleft}comp\ e{\isacharparenright}\ s\ vs\ {\isacharequal}\ {\isacharparenleft}value\ e\ s{\isacharparenright}\ {\isacharhash}\ vs{\isachardoublequote}\isamarkupfalse%
%
\begin{isamarkuptxt}%
\noindent
It will be proved by induction on \isa{e} followed by simplification.  
First, we must prove a lemma about executing the concatenation of two
instruction sequences:%
\end{isamarkuptxt}%
\isamarkuptrue%
\isamarkupfalse%
\isacommand{lemma}\ exec{\isacharunderscore}app{\isacharbrackleft}simp{\isacharbrackright}{\isacharcolon}\isanewline
\ \ {\isachardoublequote}{\isasymforall}vs{\isachardot}\ exec\ {\isacharparenleft}xs{\isacharat}ys{\isacharparenright}\ s\ vs\ {\isacharequal}\ exec\ ys\ s\ {\isacharparenleft}exec\ xs\ s\ vs{\isacharparenright}{\isachardoublequote}\isamarkupfalse%
%
\begin{isamarkuptxt}%
\noindent
This requires induction on \isa{xs} and ordinary simplification for the
base cases. In the induction step, simplification leaves us with a formula
that contains two \isa{case}-expressions over instructions. Thus we add
automatic case splitting, which finishes the proof:%
\end{isamarkuptxt}%
\isamarkuptrue%
\isacommand{apply}{\isacharparenleft}induct{\isacharunderscore}tac\ xs{\isacharcomma}\ simp{\isacharcomma}\ simp\ split{\isacharcolon}\ instr{\isachardot}split{\isacharparenright}\isanewline
\isamarkupfalse%
\isamarkupfalse%
%
\begin{isamarkuptext}%
\noindent
Note that because both \methdx{simp_all} and \methdx{auto} perform simplification, they can
be modified in the same way as \isa{simp}.  Thus the proof can be
rewritten as%
\end{isamarkuptext}%
\isamarkuptrue%
\isamarkupfalse%
\isamarkupfalse%
\isacommand{apply}{\isacharparenleft}induct{\isacharunderscore}tac\ xs{\isacharcomma}\ simp{\isacharunderscore}all\ split{\isacharcolon}\ instr{\isachardot}split{\isacharparenright}\isanewline
\isamarkupfalse%
\isamarkupfalse%
%
\begin{isamarkuptext}%
\noindent
Although this is more compact, it is less clear for the reader of the proof.

We could now go back and prove \isa{exec (comp e) s [] = [value e s]}
merely by simplification with the generalized version we just proved.
However, this is unnecessary because the generalized version fully subsumes
its instance.%
\index{compiling expressions example|)}%
\end{isamarkuptext}%
\isamarkuptrue%
\isamarkupfalse%
\isamarkupfalse%
\isamarkupfalse%
\end{isabellebody}%
%%% Local Variables:
%%% mode: latex
%%% TeX-master: "root"
%%% End:



\section{Advanced Datatypes}
\label{sec:advanced-datatypes}
\index{datatype@\isacommand {datatype} (command)|(}
\index{primrec@\isacommand {primrec} (command)|(}
%|)

This section presents advanced forms of datatypes: mutual and nested
recursion.  A series of examples will culminate in a treatment of the trie
data structure.


\subsection{Mutual Recursion}
\label{sec:datatype-mut-rec}

%
\begin{isabellebody}%
\def\isabellecontext{ABexpr}%
%
\isadelimtheory
%
\endisadelimtheory
%
\isatagtheory
%
\endisatagtheory
{\isafoldtheory}%
%
\isadelimtheory
%
\endisadelimtheory
\isamarkuptrue%
%
\begin{isamarkuptext}%
\index{datatypes!mutually recursive}%
Sometimes it is necessary to define two datatypes that depend on each
other. This is called \textbf{mutual recursion}. As an example consider a
language of arithmetic and boolean expressions where
\begin{itemize}
\item arithmetic expressions contain boolean expressions because there are
  conditional expressions like ``if $m<n$ then $n-m$ else $m-n$'',
  and
\item boolean expressions contain arithmetic expressions because of
  comparisons like ``$m<n$''.
\end{itemize}
In Isabelle this becomes%
\end{isamarkuptext}%
\isamarkupfalse%
\isacommand{datatype}\ {\isacharprime}a\ aexp\ {\isacharequal}\ IF\ \ \ {\isachardoublequote}{\isacharprime}a\ bexp{\isachardoublequote}\ {\isachardoublequote}{\isacharprime}a\ aexp{\isachardoublequote}\ {\isachardoublequote}{\isacharprime}a\ aexp{\isachardoublequote}\isanewline
\ \ \ \ \ \ \ \ \ \ \ \ \ \ \ \ \ {\isacharbar}\ Sum\ \ {\isachardoublequote}{\isacharprime}a\ aexp{\isachardoublequote}\ {\isachardoublequote}{\isacharprime}a\ aexp{\isachardoublequote}\isanewline
\ \ \ \ \ \ \ \ \ \ \ \ \ \ \ \ \ {\isacharbar}\ Diff\ {\isachardoublequote}{\isacharprime}a\ aexp{\isachardoublequote}\ {\isachardoublequote}{\isacharprime}a\ aexp{\isachardoublequote}\isanewline
\ \ \ \ \ \ \ \ \ \ \ \ \ \ \ \ \ {\isacharbar}\ Var\ {\isacharprime}a\isanewline
\ \ \ \ \ \ \ \ \ \ \ \ \ \ \ \ \ {\isacharbar}\ Num\ nat\isanewline
\isakeyword{and}\ \ \ \ \ \ {\isacharprime}a\ bexp\ {\isacharequal}\ Less\ {\isachardoublequote}{\isacharprime}a\ aexp{\isachardoublequote}\ {\isachardoublequote}{\isacharprime}a\ aexp{\isachardoublequote}\isanewline
\ \ \ \ \ \ \ \ \ \ \ \ \ \ \ \ \ {\isacharbar}\ And\ \ {\isachardoublequote}{\isacharprime}a\ bexp{\isachardoublequote}\ {\isachardoublequote}{\isacharprime}a\ bexp{\isachardoublequote}\isanewline
\ \ \ \ \ \ \ \ \ \ \ \ \ \ \ \ \ {\isacharbar}\ Neg\ \ {\isachardoublequote}{\isacharprime}a\ bexp{\isachardoublequote}\isamarkuptrue%
%
\begin{isamarkuptext}%
\noindent
Type \isa{aexp} is similar to \isa{expr} in \S\ref{sec:ExprCompiler},
except that we have added an \isa{IF} constructor,
fixed the values to be of type \isa{nat} and declared the two binary
operations \isa{Sum} and \isa{Diff}.  Boolean
expressions can be arithmetic comparisons, conjunctions and negations.
The semantics is given by two evaluation functions:%
\end{isamarkuptext}%
\isamarkupfalse%
\isacommand{consts}\ \ evala\ {\isacharcolon}{\isacharcolon}\ {\isachardoublequote}{\isacharprime}a\ aexp\ {\isasymRightarrow}\ {\isacharparenleft}{\isacharprime}a\ {\isasymRightarrow}\ nat{\isacharparenright}\ {\isasymRightarrow}\ nat{\isachardoublequote}\isanewline
\ \ \ \ \ \ \ \ evalb\ {\isacharcolon}{\isacharcolon}\ {\isachardoublequote}{\isacharprime}a\ bexp\ {\isasymRightarrow}\ {\isacharparenleft}{\isacharprime}a\ {\isasymRightarrow}\ nat{\isacharparenright}\ {\isasymRightarrow}\ bool{\isachardoublequote}\isamarkuptrue%
%
\begin{isamarkuptext}%
\noindent
Both take an expression and an environment (a mapping from variables \isa{{\isacharprime}a} to values
\isa{nat}) and return its arithmetic/boolean
value. Since the datatypes are mutually recursive, so are functions that
operate on them. Hence they need to be defined in a single \isacommand{primrec}
section:%
\end{isamarkuptext}%
\isamarkupfalse%
\isacommand{primrec}\isanewline
\ \ {\isachardoublequote}evala\ {\isacharparenleft}IF\ b\ a{\isadigit{1}}\ a{\isadigit{2}}{\isacharparenright}\ env\ {\isacharequal}\isanewline
\ \ \ \ \ {\isacharparenleft}if\ evalb\ b\ env\ then\ evala\ a{\isadigit{1}}\ env\ else\ evala\ a{\isadigit{2}}\ env{\isacharparenright}{\isachardoublequote}\isanewline
\ \ {\isachardoublequote}evala\ {\isacharparenleft}Sum\ a{\isadigit{1}}\ a{\isadigit{2}}{\isacharparenright}\ env\ {\isacharequal}\ evala\ a{\isadigit{1}}\ env\ {\isacharplus}\ evala\ a{\isadigit{2}}\ env{\isachardoublequote}\isanewline
\ \ {\isachardoublequote}evala\ {\isacharparenleft}Diff\ a{\isadigit{1}}\ a{\isadigit{2}}{\isacharparenright}\ env\ {\isacharequal}\ evala\ a{\isadigit{1}}\ env\ {\isacharminus}\ evala\ a{\isadigit{2}}\ env{\isachardoublequote}\isanewline
\ \ {\isachardoublequote}evala\ {\isacharparenleft}Var\ v{\isacharparenright}\ env\ {\isacharequal}\ env\ v{\isachardoublequote}\isanewline
\ \ {\isachardoublequote}evala\ {\isacharparenleft}Num\ n{\isacharparenright}\ env\ {\isacharequal}\ n{\isachardoublequote}\isanewline
\isanewline
\ \ {\isachardoublequote}evalb\ {\isacharparenleft}Less\ a{\isadigit{1}}\ a{\isadigit{2}}{\isacharparenright}\ env\ {\isacharequal}\ {\isacharparenleft}evala\ a{\isadigit{1}}\ env\ {\isacharless}\ evala\ a{\isadigit{2}}\ env{\isacharparenright}{\isachardoublequote}\isanewline
\ \ {\isachardoublequote}evalb\ {\isacharparenleft}And\ b{\isadigit{1}}\ b{\isadigit{2}}{\isacharparenright}\ env\ {\isacharequal}\ {\isacharparenleft}evalb\ b{\isadigit{1}}\ env\ {\isasymand}\ evalb\ b{\isadigit{2}}\ env{\isacharparenright}{\isachardoublequote}\isanewline
\ \ {\isachardoublequote}evalb\ {\isacharparenleft}Neg\ b{\isacharparenright}\ env\ {\isacharequal}\ {\isacharparenleft}{\isasymnot}\ evalb\ b\ env{\isacharparenright}{\isachardoublequote}\isamarkuptrue%
%
\begin{isamarkuptext}%
\noindent
In the same fashion we also define two functions that perform substitution:%
\end{isamarkuptext}%
\isamarkupfalse%
\isacommand{consts}\ substa\ {\isacharcolon}{\isacharcolon}\ {\isachardoublequote}{\isacharparenleft}{\isacharprime}a\ {\isasymRightarrow}\ {\isacharprime}b\ aexp{\isacharparenright}\ {\isasymRightarrow}\ {\isacharprime}a\ aexp\ {\isasymRightarrow}\ {\isacharprime}b\ aexp{\isachardoublequote}\isanewline
\ \ \ \ \ \ \ substb\ {\isacharcolon}{\isacharcolon}\ {\isachardoublequote}{\isacharparenleft}{\isacharprime}a\ {\isasymRightarrow}\ {\isacharprime}b\ aexp{\isacharparenright}\ {\isasymRightarrow}\ {\isacharprime}a\ bexp\ {\isasymRightarrow}\ {\isacharprime}b\ bexp{\isachardoublequote}\isamarkuptrue%
%
\begin{isamarkuptext}%
\noindent
The first argument is a function mapping variables to expressions, the
substitution. It is applied to all variables in the second argument. As a
result, the type of variables in the expression may change from \isa{{\isacharprime}a}
to \isa{{\isacharprime}b}. Note that there are only arithmetic and no boolean variables.%
\end{isamarkuptext}%
\isamarkupfalse%
\isacommand{primrec}\isanewline
\ \ {\isachardoublequote}substa\ s\ {\isacharparenleft}IF\ b\ a{\isadigit{1}}\ a{\isadigit{2}}{\isacharparenright}\ {\isacharequal}\isanewline
\ \ \ \ \ IF\ {\isacharparenleft}substb\ s\ b{\isacharparenright}\ {\isacharparenleft}substa\ s\ a{\isadigit{1}}{\isacharparenright}\ {\isacharparenleft}substa\ s\ a{\isadigit{2}}{\isacharparenright}{\isachardoublequote}\isanewline
\ \ {\isachardoublequote}substa\ s\ {\isacharparenleft}Sum\ a{\isadigit{1}}\ a{\isadigit{2}}{\isacharparenright}\ {\isacharequal}\ Sum\ {\isacharparenleft}substa\ s\ a{\isadigit{1}}{\isacharparenright}\ {\isacharparenleft}substa\ s\ a{\isadigit{2}}{\isacharparenright}{\isachardoublequote}\isanewline
\ \ {\isachardoublequote}substa\ s\ {\isacharparenleft}Diff\ a{\isadigit{1}}\ a{\isadigit{2}}{\isacharparenright}\ {\isacharequal}\ Diff\ {\isacharparenleft}substa\ s\ a{\isadigit{1}}{\isacharparenright}\ {\isacharparenleft}substa\ s\ a{\isadigit{2}}{\isacharparenright}{\isachardoublequote}\isanewline
\ \ {\isachardoublequote}substa\ s\ {\isacharparenleft}Var\ v{\isacharparenright}\ {\isacharequal}\ s\ v{\isachardoublequote}\isanewline
\ \ {\isachardoublequote}substa\ s\ {\isacharparenleft}Num\ n{\isacharparenright}\ {\isacharequal}\ Num\ n{\isachardoublequote}\isanewline
\isanewline
\ \ {\isachardoublequote}substb\ s\ {\isacharparenleft}Less\ a{\isadigit{1}}\ a{\isadigit{2}}{\isacharparenright}\ {\isacharequal}\ Less\ {\isacharparenleft}substa\ s\ a{\isadigit{1}}{\isacharparenright}\ {\isacharparenleft}substa\ s\ a{\isadigit{2}}{\isacharparenright}{\isachardoublequote}\isanewline
\ \ {\isachardoublequote}substb\ s\ {\isacharparenleft}And\ b{\isadigit{1}}\ b{\isadigit{2}}{\isacharparenright}\ {\isacharequal}\ And\ {\isacharparenleft}substb\ s\ b{\isadigit{1}}{\isacharparenright}\ {\isacharparenleft}substb\ s\ b{\isadigit{2}}{\isacharparenright}{\isachardoublequote}\isanewline
\ \ {\isachardoublequote}substb\ s\ {\isacharparenleft}Neg\ b{\isacharparenright}\ {\isacharequal}\ Neg\ {\isacharparenleft}substb\ s\ b{\isacharparenright}{\isachardoublequote}\isamarkuptrue%
%
\begin{isamarkuptext}%
Now we can prove a fundamental theorem about the interaction between
evaluation and substitution: applying a substitution $s$ to an expression $a$
and evaluating the result in an environment $env$ yields the same result as
evaluation $a$ in the environment that maps every variable $x$ to the value
of $s(x)$ under $env$. If you try to prove this separately for arithmetic or
boolean expressions (by induction), you find that you always need the other
theorem in the induction step. Therefore you need to state and prove both
theorems simultaneously:%
\end{isamarkuptext}%
\isamarkupfalse%
\isacommand{lemma}\ {\isachardoublequote}evala\ {\isacharparenleft}substa\ s\ a{\isacharparenright}\ env\ {\isacharequal}\ evala\ a\ {\isacharparenleft}{\isasymlambda}x{\isachardot}\ evala\ {\isacharparenleft}s\ x{\isacharparenright}\ env{\isacharparenright}\ {\isasymand}\isanewline
\ \ \ \ \ \ \ \ evalb\ {\isacharparenleft}substb\ s\ b{\isacharparenright}\ env\ {\isacharequal}\ evalb\ b\ {\isacharparenleft}{\isasymlambda}x{\isachardot}\ evala\ {\isacharparenleft}s\ x{\isacharparenright}\ env{\isacharparenright}{\isachardoublequote}\isanewline
%
\isadelimproof
%
\endisadelimproof
%
\isatagproof
\isamarkupfalse%
\isacommand{apply}{\isacharparenleft}induct{\isacharunderscore}tac\ a\ \isakeyword{and}\ b{\isacharparenright}\isamarkuptrue%
%
\begin{isamarkuptxt}%
\noindent
The resulting 8 goals (one for each constructor) are proved in one fell swoop:%
\end{isamarkuptxt}%
\isamarkupfalse%
\isacommand{apply}\ simp{\isacharunderscore}all%
\endisatagproof
{\isafoldproof}%
%
\isadelimproof
%
\endisadelimproof
\isamarkuptrue%
%
\begin{isamarkuptext}%
In general, given $n$ mutually recursive datatypes $\tau@1$, \dots, $\tau@n$,
an inductive proof expects a goal of the form
\[ P@1(x@1)\ \land \dots \land P@n(x@n) \]
where each variable $x@i$ is of type $\tau@i$. Induction is started by
\begin{isabelle}
\isacommand{apply}\isa{{\isacharparenleft}induct{\isacharunderscore}tac} $x@1$ \isacommand{and} \dots\ \isacommand{and} $x@n$\isa{{\isacharparenright}}
\end{isabelle}

\begin{exercise}
  Define a function \isa{norma} of type \isa{{\isacharprime}a\ aexp\ {\isasymRightarrow}\ {\isacharprime}a\ aexp} that
  replaces \isa{IF}s with complex boolean conditions by nested
  \isa{IF}s; it should eliminate the constructors
  \isa{And} and \isa{Neg}, leaving only \isa{Less}.
  Prove that \isa{norma}
  preserves the value of an expression and that the result of \isa{norma}
  is really normal, i.e.\ no more \isa{And}s and \isa{Neg}s occur in
  it.  ({\em Hint:} proceed as in \S\ref{sec:boolex} and read the discussion
  of type annotations following lemma \isa{subst{\isacharunderscore}id} below).
\end{exercise}%
\end{isamarkuptext}%
%
\isadelimproof
%
\endisadelimproof
%
\isatagproof
%
\endisatagproof
{\isafoldproof}%
%
\isadelimproof
%
\endisadelimproof
%
\isadelimproof
%
\endisadelimproof
%
\isatagproof
%
\endisatagproof
{\isafoldproof}%
%
\isadelimproof
%
\endisadelimproof
%
\isadelimtheory
%
\endisadelimtheory
%
\isatagtheory
%
\endisatagtheory
{\isafoldtheory}%
%
\isadelimtheory
%
\endisadelimtheory
\end{isabellebody}%
%%% Local Variables:
%%% mode: latex
%%% TeX-master: "root"
%%% End:


\subsection{Nested Recursion}
\label{sec:nested-datatype}

{\makeatother%
\begin{isabellebody}%
\def\isabellecontext{Nested}%
%
\isadelimtheory
%
\endisadelimtheory
%
\isatagtheory
%
\endisatagtheory
{\isafoldtheory}%
%
\isadelimtheory
%
\endisadelimtheory
%
\begin{isamarkuptext}%
\index{datatypes!and nested recursion}%
So far, all datatypes had the property that on the right-hand side of their
definition they occurred only at the top-level: directly below a
constructor. Now we consider \emph{nested recursion}, where the recursive
datatype occurs nested in some other datatype (but not inside itself!).
Consider the following model of terms
where function symbols can be applied to a list of arguments:%
\end{isamarkuptext}%
\isamarkuptrue%
\isacommand{datatype}\isamarkupfalse%
\ {\isacharparenleft}{\isacharprime}v{\isacharcomma}{\isacharprime}f{\isacharparenright}{\isachardoublequoteopen}term{\isachardoublequoteclose}\ {\isacharequal}\ Var\ {\isacharprime}v\ {\isacharbar}\ App\ {\isacharprime}f\ {\isachardoublequoteopen}{\isacharparenleft}{\isacharprime}v{\isacharcomma}{\isacharprime}f{\isacharparenright}term\ list{\isachardoublequoteclose}%
\begin{isamarkuptext}%
\noindent
Note that we need to quote \isa{term} on the left to avoid confusion with
the Isabelle command \isacommand{term}.
Parameter \isa{{\isacharprime}v} is the type of variables and \isa{{\isacharprime}f} the type of
function symbols.
A mathematical term like $f(x,g(y))$ becomes \isa{App\ f\ {\isacharbrackleft}Var\ x{\isacharcomma}\ App\ g\ {\isacharbrackleft}Var\ y{\isacharbrackright}{\isacharbrackright}}, where \isa{f}, \isa{g}, \isa{x}, \isa{y} are
suitable values, e.g.\ numbers or strings.

What complicates the definition of \isa{term} is the nested occurrence of
\isa{term} inside \isa{list} on the right-hand side. In principle,
nested recursion can be eliminated in favour of mutual recursion by unfolding
the offending datatypes, here \isa{list}. The result for \isa{term}
would be something like
\medskip

\begin{isabelle}%
\isacommand{datatype}\ {\isacharparenleft}{\isacharprime}a{\isacharcomma}{\isacharprime}b{\isacharparenright}{\isachardoublequote}term{\isachardoublequote}\ {\isacharequal}\ Var\ {\isacharprime}a\ {\isacharbar}\ App\ {\isacharprime}b\ {\isachardoublequote}{\isacharparenleft}{\isacharprime}a{\isacharcomma}{\isacharprime}b{\isacharparenright}term{\isacharunderscore}list{\isachardoublequote}\isanewline
\isakeyword{and}\ {\isacharparenleft}{\isacharprime}a{\isacharcomma}{\isacharprime}b{\isacharparenright}term{\isacharunderscore}list\ {\isacharequal}\ Nil\ {\isacharbar}\ Cons\ {\isachardoublequote}{\isacharparenleft}{\isacharprime}a{\isacharcomma}{\isacharprime}b{\isacharparenright}term{\isachardoublequote}\ {\isachardoublequote}{\isacharparenleft}{\isacharprime}a{\isacharcomma}{\isacharprime}b{\isacharparenright}term{\isacharunderscore}list{\isachardoublequote}\end{isabelle}%
%%% Local Variables:
%%% mode: latex
%%% TeX-master: "root"
%%% End:

\medskip

\noindent
Although we do not recommend this unfolding to the user, it shows how to
simulate nested recursion by mutual recursion.
Now we return to the initial definition of \isa{term} using
nested recursion.

Let us define a substitution function on terms. Because terms involve term
lists, we need to define two substitution functions simultaneously:%
\end{isamarkuptext}%
\isamarkuptrue%
\isacommand{primrec}\isamarkupfalse%
\isanewline
subst\ {\isacharcolon}{\isacharcolon}\ {\isachardoublequoteopen}{\isacharparenleft}{\isacharprime}v{\isasymRightarrow}{\isacharparenleft}{\isacharprime}v{\isacharcomma}{\isacharprime}f{\isacharparenright}term{\isacharparenright}\ {\isasymRightarrow}\ {\isacharparenleft}{\isacharprime}v{\isacharcomma}{\isacharprime}f{\isacharparenright}term\ \ \ \ \ \ {\isasymRightarrow}\ {\isacharparenleft}{\isacharprime}v{\isacharcomma}{\isacharprime}f{\isacharparenright}term{\isachardoublequoteclose}\ \isakeyword{and}\isanewline
substs{\isacharcolon}{\isacharcolon}\ {\isachardoublequoteopen}{\isacharparenleft}{\isacharprime}v{\isasymRightarrow}{\isacharparenleft}{\isacharprime}v{\isacharcomma}{\isacharprime}f{\isacharparenright}term{\isacharparenright}\ {\isasymRightarrow}\ {\isacharparenleft}{\isacharprime}v{\isacharcomma}{\isacharprime}f{\isacharparenright}term\ list\ {\isasymRightarrow}\ {\isacharparenleft}{\isacharprime}v{\isacharcomma}{\isacharprime}f{\isacharparenright}term\ list{\isachardoublequoteclose}\isanewline
\isakeyword{where}\isanewline
{\isachardoublequoteopen}subst\ s\ {\isacharparenleft}Var\ x{\isacharparenright}\ {\isacharequal}\ s\ x{\isachardoublequoteclose}\ {\isacharbar}\isanewline
\ \ subst{\isacharunderscore}App{\isacharcolon}\isanewline
{\isachardoublequoteopen}subst\ s\ {\isacharparenleft}App\ f\ ts{\isacharparenright}\ {\isacharequal}\ App\ f\ {\isacharparenleft}substs\ s\ ts{\isacharparenright}{\isachardoublequoteclose}\ {\isacharbar}\isanewline
\isanewline
{\isachardoublequoteopen}substs\ s\ {\isacharbrackleft}{\isacharbrackright}\ {\isacharequal}\ {\isacharbrackleft}{\isacharbrackright}{\isachardoublequoteclose}\ {\isacharbar}\isanewline
{\isachardoublequoteopen}substs\ s\ {\isacharparenleft}t\ {\isacharhash}\ ts{\isacharparenright}\ {\isacharequal}\ subst\ s\ t\ {\isacharhash}\ substs\ s\ ts{\isachardoublequoteclose}%
\begin{isamarkuptext}%
\noindent
Individual equations in a \commdx{primrec} definition may be
named as shown for \isa{subst{\isacharunderscore}App}.
The significance of this device will become apparent below.

Similarly, when proving a statement about terms inductively, we need
to prove a related statement about term lists simultaneously. For example,
the fact that the identity substitution does not change a term needs to be
strengthened and proved as follows:%
\end{isamarkuptext}%
\isamarkuptrue%
\isacommand{lemma}\isamarkupfalse%
\ subst{\isacharunderscore}id{\isacharcolon}\ {\isachardoublequoteopen}subst\ \ Var\ t\ \ {\isacharequal}\ {\isacharparenleft}t\ {\isacharcolon}{\isacharcolon}{\isacharparenleft}{\isacharprime}v{\isacharcomma}{\isacharprime}f{\isacharparenright}term{\isacharparenright}\ \ {\isasymand}\isanewline
\ \ \ \ \ \ \ \ \ \ \ \ \ \ \ \ \ \ substs\ Var\ ts\ {\isacharequal}\ {\isacharparenleft}ts{\isacharcolon}{\isacharcolon}{\isacharparenleft}{\isacharprime}v{\isacharcomma}{\isacharprime}f{\isacharparenright}term\ list{\isacharparenright}{\isachardoublequoteclose}\isanewline
%
\isadelimproof
%
\endisadelimproof
%
\isatagproof
\isacommand{apply}\isamarkupfalse%
{\isacharparenleft}induct{\isacharunderscore}tac\ t\ \isakeyword{and}\ ts{\isacharcomma}\ simp{\isacharunderscore}all{\isacharparenright}\isanewline
\isacommand{done}\isamarkupfalse%
%
\endisatagproof
{\isafoldproof}%
%
\isadelimproof
%
\endisadelimproof
%
\begin{isamarkuptext}%
\noindent
Note that \isa{Var} is the identity substitution because by definition it
leaves variables unchanged: \isa{subst\ Var\ {\isacharparenleft}Var\ x{\isacharparenright}\ {\isacharequal}\ Var\ x}. Note also
that the type annotations are necessary because otherwise there is nothing in
the goal to enforce that both halves of the goal talk about the same type
parameters \isa{{\isacharparenleft}{\isacharprime}v{\isacharcomma}{\isacharprime}f{\isacharparenright}}. As a result, induction would fail
because the two halves of the goal would be unrelated.

\begin{exercise}
The fact that substitution distributes over composition can be expressed
roughly as follows:
\begin{isabelle}%
\ \ \ \ \ subst\ {\isacharparenleft}f\ {\isasymcirc}\ g{\isacharparenright}\ t\ {\isacharequal}\ subst\ f\ {\isacharparenleft}subst\ g\ t{\isacharparenright}%
\end{isabelle}
Correct this statement (you will find that it does not type-check),
strengthen it, and prove it. (Note: \isa{{\isasymcirc}} is function composition;
its definition is found in theorem \isa{o{\isacharunderscore}def}).
\end{exercise}
\begin{exercise}\label{ex:trev-trev}
  Define a function \isa{trev} of type \isa{{\isacharparenleft}{\isacharprime}v{\isacharcomma}\ {\isacharprime}f{\isacharparenright}\ Nested{\isachardot}term\ {\isasymRightarrow}\ {\isacharparenleft}{\isacharprime}v{\isacharcomma}\ {\isacharprime}f{\isacharparenright}\ Nested{\isachardot}term}
that recursively reverses the order of arguments of all function symbols in a
  term. Prove that \isa{trev\ {\isacharparenleft}trev\ t{\isacharparenright}\ {\isacharequal}\ t}.
\end{exercise}

The experienced functional programmer may feel that our definition of
\isa{subst} is too complicated in that \isa{substs} is
unnecessary. The \isa{App}-case can be defined directly as
\begin{isabelle}%
\ \ \ \ \ subst\ s\ {\isacharparenleft}App\ f\ ts{\isacharparenright}\ {\isacharequal}\ App\ f\ {\isacharparenleft}map\ {\isacharparenleft}subst\ s{\isacharparenright}\ ts{\isacharparenright}%
\end{isabelle}
where \isa{map} is the standard list function such that
\isa{map\ f\ {\isacharbrackleft}x{\isadigit{1}}{\isacharcomma}{\isachardot}{\isachardot}{\isachardot}{\isacharcomma}xn{\isacharbrackright}\ {\isacharequal}\ {\isacharbrackleft}f\ x{\isadigit{1}}{\isacharcomma}{\isachardot}{\isachardot}{\isachardot}{\isacharcomma}f\ xn{\isacharbrackright}}. This is true, but Isabelle
insists on the conjunctive format. Fortunately, we can easily \emph{prove}
that the suggested equation holds:%
\end{isamarkuptext}%
\isamarkuptrue%
%
\isadelimproof
%
\endisadelimproof
%
\isatagproof
%
\endisatagproof
{\isafoldproof}%
%
\isadelimproof
%
\endisadelimproof
%
\isadelimproof
%
\endisadelimproof
%
\isatagproof
%
\endisatagproof
{\isafoldproof}%
%
\isadelimproof
%
\endisadelimproof
%
\isadelimproof
%
\endisadelimproof
%
\isatagproof
%
\endisatagproof
{\isafoldproof}%
%
\isadelimproof
\isanewline
%
\endisadelimproof
\isacommand{lemma}\isamarkupfalse%
\ {\isacharbrackleft}simp{\isacharbrackright}{\isacharcolon}\ {\isachardoublequoteopen}subst\ s\ {\isacharparenleft}App\ f\ ts{\isacharparenright}\ {\isacharequal}\ App\ f\ {\isacharparenleft}map\ {\isacharparenleft}subst\ s{\isacharparenright}\ ts{\isacharparenright}{\isachardoublequoteclose}\isanewline
%
\isadelimproof
%
\endisadelimproof
%
\isatagproof
\isacommand{apply}\isamarkupfalse%
{\isacharparenleft}induct{\isacharunderscore}tac\ ts{\isacharcomma}\ simp{\isacharunderscore}all{\isacharparenright}\isanewline
\isacommand{done}\isamarkupfalse%
%
\endisatagproof
{\isafoldproof}%
%
\isadelimproof
%
\endisadelimproof
%
\begin{isamarkuptext}%
\noindent
What is more, we can now disable the old defining equation as a
simplification rule:%
\end{isamarkuptext}%
\isamarkuptrue%
\isacommand{declare}\isamarkupfalse%
\ subst{\isacharunderscore}App\ {\isacharbrackleft}simp\ del{\isacharbrackright}%
\begin{isamarkuptext}%
\noindent The advantage is that now we have replaced \isa{substs} by \isa{map}, we can profit from the large number of
pre-proved lemmas about \isa{map}.  Unfortunately, inductive proofs
about type \isa{term} are still awkward because they expect a
conjunction. One could derive a new induction principle as well (see
\S\ref{sec:derive-ind}), but simpler is to stop using
\isacommand{primrec} and to define functions with \isacommand{fun}
instead.  Simple uses of \isacommand{fun} are described in
\S\ref{sec:fun} below.  Advanced applications, including functions
over nested datatypes like \isa{term}, are discussed in a
separate tutorial~\cite{isabelle-function}.

Of course, you may also combine mutual and nested recursion of datatypes. For example,
constructor \isa{Sum} in \S\ref{sec:datatype-mut-rec} could take a list of
expressions as its argument: \isa{Sum}~\isa{{\isachardoublequote}{\isacharprime}a\ aexp\ list{\isachardoublequote}}.%
\end{isamarkuptext}%
\isamarkuptrue%
%
\isadelimtheory
%
\endisadelimtheory
%
\isatagtheory
%
\endisatagtheory
{\isafoldtheory}%
%
\isadelimtheory
%
\endisadelimtheory
\end{isabellebody}%
%%% Local Variables:
%%% mode: latex
%%% TeX-master: "root"
%%% End:
}


\subsection{The Limits of Nested Recursion}
\label{sec:nested-fun-datatype}

How far can we push nested recursion? By the unfolding argument above, we can
reduce nested to mutual recursion provided the nested recursion only involves
previously defined datatypes. This does not include functions:
\begin{isabelle}
\isacommand{datatype} t = C "t \isasymRightarrow\ bool"
\end{isabelle}
This declaration is a real can of worms.
In HOL it must be ruled out because it requires a type
\isa{t} such that \isa{t} and its power set \isa{t \isasymFun\ bool} have the
same cardinality --- an impossibility. For the same reason it is not possible
to allow recursion involving the type \isa{t set}, which is isomorphic to
\isa{t \isasymFun\ bool}.

Fortunately, a limited form of recursion
involving function spaces is permitted: the recursive type may occur on the
right of a function arrow, but never on the left. Hence the above can of worms
is ruled out but the following example of a potentially 
\index{infinitely branching trees}%
infinitely branching tree is accepted:
\smallskip

%
\begin{isabellebody}%
\def\isabellecontext{Fundata}%
\isamarkupfalse%
\isacommand{datatype}\ {\isacharparenleft}{\isacharprime}a{\isacharcomma}{\isacharprime}i{\isacharparenright}bigtree\ {\isacharequal}\ Tip\ {\isacharbar}\ Br\ {\isacharprime}a\ {\isachardoublequote}{\isacharprime}i\ {\isasymRightarrow}\ {\isacharparenleft}{\isacharprime}a{\isacharcomma}{\isacharprime}i{\isacharparenright}bigtree{\isachardoublequote}\isamarkupfalse%
%
\begin{isamarkuptext}%
\noindent
Parameter \isa{{\isacharprime}a} is the type of values stored in
the \isa{Br}anches of the tree, whereas \isa{{\isacharprime}i} is the index
type over which the tree branches. If \isa{{\isacharprime}i} is instantiated to
\isa{bool}, the result is a binary tree; if it is instantiated to
\isa{nat}, we have an infinitely branching tree because each node
has as many subtrees as there are natural numbers. How can we possibly
write down such a tree? Using functional notation! For example, the term
\begin{isabelle}%
\ \ \ \ \ Br\ {\isacharparenleft}{\isadigit{0}}{\isasymColon}{\isacharprime}a{\isacharparenright}\ {\isacharparenleft}{\isasymlambda}i{\isachardot}\ Br\ i\ {\isacharparenleft}{\isasymlambda}n{\isachardot}\ Tip{\isacharparenright}{\isacharparenright}%
\end{isabelle}
of type \isa{{\isacharparenleft}nat{\isacharcomma}\ nat{\isacharparenright}\ bigtree} is the tree whose
root is labeled with 0 and whose $i$th subtree is labeled with $i$ and
has merely \isa{Tip}s as further subtrees.

Function \isa{map{\isacharunderscore}bt} applies a function to all labels in a \isa{bigtree}:%
\end{isamarkuptext}%
\isamarkuptrue%
\isacommand{consts}\ map{\isacharunderscore}bt\ {\isacharcolon}{\isacharcolon}\ {\isachardoublequote}{\isacharparenleft}{\isacharprime}a\ {\isasymRightarrow}\ {\isacharprime}b{\isacharparenright}\ {\isasymRightarrow}\ {\isacharparenleft}{\isacharprime}a{\isacharcomma}{\isacharprime}i{\isacharparenright}bigtree\ {\isasymRightarrow}\ {\isacharparenleft}{\isacharprime}b{\isacharcomma}{\isacharprime}i{\isacharparenright}bigtree{\isachardoublequote}\isanewline
\isamarkupfalse%
\isacommand{primrec}\isanewline
{\isachardoublequote}map{\isacharunderscore}bt\ f\ Tip\ \ \ \ \ \ {\isacharequal}\ Tip{\isachardoublequote}\isanewline
{\isachardoublequote}map{\isacharunderscore}bt\ f\ {\isacharparenleft}Br\ a\ F{\isacharparenright}\ {\isacharequal}\ Br\ {\isacharparenleft}f\ a{\isacharparenright}\ {\isacharparenleft}{\isasymlambda}i{\isachardot}\ map{\isacharunderscore}bt\ f\ {\isacharparenleft}F\ i{\isacharparenright}{\isacharparenright}{\isachardoublequote}\isamarkupfalse%
%
\begin{isamarkuptext}%
\noindent This is a valid \isacommand{primrec} definition because the
recursive calls of \isa{map{\isacharunderscore}bt} involve only subtrees obtained from
\isa{F}: the left-hand side. Thus termination is assured.  The
seasoned functional programmer might try expressing
\isa{{\isasymlambda}i{\isachardot}\ map{\isacharunderscore}bt\ f\ {\isacharparenleft}F\ i{\isacharparenright}} as \isa{map{\isacharunderscore}bt\ f\ {\isasymcirc}\ F}, which Isabelle 
however will reject.  Applying \isa{map{\isacharunderscore}bt} to only one of its arguments
makes the termination proof less obvious.

The following lemma has a simple proof by induction:%
\end{isamarkuptext}%
\isamarkuptrue%
\isacommand{lemma}\ {\isachardoublequote}map{\isacharunderscore}bt\ {\isacharparenleft}g\ o\ f{\isacharparenright}\ T\ {\isacharequal}\ map{\isacharunderscore}bt\ g\ {\isacharparenleft}map{\isacharunderscore}bt\ f\ T{\isacharparenright}{\isachardoublequote}\isanewline
\isamarkupfalse%
\isacommand{apply}{\isacharparenleft}induct{\isacharunderscore}tac\ T{\isacharcomma}\ simp{\isacharunderscore}all{\isacharparenright}\isanewline
\isamarkupfalse%
\isacommand{done}\isamarkupfalse%
\isamarkupfalse%
\isamarkupfalse%
%
\begin{isamarkuptxt}%
\noindent
Because of the function type, the 
the proof state after induction looks unusual.
Notice the quantified induction hypothesis:
\begin{isabelle}%
\ {\isadigit{1}}{\isachardot}\ map{\isacharunderscore}bt\ {\isacharparenleft}g\ {\isasymcirc}\ f{\isacharparenright}\ Tip\ {\isacharequal}\ map{\isacharunderscore}bt\ g\ {\isacharparenleft}map{\isacharunderscore}bt\ f\ Tip{\isacharparenright}\isanewline
\ {\isadigit{2}}{\isachardot}\ {\isasymAnd}a\ F{\isachardot}\ {\isacharparenleft}{\isasymAnd}x{\isachardot}\ map{\isacharunderscore}bt\ {\isacharparenleft}g\ {\isasymcirc}\ f{\isacharparenright}\ {\isacharparenleft}F\ x{\isacharparenright}\ {\isacharequal}\ map{\isacharunderscore}bt\ g\ {\isacharparenleft}map{\isacharunderscore}bt\ f\ {\isacharparenleft}F\ x{\isacharparenright}{\isacharparenright}{\isacharparenright}\ {\isasymLongrightarrow}\isanewline
\isaindent{\ {\isadigit{2}}{\isachardot}\ {\isasymAnd}a\ F{\isachardot}\ }map{\isacharunderscore}bt\ {\isacharparenleft}g\ {\isasymcirc}\ f{\isacharparenright}\ {\isacharparenleft}Br\ a\ F{\isacharparenright}\ {\isacharequal}\ map{\isacharunderscore}bt\ g\ {\isacharparenleft}map{\isacharunderscore}bt\ f\ {\isacharparenleft}Br\ a\ F{\isacharparenright}{\isacharparenright}%
\end{isabelle}%
\end{isamarkuptxt}%
\isamarkuptrue%
\isamarkupfalse%
\isamarkupfalse%
\end{isabellebody}%
%%% Local Variables:
%%% mode: latex
%%% TeX-master: "root"
%%% End:


If you need nested recursion on the left of a function arrow, there are
alternatives to pure HOL\@.  In the Logic for Computable Functions 
(\rmindex{LCF}), types like
\begin{isabelle}
\isacommand{datatype} lam = C "lam \isasymrightarrow\ lam"
\end{isabelle}
do indeed make sense~\cite{paulson87}.  Note the different arrow,
\isa{\isasymrightarrow} instead of \isa{\isasymRightarrow},
expressing the type of \emph{continuous} functions. 
There is even a version of LCF on top of HOL,
called \rmindex{HOLCF}~\cite{MuellerNvOS99}.
\index{datatype@\isacommand {datatype} (command)|)}
\index{primrec@\isacommand {primrec} (command)|)}


\subsection{Case Study: Tries}
\label{sec:Trie}

\index{tries|(}%
Tries are a classic search tree data structure~\cite{Knuth3-75} for fast
indexing with strings. Figure~\ref{fig:trie} gives a graphical example of a
trie containing the words ``all'', ``an'', ``ape'', ``can'', ``car'' and
``cat''.  When searching a string in a trie, the letters of the string are
examined sequentially. Each letter determines which subtrie to search next.
In this case study we model tries as a datatype, define a lookup and an
update function, and prove that they behave as expected.

\begin{figure}[htbp]
\begin{center}
\unitlength1mm
\begin{picture}(60,30)
\put( 5, 0){\makebox(0,0)[b]{l}}
\put(25, 0){\makebox(0,0)[b]{e}}
\put(35, 0){\makebox(0,0)[b]{n}}
\put(45, 0){\makebox(0,0)[b]{r}}
\put(55, 0){\makebox(0,0)[b]{t}}
%
\put( 5, 9){\line(0,-1){5}}
\put(25, 9){\line(0,-1){5}}
\put(44, 9){\line(-3,-2){9}}
\put(45, 9){\line(0,-1){5}}
\put(46, 9){\line(3,-2){9}}
%
\put( 5,10){\makebox(0,0)[b]{l}}
\put(15,10){\makebox(0,0)[b]{n}}
\put(25,10){\makebox(0,0)[b]{p}}
\put(45,10){\makebox(0,0)[b]{a}}
%
\put(14,19){\line(-3,-2){9}}
\put(15,19){\line(0,-1){5}}
\put(16,19){\line(3,-2){9}}
\put(45,19){\line(0,-1){5}}
%
\put(15,20){\makebox(0,0)[b]{a}}
\put(45,20){\makebox(0,0)[b]{c}}
%
\put(30,30){\line(-3,-2){13}}
\put(30,30){\line(3,-2){13}}
\end{picture}
\end{center}
\caption{A Sample Trie}
\label{fig:trie}
\end{figure}

Proper tries associate some value with each string. Since the
information is stored only in the final node associated with the string, many
nodes do not carry any value. This distinction is modeled with the help
of the predefined datatype \isa{option} (see {\S}\ref{sec:option}).
\begin{isabelle}%
%
\begin{isamarkuptext}%
To minimize running time, each node of a trie should contain an array that maps
letters to subtries. We have chosen a (sometimes) more space efficient
representation where the subtries are held in an association list, i.e.\ a
list of (letter,trie) pairs.  Abstracting over the alphabet \isa{'a} and the
values \isa{'v} we define a trie as follows:%
\end{isamarkuptext}%
\isacommand{datatype}~('a,'v)trie~=~Trie~~{"}'v~option{"}~~{"}('a~*~('a,'v)trie)list{"}%
\begin{isamarkuptext}%
\noindent
The first component is the optional value, the second component the
association list of subtries.  This is an example of nested recursion involving products,
which is fine because products are datatypes as well.
We define two selector functions:%
\end{isamarkuptext}%
\isacommand{consts}~value~::~{"}('a,'v)trie~{\isasymRightarrow}~'v~option{"}\isanewline
~~~~~~~alist~::~{"}('a,'v)trie~{\isasymRightarrow}~('a~*~('a,'v)trie)list{"}\isanewline
\isacommand{primrec}~{"}value(Trie~ov~al)~=~ov{"}\isanewline
\isacommand{primrec}~{"}alist(Trie~ov~al)~=~al{"}%
\begin{isamarkuptext}%
\noindent
Association lists come with a generic lookup function:%
\end{isamarkuptext}%
\isacommand{consts}~~~assoc~::~{"}('key~*~'val)list~{\isasymRightarrow}~'key~{\isasymRightarrow}~'val~option{"}\isanewline
\isacommand{primrec}~{"}assoc~[]~x~=~None{"}\isanewline
~~~~~~~~{"}assoc~(p\#ps)~x~=\isanewline
~~~~~~~~~~~(let~(a,b)~=~p~in~if~a=x~then~Some~b~else~assoc~ps~x){"}%
\begin{isamarkuptext}%
Now we can define the lookup function for tries. It descends into the trie
examining the letters of the search string one by one. As
recursion on lists is simpler than on tries, let us express this as primitive
recursion on the search string argument:%
\end{isamarkuptext}%
\isacommand{consts}~~~lookup~::~{"}('a,'v)trie~{\isasymRightarrow}~'a~list~{\isasymRightarrow}~'v~option{"}\isanewline
\isacommand{primrec}~{"}lookup~t~[]~=~value~t{"}\isanewline
~~~~~~~~{"}lookup~t~(a\#as)~=~(case~assoc~(alist~t)~a~of\isanewline
~~~~~~~~~~~~~~~~~~~~~~~~~~~~~~None~{\isasymRightarrow}~None\isanewline
~~~~~~~~~~~~~~~~~~~~~~~~~~~~|~Some~at~{\isasymRightarrow}~lookup~at~as){"}%
\begin{isamarkuptext}%
As a first simple property we prove that looking up a string in the empty
trie \isa{Trie~None~[]} always returns \isa{None}. The proof merely
distinguishes the two cases whether the search string is empty or not:%
\end{isamarkuptext}%
\isacommand{lemma}~[simp]:~{"}lookup~(Trie~None~[])~as~=~None{"}\isanewline
\isacommand{apply}(case\_tac~as,~auto)\isacommand{.}%
\begin{isamarkuptext}%
Things begin to get interesting with the definition of an update function
that adds a new (string,value) pair to a trie, overwriting the old value
associated with that string:%
\end{isamarkuptext}%
\isacommand{consts}~update~::~{"}('a,'v)trie~{\isasymRightarrow}~'a~list~{\isasymRightarrow}~'v~{\isasymRightarrow}~('a,'v)trie{"}\isanewline
\isacommand{primrec}\isanewline
~~{"}update~t~[]~~~~~v~=~Trie~(Some~v)~(alist~t){"}\isanewline
~~{"}update~t~(a\#as)~v~=\isanewline
~~~~~(let~tt~=~(case~assoc~(alist~t)~a~of\isanewline
~~~~~~~~~~~~~~~~~~None~{\isasymRightarrow}~Trie~None~[]~|~Some~at~{\isasymRightarrow}~at)\isanewline
~~~~~~in~Trie~(value~t)~((a,update~tt~as~v)\#alist~t)){"}%
\begin{isamarkuptext}%
\noindent
The base case is obvious. In the recursive case the subtrie
\isa{tt} associated with the first letter \isa{a} is extracted,
recursively updated, and then placed in front of the association list.
The old subtrie associated with \isa{a} is still in the association list
but no longer accessible via \isa{assoc}. Clearly, there is room here for
optimizations!

Before we start on any proofs about \isa{update} we tell the simplifier to
expand all \isa{let}s and to split all \isa{case}-constructs over
options:%
\end{isamarkuptext}%
\isacommand{theorems}~[simp]~=~Let\_def\isanewline
\isacommand{theorems}~[split]~=~option.split%
\begin{isamarkuptext}%
\noindent
The reason becomes clear when looking (probably after a failed proof
attempt) at the body of \isa{update}: it contains both
\isa{let} and a case distinction over type \isa{option}.

Our main goal is to prove the correct interaction of \isa{update} and
\isa{lookup}:%
\end{isamarkuptext}%
\isacommand{theorem}~{"}{\isasymforall}t~v~bs.~lookup~(update~t~as~v)~bs~=\isanewline
~~~~~~~~~~~~~~~~~~~~(if~as=bs~then~Some~v~else~lookup~t~bs){"}%
\begin{isamarkuptxt}%
\noindent
Our plan is to induct on \isa{as}; hence the remaining variables are
quantified. From the definitions it is clear that induction on either
\isa{as} or \isa{bs} is required. The choice of \isa{as} is merely
guided by the intuition that simplification of \isa{lookup} might be easier
if \isa{update} has already been simplified, which can only happen if
\isa{as} is instantiated.
The start of the proof is completely conventional:%
\end{isamarkuptxt}%
\isacommand{apply}(induct\_tac~as,~auto)%
\begin{isamarkuptxt}%
\noindent
Unfortunately, this time we are left with three intimidating looking subgoals:
\begin{isabellepar}%
~1.~\dots~{\isasymLongrightarrow}~lookup~\dots~bs~=~lookup~t~bs\isanewline
~2.~\dots~{\isasymLongrightarrow}~lookup~\dots~bs~=~lookup~t~bs\isanewline
~3.~\dots~{\isasymLongrightarrow}~lookup~\dots~bs~=~lookup~t~bs%
\end{isabellepar}%
Clearly, if we want to make headway we have to instantiate \isa{bs} as
well now. It turns out that instead of induction, case distinction
suffices:%
\end{isamarkuptxt}%
\isacommand{apply}(case\_tac[!]~bs)\isanewline
\isacommand{apply}(auto)\isacommand{.}%
\begin{isamarkuptext}%
\noindent
Both \isaindex{case_tac} and \isaindex{induct_tac}
take an optional first argument that specifies the range of subgoals they are
applied to, where \isa{!} means all subgoals, i.e.\ \isa{[1-3]} in our case. Individual
subgoal numbers are also allowed.

This proof may look surprisingly straightforward. However, note that this
comes at a cost: the proof script is unreadable because the
intermediate proof states are invisible, and we rely on the (possibly
brittle) magic of \isa{auto} (after the induction) to split the remaining
goals up in such a way that case distinction on \isa{bs} makes sense and
solves the proof. Part~\ref{Isar} shows you how to write readable and stable
proofs.%
\end{isamarkuptext}%
\end{isabelle}%

\index{tries|)}

\section{Total Recursive Functions: \isacommand{fun}}
\label{sec:fun}
\index{fun@\isacommand {fun} (command)|(}\index{functions!total|(}

Although many total functions have a natural primitive recursive definition,
this is not always the case. Arbitrary total recursive functions can be
defined by means of \isacommand{fun}: you can use full pattern matching,
recursion need not involve datatypes, and termination is proved by showing
that the arguments of all recursive calls are smaller in a suitable sense.
In this section we restrict ourselves to functions where Isabelle can prove
termination automatically. More advanced function definitions, including user
supplied termination proofs, nested recursion and partiality, are discussed
in a separate tutorial~\cite{isabelle-function}.

%
\begin{isabellebody}%
\def\isabellecontext{fun{\isadigit{0}}}%
%
\isadelimtheory
%
\endisadelimtheory
%
\isatagtheory
%
\endisatagtheory
{\isafoldtheory}%
%
\isadelimtheory
%
\endisadelimtheory
%
\begin{isamarkuptext}%
\subsection{Definition}
\label{sec:fun-examples}

Here is a simple example, the \rmindex{Fibonacci function}:%
\end{isamarkuptext}%
\isamarkuptrue%
\isacommand{fun}\isamarkupfalse%
\ fib\ {\isacharcolon}{\isacharcolon}\ {\isachardoublequoteopen}nat\ {\isasymRightarrow}\ nat{\isachardoublequoteclose}\ \isakeyword{where}\isanewline
\ \ {\isachardoublequoteopen}fib\ {\isadigit{0}}\ {\isacharequal}\ {\isadigit{0}}{\isachardoublequoteclose}\ {\isacharbar}\isanewline
\ \ {\isachardoublequoteopen}fib\ {\isacharparenleft}Suc\ {\isadigit{0}}{\isacharparenright}\ {\isacharequal}\ {\isadigit{1}}{\isachardoublequoteclose}\ {\isacharbar}\isanewline
\ \ {\isachardoublequoteopen}fib\ {\isacharparenleft}Suc{\isacharparenleft}Suc\ x{\isacharparenright}{\isacharparenright}\ {\isacharequal}\ fib\ x\ {\isacharplus}\ fib\ {\isacharparenleft}Suc\ x{\isacharparenright}{\isachardoublequoteclose}%
\begin{isamarkuptext}%
\noindent
This resembles ordinary functional programming languages. Note the obligatory
\isacommand{where} and \isa{|}. Command \isacommand{fun} declares and
defines the function in one go. Isabelle establishes termination automatically
because \isa{fib}'s argument decreases in every recursive call.

Slightly more interesting is the insertion of a fixed element
between any two elements of a list:%
\end{isamarkuptext}%
\isamarkuptrue%
\isacommand{fun}\isamarkupfalse%
\ sep\ {\isacharcolon}{\isacharcolon}\ {\isachardoublequoteopen}{\isacharprime}a\ {\isasymRightarrow}\ {\isacharprime}a\ list\ {\isasymRightarrow}\ {\isacharprime}a\ list{\isachardoublequoteclose}\ \isakeyword{where}\isanewline
\ \ {\isachardoublequoteopen}sep\ a\ {\isacharbrackleft}{\isacharbrackright}\ \ \ \ \ {\isacharequal}\ {\isacharbrackleft}{\isacharbrackright}{\isachardoublequoteclose}\ {\isacharbar}\isanewline
\ \ {\isachardoublequoteopen}sep\ a\ {\isacharbrackleft}x{\isacharbrackright}\ \ \ \ {\isacharequal}\ {\isacharbrackleft}x{\isacharbrackright}{\isachardoublequoteclose}\ {\isacharbar}\isanewline
\ \ {\isachardoublequoteopen}sep\ a\ {\isacharparenleft}x{\isacharhash}y{\isacharhash}zs{\isacharparenright}\ {\isacharequal}\ x\ {\isacharhash}\ a\ {\isacharhash}\ sep\ a\ {\isacharparenleft}y{\isacharhash}zs{\isacharparenright}{\isachardoublequoteclose}%
\begin{isamarkuptext}%
\noindent
This time the length of the list decreases with the
recursive call; the first argument is irrelevant for termination.

Pattern matching\index{pattern matching!and \isacommand{fun}}
need not be exhaustive and may employ wildcards:%
\end{isamarkuptext}%
\isamarkuptrue%
\isacommand{fun}\isamarkupfalse%
\ last\ {\isacharcolon}{\isacharcolon}\ {\isachardoublequoteopen}{\isacharprime}a\ list\ {\isasymRightarrow}\ {\isacharprime}a{\isachardoublequoteclose}\ \isakeyword{where}\isanewline
\ \ {\isachardoublequoteopen}last\ {\isacharbrackleft}x{\isacharbrackright}\ \ \ \ \ \ {\isacharequal}\ x{\isachardoublequoteclose}\ {\isacharbar}\isanewline
\ \ {\isachardoublequoteopen}last\ {\isacharparenleft}{\isacharunderscore}{\isacharhash}y{\isacharhash}zs{\isacharparenright}\ {\isacharequal}\ last\ {\isacharparenleft}y{\isacharhash}zs{\isacharparenright}{\isachardoublequoteclose}%
\begin{isamarkuptext}%
Overlapping patterns are disambiguated by taking the order of equations into
account, just as in functional programming:%
\end{isamarkuptext}%
\isamarkuptrue%
\isacommand{fun}\isamarkupfalse%
\ sep{\isadigit{1}}\ {\isacharcolon}{\isacharcolon}\ {\isachardoublequoteopen}{\isacharprime}a\ {\isasymRightarrow}\ {\isacharprime}a\ list\ {\isasymRightarrow}\ {\isacharprime}a\ list{\isachardoublequoteclose}\ \isakeyword{where}\isanewline
\ \ {\isachardoublequoteopen}sep{\isadigit{1}}\ a\ {\isacharparenleft}x{\isacharhash}y{\isacharhash}zs{\isacharparenright}\ {\isacharequal}\ x\ {\isacharhash}\ a\ {\isacharhash}\ sep{\isadigit{1}}\ a\ {\isacharparenleft}y{\isacharhash}zs{\isacharparenright}{\isachardoublequoteclose}\ {\isacharbar}\isanewline
\ \ {\isachardoublequoteopen}sep{\isadigit{1}}\ {\isacharunderscore}\ xs\ \ \ \ \ \ \ {\isacharequal}\ xs{\isachardoublequoteclose}%
\begin{isamarkuptext}%
\noindent
To guarantee that the second equation can only be applied if the first
one does not match, Isabelle internally replaces the second equation
by the two possibilities that are left: \isa{sep{\isadigit{1}}\ a\ {\isacharbrackleft}{\isacharbrackright}\ {\isacharequal}\ {\isacharbrackleft}{\isacharbrackright}} and
\isa{sep{\isadigit{1}}\ a\ {\isacharbrackleft}x{\isacharbrackright}\ {\isacharequal}\ {\isacharbrackleft}x{\isacharbrackright}}.  Thus the functions \isa{sep} and
\isa{sep{\isadigit{1}}} are identical.

Because of its pattern matching syntax, \isacommand{fun} is also useful
for the definition of non-recursive functions:%
\end{isamarkuptext}%
\isamarkuptrue%
\isacommand{fun}\isamarkupfalse%
\ swap{\isadigit{1}}{\isadigit{2}}\ {\isacharcolon}{\isacharcolon}\ {\isachardoublequoteopen}{\isacharprime}a\ list\ {\isasymRightarrow}\ {\isacharprime}a\ list{\isachardoublequoteclose}\ \isakeyword{where}\isanewline
\ \ {\isachardoublequoteopen}swap{\isadigit{1}}{\isadigit{2}}\ {\isacharparenleft}x{\isacharhash}y{\isacharhash}zs{\isacharparenright}\ {\isacharequal}\ y{\isacharhash}x{\isacharhash}zs{\isachardoublequoteclose}\ {\isacharbar}\isanewline
\ \ {\isachardoublequoteopen}swap{\isadigit{1}}{\isadigit{2}}\ zs\ \ \ \ \ \ \ {\isacharequal}\ zs{\isachardoublequoteclose}%
\begin{isamarkuptext}%
After a function~$f$ has been defined via \isacommand{fun},
its defining equations (or variants derived from them) are available
under the name $f$\isa{{\isachardot}simps} as theorems.
For example, look (via \isacommand{thm}) at
\isa{sep{\isachardot}simps} and \isa{sep{\isadigit{1}}{\isachardot}simps} to see that they define
the same function. What is more, those equations are automatically declared as
simplification rules.

\subsection{Termination}

Isabelle's automatic termination prover for \isacommand{fun} has a
fixed notion of the \emph{size} (of type \isa{nat}) of an
argument. The size of a natural number is the number itself. The size
of a list is its length. For the general case see \S\ref{sec:general-datatype}.
A recursive function is accepted if \isacommand{fun} can
show that the size of one fixed argument becomes smaller with each
recursive call.

More generally, \isacommand{fun} allows any \emph{lexicographic
combination} of size measures in case there are multiple
arguments. For example, the following version of \rmindex{Ackermann's
function} is accepted:%
\end{isamarkuptext}%
\isamarkuptrue%
\isacommand{fun}\isamarkupfalse%
\ ack{\isadigit{2}}\ {\isacharcolon}{\isacharcolon}\ {\isachardoublequoteopen}nat\ {\isasymRightarrow}\ nat\ {\isasymRightarrow}\ nat{\isachardoublequoteclose}\ \isakeyword{where}\isanewline
\ \ {\isachardoublequoteopen}ack{\isadigit{2}}\ n\ {\isadigit{0}}\ {\isacharequal}\ Suc\ n{\isachardoublequoteclose}\ {\isacharbar}\isanewline
\ \ {\isachardoublequoteopen}ack{\isadigit{2}}\ {\isadigit{0}}\ {\isacharparenleft}Suc\ m{\isacharparenright}\ {\isacharequal}\ ack{\isadigit{2}}\ {\isacharparenleft}Suc\ {\isadigit{0}}{\isacharparenright}\ m{\isachardoublequoteclose}\ {\isacharbar}\isanewline
\ \ {\isachardoublequoteopen}ack{\isadigit{2}}\ {\isacharparenleft}Suc\ n{\isacharparenright}\ {\isacharparenleft}Suc\ m{\isacharparenright}\ {\isacharequal}\ ack{\isadigit{2}}\ {\isacharparenleft}ack{\isadigit{2}}\ n\ {\isacharparenleft}Suc\ m{\isacharparenright}{\isacharparenright}\ m{\isachardoublequoteclose}%
\begin{isamarkuptext}%
The order of arguments has no influence on whether
\isacommand{fun} can prove termination of a function. For more details
see elsewhere~\cite{bulwahnKN07}.

\subsection{Simplification}
\label{sec:fun-simplification}

Upon a successful termination proof, the recursion equations become
simplification rules, just as with \isacommand{primrec}.
In most cases this works fine, but there is a subtle
problem that must be mentioned: simplification may not
terminate because of automatic splitting of \isa{if}.
\index{*if expressions!splitting of}
Let us look at an example:%
\end{isamarkuptext}%
\isamarkuptrue%
\isacommand{fun}\isamarkupfalse%
\ gcd\ {\isacharcolon}{\isacharcolon}\ {\isachardoublequoteopen}nat\ {\isasymRightarrow}\ nat\ {\isasymRightarrow}\ nat{\isachardoublequoteclose}\ \isakeyword{where}\isanewline
\ \ {\isachardoublequoteopen}gcd\ m\ n\ {\isacharequal}\ {\isacharparenleft}if\ n{\isacharequal}{\isadigit{0}}\ then\ m\ else\ gcd\ n\ {\isacharparenleft}m\ mod\ n{\isacharparenright}{\isacharparenright}{\isachardoublequoteclose}%
\begin{isamarkuptext}%
\noindent
The second argument decreases with each recursive call.
The termination condition
\begin{isabelle}%
\ \ \ \ \ n\ {\isasymnoteq}\ {\isadigit{0}}\ {\isasymLongrightarrow}\ m\ mod\ n\ {\isacharless}\ n%
\end{isabelle}
is proved automatically because it is already present as a lemma in
HOL\@.  Thus the recursion equation becomes a simplification
rule. Of course the equation is nonterminating if we are allowed to unfold
the recursive call inside the \isa{else} branch, which is why programming
languages and our simplifier don't do that. Unfortunately the simplifier does
something else that leads to the same problem: it splits 
each \isa{if}-expression unless its
condition simplifies to \isa{True} or \isa{False}.  For
example, simplification reduces
\begin{isabelle}%
\ \ \ \ \ gcd\ m\ n\ {\isacharequal}\ k%
\end{isabelle}
in one step to
\begin{isabelle}%
\ \ \ \ \ {\isacharparenleft}if\ n\ {\isacharequal}\ {\isadigit{0}}\ then\ m\ else\ gcd\ n\ {\isacharparenleft}m\ mod\ n{\isacharparenright}{\isacharparenright}\ {\isacharequal}\ k%
\end{isabelle}
where the condition cannot be reduced further, and splitting leads to
\begin{isabelle}%
\ \ \ \ \ {\isacharparenleft}n\ {\isacharequal}\ {\isadigit{0}}\ {\isasymlongrightarrow}\ m\ {\isacharequal}\ k{\isacharparenright}\ {\isasymand}\ {\isacharparenleft}n\ {\isasymnoteq}\ {\isadigit{0}}\ {\isasymlongrightarrow}\ gcd\ n\ {\isacharparenleft}m\ mod\ n{\isacharparenright}\ {\isacharequal}\ k{\isacharparenright}%
\end{isabelle}
Since the recursive call \isa{gcd\ n\ {\isacharparenleft}m\ mod\ n{\isacharparenright}} is no longer protected by
an \isa{if}, it is unfolded again, which leads to an infinite chain of
simplification steps. Fortunately, this problem can be avoided in many
different ways.

The most radical solution is to disable the offending theorem
\isa{split{\isacharunderscore}if},
as shown in \S\ref{sec:AutoCaseSplits}.  However, we do not recommend this
approach: you will often have to invoke the rule explicitly when
\isa{if} is involved.

If possible, the definition should be given by pattern matching on the left
rather than \isa{if} on the right. In the case of \isa{gcd} the
following alternative definition suggests itself:%
\end{isamarkuptext}%
\isamarkuptrue%
\isacommand{fun}\isamarkupfalse%
\ gcd{\isadigit{1}}\ {\isacharcolon}{\isacharcolon}\ {\isachardoublequoteopen}nat\ {\isasymRightarrow}\ nat\ {\isasymRightarrow}\ nat{\isachardoublequoteclose}\ \isakeyword{where}\isanewline
\ \ {\isachardoublequoteopen}gcd{\isadigit{1}}\ m\ {\isadigit{0}}\ {\isacharequal}\ m{\isachardoublequoteclose}\ {\isacharbar}\isanewline
\ \ {\isachardoublequoteopen}gcd{\isadigit{1}}\ m\ n\ {\isacharequal}\ gcd{\isadigit{1}}\ n\ {\isacharparenleft}m\ mod\ n{\isacharparenright}{\isachardoublequoteclose}%
\begin{isamarkuptext}%
\noindent
The order of equations is important: it hides the side condition
\isa{n\ {\isasymnoteq}\ {\isadigit{0}}}.  Unfortunately, not all conditionals can be
expressed by pattern matching.

A simple alternative is to replace \isa{if} by \isa{case}, 
which is also available for \isa{bool} and is not split automatically:%
\end{isamarkuptext}%
\isamarkuptrue%
\isacommand{fun}\isamarkupfalse%
\ gcd{\isadigit{2}}\ {\isacharcolon}{\isacharcolon}\ {\isachardoublequoteopen}nat\ {\isasymRightarrow}\ nat\ {\isasymRightarrow}\ nat{\isachardoublequoteclose}\ \isakeyword{where}\isanewline
\ \ {\isachardoublequoteopen}gcd{\isadigit{2}}\ m\ n\ {\isacharequal}\ {\isacharparenleft}case\ n{\isacharequal}{\isadigit{0}}\ of\ True\ {\isasymRightarrow}\ m\ {\isacharbar}\ False\ {\isasymRightarrow}\ gcd{\isadigit{2}}\ n\ {\isacharparenleft}m\ mod\ n{\isacharparenright}{\isacharparenright}{\isachardoublequoteclose}%
\begin{isamarkuptext}%
\noindent
This is probably the neatest solution next to pattern matching, and it is
always available.

A final alternative is to replace the offending simplification rules by
derived conditional ones. For \isa{gcd} it means we have to prove
these lemmas:%
\end{isamarkuptext}%
\isamarkuptrue%
\isacommand{lemma}\isamarkupfalse%
\ {\isacharbrackleft}simp{\isacharbrackright}{\isacharcolon}\ {\isachardoublequoteopen}gcd\ m\ {\isadigit{0}}\ {\isacharequal}\ m{\isachardoublequoteclose}\isanewline
%
\isadelimproof
%
\endisadelimproof
%
\isatagproof
\isacommand{apply}\isamarkupfalse%
{\isacharparenleft}simp{\isacharparenright}\isanewline
\isacommand{done}\isamarkupfalse%
%
\endisatagproof
{\isafoldproof}%
%
\isadelimproof
\isanewline
%
\endisadelimproof
\isanewline
\isacommand{lemma}\isamarkupfalse%
\ {\isacharbrackleft}simp{\isacharbrackright}{\isacharcolon}\ {\isachardoublequoteopen}n\ {\isasymnoteq}\ {\isadigit{0}}\ {\isasymLongrightarrow}\ gcd\ m\ n\ {\isacharequal}\ gcd\ n\ {\isacharparenleft}m\ mod\ n{\isacharparenright}{\isachardoublequoteclose}\isanewline
%
\isadelimproof
%
\endisadelimproof
%
\isatagproof
\isacommand{apply}\isamarkupfalse%
{\isacharparenleft}simp{\isacharparenright}\isanewline
\isacommand{done}\isamarkupfalse%
%
\endisatagproof
{\isafoldproof}%
%
\isadelimproof
%
\endisadelimproof
%
\begin{isamarkuptext}%
\noindent
Simplification terminates for these proofs because the condition of the \isa{if} simplifies to \isa{True} or \isa{False}.
Now we can disable the original simplification rule:%
\end{isamarkuptext}%
\isamarkuptrue%
\isacommand{declare}\isamarkupfalse%
\ gcd{\isachardot}simps\ {\isacharbrackleft}simp\ del{\isacharbrackright}%
\begin{isamarkuptext}%
\index{induction!recursion|(}
\index{recursion induction|(}

\subsection{Induction}
\label{sec:fun-induction}

Having defined a function we might like to prove something about it.
Since the function is recursive, the natural proof principle is
again induction. But this time the structural form of induction that comes
with datatypes is unlikely to work well --- otherwise we could have defined the
function by \isacommand{primrec}. Therefore \isacommand{fun} automatically
proves a suitable induction rule $f$\isa{{\isachardot}induct} that follows the
recursion pattern of the particular function $f$. We call this
\textbf{recursion induction}. Roughly speaking, it
requires you to prove for each \isacommand{fun} equation that the property
you are trying to establish holds for the left-hand side provided it holds
for all recursive calls on the right-hand side. Here is a simple example
involving the predefined \isa{map} functional on lists:%
\end{isamarkuptext}%
\isamarkuptrue%
\isacommand{lemma}\isamarkupfalse%
\ {\isachardoublequoteopen}map\ f\ {\isacharparenleft}sep\ x\ xs{\isacharparenright}\ {\isacharequal}\ sep\ {\isacharparenleft}f\ x{\isacharparenright}\ {\isacharparenleft}map\ f\ xs{\isacharparenright}{\isachardoublequoteclose}%
\isadelimproof
%
\endisadelimproof
%
\isatagproof
%
\begin{isamarkuptxt}%
\noindent
Note that \isa{map\ f\ xs}
is the result of applying \isa{f} to all elements of \isa{xs}. We prove
this lemma by recursion induction over \isa{sep}:%
\end{isamarkuptxt}%
\isamarkuptrue%
\isacommand{apply}\isamarkupfalse%
{\isacharparenleft}induct{\isacharunderscore}tac\ x\ xs\ rule{\isacharcolon}\ sep{\isachardot}induct{\isacharparenright}%
\begin{isamarkuptxt}%
\noindent
The resulting proof state has three subgoals corresponding to the three
clauses for \isa{sep}:
\begin{isabelle}%
\ {\isadigit{1}}{\isachardot}\ {\isasymAnd}a{\isachardot}\ map\ f\ {\isacharparenleft}sep\ a\ {\isacharbrackleft}{\isacharbrackright}{\isacharparenright}\ {\isacharequal}\ sep\ {\isacharparenleft}f\ a{\isacharparenright}\ {\isacharparenleft}map\ f\ {\isacharbrackleft}{\isacharbrackright}{\isacharparenright}\isanewline
\ {\isadigit{2}}{\isachardot}\ {\isasymAnd}a\ x{\isachardot}\ map\ f\ {\isacharparenleft}sep\ a\ {\isacharbrackleft}x{\isacharbrackright}{\isacharparenright}\ {\isacharequal}\ sep\ {\isacharparenleft}f\ a{\isacharparenright}\ {\isacharparenleft}map\ f\ {\isacharbrackleft}x{\isacharbrackright}{\isacharparenright}\isanewline
\ {\isadigit{3}}{\isachardot}\ {\isasymAnd}a\ x\ y\ zs{\isachardot}\isanewline
\isaindent{\ {\isadigit{3}}{\isachardot}\ \ \ \ }map\ f\ {\isacharparenleft}sep\ a\ {\isacharparenleft}y\ {\isacharhash}\ zs{\isacharparenright}{\isacharparenright}\ {\isacharequal}\ sep\ {\isacharparenleft}f\ a{\isacharparenright}\ {\isacharparenleft}map\ f\ {\isacharparenleft}y\ {\isacharhash}\ zs{\isacharparenright}{\isacharparenright}\ {\isasymLongrightarrow}\isanewline
\isaindent{\ {\isadigit{3}}{\isachardot}\ \ \ \ }map\ f\ {\isacharparenleft}sep\ a\ {\isacharparenleft}x\ {\isacharhash}\ y\ {\isacharhash}\ zs{\isacharparenright}{\isacharparenright}\ {\isacharequal}\ sep\ {\isacharparenleft}f\ a{\isacharparenright}\ {\isacharparenleft}map\ f\ {\isacharparenleft}x\ {\isacharhash}\ y\ {\isacharhash}\ zs{\isacharparenright}{\isacharparenright}%
\end{isabelle}
The rest is pure simplification:%
\end{isamarkuptxt}%
\isamarkuptrue%
\isacommand{apply}\isamarkupfalse%
\ simp{\isacharunderscore}all\isanewline
\isacommand{done}\isamarkupfalse%
%
\endisatagproof
{\isafoldproof}%
%
\isadelimproof
%
\endisadelimproof
%
\begin{isamarkuptext}%
\noindent The proof goes smoothly because the induction rule
follows the recursion of \isa{sep}.  Try proving the above lemma by
structural induction, and you find that you need an additional case
distinction.

In general, the format of invoking recursion induction is
\begin{quote}
\isacommand{apply}\isa{{\isacharparenleft}induct{\isacharunderscore}tac} $x@1 \dots x@n$ \isa{rule{\isacharcolon}} $f$\isa{{\isachardot}induct{\isacharparenright}}
\end{quote}\index{*induct_tac (method)}%
where $x@1~\dots~x@n$ is a list of free variables in the subgoal and $f$ the
name of a function that takes an $n$ arguments. Usually the subgoal will
contain the term $f x@1 \dots x@n$ but this need not be the case. The
induction rules do not mention $f$ at all. Here is \isa{sep{\isachardot}induct}:
\begin{isabelle}
{\isasymlbrakk}~{\isasymAnd}a.~P~a~[];\isanewline
~~{\isasymAnd}a~x.~P~a~[x];\isanewline
~~{\isasymAnd}a~x~y~zs.~P~a~(y~\#~zs)~{\isasymLongrightarrow}~P~a~(x~\#~y~\#~zs){\isasymrbrakk}\isanewline
{\isasymLongrightarrow}~P~u~v%
\end{isabelle}
It merely says that in order to prove a property \isa{P} of \isa{u} and
\isa{v} you need to prove it for the three cases where \isa{v} is the
empty list, the singleton list, and the list with at least two elements.
The final case has an induction hypothesis:  you may assume that \isa{P}
holds for the tail of that list.
\index{induction!recursion|)}
\index{recursion induction|)}%
\end{isamarkuptext}%
\isamarkuptrue%
%
\isadelimtheory
%
\endisadelimtheory
%
\isatagtheory
%
\endisatagtheory
{\isafoldtheory}%
%
\isadelimtheory
%
\endisadelimtheory
\end{isabellebody}%
%%% Local Variables:
%%% mode: latex
%%% TeX-master: "root"
%%% End:


\index{fun@\isacommand {fun} (command)|)}\index{functions!total|)}
