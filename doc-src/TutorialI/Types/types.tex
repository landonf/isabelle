\chapter{More about Types}

So far we have learned about a few basic types (for example \isa{bool} and
\isa{nat}), type abbreviations (\isacommand{types}) and recursive datatpes
(\isacommand{datatype}). This chapter will introduce the following more
advanced material:
\begin{itemize}
\item More about basic types: numbers ({\S}\ref{sec:numbers}), pairs
  ({\S}\ref{sec:products}) and records ({\S}\ref{sec:records}), and how to reason
  about them.
\item Introducing your own types: how to introduce your own new types that
  cannot be constructed with any of the basic methods ({\S}\ref{sec:typedef}).
\item Type classes: how to specify and reason about axiomatic collections of
  types ({\S}\ref{sec:axclass}).
\end{itemize}

\section{Axiomatic type classes}
\label{sec:axclass}
\index{axiomatic type class|(}
\index{*axclass|(}


The programming language Haskell has popularized the notion of type classes.
Isabelle offers the related concept of an \textbf{axiomatic type class}.
Roughly speaking, an axiomatic type class is a type class with axioms, i.e.\ 
an axiomatic specification of a class of types. Thus we can talk about a type
$t$ being in a class $c$, which is written $\tau :: c$.  This is the case of
$\tau$ satisfies the axioms of $c$. Furthermore, type classes can be
organized in a hierarchy. Thus there is the notion of a class $d$ being a
\textbf{subclass} of a class $c$, written $d < c$. This is the case if all
axioms of $c$ are also provable in $d$. Let us now introduce these concepts
by means of a running example, ordering relations.

\subsection{Overloading}
\label{sec:overloading}
\index{overloading|(}

%
\begin{isabellebody}%
\def\isabellecontext{Overloading{\isadigit{0}}}%
%
\begin{isamarkuptext}%
We start with a concept that is required for type classes but already
useful on its own: \emph{overloading}. Isabelle allows overloading: a
constant may have multiple definitions at non-overlapping types.%
\end{isamarkuptext}%
%
\isamarkupsubsubsection{An Initial Example%
}
%
\begin{isamarkuptext}%
If we want to introduce the notion of an \emph{inverse} for arbitrary types we
give it a polymorphic type%
\end{isamarkuptext}%
\isacommand{consts}\ inverse\ {\isacharcolon}{\isacharcolon}\ {\isachardoublequote}{\isacharprime}a\ {\isasymRightarrow}\ {\isacharprime}a{\isachardoublequote}%
\begin{isamarkuptext}%
\noindent
and provide different definitions at different instances:%
\end{isamarkuptext}%
\isacommand{defs}\ {\isacharparenleft}\isakeyword{overloaded}{\isacharparenright}\isanewline
inverse{\isacharunderscore}bool{\isacharcolon}\ {\isachardoublequote}inverse{\isacharparenleft}b{\isacharcolon}{\isacharcolon}bool{\isacharparenright}\ {\isasymequiv}\ {\isasymnot}\ b{\isachardoublequote}\isanewline
inverse{\isacharunderscore}set{\isacharcolon}\ \ {\isachardoublequote}inverse{\isacharparenleft}A{\isacharcolon}{\isacharcolon}{\isacharprime}a\ set{\isacharparenright}\ {\isasymequiv}\ {\isacharminus}A{\isachardoublequote}\isanewline
inverse{\isacharunderscore}pair{\isacharcolon}\ {\isachardoublequote}inverse{\isacharparenleft}p{\isacharparenright}\ {\isasymequiv}\ {\isacharparenleft}inverse{\isacharparenleft}fst\ p{\isacharparenright}{\isacharcomma}\ inverse{\isacharparenleft}snd\ p{\isacharparenright}{\isacharparenright}{\isachardoublequote}%
\begin{isamarkuptext}%
\noindent
Isabelle will not complain because the three definitions do not overlap: no
two of the three types \isa{bool}, \isa{{\isacharprime}a\ set} and \isa{{\isacharprime}a\ {\isasymtimes}\ {\isacharprime}b} have a
common instance. What is more, the recursion in \isa{inverse{\isacharunderscore}pair} is
benign because the type of \isa{inverse} becomes smaller: on the
left it is \isa{{\isacharprime}a\ {\isasymtimes}\ {\isacharprime}b\ {\isasymRightarrow}\ {\isacharprime}a\ {\isasymtimes}\ {\isacharprime}b} but on the right \isa{{\isacharprime}a\ {\isasymRightarrow}\ {\isacharprime}a} and
\isa{{\isacharprime}b\ {\isasymRightarrow}\ {\isacharprime}b}. The annotation \isa{{\isacharparenleft}}\isacommand{overloaded}\isa{{\isacharparenright}} tells Isabelle that
the definitions do intentionally define \isa{inverse} only at
instances of its declared type \isa{{\isacharprime}a\ {\isasymRightarrow}\ {\isacharprime}a} --- this merely supresses
warnings to that effect.

However, there is nothing to prevent the user from forming terms such as
\isa{inverse\ {\isacharbrackleft}{\isacharbrackright}} and proving theorems as \isa{inverse\ {\isacharbrackleft}{\isacharbrackright}\ {\isacharequal}\ inverse\ {\isacharbrackleft}{\isacharbrackright}}, although we never defined inverse on lists. We hasten to say
that there is nothing wrong with such terms and theorems. But it would be
nice if we could prevent their formation, simply because it is very likely
that the user did not mean to write what he did. Thus she had better not waste
her time pursuing it further. This requires the use of type classes.%
\end{isamarkuptext}%
\end{isabellebody}%
%%% Local Variables:
%%% mode: latex
%%% TeX-master: "root"
%%% End:

%
\begin{isabellebody}%
\def\isabellecontext{Overloading{\isadigit{1}}}%
%
\isadelimtheory
%
\endisadelimtheory
%
\isatagtheory
%
\endisatagtheory
{\isafoldtheory}%
%
\isadelimtheory
%
\endisadelimtheory
%
\isamarkupsubsubsection{Controlled Overloading with Type Classes%
}
\isamarkuptrue%
%
\begin{isamarkuptext}%
We now start with the theory of ordering relations, which we shall phrase
in terms of the two binary symbols \isa{{\isacharless}{\isacharless}} and \isa{{\isacharless}{\isacharless}{\isacharequal}}
to avoid clashes with \isa{{\isacharless}} and \isa{{\isacharless}{\isacharequal}} in theory \isa{Main}. To restrict the application of \isa{{\isacharless}{\isacharless}} and \isa{{\isacharless}{\isacharless}{\isacharequal}} we
introduce the class \isa{ordrel}:%
\end{isamarkuptext}%
\isamarkuptrue%
\isacommand{axclass}\isamarkupfalse%
\ ordrel\ {\isacharless}\ type%
\begin{isamarkuptext}%
\noindent
This introduces a new class \isa{ordrel} and makes it a subclass of
the predefined class \isa{type}, which
is the class of all HOL types.
This is a degenerate form of axiomatic type class without any axioms.
Its sole purpose is to restrict the use of overloaded constants to meaningful
instances:%
\end{isamarkuptext}%
\isamarkuptrue%
\isacommand{consts}\isamarkupfalse%
\ lt\ {\isacharcolon}{\isacharcolon}\ {\isachardoublequoteopen}{\isacharparenleft}{\isacharprime}a{\isacharcolon}{\isacharcolon}ordrel{\isacharparenright}\ {\isasymRightarrow}\ {\isacharprime}a\ {\isasymRightarrow}\ bool{\isachardoublequoteclose}\ \ \ \ \ {\isacharparenleft}\isakeyword{infixl}\ {\isachardoublequoteopen}{\isacharless}{\isacharless}{\isachardoublequoteclose}\ \ {\isadigit{5}}{\isadigit{0}}{\isacharparenright}\isanewline
\ \ \ \ \ \ \ le\ {\isacharcolon}{\isacharcolon}\ {\isachardoublequoteopen}{\isacharparenleft}{\isacharprime}a{\isacharcolon}{\isacharcolon}ordrel{\isacharparenright}\ {\isasymRightarrow}\ {\isacharprime}a\ {\isasymRightarrow}\ bool{\isachardoublequoteclose}\ \ \ \ \ {\isacharparenleft}\isakeyword{infixl}\ {\isachardoublequoteopen}{\isacharless}{\isacharless}{\isacharequal}{\isachardoublequoteclose}\ {\isadigit{5}}{\isadigit{0}}{\isacharparenright}%
\begin{isamarkuptext}%
\noindent
Note that only one occurrence of a type variable in a type needs to be
constrained with a class; the constraint is propagated to the other
occurrences automatically.

So far there are no types of class \isa{ordrel}. To breathe life
into \isa{ordrel} we need to declare a type to be an \bfindex{instance} of
\isa{ordrel}:%
\end{isamarkuptext}%
\isamarkuptrue%
\isacommand{instance}\isamarkupfalse%
\ bool\ {\isacharcolon}{\isacharcolon}\ ordrel%
\isadelimproof
%
\endisadelimproof
%
\isatagproof
%
\begin{isamarkuptxt}%
\noindent
Command \isacommand{instance} actually starts a proof, namely that
\isa{bool} satisfies all axioms of \isa{ordrel}.
There are none, but we still need to finish that proof, which we do
by invoking the \methdx{intro_classes} method:%
\end{isamarkuptxt}%
\isamarkuptrue%
\isacommand{by}\isamarkupfalse%
\ intro{\isacharunderscore}classes%
\endisatagproof
{\isafoldproof}%
%
\isadelimproof
%
\endisadelimproof
%
\begin{isamarkuptext}%
\noindent
More interesting \isacommand{instance} proofs will arise below
in the context of proper axiomatic type classes.

Although terms like \isa{False\ {\isacharless}{\isacharless}{\isacharequal}\ P} are now legal, we still need to say
what the relation symbols actually mean at type \isa{bool}:%
\end{isamarkuptext}%
\isamarkuptrue%
\isacommand{defs}\isamarkupfalse%
\ {\isacharparenleft}\isakeyword{overloaded}{\isacharparenright}\isanewline
le{\isacharunderscore}bool{\isacharunderscore}def{\isacharcolon}\ {\isachardoublequoteopen}P\ {\isacharless}{\isacharless}{\isacharequal}\ Q\ {\isasymequiv}\ P\ {\isasymlongrightarrow}\ Q{\isachardoublequoteclose}\isanewline
lt{\isacharunderscore}bool{\isacharunderscore}def{\isacharcolon}\ {\isachardoublequoteopen}P\ {\isacharless}{\isacharless}\ Q\ {\isasymequiv}\ {\isasymnot}P\ {\isasymand}\ Q{\isachardoublequoteclose}%
\begin{isamarkuptext}%
\noindent
Now \isa{False\ {\isacharless}{\isacharless}{\isacharequal}\ P} is provable:%
\end{isamarkuptext}%
\isamarkuptrue%
\isacommand{lemma}\isamarkupfalse%
\ {\isachardoublequoteopen}False\ {\isacharless}{\isacharless}{\isacharequal}\ P{\isachardoublequoteclose}\isanewline
%
\isadelimproof
%
\endisadelimproof
%
\isatagproof
\isacommand{by}\isamarkupfalse%
{\isacharparenleft}simp\ add{\isacharcolon}\ le{\isacharunderscore}bool{\isacharunderscore}def{\isacharparenright}%
\endisatagproof
{\isafoldproof}%
%
\isadelimproof
%
\endisadelimproof
%
\begin{isamarkuptext}%
\noindent
At this point, \isa{{\isacharbrackleft}{\isacharbrackright}\ {\isacharless}{\isacharless}{\isacharequal}\ {\isacharbrackleft}{\isacharbrackright}} is not even well-typed.
To make it well-typed,
we need to make lists a type of class \isa{ordrel}:%
\end{isamarkuptext}%
\isamarkuptrue%
%
\isadelimtheory
%
\endisadelimtheory
%
\isatagtheory
%
\endisatagtheory
{\isafoldtheory}%
%
\isadelimtheory
%
\endisadelimtheory
\end{isabellebody}%
%%% Local Variables:
%%% mode: latex
%%% TeX-master: "root"
%%% End:

%
\begin{isabellebody}%
\def\isabellecontext{Overloading}%
%
\isadelimtheory
%
\endisadelimtheory
%
\isatagtheory
%
\endisatagtheory
{\isafoldtheory}%
%
\isadelimtheory
%
\endisadelimtheory
%
\begin{isamarkuptext}%
Type classes allow \emph{overloading}; thus a constant may
have multiple definitions at non-overlapping types.%
\end{isamarkuptext}%
\isamarkuptrue%
%
\isamarkupsubsubsection{Overloading%
}
\isamarkuptrue%
%
\begin{isamarkuptext}%
We can introduce a binary infix addition operator \isa{{\isasymotimes}}
for arbitrary types by means of a type class:%
\end{isamarkuptext}%
\isamarkuptrue%
\isacommand{class}\isamarkupfalse%
\ plus\ {\isacharequal}\isanewline
\ \ \isakeyword{fixes}\ plus\ {\isacharcolon}{\isacharcolon}\ {\isachardoublequoteopen}{\isacharprime}a\ {\isasymRightarrow}\ {\isacharprime}a\ {\isasymRightarrow}\ {\isacharprime}a{\isachardoublequoteclose}\ {\isacharparenleft}\isakeyword{infixl}\ {\isachardoublequoteopen}{\isasymoplus}{\isachardoublequoteclose}\ {\isadigit{7}}{\isadigit{0}}{\isacharparenright}%
\begin{isamarkuptext}%
\noindent This introduces a new class \isa{plus},
along with a constant \isa{plus} with nice infix syntax.
\isa{plus} is also named \emph{class operation}.  The type
of \isa{plus} carries a sort constraint \isa{{\isachardoublequote}{\isacharprime}a\ {\isacharcolon}{\isacharcolon}\ plus{\isachardoublequote}} on its type variable, meaning that only types of class
\isa{plus} can be instantiated for \isa{{\isachardoublequote}{\isacharprime}a{\isachardoublequote}}.
To breathe life into \isa{plus} we need to declare a type
to be an \bfindex{instance} of \isa{plus}:%
\end{isamarkuptext}%
\isamarkuptrue%
\isacommand{instantiation}\isamarkupfalse%
\ nat\ {\isacharcolon}{\isacharcolon}\ plus\isanewline
\isakeyword{begin}%
\begin{isamarkuptext}%
\noindent Command \isacommand{instantiation} opens a local
theory context.  Here we can now instantiate \isa{plus} on
\isa{nat}:%
\end{isamarkuptext}%
\isamarkuptrue%
\isacommand{primrec}\isamarkupfalse%
\ plus{\isacharunderscore}nat\ {\isacharcolon}{\isacharcolon}\ {\isachardoublequoteopen}nat\ {\isasymRightarrow}\ nat\ {\isasymRightarrow}\ nat{\isachardoublequoteclose}\ \isakeyword{where}\isanewline
\ \ \ \ {\isachardoublequoteopen}{\isacharparenleft}{\isadigit{0}}{\isacharcolon}{\isacharcolon}nat{\isacharparenright}\ {\isasymoplus}\ n\ {\isacharequal}\ n{\isachardoublequoteclose}\isanewline
\ \ {\isacharbar}\ {\isachardoublequoteopen}Suc\ m\ {\isasymoplus}\ n\ {\isacharequal}\ Suc\ {\isacharparenleft}m\ {\isasymoplus}\ n{\isacharparenright}{\isachardoublequoteclose}%
\begin{isamarkuptext}%
\noindent Note that the name \isa{plus} carries a
suffix \isa{{\isacharunderscore}nat}; by default, the local name of a class operation
\isa{f} to be instantiated on type constructor \isa{{\isasymkappa}} is mangled
as \isa{f{\isacharunderscore}{\isasymkappa}}.  In case of uncertainty, these names may be inspected
using the \hyperlink{command.print-context}{\mbox{\isa{\isacommand{print{\isacharunderscore}context}}}} command or the corresponding
ProofGeneral button.

Although class \isa{plus} has no axioms, the instantiation must be
formally concluded by a (trivial) instantiation proof ``..'':%
\end{isamarkuptext}%
\isamarkuptrue%
\isacommand{instance}\isamarkupfalse%
%
\isadelimproof
\ %
\endisadelimproof
%
\isatagproof
\isacommand{{\isachardot}{\isachardot}}\isamarkupfalse%
%
\endisatagproof
{\isafoldproof}%
%
\isadelimproof
%
\endisadelimproof
%
\begin{isamarkuptext}%
\noindent More interesting \isacommand{instance} proofs will
arise below.

The instantiation is finished by an explicit%
\end{isamarkuptext}%
\isamarkuptrue%
\isacommand{end}\isamarkupfalse%
%
\begin{isamarkuptext}%
\noindent From now on, terms like \isa{Suc\ {\isacharparenleft}m\ {\isasymoplus}\ {\isadigit{2}}{\isacharparenright}} are
legal.%
\end{isamarkuptext}%
\isamarkuptrue%
\isacommand{instantiation}\isamarkupfalse%
\ {\isachardoublequoteopen}{\isacharasterisk}{\isachardoublequoteclose}\ {\isacharcolon}{\isacharcolon}\ {\isacharparenleft}plus{\isacharcomma}\ plus{\isacharparenright}\ plus\isanewline
\isakeyword{begin}%
\begin{isamarkuptext}%
\noindent Here we instantiate the product type \isa{{\isacharasterisk}} to
class \isa{plus}, given that its type arguments are of
class \isa{plus}:%
\end{isamarkuptext}%
\isamarkuptrue%
\isacommand{fun}\isamarkupfalse%
\ plus{\isacharunderscore}prod\ {\isacharcolon}{\isacharcolon}\ {\isachardoublequoteopen}{\isacharprime}a\ {\isasymtimes}\ {\isacharprime}b\ {\isasymRightarrow}\ {\isacharprime}a\ {\isasymtimes}\ {\isacharprime}b\ {\isasymRightarrow}\ {\isacharprime}a\ {\isasymtimes}\ {\isacharprime}b{\isachardoublequoteclose}\ \isakeyword{where}\isanewline
\ \ {\isachardoublequoteopen}{\isacharparenleft}x{\isacharcomma}\ y{\isacharparenright}\ {\isasymoplus}\ {\isacharparenleft}w{\isacharcomma}\ z{\isacharparenright}\ {\isacharequal}\ {\isacharparenleft}x\ {\isasymoplus}\ w{\isacharcomma}\ y\ {\isasymoplus}\ z{\isacharparenright}{\isachardoublequoteclose}%
\begin{isamarkuptext}%
\noindent Obviously, overloaded specifications may include
recursion over the syntactic structure of types.%
\end{isamarkuptext}%
\isamarkuptrue%
\isacommand{instance}\isamarkupfalse%
%
\isadelimproof
\ %
\endisadelimproof
%
\isatagproof
\isacommand{{\isachardot}{\isachardot}}\isamarkupfalse%
%
\endisatagproof
{\isafoldproof}%
%
\isadelimproof
%
\endisadelimproof
\isanewline
\isanewline
\isacommand{end}\isamarkupfalse%
%
\begin{isamarkuptext}%
\noindent This way we have encoded the canonical lifting of
binary operations to products by means of type classes.%
\end{isamarkuptext}%
\isamarkuptrue%
%
\isadelimtheory
%
\endisadelimtheory
%
\isatagtheory
%
\endisatagtheory
{\isafoldtheory}%
%
\isadelimtheory
%
\endisadelimtheory
\end{isabellebody}%
%%% Local Variables:
%%% mode: latex
%%% TeX-master: "root"
%%% End:

%
\begin{isabellebody}%
\def\isabellecontext{Overloading{\isadigit{2}}}%
%
\isadelimtheory
%
\endisadelimtheory
%
\isatagtheory
%
\endisatagtheory
{\isafoldtheory}%
%
\isadelimtheory
%
\endisadelimtheory
%
\begin{isamarkuptext}%
Of course this is not the only possible definition of the two relations.
Componentwise comparison of lists of equal length also makes sense. This time
the elements of the list must also be of class \isa{ordrel} to permit their
comparison:%
\end{isamarkuptext}%
\isamarkuptrue%
\isacommand{instance}\isamarkupfalse%
\ list\ {\isacharcolon}{\isacharcolon}\ {\isacharparenleft}ordrel{\isacharparenright}ordrel\isanewline
%
\isadelimproof
%
\endisadelimproof
%
\isatagproof
\isacommand{by}\isamarkupfalse%
\ intro{\isacharunderscore}classes%
\endisatagproof
{\isafoldproof}%
%
\isadelimproof
\isanewline
%
\endisadelimproof
\isanewline
\isacommand{defs}\isamarkupfalse%
\ {\isacharparenleft}\isakeyword{overloaded}{\isacharparenright}\isanewline
le{\isacharunderscore}list{\isacharunderscore}def{\isacharcolon}\ {\isachardoublequoteopen}xs\ {\isacharless}{\isacharless}{\isacharequal}\ {\isacharparenleft}ys{\isacharcolon}{\isacharcolon}{\isacharprime}a{\isacharcolon}{\isacharcolon}ordrel\ list{\isacharparenright}\ {\isasymequiv}\isanewline
\ \ \ \ \ \ \ \ \ \ \ \ \ \ size\ xs\ {\isacharequal}\ size\ ys\ {\isasymand}\ {\isacharparenleft}{\isasymforall}i{\isacharless}size\ xs{\isachardot}\ xs{\isacharbang}i\ {\isacharless}{\isacharless}{\isacharequal}\ ys{\isacharbang}i{\isacharparenright}{\isachardoublequoteclose}%
\begin{isamarkuptext}%
\noindent
The infix function \isa{{\isacharbang}} yields the nth element of a list.

\begin{warn}
A type constructor can be instantiated in only one way to
a given type class.  For example, our two instantiations of \isa{list} must
reside in separate theories with disjoint scopes.
\end{warn}%
\end{isamarkuptext}%
\isamarkuptrue%
%
\isamarkupsubsubsection{Predefined Overloading%
}
\isamarkuptrue%
%
\begin{isamarkuptext}%
HOL comes with a number of overloaded constants and corresponding classes.
The most important ones are listed in Table~\ref{tab:overloading} in the appendix. They are
defined on all numeric types and sometimes on other types as well, for example
$-$ and \isa{{\isasymle}} on sets.

In addition there is a special syntax for bounded quantifiers:
\begin{center}
\begin{tabular}{lcl}
\isa{{\isasymforall}x{\isasymle}y{\isachardot}\ P\ x} & \isa{{\isasymrightleftharpoons}} & \isa{{\isachardoublequote}{\isasymforall}x{\isachardot}\ x\ {\isasymle}\ y\ {\isasymlongrightarrow}\ P\ x{\isachardoublequote}} \\
\isa{{\isasymexists}x{\isasymle}y{\isachardot}\ P\ x} & \isa{{\isasymrightleftharpoons}} & \isa{{\isachardoublequote}{\isasymexists}x{\isachardot}\ x\ {\isasymle}\ y\ {\isasymand}\ P\ x{\isachardoublequote}}
\end{tabular}
\end{center}
And analogously for \isa{{\isacharless}} instead of \isa{{\isasymle}}.%
\end{isamarkuptext}%
\isamarkuptrue%
%
\isadelimtheory
%
\endisadelimtheory
%
\isatagtheory
%
\endisatagtheory
{\isafoldtheory}%
%
\isadelimtheory
%
\endisadelimtheory
\end{isabellebody}%
%%% Local Variables:
%%% mode: latex
%%% TeX-master: "root"
%%% End:


\index{overloading|)}

Finally we should remind our readers that \isa{Main} contains a much more
developed theory of orderings phrased in terms of the usual $\leq$ and
\isa{<}. It is recommended that, if possible, you base your own
ordering relations on this theory.

\index{axiomatic type class|)}
\index{*axclass|)}
