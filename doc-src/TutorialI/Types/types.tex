\chapter{More about Types}

So far we have learned about a few basic types (for example \isa{bool} and
\isa{nat}), type abbreviations (\isacommand{types}) and recursive datatpes
(\isacommand{datatype}). This chapter will introduce the following more
advanced material:
\begin{itemize}
\item More about basic types: numbers ({\S}\ref{sec:numbers}), pairs
  ({\S}\ref{sec:products}) and records ({\S}\ref{sec:records}), and how to reason
  about them.
\item Introducing your own types: how to introduce your own new types that
  cannot be constructed with any of the basic methods ({\S}\ref{sec:adv-typedef}).
\item Type classes: how to specify and reason about axiomatic collections of
  types ({\S}\ref{sec:axclass}).
\end{itemize}

\section{Basic types}

\subsection{Numbers}
\label{sec:numbers}

\subsection{Pairs}
\label{sec:products}
% Check refs to this section to see what is expected of it.
% Mention type unit

\subsection{Records}
\label{sec:records}

%
\begin{isabellebody}%
\def\isabellecontext{Typedef}%
%
\isamarkupsection{Introducing New Types%
}
%
\begin{isamarkuptext}%
\label{sec:adv-typedef}
For most applications, a combination of predefined types like \isa{bool} and
\isa{{\isasymRightarrow}} with recursive datatypes and records is quite sufficient. Very
occasionally you may feel the need for a more advanced type. If you cannot do
without that type, and you are certain that it is not definable by any of the
standard means, then read on.
\begin{warn}
  Types in HOL must be non-empty; otherwise the quantifier rules would be
  unsound, because $\exists x.\ x=x$ is a theorem.
\end{warn}%
\end{isamarkuptext}%
%
\isamarkupsubsection{Declaring New Types%
}
%
\begin{isamarkuptext}%
\label{sec:typedecl}
The most trivial way of introducing a new type is by a \bfindex{type
declaration}:%
\end{isamarkuptext}%
\isacommand{typedecl}\ my{\isacharunderscore}new{\isacharunderscore}type%
\begin{isamarkuptext}%
\noindent\indexbold{*typedecl}%
This does not define \isa{my{\isacharunderscore}new{\isacharunderscore}type} at all but merely introduces its
name. Thus we know nothing about this type, except that it is
non-empty. Such declarations without definitions are
useful if that type can be viewed as a parameter of the theory.
A typical example is given in \S\ref{sec:VMC}, where we define a transition
relation over an arbitrary type of states.

In principle we can always get rid of such type declarations by making those
types parameters of every other type, thus keeping the theory generic. In
practice, however, the resulting clutter can make types hard to read.

If you are looking for a quick and dirty way of introducing a new type
together with its properties: declare the type and state its properties as
axioms. Example:%
\end{isamarkuptext}%
\isacommand{axioms}\isanewline
just{\isacharunderscore}one{\isacharcolon}\ {\isachardoublequote}{\isasymexists}x{\isacharcolon}{\isacharcolon}my{\isacharunderscore}new{\isacharunderscore}type{\isachardot}\ {\isasymforall}y{\isachardot}\ x\ {\isacharequal}\ y{\isachardoublequote}%
\begin{isamarkuptext}%
\noindent
However, we strongly discourage this approach, except at explorative stages
of your development. It is extremely easy to write down contradictory sets of
axioms, in which case you will be able to prove everything but it will mean
nothing.  In the example above, the axiomatic approach is
unnecessary: a one-element type called \isa{unit} is already defined in HOL.%
\end{isamarkuptext}%
%
\isamarkupsubsection{Defining New Types%
}
%
\begin{isamarkuptext}%
\label{sec:typedef}
Now we come to the most general method of safely introducing a new type, the
\bfindex{type definition}. All other methods, for example
\isacommand{datatype}, are based on it. The principle is extremely simple:
any non-empty subset of an existing type can be turned into a new type.  Thus
a type definition is merely a notational device: you introduce a new name for
a subset of an existing type. This does not add any logical power to HOL,
because you could base all your work directly on the subset of the existing
type. However, the resulting theories could easily become indigestible
because instead of implicit types you would have explicit sets in your
formulae.

Let us work a simple example, the definition of a three-element type.
It is easily represented by the first three natural numbers:%
\end{isamarkuptext}%
\isacommand{typedef}\ three\ {\isacharequal}\ {\isachardoublequote}{\isacharbraceleft}n{\isachardot}\ n\ {\isasymle}\ {\isadigit{2}}{\isacharbraceright}{\isachardoublequote}%
\begin{isamarkuptxt}%
\noindent\indexbold{*typedef}%
In order to enforce that the representing set on the right-hand side is
non-empty, this definition actually starts a proof to that effect:
\begin{isabelle}%
\ {\isadigit{1}}{\isachardot}\ {\isasymexists}x{\isachardot}\ x\ {\isasymin}\ {\isacharbraceleft}n{\isachardot}\ n\ {\isasymle}\ {\isadigit{2}}{\isacharbraceright}%
\end{isabelle}
Fortunately, this is easy enough to show: take 0 as a witness.%
\end{isamarkuptxt}%
\isacommand{apply}{\isacharparenleft}rule{\isacharunderscore}tac\ x\ {\isacharequal}\ {\isadigit{0}}\ \isakeyword{in}\ exI{\isacharparenright}\isanewline
\isacommand{by}\ simp%
\begin{isamarkuptext}%
This type definition introduces the new type \isa{three} and asserts
that it is a copy of the set \isa{{\isacharbraceleft}{\isadigit{0}}{\isacharcomma}\ {\isadigit{1}}{\isacharcomma}\ {\isadigit{2}}{\isacharbraceright}}. This assertion
is expressed via a bijection between the \emph{type} \isa{three} and the
\emph{set} \isa{{\isacharbraceleft}{\isadigit{0}}{\isacharcomma}\ {\isadigit{1}}{\isacharcomma}\ {\isadigit{2}}{\isacharbraceright}}. To this end, the command declares the following
constants behind the scenes:
\begin{center}
\begin{tabular}{rcl}
\isa{three} &::& \isa{nat\ set} \\
\isa{Rep{\isacharunderscore}three} &::& \isa{three\ {\isasymRightarrow}\ nat}\\
\isa{Abs{\isacharunderscore}three} &::& \isa{nat\ {\isasymRightarrow}\ three}
\end{tabular}
\end{center}
where constant \isa{three} is explicitly defined as the representing set:
\begin{center}
\isa{three\ {\isasymequiv}\ {\isacharbraceleft}n{\isachardot}\ n\ {\isasymle}\ {\isadigit{2}}{\isacharbraceright}}\hfill(\isa{three{\isacharunderscore}def})
\end{center}
The situation is best summarized with the help of the following diagram,
where squares are types and circles are sets:
\begin{center}
\unitlength1mm
\thicklines
\begin{picture}(100,40)
\put(3,13){\framebox(15,15){\isa{three}}}
\put(55,5){\framebox(30,30){\isa{three}}}
\put(70,32){\makebox(0,0){\isa{nat}}}
\put(70,20){\circle{40}}
\put(10,15){\vector(1,0){60}}
\put(25,14){\makebox(0,0)[tl]{\isa{Rep{\isacharunderscore}three}}}
\put(70,25){\vector(-1,0){60}}
\put(25,26){\makebox(0,0)[bl]{\isa{Abs{\isacharunderscore}three}}}
\end{picture}
\end{center}
Finally, \isacommand{typedef} asserts that \isa{Rep{\isacharunderscore}three} is
surjective on the subset \isa{three} and \isa{Abs{\isacharunderscore}three} and \isa{Rep{\isacharunderscore}three} are inverses of each other:
\begin{center}
\begin{tabular}{@ {}r@ {\qquad\qquad}l@ {}}
\isa{Rep{\isacharunderscore}three\ x\ {\isasymin}\ three} & (\isa{Rep{\isacharunderscore}three}) \\
\isa{Abs{\isacharunderscore}three\ {\isacharparenleft}Rep{\isacharunderscore}three\ x{\isacharparenright}\ {\isacharequal}\ x} & (\isa{Rep{\isacharunderscore}three{\isacharunderscore}inverse}) \\
\isa{y\ {\isasymin}\ three\ {\isasymLongrightarrow}\ Rep{\isacharunderscore}three\ {\isacharparenleft}Abs{\isacharunderscore}three\ y{\isacharparenright}\ {\isacharequal}\ y} & (\isa{Abs{\isacharunderscore}three{\isacharunderscore}inverse})
\end{tabular}
\end{center}
%
From this example it should be clear what \isacommand{typedef} does
in general given a name (here \isa{three}) and a set
(here \isa{{\isacharbraceleft}n{\isachardot}\ n\ {\isasymle}\ {\isadigit{2}}{\isacharbraceright}}).

Our next step is to define the basic functions expected on the new type.
Although this depends on the type at hand, the following strategy works well:
\begin{itemize}
\item define a small kernel of basic functions that can express all other
functions you anticipate.
\item define the kernel in terms of corresponding functions on the
representing type using \isa{Abs} and \isa{Rep} to convert between the
two levels.
\end{itemize}
In our example it suffices to give the three elements of type \isa{three}
names:%
\end{isamarkuptext}%
\isacommand{constdefs}\isanewline
\ \ A{\isacharcolon}{\isacharcolon}\ three\isanewline
\ {\isachardoublequote}A\ {\isasymequiv}\ Abs{\isacharunderscore}three\ {\isadigit{0}}{\isachardoublequote}\isanewline
\ \ B{\isacharcolon}{\isacharcolon}\ three\isanewline
\ {\isachardoublequote}B\ {\isasymequiv}\ Abs{\isacharunderscore}three\ {\isadigit{1}}{\isachardoublequote}\isanewline
\ \ C\ {\isacharcolon}{\isacharcolon}\ three\isanewline
\ {\isachardoublequote}C\ {\isasymequiv}\ Abs{\isacharunderscore}three\ {\isadigit{2}}{\isachardoublequote}%
\begin{isamarkuptext}%
So far, everything was easy. But it is clear that reasoning about \isa{three} will be hell if we have to go back to \isa{nat} every time. Thus our
aim must be to raise our level of abstraction by deriving enough theorems
about type \isa{three} to characterize it completely. And those theorems
should be phrased in terms of \isa{A}, \isa{B} and \isa{C}, not \isa{Abs{\isacharunderscore}three} and \isa{Rep{\isacharunderscore}three}. Because of the simplicity of the example,
we merely need to prove that \isa{A}, \isa{B} and \isa{C} are distinct
and that they exhaust the type.

We start with a helpful version of injectivity of \isa{Abs{\isacharunderscore}three} on the
representing subset:%
\end{isamarkuptext}%
\isacommand{lemma}\ {\isacharbrackleft}simp{\isacharbrackright}{\isacharcolon}\isanewline
\ {\isachardoublequote}{\isasymlbrakk}\ x\ {\isasymin}\ three{\isacharsemicolon}\ y\ {\isasymin}\ three\ {\isasymrbrakk}\ {\isasymLongrightarrow}\ {\isacharparenleft}Abs{\isacharunderscore}three\ x\ {\isacharequal}\ Abs{\isacharunderscore}three\ y{\isacharparenright}\ {\isacharequal}\ {\isacharparenleft}x{\isacharequal}y{\isacharparenright}{\isachardoublequote}%
\begin{isamarkuptxt}%
\noindent
We prove each direction separately. From \isa{Abs{\isacharunderscore}three\ x\ {\isacharequal}\ Abs{\isacharunderscore}three\ y}
we use \isa{arg{\isacharunderscore}cong} to apply \isa{Rep{\isacharunderscore}three} to both sides,
deriving \begin{isabelle}%
Rep{\isacharunderscore}three\ {\isacharparenleft}Abs{\isacharunderscore}three\ x{\isacharparenright}\ {\isacharequal}\ Rep{\isacharunderscore}three\ {\isacharparenleft}Abs{\isacharunderscore}three\ y{\isacharparenright}%
\end{isabelle}
Thus we get the required \isa{x\ {\isacharequal}\ y} by simplification with \isa{Abs{\isacharunderscore}three{\isacharunderscore}inverse}. 
The other direction
is trivial by simplification:%
\end{isamarkuptxt}%
\isacommand{apply}{\isacharparenleft}rule\ iffI{\isacharparenright}\isanewline
\ \isacommand{apply}{\isacharparenleft}drule{\isacharunderscore}tac\ f\ {\isacharequal}\ Rep{\isacharunderscore}three\ \isakeyword{in}\ arg{\isacharunderscore}cong{\isacharparenright}\isanewline
\ \isacommand{apply}{\isacharparenleft}simp\ add{\isacharcolon}Abs{\isacharunderscore}three{\isacharunderscore}inverse{\isacharparenright}\isanewline
\isacommand{by}\ simp%
\begin{isamarkuptext}%
\noindent
Analogous lemmas can be proved in the same way for arbitrary type definitions.

Distinctness of \isa{A}, \isa{B} and \isa{C} follows immediately
if we expand their definitions and rewrite with the above simplification rule:%
\end{isamarkuptext}%
\isacommand{lemma}\ {\isachardoublequote}A\ {\isasymnoteq}\ B\ {\isasymand}\ B\ {\isasymnoteq}\ A\ {\isasymand}\ A\ {\isasymnoteq}\ C\ {\isasymand}\ C\ {\isasymnoteq}\ A\ {\isasymand}\ B\ {\isasymnoteq}\ C\ {\isasymand}\ C\ {\isasymnoteq}\ B{\isachardoublequote}\isanewline
\isacommand{by}{\isacharparenleft}simp\ add{\isacharcolon}A{\isacharunderscore}def\ B{\isacharunderscore}def\ C{\isacharunderscore}def\ three{\isacharunderscore}def{\isacharparenright}%
\begin{isamarkuptext}%
\noindent
Of course we rely on the simplifier to solve goals like \isa{{\isadigit{0}}\ {\isasymnoteq}\ {\isadigit{1}}}.

The fact that \isa{A}, \isa{B} and \isa{C} exhaust type \isa{three} is
best phrased as a case distinction theorem: if you want to prove \isa{P\ x}
(where \isa{x} is of type \isa{three}) it suffices to prove \isa{P\ A},
\isa{P\ B} and \isa{P\ C}. First we prove the analogous proposition for the
representation:%
\end{isamarkuptext}%
\isacommand{lemma}\ cases{\isacharunderscore}lemma{\isacharcolon}\ {\isachardoublequote}{\isasymlbrakk}\ Q\ {\isadigit{0}}{\isacharsemicolon}\ Q\ {\isadigit{1}}{\isacharsemicolon}\ Q\ {\isadigit{2}}{\isacharsemicolon}\ n\ {\isasymin}\ three\ {\isasymrbrakk}\ {\isasymLongrightarrow}\ \ Q\ n{\isachardoublequote}%
\begin{isamarkuptxt}%
\noindent
Expanding \isa{three{\isacharunderscore}def} yields the premise \isa{n\ {\isasymle}\ {\isadigit{2}}}. Repeated
elimination with \isa{le{\isacharunderscore}SucE}
\begin{isabelle}%
{\isasymlbrakk}{\isacharquery}m\ {\isasymle}\ Suc\ {\isacharquery}n{\isacharsemicolon}\ {\isacharquery}m\ {\isasymle}\ {\isacharquery}n\ {\isasymLongrightarrow}\ {\isacharquery}R{\isacharsemicolon}\ {\isacharquery}m\ {\isacharequal}\ Suc\ {\isacharquery}n\ {\isasymLongrightarrow}\ {\isacharquery}R{\isasymrbrakk}\ {\isasymLongrightarrow}\ {\isacharquery}R%
\end{isabelle}
reduces \isa{n\ {\isasymle}\ {\isadigit{2}}} to the three cases \isa{n\ {\isasymle}\ {\isadigit{0}}}, \isa{n\ {\isacharequal}\ {\isadigit{1}}} and
\isa{n\ {\isacharequal}\ {\isadigit{2}}} which are trivial for simplification:%
\end{isamarkuptxt}%
\isacommand{apply}{\isacharparenleft}simp\ add{\isacharcolon}three{\isacharunderscore}def{\isacharparenright}\isanewline
\isacommand{apply}{\isacharparenleft}{\isacharparenleft}erule\ le{\isacharunderscore}SucE{\isacharparenright}{\isacharplus}{\isacharparenright}\isanewline
\isacommand{apply}\ simp{\isacharunderscore}all\isanewline
\isacommand{done}%
\begin{isamarkuptext}%
Now the case distinction lemma on type \isa{three} is easy to derive if you know how to:%
\end{isamarkuptext}%
\isacommand{lemma}\ three{\isacharunderscore}cases{\isacharcolon}\ {\isachardoublequote}{\isasymlbrakk}\ P\ A{\isacharsemicolon}\ P\ B{\isacharsemicolon}\ P\ C\ {\isasymrbrakk}\ {\isasymLongrightarrow}\ P\ x{\isachardoublequote}%
\begin{isamarkuptxt}%
\noindent
We start by replacing the \isa{x} by \isa{Abs{\isacharunderscore}three\ {\isacharparenleft}Rep{\isacharunderscore}three\ x{\isacharparenright}}:%
\end{isamarkuptxt}%
\isacommand{apply}{\isacharparenleft}rule\ subst{\isacharbrackleft}OF\ Rep{\isacharunderscore}three{\isacharunderscore}inverse{\isacharbrackright}{\isacharparenright}%
\begin{isamarkuptxt}%
\noindent
This substitution step worked nicely because there was just a single
occurrence of a term of type \isa{three}, namely \isa{x}.
When we now apply \isa{cases{\isacharunderscore}lemma}, \isa{Q} becomes \isa{{\isasymlambda}n{\isachardot}\ P\ {\isacharparenleft}Abs{\isacharunderscore}three\ n{\isacharparenright}} because \isa{Rep{\isacharunderscore}three\ x} is the only term of type \isa{nat}:%
\end{isamarkuptxt}%
\isacommand{apply}{\isacharparenleft}rule\ cases{\isacharunderscore}lemma{\isacharparenright}%
\begin{isamarkuptxt}%
\begin{isabelle}%
\ {\isadigit{1}}{\isachardot}\ {\isasymlbrakk}P\ A{\isacharsemicolon}\ P\ B{\isacharsemicolon}\ P\ C{\isasymrbrakk}\ {\isasymLongrightarrow}\ P\ {\isacharparenleft}Abs{\isacharunderscore}three\ {\isadigit{0}}{\isacharparenright}\isanewline
\ {\isadigit{2}}{\isachardot}\ {\isasymlbrakk}P\ A{\isacharsemicolon}\ P\ B{\isacharsemicolon}\ P\ C{\isasymrbrakk}\ {\isasymLongrightarrow}\ P\ {\isacharparenleft}Abs{\isacharunderscore}three\ {\isadigit{1}}{\isacharparenright}\isanewline
\ {\isadigit{3}}{\isachardot}\ {\isasymlbrakk}P\ A{\isacharsemicolon}\ P\ B{\isacharsemicolon}\ P\ C{\isasymrbrakk}\ {\isasymLongrightarrow}\ P\ {\isacharparenleft}Abs{\isacharunderscore}three\ {\isadigit{2}}{\isacharparenright}\isanewline
\ {\isadigit{4}}{\isachardot}\ {\isasymlbrakk}P\ A{\isacharsemicolon}\ P\ B{\isacharsemicolon}\ P\ C{\isasymrbrakk}\ {\isasymLongrightarrow}\ Rep{\isacharunderscore}three\ x\ {\isasymin}\ three%
\end{isabelle}
The resulting subgoals are easily solved by simplification:%
\end{isamarkuptxt}%
\isacommand{apply}{\isacharparenleft}simp{\isacharunderscore}all\ add{\isacharcolon}A{\isacharunderscore}def\ B{\isacharunderscore}def\ C{\isacharunderscore}def\ Rep{\isacharunderscore}three{\isacharparenright}\isanewline
\isacommand{done}%
\begin{isamarkuptext}%
\noindent
This concludes the derivation of the characteristic theorems for
type \isa{three}.

The attentive reader has realized long ago that the
above lengthy definition can be collapsed into one line:%
\end{isamarkuptext}%
\isacommand{datatype}\ three{\isacharprime}\ {\isacharequal}\ A\ {\isacharbar}\ B\ {\isacharbar}\ C%
\begin{isamarkuptext}%
\noindent
In fact, the \isacommand{datatype} command performs internally more or less
the same derivations as we did, which gives you some idea what life would be
like without \isacommand{datatype}.

Although \isa{three} could be defined in one line, we have chosen this
example to demonstrate \isacommand{typedef} because its simplicity makes the
key concepts particularly easy to grasp. If you would like to see a
non-trivial example that cannot be defined more directly, we recommend the
definition of \emph{finite multisets} in the HOL Library.

Let us conclude by summarizing the above procedure for defining a new type.
Given some abstract axiomatic description $P$ of a type $ty$ in terms of a
set of functions $F$, this involves three steps:
\begin{enumerate}
\item Find an appropriate type $\tau$ and subset $A$ which has the desired
  properties $P$, and make a type definition based on this representation.
\item Define the required functions $F$ on $ty$ by lifting
analogous functions on the representation via $Abs_ty$ and $Rep_ty$.
\item Prove that $P$ holds for $ty$ by lifting $P$ from the representation.
\end{enumerate}
You can now forget about the representation and work solely in terms of the
abstract functions $F$ and properties $P$.%
\end{isamarkuptext}%
\end{isabellebody}%
%%% Local Variables:
%%% mode: latex
%%% TeX-master: "root"
%%% End:


\section{Axiomatic type classes}
\label{sec:axclass}
\index{axiomatic type class|(}
\index{*axclass|(}


The programming language Haskell has popularized the notion of type classes.
Isabelle offers the related concept of an \textbf{axiomatic type class}.
Roughly speaking, an axiomatic type class is a type class with axioms, i.e.\ 
an axiomatic specification of a class of types. Thus we can talk about a type
$t$ being in a class $c$, which is written $\tau :: c$.  This is the case of
$\tau$ satisfies the axioms of $c$. Furthermore, type classes can be
organized in a hierarchy. Thus there is the notion of a class $d$ being a
\textbf{subclass} of a class $c$, written $d < c$. This is the case if all
axioms of $c$ are also provable in $d$. Let us now introduce these concepts
by means of a running example, ordering relations.

\subsection{Overloading}
\label{sec:overloading}
\index{overloading|(}

%
\begin{isabellebody}%
\def\isabellecontext{Overloading{\isadigit{0}}}%
%
\begin{isamarkuptext}%
We start with a concept that is required for type classes but already
useful on its own: \emph{overloading}. Isabelle allows overloading: a
constant may have multiple definitions at non-overlapping types.%
\end{isamarkuptext}%
%
\isamarkupsubsubsection{An Initial Example%
}
%
\begin{isamarkuptext}%
If we want to introduce the notion of an \emph{inverse} for arbitrary types we
give it a polymorphic type%
\end{isamarkuptext}%
\isacommand{consts}\ inverse\ {\isacharcolon}{\isacharcolon}\ {\isachardoublequote}{\isacharprime}a\ {\isasymRightarrow}\ {\isacharprime}a{\isachardoublequote}%
\begin{isamarkuptext}%
\noindent
and provide different definitions at different instances:%
\end{isamarkuptext}%
\isacommand{defs}\ {\isacharparenleft}\isakeyword{overloaded}{\isacharparenright}\isanewline
inverse{\isacharunderscore}bool{\isacharcolon}\ {\isachardoublequote}inverse{\isacharparenleft}b{\isacharcolon}{\isacharcolon}bool{\isacharparenright}\ {\isasymequiv}\ {\isasymnot}\ b{\isachardoublequote}\isanewline
inverse{\isacharunderscore}set{\isacharcolon}\ \ {\isachardoublequote}inverse{\isacharparenleft}A{\isacharcolon}{\isacharcolon}{\isacharprime}a\ set{\isacharparenright}\ {\isasymequiv}\ {\isacharminus}A{\isachardoublequote}\isanewline
inverse{\isacharunderscore}pair{\isacharcolon}\ {\isachardoublequote}inverse{\isacharparenleft}p{\isacharparenright}\ {\isasymequiv}\ {\isacharparenleft}inverse{\isacharparenleft}fst\ p{\isacharparenright}{\isacharcomma}\ inverse{\isacharparenleft}snd\ p{\isacharparenright}{\isacharparenright}{\isachardoublequote}%
\begin{isamarkuptext}%
\noindent
Isabelle will not complain because the three definitions do not overlap: no
two of the three types \isa{bool}, \isa{{\isacharprime}a\ set} and \isa{{\isacharprime}a\ {\isasymtimes}\ {\isacharprime}b} have a
common instance. What is more, the recursion in \isa{inverse{\isacharunderscore}pair} is
benign because the type of \isa{inverse} becomes smaller: on the
left it is \isa{{\isacharprime}a\ {\isasymtimes}\ {\isacharprime}b\ {\isasymRightarrow}\ {\isacharprime}a\ {\isasymtimes}\ {\isacharprime}b} but on the right \isa{{\isacharprime}a\ {\isasymRightarrow}\ {\isacharprime}a} and
\isa{{\isacharprime}b\ {\isasymRightarrow}\ {\isacharprime}b}. The annotation \isa{{\isacharparenleft}}\isacommand{overloaded}\isa{{\isacharparenright}} tells Isabelle that
the definitions do intentionally define \isa{inverse} only at
instances of its declared type \isa{{\isacharprime}a\ {\isasymRightarrow}\ {\isacharprime}a} --- this merely supresses
warnings to that effect.

However, there is nothing to prevent the user from forming terms such as
\isa{inverse\ {\isacharbrackleft}{\isacharbrackright}} and proving theorems as \isa{inverse\ {\isacharbrackleft}{\isacharbrackright}\ {\isacharequal}\ inverse\ {\isacharbrackleft}{\isacharbrackright}}, although we never defined inverse on lists. We hasten to say
that there is nothing wrong with such terms and theorems. But it would be
nice if we could prevent their formation, simply because it is very likely
that the user did not mean to write what he did. Thus she had better not waste
her time pursuing it further. This requires the use of type classes.%
\end{isamarkuptext}%
\end{isabellebody}%
%%% Local Variables:
%%% mode: latex
%%% TeX-master: "root"
%%% End:

%
\begin{isabellebody}%
\def\isabellecontext{Overloading{\isadigit{1}}}%
%
\isadelimtheory
%
\endisadelimtheory
%
\isatagtheory
%
\endisatagtheory
{\isafoldtheory}%
%
\isadelimtheory
%
\endisadelimtheory
%
\isamarkupsubsubsection{Controlled Overloading with Type Classes%
}
\isamarkuptrue%
%
\begin{isamarkuptext}%
We now start with the theory of ordering relations, which we shall phrase
in terms of the two binary symbols \isa{{\isacharless}{\isacharless}} and \isa{{\isacharless}{\isacharless}{\isacharequal}}
to avoid clashes with \isa{{\isacharless}} and \isa{{\isacharless}{\isacharequal}} in theory \isa{Main}. To restrict the application of \isa{{\isacharless}{\isacharless}} and \isa{{\isacharless}{\isacharless}{\isacharequal}} we
introduce the class \isa{ordrel}:%
\end{isamarkuptext}%
\isamarkuptrue%
\isacommand{axclass}\isamarkupfalse%
\ ordrel\ {\isacharless}\ type%
\begin{isamarkuptext}%
\noindent
This introduces a new class \isa{ordrel} and makes it a subclass of
the predefined class \isa{type}, which
is the class of all HOL types.
This is a degenerate form of axiomatic type class without any axioms.
Its sole purpose is to restrict the use of overloaded constants to meaningful
instances:%
\end{isamarkuptext}%
\isamarkuptrue%
\isacommand{consts}\isamarkupfalse%
\ lt\ {\isacharcolon}{\isacharcolon}\ {\isachardoublequoteopen}{\isacharparenleft}{\isacharprime}a{\isacharcolon}{\isacharcolon}ordrel{\isacharparenright}\ {\isasymRightarrow}\ {\isacharprime}a\ {\isasymRightarrow}\ bool{\isachardoublequoteclose}\ \ \ \ \ {\isacharparenleft}\isakeyword{infixl}\ {\isachardoublequoteopen}{\isacharless}{\isacharless}{\isachardoublequoteclose}\ \ {\isadigit{5}}{\isadigit{0}}{\isacharparenright}\isanewline
\ \ \ \ \ \ \ le\ {\isacharcolon}{\isacharcolon}\ {\isachardoublequoteopen}{\isacharparenleft}{\isacharprime}a{\isacharcolon}{\isacharcolon}ordrel{\isacharparenright}\ {\isasymRightarrow}\ {\isacharprime}a\ {\isasymRightarrow}\ bool{\isachardoublequoteclose}\ \ \ \ \ {\isacharparenleft}\isakeyword{infixl}\ {\isachardoublequoteopen}{\isacharless}{\isacharless}{\isacharequal}{\isachardoublequoteclose}\ {\isadigit{5}}{\isadigit{0}}{\isacharparenright}%
\begin{isamarkuptext}%
\noindent
Note that only one occurrence of a type variable in a type needs to be
constrained with a class; the constraint is propagated to the other
occurrences automatically.

So far there are no types of class \isa{ordrel}. To breathe life
into \isa{ordrel} we need to declare a type to be an \bfindex{instance} of
\isa{ordrel}:%
\end{isamarkuptext}%
\isamarkuptrue%
\isacommand{instance}\isamarkupfalse%
\ bool\ {\isacharcolon}{\isacharcolon}\ ordrel%
\isadelimproof
%
\endisadelimproof
%
\isatagproof
%
\begin{isamarkuptxt}%
\noindent
Command \isacommand{instance} actually starts a proof, namely that
\isa{bool} satisfies all axioms of \isa{ordrel}.
There are none, but we still need to finish that proof, which we do
by invoking the \methdx{intro_classes} method:%
\end{isamarkuptxt}%
\isamarkuptrue%
\isacommand{by}\isamarkupfalse%
\ intro{\isacharunderscore}classes%
\endisatagproof
{\isafoldproof}%
%
\isadelimproof
%
\endisadelimproof
%
\begin{isamarkuptext}%
\noindent
More interesting \isacommand{instance} proofs will arise below
in the context of proper axiomatic type classes.

Although terms like \isa{False\ {\isacharless}{\isacharless}{\isacharequal}\ P} are now legal, we still need to say
what the relation symbols actually mean at type \isa{bool}:%
\end{isamarkuptext}%
\isamarkuptrue%
\isacommand{defs}\isamarkupfalse%
\ {\isacharparenleft}\isakeyword{overloaded}{\isacharparenright}\isanewline
le{\isacharunderscore}bool{\isacharunderscore}def{\isacharcolon}\ {\isachardoublequoteopen}P\ {\isacharless}{\isacharless}{\isacharequal}\ Q\ {\isasymequiv}\ P\ {\isasymlongrightarrow}\ Q{\isachardoublequoteclose}\isanewline
lt{\isacharunderscore}bool{\isacharunderscore}def{\isacharcolon}\ {\isachardoublequoteopen}P\ {\isacharless}{\isacharless}\ Q\ {\isasymequiv}\ {\isasymnot}P\ {\isasymand}\ Q{\isachardoublequoteclose}%
\begin{isamarkuptext}%
\noindent
Now \isa{False\ {\isacharless}{\isacharless}{\isacharequal}\ P} is provable:%
\end{isamarkuptext}%
\isamarkuptrue%
\isacommand{lemma}\isamarkupfalse%
\ {\isachardoublequoteopen}False\ {\isacharless}{\isacharless}{\isacharequal}\ P{\isachardoublequoteclose}\isanewline
%
\isadelimproof
%
\endisadelimproof
%
\isatagproof
\isacommand{by}\isamarkupfalse%
{\isacharparenleft}simp\ add{\isacharcolon}\ le{\isacharunderscore}bool{\isacharunderscore}def{\isacharparenright}%
\endisatagproof
{\isafoldproof}%
%
\isadelimproof
%
\endisadelimproof
%
\begin{isamarkuptext}%
\noindent
At this point, \isa{{\isacharbrackleft}{\isacharbrackright}\ {\isacharless}{\isacharless}{\isacharequal}\ {\isacharbrackleft}{\isacharbrackright}} is not even well-typed.
To make it well-typed,
we need to make lists a type of class \isa{ordrel}:%
\end{isamarkuptext}%
\isamarkuptrue%
%
\isadelimtheory
%
\endisadelimtheory
%
\isatagtheory
%
\endisatagtheory
{\isafoldtheory}%
%
\isadelimtheory
%
\endisadelimtheory
\end{isabellebody}%
%%% Local Variables:
%%% mode: latex
%%% TeX-master: "root"
%%% End:

%
\begin{isabellebody}%
\def\isabellecontext{Overloading}%
%
\isadelimtheory
%
\endisadelimtheory
%
\isatagtheory
%
\endisatagtheory
{\isafoldtheory}%
%
\isadelimtheory
%
\endisadelimtheory
%
\begin{isamarkuptext}%
Type classes allow \emph{overloading}; thus a constant may
have multiple definitions at non-overlapping types.%
\end{isamarkuptext}%
\isamarkuptrue%
%
\isamarkupsubsubsection{Overloading%
}
\isamarkuptrue%
%
\begin{isamarkuptext}%
We can introduce a binary infix addition operator \isa{{\isasymotimes}}
for arbitrary types by means of a type class:%
\end{isamarkuptext}%
\isamarkuptrue%
\isacommand{class}\isamarkupfalse%
\ plus\ {\isacharequal}\isanewline
\ \ \isakeyword{fixes}\ plus\ {\isacharcolon}{\isacharcolon}\ {\isachardoublequoteopen}{\isacharprime}a\ {\isasymRightarrow}\ {\isacharprime}a\ {\isasymRightarrow}\ {\isacharprime}a{\isachardoublequoteclose}\ {\isacharparenleft}\isakeyword{infixl}\ {\isachardoublequoteopen}{\isasymoplus}{\isachardoublequoteclose}\ {\isadigit{7}}{\isadigit{0}}{\isacharparenright}%
\begin{isamarkuptext}%
\noindent This introduces a new class \isa{plus},
along with a constant \isa{plus} with nice infix syntax.
\isa{plus} is also named \emph{class operation}.  The type
of \isa{plus} carries a sort constraint \isa{{\isachardoublequote}{\isacharprime}a\ {\isacharcolon}{\isacharcolon}\ plus{\isachardoublequote}} on its type variable, meaning that only types of class
\isa{plus} can be instantiated for \isa{{\isachardoublequote}{\isacharprime}a{\isachardoublequote}}.
To breathe life into \isa{plus} we need to declare a type
to be an \bfindex{instance} of \isa{plus}:%
\end{isamarkuptext}%
\isamarkuptrue%
\isacommand{instantiation}\isamarkupfalse%
\ nat\ {\isacharcolon}{\isacharcolon}\ plus\isanewline
\isakeyword{begin}%
\begin{isamarkuptext}%
\noindent Command \isacommand{instantiation} opens a local
theory context.  Here we can now instantiate \isa{plus} on
\isa{nat}:%
\end{isamarkuptext}%
\isamarkuptrue%
\isacommand{primrec}\isamarkupfalse%
\ plus{\isacharunderscore}nat\ {\isacharcolon}{\isacharcolon}\ {\isachardoublequoteopen}nat\ {\isasymRightarrow}\ nat\ {\isasymRightarrow}\ nat{\isachardoublequoteclose}\ \isakeyword{where}\isanewline
\ \ \ \ {\isachardoublequoteopen}{\isacharparenleft}{\isadigit{0}}{\isacharcolon}{\isacharcolon}nat{\isacharparenright}\ {\isasymoplus}\ n\ {\isacharequal}\ n{\isachardoublequoteclose}\isanewline
\ \ {\isacharbar}\ {\isachardoublequoteopen}Suc\ m\ {\isasymoplus}\ n\ {\isacharequal}\ Suc\ {\isacharparenleft}m\ {\isasymoplus}\ n{\isacharparenright}{\isachardoublequoteclose}%
\begin{isamarkuptext}%
\noindent Note that the name \isa{plus} carries a
suffix \isa{{\isacharunderscore}nat}; by default, the local name of a class operation
\isa{f} to be instantiated on type constructor \isa{{\isasymkappa}} is mangled
as \isa{f{\isacharunderscore}{\isasymkappa}}.  In case of uncertainty, these names may be inspected
using the \hyperlink{command.print-context}{\mbox{\isa{\isacommand{print{\isacharunderscore}context}}}} command or the corresponding
ProofGeneral button.

Although class \isa{plus} has no axioms, the instantiation must be
formally concluded by a (trivial) instantiation proof ``..'':%
\end{isamarkuptext}%
\isamarkuptrue%
\isacommand{instance}\isamarkupfalse%
%
\isadelimproof
\ %
\endisadelimproof
%
\isatagproof
\isacommand{{\isachardot}{\isachardot}}\isamarkupfalse%
%
\endisatagproof
{\isafoldproof}%
%
\isadelimproof
%
\endisadelimproof
%
\begin{isamarkuptext}%
\noindent More interesting \isacommand{instance} proofs will
arise below.

The instantiation is finished by an explicit%
\end{isamarkuptext}%
\isamarkuptrue%
\isacommand{end}\isamarkupfalse%
%
\begin{isamarkuptext}%
\noindent From now on, terms like \isa{Suc\ {\isacharparenleft}m\ {\isasymoplus}\ {\isadigit{2}}{\isacharparenright}} are
legal.%
\end{isamarkuptext}%
\isamarkuptrue%
\isacommand{instantiation}\isamarkupfalse%
\ {\isachardoublequoteopen}{\isacharasterisk}{\isachardoublequoteclose}\ {\isacharcolon}{\isacharcolon}\ {\isacharparenleft}plus{\isacharcomma}\ plus{\isacharparenright}\ plus\isanewline
\isakeyword{begin}%
\begin{isamarkuptext}%
\noindent Here we instantiate the product type \isa{{\isacharasterisk}} to
class \isa{plus}, given that its type arguments are of
class \isa{plus}:%
\end{isamarkuptext}%
\isamarkuptrue%
\isacommand{fun}\isamarkupfalse%
\ plus{\isacharunderscore}prod\ {\isacharcolon}{\isacharcolon}\ {\isachardoublequoteopen}{\isacharprime}a\ {\isasymtimes}\ {\isacharprime}b\ {\isasymRightarrow}\ {\isacharprime}a\ {\isasymtimes}\ {\isacharprime}b\ {\isasymRightarrow}\ {\isacharprime}a\ {\isasymtimes}\ {\isacharprime}b{\isachardoublequoteclose}\ \isakeyword{where}\isanewline
\ \ {\isachardoublequoteopen}{\isacharparenleft}x{\isacharcomma}\ y{\isacharparenright}\ {\isasymoplus}\ {\isacharparenleft}w{\isacharcomma}\ z{\isacharparenright}\ {\isacharequal}\ {\isacharparenleft}x\ {\isasymoplus}\ w{\isacharcomma}\ y\ {\isasymoplus}\ z{\isacharparenright}{\isachardoublequoteclose}%
\begin{isamarkuptext}%
\noindent Obviously, overloaded specifications may include
recursion over the syntactic structure of types.%
\end{isamarkuptext}%
\isamarkuptrue%
\isacommand{instance}\isamarkupfalse%
%
\isadelimproof
\ %
\endisadelimproof
%
\isatagproof
\isacommand{{\isachardot}{\isachardot}}\isamarkupfalse%
%
\endisatagproof
{\isafoldproof}%
%
\isadelimproof
%
\endisadelimproof
\isanewline
\isanewline
\isacommand{end}\isamarkupfalse%
%
\begin{isamarkuptext}%
\noindent This way we have encoded the canonical lifting of
binary operations to products by means of type classes.%
\end{isamarkuptext}%
\isamarkuptrue%
%
\isadelimtheory
%
\endisadelimtheory
%
\isatagtheory
%
\endisatagtheory
{\isafoldtheory}%
%
\isadelimtheory
%
\endisadelimtheory
\end{isabellebody}%
%%% Local Variables:
%%% mode: latex
%%% TeX-master: "root"
%%% End:

%
\begin{isabellebody}%
\def\isabellecontext{Overloading{\isadigit{2}}}%
%
\isadelimtheory
%
\endisadelimtheory
%
\isatagtheory
%
\endisatagtheory
{\isafoldtheory}%
%
\isadelimtheory
%
\endisadelimtheory
%
\begin{isamarkuptext}%
Of course this is not the only possible definition of the two relations.
Componentwise comparison of lists of equal length also makes sense. This time
the elements of the list must also be of class \isa{ordrel} to permit their
comparison:%
\end{isamarkuptext}%
\isamarkuptrue%
\isacommand{instance}\isamarkupfalse%
\ list\ {\isacharcolon}{\isacharcolon}\ {\isacharparenleft}ordrel{\isacharparenright}ordrel\isanewline
%
\isadelimproof
%
\endisadelimproof
%
\isatagproof
\isacommand{by}\isamarkupfalse%
\ intro{\isacharunderscore}classes%
\endisatagproof
{\isafoldproof}%
%
\isadelimproof
\isanewline
%
\endisadelimproof
\isanewline
\isacommand{defs}\isamarkupfalse%
\ {\isacharparenleft}\isakeyword{overloaded}{\isacharparenright}\isanewline
le{\isacharunderscore}list{\isacharunderscore}def{\isacharcolon}\ {\isachardoublequoteopen}xs\ {\isacharless}{\isacharless}{\isacharequal}\ {\isacharparenleft}ys{\isacharcolon}{\isacharcolon}{\isacharprime}a{\isacharcolon}{\isacharcolon}ordrel\ list{\isacharparenright}\ {\isasymequiv}\isanewline
\ \ \ \ \ \ \ \ \ \ \ \ \ \ size\ xs\ {\isacharequal}\ size\ ys\ {\isasymand}\ {\isacharparenleft}{\isasymforall}i{\isacharless}size\ xs{\isachardot}\ xs{\isacharbang}i\ {\isacharless}{\isacharless}{\isacharequal}\ ys{\isacharbang}i{\isacharparenright}{\isachardoublequoteclose}%
\begin{isamarkuptext}%
\noindent
The infix function \isa{{\isacharbang}} yields the nth element of a list.

\begin{warn}
A type constructor can be instantiated in only one way to
a given type class.  For example, our two instantiations of \isa{list} must
reside in separate theories with disjoint scopes.
\end{warn}%
\end{isamarkuptext}%
\isamarkuptrue%
%
\isamarkupsubsubsection{Predefined Overloading%
}
\isamarkuptrue%
%
\begin{isamarkuptext}%
HOL comes with a number of overloaded constants and corresponding classes.
The most important ones are listed in Table~\ref{tab:overloading} in the appendix. They are
defined on all numeric types and sometimes on other types as well, for example
$-$ and \isa{{\isasymle}} on sets.

In addition there is a special syntax for bounded quantifiers:
\begin{center}
\begin{tabular}{lcl}
\isa{{\isasymforall}x{\isasymle}y{\isachardot}\ P\ x} & \isa{{\isasymrightleftharpoons}} & \isa{{\isachardoublequote}{\isasymforall}x{\isachardot}\ x\ {\isasymle}\ y\ {\isasymlongrightarrow}\ P\ x{\isachardoublequote}} \\
\isa{{\isasymexists}x{\isasymle}y{\isachardot}\ P\ x} & \isa{{\isasymrightleftharpoons}} & \isa{{\isachardoublequote}{\isasymexists}x{\isachardot}\ x\ {\isasymle}\ y\ {\isasymand}\ P\ x{\isachardoublequote}}
\end{tabular}
\end{center}
And analogously for \isa{{\isacharless}} instead of \isa{{\isasymle}}.%
\end{isamarkuptext}%
\isamarkuptrue%
%
\isadelimtheory
%
\endisadelimtheory
%
\isatagtheory
%
\endisatagtheory
{\isafoldtheory}%
%
\isadelimtheory
%
\endisadelimtheory
\end{isabellebody}%
%%% Local Variables:
%%% mode: latex
%%% TeX-master: "root"
%%% End:


\index{overloading|)}

%
\begin{isabellebody}%
\def\isabellecontext{Axioms}%
%
\isamarkupsubsection{Axioms%
}
%
\begin{isamarkuptext}%
Now we want to attach axioms to our classes. Then we can reason on the
level of classes and the results will be applicable to all types in a class,
just as in axiomatic mathematics. These ideas are demonstrated by means of
our above ordering relations.%
\end{isamarkuptext}%
%
\isamarkupsubsubsection{Partial orders%
}
%
\begin{isamarkuptext}%
A \emph{partial order} is a subclass of \isa{ordrel}
where certain axioms need to hold:%
\end{isamarkuptext}%
\isacommand{axclass}\ parord\ {\isacharless}\ ordrel\isanewline
refl{\isacharcolon}\ \ \ \ {\isachardoublequote}x\ {\isacharless}{\isacharless}{\isacharequal}\ x{\isachardoublequote}\isanewline
trans{\isacharcolon}\ \ \ {\isachardoublequote}{\isasymlbrakk}\ x\ {\isacharless}{\isacharless}{\isacharequal}\ y{\isacharsemicolon}\ y\ {\isacharless}{\isacharless}{\isacharequal}\ z\ {\isasymrbrakk}\ {\isasymLongrightarrow}\ x\ {\isacharless}{\isacharless}{\isacharequal}\ z{\isachardoublequote}\isanewline
antisym{\isacharcolon}\ {\isachardoublequote}{\isasymlbrakk}\ x\ {\isacharless}{\isacharless}{\isacharequal}\ y{\isacharsemicolon}\ y\ {\isacharless}{\isacharless}{\isacharequal}\ x\ {\isasymrbrakk}\ {\isasymLongrightarrow}\ x\ {\isacharequal}\ y{\isachardoublequote}\isanewline
less{\isacharunderscore}le{\isacharcolon}\ {\isachardoublequote}x\ {\isacharless}{\isacharless}\ y\ {\isacharequal}\ {\isacharparenleft}x\ {\isacharless}{\isacharless}{\isacharequal}\ y\ {\isasymand}\ x\ {\isasymnoteq}\ y{\isacharparenright}{\isachardoublequote}%
\begin{isamarkuptext}%
\noindent
The first three axioms are the familiar ones, and the final one
requires that \isa{{\isacharless}{\isacharless}} and \isa{{\isacharless}{\isacharless}{\isacharequal}} are related as expected.
Note that behind the scenes, Isabelle has restricted the axioms to class
\isa{parord}. For example, this is what \isa{refl} really looks like:
\isa{{\isacharparenleft}{\isacharquery}x{\isasymColon}{\isacharquery}{\isacharprime}a{\isacharparenright}\ {\isacharless}{\isacharless}{\isacharequal}\ {\isacharquery}x}.

We have not made \isa{less{\isacharunderscore}le} a global definition because it would
fix once and for all that \isa{{\isacharless}{\isacharless}} is defined in terms of \isa{{\isacharless}{\isacharless}{\isacharequal}}.
There are however situations where it is the other way around, which such a
definition would complicate. The slight drawback of the above class is that
we need to define both \isa{{\isacharless}{\isacharless}{\isacharequal}} and \isa{{\isacharless}{\isacharless}} for each instance.

We can now prove simple theorems in this abstract setting, for example
that \isa{{\isacharless}{\isacharless}} is not symmetric:%
\end{isamarkuptext}%
\isacommand{lemma}\ {\isacharbrackleft}simp{\isacharbrackright}{\isacharcolon}\ {\isachardoublequote}{\isacharparenleft}x{\isacharcolon}{\isacharcolon}{\isacharprime}a{\isacharcolon}{\isacharcolon}parord{\isacharparenright}\ {\isacharless}{\isacharless}\ y\ {\isasymLongrightarrow}\ {\isacharparenleft}{\isasymnot}\ y\ {\isacharless}{\isacharless}\ x{\isacharparenright}\ {\isacharequal}\ True{\isachardoublequote}%
\begin{isamarkuptxt}%
\noindent
The conclusion is not simply \isa{{\isasymnot}\ y\ {\isacharless}{\isacharless}\ x} because the preprocessor
of the simplifier would turn this into \isa{{\isacharparenleft}y\ {\isacharless}{\isacharless}\ x{\isacharparenright}\ {\isacharequal}\ False}, thus yielding
a nonterminating rewrite rule. In the above form it is a generally useful
rule.
The type constraint is necessary because otherwise Isabelle would only assume
\isa{{\isacharprime}a{\isacharcolon}{\isacharcolon}ordrel} (as required in the type of \isa{{\isacharless}{\isacharless}}), in
which case the proposition is not a theorem.  The proof is easy:%
\end{isamarkuptxt}%
\isacommand{by}{\isacharparenleft}simp\ add{\isacharcolon}less{\isacharunderscore}le\ antisym{\isacharparenright}%
\begin{isamarkuptext}%
We could now continue in this vein and develop a whole theory of
results about partial orders. Eventually we will want to apply these results
to concrete types, namely the instances of the class. Thus we first need to
prove that the types in question, for example \isa{bool}, are indeed
instances of \isa{parord}:%
\end{isamarkuptext}%
\isacommand{instance}\ bool\ {\isacharcolon}{\isacharcolon}\ parord\isanewline
\isacommand{apply}\ intro{\isacharunderscore}classes%
\begin{isamarkuptxt}%
\noindent
This time \isa{intro{\isacharunderscore}classes} leaves us with the four axioms,
specialized to type \isa{bool}, as subgoals:
\begin{isabelle}%
\ {\isadigit{1}}{\isachardot}\ {\isasymAnd}x{\isasymColon}bool{\isachardot}\ x\ {\isacharless}{\isacharless}{\isacharequal}\ x\isanewline
\ {\isadigit{2}}{\isachardot}\ {\isasymAnd}{\isacharparenleft}x{\isasymColon}bool{\isacharparenright}\ {\isacharparenleft}y{\isasymColon}bool{\isacharparenright}\ z{\isasymColon}bool{\isachardot}\ x\ {\isacharless}{\isacharless}{\isacharequal}\ y\ {\isasymLongrightarrow}\ y\ {\isacharless}{\isacharless}{\isacharequal}\ z\ {\isasymLongrightarrow}\ x\ {\isacharless}{\isacharless}{\isacharequal}\ z\isanewline
\ {\isadigit{3}}{\isachardot}\ {\isasymAnd}{\isacharparenleft}x{\isasymColon}bool{\isacharparenright}\ y{\isasymColon}bool{\isachardot}\ x\ {\isacharless}{\isacharless}{\isacharequal}\ y\ {\isasymLongrightarrow}\ y\ {\isacharless}{\isacharless}{\isacharequal}\ x\ {\isasymLongrightarrow}\ x\ {\isacharequal}\ y\isanewline
\ {\isadigit{4}}{\isachardot}\ {\isasymAnd}{\isacharparenleft}x{\isasymColon}bool{\isacharparenright}\ y{\isasymColon}bool{\isachardot}\ {\isacharparenleft}x\ {\isacharless}{\isacharless}\ y{\isacharparenright}\ {\isacharequal}\ {\isacharparenleft}x\ {\isacharless}{\isacharless}{\isacharequal}\ y\ {\isasymand}\ x\ {\isasymnoteq}\ y{\isacharparenright}%
\end{isabelle}
Fortunately, the proof is easy for blast, once we have unfolded the definitions
of \isa{{\isacharless}{\isacharless}} and \isa{{\isacharless}{\isacharless}{\isacharequal}} at \isa{bool}:%
\end{isamarkuptxt}%
\isacommand{apply}{\isacharparenleft}simp{\isacharunderscore}all\ {\isacharparenleft}no{\isacharunderscore}asm{\isacharunderscore}use{\isacharparenright}\ only{\isacharcolon}\ le{\isacharunderscore}bool{\isacharunderscore}def\ less{\isacharunderscore}bool{\isacharunderscore}def{\isacharparenright}\isanewline
\isacommand{by}{\isacharparenleft}blast{\isacharcomma}\ blast{\isacharcomma}\ blast{\isacharcomma}\ blast{\isacharparenright}%
\begin{isamarkuptext}%
\noindent
Can you figure out why we have to include \isa{{\isacharparenleft}no{\isacharunderscore}asm{\isacharunderscore}use{\isacharparenright}}?

We can now apply our single lemma above in the context of booleans:%
\end{isamarkuptext}%
\isacommand{lemma}\ {\isachardoublequote}{\isacharparenleft}P{\isacharcolon}{\isacharcolon}bool{\isacharparenright}\ {\isacharless}{\isacharless}\ Q\ {\isasymLongrightarrow}\ {\isasymnot}{\isacharparenleft}Q\ {\isacharless}{\isacharless}\ P{\isacharparenright}{\isachardoublequote}\isanewline
\isacommand{by}\ simp%
\begin{isamarkuptext}%
\noindent
The effect is not stunning but demonstrates the principle.  It also shows
that tools like the simplifier can deal with generic rules as well. Moreover,
it should be clear that the main advantage of the axiomatic method is that
theorems can be proved in the abstract and one does not need to repeat the
proof for each instance.%
\end{isamarkuptext}%
%
\isamarkupsubsubsection{Linear orders%
}
%
\begin{isamarkuptext}%
If any two elements of a partial order are comparable it is a
\emph{linear} or \emph{total} order:%
\end{isamarkuptext}%
\isacommand{axclass}\ linord\ {\isacharless}\ parord\isanewline
total{\isacharcolon}\ {\isachardoublequote}x\ {\isacharless}{\isacharless}{\isacharequal}\ y\ {\isasymor}\ y\ {\isacharless}{\isacharless}{\isacharequal}\ x{\isachardoublequote}%
\begin{isamarkuptext}%
\noindent
By construction, \isa{linord} inherits all axioms from \isa{parord}.
Therefore we can show that totality can be expressed in terms of \isa{{\isacharless}{\isacharless}}
as follows:%
\end{isamarkuptext}%
\isacommand{lemma}\ {\isachardoublequote}{\isasymAnd}x{\isacharcolon}{\isacharcolon}{\isacharprime}a{\isacharcolon}{\isacharcolon}linord{\isachardot}\ x{\isacharless}{\isacharless}y\ {\isasymor}\ x{\isacharequal}y\ {\isasymor}\ y{\isacharless}{\isacharless}x{\isachardoublequote}\isanewline
\isacommand{apply}{\isacharparenleft}simp\ add{\isacharcolon}\ less{\isacharunderscore}le{\isacharparenright}\isanewline
\isacommand{apply}{\isacharparenleft}insert\ total{\isacharparenright}\isanewline
\isacommand{apply}\ blast\isanewline
\isacommand{done}%
\begin{isamarkuptext}%
Linear orders are an example of subclassing by construction, which is the most
common case. It is also possible to prove additional subclass relationships
later on, i.e.\ subclassing by proof. This is the topic of the following
paragraph.%
\end{isamarkuptext}%
%
\isamarkupsubsubsection{Strict orders%
}
%
\begin{isamarkuptext}%
An alternative axiomatization of partial orders takes \isa{{\isacharless}{\isacharless}} rather than
\isa{{\isacharless}{\isacharless}{\isacharequal}} as the primary concept. The result is a \emph{strict} order:%
\end{isamarkuptext}%
\isacommand{axclass}\ strord\ {\isacharless}\ ordrel\isanewline
irrefl{\isacharcolon}\ \ \ \ \ {\isachardoublequote}{\isasymnot}\ x\ {\isacharless}{\isacharless}\ x{\isachardoublequote}\isanewline
less{\isacharunderscore}trans{\isacharcolon}\ {\isachardoublequote}{\isasymlbrakk}\ x\ {\isacharless}{\isacharless}\ y{\isacharsemicolon}\ y\ {\isacharless}{\isacharless}\ z\ {\isasymrbrakk}\ {\isasymLongrightarrow}\ x\ {\isacharless}{\isacharless}\ z{\isachardoublequote}\isanewline
le{\isacharunderscore}less{\isacharcolon}\ \ \ \ {\isachardoublequote}x\ {\isacharless}{\isacharless}{\isacharequal}\ y\ {\isacharequal}\ {\isacharparenleft}x\ {\isacharless}{\isacharless}\ y\ {\isasymor}\ x\ {\isacharequal}\ y{\isacharparenright}{\isachardoublequote}%
\begin{isamarkuptext}%
\noindent
It is well known that partial orders are the same as strict orders. Let us
prove one direction, namely that partial orders are a subclass of strict
orders. The proof follows the ususal pattern:%
\end{isamarkuptext}%
\isacommand{instance}\ parord\ {\isacharless}\ strord\isanewline
\isacommand{apply}\ intro{\isacharunderscore}classes\isanewline
\isacommand{apply}{\isacharparenleft}simp{\isacharunderscore}all\ {\isacharparenleft}no{\isacharunderscore}asm{\isacharunderscore}use{\isacharparenright}\ add{\isacharcolon}less{\isacharunderscore}le{\isacharparenright}\isanewline
\ \ \isacommand{apply}{\isacharparenleft}blast\ intro{\isacharcolon}\ trans\ antisym{\isacharparenright}\isanewline
\ \isacommand{apply}{\isacharparenleft}blast\ intro{\isacharcolon}\ refl{\isacharparenright}\isanewline
\isacommand{done}%
\begin{isamarkuptext}%
The subclass relation must always be acyclic. Therefore Isabelle will
complain if you also prove the relationship \isa{strord\ {\isacharless}\ parord}.%
\end{isamarkuptext}%
%
\isamarkupsubsubsection{Multiple inheritance and sorts%
}
%
\begin{isamarkuptext}%
A class may inherit from more than one direct superclass. This is called
multiple inheritance and is certainly permitted. For example we could define
the classes of well-founded orderings and well-orderings:%
\end{isamarkuptext}%
\isacommand{axclass}\ wford\ {\isacharless}\ parord\isanewline
wford{\isacharcolon}\ {\isachardoublequote}wf\ {\isacharbraceleft}{\isacharparenleft}y{\isacharcomma}x{\isacharparenright}{\isachardot}\ y\ {\isacharless}{\isacharless}\ x{\isacharbraceright}{\isachardoublequote}\isanewline
\isanewline
\isacommand{axclass}\ wellord\ {\isacharless}\ linord{\isacharcomma}\ wford%
\begin{isamarkuptext}%
\noindent
The last line expresses the usual definition: a well-ordering is a linear
well-founded ordering. The result is the subclass diagram in
Figure~\ref{fig:subclass}.

\begin{figure}[htbp]
\[
\begin{array}{r@ {}r@ {}c@ {}l@ {}l}
& & \isa{term}\\
& & |\\
& & \isa{ordrel}\\
& & |\\
& & \isa{strord}\\
& & |\\
& & \isa{parord} \\
& / & & \backslash \\
\isa{linord} & & & & \isa{wford} \\
& \backslash & & / \\
& & \isa{wellord}
\end{array}
\]
\caption{Subclass diagramm}
\label{fig:subclass}
\end{figure}

Since class \isa{wellord} does not introduce any new axioms, it can simply
be viewed as the intersection of the two classes \isa{linord} and \isa{wford}. Such intersections need not be given a new name but can be created on
the fly: the expression $\{C@1,\dots,C@n\}$, where the $C@i$ are classes,
represents the intersection of the $C@i$. Such an expression is called a
\bfindex{sort}, and sorts can appear in most places where we have only shown
classes so far, for example in type constraints: \isa{{\isacharprime}a{\isacharcolon}{\isacharcolon}{\isacharbraceleft}linord{\isacharcomma}wford{\isacharbraceright}}.
In fact, \isa{{\isacharprime}a{\isacharcolon}{\isacharcolon}ord} is short for \isa{{\isacharprime}a{\isacharcolon}{\isacharcolon}{\isacharbraceleft}ord{\isacharbraceright}}.
However, we do not pursue this rarefied concept further.

This concludes our demonstration of type classes based on orderings.  We
remind our readers that \isa{Main} contains a theory of
orderings phrased in terms of the usual \isa{{\isasymle}} and \isa{{\isacharless}}.
It is recommended that, if possible,
you base your own ordering relations on this theory.%
\end{isamarkuptext}%
%
\isamarkupsubsubsection{Inconsistencies%
}
%
\begin{isamarkuptext}%
The reader may be wondering what happens if we, maybe accidentally,
attach an inconsistent set of axioms to a class. So far we have always
avoided to add new axioms to HOL for fear of inconsistencies and suddenly it
seems that we are throwing all caution to the wind. So why is there no
problem?

The point is that by construction, all type variables in the axioms of an
\isacommand{axclass} are automatically constrained with the class being
defined (as shown for axiom \isa{refl} above). These constraints are
always carried around and Isabelle takes care that they are never lost,
unless the type variable is instantiated with a type that has been shown to
belong to that class. Thus you may be able to prove \isa{False}
from your axioms, but Isabelle will remind you that this
theorem has the hidden hypothesis that the class is non-empty.%
\end{isamarkuptext}%
\end{isabellebody}%
%%% Local Variables:
%%% mode: latex
%%% TeX-master: "root"
%%% End:


\index{axiomatic type class|)}
\index{*axclass|)}
