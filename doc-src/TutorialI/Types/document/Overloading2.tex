%
\begin{isabellebody}%
\def\isabellecontext{Overloading{\isadigit{2}}}%
%
\begin{isamarkuptext}%
Of course this is not the only possible definition of the two relations.
Componentwise comparison of lists of equal length also makes sense. This time
the elements of the list must also be of class \isa{ordrel} to permit their
comparison:%
\end{isamarkuptext}%
\isacommand{instance}\ list\ {\isacharcolon}{\isacharcolon}\ {\isacharparenleft}ordrel{\isacharparenright}ordrel\isanewline
\isacommand{by}\ intro{\isacharunderscore}classes\isanewline
\isanewline
\isacommand{defs}\ {\isacharparenleft}\isakeyword{overloaded}{\isacharparenright}\isanewline
le{\isacharunderscore}list{\isacharunderscore}def{\isacharcolon}\ {\isachardoublequote}xs\ {\isacharless}{\isacharless}{\isacharequal}\ {\isacharparenleft}ys{\isacharcolon}{\isacharcolon}{\isacharprime}a{\isacharcolon}{\isacharcolon}ordrel\ list{\isacharparenright}\ {\isasymequiv}\isanewline
\ \ \ \ \ \ \ \ \ \ \ \ \ \ size\ xs\ {\isacharequal}\ size\ ys\ {\isasymand}\ {\isacharparenleft}{\isasymforall}i{\isacharless}size\ xs{\isachardot}\ xs{\isacharbang}i\ {\isacharless}{\isacharless}{\isacharequal}\ ys{\isacharbang}i{\isacharparenright}{\isachardoublequote}%
\begin{isamarkuptext}%
\noindent
The infix function \isa{{\isacharbang}} yields the nth element of a list.%
\end{isamarkuptext}%
%
\isamarkupsubsubsection{Predefined overloading%
}
%
\begin{isamarkuptext}%
HOL comes with a number of overloaded constants and corresponding classes.
The most important ones are listed in Table~\ref{tab:overloading}. They are
defined on all numeric types and somtimes on other types as well, for example
\isa{{\isacharminus}}, \isa{{\isasymle}} and \isa{{\isacharless}} on sets.

\begin{table}[htbp]
\begin{center}
\begin{tabular}{lll}
Constant & Type & Syntax \\
\hline
\isa{{\isadigit{0}}} & \isa{{\isacharprime}a{\isacharcolon}{\isacharcolon}zero} \\
\isa{{\isacharplus}} & \isa{{\isacharparenleft}{\isacharprime}a{\isacharcolon}{\isacharcolon}plus{\isacharparenright}\ {\isasymRightarrow}\ {\isacharprime}a\ {\isasymRightarrow}\ {\isacharprime}a} & (infixl 65) \\
\isa{{\isacharminus}} & \isa{{\isacharparenleft}{\isacharprime}a{\isacharcolon}{\isacharcolon}minus{\isacharparenright}\ {\isasymRightarrow}\ {\isacharprime}a\ {\isasymRightarrow}\ {\isacharprime}a} &  (infixl 65) \\
\isa{{\isacharasterisk}} & \isa{{\isacharparenleft}{\isacharprime}a{\isacharcolon}{\isacharcolon}times{\isacharparenright}\ {\isasymRightarrow}\ {\isacharprime}a\ {\isasymRightarrow}\ {\isacharprime}a} & (infixl 70) \\
\isa{{\isacharcircum}} & \isa{{\isacharparenleft}{\isacharprime}a{\isacharcolon}{\isacharcolon}power{\isacharparenright}\ {\isasymRightarrow}\ nat\ {\isasymRightarrow}\ {\isacharprime}a} & (infixr 80) \\
\isa{{\isacharminus}} & \isa{{\isacharparenleft}{\isacharprime}a{\isacharcolon}{\isacharcolon}minus{\isacharparenright}\ {\isasymRightarrow}\ {\isacharprime}a} \\
\isa{abs} &  \isa{{\isacharparenleft}{\isacharprime}a{\isacharcolon}{\isacharcolon}minus{\isacharparenright}\ {\isasymRightarrow}\ {\isacharprime}a} & ${\mid} x {\mid}$\\
\isa{{\isasymle}} & \isa{{\isacharparenleft}{\isacharprime}a{\isacharcolon}{\isacharcolon}ord{\isacharparenright}\ {\isasymRightarrow}\ {\isacharprime}a\ {\isasymRightarrow}\ bool} & (infixl 50) \\
\isa{{\isacharless}} & \isa{{\isacharparenleft}{\isacharprime}a{\isacharcolon}{\isacharcolon}ord{\isacharparenright}\ {\isasymRightarrow}\ {\isacharprime}a\ {\isasymRightarrow}\ bool} & (infixl 50) \\
\isa{min} &  \isa{{\isacharparenleft}{\isacharprime}a{\isacharcolon}{\isacharcolon}ord{\isacharparenright}\ {\isasymRightarrow}\ {\isacharprime}a\ {\isasymRightarrow}\ {\isacharprime}a} \\
\isa{max} &  \isa{{\isacharparenleft}{\isacharprime}a{\isacharcolon}{\isacharcolon}ord{\isacharparenright}\ {\isasymRightarrow}\ {\isacharprime}a\ {\isasymRightarrow}\ {\isacharprime}a} \\
\isa{Least} & \isa{{\isacharparenleft}{\isacharprime}a{\isacharcolon}{\isacharcolon}ord\ {\isasymRightarrow}\ bool{\isacharparenright}\ {\isasymRightarrow}\ {\isacharprime}a} &
\isa{LEAST}$~x.~P$
\end{tabular}
\caption{Overloaded constants in HOL}
\label{tab:overloading}
\end{center}
\end{table}

In addition there is a special input syntax for bounded quantifiers:
\begin{center}
\begin{tabular}{lcl}
\isa{{\isasymforall}x\ {\isasymle}\ y{\isachardot}\ P\ x} & \isa{{\isasymequiv}} & \isa{{\isasymforall}x{\isachardot}\ x\ {\isasymle}\ y\ {\isasymlongrightarrow}\ P\ x} \\
\isa{{\isasymexists}x\ {\isasymle}\ y{\isachardot}\ P\ x} & \isa{{\isasymequiv}} & \isa{{\isasymexists}x{\isachardot}\ x\ {\isasymle}\ y\ {\isasymand}\ P\ x}
\end{tabular}
\end{center}
And analogously for \isa{{\isacharless}} instead of \isa{{\isasymle}}.
The form on the left is translated into the one on the right upon input but it is not
translated back upon output.%
\end{isamarkuptext}%
\end{isabellebody}%
%%% Local Variables:
%%% mode: latex
%%% TeX-master: "root"
%%% End:
