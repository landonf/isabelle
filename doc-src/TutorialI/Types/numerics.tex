% $Id$

\section{Numbers}
\label{sec:numbers}

Until now, our numerical examples have used the type of \textbf{natural
numbers},
\isa{nat}.  This is a recursive datatype generated by the constructors
zero  and successor, so it works well with inductive proofs and primitive
recursive function definitions.  HOL also provides the type
\isa{int} of \textbf{integers}, which lack induction but support true
subtraction.  The integers are preferable to the natural numbers for reasoning about
complicated arithmetic expressions, even for some expressions whose
value is non-negative.  The logic HOL-Real also has the type
\isa{real} of real numbers.  Isabelle has no subtyping,  so the numeric
types are distinct and there are  functions to convert between them.
Fortunately most numeric operations are overloaded: the same symbol can be
used at all numeric types. Table~\ref{tab:overloading} in the appendix
shows the most important operations, together with the priorities of the
infix symbols.


Many theorems involving numeric types can be proved automatically by
Isabelle's arithmetic decision procedure, the method
\isa{arith}.  Linear arithmetic comprises addition, subtraction
and multiplication by constant factors; subterms involving other operators
are regarded as variables.  The procedure can be slow, especially if the
subgoal to be proved involves subtraction over type \isa{nat}, which 
causes case splits.  

The simplifier reduces arithmetic expressions in other
ways, such as dividing through by common factors.  For problems that lie
outside the scope of automation, HOL provides hundreds of
theorems about multiplication, division, etc., that can be brought to
bear.  You can locate them using Proof General's Find
button.  A few lemmas are given below to show what
is available.

\subsection{Numeric Literals}
\label{sec:numerals}

Literals are available for the types of natural numbers, integers 
and reals and denote integer values of arbitrary size. 
They begin 
with a number sign (\isa{\#}), have an optional minus sign (\isa{-}) and 
then one or more decimal digits. Examples are \isa{\#0}, \isa{\#-3} 
and \isa{\#441223334678}.

Literals look like constants, but they abbreviate 
terms, representing the number in a two's complement binary notation. 
Isabelle performs arithmetic on literals by rewriting rather 
than using the hardware arithmetic. In most cases arithmetic 
is fast enough, even for large numbers. The arithmetic operations 
provided for literals include addition, subtraction, multiplication, 
integer division and remainder.  Fractions of literals (expressed using
division) are reduced to lowest terms.

\begin{warn}
The arithmetic operators are 
overloaded, so you must be careful to ensure that each numeric 
expression refers to a specific type, if necessary by inserting 
type constraints.  Here is an example of what can go wrong:
\par
\begin{isabelle}
\isacommand{lemma}\ "\#2\ *\ m\ =\ m\ +\ m"
\end{isabelle}
%
Carefully observe how Isabelle displays the subgoal:
\begin{isabelle}
\ 1.\ (\#2::'a)\ *\ m\ =\ m\ +\ m
\end{isabelle}
The type \isa{'a} given for the literal \isa{\#2} warns us that no numeric
type has been specified.  The problem is underspecified.  Given a type
constraint such as \isa{nat}, \isa{int} or \isa{real}, it becomes trivial.
\end{warn}

\begin{warn}
Numeric literals are not constructors and therefore must not be used in
patterns.  For example, this declaration is rejected:
\begin{isabelle}
\isacommand{recdef}\ h\ "\isacharbraceleft \isacharbraceright "\isanewline
"h\ \#3\ =\ \#2"\isanewline
"h\ i\ \ =\ i"
\end{isabelle}

You should use a conditional expression instead:
\begin{isabelle}
"h\ i\ =\ (if\ i\ =\ \#3\ then\ \#2\ else\ i)"
\end{isabelle}
\end{warn}



\subsection{The Type of Natural Numbers, {\tt\slshape nat}}

This type requires no introduction: we have been using it from the
beginning.  Hundreds of theorems about the natural numbers are
proved in the theories \isa{Nat}, \isa{NatArith} and \isa{Divides}.  Only
in exceptional circumstances should you resort to induction.

\subsubsection{Literals}
The notational options for the natural numbers can be confusing. The 
constant \isa{0} is overloaded to serve as the neutral value 
in a variety of additive types. The symbols \isa{1} and \isa{2} are 
not constants but abbreviations for \isa{Suc 0} and \isa{Suc(Suc 0)},
respectively. The literals \isa{\#0}, \isa{\#1} and \isa{\#2}  are
syntactically different from \isa{0}, \isa{1} and \isa{2}. You  will
sometimes prefer one notation to the other. Literals are  obviously
necessary to express large values, while \isa{0} and \isa{Suc}  are needed
in order to match many theorems, including the rewrite  rules for primitive
recursive functions. The following default  simplification rules replace
small literals by zero and successor: 
\begin{isabelle}
\#0\ =\ 0
\rulename{numeral_0_eq_0}\isanewline
\#1\ =\ 1
\rulename{numeral_1_eq_1}\isanewline
\#2\ +\ n\ =\ Suc\ (Suc\ n)
\rulename{add_2_eq_Suc}\isanewline
n\ +\ \#2\ =\ Suc\ (Suc\ n)
\rulename{add_2_eq_Suc'}
\end{isabelle}
In special circumstances, you may wish to remove or reorient 
these rules. 

\subsubsection{Typical lemmas}
Inequalities involving addition and subtraction alone can be proved
automatically.  Lemmas such as these can be used to prove inequalities
involving multiplication and division:
\begin{isabelle}
\isasymlbrakk i\ \isasymle \ j;\ k\ \isasymle \ l\isasymrbrakk \ \isasymLongrightarrow \ i\ *\ k\ \isasymle \ j\ *\ l%
\rulename{mult_le_mono}\isanewline
\isasymlbrakk i\ <\ j;\ 0\ <\ k\isasymrbrakk \ \isasymLongrightarrow \ i\
*\ k\ <\ j\ *\ k%
\rulename{mult_less_mono1}\isanewline
m\ \isasymle \ n\ \isasymLongrightarrow \ m\ div\ k\ \isasymle \ n\ div\ k%
\rulename{div_le_mono}
\end{isabelle}
%
Various distributive laws concerning multiplication are available:
\begin{isabelle}
(m\ +\ n)\ *\ k\ =\ m\ *\ k\ +\ n\ *\ k%
\rulename{add_mult_distrib}\isanewline
(m\ -\ n)\ *\ k\ =\ m\ *\ k\ -\ n\ *\ k%
\rulename{diff_mult_distrib}\isanewline
(m\ mod\ n)\ *\ k\ =\ (m\ *\ k)\ mod\ (n\ *\ k)
\rulename{mod_mult_distrib}
\end{isabelle}

\subsubsection{Division}
The infix operators \isa{div} and \isa{mod} are overloaded.
Isabelle/HOL provides the basic facts about quotient and remainder
on the natural numbers:
\begin{isabelle}
m\ mod\ n\ =\ (if\ m\ <\ n\ then\ m\ else\ (m\ -\ n)\ mod\ n)
\rulename{mod_if}\isanewline
m\ div\ n\ *\ n\ +\ m\ mod\ n\ =\ m%
\rulename{mod_div_equality}
\end{isabelle}

Many less obvious facts about quotient and remainder are also provided. 
Here is a selection:
\begin{isabelle}
a\ *\ b\ div\ c\ =\ a\ *\ (b\ div\ c)\ +\ a\ *\ (b\ mod\ c)\ div\ c%
\rulename{div_mult1_eq}\isanewline
a\ *\ b\ mod\ c\ =\ a\ *\ (b\ mod\ c)\ mod\ c%
\rulename{mod_mult1_eq}\isanewline
a\ div\ (b*c)\ =\ a\ div\ b\ div\ c%
\rulename{div_mult2_eq}\isanewline
a\ mod\ (b*c)\ =\ b * (a\ div\ b\ mod\ c)\ +\ a\ mod\ b%
\rulename{mod_mult2_eq}\isanewline
0\ <\ c\ \isasymLongrightarrow \ (c\ *\ a)\ div\ (c\ *\ b)\ =\ a\ div\ b%
\rulename{div_mult_mult1}
\end{isabelle}

Surprisingly few of these results depend upon the
divisors' being nonzero.  That is because division by
zero yields zero:
\begin{isabelle}
a\ div\ 0\ =\ 0
\rulename{DIVISION_BY_ZERO_DIV}\isanewline
a\ mod\ 0\ =\ a%
\rulename{DIVISION_BY_ZERO_MOD}
\end{isabelle}
As a concession to convention, these equations are not installed as default
simplification rules.  In \isa{div_mult_mult1} above, one of
the two divisors (namely~\isa{c}) must still be nonzero.

The \textbf{divides} relation has the standard definition, which
is overloaded over all numeric types: 
\begin{isabelle}
m\ dvd\ n\ \isasymequiv\ {\isasymexists}k.\ n\ =\ m\ *\ k
\rulename{dvd_def}
\end{isabelle}
%
Section~\ref{sec:proving-euclid} discusses proofs involving this
relation.  Here are some of the facts proved about it:
\begin{isabelle}
\isasymlbrakk m\ dvd\ n;\ n\ dvd\ m\isasymrbrakk \ \isasymLongrightarrow \ m\ =\ n%
\rulename{dvd_anti_sym}\isanewline
\isasymlbrakk k\ dvd\ m;\ k\ dvd\ n\isasymrbrakk \ \isasymLongrightarrow \ k\ dvd\ (m\ +\ n)
\rulename{dvd_add}
\end{isabelle}

\subsubsection{Simplifier Tricks}
The rule \isa{diff_mult_distrib} shown above is one of the few facts
about \isa{m\ -\ n} that is not subject to
the condition \isa{n\ \isasymle \  m}.  Natural number subtraction has few
nice properties; often you should remove it by simplifying with this split
rule:
\begin{isabelle}
P(a-b)\ =\ ((a<b\ \isasymlongrightarrow \ P\
0)\ \isasymand \ (\isasymforall d.\ a\ =\ b+d\ \isasymlongrightarrow \ P\
d))
\rulename{nat_diff_split}
\end{isabelle}
For example, it proves the following fact, which lies outside the scope of
linear arithmetic:
\begin{isabelle}
\isacommand{lemma}\ "(n-1)*(n+1)\ =\ n*n\ -\ 1"\isanewline
\isacommand{apply}\ (simp\ split:\ nat_diff_split)\isanewline
\isacommand{done}
\end{isabelle}

Suppose that two expressions are equal, differing only in 
associativity and commutativity of addition.  Simplifying with the
following equations sorts the terms and groups them to the right, making
the two expressions identical:
\begin{isabelle}
m\ +\ n\ +\ k\ =\ m\ +\ (n\ +\ k)
\rulename{add_assoc}\isanewline
m\ +\ n\ =\ n\ +\ m%
\rulename{add_commute}\isanewline
x\ +\ (y\ +\ z)\ =\ y\ +\ (x\
+\ z)
\rulename{add_left_commute}
\end{isabelle}
The name \isa{add_ac} refers to the list of all three theorems, similarly
there is \isa{mult_ac}.  Here is an example of the sorting effect.  Start
with this goal:
\begin{isabelle}
\ 1.\ Suc\ (i\ +\ j\ *\ l\ *\ k\ +\ m\ *\ n)\ =\
f\ (n\ *\ m\ +\ i\ +\ k\ *\ j\ *\ l)
\end{isabelle}
%
Simplify using  \isa{add_ac} and \isa{mult_ac}:
\begin{isabelle}
\isacommand{apply}\ (simp\ add:\ add_ac\ mult_ac)
\end{isabelle}
%
Here is the resulting subgoal:
\begin{isabelle}
\ 1.\ Suc\ (i\ +\ (m\ *\ n\ +\ j\ *\ (k\ *\ l)))\
=\ f\ (i\ +\ (m\ *\ n\ +\ j\ *\ (k\ *\ l)))%
\end{isabelle}


\subsection{The Type of Integers, {\tt\slshape int}}

Reasoning methods resemble those for the natural numbers, but induction and
the constant \isa{Suc} are not available.  HOL provides many lemmas
for proving inequalities involving integer multiplication and division,
similar to those shown above for type~\isa{nat}.  

The absolute value function \isa{abs} is overloaded for the numeric types.
It is defined for the integers; we have for example the obvious law
\begin{isabelle}
\isasymbar x\ *\ y\isasymbar \ =\ \isasymbar x\isasymbar \ *\ \isasymbar y\isasymbar 
\rulename{abs_mult}
\end{isabelle}

\begin{warn}
The absolute value bars shown above cannot be typed on a keyboard.  They
can be entered using the X-symbol package.  In \textsc{ascii}, type \isa{abs x} to
get \isa{\isasymbar x\isasymbar}.
\end{warn}

The \isa{arith} method can prove facts about \isa{abs} automatically, 
though as it does so by case analysis, the cost can be exponential.
\begin{isabelle}
\isacommand{lemma}\ "abs\ (x+y)\ \isasymle \ abs\ x\ +\ abs\ (y\ ::\ int)"\isanewline
\isacommand{by}\ arith
\end{isabelle}

Concerning simplifier tricks, we have no need to eliminate subtraction: it
is well-behaved.  As with the natural numbers, the simplifier can sort the
operands of sums and products.  The name \isa{zadd_ac} refers to the
associativity and commutativity theorems for integer addition, while
\isa{zmult_ac} has the analogous theorems for multiplication.  The
prefix~\isa{z} in many theorem names recalls the use of $\mathbb{Z}$ to
denote the set of integers.

For division and remainder, the treatment of negative divisors follows
mathematical practice: the sign of the remainder follows that
of the divisor:
\begin{isabelle}
\#0\ <\ b\ \isasymLongrightarrow \ \#0\ \isasymle \ a\ mod\ b%
\rulename{pos_mod_sign}\isanewline
\#0\ <\ b\ \isasymLongrightarrow \ a\ mod\ b\ <\ b%
\rulename{pos_mod_bound}\isanewline
b\ <\ \#0\ \isasymLongrightarrow \ a\ mod\ b\ \isasymle \ \#0
\rulename{neg_mod_sign}\isanewline
b\ <\ \#0\ \isasymLongrightarrow \ b\ <\ a\ mod\ b%
\rulename{neg_mod_bound}
\end{isabelle}
ML treats negative divisors in the same way, but most computer hardware
treats signed operands using the same rules as for multiplication.
Many facts about quotients and remainders are provided:
\begin{isabelle}
(a\ +\ b)\ div\ c\ =\isanewline
a\ div\ c\ +\ b\ div\ c\ +\ (a\ mod\ c\ +\ b\ mod\ c)\ div\ c%
\rulename{zdiv_zadd1_eq}
\par\smallskip
(a\ +\ b)\ mod\ c\ =\ (a\ mod\ c\ +\ b\ mod\ c)\ mod\ c%
\rulename{zmod_zadd1_eq}
\end{isabelle}

\begin{isabelle}
(a\ *\ b)\ div\ c\ =\ a\ *\ (b\ div\ c)\ +\ a\ *\ (b\ mod\ c)\ div\ c%
\rulename{zdiv_zmult1_eq}\isanewline
(a\ *\ b)\ mod\ c\ =\ a\ *\ (b\ mod\ c)\ mod\ c%
\rulename{zmod_zmult1_eq}
\end{isabelle}

\begin{isabelle}
\#0\ <\ c\ \isasymLongrightarrow \ a\ div\ (b*c)\ =\ a\ div\ b\ div\ c%
\rulename{zdiv_zmult2_eq}\isanewline
\#0\ <\ c\ \isasymLongrightarrow \ a\ mod\ (b*c)\ =\ b*(a\ div\ b\ mod\
c)\ +\ a\ mod\ b%
\rulename{zmod_zmult2_eq}
\end{isabelle}
The last two differ from their natural number analogues by requiring
\isa{c} to be positive.  Since division by zero yields zero, we could allow
\isa{c} to be zero.  However, \isa{c} cannot be negative: a counterexample
is
$\isa{a} = 7$, $\isa{b} = 2$ and $\isa{c} = -3$, when the left-hand side of
\isa{zdiv_zmult2_eq} is $-2$ while the right-hand side is~$-1$.


\subsection{The Type of Real Numbers, {\tt\slshape real}}

The real numbers enjoy two significant properties that the integers lack. 
They are
\textbf{dense}: between every two distinct real numbers there lies another.
This property follows from the division laws, since if $x<y$ then between
them lies $(x+y)/2$.  The second property is that they are
\textbf{complete}: every set of reals that is bounded above has a least
upper bound.  Completeness distinguishes the reals from the rationals, for
which the set $\{x\mid x^2<2\}$ has no least upper bound.  (It could only be
$\surd2$, which is irrational.)
The formalization of completeness is complicated; rather than
reproducing it here, we refer you to the theory \texttt{RComplete} in
directory \texttt{Real}.
Density, however, is trivial to express:
\begin{isabelle}
x\ <\ y\ \isasymLongrightarrow \ \isasymexists r.\ x\ <\ r\ \isasymand \ r\ <\ y%
\rulename{real_dense}
\end{isabelle}

Here is a selection of rules about the division operator.  The following
are installed as default simplification rules in order to express
combinations of products and quotients as rational expressions:
\begin{isabelle}
x\ *\ (y\ /\ z)\ =\ x\ *\ y\ /\ z
\rulename{real_times_divide1_eq}\isanewline
y\ /\ z\ *\ x\ =\ y\ *\ x\ /\ z
\rulename{real_times_divide2_eq}\isanewline
x\ /\ (y\ /\ z)\ =\ x\ *\ z\ /\ y
\rulename{real_divide_divide1_eq}\isanewline
x\ /\ y\ /\ z\ =\ x\ /\ (y\ *\ z)
\rulename{real_divide_divide2_eq}
\end{isabelle}

Signs are extracted from quotients in the hope that complementary terms can
then be cancelled:
\begin{isabelle}
-\ x\ /\ y\ =\ -\ (x\ /\ y)
\rulename{real_minus_divide_eq}\isanewline
x\ /\ -\ y\ =\ -\ (x\ /\ y)
\rulename{real_divide_minus_eq}
\end{isabelle}

The following distributive law is available, but it is not installed as a
simplification rule.
\begin{isabelle}
(x\ +\ y)\ /\ z\ =\ x\ /\ z\ +\ y\ /\ z%
\rulename{real_add_divide_distrib}
\end{isabelle}

As with the other numeric types, the simplifier can sort the operands of
addition and multiplication.  The name \isa{real_add_ac} refers to the
associativity and commutativity theorems for addition, while similarly
\isa{real_mult_ac} contains those properties for multiplication. 

The absolute value function \isa{abs} is
defined for the reals, along with many theorems such as this one about
exponentiation:
\begin{isabelle}
\isasymbar r\isasymbar \ \isacharcircum \ n\ =\ \isasymbar r\ \isacharcircum \ n\isasymbar 
\rulename{realpow_abs}
\end{isabelle}

Numeric literals for type \isa{real} have the same syntax as those for type
\isa{int} and only express integral values.  Fractions expressed
using the division operator are automatically simplified to lowest terms:
\begin{isabelle}
\ 1.\ P\ ((\#3\ /\ \#4)\ *\ (\#8\ /\ \#15))\isanewline
\isacommand{apply} simp\isanewline
\ 1.\ P\ (\#2\ /\ \#5)
\end{isabelle}
Exponentiation can express floating-point values such as
\isa{\#2 * \#10\isacharcircum\#6}, but at present no special simplification
is performed.


\begin{warn}
Type \isa{real} is only available in the logic HOL-Real, which
is  HOL extended with the rather substantial development of the real
numbers.  Base your theory upon theory
\isa{Real}, not the usual \isa{Main}.  Launch Isabelle using the command 
\begin{verbatim}
Isabelle -l HOL-Real
\end{verbatim}
\end{warn}

Also distributed with Isabelle is HOL-Hyperreal,
whose theory \isa{Hyperreal} defines the type \isa{hypreal} of non-standard
reals.  These
\textbf{hyperreals} include infinitesimals, which represent infinitely
small and infinitely large quantities; they facilitate proofs
about limits, differentiation and integration~\cite{fleuriot-jcm}.  The
development defines an infinitely large number, \isa{omega} and an
infinitely small positive number, \isa{epsilon}.  The 
relation $x\approx y$ means ``$x$ is infinitely close to~$y$''.
