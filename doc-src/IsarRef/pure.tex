
\chapter{Basic Isar Language Elements}\label{ch:pure-syntax}

Subsequently, we introduce the main part of Pure Isar theory and proof
commands, together with fundamental proof methods and attributes.
Chapter~\ref{ch:gen-tools} describes further Isar elements provided by generic
tools and packages (such as the Simplifier) that are either part of Pure
Isabelle or pre-installed by most object logics.  Chapter~\ref{ch:hol-tools}
refers to actual object-logic specific elements of Isabelle/HOL.

\medskip

Isar commands may be either \emph{proper} document constructors, or
\emph{improper commands}.  Some proof methods and attributes introduced later
are classified as improper as well.  Improper Isar language elements, which
are subsequently marked by $^*$, are often helpful when developing proof
documents, while their use is discouraged for the final outcome.  Typical
examples are diagnostic commands that print terms or theorems according to the
current context; other commands even emulate old-style tactical theorem
proving.


\section{Theory commands}

\subsection{Defining theories}\label{sec:begin-thy}

\indexisarcmd{header}\indexisarcmd{theory}\indexisarcmd{end}\indexisarcmd{context}
\begin{matharray}{rcl}
  \isarcmd{header} & : & \isarkeep{toplevel} \\
  \isarcmd{theory} & : & \isartrans{toplevel}{theory} \\
  \isarcmd{context}^* & : & \isartrans{toplevel}{theory} \\
  \isarcmd{end} & : & \isartrans{theory}{toplevel} \\
\end{matharray}

Isabelle/Isar ``new-style'' theories are either defined via theory files or
interactively.  Both theory-level specifications and proofs are handled
uniformly --- occasionally definitional mechanisms even require some explicit
proof as well.  In contrast, ``old-style'' Isabelle theories support batch
processing only, with the proof scripts collected in separate ML files.

The first actual command of any theory has to be $\THEORY$, starting a new
theory based on the merge of existing ones.  Just preceding $\THEORY$, there
may be an optional $\isarkeyword{header}$ declaration, which is relevant to
document preparation only; it acts very much like a special pre-theory markup
command (cf.\ \S\ref{sec:markup-thy} and \S\ref{sec:markup-thy}).  The theory
context may be also changed by $\CONTEXT$ without creating a new theory.  In
both cases, $\END$ concludes the theory development; it has to be the very
last command of any theory file.

\begin{rail}
  'header' text
  ;
  'theory' name '=' (name + '+') filespecs? ':'
  ;
  'context' name
  ;
  'end'
  ;;

  filespecs: 'files' ((name | parname) +);
\end{rail}

\begin{descr}
\item [$\isarkeyword{header}~text$] provides plain text markup just preceding
  the formal beginning of a theory.  In actual document preparation the
  corresponding {\LaTeX} macro \verb,\isamarkupheader, may be redefined to
  produce chapter or section headings.  See also \S\ref{sec:markup-thy} and
  \S\ref{sec:markup-prf} for further markup commands.
  
\item [$\THEORY~A = B@1 + \cdots + B@n\colon$] commences a new theory $A$
  based on existing ones $B@1 + \cdots + B@n$.  Isabelle's theory loader
  system ensures that any of the base theories are properly loaded (and fully
  up-to-date when $\THEORY$ is executed interactively).  The optional
  $\isarkeyword{files}$ specification declares additional dependencies on ML
  files.  Unless put in parentheses, any file will be loaded immediately via
  $\isarcmd{use}$ (see also \S\ref{sec:ML}).  The optional ML file
  \texttt{$A$.ML} that may be associated with any theory should \emph{not} be
  included in $\isarkeyword{files}$, though.
  
\item [$\CONTEXT~B$] enters an existing theory context, basically in read-only
  mode, so only a limited set of commands may be performed without destroying
  the theory.  Just as for $\THEORY$, the theory loader ensures that $B$ is
  loaded and up-to-date.
  
\item [$\END$] concludes the current theory definition or context switch.
Note that this command cannot be undone, but the whole theory definition has
to be retracted.
\end{descr}


\subsection{Theory markup commands}\label{sec:markup-thy}

\indexisarcmd{chapter}\indexisarcmd{section}\indexisarcmd{subsection}
\indexisarcmd{subsubsection}\indexisarcmd{text}\indexisarcmd{text-raw}
\begin{matharray}{rcl}
  \isarcmd{chapter} & : & \isartrans{theory}{theory} \\
  \isarcmd{section} & : & \isartrans{theory}{theory} \\
  \isarcmd{subsection} & : & \isartrans{theory}{theory} \\
  \isarcmd{subsubsection} & : & \isartrans{theory}{theory} \\
  \isarcmd{text} & : & \isartrans{theory}{theory} \\
  \isarcmd{text_raw} & : & \isartrans{theory}{theory} \\
\end{matharray}

Apart from formal comments (see \S\ref{sec:comments}), markup commands provide
a structured way to insert text into the document generated from a theory (see
\cite{isabelle-sys} for more information on Isabelle's document preparation
tools).

\railalias{textraw}{text\_raw}
\railterm{textraw}

\begin{rail}
  ('chapter' | 'section' | 'subsection' | 'subsubsection' | 'text' | textraw) text
  ;
\end{rail}

\begin{descr}
\item [$\isarkeyword{chapter}$, $\isarkeyword{section}$,
  $\isarkeyword{subsection}$, and $\isarkeyword{subsubsection}$] mark chapter
  and section headings.
\item [$\TEXT$] specifies paragraphs of plain text, including references to
  formal entities.\footnote{The latter feature is not yet supported.
    Nevertheless, any source text of the form
    ``\texttt{\at\ttlbrace$\dots$\ttrbrace}'' should be considered as reserved
    for future use.}
\item [$\isarkeyword{text_raw}$] inserts {\LaTeX} source into the output,
  without additional markup.  Thus the full range of document manipulations
  becomes available.  A typical application would be to emit
  \verb,\begin{comment}, and \verb,\end{comment}, commands to exclude certain
  parts from the final document.\footnote{This requires the \texttt{comment}
    package to be included in {\LaTeX}, of course.}
\end{descr}

Any of these markup elements corresponds to a {\LaTeX} command with the name
prefixed by \verb,\isamarkup,.  For the sectioning commands this is a plain
macro with a single argument, e.g.\ \verb,\isamarkupchapter{,\dots\verb,}, for
$\isarkeyword{chapter}$.  The $\isarkeyword{text}$ markup results in a
{\LaTeX} environment \verb,\begin{isamarkuptext}, {\dots}
  \verb,\end{isamarkuptext},, while $\isarkeyword{text_raw}$ causes the text
to be inserted directly into the {\LaTeX} source.

\medskip

Additional markup commands are available for proofs (see
\S\ref{sec:markup-prf}).  Also note that the $\isarkeyword{header}$
declaration (see \S\ref{sec:begin-thy}) admits to insert section markup just
preceding the actual theory definition.


\subsection{Type classes and sorts}\label{sec:classes}

\indexisarcmd{classes}\indexisarcmd{classrel}\indexisarcmd{defaultsort}
\begin{matharray}{rcl}
  \isarcmd{classes} & : & \isartrans{theory}{theory} \\
  \isarcmd{classrel} & : & \isartrans{theory}{theory} \\
  \isarcmd{defaultsort} & : & \isartrans{theory}{theory} \\
\end{matharray}

\begin{rail}
  'classes' (classdecl comment? +)
  ;
  'classrel' nameref '<' nameref comment?
  ;
  'defaultsort' sort comment?
  ;
\end{rail}

\begin{descr}
\item [$\isarkeyword{classes}~c<\vec c$] declares class $c$ to be a subclass
  of existing classes $\vec c$.  Cyclic class structures are ruled out.
\item [$\isarkeyword{classrel}~c@1<c@2$] states a subclass relation between
  existing classes $c@1$ and $c@2$.  This is done axiomatically!  The
  $\INSTANCE$ command (see \S\ref{sec:axclass}) provides a way to introduce
  proven class relations.
\item [$\isarkeyword{defaultsort}~s$] makes sort $s$ the new default sort for
  any type variables given without sort constraints.  Usually, the default
  sort would be only changed when defining new object-logics.
\end{descr}


\subsection{Primitive types and type abbreviations}\label{sec:types-pure}

\indexisarcmd{typedecl}\indexisarcmd{types}\indexisarcmd{nonterminals}\indexisarcmd{arities}
\begin{matharray}{rcl}
  \isarcmd{types} & : & \isartrans{theory}{theory} \\
  \isarcmd{typedecl} & : & \isartrans{theory}{theory} \\
  \isarcmd{nonterminals} & : & \isartrans{theory}{theory} \\
  \isarcmd{arities} & : & \isartrans{theory}{theory} \\
\end{matharray}

\begin{rail}
  'types' (typespec '=' type infix? comment? +)
  ;
  'typedecl' typespec infix? comment?
  ;
  'nonterminals' (name +) comment?
  ;
  'arities' (nameref '::' arity comment? +)
  ;
\end{rail}

\begin{descr}
\item [$\TYPES~(\vec\alpha)t = \tau$] introduces \emph{type synonym}
  $(\vec\alpha)t$ for existing type $\tau$.  Unlike actual type definitions,
  as are available in Isabelle/HOL for example, type synonyms are just purely
  syntactic abbreviations without any logical significance.  Internally, type
  synonyms are fully expanded.
\item [$\isarkeyword{typedecl}~(\vec\alpha)t$] declares a new type constructor
  $t$, intended as an actual logical type.  Note that object-logics such as
  Isabelle/HOL override $\isarkeyword{typedecl}$ by their own version.
\item [$\isarkeyword{nonterminals}~\vec c$] declares $0$-ary type constructors
  $\vec c$ to act as purely syntactic types, i.e.\ nonterminal symbols of
  Isabelle's inner syntax of terms or types.
\item [$\isarkeyword{arities}~t::(\vec s)s$] augments Isabelle's order-sorted
  signature of types by new type constructor arities.  This is done
  axiomatically!  The $\INSTANCE$ command (see \S\ref{sec:axclass}) provides a
  way to introduce proven type arities.
\end{descr}


\subsection{Constants and simple definitions}\label{sec:consts}

\indexisarcmd{consts}\indexisarcmd{defs}\indexisarcmd{constdefs}\indexoutertoken{constdecl}
\begin{matharray}{rcl}
  \isarcmd{consts} & : & \isartrans{theory}{theory} \\
  \isarcmd{defs} & : & \isartrans{theory}{theory} \\
  \isarcmd{constdefs} & : & \isartrans{theory}{theory} \\
\end{matharray}

\begin{rail}
  'consts' (constdecl +)
  ;
  'defs' ('(overloaded)')? (axmdecl prop comment? +)
  ;
  'constdefs' (constdecl prop comment? +)
  ;

  constdecl: name '::' type mixfix? comment?
  ;
\end{rail}

\begin{descr}
\item [$\CONSTS~c::\sigma$] declares constant $c$ to have any instance of type
  scheme $\sigma$.  The optional mixfix annotations may attach concrete syntax
  to the constants declared.

\item [$\DEFS~name: eqn$] introduces $eqn$ as a definitional axiom for some
  existing constant.  See \cite[\S6]{isabelle-ref} for more details on the
  form of equations admitted as constant definitions.
  
  The $overloaded$ option declares definitions to be potentially overloaded.
  Unless this option is given, a warning message would be issued for any
  definitional equation with a more special type than that of the
  corresponding constant declaration.

\item [$\isarkeyword{constdefs}~c::\sigma~eqn$] combines declarations and
  definitions of constants, using the canonical name $c_def$ for the
  definitional axiom.
\end{descr}


\subsection{Syntax and translations}\label{sec:syn-trans}

\indexisarcmd{syntax}\indexisarcmd{translations}
\begin{matharray}{rcl}
  \isarcmd{syntax} & : & \isartrans{theory}{theory} \\
  \isarcmd{translations} & : & \isartrans{theory}{theory} \\
\end{matharray}

\railalias{rightleftharpoons}{\isasymrightleftharpoons}
\railterm{rightleftharpoons}

\railalias{rightharpoonup}{\isasymrightharpoonup}
\railterm{rightharpoonup}

\railalias{leftharpoondown}{\isasymleftharpoondown}
\railterm{leftharpoondown}

\begin{rail}
  'syntax' ('(' ( name | 'output' | name 'output' ) ')')? (constdecl +)
  ;
  'translations' (transpat ('==' | '=>' | '<=' | rightleftharpoons | rightharpoonup | leftharpoondown) transpat comment? +)
  ;
  transpat: ('(' nameref ')')? string
  ;
\end{rail}

\begin{descr}
\item [$\isarkeyword{syntax}~(mode)~decls$] is similar to $\CONSTS~decls$,
  except that the actual logical signature extension is omitted.  Thus the
  context free grammar of Isabelle's inner syntax may be augmented in
  arbitrary ways, independently of the logic.  The $mode$ argument refers to
  the print mode that the grammar rules belong; unless the \texttt{output}
  flag is given, all productions are added both to the input and output
  grammar.
\item [$\isarkeyword{translations}~rules$] specifies syntactic translation
  rules (i.e.\ \emph{macros}): parse~/ print rules (\texttt{==} or
  \isasymrightleftharpoons), parse rules (\texttt{=>} or
  \isasymrightharpoonup), or print rules (\texttt{<=} or
  \isasymleftharpoondown).  Translation patterns may be prefixed by the
  syntactic category to be used for parsing; the default is \texttt{logic}.
\end{descr}


\subsection{Axioms and theorems}\label{sec:axms-thms}

\indexisarcmd{axioms}\indexisarcmd{theorems}\indexisarcmd{lemmas}
\begin{matharray}{rcl}
  \isarcmd{axioms} & : & \isartrans{theory}{theory} \\
  \isarcmd{theorems} & : & \isartrans{theory}{theory} \\
  \isarcmd{lemmas} & : & \isartrans{theory}{theory} \\
\end{matharray}

\begin{rail}
  'axioms' (axmdecl prop comment? +)
  ;
  ('theorems' | 'lemmas') (thmdef? thmrefs comment? + 'and')
  ;
\end{rail}

\begin{descr}
\item [$\isarkeyword{axioms}~a: \phi$] introduces arbitrary statements as
  axioms of the meta-logic.  In fact, axioms are ``axiomatic theorems'', and
  may be referred later just as any other theorem.
  
  Axioms are usually only introduced when declaring new logical systems.
  Everyday work is typically done the hard way, with proper definitions and
  actual proven theorems.
\item [$\isarkeyword{theorems}~a = \vec b$] stores lists of existing theorems.
  Typical applications would also involve attributes, to declare Simplifier
  rules, for example.
\item [$\isarkeyword{lemmas}$] is similar to $\isarkeyword{theorems}$, but
  tags the results as ``lemma''.
\end{descr}


\subsection{Name spaces}

\indexisarcmd{global}\indexisarcmd{local}\indexisarcmd{hide}
\begin{matharray}{rcl}
  \isarcmd{global} & : & \isartrans{theory}{theory} \\
  \isarcmd{local} & : & \isartrans{theory}{theory} \\
  \isarcmd{hide} & : & \isartrans{theory}{theory} \\
\end{matharray}

\begin{rail}
  'global' comment?
  ;
  'local' comment?
  ;
  'hide' name (nameref + ) comment?
  ;
\end{rail}

Isabelle organizes any kind of name declarations (of types, constants,
theorems etc.) by separate hierarchically structured name spaces.  Normally
the user does not have to control the behavior of name spaces by hand, yet the
following commands provide some way to do so.

\begin{descr}
\item [$\isarkeyword{global}$ and $\isarkeyword{local}$] change the current
  name declaration mode.  Initially, theories start in $\isarkeyword{local}$
  mode, causing all names to be automatically qualified by the theory name.
  Changing this to $\isarkeyword{global}$ causes all names to be declared
  without the theory prefix, until $\isarkeyword{local}$ is declared again.
  
  Note that global names are prone to get hidden accidently later, when
  qualified names of the same base name are introduced.
  
\item [$\isarkeyword{hide}~space~names$] removes declarations from a given
  name space (which may be $class$, $type$, or $const$).  Hidden objects
  remain valid within the logic, but are inaccessible from user input.  In
  output, the special qualifier ``$\mathord?\mathord?$'' is prefixed to the
  full internal name.
  
  Unqualified (global) names may not be hidden deliberately.
\end{descr}


\subsection{Incorporating ML code}\label{sec:ML}

\indexisarcmd{use}\indexisarcmd{ML}\indexisarcmd{ML-command}
\indexisarcmd{ML-setup}\indexisarcmd{setup}
\indexisarcmd{method-setup}
\begin{matharray}{rcl}
  \isarcmd{use} & : & \isartrans{\cdot}{\cdot} \\
  \isarcmd{ML} & : & \isartrans{\cdot}{\cdot} \\
  \isarcmd{ML_command} & : & \isartrans{\cdot}{\cdot} \\
  \isarcmd{ML_setup} & : & \isartrans{theory}{theory} \\
  \isarcmd{setup} & : & \isartrans{theory}{theory} \\
  \isarcmd{method_setup} & : & \isartrans{theory}{theory} \\
\end{matharray}

\railalias{MLsetup}{ML\_setup}
\railterm{MLsetup}

\railalias{methodsetup}{method\_setup}
\railterm{methodsetup}

\railalias{MLcommand}{ML\_command}
\railterm{MLcommand}

\begin{rail}
  'use' name comment?
  ;
  ('ML' | MLcommand | MLsetup | 'setup') text comment?
  ;
  methodsetup name '=' text text comment?
  ;
\end{rail}

\begin{descr}
\item [$\isarkeyword{use}~file$] reads and executes ML commands from $file$.
  The current theory context (if present) is passed down to the ML session,
  but may not be modified.  Furthermore, the file name is checked with the
  $\isarkeyword{files}$ dependency declaration given in the theory header (see
  also \S\ref{sec:begin-thy}).
  
\item [$\isarkeyword{ML}~text$ and $\isarkeyword{ML_command}~text$] execute ML
  commands from $text$.  The theory context is passed in the same way as for
  $\isarkeyword{use}$, but may not be changed.  Note that
  $\isarkeyword{ML_command}$ is less verbose than plain $\isarkeyword{ML}$.
  
\item [$\isarkeyword{ML_setup}~text$] executes ML commands from $text$.  The
  theory context is passed down to the ML session, and fetched back
  afterwards.  Thus $text$ may actually change the theory as a side effect.
  
\item [$\isarkeyword{setup}~text$] changes the current theory context by
  applying $text$, which refers to an ML expression of type
  \texttt{(theory~->~theory)~list}.  The $\isarkeyword{setup}$ command is the
  canonical way to initialize any object-logic specific tools and packages
  written in ML.
  
\item [$\isarkeyword{method_setup}~name = text~description$] defines a proof
  method in the current theory.  The given $text$ has to be an ML expression
  of type \texttt{Args.src -> Proof.context -> Proof.method}.  Parsing
  concrete method syntax from \texttt{Args.src} input can be quite tedious in
  general.  The following simple examples are for methods without any explicit
  arguments, or a list of theorems, respectively.

{\footnotesize
\begin{verbatim}
 Method.no_args (Method.METHOD (fn facts => foobar_tac))
 Method.thms_args (fn thms => Method.METHOD (fn facts => foobar_tac))
\end{verbatim}
}

Note that mere tactic emulations may ignore the \texttt{facts} parameter
above.  Proper proof methods would do something ``appropriate'' with the list
of current facts, though.  Single-rule methods usually do strict
forward-chaining (e.g.\ by using \texttt{Method.multi_resolves}), while
automatic ones just insert the facts using \texttt{Method.insert_tac} before
applying the main tactic.
\end{descr}


\subsection{Syntax translation functions}

\indexisarcmd{parse-ast-translation}\indexisarcmd{parse-translation}
\indexisarcmd{print-translation}\indexisarcmd{typed-print-translation}
\indexisarcmd{print-ast-translation}\indexisarcmd{token-translation}
\begin{matharray}{rcl}
  \isarcmd{parse_ast_translation} & : & \isartrans{theory}{theory} \\
  \isarcmd{parse_translation} & : & \isartrans{theory}{theory} \\
  \isarcmd{print_translation} & : & \isartrans{theory}{theory} \\
  \isarcmd{typed_print_translation} & : & \isartrans{theory}{theory} \\
  \isarcmd{print_ast_translation} & : & \isartrans{theory}{theory} \\
  \isarcmd{token_translation} & : & \isartrans{theory}{theory} \\
\end{matharray}

\railalias{parseasttranslation}{parse\_ast\_translation}
\railterm{parseasttranslation}

\railalias{parsetranslation}{parse\_translation}
\railterm{parsetranslation}

\railalias{printtranslation}{print\_translation}
\railterm{printtranslation}

\railalias{typedprinttranslation}{typed\_print\_translation}
\railterm{typedprinttranslation}

\railalias{printasttranslation}{print\_ast\_translation}
\railterm{printasttranslation}

\railalias{tokentranslation}{token\_translation}
\railterm{tokentranslation}

\begin{rail}
  ( parseasttranslation | parsetranslation | printtranslation | typedprinttranslation |
  printasttranslation | tokentranslation ) text comment?
\end{rail}

Syntax translation functions written in ML admit almost arbitrary
manipulations of Isabelle's inner syntax.  Any of the above commands have a
single \railqtoken{text} argument that refers to an ML expression of
appropriate type.

\begin{ttbox}
val parse_ast_translation   : (string * (ast list -> ast)) list
val parse_translation       : (string * (term list -> term)) list
val print_translation       : (string * (term list -> term)) list
val typed_print_translation :
  (string * (bool -> typ -> term list -> term)) list
val print_ast_translation   : (string * (ast list -> ast)) list
val token_translation       :
  (string * string * (string -> string * real)) list
\end{ttbox}
See \cite[\S8]{isabelle-ref} for more information on syntax transformations.


\subsection{Oracles}

\indexisarcmd{oracle}
\begin{matharray}{rcl}
  \isarcmd{oracle} & : & \isartrans{theory}{theory} \\
\end{matharray}

Oracles provide an interface to external reasoning systems, without giving up
control completely --- each theorem carries a derivation object recording any
oracle invocation.  See \cite[\S6]{isabelle-ref} for more information.

\begin{rail}
  'oracle' name '=' text comment?
  ;
\end{rail}

\begin{descr}
\item [$\isarkeyword{oracle}~name=text$] declares oracle $name$ to be ML
  function $text$, which has to be of type
  \texttt{Sign.sg~*~Object.T~->~term}.
\end{descr}


\section{Proof commands}

Proof commands perform transitions of Isar/VM machine configurations, which
are block-structured, consisting of a stack of nodes with three main
components: logical proof context, current facts, and open goals.  Isar/VM
transitions are \emph{typed} according to the following three different modes
of operation:
\begin{descr}
\item [$proof(prove)$] means that a new goal has just been stated that is now
  to be \emph{proven}; the next command may refine it by some proof method,
  and enter a sub-proof to establish the actual result.
\item [$proof(state)$] is like an internal theory mode: the context may be
  augmented by \emph{stating} additional assumptions, intermediate results
  etc.
\item [$proof(chain)$] is intermediate between $proof(state)$ and
  $proof(prove)$: existing facts (i.e.\ the contents of the special ``$this$''
  register) have been just picked up in order to be used when refining the
  goal claimed next.
\end{descr}


\subsection{Proof markup commands}\label{sec:markup-prf}

\indexisarcmd{sect}\indexisarcmd{subsect}\indexisarcmd{subsubsect}
\indexisarcmd{txt}\indexisarcmd{txt-raw}
\begin{matharray}{rcl}
  \isarcmd{sect} & : & \isartrans{proof}{proof} \\
  \isarcmd{subsect} & : & \isartrans{proof}{proof} \\
  \isarcmd{subsubsect} & : & \isartrans{proof}{proof} \\
  \isarcmd{txt} & : & \isartrans{proof}{proof} \\
  \isarcmd{txt_raw} & : & \isartrans{proof}{proof} \\
\end{matharray}

These markup commands for proof mode closely correspond to the ones of theory
mode (see \S\ref{sec:markup-thy}).

\railalias{txtraw}{txt\_raw}
\railterm{txtraw}

\begin{rail}
  ('sect' | 'subsect' | 'subsubsect' | 'txt' | txtraw) text
  ;
\end{rail}


\subsection{Proof context}\label{sec:proof-context}

\indexisarcmd{fix}\indexisarcmd{assume}\indexisarcmd{presume}\indexisarcmd{def}
\begin{matharray}{rcl}
  \isarcmd{fix} & : & \isartrans{proof(state)}{proof(state)} \\
  \isarcmd{assume} & : & \isartrans{proof(state)}{proof(state)} \\
  \isarcmd{presume} & : & \isartrans{proof(state)}{proof(state)} \\
  \isarcmd{def} & : & \isartrans{proof(state)}{proof(state)} \\
\end{matharray}

The logical proof context consists of fixed variables and assumptions.  The
former closely correspond to Skolem constants, or meta-level universal
quantification as provided by the Isabelle/Pure logical framework.
Introducing some \emph{arbitrary, but fixed} variable via $\FIX x$ results in
a local value that may be used in the subsequent proof as any other variable
or constant.  Furthermore, any result $\edrv \phi[x]$ exported from the
context will be universally closed wrt.\ $x$ at the outermost level: $\edrv
\All x \phi$ (this is expressed using Isabelle's meta-variables).

Similarly, introducing some assumption $\chi$ has two effects.  On the one
hand, a local theorem is created that may be used as a fact in subsequent
proof steps.  On the other hand, any result $\chi \drv \phi$ exported from the
context becomes conditional wrt.\ the assumption: $\edrv \chi \Imp \phi$.
Thus, solving an enclosing goal using such a result would basically introduce
a new subgoal stemming from the assumption.  How this situation is handled
depends on the actual version of assumption command used: while $\ASSUMENAME$
insists on solving the subgoal by unification with some premise of the goal,
$\PRESUMENAME$ leaves the subgoal unchanged in order to be proved later by the
user.

Local definitions, introduced by $\DEF{}{x \equiv t}$, are achieved by
combining $\FIX x$ with another version of assumption that causes any
hypothetical equation $x \equiv t$ to be eliminated by the reflexivity rule.
Thus, exporting some result $x \equiv t \drv \phi[x]$ yields $\edrv \phi[t]$.

\begin{rail}
  'fix' (vars + 'and') comment?
  ;
  ('assume' | 'presume') (assm comment? + 'and')
  ;
  'def' thmdecl? \\ name '==' term termpat? comment?
  ;

  var: name ('::' type)?
  ;
  vars: (name+) ('::' type)?
  ;
  assm: thmdecl? (prop proppat? +)
  ;
\end{rail}

\begin{descr}
\item [$\FIX{\vec x}$] introduces local \emph{arbitrary, but fixed} variables
  $\vec x$.
\item [$\ASSUME{a}{\vec\phi}$ and $\PRESUME{a}{\vec\phi}$] introduce local
  theorems $\vec\phi$ by assumption.  Subsequent results applied to an
  enclosing goal (e.g.\ by $\SHOWNAME$) are handled as follows: $\ASSUMENAME$
  expects to be able to unify with existing premises in the goal, while
  $\PRESUMENAME$ leaves $\vec\phi$ as new subgoals.
  
  Several lists of assumptions may be given (separated by
  $\isarkeyword{and}$); the resulting list of current facts consists of all of
  these concatenated.
\item [$\DEF{a}{x \equiv t}$] introduces a local (non-polymorphic) definition.
  In results exported from the context, $x$ is replaced by $t$.  Basically,
  $\DEF{}{x \equiv t}$ abbreviates $\FIX{x}~\ASSUME{}{x \equiv t}$, with the
  resulting hypothetical equation solved by reflexivity.
  
  The default name for the definitional equation is $x_def$.
\end{descr}

The special name $prems$\indexisarthm{prems} refers to all assumptions of the
current context as a list of theorems.


\subsection{Facts and forward chaining}

\indexisarcmd{note}\indexisarcmd{then}\indexisarcmd{from}\indexisarcmd{with}
\begin{matharray}{rcl}
  \isarcmd{note} & : & \isartrans{proof(state)}{proof(state)} \\
  \isarcmd{then} & : & \isartrans{proof(state)}{proof(chain)} \\
  \isarcmd{from} & : & \isartrans{proof(state)}{proof(chain)} \\
  \isarcmd{with} & : & \isartrans{proof(state)}{proof(chain)} \\
\end{matharray}

New facts are established either by assumption or proof of local statements.
Any fact will usually be involved in further proofs, either as explicit
arguments of proof methods, or when forward chaining towards the next goal via
$\THEN$ (and variants).  Note that the special theorem name
$this$\indexisarthm{this} refers to the most recently established facts.
\begin{rail}
  'note' (thmdef? thmrefs comment? + 'and')
  ;
  'then' comment?
  ;
  ('from' | 'with') (thmrefs comment? + 'and')
  ;
\end{rail}

\begin{descr}
\item [$\NOTE{a}{\vec b}$] recalls existing facts $\vec b$, binding the result
  as $a$.  Note that attributes may be involved as well, both on the left and
  right hand sides.
\item [$\THEN$] indicates forward chaining by the current facts in order to
  establish the goal to be claimed next.  The initial proof method invoked to
  refine that will be offered the facts to do ``anything appropriate'' (cf.\ 
  also \S\ref{sec:proof-steps}).  For example, method $rule$ (see
  \S\ref{sec:pure-meth-att}) would typically do an elimination rather than an
  introduction.  Automatic methods usually insert the facts into the goal
  state before operation.  This provides a simple scheme to control relevance
  of facts in automated proof search.
\item [$\FROM{\vec b}$] abbreviates $\NOTE{}{\vec b}~\THEN$; thus $\THEN$ is
  equivalent to $\FROM{this}$.
\item [$\WITH{\vec b}$] abbreviates $\FROM{\vec b~facts}$; thus the forward
  chaining is from earlier facts together with the current ones.
\end{descr}

Basic proof methods (such as $rule$, see \S\ref{sec:pure-meth-att}) expect
multiple facts to be given in their proper order, corresponding to a prefix of
the premises of the rule involved.  Note that positions may be easily skipped
using something like $\FROM{\Text{\texttt{_}}~a~b}$, for example.  This
involves the trivial rule $\PROP\psi \Imp \PROP\psi$, which happens to be
bound in Isabelle/Pure as ``\texttt{_}''
(underscore).\indexisarthm{_@\texttt{_}}

Forward chaining with an empty list of theorems is the same as not chaining.
Thus $\FROM{nothing}$ has no effect apart from entering $prove(chain)$ mode,
since $nothing$\indexisarthm{nothing} is bound to the empty list of facts.


\subsection{Goal statements}

\indexisarcmd{theorem}\indexisarcmd{lemma}
\indexisarcmd{have}\indexisarcmd{show}\indexisarcmd{hence}\indexisarcmd{thus}
\begin{matharray}{rcl}
  \isarcmd{theorem} & : & \isartrans{theory}{proof(prove)} \\
  \isarcmd{lemma} & : & \isartrans{theory}{proof(prove)} \\
  \isarcmd{have} & : & \isartrans{proof(state) ~|~ proof(chain)}{proof(prove)} \\
  \isarcmd{show} & : & \isartrans{proof(state) ~|~ proof(chain)}{proof(prove)} \\
  \isarcmd{hence} & : & \isartrans{proof(state)}{proof(prove)} \\
  \isarcmd{thus} & : & \isartrans{proof(state)}{proof(prove)} \\
\end{matharray}

Proof mode is entered from theory mode by initial goal commands $\THEOREMNAME$
and $\LEMMANAME$.  New local goals may be claimed within proof mode as well.
Four variants are available, indicating whether the result is meant to solve
some pending goal or whether forward chaining is indicated.

\begin{rail}
  ('theorem' | 'lemma') goal
  ;
  ('have' | 'show' | 'hence' | 'thus') goal
  ;

  goal: thmdecl? prop proppat? comment?
  ;
\end{rail}

\begin{descr}
\item [$\THEOREM{a}{\phi}$] enters proof mode with $\phi$ as main goal,
  eventually resulting in some theorem $\turn \phi$ to be put back into the
  theory.
\item [$\LEMMA{a}{\phi}$] is similar to $\THEOREMNAME$, but tags the result as
  ``lemma''.
\item [$\HAVE{a}{\phi}$] claims a local goal, eventually resulting in a
  theorem with the current assumption context as hypotheses.
\item [$\SHOW{a}{\phi}$] is similar to $\HAVE{a}{\phi}$, but solves some
  pending goal with the result \emph{exported} into the corresponding context
  (cf.\ \S\ref{sec:proof-context}).
\item [$\HENCENAME$] abbreviates $\THEN~\HAVENAME$, i.e.\ claims a local goal
  to be proven by forward chaining the current facts.  Note that $\HENCENAME$
  is also equivalent to $\FROM{this}~\HAVENAME$.
\item [$\THUSNAME$] abbreviates $\THEN~\SHOWNAME$.  Note that $\THUSNAME$ is
  also equivalent to $\FROM{this}~\SHOWNAME$.
\end{descr}

Any goal statement causes some term abbreviations (such as $\Var{thesis}$,
$\dots$) to be bound automatically, see also \S\ref{sec:term-abbrev}.
Furthermore, the local context of a (non-atomic) goal is provided via the case
name $antecedent$\indexisarcase{antecedent}, see also \S\ref{sec:cases}.

\medskip

\begin{warn}
  Isabelle/Isar suffers theory-level goal statements to contain \emph{unbound
    schematic variables}, although this does not conform to the aim of
  human-readable proof documents!  The main problem with schematic goals is
  that the actual outcome is usually hard to predict, depending on the
  behavior of the actual proof methods applied during the reasoning.  Note
  that most semi-automated methods heavily depend on several kinds of implicit
  rule declarations within the current theory context.  As this would also
  result in non-compositional checking of sub-proofs, \emph{local goals} are
  not allowed to be schematic at all.
  
  Nevertheless, schematic goals do have their use in Prolog-style interactive
  synthesis of proven results, usually by stepwise refinement via emulation of
  traditional Isabelle tactic scripts (see also \S\ref{sec:tactic-commands}).
  In any case, users should know what they are doing!
\end{warn}


\subsection{Initial and terminal proof steps}\label{sec:proof-steps}

\indexisarcmd{proof}\indexisarcmd{qed}\indexisarcmd{by}
\indexisarcmd{.}\indexisarcmd{..}\indexisarcmd{sorry}
\begin{matharray}{rcl}
  \isarcmd{proof} & : & \isartrans{proof(prove)}{proof(state)} \\
  \isarcmd{qed} & : & \isartrans{proof(state)}{proof(state) ~|~ theory} \\
  \isarcmd{by} & : & \isartrans{proof(prove)}{proof(state) ~|~ theory} \\
  \isarcmd{.\,.} & : & \isartrans{proof(prove)}{proof(state) ~|~ theory} \\
  \isarcmd{.} & : & \isartrans{proof(prove)}{proof(state) ~|~ theory} \\
  \isarcmd{sorry} & : & \isartrans{proof(prove)}{proof(state) ~|~ theory} \\
\end{matharray}

Arbitrary goal refinement via tactics is considered harmful.  Properly, the
Isar framework admits proof methods to be invoked in two places only.
\begin{enumerate}
\item An \emph{initial} refinement step $\PROOF{m@1}$ reduces a newly stated
  goal to a number of sub-goals that are to be solved later.  Facts are passed
  to $m@1$ for forward chaining, if so indicated by $proof(chain)$ mode.
  
\item A \emph{terminal} conclusion step $\QED{m@2}$ is intended to solve
  remaining goals.  No facts are passed to $m@2$.
\end{enumerate}

The only other proper way to affect pending goals is by $\SHOWNAME$, which
involves an explicit statement of what is to be solved.

\medskip

Also note that initial proof methods should either solve the goal completely,
or constitute some well-understood reduction to new sub-goals.  Arbitrary
automatic proof tools that are prone leave a large number of badly structured
sub-goals are no help in continuing the proof document in any intelligible
way.

\medskip

Unless given explicitly by the user, the default initial method is ``$rule$'',
which applies a single standard elimination or introduction rule according to
the topmost symbol involved.  There is no separate default terminal method.
Any remaining goals are always solved by assumption in the very last step.

\begin{rail}
  'proof' interest? meth? comment?
  ;
  'qed' meth? comment?
  ;
  'by' meth meth? comment?
  ;
  ('.' | '..' | 'sorry') comment?
  ;

  meth: method interest?
  ;
\end{rail}

\begin{descr}
\item [$\PROOF{m@1}$] refines the goal by proof method $m@1$; facts for
  forward chaining are passed if so indicated by $proof(chain)$ mode.
\item [$\QED{m@2}$] refines any remaining goals by proof method $m@2$ and
  concludes the sub-proof by assumption.  If the goal had been $\SHOWNAME$ (or
  $\THUSNAME$), some pending sub-goal is solved as well by the rule resulting
  from the result \emph{exported} into the enclosing goal context.  Thus
  $\QEDNAME$ may fail for two reasons: either $m@2$ fails, or the resulting
  rule does not fit to any pending goal\footnote{This includes any additional
    ``strong'' assumptions as introduced by $\ASSUMENAME$.} of the enclosing
  context.  Debugging such a situation might involve temporarily changing
  $\SHOWNAME$ into $\HAVENAME$, or weakening the local context by replacing
  some occurrences of $\ASSUMENAME$ by $\PRESUMENAME$.
\item [$\BYY{m@1}{m@2}$] is a \emph{terminal proof}\index{proof!terminal}; it
  abbreviates $\PROOF{m@1}~\QED{m@2}$, with backtracking across both methods,
  though.  Debugging an unsuccessful $\BYY{m@1}{m@2}$ commands might be done
  by expanding its definition; in many cases $\PROOF{m@1}$ is already
  sufficient to see what is going wrong.
\item [``$\DDOT$''] is a \emph{default proof}\index{proof!default}; it
  abbreviates $\BY{rule}$.
\item [``$\DOT$''] is a \emph{trivial proof}\index{proof!trivial}; it
  abbreviates $\BY{this}$.
\item [$\SORRY$] is a \emph{fake proof}\index{proof!fake}; provided that the
  \texttt{quick_and_dirty} flag is enabled, $\SORRY$ pretends to solve the
  goal without further ado.  Of course, the result would be a fake theorem
  only, involving some oracle in its internal derivation object (this is
  indicated as ``$[!]$'' in the printed result).  The main application of
  $\SORRY$ is to support experimentation and top-down proof development.
\end{descr}


\subsection{Fundamental methods and attributes}\label{sec:pure-meth-att}

The following proof methods and attributes refer to basic logical operations
of Isar.  Further methods and attributes are provided by several generic and
object-logic specific tools and packages (see chapters \ref{ch:gen-tools} and
\ref{ch:hol-tools}).

\indexisarmeth{assumption}\indexisarmeth{this}\indexisarmeth{rule}\indexisarmeth{$-$}
\indexisaratt{intro}\indexisaratt{elim}\indexisaratt{dest}\indexisaratt{rule}
\indexisaratt{OF}\indexisaratt{of}
\begin{matharray}{rcl}
  assumption & : & \isarmeth \\
  this & : & \isarmeth \\
  rule & : & \isarmeth \\
  - & : & \isarmeth \\
  OF & : & \isaratt \\
  of & : & \isaratt \\
  intro & : & \isaratt \\
  elim & : & \isaratt \\
  dest & : & \isaratt \\
  rule & : & \isaratt \\
\end{matharray}

\begin{rail}
  'rule' thmrefs?
  ;
  'OF' thmrefs
  ;
  'of' insts ('concl' ':' insts)?
  ;
  'rule' 'del'
  ;
\end{rail}

\begin{descr}
\item [$assumption$] solves some goal by a single assumption step.  Any facts
  given (${} \le 1$) are guaranteed to participate in the refinement.  Recall
  that $\QEDNAME$ (see \S\ref{sec:proof-steps}) already concludes any
  remaining sub-goals by assumption.
\item [$this$] applies all of the current facts directly as rules.  Recall
  that ``$\DOT$'' (dot) abbreviates $\BY{this}$.
\item [$rule~\vec a$] applies some rule given as argument in backward manner;
  facts are used to reduce the rule before applying it to the goal.  Thus
  $rule$ without facts is plain \emph{introduction}, while with facts it
  becomes \emph{elimination}.
  
  When no arguments are given, the $rule$ method tries to pick appropriate
  rules automatically, as declared in the current context using the $intro$,
  $elim$, $dest$ attributes (see below).  This is the default behavior of
  $\PROOFNAME$ and ``$\DDOT$'' (double-dot) steps (see
  \S\ref{sec:proof-steps}).
\item [``$-$''] does nothing but insert the forward chaining facts as premises
  into the goal.  Note that command $\PROOFNAME$ without any method actually
  performs a single reduction step using the $rule$ method; thus a plain
  \emph{do-nothing} proof step would be $\PROOF{-}$ rather than $\PROOFNAME$
  alone.
\item [$OF~\vec a$] applies some theorem to given rules $\vec a$ (in
  parallel).  This corresponds to the \texttt{MRS} operator in ML
  \cite[\S5]{isabelle-ref}, but note the reversed order.  Positions may be
  skipped by including ``$\_$'' (underscore) as argument.
\item [$of~\vec t$] performs positional instantiation.  The terms $\vec t$ are
  substituted for any schematic variables occurring in a theorem from left to
  right; ``\texttt{_}'' (underscore) indicates to skip a position.  Arguments
  following a ``$concl\colon$'' specification refer to positions of the
  conclusion of a rule.
\item [$intro$, $elim$, and $dest$] declare introduction, elimination, and
  destruct rules, respectively.  Note that the classical reasoner (see
  \S\ref{sec:classical-basic}) introduces different versions of these
  attributes, and the $rule$ method, too.  In object-logics with classical
  reasoning enabled, the latter version should be used all the time to avoid
  confusion!
\item [$rule~del$] undeclares introduction, elimination, or destruct rules.
\end{descr}


\subsection{Term abbreviations}\label{sec:term-abbrev}

\indexisarcmd{let}
\begin{matharray}{rcl}
  \isarcmd{let} & : & \isartrans{proof(state)}{proof(state)} \\
  \isarkeyword{is} & : & syntax \\
\end{matharray}

Abbreviations may be either bound by explicit $\LET{p \equiv t}$ statements,
or by annotating assumptions or goal statements with a list of patterns
$\ISS{p@1\;\dots}{p@n}$.  In both cases, higher-order matching is invoked to
bind extra-logical term variables, which may be either named schematic
variables of the form $\Var{x}$, or nameless dummies ``\texttt{_}''
(underscore).\indexisarvar{_@\texttt{_}} Note that in the $\LETNAME$ form the
patterns occur on the left-hand side, while the $\ISNAME$ patterns are in
postfix position.

Polymorphism of term bindings is handled in Hindley-Milner style, as in ML.
Type variables referring to local assumptions or open goal statements are
\emph{fixed}, while those of finished results or bound by $\LETNAME$ may occur
in \emph{arbitrary} instances later.  Even though actual polymorphism should
be rarely used in practice, this mechanism is essential to achieve proper
incremental type-inference, as the user proceeds to build up the Isar proof
text.

\medskip

Term abbreviations are quite different from actual local definitions as
introduced via $\DEFNAME$ (see \S\ref{sec:proof-context}).  The latter are
visible within the logic as actual equations, while abbreviations disappear
during the input process just after type checking.  Also note that $\DEFNAME$
does not support polymorphism.

\begin{rail}
  'let' ((term + 'and') '=' term comment? + 'and')
  ;  
\end{rail}

The syntax of $\ISNAME$ patterns follows \railnonterm{termpat} or
\railnonterm{proppat} (see \S\ref{sec:term-pats}).

\begin{descr}
\item [$\LET{\vec p = \vec t}$] binds any text variables in patters $\vec p$
  by simultaneous higher-order matching against terms $\vec t$.
\item [$\IS{\vec p}$] resembles $\LETNAME$, but matches $\vec p$ against the
  preceding statement.  Also note that $\ISNAME$ is not a separate command,
  but part of others (such as $\ASSUMENAME$, $\HAVENAME$ etc.).
\end{descr}

Some \emph{automatic} term abbreviations\index{term abbreviations} for goals
and facts are available as well.  For any open goal,
$\Var{thesis}$\indexisarvar{thesis} refers to its object-level statement,
abstracted over any meta-level parameters (if present).  Likewise,
$\Var{this}$\indexisarvar{this} is bound for fact statements resulting from
assumptions or finished goals.  In case $\Var{this}$ refers to an object-logic
statement that is an application $f(t)$, then $t$ is bound to the special text
variable ``$\dots$''\indexisarvar{\dots} (three dots).  The canonical
application of the latter are calculational proofs (see
\S\ref{sec:calculation}).

%FIXME !?

% A few \emph{automatic} term abbreviations\index{term abbreviations} for goals
% and facts are available as well.  For any open goal,
% $\Var{thesis_prop}$\indexisarvar{thesis-prop} refers to the full proposition
% (which may be a rule), $\Var{thesis_concl}$\indexisarvar{thesis-concl} to its
% (atomic) conclusion, and $\Var{thesis}$\indexisarvar{thesis} to its
% object-level statement.  The latter two abstract over any meta-level
% parameters.

% Fact statements resulting from assumptions or finished goals are bound as
% $\Var{this_prop}$\indexisarvar{this-prop},
% $\Var{this_concl}$\indexisarvar{this-concl}, and
% $\Var{this}$\indexisarvar{this}, similar to $\Var{thesis}$ above.  In case
% $\Var{this}$ refers to an object-logic statement that is an application
% $f(t)$, then $t$ is bound to the special text variable
% ``$\dots$''\indexisarvar{\dots} (three dots).  The canonical application of
% the latter are calculational proofs (see \S\ref{sec:calculation}).


\subsection{Block structure}

\indexisarcmd{next}\indexisarcmd{\{}\indexisarcmd{\}}
\begin{matharray}{rcl}
  \NEXT & : & \isartrans{proof(state)}{proof(state)} \\
  \BG & : & \isartrans{proof(state)}{proof(state)} \\
  \EN & : & \isartrans{proof(state)}{proof(state)} \\
\end{matharray}

\railalias{lbrace}{\ttlbrace}
\railterm{lbrace}

\railalias{rbrace}{\ttrbrace}
\railterm{rbrace}

\begin{rail}
  'next' comment?
  ;
  lbrace comment?
  ;
  rbrace comment?
  ;
\end{rail}

While Isar is inherently block-structured, opening and closing blocks is
mostly handled rather casually, with little explicit user-intervention.  Any
local goal statement automatically opens \emph{two} blocks, which are closed
again when concluding the sub-proof (by $\QEDNAME$ etc.).  Sections of
different context within a sub-proof may be switched via $\NEXT$, which is
just a single block-close followed by block-open again.  Thus the effect of
$\NEXT$ to reset the local proof context. There is no goal focus involved
here!

For slightly more advanced applications, there are explicit block parentheses
as well.  These typically achieve a stronger forward style of reasoning.

\begin{descr}
\item [$\NEXT$] switches to a fresh block within a sub-proof, resetting the
  local context to the initial one.
\item [$\BG$ and $\EN$] explicitly open and close blocks.  Any current facts
  pass through ``$\BG$'' unchanged, while ``$\EN$'' causes any result to be
  \emph{exported} into the enclosing context.  Thus fixed variables are
  generalized, assumptions discharged, and local definitions unfolded (cf.\ 
  \S\ref{sec:proof-context}).  There is no difference of $\ASSUMENAME$ and
  $\PRESUMENAME$ in this mode of forward reasoning --- in contrast to plain
  backward reasoning with the result exported at $\SHOWNAME$ time.
\end{descr}


\subsection{Emulating tactic scripts}\label{sec:tactic-commands}

The Isar provides separate commands to accommodate tactic-style proof scripts
within the same system.  While being outside the orthodox Isar proof language,
these might come in handy for interactive exploration and debugging, or even
actual tactical proof within new-style theories (to benefit from document
preparation, for example).  See also \S\ref{sec:tactics} for actual tactics,
that have been encapsulated as proof methods.  Proper proof methods may be
used in scripts, too.

\indexisarcmd{apply}\indexisarcmd{apply-end}\indexisarcmd{done}
\indexisarcmd{defer}\indexisarcmd{prefer}\indexisarcmd{back}
\indexisarcmd{declare}
\begin{matharray}{rcl}
  \isarcmd{apply}^* & : & \isartrans{proof(prove)}{proof(prove)} \\
  \isarcmd{apply_end}^* & : & \isartrans{proof(state)}{proof(state)} \\
  \isarcmd{done}^* & : & \isartrans{proof(prove)}{proof(state)} \\
  \isarcmd{defer}^* & : & \isartrans{proof}{proof} \\
  \isarcmd{prefer}^* & : & \isartrans{proof}{proof} \\
  \isarcmd{back}^* & : & \isartrans{proof}{proof} \\
  \isarcmd{declare}^* & : & \isartrans{theory}{theory} \\
\end{matharray}

\railalias{applyend}{apply\_end}
\railterm{applyend}

\begin{rail}
  ( 'apply' | applyend ) method comment?
  ;
  'done' comment?
  ;
  'defer' nat? comment?
  ;
  'prefer' nat comment?
  ;
  'back' comment?
  ;
  'declare' thmrefs comment?
  ;
\end{rail}

\begin{descr}
\item [$\APPLY{m}$] applies proof method $m$ in initial position, but unlike
  $\PROOFNAME$ it retains ``$proof(prove)$'' mode.  Thus consecutive method
  applications may be given just as in tactic scripts.
  
  Facts are passed to $m$ as indicated by the goal's forward-chain mode, and
  are \emph{consumed} afterwards.  Thus any further $\APPLYNAME$ command would
  always work in a purely backward manner.
  
\item [$\isarkeyword{apply_end}~(m)$] applies proof method $m$ as if in
  terminal position.  Basically, this simulates a multi-step tactic script for
  $\QEDNAME$, but may be given anywhere within the proof body.
  
  No facts are passed to $m$.  Furthermore, the static context is that of the
  enclosing goal (as for actual $\QEDNAME$).  Thus the proof method may not
  refer to any assumptions introduced in the current body, for example.

\item [$\isarkeyword{done}$] completes a proof script, provided that the
  current goal state is already solved completely.  Note that actual
  structured proof commands (e.g.\ ``$\DOT$'' or $\SORRY$) may be used to
  conclude proof scripts as well.

\item [$\isarkeyword{defer}~n$ and $\isarkeyword{prefer}~n$] shuffle the list
  of pending goals: $defer$ puts off goal $n$ to the end of the list ($n = 1$
  by default), while $prefer$ brings goal $n$ to the top.

\item [$\isarkeyword{back}$] does back-tracking over the result sequence of
  the latest proof command.\footnote{Unlike the ML function \texttt{back}
    \cite{isabelle-ref}, the Isar command does not search upwards for further
    branch points.} Basically, any proof command may return multiple results.
  
\item [$\isarkeyword{declare}~thms$] declares theorems to the current theory
  context.  No theorem binding is involved here, unlike
  $\isarkeyword{theorems}$ or $\isarkeyword{lemmas}$ (cf.\ 
  \S\ref{sec:axms-thms}).  So $\isarkeyword{declare}$ only has the effect of
  applying attributes as included in the theorem specification.
\end{descr}

Any proper Isar proof method may be used with tactic script commands such as
$\APPLYNAME$.  A few additional emulations of actual tactics are provided as
well; these would be never used in actual structured proofs, of course.


\subsection{Meta-linguistic features}

\indexisarcmd{oops}
\begin{matharray}{rcl}
  \isarcmd{oops} & : & \isartrans{proof}{theory} \\
\end{matharray}

The $\OOPS$ command discontinues the current proof attempt, while considering
the partial proof text as properly processed.  This is conceptually quite
different from ``faking'' actual proofs via $\SORRY$ (see
\S\ref{sec:proof-steps}): $\OOPS$ does not observe the proof structure at all,
but goes back right to the theory level.  Furthermore, $\OOPS$ does not
produce any result theorem --- there is no claim to be able to complete the
proof anyhow.

A typical application of $\OOPS$ is to explain Isar proofs \emph{within} the
system itself, in conjunction with the document preparation tools of Isabelle
described in \cite{isabelle-sys}.  Thus partial or even wrong proof attempts
can be discussed in a logically sound manner.  Note that the Isabelle {\LaTeX}
macros can be easily adapted to print something like ``$\dots$'' instead of an
``$\OOPS$'' keyword.

\medskip The $\OOPS$ command is undoable, unlike $\isarkeyword{kill}$ (see
\S\ref{sec:history}).  The effect is to get back to the theory \emph{before}
the opening of the proof.


\section{Other commands}

\subsection{Diagnostics}

\indexisarcmd{pr}\indexisarcmd{thm}\indexisarcmd{term}\indexisarcmd{prop}\indexisarcmd{typ}
\begin{matharray}{rcl}
  \isarcmd{pr}^* & : & \isarkeep{\cdot} \\
  \isarcmd{thm}^* & : & \isarkeep{theory~|~proof} \\
  \isarcmd{term}^* & : & \isarkeep{theory~|~proof} \\
  \isarcmd{prop}^* & : & \isarkeep{theory~|~proof} \\
  \isarcmd{typ}^* & : & \isarkeep{theory~|~proof} \\
\end{matharray}

These diagnostic commands assist interactive development.  Note that $undo$
does not apply here, the theory or proof configuration is not changed.

\begin{rail}
  'pr' modes? nat? (',' nat)?
  ;
  'thm' modes? thmrefs comment?
  ;
  'term' modes? term comment?
  ;
  'prop' modes? prop comment?
  ;
  'typ' modes? type comment?
  ;

  modes: '(' (name + ) ')'
  ;
\end{rail}

\begin{descr}
\item [$\isarkeyword{pr}~goals, prems$] prints the current proof state (if
  present), including the proof context, current facts and goals.  The
  optional limit arguments affect the number of goals and premises to be
  displayed, which is initially 10 for both.  Omitting limit values leaves the
  current setting unchanged.
\item [$\isarkeyword{thm}~\vec a$] retrieves theorems from the current theory
  or proof context.  Note that any attributes included in the theorem
  specifications are applied to a temporary context derived from the current
  theory or proof; the result is discarded, i.e.\ attributes involved in $\vec
  a$ do not have any permanent effect.
\item [$\isarkeyword{term}~t$ and $\isarkeyword{prop}~\phi$] read, type-check
  and print terms or propositions according to the current theory or proof
  context; the inferred type of $t$ is output as well.  Note that these
  commands are also useful in inspecting the current environment of term
  abbreviations.
\item [$\isarkeyword{typ}~\tau$] reads and prints types of the meta-logic
  according to the current theory or proof context.
\end{descr}

All of the diagnostic commands above admit a list of $modes$ to be specified,
which is appended to the current print mode (see also \cite{isabelle-ref}).
Thus the output behavior may be modified according particular print mode
features.  For example, $\isarkeyword{pr}~(latex~xsymbols~symbols)$ would
print the current proof state with mathematical symbols and special characters
represented in {\LaTeX} source, according to the Isabelle style
\cite{isabelle-sys}.

Note that antiquotations (cf.\ \S\ref{sec:antiq}) provide a more systematic
way to include formal items into the printed text document.


\subsection{Inspecting the context}

\indexisarcmd{print-facts}\indexisarcmd{print-binds}
\indexisarcmd{print-commands}\indexisarcmd{print-syntax}
\indexisarcmd{print-methods}\indexisarcmd{print-attributes}
\begin{matharray}{rcl}
  \isarcmd{print_commands}^* & : & \isarkeep{\cdot} \\
  \isarcmd{print_syntax}^* & : & \isarkeep{theory~|~proof} \\
  \isarcmd{print_methods}^* & : & \isarkeep{theory~|~proof} \\
  \isarcmd{print_attributes}^* & : & \isarkeep{theory~|~proof} \\
  \isarcmd{print_facts}^* & : & \isarkeep{proof} \\
  \isarcmd{print_binds}^* & : & \isarkeep{proof} \\
\end{matharray}

These commands print parts of the theory and proof context.  Note that there
are some further ones available, such as for the set of rules declared for
simplifications.

\begin{descr}
\item [$\isarkeyword{print_commands}$] prints Isabelle's outer theory syntax,
  including keywords and command.
\item [$\isarkeyword{print_syntax}$] prints the inner syntax of types and
  terms, depending on the current context.  The output can be very verbose,
  including grammar tables and syntax translation rules.  See \cite[\S7,
  \S8]{isabelle-ref} for further information on Isabelle's inner syntax.
\item [$\isarkeyword{print_methods}$] all proof methods available in the
  current theory context.
\item [$\isarkeyword{print_attributes}$] all attributes available in the
  current theory context.
\item [$\isarkeyword{print_facts}$] prints any named facts of the current
  context, including assumptions and local results.
\item [$\isarkeyword{print_binds}$] prints all term abbreviations present in
  the context.
\end{descr}


\subsection{History commands}\label{sec:history}

\indexisarcmd{undo}\indexisarcmd{redo}\indexisarcmd{kill}
\begin{matharray}{rcl}
  \isarcmd{undo}^{{*}{*}} & : & \isarkeep{\cdot} \\
  \isarcmd{redo}^{{*}{*}} & : & \isarkeep{\cdot} \\
  \isarcmd{kill}^{{*}{*}} & : & \isarkeep{\cdot} \\
\end{matharray}

The Isabelle/Isar top-level maintains a two-stage history, for theory and
proof state transformation.  Basically, any command can be undone using
$\isarkeyword{undo}$, excluding mere diagnostic elements.  Its effect may be
revoked via $\isarkeyword{redo}$, unless the corresponding the
$\isarkeyword{undo}$ step has crossed the beginning of a proof or theory.  The
$\isarkeyword{kill}$ command aborts the current history node altogether,
discontinuing a proof or even the whole theory.  This operation is \emph{not}
undoable.

\begin{warn}
  History commands should never be used with user interfaces such as
  Proof~General \cite{proofgeneral,Aspinall:TACAS:2000}, which takes care of
  stepping forth and back itself.  Interfering by manual $\isarkeyword{undo}$,
  $\isarkeyword{redo}$, or even $\isarkeyword{kill}$ commands would quickly
  result in utter confusion.
\end{warn}


\subsection{System operations}

\indexisarcmd{cd}\indexisarcmd{pwd}\indexisarcmd{use-thy}\indexisarcmd{use-thy-only}
\indexisarcmd{update-thy}\indexisarcmd{update-thy-only}
\begin{matharray}{rcl}
  \isarcmd{cd}^* & : & \isarkeep{\cdot} \\
  \isarcmd{pwd}^* & : & \isarkeep{\cdot} \\
  \isarcmd{use_thy}^* & : & \isarkeep{\cdot} \\
  \isarcmd{use_thy_only}^* & : & \isarkeep{\cdot} \\
  \isarcmd{update_thy}^* & : & \isarkeep{\cdot} \\
  \isarcmd{update_thy_only}^* & : & \isarkeep{\cdot} \\
\end{matharray}

\begin{descr}
\item [$\isarkeyword{cd}~name$] changes the current directory of the Isabelle
  process.
\item [$\isarkeyword{pwd}~$] prints the current working directory.
\item [$\isarkeyword{use_thy}$, $\isarkeyword{use_thy_only}$,
  $\isarkeyword{update_thy}$, $\isarkeyword{update_thy_only}$] load some
  theory given as $name$ argument.  These commands are basically the same as
  the corresponding ML functions\footnote{The ML versions also change the
    implicit theory context to that of the theory loaded.}  (see also
  \cite[\S1,\S6]{isabelle-ref}).  Note that both the ML and Isar versions may
  load new- and old-style theories alike.
\end{descr}

These system commands are scarcely used when working with the Proof~General
interface, since loading of theories is done fully transparently.


%%% Local Variables: 
%%% mode: latex
%%% TeX-master: "isar-ref"
%%% End: 
