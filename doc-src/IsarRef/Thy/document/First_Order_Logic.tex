%
\begin{isabellebody}%
\def\isabellecontext{First{\isacharunderscore}Order{\isacharunderscore}Logic}%
%
\isamarkupheader{Example: First-Order Logic%
}
\isamarkuptrue%
%
\isadelimvisible
%
\endisadelimvisible
%
\isatagvisible
\isacommand{theory}\isamarkupfalse%
\ First{\isacharunderscore}Order{\isacharunderscore}Logic\isanewline
\isakeyword{imports}\ Pure\isanewline
\isakeyword{begin}%
\endisatagvisible
{\isafoldvisible}%
%
\isadelimvisible
%
\endisadelimvisible
%
\begin{isamarkuptext}%
\noindent In order to commence a new object-logic within
  Isabelle/Pure we introduce abstract syntactic categories \isa{{\isachardoublequote}i{\isachardoublequote}}
  for individuals and \isa{{\isachardoublequote}o{\isachardoublequote}} for object-propositions.  The latter
  is embedded into the language of Pure propositions by means of a
  separate judgment.%
\end{isamarkuptext}%
\isamarkuptrue%
\isacommand{typedecl}\isamarkupfalse%
\ i\isanewline
\isacommand{typedecl}\isamarkupfalse%
\ o\isanewline
\isanewline
\isacommand{judgment}\isamarkupfalse%
\isanewline
\ \ Trueprop\ {\isacharcolon}{\isacharcolon}\ {\isachardoublequoteopen}o\ {\isasymRightarrow}\ prop{\isachardoublequoteclose}\ \ \ \ {\isacharparenleft}{\isachardoublequoteopen}{\isacharunderscore}{\isachardoublequoteclose}\ {\isadigit{5}}{\isacharparenright}%
\begin{isamarkuptext}%
\noindent Note that the object-logic judgement is implicit in the
  syntax: writing \isa{A} produces \isa{{\isachardoublequote}Trueprop\ A{\isachardoublequote}} internally.
  From the Pure perspective this means ``\isa{A} is derivable in the
  object-logic''.%
\end{isamarkuptext}%
\isamarkuptrue%
%
\isamarkupsubsection{Equational reasoning \label{sec:framework-ex-equal}%
}
\isamarkuptrue%
%
\begin{isamarkuptext}%
Equality is axiomatized as a binary predicate on individuals, with
  reflexivity as introduction, and substitution as elimination
  principle.  Note that the latter is particularly convenient in a
  framework like Isabelle, because syntactic congruences are
  implicitly produced by unification of \isa{{\isachardoublequote}B\ x{\isachardoublequote}} against
  expressions containing occurrences of \isa{x}.%
\end{isamarkuptext}%
\isamarkuptrue%
\isacommand{axiomatization}\isamarkupfalse%
\isanewline
\ \ equal\ {\isacharcolon}{\isacharcolon}\ {\isachardoublequoteopen}i\ {\isasymRightarrow}\ i\ {\isasymRightarrow}\ o{\isachardoublequoteclose}\ \ {\isacharparenleft}\isakeyword{infix}\ {\isachardoublequoteopen}{\isacharequal}{\isachardoublequoteclose}\ {\isadigit{5}}{\isadigit{0}}{\isacharparenright}\isanewline
\isakeyword{where}\isanewline
\ \ refl\ {\isacharbrackleft}intro{\isacharbrackright}{\isacharcolon}\ {\isachardoublequoteopen}x\ {\isacharequal}\ x{\isachardoublequoteclose}\ \isakeyword{and}\isanewline
\ \ subst\ {\isacharbrackleft}elim{\isacharbrackright}{\isacharcolon}\ {\isachardoublequoteopen}x\ {\isacharequal}\ y\ {\isasymLongrightarrow}\ B\ x\ {\isasymLongrightarrow}\ B\ y{\isachardoublequoteclose}%
\begin{isamarkuptext}%
\noindent Substitution is very powerful, but also hard to control in
  full generality.  We derive some common symmetry~/ transitivity
  schemes of as particular consequences.%
\end{isamarkuptext}%
\isamarkuptrue%
\isacommand{theorem}\isamarkupfalse%
\ sym\ {\isacharbrackleft}sym{\isacharbrackright}{\isacharcolon}\isanewline
\ \ \isakeyword{assumes}\ {\isachardoublequoteopen}x\ {\isacharequal}\ y{\isachardoublequoteclose}\isanewline
\ \ \isakeyword{shows}\ {\isachardoublequoteopen}y\ {\isacharequal}\ x{\isachardoublequoteclose}\isanewline
%
\isadelimproof
%
\endisadelimproof
%
\isatagproof
\isacommand{proof}\isamarkupfalse%
\ {\isacharminus}\isanewline
\ \ \isacommand{have}\isamarkupfalse%
\ {\isachardoublequoteopen}x\ {\isacharequal}\ x{\isachardoublequoteclose}\ \isacommand{{\isachardot}{\isachardot}}\isamarkupfalse%
\isanewline
\ \ \isacommand{with}\isamarkupfalse%
\ {\isacharbackquoteopen}x\ {\isacharequal}\ y{\isacharbackquoteclose}\ \isacommand{show}\isamarkupfalse%
\ {\isachardoublequoteopen}y\ {\isacharequal}\ x{\isachardoublequoteclose}\ \isacommand{{\isachardot}{\isachardot}}\isamarkupfalse%
\isanewline
\isacommand{qed}\isamarkupfalse%
%
\endisatagproof
{\isafoldproof}%
%
\isadelimproof
\isanewline
%
\endisadelimproof
\isanewline
\isacommand{theorem}\isamarkupfalse%
\ forw{\isacharunderscore}subst\ {\isacharbrackleft}trans{\isacharbrackright}{\isacharcolon}\isanewline
\ \ \isakeyword{assumes}\ {\isachardoublequoteopen}y\ {\isacharequal}\ x{\isachardoublequoteclose}\ \isakeyword{and}\ {\isachardoublequoteopen}B\ x{\isachardoublequoteclose}\isanewline
\ \ \isakeyword{shows}\ {\isachardoublequoteopen}B\ y{\isachardoublequoteclose}\isanewline
%
\isadelimproof
%
\endisadelimproof
%
\isatagproof
\isacommand{proof}\isamarkupfalse%
\ {\isacharminus}\isanewline
\ \ \isacommand{from}\isamarkupfalse%
\ {\isacharbackquoteopen}y\ {\isacharequal}\ x{\isacharbackquoteclose}\ \isacommand{have}\isamarkupfalse%
\ {\isachardoublequoteopen}x\ {\isacharequal}\ y{\isachardoublequoteclose}\ \isacommand{{\isachardot}{\isachardot}}\isamarkupfalse%
\isanewline
\ \ \isacommand{from}\isamarkupfalse%
\ this\ \isakeyword{and}\ {\isacharbackquoteopen}B\ x{\isacharbackquoteclose}\ \isacommand{show}\isamarkupfalse%
\ {\isachardoublequoteopen}B\ y{\isachardoublequoteclose}\ \isacommand{{\isachardot}{\isachardot}}\isamarkupfalse%
\isanewline
\isacommand{qed}\isamarkupfalse%
%
\endisatagproof
{\isafoldproof}%
%
\isadelimproof
\isanewline
%
\endisadelimproof
\isanewline
\isacommand{theorem}\isamarkupfalse%
\ back{\isacharunderscore}subst\ {\isacharbrackleft}trans{\isacharbrackright}{\isacharcolon}\isanewline
\ \ \isakeyword{assumes}\ {\isachardoublequoteopen}B\ x{\isachardoublequoteclose}\ \isakeyword{and}\ {\isachardoublequoteopen}x\ {\isacharequal}\ y{\isachardoublequoteclose}\isanewline
\ \ \isakeyword{shows}\ {\isachardoublequoteopen}B\ y{\isachardoublequoteclose}\isanewline
%
\isadelimproof
%
\endisadelimproof
%
\isatagproof
\isacommand{proof}\isamarkupfalse%
\ {\isacharminus}\isanewline
\ \ \isacommand{from}\isamarkupfalse%
\ {\isacharbackquoteopen}x\ {\isacharequal}\ y{\isacharbackquoteclose}\ \isakeyword{and}\ {\isacharbackquoteopen}B\ x{\isacharbackquoteclose}\isanewline
\ \ \isacommand{show}\isamarkupfalse%
\ {\isachardoublequoteopen}B\ y{\isachardoublequoteclose}\ \isacommand{{\isachardot}{\isachardot}}\isamarkupfalse%
\isanewline
\isacommand{qed}\isamarkupfalse%
%
\endisatagproof
{\isafoldproof}%
%
\isadelimproof
\isanewline
%
\endisadelimproof
\isanewline
\isacommand{theorem}\isamarkupfalse%
\ trans\ {\isacharbrackleft}trans{\isacharbrackright}{\isacharcolon}\isanewline
\ \ \isakeyword{assumes}\ {\isachardoublequoteopen}x\ {\isacharequal}\ y{\isachardoublequoteclose}\ \isakeyword{and}\ {\isachardoublequoteopen}y\ {\isacharequal}\ z{\isachardoublequoteclose}\isanewline
\ \ \isakeyword{shows}\ {\isachardoublequoteopen}x\ {\isacharequal}\ z{\isachardoublequoteclose}\isanewline
%
\isadelimproof
%
\endisadelimproof
%
\isatagproof
\isacommand{proof}\isamarkupfalse%
\ {\isacharminus}\isanewline
\ \ \isacommand{from}\isamarkupfalse%
\ {\isacharbackquoteopen}y\ {\isacharequal}\ z{\isacharbackquoteclose}\ \isakeyword{and}\ {\isacharbackquoteopen}x\ {\isacharequal}\ y{\isacharbackquoteclose}\isanewline
\ \ \isacommand{show}\isamarkupfalse%
\ {\isachardoublequoteopen}x\ {\isacharequal}\ z{\isachardoublequoteclose}\ \isacommand{{\isachardot}{\isachardot}}\isamarkupfalse%
\isanewline
\isacommand{qed}\isamarkupfalse%
%
\endisatagproof
{\isafoldproof}%
%
\isadelimproof
%
\endisadelimproof
%
\isamarkupsubsection{Basic group theory%
}
\isamarkuptrue%
%
\begin{isamarkuptext}%
As an example for equational reasoning we consider some bits of
  group theory.  The subsequent locale definition postulates group
  operations and axioms; we also derive some consequences of this
  specification.%
\end{isamarkuptext}%
\isamarkuptrue%
\isacommand{locale}\isamarkupfalse%
\ group\ {\isacharequal}\isanewline
\ \ \isakeyword{fixes}\ prod\ {\isacharcolon}{\isacharcolon}\ {\isachardoublequoteopen}i\ {\isasymRightarrow}\ i\ {\isasymRightarrow}\ i{\isachardoublequoteclose}\ \ {\isacharparenleft}\isakeyword{infix}\ {\isachardoublequoteopen}{\isasymcirc}{\isachardoublequoteclose}\ {\isadigit{7}}{\isadigit{0}}{\isacharparenright}\isanewline
\ \ \ \ \isakeyword{and}\ inv\ {\isacharcolon}{\isacharcolon}\ {\isachardoublequoteopen}i\ {\isasymRightarrow}\ i{\isachardoublequoteclose}\ \ {\isacharparenleft}{\isachardoublequoteopen}{\isacharparenleft}{\isacharunderscore}{\isasyminverse}{\isacharparenright}{\isachardoublequoteclose}\ {\isacharbrackleft}{\isadigit{1}}{\isadigit{0}}{\isadigit{0}}{\isadigit{0}}{\isacharbrackright}\ {\isadigit{9}}{\isadigit{9}}{\isadigit{9}}{\isacharparenright}\isanewline
\ \ \ \ \isakeyword{and}\ unit\ {\isacharcolon}{\isacharcolon}\ i\ \ {\isacharparenleft}{\isachardoublequoteopen}{\isadigit{1}}{\isachardoublequoteclose}{\isacharparenright}\isanewline
\ \ \isakeyword{assumes}\ assoc{\isacharcolon}\ {\isachardoublequoteopen}{\isacharparenleft}x\ {\isasymcirc}\ y{\isacharparenright}\ {\isasymcirc}\ z\ {\isacharequal}\ x\ {\isasymcirc}\ {\isacharparenleft}y\ {\isasymcirc}\ z{\isacharparenright}{\isachardoublequoteclose}\isanewline
\ \ \ \ \isakeyword{and}\ left{\isacharunderscore}unit{\isacharcolon}\ \ {\isachardoublequoteopen}{\isadigit{1}}\ {\isasymcirc}\ x\ {\isacharequal}\ x{\isachardoublequoteclose}\isanewline
\ \ \ \ \isakeyword{and}\ left{\isacharunderscore}inv{\isacharcolon}\ {\isachardoublequoteopen}x{\isasyminverse}\ {\isasymcirc}\ x\ {\isacharequal}\ {\isadigit{1}}{\isachardoublequoteclose}\isanewline
\isakeyword{begin}\isanewline
\isanewline
\isacommand{theorem}\isamarkupfalse%
\ right{\isacharunderscore}inv{\isacharcolon}\ {\isachardoublequoteopen}x\ {\isasymcirc}\ x{\isasyminverse}\ {\isacharequal}\ {\isadigit{1}}{\isachardoublequoteclose}\isanewline
%
\isadelimproof
%
\endisadelimproof
%
\isatagproof
\isacommand{proof}\isamarkupfalse%
\ {\isacharminus}\isanewline
\ \ \isacommand{have}\isamarkupfalse%
\ {\isachardoublequoteopen}x\ {\isasymcirc}\ x{\isasyminverse}\ {\isacharequal}\ {\isadigit{1}}\ {\isasymcirc}\ {\isacharparenleft}x\ {\isasymcirc}\ x{\isasyminverse}{\isacharparenright}{\isachardoublequoteclose}\ \isacommand{by}\isamarkupfalse%
\ {\isacharparenleft}rule\ left{\isacharunderscore}unit\ {\isacharbrackleft}symmetric{\isacharbrackright}{\isacharparenright}\isanewline
\ \ \isacommand{also}\isamarkupfalse%
\ \isacommand{have}\isamarkupfalse%
\ {\isachardoublequoteopen}{\isasymdots}\ {\isacharequal}\ {\isacharparenleft}{\isadigit{1}}\ {\isasymcirc}\ x{\isacharparenright}\ {\isasymcirc}\ x{\isasyminverse}{\isachardoublequoteclose}\ \isacommand{by}\isamarkupfalse%
\ {\isacharparenleft}rule\ assoc\ {\isacharbrackleft}symmetric{\isacharbrackright}{\isacharparenright}\isanewline
\ \ \isacommand{also}\isamarkupfalse%
\ \isacommand{have}\isamarkupfalse%
\ {\isachardoublequoteopen}{\isadigit{1}}\ {\isacharequal}\ {\isacharparenleft}x{\isasyminverse}{\isacharparenright}{\isasyminverse}\ {\isasymcirc}\ x{\isasyminverse}{\isachardoublequoteclose}\ \isacommand{by}\isamarkupfalse%
\ {\isacharparenleft}rule\ left{\isacharunderscore}inv\ {\isacharbrackleft}symmetric{\isacharbrackright}{\isacharparenright}\isanewline
\ \ \isacommand{also}\isamarkupfalse%
\ \isacommand{have}\isamarkupfalse%
\ {\isachardoublequoteopen}{\isasymdots}\ {\isasymcirc}\ x\ {\isacharequal}\ {\isacharparenleft}x{\isasyminverse}{\isacharparenright}{\isasyminverse}\ {\isasymcirc}\ {\isacharparenleft}x{\isasyminverse}\ {\isasymcirc}\ x{\isacharparenright}{\isachardoublequoteclose}\ \isacommand{by}\isamarkupfalse%
\ {\isacharparenleft}rule\ assoc{\isacharparenright}\isanewline
\ \ \isacommand{also}\isamarkupfalse%
\ \isacommand{have}\isamarkupfalse%
\ {\isachardoublequoteopen}x{\isasyminverse}\ {\isasymcirc}\ x\ {\isacharequal}\ {\isadigit{1}}{\isachardoublequoteclose}\ \isacommand{by}\isamarkupfalse%
\ {\isacharparenleft}rule\ left{\isacharunderscore}inv{\isacharparenright}\isanewline
\ \ \isacommand{also}\isamarkupfalse%
\ \isacommand{have}\isamarkupfalse%
\ {\isachardoublequoteopen}{\isacharparenleft}{\isacharparenleft}x{\isasyminverse}{\isacharparenright}{\isasyminverse}\ {\isasymcirc}\ {\isasymdots}{\isacharparenright}\ {\isasymcirc}\ x{\isasyminverse}\ {\isacharequal}\ {\isacharparenleft}x{\isasyminverse}{\isacharparenright}{\isasyminverse}\ {\isasymcirc}\ {\isacharparenleft}{\isadigit{1}}\ {\isasymcirc}\ x{\isasyminverse}{\isacharparenright}{\isachardoublequoteclose}\ \isacommand{by}\isamarkupfalse%
\ {\isacharparenleft}rule\ assoc{\isacharparenright}\isanewline
\ \ \isacommand{also}\isamarkupfalse%
\ \isacommand{have}\isamarkupfalse%
\ {\isachardoublequoteopen}{\isadigit{1}}\ {\isasymcirc}\ x{\isasyminverse}\ {\isacharequal}\ x{\isasyminverse}{\isachardoublequoteclose}\ \isacommand{by}\isamarkupfalse%
\ {\isacharparenleft}rule\ left{\isacharunderscore}unit{\isacharparenright}\isanewline
\ \ \isacommand{also}\isamarkupfalse%
\ \isacommand{have}\isamarkupfalse%
\ {\isachardoublequoteopen}{\isacharparenleft}x{\isasyminverse}{\isacharparenright}{\isasyminverse}\ {\isasymcirc}\ {\isasymdots}\ {\isacharequal}\ {\isadigit{1}}{\isachardoublequoteclose}\ \isacommand{by}\isamarkupfalse%
\ {\isacharparenleft}rule\ left{\isacharunderscore}inv{\isacharparenright}\isanewline
\ \ \isacommand{finally}\isamarkupfalse%
\ \isacommand{show}\isamarkupfalse%
\ {\isachardoublequoteopen}x\ {\isasymcirc}\ x{\isasyminverse}\ {\isacharequal}\ {\isadigit{1}}{\isachardoublequoteclose}\ \isacommand{{\isachardot}}\isamarkupfalse%
\isanewline
\isacommand{qed}\isamarkupfalse%
%
\endisatagproof
{\isafoldproof}%
%
\isadelimproof
\isanewline
%
\endisadelimproof
\isanewline
\isacommand{theorem}\isamarkupfalse%
\ right{\isacharunderscore}unit{\isacharcolon}\ {\isachardoublequoteopen}x\ {\isasymcirc}\ {\isadigit{1}}\ {\isacharequal}\ x{\isachardoublequoteclose}\isanewline
%
\isadelimproof
%
\endisadelimproof
%
\isatagproof
\isacommand{proof}\isamarkupfalse%
\ {\isacharminus}\isanewline
\ \ \isacommand{have}\isamarkupfalse%
\ {\isachardoublequoteopen}{\isadigit{1}}\ {\isacharequal}\ x{\isasyminverse}\ {\isasymcirc}\ x{\isachardoublequoteclose}\ \isacommand{by}\isamarkupfalse%
\ {\isacharparenleft}rule\ left{\isacharunderscore}inv\ {\isacharbrackleft}symmetric{\isacharbrackright}{\isacharparenright}\isanewline
\ \ \isacommand{also}\isamarkupfalse%
\ \isacommand{have}\isamarkupfalse%
\ {\isachardoublequoteopen}x\ {\isasymcirc}\ {\isasymdots}\ {\isacharequal}\ {\isacharparenleft}x\ {\isasymcirc}\ x{\isasyminverse}{\isacharparenright}\ {\isasymcirc}\ x{\isachardoublequoteclose}\ \isacommand{by}\isamarkupfalse%
\ {\isacharparenleft}rule\ assoc\ {\isacharbrackleft}symmetric{\isacharbrackright}{\isacharparenright}\isanewline
\ \ \isacommand{also}\isamarkupfalse%
\ \isacommand{have}\isamarkupfalse%
\ {\isachardoublequoteopen}x\ {\isasymcirc}\ x{\isasyminverse}\ {\isacharequal}\ {\isadigit{1}}{\isachardoublequoteclose}\ \isacommand{by}\isamarkupfalse%
\ {\isacharparenleft}rule\ right{\isacharunderscore}inv{\isacharparenright}\isanewline
\ \ \isacommand{also}\isamarkupfalse%
\ \isacommand{have}\isamarkupfalse%
\ {\isachardoublequoteopen}{\isasymdots}\ {\isasymcirc}\ x\ {\isacharequal}\ x{\isachardoublequoteclose}\ \isacommand{by}\isamarkupfalse%
\ {\isacharparenleft}rule\ left{\isacharunderscore}unit{\isacharparenright}\isanewline
\ \ \isacommand{finally}\isamarkupfalse%
\ \isacommand{show}\isamarkupfalse%
\ {\isachardoublequoteopen}x\ {\isasymcirc}\ {\isadigit{1}}\ {\isacharequal}\ x{\isachardoublequoteclose}\ \isacommand{{\isachardot}}\isamarkupfalse%
\isanewline
\isacommand{qed}\isamarkupfalse%
%
\endisatagproof
{\isafoldproof}%
%
\isadelimproof
%
\endisadelimproof
%
\begin{isamarkuptext}%
\noindent Reasoning from basic axioms is often tedious.  Our proofs
  work by producing various instances of the given rules (potentially
  the symmetric form) using the pattern ``\hyperlink{command.have}{\mbox{\isa{\isacommand{have}}}}~\isa{eq}~\hyperlink{command.by}{\mbox{\isa{\isacommand{by}}}}~\isa{{\isachardoublequote}{\isacharparenleft}rule\ r{\isacharparenright}{\isachardoublequote}}'' and composing the chain of
  results via \hyperlink{command.also}{\mbox{\isa{\isacommand{also}}}}/\hyperlink{command.finally}{\mbox{\isa{\isacommand{finally}}}}.  These steps may
  involve any of the transitivity rules declared in
  \secref{sec:framework-ex-equal}, namely \isa{trans} in combining
  the first two results in \isa{right{\isacharunderscore}inv} and in the final steps of
  both proofs, \isa{forw{\isacharunderscore}subst} in the first combination of \isa{right{\isacharunderscore}unit}, and \isa{back{\isacharunderscore}subst} in all other calculational steps.

  Occasional substitutions in calculations are adequate, but should
  not be over-emphasized.  The other extreme is to compose a chain by
  plain transitivity only, with replacements occurring always in
  topmost position. For example:%
\end{isamarkuptext}%
\isamarkuptrue%
%
\isadelimproof
%
\endisadelimproof
%
\isatagproof
\ \ \isacommand{have}\isamarkupfalse%
\ {\isachardoublequoteopen}x\ {\isasymcirc}\ {\isadigit{1}}\ {\isacharequal}\ x\ {\isasymcirc}\ {\isacharparenleft}x{\isasyminverse}\ {\isasymcirc}\ x{\isacharparenright}{\isachardoublequoteclose}\ \isacommand{unfolding}\isamarkupfalse%
\ left{\isacharunderscore}inv\ \isacommand{{\isachardot}{\isachardot}}\isamarkupfalse%
\isanewline
\ \ \isacommand{also}\isamarkupfalse%
\ \isacommand{have}\isamarkupfalse%
\ {\isachardoublequoteopen}{\isasymdots}\ {\isacharequal}\ {\isacharparenleft}x\ {\isasymcirc}\ x{\isasyminverse}{\isacharparenright}\ {\isasymcirc}\ x{\isachardoublequoteclose}\ \isacommand{unfolding}\isamarkupfalse%
\ assoc\ \isacommand{{\isachardot}{\isachardot}}\isamarkupfalse%
\isanewline
\ \ \isacommand{also}\isamarkupfalse%
\ \isacommand{have}\isamarkupfalse%
\ {\isachardoublequoteopen}{\isasymdots}\ {\isacharequal}\ {\isadigit{1}}\ {\isasymcirc}\ x{\isachardoublequoteclose}\ \isacommand{unfolding}\isamarkupfalse%
\ right{\isacharunderscore}inv\ \isacommand{{\isachardot}{\isachardot}}\isamarkupfalse%
\isanewline
\ \ \isacommand{also}\isamarkupfalse%
\ \isacommand{have}\isamarkupfalse%
\ {\isachardoublequoteopen}{\isasymdots}\ {\isacharequal}\ x{\isachardoublequoteclose}\ \isacommand{unfolding}\isamarkupfalse%
\ left{\isacharunderscore}unit\ \isacommand{{\isachardot}{\isachardot}}\isamarkupfalse%
\isanewline
\ \ \isacommand{finally}\isamarkupfalse%
\ \isacommand{have}\isamarkupfalse%
\ {\isachardoublequoteopen}x\ {\isasymcirc}\ {\isadigit{1}}\ {\isacharequal}\ x{\isachardoublequoteclose}\ \isacommand{{\isachardot}}\isamarkupfalse%
%
\endisatagproof
{\isafoldproof}%
%
\isadelimproof
%
\endisadelimproof
%
\begin{isamarkuptext}%
\noindent Here we have re-used the built-in mechanism for unfolding
  definitions in order to normalize each equational problem.  A more
  realistic object-logic would include proper setup for the Simplifier
  (\secref{sec:simplifier}), the main automated tool for equational
  reasoning in Isabelle.  Then ``\hyperlink{command.unfolding}{\mbox{\isa{\isacommand{unfolding}}}}~\isa{left{\isacharunderscore}inv}~\hyperlink{command.ddot}{\mbox{\isa{\isacommand{{\isachardot}{\isachardot}}}}}'' would become ``\hyperlink{command.by}{\mbox{\isa{\isacommand{by}}}}~\isa{{\isachardoublequote}{\isacharparenleft}simp\ only{\isacharcolon}\ left{\isacharunderscore}inv{\isacharparenright}{\isachardoublequote}}'' etc.%
\end{isamarkuptext}%
\isamarkuptrue%
\isacommand{end}\isamarkupfalse%
%
\isamarkupsubsection{Propositional logic \label{sec:framework-ex-prop}%
}
\isamarkuptrue%
%
\begin{isamarkuptext}%
We axiomatize basic connectives of propositional logic: implication,
  disjunction, and conjunction.  The associated rules are modeled
  after Gentzen's system of Natural Deduction \cite{Gentzen:1935}.%
\end{isamarkuptext}%
\isamarkuptrue%
\isacommand{axiomatization}\isamarkupfalse%
\isanewline
\ \ imp\ {\isacharcolon}{\isacharcolon}\ {\isachardoublequoteopen}o\ {\isasymRightarrow}\ o\ {\isasymRightarrow}\ o{\isachardoublequoteclose}\ \ {\isacharparenleft}\isakeyword{infixr}\ {\isachardoublequoteopen}{\isasymlongrightarrow}{\isachardoublequoteclose}\ {\isadigit{2}}{\isadigit{5}}{\isacharparenright}\ \isakeyword{where}\isanewline
\ \ impI\ {\isacharbrackleft}intro{\isacharbrackright}{\isacharcolon}\ {\isachardoublequoteopen}{\isacharparenleft}A\ {\isasymLongrightarrow}\ B{\isacharparenright}\ {\isasymLongrightarrow}\ A\ {\isasymlongrightarrow}\ B{\isachardoublequoteclose}\ \isakeyword{and}\isanewline
\ \ impD\ {\isacharbrackleft}dest{\isacharbrackright}{\isacharcolon}\ {\isachardoublequoteopen}{\isacharparenleft}A\ {\isasymlongrightarrow}\ B{\isacharparenright}\ {\isasymLongrightarrow}\ A\ {\isasymLongrightarrow}\ B{\isachardoublequoteclose}\isanewline
\isanewline
\isacommand{axiomatization}\isamarkupfalse%
\isanewline
\ \ disj\ {\isacharcolon}{\isacharcolon}\ {\isachardoublequoteopen}o\ {\isasymRightarrow}\ o\ {\isasymRightarrow}\ o{\isachardoublequoteclose}\ \ {\isacharparenleft}\isakeyword{infixr}\ {\isachardoublequoteopen}{\isasymor}{\isachardoublequoteclose}\ {\isadigit{3}}{\isadigit{0}}{\isacharparenright}\ \isakeyword{where}\isanewline
\ \ disjI\isactrlisub {\isadigit{1}}\ {\isacharbrackleft}intro{\isacharbrackright}{\isacharcolon}\ {\isachardoublequoteopen}A\ {\isasymLongrightarrow}\ A\ {\isasymor}\ B{\isachardoublequoteclose}\ \isakeyword{and}\isanewline
\ \ disjI\isactrlisub {\isadigit{2}}\ {\isacharbrackleft}intro{\isacharbrackright}{\isacharcolon}\ {\isachardoublequoteopen}B\ {\isasymLongrightarrow}\ A\ {\isasymor}\ B{\isachardoublequoteclose}\ \isakeyword{and}\isanewline
\ \ disjE\ {\isacharbrackleft}elim{\isacharbrackright}{\isacharcolon}\ {\isachardoublequoteopen}A\ {\isasymor}\ B\ {\isasymLongrightarrow}\ {\isacharparenleft}A\ {\isasymLongrightarrow}\ C{\isacharparenright}\ {\isasymLongrightarrow}\ {\isacharparenleft}B\ {\isasymLongrightarrow}\ C{\isacharparenright}\ {\isasymLongrightarrow}\ C{\isachardoublequoteclose}\isanewline
\isanewline
\isacommand{axiomatization}\isamarkupfalse%
\isanewline
\ \ conj\ {\isacharcolon}{\isacharcolon}\ {\isachardoublequoteopen}o\ {\isasymRightarrow}\ o\ {\isasymRightarrow}\ o{\isachardoublequoteclose}\ \ {\isacharparenleft}\isakeyword{infixr}\ {\isachardoublequoteopen}{\isasymand}{\isachardoublequoteclose}\ {\isadigit{3}}{\isadigit{5}}{\isacharparenright}\ \isakeyword{where}\isanewline
\ \ conjI\ {\isacharbrackleft}intro{\isacharbrackright}{\isacharcolon}\ {\isachardoublequoteopen}A\ {\isasymLongrightarrow}\ B\ {\isasymLongrightarrow}\ A\ {\isasymand}\ B{\isachardoublequoteclose}\ \isakeyword{and}\isanewline
\ \ conjD\isactrlisub {\isadigit{1}}{\isacharcolon}\ {\isachardoublequoteopen}A\ {\isasymand}\ B\ {\isasymLongrightarrow}\ A{\isachardoublequoteclose}\ \isakeyword{and}\isanewline
\ \ conjD\isactrlisub {\isadigit{2}}{\isacharcolon}\ {\isachardoublequoteopen}A\ {\isasymand}\ B\ {\isasymLongrightarrow}\ B{\isachardoublequoteclose}%
\begin{isamarkuptext}%
\noindent The conjunctive destructions have the disadvantage that
  decomposing \isa{{\isachardoublequote}A\ {\isasymand}\ B{\isachardoublequote}} involves an immediate decision which
  component should be projected.  The more convenient simultaneous
  elimination \isa{{\isachardoublequote}A\ {\isasymand}\ B\ {\isasymLongrightarrow}\ {\isacharparenleft}A\ {\isasymLongrightarrow}\ B\ {\isasymLongrightarrow}\ C{\isacharparenright}\ {\isasymLongrightarrow}\ C{\isachardoublequote}} can be derived as
  follows:%
\end{isamarkuptext}%
\isamarkuptrue%
\isacommand{theorem}\isamarkupfalse%
\ conjE\ {\isacharbrackleft}elim{\isacharbrackright}{\isacharcolon}\isanewline
\ \ \isakeyword{assumes}\ {\isachardoublequoteopen}A\ {\isasymand}\ B{\isachardoublequoteclose}\isanewline
\ \ \isakeyword{obtains}\ A\ \isakeyword{and}\ B\isanewline
%
\isadelimproof
%
\endisadelimproof
%
\isatagproof
\isacommand{proof}\isamarkupfalse%
\isanewline
\ \ \isacommand{from}\isamarkupfalse%
\ {\isacharbackquoteopen}A\ {\isasymand}\ B{\isacharbackquoteclose}\ \isacommand{show}\isamarkupfalse%
\ A\ \isacommand{by}\isamarkupfalse%
\ {\isacharparenleft}rule\ conjD\isactrlisub {\isadigit{1}}{\isacharparenright}\isanewline
\ \ \isacommand{from}\isamarkupfalse%
\ {\isacharbackquoteopen}A\ {\isasymand}\ B{\isacharbackquoteclose}\ \isacommand{show}\isamarkupfalse%
\ B\ \isacommand{by}\isamarkupfalse%
\ {\isacharparenleft}rule\ conjD\isactrlisub {\isadigit{2}}{\isacharparenright}\isanewline
\isacommand{qed}\isamarkupfalse%
%
\endisatagproof
{\isafoldproof}%
%
\isadelimproof
%
\endisadelimproof
%
\begin{isamarkuptext}%
\noindent Here is an example of swapping conjuncts with a single
  intermediate elimination step:%
\end{isamarkuptext}%
\isamarkuptrue%
%
\isadelimproof
%
\endisadelimproof
%
\isatagproof
\ \ \isacommand{assume}\isamarkupfalse%
\ {\isachardoublequoteopen}A\ {\isasymand}\ B{\isachardoublequoteclose}\isanewline
\ \ \isacommand{then}\isamarkupfalse%
\ \isacommand{obtain}\isamarkupfalse%
\ B\ \isakeyword{and}\ A\ \isacommand{{\isachardot}{\isachardot}}\isamarkupfalse%
\isanewline
\ \ \isacommand{then}\isamarkupfalse%
\ \isacommand{have}\isamarkupfalse%
\ {\isachardoublequoteopen}B\ {\isasymand}\ A{\isachardoublequoteclose}\ \isacommand{{\isachardot}{\isachardot}}\isamarkupfalse%
%
\endisatagproof
{\isafoldproof}%
%
\isadelimproof
%
\endisadelimproof
%
\begin{isamarkuptext}%
\noindent Note that the analogous elimination rule for disjunction
  ``\isa{{\isachardoublequote}{\isasymASSUMES}\ A\ {\isasymor}\ B\ {\isasymOBTAINS}\ A\ {\isasymBBAR}\ B{\isachardoublequote}}'' coincides with
  the original axiomatization of \isa{disjE}.

  \medskip We continue propositional logic by introducing absurdity
  with its characteristic elimination.  Plain truth may then be
  defined as a proposition that is trivially true.%
\end{isamarkuptext}%
\isamarkuptrue%
\isacommand{axiomatization}\isamarkupfalse%
\isanewline
\ \ false\ {\isacharcolon}{\isacharcolon}\ o\ \ {\isacharparenleft}{\isachardoublequoteopen}{\isasymbottom}{\isachardoublequoteclose}{\isacharparenright}\ \isakeyword{where}\isanewline
\ \ falseE\ {\isacharbrackleft}elim{\isacharbrackright}{\isacharcolon}\ {\isachardoublequoteopen}{\isasymbottom}\ {\isasymLongrightarrow}\ A{\isachardoublequoteclose}\isanewline
\isanewline
\isacommand{definition}\isamarkupfalse%
\isanewline
\ \ true\ {\isacharcolon}{\isacharcolon}\ o\ \ {\isacharparenleft}{\isachardoublequoteopen}{\isasymtop}{\isachardoublequoteclose}{\isacharparenright}\ \isakeyword{where}\isanewline
\ \ {\isachardoublequoteopen}{\isasymtop}\ {\isasymequiv}\ {\isasymbottom}\ {\isasymlongrightarrow}\ {\isasymbottom}{\isachardoublequoteclose}\isanewline
\isanewline
\isacommand{theorem}\isamarkupfalse%
\ trueI\ {\isacharbrackleft}intro{\isacharbrackright}{\isacharcolon}\ {\isasymtop}\isanewline
%
\isadelimproof
\ \ %
\endisadelimproof
%
\isatagproof
\isacommand{unfolding}\isamarkupfalse%
\ true{\isacharunderscore}def\ \isacommand{{\isachardot}{\isachardot}}\isamarkupfalse%
%
\endisatagproof
{\isafoldproof}%
%
\isadelimproof
%
\endisadelimproof
%
\begin{isamarkuptext}%
\medskip\noindent Now negation represents an implication towards
  absurdity:%
\end{isamarkuptext}%
\isamarkuptrue%
\isacommand{definition}\isamarkupfalse%
\isanewline
\ \ not\ {\isacharcolon}{\isacharcolon}\ {\isachardoublequoteopen}o\ {\isasymRightarrow}\ o{\isachardoublequoteclose}\ \ {\isacharparenleft}{\isachardoublequoteopen}{\isasymnot}\ {\isacharunderscore}{\isachardoublequoteclose}\ {\isacharbrackleft}{\isadigit{4}}{\isadigit{0}}{\isacharbrackright}\ {\isadigit{4}}{\isadigit{0}}{\isacharparenright}\ \isakeyword{where}\isanewline
\ \ {\isachardoublequoteopen}{\isasymnot}\ A\ {\isasymequiv}\ A\ {\isasymlongrightarrow}\ {\isasymbottom}{\isachardoublequoteclose}\isanewline
\isanewline
\isacommand{theorem}\isamarkupfalse%
\ notI\ {\isacharbrackleft}intro{\isacharbrackright}{\isacharcolon}\isanewline
\ \ \isakeyword{assumes}\ {\isachardoublequoteopen}A\ {\isasymLongrightarrow}\ {\isasymbottom}{\isachardoublequoteclose}\isanewline
\ \ \isakeyword{shows}\ {\isachardoublequoteopen}{\isasymnot}\ A{\isachardoublequoteclose}\isanewline
%
\isadelimproof
%
\endisadelimproof
%
\isatagproof
\isacommand{unfolding}\isamarkupfalse%
\ not{\isacharunderscore}def\isanewline
\isacommand{proof}\isamarkupfalse%
\isanewline
\ \ \isacommand{assume}\isamarkupfalse%
\ A\isanewline
\ \ \isacommand{then}\isamarkupfalse%
\ \isacommand{show}\isamarkupfalse%
\ {\isasymbottom}\ \isacommand{by}\isamarkupfalse%
\ {\isacharparenleft}rule\ {\isacharbackquoteopen}A\ {\isasymLongrightarrow}\ {\isasymbottom}{\isacharbackquoteclose}{\isacharparenright}\isanewline
\isacommand{qed}\isamarkupfalse%
%
\endisatagproof
{\isafoldproof}%
%
\isadelimproof
\isanewline
%
\endisadelimproof
\isanewline
\isacommand{theorem}\isamarkupfalse%
\ notE\ {\isacharbrackleft}elim{\isacharbrackright}{\isacharcolon}\isanewline
\ \ \isakeyword{assumes}\ {\isachardoublequoteopen}{\isasymnot}\ A{\isachardoublequoteclose}\ \isakeyword{and}\ A\isanewline
\ \ \isakeyword{shows}\ B\isanewline
%
\isadelimproof
%
\endisadelimproof
%
\isatagproof
\isacommand{proof}\isamarkupfalse%
\ {\isacharminus}\isanewline
\ \ \isacommand{from}\isamarkupfalse%
\ {\isacharbackquoteopen}{\isasymnot}\ A{\isacharbackquoteclose}\ \isacommand{have}\isamarkupfalse%
\ {\isachardoublequoteopen}A\ {\isasymlongrightarrow}\ {\isasymbottom}{\isachardoublequoteclose}\ \isacommand{unfolding}\isamarkupfalse%
\ not{\isacharunderscore}def\ \isacommand{{\isachardot}}\isamarkupfalse%
\isanewline
\ \ \isacommand{from}\isamarkupfalse%
\ {\isacharbackquoteopen}A\ {\isasymlongrightarrow}\ {\isasymbottom}{\isacharbackquoteclose}\ \isakeyword{and}\ {\isacharbackquoteopen}A{\isacharbackquoteclose}\ \isacommand{have}\isamarkupfalse%
\ {\isasymbottom}\ \isacommand{{\isachardot}{\isachardot}}\isamarkupfalse%
\isanewline
\ \ \isacommand{then}\isamarkupfalse%
\ \isacommand{show}\isamarkupfalse%
\ B\ \isacommand{{\isachardot}{\isachardot}}\isamarkupfalse%
\isanewline
\isacommand{qed}\isamarkupfalse%
%
\endisatagproof
{\isafoldproof}%
%
\isadelimproof
%
\endisadelimproof
%
\isamarkupsubsection{Classical logic%
}
\isamarkuptrue%
%
\begin{isamarkuptext}%
Subsequently we state the principle of classical contradiction as a
  local assumption.  Thus we refrain from forcing the object-logic
  into the classical perspective.  Within that context, we may derive
  well-known consequences of the classical principle.%
\end{isamarkuptext}%
\isamarkuptrue%
\isacommand{locale}\isamarkupfalse%
\ classical\ {\isacharequal}\isanewline
\ \ \isakeyword{assumes}\ classical{\isacharcolon}\ {\isachardoublequoteopen}{\isacharparenleft}{\isasymnot}\ C\ {\isasymLongrightarrow}\ C{\isacharparenright}\ {\isasymLongrightarrow}\ C{\isachardoublequoteclose}\isanewline
\isakeyword{begin}\isanewline
\isanewline
\isacommand{theorem}\isamarkupfalse%
\ double{\isacharunderscore}negation{\isacharcolon}\isanewline
\ \ \isakeyword{assumes}\ {\isachardoublequoteopen}{\isasymnot}\ {\isasymnot}\ C{\isachardoublequoteclose}\isanewline
\ \ \isakeyword{shows}\ C\isanewline
%
\isadelimproof
%
\endisadelimproof
%
\isatagproof
\isacommand{proof}\isamarkupfalse%
\ {\isacharparenleft}rule\ classical{\isacharparenright}\isanewline
\ \ \isacommand{assume}\isamarkupfalse%
\ {\isachardoublequoteopen}{\isasymnot}\ C{\isachardoublequoteclose}\isanewline
\ \ \isacommand{with}\isamarkupfalse%
\ {\isacharbackquoteopen}{\isasymnot}\ {\isasymnot}\ C{\isacharbackquoteclose}\ \isacommand{show}\isamarkupfalse%
\ C\ \isacommand{{\isachardot}{\isachardot}}\isamarkupfalse%
\isanewline
\isacommand{qed}\isamarkupfalse%
%
\endisatagproof
{\isafoldproof}%
%
\isadelimproof
\isanewline
%
\endisadelimproof
\isanewline
\isacommand{theorem}\isamarkupfalse%
\ tertium{\isacharunderscore}non{\isacharunderscore}datur{\isacharcolon}\ {\isachardoublequoteopen}C\ {\isasymor}\ {\isasymnot}\ C{\isachardoublequoteclose}\isanewline
%
\isadelimproof
%
\endisadelimproof
%
\isatagproof
\isacommand{proof}\isamarkupfalse%
\ {\isacharparenleft}rule\ double{\isacharunderscore}negation{\isacharparenright}\isanewline
\ \ \isacommand{show}\isamarkupfalse%
\ {\isachardoublequoteopen}{\isasymnot}\ {\isasymnot}\ {\isacharparenleft}C\ {\isasymor}\ {\isasymnot}\ C{\isacharparenright}{\isachardoublequoteclose}\isanewline
\ \ \isacommand{proof}\isamarkupfalse%
\isanewline
\ \ \ \ \isacommand{assume}\isamarkupfalse%
\ {\isachardoublequoteopen}{\isasymnot}\ {\isacharparenleft}C\ {\isasymor}\ {\isasymnot}\ C{\isacharparenright}{\isachardoublequoteclose}\isanewline
\ \ \ \ \isacommand{have}\isamarkupfalse%
\ {\isachardoublequoteopen}{\isasymnot}\ C{\isachardoublequoteclose}\isanewline
\ \ \ \ \isacommand{proof}\isamarkupfalse%
\isanewline
\ \ \ \ \ \ \isacommand{assume}\isamarkupfalse%
\ C\ \isacommand{then}\isamarkupfalse%
\ \isacommand{have}\isamarkupfalse%
\ {\isachardoublequoteopen}C\ {\isasymor}\ {\isasymnot}\ C{\isachardoublequoteclose}\ \isacommand{{\isachardot}{\isachardot}}\isamarkupfalse%
\isanewline
\ \ \ \ \ \ \isacommand{with}\isamarkupfalse%
\ {\isacharbackquoteopen}{\isasymnot}\ {\isacharparenleft}C\ {\isasymor}\ {\isasymnot}\ C{\isacharparenright}{\isacharbackquoteclose}\ \isacommand{show}\isamarkupfalse%
\ {\isasymbottom}\ \isacommand{{\isachardot}{\isachardot}}\isamarkupfalse%
\isanewline
\ \ \ \ \isacommand{qed}\isamarkupfalse%
\isanewline
\ \ \ \ \isacommand{then}\isamarkupfalse%
\ \isacommand{have}\isamarkupfalse%
\ {\isachardoublequoteopen}C\ {\isasymor}\ {\isasymnot}\ C{\isachardoublequoteclose}\ \isacommand{{\isachardot}{\isachardot}}\isamarkupfalse%
\isanewline
\ \ \ \ \isacommand{with}\isamarkupfalse%
\ {\isacharbackquoteopen}{\isasymnot}\ {\isacharparenleft}C\ {\isasymor}\ {\isasymnot}\ C{\isacharparenright}{\isacharbackquoteclose}\ \isacommand{show}\isamarkupfalse%
\ {\isasymbottom}\ \isacommand{{\isachardot}{\isachardot}}\isamarkupfalse%
\isanewline
\ \ \isacommand{qed}\isamarkupfalse%
\isanewline
\isacommand{qed}\isamarkupfalse%
%
\endisatagproof
{\isafoldproof}%
%
\isadelimproof
%
\endisadelimproof
%
\begin{isamarkuptext}%
\noindent These examples illustrate both classical reasoning and
  non-trivial propositional proofs in general.  All three rules
  characterize classical logic independently, but the original rule is
  already the most convenient to use, because it leaves the conclusion
  unchanged.  Note that \isa{{\isachardoublequote}{\isacharparenleft}{\isasymnot}\ C\ {\isasymLongrightarrow}\ C{\isacharparenright}\ {\isasymLongrightarrow}\ C{\isachardoublequote}} fits again into our
  format for eliminations, despite the additional twist that the
  context refers to the main conclusion.  So we may write \isa{classical} as the Isar statement ``\isa{{\isachardoublequote}{\isasymOBTAINS}\ {\isasymnot}\ thesis{\isachardoublequote}}''.
  This also explains nicely how classical reasoning really works:
  whatever the main \isa{thesis} might be, we may always assume its
  negation!%
\end{isamarkuptext}%
\isamarkuptrue%
\isacommand{end}\isamarkupfalse%
%
\isamarkupsubsection{Quantifiers \label{sec:framework-ex-quant}%
}
\isamarkuptrue%
%
\begin{isamarkuptext}%
Representing quantifiers is easy, thanks to the higher-order nature
  of the underlying framework.  According to the well-known technique
  introduced by Church \cite{church40}, quantifiers are operators on
  predicates, which are syntactically represented as \isa{{\isachardoublequote}{\isasymlambda}{\isachardoublequote}}-terms
  of type \isa{{\isachardoublequote}i\ {\isasymRightarrow}\ o{\isachardoublequote}}.  Binder notation turns \isa{{\isachardoublequote}All\ {\isacharparenleft}{\isasymlambda}x{\isachardot}\ B\ x{\isacharparenright}{\isachardoublequote}} into \isa{{\isachardoublequote}{\isasymforall}x{\isachardot}\ B\ x{\isachardoublequote}} etc.%
\end{isamarkuptext}%
\isamarkuptrue%
\isacommand{axiomatization}\isamarkupfalse%
\isanewline
\ \ All\ {\isacharcolon}{\isacharcolon}\ {\isachardoublequoteopen}{\isacharparenleft}i\ {\isasymRightarrow}\ o{\isacharparenright}\ {\isasymRightarrow}\ o{\isachardoublequoteclose}\ \ {\isacharparenleft}\isakeyword{binder}\ {\isachardoublequoteopen}{\isasymforall}{\isachardoublequoteclose}\ {\isadigit{1}}{\isadigit{0}}{\isacharparenright}\ \isakeyword{where}\isanewline
\ \ allI\ {\isacharbrackleft}intro{\isacharbrackright}{\isacharcolon}\ {\isachardoublequoteopen}{\isacharparenleft}{\isasymAnd}x{\isachardot}\ B\ x{\isacharparenright}\ {\isasymLongrightarrow}\ {\isasymforall}x{\isachardot}\ B\ x{\isachardoublequoteclose}\ \isakeyword{and}\isanewline
\ \ allD\ {\isacharbrackleft}dest{\isacharbrackright}{\isacharcolon}\ {\isachardoublequoteopen}{\isacharparenleft}{\isasymforall}x{\isachardot}\ B\ x{\isacharparenright}\ {\isasymLongrightarrow}\ B\ a{\isachardoublequoteclose}\isanewline
\isanewline
\isacommand{axiomatization}\isamarkupfalse%
\isanewline
\ \ Ex\ {\isacharcolon}{\isacharcolon}\ {\isachardoublequoteopen}{\isacharparenleft}i\ {\isasymRightarrow}\ o{\isacharparenright}\ {\isasymRightarrow}\ o{\isachardoublequoteclose}\ \ {\isacharparenleft}\isakeyword{binder}\ {\isachardoublequoteopen}{\isasymexists}{\isachardoublequoteclose}\ {\isadigit{1}}{\isadigit{0}}{\isacharparenright}\ \isakeyword{where}\isanewline
\ \ exI\ {\isacharbrackleft}intro{\isacharbrackright}{\isacharcolon}\ {\isachardoublequoteopen}B\ a\ {\isasymLongrightarrow}\ {\isacharparenleft}{\isasymexists}x{\isachardot}\ B\ x{\isacharparenright}{\isachardoublequoteclose}\ \isakeyword{and}\isanewline
\ \ exE\ {\isacharbrackleft}elim{\isacharbrackright}{\isacharcolon}\ {\isachardoublequoteopen}{\isacharparenleft}{\isasymexists}x{\isachardot}\ B\ x{\isacharparenright}\ {\isasymLongrightarrow}\ {\isacharparenleft}{\isasymAnd}x{\isachardot}\ B\ x\ {\isasymLongrightarrow}\ C{\isacharparenright}\ {\isasymLongrightarrow}\ C{\isachardoublequoteclose}%
\begin{isamarkuptext}%
\noindent The statement of \isa{exE} corresponds to ``\isa{{\isachardoublequote}{\isasymASSUMES}\ {\isasymexists}x{\isachardot}\ B\ x\ {\isasymOBTAINS}\ x\ {\isasymWHERE}\ B\ x{\isachardoublequote}}'' in Isar.  In the
  subsequent example we illustrate quantifier reasoning involving all
  four rules:%
\end{isamarkuptext}%
\isamarkuptrue%
\isacommand{theorem}\isamarkupfalse%
\isanewline
\ \ \isakeyword{assumes}\ {\isachardoublequoteopen}{\isasymexists}x{\isachardot}\ {\isasymforall}y{\isachardot}\ R\ x\ y{\isachardoublequoteclose}\isanewline
\ \ \isakeyword{shows}\ {\isachardoublequoteopen}{\isasymforall}y{\isachardot}\ {\isasymexists}x{\isachardot}\ R\ x\ y{\isachardoublequoteclose}\isanewline
%
\isadelimproof
%
\endisadelimproof
%
\isatagproof
\isacommand{proof}\isamarkupfalse%
\ \ \ \ %
\isamarkupcmt{\isa{{\isachardoublequote}{\isasymforall}{\isachardoublequote}} introduction%
}
\isanewline
\ \ \isacommand{obtain}\isamarkupfalse%
\ x\ \isakeyword{where}\ {\isachardoublequoteopen}{\isasymforall}y{\isachardot}\ R\ x\ y{\isachardoublequoteclose}\ \isacommand{using}\isamarkupfalse%
\ {\isacharbackquoteopen}{\isasymexists}x{\isachardot}\ {\isasymforall}y{\isachardot}\ R\ x\ y{\isacharbackquoteclose}\ \isacommand{{\isachardot}{\isachardot}}\isamarkupfalse%
\ \ \ \ %
\isamarkupcmt{\isa{{\isachardoublequote}{\isasymexists}{\isachardoublequote}} elimination%
}
\isanewline
\ \ \isacommand{fix}\isamarkupfalse%
\ y\ \isacommand{have}\isamarkupfalse%
\ {\isachardoublequoteopen}R\ x\ y{\isachardoublequoteclose}\ \isacommand{using}\isamarkupfalse%
\ {\isacharbackquoteopen}{\isasymforall}y{\isachardot}\ R\ x\ y{\isacharbackquoteclose}\ \isacommand{{\isachardot}{\isachardot}}\isamarkupfalse%
\ \ \ \ %
\isamarkupcmt{\isa{{\isachardoublequote}{\isasymforall}{\isachardoublequote}} destruction%
}
\isanewline
\ \ \isacommand{then}\isamarkupfalse%
\ \isacommand{show}\isamarkupfalse%
\ {\isachardoublequoteopen}{\isasymexists}x{\isachardot}\ R\ x\ y{\isachardoublequoteclose}\ \isacommand{{\isachardot}{\isachardot}}\isamarkupfalse%
\ \ \ \ %
\isamarkupcmt{\isa{{\isachardoublequote}{\isasymexists}{\isachardoublequote}} introduction%
}
\isanewline
\isacommand{qed}\isamarkupfalse%
%
\endisatagproof
{\isafoldproof}%
%
\isadelimproof
%
\endisadelimproof
%
\isamarkupsubsection{Canonical reasoning patterns%
}
\isamarkuptrue%
%
\begin{isamarkuptext}%
The main rules of first-order predicate logic from
  \secref{sec:framework-ex-prop} and \secref{sec:framework-ex-quant}
  can now be summarized as follows, using the native Isar statement
  format of \secref{sec:framework-stmt}.

  \medskip
  \begin{tabular}{l}
  \isa{{\isachardoublequote}impI{\isacharcolon}\ {\isasymASSUMES}\ A\ {\isasymLongrightarrow}\ B\ {\isasymSHOWS}\ A\ {\isasymlongrightarrow}\ B{\isachardoublequote}} \\
  \isa{{\isachardoublequote}impD{\isacharcolon}\ {\isasymASSUMES}\ A\ {\isasymlongrightarrow}\ B\ {\isasymAND}\ A\ {\isasymSHOWS}\ B{\isachardoublequote}} \\[1ex]

  \isa{{\isachardoublequote}disjI\isactrlisub {\isadigit{1}}{\isacharcolon}\ {\isasymASSUMES}\ A\ {\isasymSHOWS}\ A\ {\isasymor}\ B{\isachardoublequote}} \\
  \isa{{\isachardoublequote}disjI\isactrlisub {\isadigit{2}}{\isacharcolon}\ {\isasymASSUMES}\ B\ {\isasymSHOWS}\ A\ {\isasymor}\ B{\isachardoublequote}} \\
  \isa{{\isachardoublequote}disjE{\isacharcolon}\ {\isasymASSUMES}\ A\ {\isasymor}\ B\ {\isasymOBTAINS}\ A\ {\isasymBBAR}\ B{\isachardoublequote}} \\[1ex]

  \isa{{\isachardoublequote}conjI{\isacharcolon}\ {\isasymASSUMES}\ A\ {\isasymAND}\ B\ {\isasymSHOWS}\ A\ {\isasymand}\ B{\isachardoublequote}} \\
  \isa{{\isachardoublequote}conjE{\isacharcolon}\ {\isasymASSUMES}\ A\ {\isasymand}\ B\ {\isasymOBTAINS}\ A\ {\isasymAND}\ B{\isachardoublequote}} \\[1ex]

  \isa{{\isachardoublequote}falseE{\isacharcolon}\ {\isasymASSUMES}\ {\isasymbottom}\ {\isasymSHOWS}\ A{\isachardoublequote}} \\
  \isa{{\isachardoublequote}trueI{\isacharcolon}\ {\isasymSHOWS}\ {\isasymtop}{\isachardoublequote}} \\[1ex]

  \isa{{\isachardoublequote}notI{\isacharcolon}\ {\isasymASSUMES}\ A\ {\isasymLongrightarrow}\ {\isasymbottom}\ {\isasymSHOWS}\ {\isasymnot}\ A{\isachardoublequote}} \\
  \isa{{\isachardoublequote}notE{\isacharcolon}\ {\isasymASSUMES}\ {\isasymnot}\ A\ {\isasymAND}\ A\ {\isasymSHOWS}\ B{\isachardoublequote}} \\[1ex]

  \isa{{\isachardoublequote}allI{\isacharcolon}\ {\isasymASSUMES}\ {\isasymAnd}x{\isachardot}\ B\ x\ {\isasymSHOWS}\ {\isasymforall}x{\isachardot}\ B\ x{\isachardoublequote}} \\
  \isa{{\isachardoublequote}allE{\isacharcolon}\ {\isasymASSUMES}\ {\isasymforall}x{\isachardot}\ B\ x\ {\isasymSHOWS}\ B\ a{\isachardoublequote}} \\[1ex]

  \isa{{\isachardoublequote}exI{\isacharcolon}\ {\isasymASSUMES}\ B\ a\ {\isasymSHOWS}\ {\isasymexists}x{\isachardot}\ B\ x{\isachardoublequote}} \\
  \isa{{\isachardoublequote}exE{\isacharcolon}\ {\isasymASSUMES}\ {\isasymexists}x{\isachardot}\ B\ x\ {\isasymOBTAINS}\ a\ {\isasymWHERE}\ B\ a{\isachardoublequote}}
  \end{tabular}
  \medskip

  \noindent This essentially provides a declarative reading of Pure
  rules as Isar reasoning patterns: the rule statements tells how a
  canonical proof outline shall look like.  Since the above rules have
  already been declared as \hyperlink{attribute.Pure.intro}{\mbox{\isa{intro}}}, \hyperlink{attribute.Pure.elim}{\mbox{\isa{elim}}}, \hyperlink{attribute.Pure.dest}{\mbox{\isa{dest}}} --- each according to its
  particular shape --- we can immediately write Isar proof texts as
  follows:%
\end{isamarkuptext}%
\isamarkuptrue%
%
\isadelimproof
%
\endisadelimproof
%
\isatagproof
%
\begin{minipage}[t]{0.4\textwidth}
\isanewline
\ \ \isacommand{have}\isamarkupfalse%
\ {\isachardoublequoteopen}A\ {\isasymlongrightarrow}\ B{\isachardoublequoteclose}\isanewline
\ \ \isacommand{proof}\isamarkupfalse%
\isanewline
\ \ \ \ \isacommand{assume}\isamarkupfalse%
\ A\isanewline
\ \ \ \ \isacommand{show}\isamarkupfalse%
\ B%
\endisatagproof
{\isafoldproof}%
%
\isadelimproof
%
\endisadelimproof
%
\isadelimnoproof
\ %
\endisadelimnoproof
%
\isatagnoproof
\isacommand{sorry}\isamarkupfalse%
%
\endisatagnoproof
{\isafoldnoproof}%
%
\isadelimnoproof
\isanewline
%
\endisadelimnoproof
%
\isadelimproof
\ \ %
\endisadelimproof
%
\isatagproof
\isacommand{qed}\isamarkupfalse%
%
\end{minipage}\qquad\begin{minipage}[t]{0.4\textwidth}
\isanewline
\ \ \isacommand{have}\isamarkupfalse%
\ {\isachardoublequoteopen}A\ {\isasymlongrightarrow}\ B{\isachardoublequoteclose}\ \isakeyword{and}\ A%
\endisatagproof
{\isafoldproof}%
%
\isadelimproof
%
\endisadelimproof
%
\isadelimnoproof
\ %
\endisadelimnoproof
%
\isatagnoproof
\isacommand{sorry}\isamarkupfalse%
%
\endisatagnoproof
{\isafoldnoproof}%
%
\isadelimnoproof
\isanewline
%
\endisadelimnoproof
%
\isadelimproof
\ \ %
\endisadelimproof
%
\isatagproof
\isacommand{then}\isamarkupfalse%
\ \isacommand{have}\isamarkupfalse%
\ B\ \isacommand{{\isachardot}{\isachardot}}\isamarkupfalse%
%
\end{minipage}\\[3ex]\begin{minipage}[t]{0.4\textwidth}
\isanewline
\ \ \isacommand{have}\isamarkupfalse%
\ A%
\endisatagproof
{\isafoldproof}%
%
\isadelimproof
%
\endisadelimproof
%
\isadelimnoproof
\ %
\endisadelimnoproof
%
\isatagnoproof
\isacommand{sorry}\isamarkupfalse%
%
\endisatagnoproof
{\isafoldnoproof}%
%
\isadelimnoproof
\isanewline
%
\endisadelimnoproof
%
\isadelimproof
\ \ %
\endisadelimproof
%
\isatagproof
\isacommand{then}\isamarkupfalse%
\ \isacommand{have}\isamarkupfalse%
\ {\isachardoublequoteopen}A\ {\isasymor}\ B{\isachardoublequoteclose}\ \isacommand{{\isachardot}{\isachardot}}\isamarkupfalse%
\isanewline
\isanewline
\ \ \isacommand{have}\isamarkupfalse%
\ B%
\endisatagproof
{\isafoldproof}%
%
\isadelimproof
%
\endisadelimproof
%
\isadelimnoproof
\ %
\endisadelimnoproof
%
\isatagnoproof
\isacommand{sorry}\isamarkupfalse%
%
\endisatagnoproof
{\isafoldnoproof}%
%
\isadelimnoproof
\isanewline
%
\endisadelimnoproof
%
\isadelimproof
\ \ %
\endisadelimproof
%
\isatagproof
\isacommand{then}\isamarkupfalse%
\ \isacommand{have}\isamarkupfalse%
\ {\isachardoublequoteopen}A\ {\isasymor}\ B{\isachardoublequoteclose}\ \isacommand{{\isachardot}{\isachardot}}\isamarkupfalse%
%
\end{minipage}\qquad\begin{minipage}[t]{0.4\textwidth}
\isanewline
\ \ \isacommand{have}\isamarkupfalse%
\ {\isachardoublequoteopen}A\ {\isasymor}\ B{\isachardoublequoteclose}%
\endisatagproof
{\isafoldproof}%
%
\isadelimproof
%
\endisadelimproof
%
\isadelimnoproof
\ %
\endisadelimnoproof
%
\isatagnoproof
\isacommand{sorry}\isamarkupfalse%
%
\endisatagnoproof
{\isafoldnoproof}%
%
\isadelimnoproof
\isanewline
%
\endisadelimnoproof
%
\isadelimproof
\ \ %
\endisadelimproof
%
\isatagproof
\isacommand{then}\isamarkupfalse%
\ \isacommand{have}\isamarkupfalse%
\ C\isanewline
\ \ \isacommand{proof}\isamarkupfalse%
\isanewline
\ \ \ \ \isacommand{assume}\isamarkupfalse%
\ A\isanewline
\ \ \ \ \isacommand{then}\isamarkupfalse%
\ \isacommand{show}\isamarkupfalse%
\ C%
\endisatagproof
{\isafoldproof}%
%
\isadelimproof
%
\endisadelimproof
%
\isadelimnoproof
\ %
\endisadelimnoproof
%
\isatagnoproof
\isacommand{sorry}\isamarkupfalse%
%
\endisatagnoproof
{\isafoldnoproof}%
%
\isadelimnoproof
\isanewline
%
\endisadelimnoproof
%
\isadelimproof
\ \ %
\endisadelimproof
%
\isatagproof
\isacommand{next}\isamarkupfalse%
\isanewline
\ \ \ \ \isacommand{assume}\isamarkupfalse%
\ B\isanewline
\ \ \ \ \isacommand{then}\isamarkupfalse%
\ \isacommand{show}\isamarkupfalse%
\ C%
\endisatagproof
{\isafoldproof}%
%
\isadelimproof
%
\endisadelimproof
%
\isadelimnoproof
\ %
\endisadelimnoproof
%
\isatagnoproof
\isacommand{sorry}\isamarkupfalse%
%
\endisatagnoproof
{\isafoldnoproof}%
%
\isadelimnoproof
\isanewline
%
\endisadelimnoproof
%
\isadelimproof
\ \ %
\endisadelimproof
%
\isatagproof
\isacommand{qed}\isamarkupfalse%
%
\end{minipage}\\[3ex]\begin{minipage}[t]{0.4\textwidth}
\isanewline
\ \ \isacommand{have}\isamarkupfalse%
\ A\ \isakeyword{and}\ B%
\endisatagproof
{\isafoldproof}%
%
\isadelimproof
%
\endisadelimproof
%
\isadelimnoproof
\ %
\endisadelimnoproof
%
\isatagnoproof
\isacommand{sorry}\isamarkupfalse%
%
\endisatagnoproof
{\isafoldnoproof}%
%
\isadelimnoproof
\isanewline
%
\endisadelimnoproof
%
\isadelimproof
\ \ %
\endisadelimproof
%
\isatagproof
\isacommand{then}\isamarkupfalse%
\ \isacommand{have}\isamarkupfalse%
\ {\isachardoublequoteopen}A\ {\isasymand}\ B{\isachardoublequoteclose}\ \isacommand{{\isachardot}{\isachardot}}\isamarkupfalse%
%
\end{minipage}\qquad\begin{minipage}[t]{0.4\textwidth}
\isanewline
\ \ \isacommand{have}\isamarkupfalse%
\ {\isachardoublequoteopen}A\ {\isasymand}\ B{\isachardoublequoteclose}%
\endisatagproof
{\isafoldproof}%
%
\isadelimproof
%
\endisadelimproof
%
\isadelimnoproof
\ %
\endisadelimnoproof
%
\isatagnoproof
\isacommand{sorry}\isamarkupfalse%
%
\endisatagnoproof
{\isafoldnoproof}%
%
\isadelimnoproof
\isanewline
%
\endisadelimnoproof
%
\isadelimproof
\ \ %
\endisadelimproof
%
\isatagproof
\isacommand{then}\isamarkupfalse%
\ \isacommand{obtain}\isamarkupfalse%
\ A\ \isakeyword{and}\ B\ \isacommand{{\isachardot}{\isachardot}}\isamarkupfalse%
%
\end{minipage}\\[3ex]\begin{minipage}[t]{0.4\textwidth}
\isanewline
\ \ \isacommand{have}\isamarkupfalse%
\ {\isachardoublequoteopen}{\isasymbottom}{\isachardoublequoteclose}%
\endisatagproof
{\isafoldproof}%
%
\isadelimproof
%
\endisadelimproof
%
\isadelimnoproof
\ %
\endisadelimnoproof
%
\isatagnoproof
\isacommand{sorry}\isamarkupfalse%
%
\endisatagnoproof
{\isafoldnoproof}%
%
\isadelimnoproof
\isanewline
%
\endisadelimnoproof
%
\isadelimproof
\ \ %
\endisadelimproof
%
\isatagproof
\isacommand{then}\isamarkupfalse%
\ \isacommand{have}\isamarkupfalse%
\ A\ \isacommand{{\isachardot}{\isachardot}}\isamarkupfalse%
%
\end{minipage}\qquad\begin{minipage}[t]{0.4\textwidth}
\isanewline
\ \ \isacommand{have}\isamarkupfalse%
\ {\isachardoublequoteopen}{\isasymtop}{\isachardoublequoteclose}\ \isacommand{{\isachardot}{\isachardot}}\isamarkupfalse%
%
\end{minipage}\\[3ex]\begin{minipage}[t]{0.4\textwidth}
\isanewline
\ \ \isacommand{have}\isamarkupfalse%
\ {\isachardoublequoteopen}{\isasymnot}\ A{\isachardoublequoteclose}\isanewline
\ \ \isacommand{proof}\isamarkupfalse%
\isanewline
\ \ \ \ \isacommand{assume}\isamarkupfalse%
\ A\isanewline
\ \ \ \ \isacommand{then}\isamarkupfalse%
\ \isacommand{show}\isamarkupfalse%
\ {\isachardoublequoteopen}{\isasymbottom}{\isachardoublequoteclose}%
\endisatagproof
{\isafoldproof}%
%
\isadelimproof
%
\endisadelimproof
%
\isadelimnoproof
\ %
\endisadelimnoproof
%
\isatagnoproof
\isacommand{sorry}\isamarkupfalse%
%
\endisatagnoproof
{\isafoldnoproof}%
%
\isadelimnoproof
\isanewline
%
\endisadelimnoproof
%
\isadelimproof
\ \ %
\endisadelimproof
%
\isatagproof
\isacommand{qed}\isamarkupfalse%
%
\end{minipage}\qquad\begin{minipage}[t]{0.4\textwidth}
\isanewline
\ \ \isacommand{have}\isamarkupfalse%
\ {\isachardoublequoteopen}{\isasymnot}\ A{\isachardoublequoteclose}\ \isakeyword{and}\ A%
\endisatagproof
{\isafoldproof}%
%
\isadelimproof
%
\endisadelimproof
%
\isadelimnoproof
\ %
\endisadelimnoproof
%
\isatagnoproof
\isacommand{sorry}\isamarkupfalse%
%
\endisatagnoproof
{\isafoldnoproof}%
%
\isadelimnoproof
\isanewline
%
\endisadelimnoproof
%
\isadelimproof
\ \ %
\endisadelimproof
%
\isatagproof
\isacommand{then}\isamarkupfalse%
\ \isacommand{have}\isamarkupfalse%
\ B\ \isacommand{{\isachardot}{\isachardot}}\isamarkupfalse%
%
\end{minipage}\\[3ex]\begin{minipage}[t]{0.4\textwidth}
\isanewline
\ \ \isacommand{have}\isamarkupfalse%
\ {\isachardoublequoteopen}{\isasymforall}x{\isachardot}\ B\ x{\isachardoublequoteclose}\isanewline
\ \ \isacommand{proof}\isamarkupfalse%
\isanewline
\ \ \ \ \isacommand{fix}\isamarkupfalse%
\ x\isanewline
\ \ \ \ \isacommand{show}\isamarkupfalse%
\ {\isachardoublequoteopen}B\ x{\isachardoublequoteclose}%
\endisatagproof
{\isafoldproof}%
%
\isadelimproof
%
\endisadelimproof
%
\isadelimnoproof
\ %
\endisadelimnoproof
%
\isatagnoproof
\isacommand{sorry}\isamarkupfalse%
%
\endisatagnoproof
{\isafoldnoproof}%
%
\isadelimnoproof
\isanewline
%
\endisadelimnoproof
%
\isadelimproof
\ \ %
\endisadelimproof
%
\isatagproof
\isacommand{qed}\isamarkupfalse%
%
\end{minipage}\qquad\begin{minipage}[t]{0.4\textwidth}
\isanewline
\ \ \isacommand{have}\isamarkupfalse%
\ {\isachardoublequoteopen}{\isasymforall}x{\isachardot}\ B\ x{\isachardoublequoteclose}%
\endisatagproof
{\isafoldproof}%
%
\isadelimproof
%
\endisadelimproof
%
\isadelimnoproof
\ %
\endisadelimnoproof
%
\isatagnoproof
\isacommand{sorry}\isamarkupfalse%
%
\endisatagnoproof
{\isafoldnoproof}%
%
\isadelimnoproof
\isanewline
%
\endisadelimnoproof
%
\isadelimproof
\ \ %
\endisadelimproof
%
\isatagproof
\isacommand{then}\isamarkupfalse%
\ \isacommand{have}\isamarkupfalse%
\ {\isachardoublequoteopen}B\ a{\isachardoublequoteclose}\ \isacommand{{\isachardot}{\isachardot}}\isamarkupfalse%
%
\end{minipage}\\[3ex]\begin{minipage}[t]{0.4\textwidth}
\isanewline
\ \ \isacommand{have}\isamarkupfalse%
\ {\isachardoublequoteopen}{\isasymexists}x{\isachardot}\ B\ x{\isachardoublequoteclose}\isanewline
\ \ \isacommand{proof}\isamarkupfalse%
\isanewline
\ \ \ \ \isacommand{show}\isamarkupfalse%
\ {\isachardoublequoteopen}B\ a{\isachardoublequoteclose}%
\endisatagproof
{\isafoldproof}%
%
\isadelimproof
%
\endisadelimproof
%
\isadelimnoproof
\ %
\endisadelimnoproof
%
\isatagnoproof
\isacommand{sorry}\isamarkupfalse%
%
\endisatagnoproof
{\isafoldnoproof}%
%
\isadelimnoproof
\isanewline
%
\endisadelimnoproof
%
\isadelimproof
\ \ %
\endisadelimproof
%
\isatagproof
\isacommand{qed}\isamarkupfalse%
%
\end{minipage}\qquad\begin{minipage}[t]{0.4\textwidth}
\isanewline
\ \ \isacommand{have}\isamarkupfalse%
\ {\isachardoublequoteopen}{\isasymexists}x{\isachardot}\ B\ x{\isachardoublequoteclose}%
\endisatagproof
{\isafoldproof}%
%
\isadelimproof
%
\endisadelimproof
%
\isadelimnoproof
\ %
\endisadelimnoproof
%
\isatagnoproof
\isacommand{sorry}\isamarkupfalse%
%
\endisatagnoproof
{\isafoldnoproof}%
%
\isadelimnoproof
\isanewline
%
\endisadelimnoproof
%
\isadelimproof
\ \ %
\endisadelimproof
%
\isatagproof
\isacommand{then}\isamarkupfalse%
\ \isacommand{obtain}\isamarkupfalse%
\ a\ \isakeyword{where}\ {\isachardoublequoteopen}B\ a{\isachardoublequoteclose}\ \isacommand{{\isachardot}{\isachardot}}\isamarkupfalse%
%
\end{minipage}
%
\endisatagproof
{\isafoldproof}%
%
\isadelimproof
%
\endisadelimproof
%
\begin{isamarkuptext}%
\bigskip\noindent Of course, these proofs are merely examples.  As
  sketched in \secref{sec:framework-subproof}, there is a fair amount
  of flexibility in expressing Pure deductions in Isar.  Here the user
  is asked to express himself adequately, aiming at proof texts of
  literary quality.%
\end{isamarkuptext}%
\isamarkuptrue%
%
\isadelimvisible
%
\endisadelimvisible
%
\isatagvisible
\isacommand{end}\isamarkupfalse%
%
\endisatagvisible
{\isafoldvisible}%
%
\isadelimvisible
%
\endisadelimvisible
\isanewline
\end{isabellebody}%
%%% Local Variables:
%%% mode: latex
%%% TeX-master: "root"
%%% End:
