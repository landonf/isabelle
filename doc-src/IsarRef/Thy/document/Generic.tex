%
\begin{isabellebody}%
\def\isabellecontext{Generic}%
%
\isadelimtheory
%
\endisadelimtheory
%
\isatagtheory
\isacommand{theory}\isamarkupfalse%
\ Generic\isanewline
\isakeyword{imports}\ Main\isanewline
\isakeyword{begin}%
\endisatagtheory
{\isafoldtheory}%
%
\isadelimtheory
%
\endisadelimtheory
%
\isamarkupchapter{Generic tools and packages \label{ch:gen-tools}%
}
\isamarkuptrue%
%
\isamarkupsection{Configuration options%
}
\isamarkuptrue%
%
\begin{isamarkuptext}%
Isabelle/Pure maintains a record of named configuration
  options within the theory or proof context, with values of type
  \verb|bool|, \verb|int|, \verb|real|, or \verb|string|.  Tools may declare options in ML, and then refer to these
  values (relative to the context).  Thus global reference variables
  are easily avoided.  The user may change the value of a
  configuration option by means of an associated attribute of the same
  name.  This form of context declaration works particularly well with
  commands such as \hyperlink{command.declare}{\mbox{\isa{\isacommand{declare}}}} or \hyperlink{command.using}{\mbox{\isa{\isacommand{using}}}}.

  For historical reasons, some tools cannot take the full proof
  context into account and merely refer to the background theory.
  This is accommodated by configuration options being declared as
  ``global'', which may not be changed within a local context.

  \begin{matharray}{rcll}
    \indexdef{}{command}{print\_configs}\hypertarget{command.print-configs}{\hyperlink{command.print-configs}{\mbox{\isa{\isacommand{print{\isaliteral{5F}{\isacharunderscore}}configs}}}}} & : & \isa{{\isaliteral{22}{\isachardoublequote}}context\ {\isaliteral{5C3C72696768746172726F773E}{\isasymrightarrow}}{\isaliteral{22}{\isachardoublequote}}} \\
  \end{matharray}

  \begin{railoutput}
\rail@begin{6}{\isa{}}
\rail@nont{\hyperlink{syntax.name}{\mbox{\isa{name}}}}[]
\rail@bar
\rail@nextbar{1}
\rail@term{\isa{{\isaliteral{3D}{\isacharequal}}}}[]
\rail@bar
\rail@term{\isa{true}}[]
\rail@nextbar{2}
\rail@term{\isa{false}}[]
\rail@nextbar{3}
\rail@nont{\hyperlink{syntax.int}{\mbox{\isa{int}}}}[]
\rail@nextbar{4}
\rail@nont{\hyperlink{syntax.float}{\mbox{\isa{float}}}}[]
\rail@nextbar{5}
\rail@nont{\hyperlink{syntax.name}{\mbox{\isa{name}}}}[]
\rail@endbar
\rail@endbar
\rail@end
\end{railoutput}


  \begin{description}
  
  \item \hyperlink{command.print-configs}{\mbox{\isa{\isacommand{print{\isaliteral{5F}{\isacharunderscore}}configs}}}} prints the available configuration
  options, with names, types, and current values.
  
  \item \isa{{\isaliteral{22}{\isachardoublequote}}name\ {\isaliteral{3D}{\isacharequal}}\ value{\isaliteral{22}{\isachardoublequote}}} as an attribute expression modifies the
  named option, with the syntax of the value depending on the option's
  type.  For \verb|bool| the default value is \isa{true}.  Any
  attempt to change a global option in a local context is ignored.

  \end{description}%
\end{isamarkuptext}%
\isamarkuptrue%
%
\isamarkupsection{Basic proof tools%
}
\isamarkuptrue%
%
\isamarkupsubsection{Miscellaneous methods and attributes \label{sec:misc-meth-att}%
}
\isamarkuptrue%
%
\begin{isamarkuptext}%
\begin{matharray}{rcl}
    \indexdef{}{method}{unfold}\hypertarget{method.unfold}{\hyperlink{method.unfold}{\mbox{\isa{unfold}}}} & : & \isa{method} \\
    \indexdef{}{method}{fold}\hypertarget{method.fold}{\hyperlink{method.fold}{\mbox{\isa{fold}}}} & : & \isa{method} \\
    \indexdef{}{method}{insert}\hypertarget{method.insert}{\hyperlink{method.insert}{\mbox{\isa{insert}}}} & : & \isa{method} \\[0.5ex]
    \indexdef{}{method}{erule}\hypertarget{method.erule}{\hyperlink{method.erule}{\mbox{\isa{erule}}}}\isa{{\isaliteral{22}{\isachardoublequote}}\isaliteral{5C3C5E7375703E}{}\isactrlsup {\isaliteral{2A}{\isacharasterisk}}{\isaliteral{22}{\isachardoublequote}}} & : & \isa{method} \\
    \indexdef{}{method}{drule}\hypertarget{method.drule}{\hyperlink{method.drule}{\mbox{\isa{drule}}}}\isa{{\isaliteral{22}{\isachardoublequote}}\isaliteral{5C3C5E7375703E}{}\isactrlsup {\isaliteral{2A}{\isacharasterisk}}{\isaliteral{22}{\isachardoublequote}}} & : & \isa{method} \\
    \indexdef{}{method}{frule}\hypertarget{method.frule}{\hyperlink{method.frule}{\mbox{\isa{frule}}}}\isa{{\isaliteral{22}{\isachardoublequote}}\isaliteral{5C3C5E7375703E}{}\isactrlsup {\isaliteral{2A}{\isacharasterisk}}{\isaliteral{22}{\isachardoublequote}}} & : & \isa{method} \\
    \indexdef{}{method}{succeed}\hypertarget{method.succeed}{\hyperlink{method.succeed}{\mbox{\isa{succeed}}}} & : & \isa{method} \\
    \indexdef{}{method}{fail}\hypertarget{method.fail}{\hyperlink{method.fail}{\mbox{\isa{fail}}}} & : & \isa{method} \\
  \end{matharray}

  \begin{railoutput}
\rail@begin{3}{\isa{}}
\rail@bar
\rail@term{\hyperlink{method.fold}{\mbox{\isa{fold}}}}[]
\rail@nextbar{1}
\rail@term{\hyperlink{method.unfold}{\mbox{\isa{unfold}}}}[]
\rail@nextbar{2}
\rail@term{\hyperlink{method.insert}{\mbox{\isa{insert}}}}[]
\rail@endbar
\rail@nont{\hyperlink{syntax.thmrefs}{\mbox{\isa{thmrefs}}}}[]
\rail@end
\rail@begin{3}{\isa{}}
\rail@bar
\rail@term{\hyperlink{method.erule}{\mbox{\isa{erule}}}}[]
\rail@nextbar{1}
\rail@term{\hyperlink{method.drule}{\mbox{\isa{drule}}}}[]
\rail@nextbar{2}
\rail@term{\hyperlink{method.frule}{\mbox{\isa{frule}}}}[]
\rail@endbar
\rail@bar
\rail@nextbar{1}
\rail@term{\isa{{\isaliteral{28}{\isacharparenleft}}}}[]
\rail@nont{\hyperlink{syntax.nat}{\mbox{\isa{nat}}}}[]
\rail@term{\isa{{\isaliteral{29}{\isacharparenright}}}}[]
\rail@endbar
\rail@nont{\hyperlink{syntax.thmrefs}{\mbox{\isa{thmrefs}}}}[]
\rail@end
\end{railoutput}


  \begin{description}
  
  \item \hyperlink{method.unfold}{\mbox{\isa{unfold}}}~\isa{{\isaliteral{22}{\isachardoublequote}}a\isaliteral{5C3C5E7375623E}{}\isactrlsub {\isadigit{1}}\ {\isaliteral{5C3C646F74733E}{\isasymdots}}\ a\isaliteral{5C3C5E7375623E}{}\isactrlsub n{\isaliteral{22}{\isachardoublequote}}} and \hyperlink{method.fold}{\mbox{\isa{fold}}}~\isa{{\isaliteral{22}{\isachardoublequote}}a\isaliteral{5C3C5E7375623E}{}\isactrlsub {\isadigit{1}}\ {\isaliteral{5C3C646F74733E}{\isasymdots}}\ a\isaliteral{5C3C5E7375623E}{}\isactrlsub n{\isaliteral{22}{\isachardoublequote}}} expand (or fold back) the given definitions throughout
  all goals; any chained facts provided are inserted into the goal and
  subject to rewriting as well.

  \item \hyperlink{method.insert}{\mbox{\isa{insert}}}~\isa{{\isaliteral{22}{\isachardoublequote}}a\isaliteral{5C3C5E7375623E}{}\isactrlsub {\isadigit{1}}\ {\isaliteral{5C3C646F74733E}{\isasymdots}}\ a\isaliteral{5C3C5E7375623E}{}\isactrlsub n{\isaliteral{22}{\isachardoublequote}}} inserts theorems as facts
  into all goals of the proof state.  Note that current facts
  indicated for forward chaining are ignored.

  \item \hyperlink{method.erule}{\mbox{\isa{erule}}}~\isa{{\isaliteral{22}{\isachardoublequote}}a\isaliteral{5C3C5E7375623E}{}\isactrlsub {\isadigit{1}}\ {\isaliteral{5C3C646F74733E}{\isasymdots}}\ a\isaliteral{5C3C5E7375623E}{}\isactrlsub n{\isaliteral{22}{\isachardoublequote}}}, \hyperlink{method.drule}{\mbox{\isa{drule}}}~\isa{{\isaliteral{22}{\isachardoublequote}}a\isaliteral{5C3C5E7375623E}{}\isactrlsub {\isadigit{1}}\ {\isaliteral{5C3C646F74733E}{\isasymdots}}\ a\isaliteral{5C3C5E7375623E}{}\isactrlsub n{\isaliteral{22}{\isachardoublequote}}}, and \hyperlink{method.frule}{\mbox{\isa{frule}}}~\isa{{\isaliteral{22}{\isachardoublequote}}a\isaliteral{5C3C5E7375623E}{}\isactrlsub {\isadigit{1}}\ {\isaliteral{5C3C646F74733E}{\isasymdots}}\ a\isaliteral{5C3C5E7375623E}{}\isactrlsub n{\isaliteral{22}{\isachardoublequote}}} are similar to the basic \hyperlink{method.rule}{\mbox{\isa{rule}}}
  method (see \secref{sec:pure-meth-att}), but apply rules by
  elim-resolution, destruct-resolution, and forward-resolution,
  respectively \cite{isabelle-implementation}.  The optional natural
  number argument (default 0) specifies additional assumption steps to
  be performed here.

  Note that these methods are improper ones, mainly serving for
  experimentation and tactic script emulation.  Different modes of
  basic rule application are usually expressed in Isar at the proof
  language level, rather than via implicit proof state manipulations.
  For example, a proper single-step elimination would be done using
  the plain \hyperlink{method.rule}{\mbox{\isa{rule}}} method, with forward chaining of current
  facts.

  \item \hyperlink{method.succeed}{\mbox{\isa{succeed}}} yields a single (unchanged) result; it is
  the identity of the ``\isa{{\isaliteral{22}{\isachardoublequote}}{\isaliteral{2C}{\isacharcomma}}{\isaliteral{22}{\isachardoublequote}}}'' method combinator (cf.\
  \secref{sec:proof-meth}).

  \item \hyperlink{method.fail}{\mbox{\isa{fail}}} yields an empty result sequence; it is the
  identity of the ``\isa{{\isaliteral{22}{\isachardoublequote}}{\isaliteral{7C}{\isacharbar}}{\isaliteral{22}{\isachardoublequote}}}'' method combinator (cf.\
  \secref{sec:proof-meth}).

  \end{description}

  \begin{matharray}{rcl}
    \indexdef{}{attribute}{tagged}\hypertarget{attribute.tagged}{\hyperlink{attribute.tagged}{\mbox{\isa{tagged}}}} & : & \isa{attribute} \\
    \indexdef{}{attribute}{untagged}\hypertarget{attribute.untagged}{\hyperlink{attribute.untagged}{\mbox{\isa{untagged}}}} & : & \isa{attribute} \\[0.5ex]
    \indexdef{}{attribute}{THEN}\hypertarget{attribute.THEN}{\hyperlink{attribute.THEN}{\mbox{\isa{THEN}}}} & : & \isa{attribute} \\
    \indexdef{}{attribute}{COMP}\hypertarget{attribute.COMP}{\hyperlink{attribute.COMP}{\mbox{\isa{COMP}}}} & : & \isa{attribute} \\[0.5ex]
    \indexdef{}{attribute}{unfolded}\hypertarget{attribute.unfolded}{\hyperlink{attribute.unfolded}{\mbox{\isa{unfolded}}}} & : & \isa{attribute} \\
    \indexdef{}{attribute}{folded}\hypertarget{attribute.folded}{\hyperlink{attribute.folded}{\mbox{\isa{folded}}}} & : & \isa{attribute} \\[0.5ex]
    \indexdef{}{attribute}{rotated}\hypertarget{attribute.rotated}{\hyperlink{attribute.rotated}{\mbox{\isa{rotated}}}} & : & \isa{attribute} \\
    \indexdef{Pure}{attribute}{elim\_format}\hypertarget{attribute.Pure.elim-format}{\hyperlink{attribute.Pure.elim-format}{\mbox{\isa{elim{\isaliteral{5F}{\isacharunderscore}}format}}}} & : & \isa{attribute} \\
    \indexdef{}{attribute}{standard}\hypertarget{attribute.standard}{\hyperlink{attribute.standard}{\mbox{\isa{standard}}}}\isa{{\isaliteral{22}{\isachardoublequote}}\isaliteral{5C3C5E7375703E}{}\isactrlsup {\isaliteral{2A}{\isacharasterisk}}{\isaliteral{22}{\isachardoublequote}}} & : & \isa{attribute} \\
    \indexdef{}{attribute}{no\_vars}\hypertarget{attribute.no-vars}{\hyperlink{attribute.no-vars}{\mbox{\isa{no{\isaliteral{5F}{\isacharunderscore}}vars}}}}\isa{{\isaliteral{22}{\isachardoublequote}}\isaliteral{5C3C5E7375703E}{}\isactrlsup {\isaliteral{2A}{\isacharasterisk}}{\isaliteral{22}{\isachardoublequote}}} & : & \isa{attribute} \\
  \end{matharray}

  \begin{railoutput}
\rail@begin{1}{\isa{}}
\rail@term{\hyperlink{attribute.tagged}{\mbox{\isa{tagged}}}}[]
\rail@nont{\hyperlink{syntax.name}{\mbox{\isa{name}}}}[]
\rail@nont{\hyperlink{syntax.name}{\mbox{\isa{name}}}}[]
\rail@end
\rail@begin{1}{\isa{}}
\rail@term{\hyperlink{attribute.untagged}{\mbox{\isa{untagged}}}}[]
\rail@nont{\hyperlink{syntax.name}{\mbox{\isa{name}}}}[]
\rail@end
\rail@begin{2}{\isa{}}
\rail@bar
\rail@term{\hyperlink{attribute.THEN}{\mbox{\isa{THEN}}}}[]
\rail@nextbar{1}
\rail@term{\hyperlink{attribute.COMP}{\mbox{\isa{COMP}}}}[]
\rail@endbar
\rail@bar
\rail@nextbar{1}
\rail@term{\isa{{\isaliteral{5B}{\isacharbrackleft}}}}[]
\rail@nont{\hyperlink{syntax.nat}{\mbox{\isa{nat}}}}[]
\rail@term{\isa{{\isaliteral{5D}{\isacharbrackright}}}}[]
\rail@endbar
\rail@nont{\hyperlink{syntax.thmref}{\mbox{\isa{thmref}}}}[]
\rail@end
\rail@begin{2}{\isa{}}
\rail@bar
\rail@term{\hyperlink{attribute.unfolded}{\mbox{\isa{unfolded}}}}[]
\rail@nextbar{1}
\rail@term{\hyperlink{attribute.folded}{\mbox{\isa{folded}}}}[]
\rail@endbar
\rail@nont{\hyperlink{syntax.thmrefs}{\mbox{\isa{thmrefs}}}}[]
\rail@end
\rail@begin{2}{\isa{}}
\rail@term{\hyperlink{attribute.rotated}{\mbox{\isa{rotated}}}}[]
\rail@bar
\rail@nextbar{1}
\rail@nont{\hyperlink{syntax.int}{\mbox{\isa{int}}}}[]
\rail@endbar
\rail@end
\end{railoutput}


  \begin{description}

  \item \hyperlink{attribute.tagged}{\mbox{\isa{tagged}}}~\isa{{\isaliteral{22}{\isachardoublequote}}name\ value{\isaliteral{22}{\isachardoublequote}}} and \hyperlink{attribute.untagged}{\mbox{\isa{untagged}}}~\isa{name} add and remove \emph{tags} of some theorem.
  Tags may be any list of string pairs that serve as formal comment.
  The first string is considered the tag name, the second its value.
  Note that \hyperlink{attribute.untagged}{\mbox{\isa{untagged}}} removes any tags of the same name.

  \item \hyperlink{attribute.THEN}{\mbox{\isa{THEN}}}~\isa{a} and \hyperlink{attribute.COMP}{\mbox{\isa{COMP}}}~\isa{a}
  compose rules by resolution.  \hyperlink{attribute.THEN}{\mbox{\isa{THEN}}} resolves with the
  first premise of \isa{a} (an alternative position may be also
  specified); the \hyperlink{attribute.COMP}{\mbox{\isa{COMP}}} version skips the automatic
  lifting process that is normally intended (cf.\ \verb|op RS| and
  \verb|op COMP| in \cite{isabelle-implementation}).
  
  \item \hyperlink{attribute.unfolded}{\mbox{\isa{unfolded}}}~\isa{{\isaliteral{22}{\isachardoublequote}}a\isaliteral{5C3C5E7375623E}{}\isactrlsub {\isadigit{1}}\ {\isaliteral{5C3C646F74733E}{\isasymdots}}\ a\isaliteral{5C3C5E7375623E}{}\isactrlsub n{\isaliteral{22}{\isachardoublequote}}} and \hyperlink{attribute.folded}{\mbox{\isa{folded}}}~\isa{{\isaliteral{22}{\isachardoublequote}}a\isaliteral{5C3C5E7375623E}{}\isactrlsub {\isadigit{1}}\ {\isaliteral{5C3C646F74733E}{\isasymdots}}\ a\isaliteral{5C3C5E7375623E}{}\isactrlsub n{\isaliteral{22}{\isachardoublequote}}} expand and fold back again the given
  definitions throughout a rule.

  \item \hyperlink{attribute.rotated}{\mbox{\isa{rotated}}}~\isa{n} rotate the premises of a
  theorem by \isa{n} (default 1).

  \item \hyperlink{attribute.Pure.elim-format}{\mbox{\isa{elim{\isaliteral{5F}{\isacharunderscore}}format}}} turns a destruction rule into
  elimination rule format, by resolving with the rule \isa{{\isaliteral{22}{\isachardoublequote}}PROP\ A\ {\isaliteral{5C3C4C6F6E6772696768746172726F773E}{\isasymLongrightarrow}}\ {\isaliteral{28}{\isacharparenleft}}PROP\ A\ {\isaliteral{5C3C4C6F6E6772696768746172726F773E}{\isasymLongrightarrow}}\ PROP\ B{\isaliteral{29}{\isacharparenright}}\ {\isaliteral{5C3C4C6F6E6772696768746172726F773E}{\isasymLongrightarrow}}\ PROP\ B{\isaliteral{22}{\isachardoublequote}}}.
  
  Note that the Classical Reasoner (\secref{sec:classical}) provides
  its own version of this operation.

  \item \hyperlink{attribute.standard}{\mbox{\isa{standard}}} puts a theorem into the standard form of
  object-rules at the outermost theory level.  Note that this
  operation violates the local proof context (including active
  locales).

  \item \hyperlink{attribute.no-vars}{\mbox{\isa{no{\isaliteral{5F}{\isacharunderscore}}vars}}} replaces schematic variables by free
  ones; this is mainly for tuning output of pretty printed theorems.

  \end{description}%
\end{isamarkuptext}%
\isamarkuptrue%
%
\isamarkupsubsection{Low-level equational reasoning%
}
\isamarkuptrue%
%
\begin{isamarkuptext}%
\begin{matharray}{rcl}
    \indexdef{}{method}{subst}\hypertarget{method.subst}{\hyperlink{method.subst}{\mbox{\isa{subst}}}} & : & \isa{method} \\
    \indexdef{}{method}{hypsubst}\hypertarget{method.hypsubst}{\hyperlink{method.hypsubst}{\mbox{\isa{hypsubst}}}} & : & \isa{method} \\
    \indexdef{}{method}{split}\hypertarget{method.split}{\hyperlink{method.split}{\mbox{\isa{split}}}} & : & \isa{method} \\
  \end{matharray}

  \begin{railoutput}
\rail@begin{3}{\isa{}}
\rail@term{\hyperlink{method.subst}{\mbox{\isa{subst}}}}[]
\rail@bar
\rail@nextbar{1}
\rail@term{\isa{{\isaliteral{28}{\isacharparenleft}}}}[]
\rail@term{\isa{asm}}[]
\rail@term{\isa{{\isaliteral{29}{\isacharparenright}}}}[]
\rail@endbar
\rail@bar
\rail@nextbar{1}
\rail@term{\isa{{\isaliteral{28}{\isacharparenleft}}}}[]
\rail@plus
\rail@nont{\hyperlink{syntax.nat}{\mbox{\isa{nat}}}}[]
\rail@nextplus{2}
\rail@endplus
\rail@term{\isa{{\isaliteral{29}{\isacharparenright}}}}[]
\rail@endbar
\rail@nont{\hyperlink{syntax.thmref}{\mbox{\isa{thmref}}}}[]
\rail@end
\rail@begin{2}{\isa{}}
\rail@term{\hyperlink{method.split}{\mbox{\isa{split}}}}[]
\rail@bar
\rail@nextbar{1}
\rail@term{\isa{{\isaliteral{28}{\isacharparenleft}}}}[]
\rail@term{\isa{asm}}[]
\rail@term{\isa{{\isaliteral{29}{\isacharparenright}}}}[]
\rail@endbar
\rail@nont{\hyperlink{syntax.thmrefs}{\mbox{\isa{thmrefs}}}}[]
\rail@end
\end{railoutput}


  These methods provide low-level facilities for equational reasoning
  that are intended for specialized applications only.  Normally,
  single step calculations would be performed in a structured text
  (see also \secref{sec:calculation}), while the Simplifier methods
  provide the canonical way for automated normalization (see
  \secref{sec:simplifier}).

  \begin{description}

  \item \hyperlink{method.subst}{\mbox{\isa{subst}}}~\isa{eq} performs a single substitution step
  using rule \isa{eq}, which may be either a meta or object
  equality.

  \item \hyperlink{method.subst}{\mbox{\isa{subst}}}~\isa{{\isaliteral{22}{\isachardoublequote}}{\isaliteral{28}{\isacharparenleft}}asm{\isaliteral{29}{\isacharparenright}}\ eq{\isaliteral{22}{\isachardoublequote}}} substitutes in an
  assumption.

  \item \hyperlink{method.subst}{\mbox{\isa{subst}}}~\isa{{\isaliteral{22}{\isachardoublequote}}{\isaliteral{28}{\isacharparenleft}}i\ {\isaliteral{5C3C646F74733E}{\isasymdots}}\ j{\isaliteral{29}{\isacharparenright}}\ eq{\isaliteral{22}{\isachardoublequote}}} performs several
  substitutions in the conclusion. The numbers \isa{i} to \isa{j}
  indicate the positions to substitute at.  Positions are ordered from
  the top of the term tree moving down from left to right. For
  example, in \isa{{\isaliteral{22}{\isachardoublequote}}{\isaliteral{28}{\isacharparenleft}}a\ {\isaliteral{2B}{\isacharplus}}\ b{\isaliteral{29}{\isacharparenright}}\ {\isaliteral{2B}{\isacharplus}}\ {\isaliteral{28}{\isacharparenleft}}c\ {\isaliteral{2B}{\isacharplus}}\ d{\isaliteral{29}{\isacharparenright}}{\isaliteral{22}{\isachardoublequote}}} there are three positions
  where commutativity of \isa{{\isaliteral{22}{\isachardoublequote}}{\isaliteral{2B}{\isacharplus}}{\isaliteral{22}{\isachardoublequote}}} is applicable: 1 refers to \isa{{\isaliteral{22}{\isachardoublequote}}a\ {\isaliteral{2B}{\isacharplus}}\ b{\isaliteral{22}{\isachardoublequote}}}, 2 to the whole term, and 3 to \isa{{\isaliteral{22}{\isachardoublequote}}c\ {\isaliteral{2B}{\isacharplus}}\ d{\isaliteral{22}{\isachardoublequote}}}.

  If the positions in the list \isa{{\isaliteral{22}{\isachardoublequote}}{\isaliteral{28}{\isacharparenleft}}i\ {\isaliteral{5C3C646F74733E}{\isasymdots}}\ j{\isaliteral{29}{\isacharparenright}}{\isaliteral{22}{\isachardoublequote}}} are non-overlapping
  (e.g.\ \isa{{\isaliteral{22}{\isachardoublequote}}{\isaliteral{28}{\isacharparenleft}}{\isadigit{2}}\ {\isadigit{3}}{\isaliteral{29}{\isacharparenright}}{\isaliteral{22}{\isachardoublequote}}} in \isa{{\isaliteral{22}{\isachardoublequote}}{\isaliteral{28}{\isacharparenleft}}a\ {\isaliteral{2B}{\isacharplus}}\ b{\isaliteral{29}{\isacharparenright}}\ {\isaliteral{2B}{\isacharplus}}\ {\isaliteral{28}{\isacharparenleft}}c\ {\isaliteral{2B}{\isacharplus}}\ d{\isaliteral{29}{\isacharparenright}}{\isaliteral{22}{\isachardoublequote}}}) you may
  assume all substitutions are performed simultaneously.  Otherwise
  the behaviour of \isa{subst} is not specified.

  \item \hyperlink{method.subst}{\mbox{\isa{subst}}}~\isa{{\isaliteral{22}{\isachardoublequote}}{\isaliteral{28}{\isacharparenleft}}asm{\isaliteral{29}{\isacharparenright}}\ {\isaliteral{28}{\isacharparenleft}}i\ {\isaliteral{5C3C646F74733E}{\isasymdots}}\ j{\isaliteral{29}{\isacharparenright}}\ eq{\isaliteral{22}{\isachardoublequote}}} performs the
  substitutions in the assumptions. The positions refer to the
  assumptions in order from left to right.  For example, given in a
  goal of the form \isa{{\isaliteral{22}{\isachardoublequote}}P\ {\isaliteral{28}{\isacharparenleft}}a\ {\isaliteral{2B}{\isacharplus}}\ b{\isaliteral{29}{\isacharparenright}}\ {\isaliteral{5C3C4C6F6E6772696768746172726F773E}{\isasymLongrightarrow}}\ P\ {\isaliteral{28}{\isacharparenleft}}c\ {\isaliteral{2B}{\isacharplus}}\ d{\isaliteral{29}{\isacharparenright}}\ {\isaliteral{5C3C4C6F6E6772696768746172726F773E}{\isasymLongrightarrow}}\ {\isaliteral{5C3C646F74733E}{\isasymdots}}{\isaliteral{22}{\isachardoublequote}}}, position 1 of
  commutativity of \isa{{\isaliteral{22}{\isachardoublequote}}{\isaliteral{2B}{\isacharplus}}{\isaliteral{22}{\isachardoublequote}}} is the subterm \isa{{\isaliteral{22}{\isachardoublequote}}a\ {\isaliteral{2B}{\isacharplus}}\ b{\isaliteral{22}{\isachardoublequote}}} and
  position 2 is the subterm \isa{{\isaliteral{22}{\isachardoublequote}}c\ {\isaliteral{2B}{\isacharplus}}\ d{\isaliteral{22}{\isachardoublequote}}}.

  \item \hyperlink{method.hypsubst}{\mbox{\isa{hypsubst}}} performs substitution using some
  assumption; this only works for equations of the form \isa{{\isaliteral{22}{\isachardoublequote}}x\ {\isaliteral{3D}{\isacharequal}}\ t{\isaliteral{22}{\isachardoublequote}}} where \isa{x} is a free or bound variable.

  \item \hyperlink{method.split}{\mbox{\isa{split}}}~\isa{{\isaliteral{22}{\isachardoublequote}}a\isaliteral{5C3C5E7375623E}{}\isactrlsub {\isadigit{1}}\ {\isaliteral{5C3C646F74733E}{\isasymdots}}\ a\isaliteral{5C3C5E7375623E}{}\isactrlsub n{\isaliteral{22}{\isachardoublequote}}} performs single-step case
  splitting using the given rules.  By default, splitting is performed
  in the conclusion of a goal; the \isa{{\isaliteral{22}{\isachardoublequote}}{\isaliteral{28}{\isacharparenleft}}asm{\isaliteral{29}{\isacharparenright}}{\isaliteral{22}{\isachardoublequote}}} option indicates to
  operate on assumptions instead.
  
  Note that the \hyperlink{method.simp}{\mbox{\isa{simp}}} method already involves repeated
  application of split rules as declared in the current context.

  \end{description}%
\end{isamarkuptext}%
\isamarkuptrue%
%
\isamarkupsubsection{Further tactic emulations \label{sec:tactics}%
}
\isamarkuptrue%
%
\begin{isamarkuptext}%
The following improper proof methods emulate traditional tactics.
  These admit direct access to the goal state, which is normally
  considered harmful!  In particular, this may involve both numbered
  goal addressing (default 1), and dynamic instantiation within the
  scope of some subgoal.

  \begin{warn}
    Dynamic instantiations refer to universally quantified parameters
    of a subgoal (the dynamic context) rather than fixed variables and
    term abbreviations of a (static) Isar context.
  \end{warn}

  Tactic emulation methods, unlike their ML counterparts, admit
  simultaneous instantiation from both dynamic and static contexts.
  If names occur in both contexts goal parameters hide locally fixed
  variables.  Likewise, schematic variables refer to term
  abbreviations, if present in the static context.  Otherwise the
  schematic variable is interpreted as a schematic variable and left
  to be solved by unification with certain parts of the subgoal.

  Note that the tactic emulation proof methods in Isabelle/Isar are
  consistently named \isa{foo{\isaliteral{5F}{\isacharunderscore}}tac}.  Note also that variable names
  occurring on left hand sides of instantiations must be preceded by a
  question mark if they coincide with a keyword or contain dots.  This
  is consistent with the attribute \hyperlink{attribute.where}{\mbox{\isa{where}}} (see
  \secref{sec:pure-meth-att}).

  \begin{matharray}{rcl}
    \indexdef{}{method}{rule\_tac}\hypertarget{method.rule-tac}{\hyperlink{method.rule-tac}{\mbox{\isa{rule{\isaliteral{5F}{\isacharunderscore}}tac}}}}\isa{{\isaliteral{22}{\isachardoublequote}}\isaliteral{5C3C5E7375703E}{}\isactrlsup {\isaliteral{2A}{\isacharasterisk}}{\isaliteral{22}{\isachardoublequote}}} & : & \isa{method} \\
    \indexdef{}{method}{erule\_tac}\hypertarget{method.erule-tac}{\hyperlink{method.erule-tac}{\mbox{\isa{erule{\isaliteral{5F}{\isacharunderscore}}tac}}}}\isa{{\isaliteral{22}{\isachardoublequote}}\isaliteral{5C3C5E7375703E}{}\isactrlsup {\isaliteral{2A}{\isacharasterisk}}{\isaliteral{22}{\isachardoublequote}}} & : & \isa{method} \\
    \indexdef{}{method}{drule\_tac}\hypertarget{method.drule-tac}{\hyperlink{method.drule-tac}{\mbox{\isa{drule{\isaliteral{5F}{\isacharunderscore}}tac}}}}\isa{{\isaliteral{22}{\isachardoublequote}}\isaliteral{5C3C5E7375703E}{}\isactrlsup {\isaliteral{2A}{\isacharasterisk}}{\isaliteral{22}{\isachardoublequote}}} & : & \isa{method} \\
    \indexdef{}{method}{frule\_tac}\hypertarget{method.frule-tac}{\hyperlink{method.frule-tac}{\mbox{\isa{frule{\isaliteral{5F}{\isacharunderscore}}tac}}}}\isa{{\isaliteral{22}{\isachardoublequote}}\isaliteral{5C3C5E7375703E}{}\isactrlsup {\isaliteral{2A}{\isacharasterisk}}{\isaliteral{22}{\isachardoublequote}}} & : & \isa{method} \\
    \indexdef{}{method}{cut\_tac}\hypertarget{method.cut-tac}{\hyperlink{method.cut-tac}{\mbox{\isa{cut{\isaliteral{5F}{\isacharunderscore}}tac}}}}\isa{{\isaliteral{22}{\isachardoublequote}}\isaliteral{5C3C5E7375703E}{}\isactrlsup {\isaliteral{2A}{\isacharasterisk}}{\isaliteral{22}{\isachardoublequote}}} & : & \isa{method} \\
    \indexdef{}{method}{thin\_tac}\hypertarget{method.thin-tac}{\hyperlink{method.thin-tac}{\mbox{\isa{thin{\isaliteral{5F}{\isacharunderscore}}tac}}}}\isa{{\isaliteral{22}{\isachardoublequote}}\isaliteral{5C3C5E7375703E}{}\isactrlsup {\isaliteral{2A}{\isacharasterisk}}{\isaliteral{22}{\isachardoublequote}}} & : & \isa{method} \\
    \indexdef{}{method}{subgoal\_tac}\hypertarget{method.subgoal-tac}{\hyperlink{method.subgoal-tac}{\mbox{\isa{subgoal{\isaliteral{5F}{\isacharunderscore}}tac}}}}\isa{{\isaliteral{22}{\isachardoublequote}}\isaliteral{5C3C5E7375703E}{}\isactrlsup {\isaliteral{2A}{\isacharasterisk}}{\isaliteral{22}{\isachardoublequote}}} & : & \isa{method} \\
    \indexdef{}{method}{rename\_tac}\hypertarget{method.rename-tac}{\hyperlink{method.rename-tac}{\mbox{\isa{rename{\isaliteral{5F}{\isacharunderscore}}tac}}}}\isa{{\isaliteral{22}{\isachardoublequote}}\isaliteral{5C3C5E7375703E}{}\isactrlsup {\isaliteral{2A}{\isacharasterisk}}{\isaliteral{22}{\isachardoublequote}}} & : & \isa{method} \\
    \indexdef{}{method}{rotate\_tac}\hypertarget{method.rotate-tac}{\hyperlink{method.rotate-tac}{\mbox{\isa{rotate{\isaliteral{5F}{\isacharunderscore}}tac}}}}\isa{{\isaliteral{22}{\isachardoublequote}}\isaliteral{5C3C5E7375703E}{}\isactrlsup {\isaliteral{2A}{\isacharasterisk}}{\isaliteral{22}{\isachardoublequote}}} & : & \isa{method} \\
    \indexdef{}{method}{tactic}\hypertarget{method.tactic}{\hyperlink{method.tactic}{\mbox{\isa{tactic}}}}\isa{{\isaliteral{22}{\isachardoublequote}}\isaliteral{5C3C5E7375703E}{}\isactrlsup {\isaliteral{2A}{\isacharasterisk}}{\isaliteral{22}{\isachardoublequote}}} & : & \isa{method} \\
    \indexdef{}{method}{raw\_tactic}\hypertarget{method.raw-tactic}{\hyperlink{method.raw-tactic}{\mbox{\isa{raw{\isaliteral{5F}{\isacharunderscore}}tactic}}}}\isa{{\isaliteral{22}{\isachardoublequote}}\isaliteral{5C3C5E7375703E}{}\isactrlsup {\isaliteral{2A}{\isacharasterisk}}{\isaliteral{22}{\isachardoublequote}}} & : & \isa{method} \\
  \end{matharray}

  \begin{railoutput}
\rail@begin{6}{\isa{}}
\rail@bar
\rail@term{\hyperlink{method.rule-tac}{\mbox{\isa{rule{\isaliteral{5F}{\isacharunderscore}}tac}}}}[]
\rail@nextbar{1}
\rail@term{\hyperlink{method.erule-tac}{\mbox{\isa{erule{\isaliteral{5F}{\isacharunderscore}}tac}}}}[]
\rail@nextbar{2}
\rail@term{\hyperlink{method.drule-tac}{\mbox{\isa{drule{\isaliteral{5F}{\isacharunderscore}}tac}}}}[]
\rail@nextbar{3}
\rail@term{\hyperlink{method.frule-tac}{\mbox{\isa{frule{\isaliteral{5F}{\isacharunderscore}}tac}}}}[]
\rail@nextbar{4}
\rail@term{\hyperlink{method.cut-tac}{\mbox{\isa{cut{\isaliteral{5F}{\isacharunderscore}}tac}}}}[]
\rail@nextbar{5}
\rail@term{\hyperlink{method.thin-tac}{\mbox{\isa{thin{\isaliteral{5F}{\isacharunderscore}}tac}}}}[]
\rail@endbar
\rail@bar
\rail@nextbar{1}
\rail@nont{\hyperlink{syntax.goalspec}{\mbox{\isa{goalspec}}}}[]
\rail@endbar
\rail@bar
\rail@nont{\isa{dynamic{\isaliteral{5F}{\isacharunderscore}}insts}}[]
\rail@term{\isa{\isakeyword{in}}}[]
\rail@nont{\hyperlink{syntax.thmref}{\mbox{\isa{thmref}}}}[]
\rail@nextbar{1}
\rail@nont{\hyperlink{syntax.thmrefs}{\mbox{\isa{thmrefs}}}}[]
\rail@endbar
\rail@end
\rail@begin{2}{\isa{}}
\rail@term{\hyperlink{method.subgoal-tac}{\mbox{\isa{subgoal{\isaliteral{5F}{\isacharunderscore}}tac}}}}[]
\rail@bar
\rail@nextbar{1}
\rail@nont{\hyperlink{syntax.goalspec}{\mbox{\isa{goalspec}}}}[]
\rail@endbar
\rail@plus
\rail@nont{\hyperlink{syntax.prop}{\mbox{\isa{prop}}}}[]
\rail@nextplus{1}
\rail@endplus
\rail@end
\rail@begin{2}{\isa{}}
\rail@term{\hyperlink{method.rename-tac}{\mbox{\isa{rename{\isaliteral{5F}{\isacharunderscore}}tac}}}}[]
\rail@bar
\rail@nextbar{1}
\rail@nont{\hyperlink{syntax.goalspec}{\mbox{\isa{goalspec}}}}[]
\rail@endbar
\rail@plus
\rail@nont{\hyperlink{syntax.name}{\mbox{\isa{name}}}}[]
\rail@nextplus{1}
\rail@endplus
\rail@end
\rail@begin{2}{\isa{}}
\rail@term{\hyperlink{method.rotate-tac}{\mbox{\isa{rotate{\isaliteral{5F}{\isacharunderscore}}tac}}}}[]
\rail@bar
\rail@nextbar{1}
\rail@nont{\hyperlink{syntax.goalspec}{\mbox{\isa{goalspec}}}}[]
\rail@endbar
\rail@bar
\rail@nextbar{1}
\rail@nont{\hyperlink{syntax.int}{\mbox{\isa{int}}}}[]
\rail@endbar
\rail@end
\rail@begin{2}{\isa{}}
\rail@bar
\rail@term{\hyperlink{method.tactic}{\mbox{\isa{tactic}}}}[]
\rail@nextbar{1}
\rail@term{\hyperlink{method.raw-tactic}{\mbox{\isa{raw{\isaliteral{5F}{\isacharunderscore}}tactic}}}}[]
\rail@endbar
\rail@nont{\hyperlink{syntax.text}{\mbox{\isa{text}}}}[]
\rail@end
\rail@begin{2}{\isa{dynamic{\isaliteral{5F}{\isacharunderscore}}insts}}
\rail@plus
\rail@nont{\hyperlink{syntax.name}{\mbox{\isa{name}}}}[]
\rail@term{\isa{{\isaliteral{3D}{\isacharequal}}}}[]
\rail@nont{\hyperlink{syntax.term}{\mbox{\isa{term}}}}[]
\rail@nextplus{1}
\rail@cterm{\isa{\isakeyword{and}}}[]
\rail@endplus
\rail@end
\end{railoutput}


\begin{description}

  \item \hyperlink{method.rule-tac}{\mbox{\isa{rule{\isaliteral{5F}{\isacharunderscore}}tac}}} etc. do resolution of rules with explicit
  instantiation.  This works the same way as the ML tactics \verb|res_inst_tac| etc. (see \cite{isabelle-implementation})

  Multiple rules may be only given if there is no instantiation; then
  \hyperlink{method.rule-tac}{\mbox{\isa{rule{\isaliteral{5F}{\isacharunderscore}}tac}}} is the same as \verb|resolve_tac| in ML (see
  \cite{isabelle-implementation}).

  \item \hyperlink{method.cut-tac}{\mbox{\isa{cut{\isaliteral{5F}{\isacharunderscore}}tac}}} inserts facts into the proof state as
  assumption of a subgoal, see also \verb|Tactic.cut_facts_tac| in
  \cite{isabelle-implementation}.  Note that the scope of schematic
  variables is spread over the main goal statement.  Instantiations
  may be given as well, see also ML tactic \verb|cut_inst_tac| in
  \cite{isabelle-implementation}.

  \item \hyperlink{method.thin-tac}{\mbox{\isa{thin{\isaliteral{5F}{\isacharunderscore}}tac}}}~\isa{{\isaliteral{5C3C7068693E}{\isasymphi}}} deletes the specified assumption
  from a subgoal; note that \isa{{\isaliteral{5C3C7068693E}{\isasymphi}}} may contain schematic variables.
  See also \verb|thin_tac| in \cite{isabelle-implementation}.

  \item \hyperlink{method.subgoal-tac}{\mbox{\isa{subgoal{\isaliteral{5F}{\isacharunderscore}}tac}}}~\isa{{\isaliteral{5C3C7068693E}{\isasymphi}}} adds \isa{{\isaliteral{5C3C7068693E}{\isasymphi}}} as an
  assumption to a subgoal.  See also \verb|subgoal_tac| and \verb|subgoals_tac| in \cite{isabelle-implementation}.

  \item \hyperlink{method.rename-tac}{\mbox{\isa{rename{\isaliteral{5F}{\isacharunderscore}}tac}}}~\isa{{\isaliteral{22}{\isachardoublequote}}x\isaliteral{5C3C5E7375623E}{}\isactrlsub {\isadigit{1}}\ {\isaliteral{5C3C646F74733E}{\isasymdots}}\ x\isaliteral{5C3C5E7375623E}{}\isactrlsub n{\isaliteral{22}{\isachardoublequote}}} renames parameters of a
  goal according to the list \isa{{\isaliteral{22}{\isachardoublequote}}x\isaliteral{5C3C5E7375623E}{}\isactrlsub {\isadigit{1}}{\isaliteral{2C}{\isacharcomma}}\ {\isaliteral{5C3C646F74733E}{\isasymdots}}{\isaliteral{2C}{\isacharcomma}}\ x\isaliteral{5C3C5E7375623E}{}\isactrlsub n{\isaliteral{22}{\isachardoublequote}}}, which refers to the
  \emph{suffix} of variables.

  \item \hyperlink{method.rotate-tac}{\mbox{\isa{rotate{\isaliteral{5F}{\isacharunderscore}}tac}}}~\isa{n} rotates the assumptions of a
  goal by \isa{n} positions: from right to left if \isa{n} is
  positive, and from left to right if \isa{n} is negative; the
  default value is 1.  See also \verb|rotate_tac| in
  \cite{isabelle-implementation}.

  \item \hyperlink{method.tactic}{\mbox{\isa{tactic}}}~\isa{{\isaliteral{22}{\isachardoublequote}}text{\isaliteral{22}{\isachardoublequote}}} produces a proof method from
  any ML text of type \verb|tactic|.  Apart from the usual ML
  environment and the current proof context, the ML code may refer to
  the locally bound values \verb|facts|, which indicates any
  current facts used for forward-chaining.

  \item \hyperlink{method.raw-tactic}{\mbox{\isa{raw{\isaliteral{5F}{\isacharunderscore}}tactic}}} is similar to \hyperlink{method.tactic}{\mbox{\isa{tactic}}}, but
  presents the goal state in its raw internal form, where simultaneous
  subgoals appear as conjunction of the logical framework instead of
  the usual split into several subgoals.  While feature this is useful
  for debugging of complex method definitions, it should not never
  appear in production theories.

  \end{description}%
\end{isamarkuptext}%
\isamarkuptrue%
%
\isamarkupsection{The Simplifier \label{sec:simplifier}%
}
\isamarkuptrue%
%
\isamarkupsubsection{Simplification methods%
}
\isamarkuptrue%
%
\begin{isamarkuptext}%
\begin{matharray}{rcl}
    \indexdef{}{method}{simp}\hypertarget{method.simp}{\hyperlink{method.simp}{\mbox{\isa{simp}}}} & : & \isa{method} \\
    \indexdef{}{method}{simp\_all}\hypertarget{method.simp-all}{\hyperlink{method.simp-all}{\mbox{\isa{simp{\isaliteral{5F}{\isacharunderscore}}all}}}} & : & \isa{method} \\
  \end{matharray}

  \begin{railoutput}
\rail@begin{2}{\isa{}}
\rail@bar
\rail@term{\hyperlink{method.simp}{\mbox{\isa{simp}}}}[]
\rail@nextbar{1}
\rail@term{\hyperlink{method.simp-all}{\mbox{\isa{simp{\isaliteral{5F}{\isacharunderscore}}all}}}}[]
\rail@endbar
\rail@bar
\rail@nextbar{1}
\rail@nont{\isa{opt}}[]
\rail@endbar
\rail@plus
\rail@nextplus{1}
\rail@cnont{\hyperlink{syntax.simpmod}{\mbox{\isa{simpmod}}}}[]
\rail@endplus
\rail@end
\rail@begin{4}{\isa{opt}}
\rail@term{\isa{{\isaliteral{28}{\isacharparenleft}}}}[]
\rail@bar
\rail@term{\isa{no{\isaliteral{5F}{\isacharunderscore}}asm}}[]
\rail@nextbar{1}
\rail@term{\isa{no{\isaliteral{5F}{\isacharunderscore}}asm{\isaliteral{5F}{\isacharunderscore}}simp}}[]
\rail@nextbar{2}
\rail@term{\isa{no{\isaliteral{5F}{\isacharunderscore}}asm{\isaliteral{5F}{\isacharunderscore}}use}}[]
\rail@nextbar{3}
\rail@term{\isa{asm{\isaliteral{5F}{\isacharunderscore}}lr}}[]
\rail@endbar
\rail@term{\isa{{\isaliteral{29}{\isacharparenright}}}}[]
\rail@end
\rail@begin{9}{\indexdef{}{syntax}{simpmod}\hypertarget{syntax.simpmod}{\hyperlink{syntax.simpmod}{\mbox{\isa{simpmod}}}}}
\rail@bar
\rail@term{\isa{add}}[]
\rail@nextbar{1}
\rail@term{\isa{del}}[]
\rail@nextbar{2}
\rail@term{\isa{only}}[]
\rail@nextbar{3}
\rail@term{\isa{cong}}[]
\rail@bar
\rail@nextbar{4}
\rail@term{\isa{add}}[]
\rail@nextbar{5}
\rail@term{\isa{del}}[]
\rail@endbar
\rail@nextbar{6}
\rail@term{\isa{split}}[]
\rail@bar
\rail@nextbar{7}
\rail@term{\isa{add}}[]
\rail@nextbar{8}
\rail@term{\isa{del}}[]
\rail@endbar
\rail@endbar
\rail@term{\isa{{\isaliteral{3A}{\isacharcolon}}}}[]
\rail@nont{\hyperlink{syntax.thmrefs}{\mbox{\isa{thmrefs}}}}[]
\rail@end
\end{railoutput}


  \begin{description}

  \item \hyperlink{method.simp}{\mbox{\isa{simp}}} invokes the Simplifier, after declaring
  additional rules according to the arguments given.  Note that the
  \isa{only} modifier first removes all other rewrite rules,
  congruences, and looper tactics (including splits), and then behaves
  like \isa{add}.

  \medskip The \isa{cong} modifiers add or delete Simplifier
  congruence rules (see also \cite{isabelle-ref}), the default is to
  add.

  \medskip The \isa{split} modifiers add or delete rules for the
  Splitter (see also \cite{isabelle-ref}), the default is to add.
  This works only if the Simplifier method has been properly setup to
  include the Splitter (all major object logics such HOL, HOLCF, FOL,
  ZF do this already).

  \item \hyperlink{method.simp-all}{\mbox{\isa{simp{\isaliteral{5F}{\isacharunderscore}}all}}} is similar to \hyperlink{method.simp}{\mbox{\isa{simp}}}, but acts on
  all goals (backwards from the last to the first one).

  \end{description}

  By default the Simplifier methods take local assumptions fully into
  account, using equational assumptions in the subsequent
  normalization process, or simplifying assumptions themselves (cf.\
  \verb|asm_full_simp_tac| in \cite{isabelle-ref}).  In structured
  proofs this is usually quite well behaved in practice: just the
  local premises of the actual goal are involved, additional facts may
  be inserted via explicit forward-chaining (via \hyperlink{command.then}{\mbox{\isa{\isacommand{then}}}},
  \hyperlink{command.from}{\mbox{\isa{\isacommand{from}}}}, \hyperlink{command.using}{\mbox{\isa{\isacommand{using}}}} etc.).

  Additional Simplifier options may be specified to tune the behavior
  further (mostly for unstructured scripts with many accidental local
  facts): ``\isa{{\isaliteral{22}{\isachardoublequote}}{\isaliteral{28}{\isacharparenleft}}no{\isaliteral{5F}{\isacharunderscore}}asm{\isaliteral{29}{\isacharparenright}}{\isaliteral{22}{\isachardoublequote}}}'' means assumptions are ignored
  completely (cf.\ \verb|simp_tac|), ``\isa{{\isaliteral{22}{\isachardoublequote}}{\isaliteral{28}{\isacharparenleft}}no{\isaliteral{5F}{\isacharunderscore}}asm{\isaliteral{5F}{\isacharunderscore}}simp{\isaliteral{29}{\isacharparenright}}{\isaliteral{22}{\isachardoublequote}}}'' means
  assumptions are used in the simplification of the conclusion but are
  not themselves simplified (cf.\ \verb|asm_simp_tac|), and ``\isa{{\isaliteral{22}{\isachardoublequote}}{\isaliteral{28}{\isacharparenleft}}no{\isaliteral{5F}{\isacharunderscore}}asm{\isaliteral{5F}{\isacharunderscore}}use{\isaliteral{29}{\isacharparenright}}{\isaliteral{22}{\isachardoublequote}}}'' means assumptions are simplified but are not used
  in the simplification of each other or the conclusion (cf.\ \verb|full_simp_tac|).  For compatibility reasons, there is also an option
  ``\isa{{\isaliteral{22}{\isachardoublequote}}{\isaliteral{28}{\isacharparenleft}}asm{\isaliteral{5F}{\isacharunderscore}}lr{\isaliteral{29}{\isacharparenright}}{\isaliteral{22}{\isachardoublequote}}}'', which means that an assumption is only used
  for simplifying assumptions which are to the right of it (cf.\ \verb|asm_lr_simp_tac|).

  The configuration option \isa{{\isaliteral{22}{\isachardoublequote}}depth{\isaliteral{5F}{\isacharunderscore}}limit{\isaliteral{22}{\isachardoublequote}}} limits the number of
  recursive invocations of the simplifier during conditional
  rewriting.

  \medskip The Splitter package is usually configured to work as part
  of the Simplifier.  The effect of repeatedly applying \verb|split_tac| can be simulated by ``\isa{{\isaliteral{22}{\isachardoublequote}}{\isaliteral{28}{\isacharparenleft}}simp\ only{\isaliteral{3A}{\isacharcolon}}\ split{\isaliteral{3A}{\isacharcolon}}\ a\isaliteral{5C3C5E7375623E}{}\isactrlsub {\isadigit{1}}\ {\isaliteral{5C3C646F74733E}{\isasymdots}}\ a\isaliteral{5C3C5E7375623E}{}\isactrlsub n{\isaliteral{29}{\isacharparenright}}{\isaliteral{22}{\isachardoublequote}}}''.  There is also a separate \isa{split}
  method available for single-step case splitting.%
\end{isamarkuptext}%
\isamarkuptrue%
%
\isamarkupsubsection{Declaring rules%
}
\isamarkuptrue%
%
\begin{isamarkuptext}%
\begin{matharray}{rcl}
    \indexdef{}{command}{print\_simpset}\hypertarget{command.print-simpset}{\hyperlink{command.print-simpset}{\mbox{\isa{\isacommand{print{\isaliteral{5F}{\isacharunderscore}}simpset}}}}}\isa{{\isaliteral{22}{\isachardoublequote}}\isaliteral{5C3C5E7375703E}{}\isactrlsup {\isaliteral{2A}{\isacharasterisk}}{\isaliteral{22}{\isachardoublequote}}} & : & \isa{{\isaliteral{22}{\isachardoublequote}}context\ {\isaliteral{5C3C72696768746172726F773E}{\isasymrightarrow}}{\isaliteral{22}{\isachardoublequote}}} \\
    \indexdef{}{attribute}{simp}\hypertarget{attribute.simp}{\hyperlink{attribute.simp}{\mbox{\isa{simp}}}} & : & \isa{attribute} \\
    \indexdef{}{attribute}{cong}\hypertarget{attribute.cong}{\hyperlink{attribute.cong}{\mbox{\isa{cong}}}} & : & \isa{attribute} \\
    \indexdef{}{attribute}{split}\hypertarget{attribute.split}{\hyperlink{attribute.split}{\mbox{\isa{split}}}} & : & \isa{attribute} \\
  \end{matharray}

  \begin{railoutput}
\rail@begin{3}{\isa{}}
\rail@bar
\rail@term{\hyperlink{attribute.simp}{\mbox{\isa{simp}}}}[]
\rail@nextbar{1}
\rail@term{\hyperlink{attribute.cong}{\mbox{\isa{cong}}}}[]
\rail@nextbar{2}
\rail@term{\hyperlink{attribute.split}{\mbox{\isa{split}}}}[]
\rail@endbar
\rail@bar
\rail@nextbar{1}
\rail@term{\isa{add}}[]
\rail@nextbar{2}
\rail@term{\isa{del}}[]
\rail@endbar
\rail@end
\end{railoutput}


  \begin{description}

  \item \hyperlink{command.print-simpset}{\mbox{\isa{\isacommand{print{\isaliteral{5F}{\isacharunderscore}}simpset}}}} prints the collection of rules
  declared to the Simplifier, which is also known as ``simpset''
  internally \cite{isabelle-ref}.

  \item \hyperlink{attribute.simp}{\mbox{\isa{simp}}} declares simplification rules.

  \item \hyperlink{attribute.cong}{\mbox{\isa{cong}}} declares congruence rules.

  \item \hyperlink{attribute.split}{\mbox{\isa{split}}} declares case split rules.

  \end{description}%
\end{isamarkuptext}%
\isamarkuptrue%
%
\isamarkupsubsection{Simplification procedures%
}
\isamarkuptrue%
%
\begin{isamarkuptext}%
\begin{matharray}{rcl}
    \indexdef{}{command}{simproc\_setup}\hypertarget{command.simproc-setup}{\hyperlink{command.simproc-setup}{\mbox{\isa{\isacommand{simproc{\isaliteral{5F}{\isacharunderscore}}setup}}}}} & : & \isa{{\isaliteral{22}{\isachardoublequote}}local{\isaliteral{5F}{\isacharunderscore}}theory\ {\isaliteral{5C3C72696768746172726F773E}{\isasymrightarrow}}\ local{\isaliteral{5F}{\isacharunderscore}}theory{\isaliteral{22}{\isachardoublequote}}} \\
    simproc & : & \isa{attribute} \\
  \end{matharray}

  \begin{railoutput}
\rail@begin{6}{\isa{}}
\rail@term{\hyperlink{command.simproc-setup}{\mbox{\isa{\isacommand{simproc{\isaliteral{5F}{\isacharunderscore}}setup}}}}}[]
\rail@nont{\hyperlink{syntax.name}{\mbox{\isa{name}}}}[]
\rail@term{\isa{{\isaliteral{28}{\isacharparenleft}}}}[]
\rail@plus
\rail@nont{\hyperlink{syntax.term}{\mbox{\isa{term}}}}[]
\rail@nextplus{1}
\rail@cterm{\isa{{\isaliteral{7C}{\isacharbar}}}}[]
\rail@endplus
\rail@term{\isa{{\isaliteral{29}{\isacharparenright}}}}[]
\rail@term{\isa{{\isaliteral{3D}{\isacharequal}}}}[]
\rail@nont{\hyperlink{syntax.text}{\mbox{\isa{text}}}}[]
\rail@cr{3}
\rail@bar
\rail@nextbar{4}
\rail@term{\isa{\isakeyword{identifier}}}[]
\rail@plus
\rail@nont{\hyperlink{syntax.nameref}{\mbox{\isa{nameref}}}}[]
\rail@nextplus{5}
\rail@endplus
\rail@endbar
\rail@end
\rail@begin{3}{\isa{}}
\rail@term{\hyperlink{attribute.simproc}{\mbox{\isa{simproc}}}}[]
\rail@bar
\rail@bar
\rail@nextbar{1}
\rail@term{\isa{add}}[]
\rail@term{\isa{{\isaliteral{3A}{\isacharcolon}}}}[]
\rail@endbar
\rail@nextbar{2}
\rail@term{\isa{del}}[]
\rail@term{\isa{{\isaliteral{3A}{\isacharcolon}}}}[]
\rail@endbar
\rail@plus
\rail@nont{\hyperlink{syntax.name}{\mbox{\isa{name}}}}[]
\rail@nextplus{1}
\rail@endplus
\rail@end
\end{railoutput}


  \begin{description}

  \item \hyperlink{command.simproc-setup}{\mbox{\isa{\isacommand{simproc{\isaliteral{5F}{\isacharunderscore}}setup}}}} defines a named simplification
  procedure that is invoked by the Simplifier whenever any of the
  given term patterns match the current redex.  The implementation,
  which is provided as ML source text, needs to be of type \verb|morphism -> simpset -> cterm -> thm option|, where the \verb|cterm| represents the current redex \isa{r} and the result is
  supposed to be some proven rewrite rule \isa{{\isaliteral{22}{\isachardoublequote}}r\ {\isaliteral{5C3C65717569763E}{\isasymequiv}}\ r{\isaliteral{27}{\isacharprime}}{\isaliteral{22}{\isachardoublequote}}} (or a
  generalized version), or \verb|NONE| to indicate failure.  The
  \verb|simpset| argument holds the full context of the current
  Simplifier invocation, including the actual Isar proof context.  The
  \verb|morphism| informs about the difference of the original
  compilation context wrt.\ the one of the actual application later
  on.  The optional \hyperlink{keyword.identifier}{\mbox{\isa{\isakeyword{identifier}}}} specifies theorems that
  represent the logical content of the abstract theory of this
  simproc.

  Morphisms and identifiers are only relevant for simprocs that are
  defined within a local target context, e.g.\ in a locale.

  \item \isa{{\isaliteral{22}{\isachardoublequote}}simproc\ add{\isaliteral{3A}{\isacharcolon}}\ name{\isaliteral{22}{\isachardoublequote}}} and \isa{{\isaliteral{22}{\isachardoublequote}}simproc\ del{\isaliteral{3A}{\isacharcolon}}\ name{\isaliteral{22}{\isachardoublequote}}}
  add or delete named simprocs to the current Simplifier context.  The
  default is to add a simproc.  Note that \hyperlink{command.simproc-setup}{\mbox{\isa{\isacommand{simproc{\isaliteral{5F}{\isacharunderscore}}setup}}}}
  already adds the new simproc to the subsequent context.

  \end{description}%
\end{isamarkuptext}%
\isamarkuptrue%
%
\isamarkupsubsection{Forward simplification%
}
\isamarkuptrue%
%
\begin{isamarkuptext}%
\begin{matharray}{rcl}
    \indexdef{}{attribute}{simplified}\hypertarget{attribute.simplified}{\hyperlink{attribute.simplified}{\mbox{\isa{simplified}}}} & : & \isa{attribute} \\
  \end{matharray}

  \begin{railoutput}
\rail@begin{2}{\isa{}}
\rail@term{\hyperlink{attribute.simplified}{\mbox{\isa{simplified}}}}[]
\rail@bar
\rail@nextbar{1}
\rail@nont{\isa{opt}}[]
\rail@endbar
\rail@bar
\rail@nextbar{1}
\rail@nont{\hyperlink{syntax.thmrefs}{\mbox{\isa{thmrefs}}}}[]
\rail@endbar
\rail@end
\rail@begin{3}{\isa{opt}}
\rail@term{\isa{{\isaliteral{28}{\isacharparenleft}}}}[]
\rail@bar
\rail@term{\isa{no{\isaliteral{5F}{\isacharunderscore}}asm}}[]
\rail@nextbar{1}
\rail@term{\isa{no{\isaliteral{5F}{\isacharunderscore}}asm{\isaliteral{5F}{\isacharunderscore}}simp}}[]
\rail@nextbar{2}
\rail@term{\isa{no{\isaliteral{5F}{\isacharunderscore}}asm{\isaliteral{5F}{\isacharunderscore}}use}}[]
\rail@endbar
\rail@term{\isa{{\isaliteral{29}{\isacharparenright}}}}[]
\rail@end
\end{railoutput}


  \begin{description}
  
  \item \hyperlink{attribute.simplified}{\mbox{\isa{simplified}}}~\isa{{\isaliteral{22}{\isachardoublequote}}a\isaliteral{5C3C5E7375623E}{}\isactrlsub {\isadigit{1}}\ {\isaliteral{5C3C646F74733E}{\isasymdots}}\ a\isaliteral{5C3C5E7375623E}{}\isactrlsub n{\isaliteral{22}{\isachardoublequote}}} causes a theorem to
  be simplified, either by exactly the specified rules \isa{{\isaliteral{22}{\isachardoublequote}}a\isaliteral{5C3C5E7375623E}{}\isactrlsub {\isadigit{1}}{\isaliteral{2C}{\isacharcomma}}\ {\isaliteral{5C3C646F74733E}{\isasymdots}}{\isaliteral{2C}{\isacharcomma}}\ a\isaliteral{5C3C5E7375623E}{}\isactrlsub n{\isaliteral{22}{\isachardoublequote}}}, or the implicit Simplifier context if no arguments are given.
  The result is fully simplified by default, including assumptions and
  conclusion; the options \isa{no{\isaliteral{5F}{\isacharunderscore}}asm} etc.\ tune the Simplifier in
  the same way as the for the \isa{simp} method.

  Note that forward simplification restricts the simplifier to its
  most basic operation of term rewriting; solver and looper tactics
  \cite{isabelle-ref} are \emph{not} involved here.  The \isa{simplified} attribute should be only rarely required under normal
  circumstances.

  \end{description}%
\end{isamarkuptext}%
\isamarkuptrue%
%
\isamarkupsection{The Classical Reasoner \label{sec:classical}%
}
\isamarkuptrue%
%
\isamarkupsubsection{Basic methods%
}
\isamarkuptrue%
%
\begin{isamarkuptext}%
\begin{matharray}{rcl}
    \indexdef{}{method}{rule}\hypertarget{method.rule}{\hyperlink{method.rule}{\mbox{\isa{rule}}}} & : & \isa{method} \\
    \indexdef{}{method}{contradiction}\hypertarget{method.contradiction}{\hyperlink{method.contradiction}{\mbox{\isa{contradiction}}}} & : & \isa{method} \\
    \indexdef{}{method}{intro}\hypertarget{method.intro}{\hyperlink{method.intro}{\mbox{\isa{intro}}}} & : & \isa{method} \\
    \indexdef{}{method}{elim}\hypertarget{method.elim}{\hyperlink{method.elim}{\mbox{\isa{elim}}}} & : & \isa{method} \\
  \end{matharray}

  \begin{railoutput}
\rail@begin{3}{\isa{}}
\rail@bar
\rail@term{\hyperlink{method.rule}{\mbox{\isa{rule}}}}[]
\rail@nextbar{1}
\rail@term{\hyperlink{method.intro}{\mbox{\isa{intro}}}}[]
\rail@nextbar{2}
\rail@term{\hyperlink{method.elim}{\mbox{\isa{elim}}}}[]
\rail@endbar
\rail@bar
\rail@nextbar{1}
\rail@nont{\hyperlink{syntax.thmrefs}{\mbox{\isa{thmrefs}}}}[]
\rail@endbar
\rail@end
\end{railoutput}


  \begin{description}

  \item \hyperlink{method.rule}{\mbox{\isa{rule}}} as offered by the Classical Reasoner is a
  refinement over the primitive one (see \secref{sec:pure-meth-att}).
  Both versions essentially work the same, but the classical version
  observes the classical rule context in addition to that of
  Isabelle/Pure.

  Common object logics (HOL, ZF, etc.) declare a rich collection of
  classical rules (even if these would qualify as intuitionistic
  ones), but only few declarations to the rule context of
  Isabelle/Pure (\secref{sec:pure-meth-att}).

  \item \hyperlink{method.contradiction}{\mbox{\isa{contradiction}}} solves some goal by contradiction,
  deriving any result from both \isa{{\isaliteral{22}{\isachardoublequote}}{\isaliteral{5C3C6E6F743E}{\isasymnot}}\ A{\isaliteral{22}{\isachardoublequote}}} and \isa{A}.  Chained
  facts, which are guaranteed to participate, may appear in either
  order.

  \item \hyperlink{method.intro}{\mbox{\isa{intro}}} and \hyperlink{method.elim}{\mbox{\isa{elim}}} repeatedly refine some goal
  by intro- or elim-resolution, after having inserted any chained
  facts.  Exactly the rules given as arguments are taken into account;
  this allows fine-tuned decomposition of a proof problem, in contrast
  to common automated tools.

  \end{description}%
\end{isamarkuptext}%
\isamarkuptrue%
%
\isamarkupsubsection{Automated methods%
}
\isamarkuptrue%
%
\begin{isamarkuptext}%
\begin{matharray}{rcl}
    \indexdef{}{method}{blast}\hypertarget{method.blast}{\hyperlink{method.blast}{\mbox{\isa{blast}}}} & : & \isa{method} \\
    \indexdef{}{method}{fast}\hypertarget{method.fast}{\hyperlink{method.fast}{\mbox{\isa{fast}}}} & : & \isa{method} \\
    \indexdef{}{method}{slow}\hypertarget{method.slow}{\hyperlink{method.slow}{\mbox{\isa{slow}}}} & : & \isa{method} \\
    \indexdef{}{method}{best}\hypertarget{method.best}{\hyperlink{method.best}{\mbox{\isa{best}}}} & : & \isa{method} \\
    \indexdef{}{method}{safe}\hypertarget{method.safe}{\hyperlink{method.safe}{\mbox{\isa{safe}}}} & : & \isa{method} \\
    \indexdef{}{method}{clarify}\hypertarget{method.clarify}{\hyperlink{method.clarify}{\mbox{\isa{clarify}}}} & : & \isa{method} \\
  \end{matharray}

  \begin{railoutput}
\rail@begin{2}{\isa{}}
\rail@term{\hyperlink{method.blast}{\mbox{\isa{blast}}}}[]
\rail@bar
\rail@nextbar{1}
\rail@nont{\hyperlink{syntax.nat}{\mbox{\isa{nat}}}}[]
\rail@endbar
\rail@plus
\rail@nextplus{1}
\rail@cnont{\hyperlink{syntax.clamod}{\mbox{\isa{clamod}}}}[]
\rail@endplus
\rail@end
\rail@begin{5}{\isa{}}
\rail@bar
\rail@term{\hyperlink{method.fast}{\mbox{\isa{fast}}}}[]
\rail@nextbar{1}
\rail@term{\hyperlink{method.slow}{\mbox{\isa{slow}}}}[]
\rail@nextbar{2}
\rail@term{\hyperlink{method.best}{\mbox{\isa{best}}}}[]
\rail@nextbar{3}
\rail@term{\hyperlink{method.safe}{\mbox{\isa{safe}}}}[]
\rail@nextbar{4}
\rail@term{\hyperlink{method.clarify}{\mbox{\isa{clarify}}}}[]
\rail@endbar
\rail@plus
\rail@nextplus{1}
\rail@cnont{\hyperlink{syntax.clamod}{\mbox{\isa{clamod}}}}[]
\rail@endplus
\rail@end
\rail@begin{4}{\indexdef{}{syntax}{clamod}\hypertarget{syntax.clamod}{\hyperlink{syntax.clamod}{\mbox{\isa{clamod}}}}}
\rail@bar
\rail@bar
\rail@term{\isa{intro}}[]
\rail@nextbar{1}
\rail@term{\isa{elim}}[]
\rail@nextbar{2}
\rail@term{\isa{dest}}[]
\rail@endbar
\rail@bar
\rail@term{\isa{{\isaliteral{21}{\isacharbang}}}}[]
\rail@nextbar{1}
\rail@nextbar{2}
\rail@term{\isa{{\isaliteral{3F}{\isacharquery}}}}[]
\rail@endbar
\rail@nextbar{3}
\rail@term{\isa{del}}[]
\rail@endbar
\rail@term{\isa{{\isaliteral{3A}{\isacharcolon}}}}[]
\rail@nont{\hyperlink{syntax.thmrefs}{\mbox{\isa{thmrefs}}}}[]
\rail@end
\end{railoutput}


  \begin{description}

  \item \hyperlink{method.blast}{\mbox{\isa{blast}}} refers to the classical tableau prover (see
  \verb|blast_tac| in \cite{isabelle-ref}).  The optional argument
  specifies a user-supplied search bound (default 20).

  \item \hyperlink{method.fast}{\mbox{\isa{fast}}}, \hyperlink{method.slow}{\mbox{\isa{slow}}}, \hyperlink{method.best}{\mbox{\isa{best}}}, \hyperlink{method.safe}{\mbox{\isa{safe}}}, and \hyperlink{method.clarify}{\mbox{\isa{clarify}}} refer to the generic classical
  reasoner.  See \verb|fast_tac|, \verb|slow_tac|, \verb|best_tac|, \verb|safe_tac|, and \verb|clarify_tac| in \cite{isabelle-ref} for more
  information.

  \end{description}

  Any of the above methods support additional modifiers of the context
  of classical rules.  Their semantics is analogous to the attributes
  given before.  Facts provided by forward chaining are inserted into
  the goal before commencing proof search.%
\end{isamarkuptext}%
\isamarkuptrue%
%
\isamarkupsubsection{Combined automated methods \label{sec:clasimp}%
}
\isamarkuptrue%
%
\begin{isamarkuptext}%
\begin{matharray}{rcl}
    \indexdef{}{method}{auto}\hypertarget{method.auto}{\hyperlink{method.auto}{\mbox{\isa{auto}}}} & : & \isa{method} \\
    \indexdef{}{method}{force}\hypertarget{method.force}{\hyperlink{method.force}{\mbox{\isa{force}}}} & : & \isa{method} \\
    \indexdef{}{method}{clarsimp}\hypertarget{method.clarsimp}{\hyperlink{method.clarsimp}{\mbox{\isa{clarsimp}}}} & : & \isa{method} \\
    \indexdef{}{method}{fastsimp}\hypertarget{method.fastsimp}{\hyperlink{method.fastsimp}{\mbox{\isa{fastsimp}}}} & : & \isa{method} \\
    \indexdef{}{method}{slowsimp}\hypertarget{method.slowsimp}{\hyperlink{method.slowsimp}{\mbox{\isa{slowsimp}}}} & : & \isa{method} \\
    \indexdef{}{method}{bestsimp}\hypertarget{method.bestsimp}{\hyperlink{method.bestsimp}{\mbox{\isa{bestsimp}}}} & : & \isa{method} \\
  \end{matharray}

  \begin{railoutput}
\rail@begin{2}{\isa{}}
\rail@term{\hyperlink{method.auto}{\mbox{\isa{auto}}}}[]
\rail@bar
\rail@nextbar{1}
\rail@nont{\hyperlink{syntax.nat}{\mbox{\isa{nat}}}}[]
\rail@nont{\hyperlink{syntax.nat}{\mbox{\isa{nat}}}}[]
\rail@endbar
\rail@plus
\rail@nextplus{1}
\rail@cnont{\hyperlink{syntax.clasimpmod}{\mbox{\isa{clasimpmod}}}}[]
\rail@endplus
\rail@end
\rail@begin{5}{\isa{}}
\rail@bar
\rail@term{\hyperlink{method.force}{\mbox{\isa{force}}}}[]
\rail@nextbar{1}
\rail@term{\hyperlink{method.clarsimp}{\mbox{\isa{clarsimp}}}}[]
\rail@nextbar{2}
\rail@term{\hyperlink{method.fastsimp}{\mbox{\isa{fastsimp}}}}[]
\rail@nextbar{3}
\rail@term{\hyperlink{method.slowsimp}{\mbox{\isa{slowsimp}}}}[]
\rail@nextbar{4}
\rail@term{\hyperlink{method.bestsimp}{\mbox{\isa{bestsimp}}}}[]
\rail@endbar
\rail@plus
\rail@nextplus{1}
\rail@cnont{\hyperlink{syntax.clasimpmod}{\mbox{\isa{clasimpmod}}}}[]
\rail@endplus
\rail@end
\rail@begin{14}{\indexdef{}{syntax}{clasimpmod}\hypertarget{syntax.clasimpmod}{\hyperlink{syntax.clasimpmod}{\mbox{\isa{clasimpmod}}}}}
\rail@bar
\rail@term{\isa{simp}}[]
\rail@bar
\rail@nextbar{1}
\rail@term{\isa{add}}[]
\rail@nextbar{2}
\rail@term{\isa{del}}[]
\rail@nextbar{3}
\rail@term{\isa{only}}[]
\rail@endbar
\rail@nextbar{4}
\rail@bar
\rail@term{\isa{cong}}[]
\rail@nextbar{5}
\rail@term{\isa{split}}[]
\rail@endbar
\rail@bar
\rail@nextbar{5}
\rail@term{\isa{add}}[]
\rail@nextbar{6}
\rail@term{\isa{del}}[]
\rail@endbar
\rail@nextbar{7}
\rail@term{\isa{iff}}[]
\rail@bar
\rail@bar
\rail@nextbar{8}
\rail@term{\isa{add}}[]
\rail@endbar
\rail@bar
\rail@nextbar{8}
\rail@term{\isa{{\isaliteral{3F}{\isacharquery}}}}[]
\rail@endbar
\rail@nextbar{9}
\rail@term{\isa{del}}[]
\rail@endbar
\rail@nextbar{10}
\rail@bar
\rail@bar
\rail@term{\isa{intro}}[]
\rail@nextbar{11}
\rail@term{\isa{elim}}[]
\rail@nextbar{12}
\rail@term{\isa{dest}}[]
\rail@endbar
\rail@bar
\rail@term{\isa{{\isaliteral{21}{\isacharbang}}}}[]
\rail@nextbar{11}
\rail@nextbar{12}
\rail@term{\isa{{\isaliteral{3F}{\isacharquery}}}}[]
\rail@endbar
\rail@nextbar{13}
\rail@term{\isa{del}}[]
\rail@endbar
\rail@endbar
\rail@term{\isa{{\isaliteral{3A}{\isacharcolon}}}}[]
\rail@nont{\hyperlink{syntax.thmrefs}{\mbox{\isa{thmrefs}}}}[]
\rail@end
\end{railoutput}


  \begin{description}

  \item \hyperlink{method.auto}{\mbox{\isa{auto}}}, \hyperlink{method.force}{\mbox{\isa{force}}}, \hyperlink{method.clarsimp}{\mbox{\isa{clarsimp}}}, \hyperlink{method.fastsimp}{\mbox{\isa{fastsimp}}}, \hyperlink{method.slowsimp}{\mbox{\isa{slowsimp}}}, and \hyperlink{method.bestsimp}{\mbox{\isa{bestsimp}}} provide access
  to Isabelle's combined simplification and classical reasoning
  tactics.  These correspond to \verb|auto_tac|, \verb|force_tac|, \verb|clarsimp_tac|, and Classical Reasoner tactics with the Simplifier
  added as wrapper, see \cite{isabelle-ref} for more information.  The
  modifier arguments correspond to those given in
  \secref{sec:simplifier} and \secref{sec:classical}.  Just note that
  the ones related to the Simplifier are prefixed by \isa{simp}
  here.

  Facts provided by forward chaining are inserted into the goal before
  doing the search.

  \end{description}%
\end{isamarkuptext}%
\isamarkuptrue%
%
\isamarkupsubsection{Declaring rules%
}
\isamarkuptrue%
%
\begin{isamarkuptext}%
\begin{matharray}{rcl}
    \indexdef{}{command}{print\_claset}\hypertarget{command.print-claset}{\hyperlink{command.print-claset}{\mbox{\isa{\isacommand{print{\isaliteral{5F}{\isacharunderscore}}claset}}}}}\isa{{\isaliteral{22}{\isachardoublequote}}\isaliteral{5C3C5E7375703E}{}\isactrlsup {\isaliteral{2A}{\isacharasterisk}}{\isaliteral{22}{\isachardoublequote}}} & : & \isa{{\isaliteral{22}{\isachardoublequote}}context\ {\isaliteral{5C3C72696768746172726F773E}{\isasymrightarrow}}{\isaliteral{22}{\isachardoublequote}}} \\
    \indexdef{}{attribute}{intro}\hypertarget{attribute.intro}{\hyperlink{attribute.intro}{\mbox{\isa{intro}}}} & : & \isa{attribute} \\
    \indexdef{}{attribute}{elim}\hypertarget{attribute.elim}{\hyperlink{attribute.elim}{\mbox{\isa{elim}}}} & : & \isa{attribute} \\
    \indexdef{}{attribute}{dest}\hypertarget{attribute.dest}{\hyperlink{attribute.dest}{\mbox{\isa{dest}}}} & : & \isa{attribute} \\
    \indexdef{}{attribute}{rule}\hypertarget{attribute.rule}{\hyperlink{attribute.rule}{\mbox{\isa{rule}}}} & : & \isa{attribute} \\
    \indexdef{}{attribute}{iff}\hypertarget{attribute.iff}{\hyperlink{attribute.iff}{\mbox{\isa{iff}}}} & : & \isa{attribute} \\
  \end{matharray}

  \begin{railoutput}
\rail@begin{3}{\isa{}}
\rail@bar
\rail@term{\hyperlink{attribute.intro}{\mbox{\isa{intro}}}}[]
\rail@nextbar{1}
\rail@term{\hyperlink{attribute.elim}{\mbox{\isa{elim}}}}[]
\rail@nextbar{2}
\rail@term{\hyperlink{attribute.dest}{\mbox{\isa{dest}}}}[]
\rail@endbar
\rail@bar
\rail@term{\isa{{\isaliteral{21}{\isacharbang}}}}[]
\rail@nextbar{1}
\rail@nextbar{2}
\rail@term{\isa{{\isaliteral{3F}{\isacharquery}}}}[]
\rail@endbar
\rail@bar
\rail@nextbar{1}
\rail@nont{\hyperlink{syntax.nat}{\mbox{\isa{nat}}}}[]
\rail@endbar
\rail@end
\rail@begin{1}{\isa{}}
\rail@term{\hyperlink{attribute.rule}{\mbox{\isa{rule}}}}[]
\rail@term{\isa{del}}[]
\rail@end
\rail@begin{3}{\isa{}}
\rail@term{\hyperlink{attribute.iff}{\mbox{\isa{iff}}}}[]
\rail@bar
\rail@bar
\rail@nextbar{1}
\rail@term{\isa{add}}[]
\rail@endbar
\rail@bar
\rail@nextbar{1}
\rail@term{\isa{{\isaliteral{3F}{\isacharquery}}}}[]
\rail@endbar
\rail@nextbar{2}
\rail@term{\isa{del}}[]
\rail@endbar
\rail@end
\end{railoutput}


  \begin{description}

  \item \hyperlink{command.print-claset}{\mbox{\isa{\isacommand{print{\isaliteral{5F}{\isacharunderscore}}claset}}}} prints the collection of rules
  declared to the Classical Reasoner, which is also known as
  ``claset'' internally \cite{isabelle-ref}.
  
  \item \hyperlink{attribute.intro}{\mbox{\isa{intro}}}, \hyperlink{attribute.elim}{\mbox{\isa{elim}}}, and \hyperlink{attribute.dest}{\mbox{\isa{dest}}}
  declare introduction, elimination, and destruction rules,
  respectively.  By default, rules are considered as \emph{unsafe}
  (i.e.\ not applied blindly without backtracking), while ``\isa{{\isaliteral{22}{\isachardoublequote}}{\isaliteral{21}{\isacharbang}}{\isaliteral{22}{\isachardoublequote}}}'' classifies as \emph{safe}.  Rule declarations marked by
  ``\isa{{\isaliteral{22}{\isachardoublequote}}{\isaliteral{3F}{\isacharquery}}{\isaliteral{22}{\isachardoublequote}}}'' coincide with those of Isabelle/Pure, cf.\
  \secref{sec:pure-meth-att} (i.e.\ are only applied in single steps
  of the \hyperlink{method.rule}{\mbox{\isa{rule}}} method).  The optional natural number
  specifies an explicit weight argument, which is ignored by automated
  tools, but determines the search order of single rule steps.

  \item \hyperlink{attribute.rule}{\mbox{\isa{rule}}}~\isa{del} deletes introduction,
  elimination, or destruction rules from the context.

  \item \hyperlink{attribute.iff}{\mbox{\isa{iff}}} declares logical equivalences to the
  Simplifier and the Classical reasoner at the same time.
  Non-conditional rules result in a ``safe'' introduction and
  elimination pair; conditional ones are considered ``unsafe''.  Rules
  with negative conclusion are automatically inverted (using \isa{{\isaliteral{22}{\isachardoublequote}}{\isaliteral{5C3C6E6F743E}{\isasymnot}}{\isaliteral{22}{\isachardoublequote}}}-elimination internally).

  The ``\isa{{\isaliteral{22}{\isachardoublequote}}{\isaliteral{3F}{\isacharquery}}{\isaliteral{22}{\isachardoublequote}}}'' version of \hyperlink{attribute.iff}{\mbox{\isa{iff}}} declares rules to
  the Isabelle/Pure context only, and omits the Simplifier
  declaration.

  \end{description}%
\end{isamarkuptext}%
\isamarkuptrue%
%
\isamarkupsubsection{Classical operations%
}
\isamarkuptrue%
%
\begin{isamarkuptext}%
\begin{matharray}{rcl}
    \indexdef{}{attribute}{swapped}\hypertarget{attribute.swapped}{\hyperlink{attribute.swapped}{\mbox{\isa{swapped}}}} & : & \isa{attribute} \\
  \end{matharray}

  \begin{description}

  \item \hyperlink{attribute.swapped}{\mbox{\isa{swapped}}} turns an introduction rule into an
  elimination, by resolving with the classical swap principle \isa{{\isaliteral{22}{\isachardoublequote}}{\isaliteral{28}{\isacharparenleft}}{\isaliteral{5C3C6E6F743E}{\isasymnot}}\ B\ {\isaliteral{5C3C4C6F6E6772696768746172726F773E}{\isasymLongrightarrow}}\ A{\isaliteral{29}{\isacharparenright}}\ {\isaliteral{5C3C4C6F6E6772696768746172726F773E}{\isasymLongrightarrow}}\ {\isaliteral{28}{\isacharparenleft}}{\isaliteral{5C3C6E6F743E}{\isasymnot}}\ A\ {\isaliteral{5C3C4C6F6E6772696768746172726F773E}{\isasymLongrightarrow}}\ B{\isaliteral{29}{\isacharparenright}}{\isaliteral{22}{\isachardoublequote}}}.

  \end{description}%
\end{isamarkuptext}%
\isamarkuptrue%
%
\isamarkupsection{Object-logic setup \label{sec:object-logic}%
}
\isamarkuptrue%
%
\begin{isamarkuptext}%
\begin{matharray}{rcl}
    \indexdef{}{command}{judgment}\hypertarget{command.judgment}{\hyperlink{command.judgment}{\mbox{\isa{\isacommand{judgment}}}}} & : & \isa{{\isaliteral{22}{\isachardoublequote}}theory\ {\isaliteral{5C3C72696768746172726F773E}{\isasymrightarrow}}\ theory{\isaliteral{22}{\isachardoublequote}}} \\
    \indexdef{}{method}{atomize}\hypertarget{method.atomize}{\hyperlink{method.atomize}{\mbox{\isa{atomize}}}} & : & \isa{method} \\
    \indexdef{}{attribute}{atomize}\hypertarget{attribute.atomize}{\hyperlink{attribute.atomize}{\mbox{\isa{atomize}}}} & : & \isa{attribute} \\
    \indexdef{}{attribute}{rule\_format}\hypertarget{attribute.rule-format}{\hyperlink{attribute.rule-format}{\mbox{\isa{rule{\isaliteral{5F}{\isacharunderscore}}format}}}} & : & \isa{attribute} \\
    \indexdef{}{attribute}{rulify}\hypertarget{attribute.rulify}{\hyperlink{attribute.rulify}{\mbox{\isa{rulify}}}} & : & \isa{attribute} \\
  \end{matharray}

  The very starting point for any Isabelle object-logic is a ``truth
  judgment'' that links object-level statements to the meta-logic
  (with its minimal language of \isa{prop} that covers universal
  quantification \isa{{\isaliteral{22}{\isachardoublequote}}{\isaliteral{5C3C416E643E}{\isasymAnd}}{\isaliteral{22}{\isachardoublequote}}} and implication \isa{{\isaliteral{22}{\isachardoublequote}}{\isaliteral{5C3C4C6F6E6772696768746172726F773E}{\isasymLongrightarrow}}{\isaliteral{22}{\isachardoublequote}}}).

  Common object-logics are sufficiently expressive to internalize rule
  statements over \isa{{\isaliteral{22}{\isachardoublequote}}{\isaliteral{5C3C416E643E}{\isasymAnd}}{\isaliteral{22}{\isachardoublequote}}} and \isa{{\isaliteral{22}{\isachardoublequote}}{\isaliteral{5C3C4C6F6E6772696768746172726F773E}{\isasymLongrightarrow}}{\isaliteral{22}{\isachardoublequote}}} within their own
  language.  This is useful in certain situations where a rule needs
  to be viewed as an atomic statement from the meta-level perspective,
  e.g.\ \isa{{\isaliteral{22}{\isachardoublequote}}{\isaliteral{5C3C416E643E}{\isasymAnd}}x{\isaliteral{2E}{\isachardot}}\ x\ {\isaliteral{5C3C696E3E}{\isasymin}}\ A\ {\isaliteral{5C3C4C6F6E6772696768746172726F773E}{\isasymLongrightarrow}}\ P\ x{\isaliteral{22}{\isachardoublequote}}} versus \isa{{\isaliteral{22}{\isachardoublequote}}{\isaliteral{5C3C666F72616C6C3E}{\isasymforall}}x\ {\isaliteral{5C3C696E3E}{\isasymin}}\ A{\isaliteral{2E}{\isachardot}}\ P\ x{\isaliteral{22}{\isachardoublequote}}}.

  From the following language elements, only the \hyperlink{method.atomize}{\mbox{\isa{atomize}}}
  method and \hyperlink{attribute.rule-format}{\mbox{\isa{rule{\isaliteral{5F}{\isacharunderscore}}format}}} attribute are occasionally
  required by end-users, the rest is for those who need to setup their
  own object-logic.  In the latter case existing formulations of
  Isabelle/FOL or Isabelle/HOL may be taken as realistic examples.

  Generic tools may refer to the information provided by object-logic
  declarations internally.

  \begin{railoutput}
\rail@begin{1}{\isa{}}
\rail@term{\hyperlink{command.judgment}{\mbox{\isa{\isacommand{judgment}}}}}[]
\rail@nont{\hyperlink{syntax.constdecl}{\mbox{\isa{constdecl}}}}[]
\rail@end
\rail@begin{2}{\isa{}}
\rail@term{\hyperlink{attribute.atomize}{\mbox{\isa{atomize}}}}[]
\rail@bar
\rail@nextbar{1}
\rail@term{\isa{{\isaliteral{28}{\isacharparenleft}}}}[]
\rail@term{\isa{full}}[]
\rail@term{\isa{{\isaliteral{29}{\isacharparenright}}}}[]
\rail@endbar
\rail@end
\rail@begin{2}{\isa{}}
\rail@term{\hyperlink{attribute.rule-format}{\mbox{\isa{rule{\isaliteral{5F}{\isacharunderscore}}format}}}}[]
\rail@bar
\rail@nextbar{1}
\rail@term{\isa{{\isaliteral{28}{\isacharparenleft}}}}[]
\rail@term{\isa{noasm}}[]
\rail@term{\isa{{\isaliteral{29}{\isacharparenright}}}}[]
\rail@endbar
\rail@end
\end{railoutput}


  \begin{description}
  
  \item \hyperlink{command.judgment}{\mbox{\isa{\isacommand{judgment}}}}~\isa{{\isaliteral{22}{\isachardoublequote}}c\ {\isaliteral{3A}{\isacharcolon}}{\isaliteral{3A}{\isacharcolon}}\ {\isaliteral{5C3C7369676D613E}{\isasymsigma}}\ {\isaliteral{28}{\isacharparenleft}}mx{\isaliteral{29}{\isacharparenright}}{\isaliteral{22}{\isachardoublequote}}} declares constant
  \isa{c} as the truth judgment of the current object-logic.  Its
  type \isa{{\isaliteral{5C3C7369676D613E}{\isasymsigma}}} should specify a coercion of the category of
  object-level propositions to \isa{prop} of the Pure meta-logic;
  the mixfix annotation \isa{{\isaliteral{22}{\isachardoublequote}}{\isaliteral{28}{\isacharparenleft}}mx{\isaliteral{29}{\isacharparenright}}{\isaliteral{22}{\isachardoublequote}}} would typically just link the
  object language (internally of syntactic category \isa{logic})
  with that of \isa{prop}.  Only one \hyperlink{command.judgment}{\mbox{\isa{\isacommand{judgment}}}}
  declaration may be given in any theory development.
  
  \item \hyperlink{method.atomize}{\mbox{\isa{atomize}}} (as a method) rewrites any non-atomic
  premises of a sub-goal, using the meta-level equations declared via
  \hyperlink{attribute.atomize}{\mbox{\isa{atomize}}} (as an attribute) beforehand.  As a result,
  heavily nested goals become amenable to fundamental operations such
  as resolution (cf.\ the \hyperlink{method.Pure.rule}{\mbox{\isa{rule}}} method).  Giving the ``\isa{{\isaliteral{22}{\isachardoublequote}}{\isaliteral{28}{\isacharparenleft}}full{\isaliteral{29}{\isacharparenright}}{\isaliteral{22}{\isachardoublequote}}}'' option here means to turn the whole subgoal into an
  object-statement (if possible), including the outermost parameters
  and assumptions as well.

  A typical collection of \hyperlink{attribute.atomize}{\mbox{\isa{atomize}}} rules for a particular
  object-logic would provide an internalization for each of the
  connectives of \isa{{\isaliteral{22}{\isachardoublequote}}{\isaliteral{5C3C416E643E}{\isasymAnd}}{\isaliteral{22}{\isachardoublequote}}}, \isa{{\isaliteral{22}{\isachardoublequote}}{\isaliteral{5C3C4C6F6E6772696768746172726F773E}{\isasymLongrightarrow}}{\isaliteral{22}{\isachardoublequote}}}, and \isa{{\isaliteral{22}{\isachardoublequote}}{\isaliteral{5C3C65717569763E}{\isasymequiv}}{\isaliteral{22}{\isachardoublequote}}}.
  Meta-level conjunction should be covered as well (this is
  particularly important for locales, see \secref{sec:locale}).

  \item \hyperlink{attribute.rule-format}{\mbox{\isa{rule{\isaliteral{5F}{\isacharunderscore}}format}}} rewrites a theorem by the equalities
  declared as \hyperlink{attribute.rulify}{\mbox{\isa{rulify}}} rules in the current object-logic.
  By default, the result is fully normalized, including assumptions
  and conclusions at any depth.  The \isa{{\isaliteral{22}{\isachardoublequote}}{\isaliteral{28}{\isacharparenleft}}no{\isaliteral{5F}{\isacharunderscore}}asm{\isaliteral{29}{\isacharparenright}}{\isaliteral{22}{\isachardoublequote}}} option
  restricts the transformation to the conclusion of a rule.

  In common object-logics (HOL, FOL, ZF), the effect of \hyperlink{attribute.rule-format}{\mbox{\isa{rule{\isaliteral{5F}{\isacharunderscore}}format}}} is to replace (bounded) universal quantification
  (\isa{{\isaliteral{22}{\isachardoublequote}}{\isaliteral{5C3C666F72616C6C3E}{\isasymforall}}{\isaliteral{22}{\isachardoublequote}}}) and implication (\isa{{\isaliteral{22}{\isachardoublequote}}{\isaliteral{5C3C6C6F6E6772696768746172726F773E}{\isasymlongrightarrow}}{\isaliteral{22}{\isachardoublequote}}}) by the corresponding
  rule statements over \isa{{\isaliteral{22}{\isachardoublequote}}{\isaliteral{5C3C416E643E}{\isasymAnd}}{\isaliteral{22}{\isachardoublequote}}} and \isa{{\isaliteral{22}{\isachardoublequote}}{\isaliteral{5C3C4C6F6E6772696768746172726F773E}{\isasymLongrightarrow}}{\isaliteral{22}{\isachardoublequote}}}.

  \end{description}%
\end{isamarkuptext}%
\isamarkuptrue%
%
\isadelimtheory
%
\endisadelimtheory
%
\isatagtheory
\isacommand{end}\isamarkupfalse%
%
\endisatagtheory
{\isafoldtheory}%
%
\isadelimtheory
%
\endisadelimtheory
\isanewline
\end{isabellebody}%
%%% Local Variables:
%%% mode: latex
%%% TeX-master: "root"
%%% End:
