%
\begin{isabellebody}%
\def\isabellecontext{Synopsis}%
%
\isadelimtheory
%
\endisadelimtheory
%
\isatagtheory
\isacommand{theory}\isamarkupfalse%
\ Synopsis\isanewline
\isakeyword{imports}\ Base\ Main\isanewline
\isakeyword{begin}%
\endisatagtheory
{\isafoldtheory}%
%
\isadelimtheory
%
\endisadelimtheory
%
\isamarkupchapter{Synopsis%
}
\isamarkuptrue%
%
\isamarkupsection{Notepad%
}
\isamarkuptrue%
%
\begin{isamarkuptext}%
An Isar proof body serves as mathematical notepad to compose logical
  content, consisting of types, terms, facts.%
\end{isamarkuptext}%
\isamarkuptrue%
%
\isamarkupsubsection{Types and terms%
}
\isamarkuptrue%
\isacommand{notepad}\isamarkupfalse%
\isanewline
\isakeyword{begin}%
\isadelimproof
%
\endisadelimproof
%
\isatagproof
%
\begin{isamarkuptxt}%
Locally fixed entities:%
\end{isamarkuptxt}%
\isamarkuptrue%
\ \ \isacommand{fix}\isamarkupfalse%
\ x\ \ \ %
\isamarkupcmt{local constant, without any type information yet%
}
\isanewline
\ \ \isacommand{fix}\isamarkupfalse%
\ x\ {\isaliteral{3A}{\isacharcolon}}{\isaliteral{3A}{\isacharcolon}}\ {\isaliteral{27}{\isacharprime}}a\ \ %
\isamarkupcmt{variant with explicit type-constraint for subsequent use%
}
\isanewline
\isanewline
\ \ \isacommand{fix}\isamarkupfalse%
\ a\ b\isanewline
\ \ \isacommand{assume}\isamarkupfalse%
\ {\isaliteral{22}{\isachardoublequoteopen}}a\ {\isaliteral{3D}{\isacharequal}}\ b{\isaliteral{22}{\isachardoublequoteclose}}\ \ %
\isamarkupcmt{type assignment at first occurrence in concrete term%
}
%
\begin{isamarkuptxt}%
Definitions (non-polymorphic):%
\end{isamarkuptxt}%
\isamarkuptrue%
\ \ \isacommand{def}\isamarkupfalse%
\ x\ {\isaliteral{5C3C65717569763E}{\isasymequiv}}\ {\isaliteral{22}{\isachardoublequoteopen}}t{\isaliteral{3A}{\isacharcolon}}{\isaliteral{3A}{\isacharcolon}}{\isaliteral{27}{\isacharprime}}a{\isaliteral{22}{\isachardoublequoteclose}}%
\begin{isamarkuptxt}%
Abbreviations (polymorphic):%
\end{isamarkuptxt}%
\isamarkuptrue%
\ \ \isacommand{let}\isamarkupfalse%
\ {\isaliteral{3F}{\isacharquery}}f\ {\isaliteral{3D}{\isacharequal}}\ {\isaliteral{22}{\isachardoublequoteopen}}{\isaliteral{5C3C6C616D6264613E}{\isasymlambda}}x{\isaliteral{2E}{\isachardot}}\ x{\isaliteral{22}{\isachardoublequoteclose}}%
\endisatagproof
{\isafoldproof}%
%
\isadelimproof
%
\endisadelimproof
\isanewline
\ \ \isacommand{term}\isamarkupfalse%
\ {\isaliteral{22}{\isachardoublequoteopen}}{\isaliteral{3F}{\isacharquery}}f\ {\isaliteral{3F}{\isacharquery}}f{\isaliteral{22}{\isachardoublequoteclose}}%
\isadelimproof
%
\endisadelimproof
%
\isatagproof
%
\begin{isamarkuptxt}%
Notation:%
\end{isamarkuptxt}%
\isamarkuptrue%
\ \ \isacommand{write}\isamarkupfalse%
\ x\ \ {\isaliteral{28}{\isacharparenleft}}{\isaliteral{22}{\isachardoublequoteopen}}{\isaliteral{2A}{\isacharasterisk}}{\isaliteral{2A}{\isacharasterisk}}{\isaliteral{2A}{\isacharasterisk}}{\isaliteral{22}{\isachardoublequoteclose}}{\isaliteral{29}{\isacharparenright}}%
\endisatagproof
{\isafoldproof}%
%
\isadelimproof
%
\endisadelimproof
\isanewline
\isacommand{end}\isamarkupfalse%
%
\isamarkupsubsection{Facts%
}
\isamarkuptrue%
%
\begin{isamarkuptext}%
A fact is a simultaneous list of theorems.%
\end{isamarkuptext}%
\isamarkuptrue%
%
\isamarkupsubsubsection{Producing facts%
}
\isamarkuptrue%
\isacommand{notepad}\isamarkupfalse%
\isanewline
\isakeyword{begin}%
\isadelimproof
%
\endisadelimproof
%
\isatagproof
%
\begin{isamarkuptxt}%
Via assumption (``lambda''):%
\end{isamarkuptxt}%
\isamarkuptrue%
\ \ \isacommand{assume}\isamarkupfalse%
\ a{\isaliteral{3A}{\isacharcolon}}\ A%
\begin{isamarkuptxt}%
Via proof (``let''):%
\end{isamarkuptxt}%
\isamarkuptrue%
\ \ \isacommand{have}\isamarkupfalse%
\ b{\isaliteral{3A}{\isacharcolon}}\ B\ \isacommand{sorry}\isamarkupfalse%
%
\begin{isamarkuptxt}%
Via abbreviation (``let''):%
\end{isamarkuptxt}%
\isamarkuptrue%
\ \ \isacommand{note}\isamarkupfalse%
\ c\ {\isaliteral{3D}{\isacharequal}}\ a\ b%
\endisatagproof
{\isafoldproof}%
%
\isadelimproof
%
\endisadelimproof
\isanewline
\isanewline
\isacommand{end}\isamarkupfalse%
%
\isamarkupsubsubsection{Referencing facts%
}
\isamarkuptrue%
\isacommand{notepad}\isamarkupfalse%
\isanewline
\isakeyword{begin}%
\isadelimproof
%
\endisadelimproof
%
\isatagproof
%
\begin{isamarkuptxt}%
Via explicit name:%
\end{isamarkuptxt}%
\isamarkuptrue%
\ \ \isacommand{assume}\isamarkupfalse%
\ a{\isaliteral{3A}{\isacharcolon}}\ A\isanewline
\ \ \isacommand{note}\isamarkupfalse%
\ a%
\begin{isamarkuptxt}%
Via implicit name:%
\end{isamarkuptxt}%
\isamarkuptrue%
\ \ \isacommand{assume}\isamarkupfalse%
\ A\isanewline
\ \ \isacommand{note}\isamarkupfalse%
\ this%
\begin{isamarkuptxt}%
Via literal proposition (unification with results from the proof text):%
\end{isamarkuptxt}%
\isamarkuptrue%
\ \ \isacommand{assume}\isamarkupfalse%
\ A\isanewline
\ \ \isacommand{note}\isamarkupfalse%
\ {\isaliteral{60}{\isacharbackquoteopen}}A{\isaliteral{60}{\isacharbackquoteclose}}\isanewline
\isanewline
\ \ \isacommand{assume}\isamarkupfalse%
\ {\isaliteral{22}{\isachardoublequoteopen}}{\isaliteral{5C3C416E643E}{\isasymAnd}}x{\isaliteral{2E}{\isachardot}}\ B\ x{\isaliteral{22}{\isachardoublequoteclose}}\isanewline
\ \ \isacommand{note}\isamarkupfalse%
\ {\isaliteral{60}{\isacharbackquoteopen}}B\ a{\isaliteral{60}{\isacharbackquoteclose}}\isanewline
\ \ \isacommand{note}\isamarkupfalse%
\ {\isaliteral{60}{\isacharbackquoteopen}}B\ b{\isaliteral{60}{\isacharbackquoteclose}}%
\endisatagproof
{\isafoldproof}%
%
\isadelimproof
%
\endisadelimproof
\isanewline
\isacommand{end}\isamarkupfalse%
%
\isamarkupsubsubsection{Manipulating facts%
}
\isamarkuptrue%
\isacommand{notepad}\isamarkupfalse%
\isanewline
\isakeyword{begin}%
\isadelimproof
%
\endisadelimproof
%
\isatagproof
%
\begin{isamarkuptxt}%
Instantiation:%
\end{isamarkuptxt}%
\isamarkuptrue%
\ \ \isacommand{assume}\isamarkupfalse%
\ a{\isaliteral{3A}{\isacharcolon}}\ {\isaliteral{22}{\isachardoublequoteopen}}{\isaliteral{5C3C416E643E}{\isasymAnd}}x{\isaliteral{2E}{\isachardot}}\ B\ x{\isaliteral{22}{\isachardoublequoteclose}}\isanewline
\ \ \isacommand{note}\isamarkupfalse%
\ a\isanewline
\ \ \isacommand{note}\isamarkupfalse%
\ a\ {\isaliteral{5B}{\isacharbrackleft}}of\ b{\isaliteral{5D}{\isacharbrackright}}\isanewline
\ \ \isacommand{note}\isamarkupfalse%
\ a\ {\isaliteral{5B}{\isacharbrackleft}}\isakeyword{where}\ x\ {\isaliteral{3D}{\isacharequal}}\ b{\isaliteral{5D}{\isacharbrackright}}%
\begin{isamarkuptxt}%
Backchaining:%
\end{isamarkuptxt}%
\isamarkuptrue%
\ \ \isacommand{assume}\isamarkupfalse%
\ {\isadigit{1}}{\isaliteral{3A}{\isacharcolon}}\ A\isanewline
\ \ \isacommand{assume}\isamarkupfalse%
\ {\isadigit{2}}{\isaliteral{3A}{\isacharcolon}}\ {\isaliteral{22}{\isachardoublequoteopen}}A\ {\isaliteral{5C3C4C6F6E6772696768746172726F773E}{\isasymLongrightarrow}}\ C{\isaliteral{22}{\isachardoublequoteclose}}\isanewline
\ \ \isacommand{note}\isamarkupfalse%
\ {\isadigit{2}}\ {\isaliteral{5B}{\isacharbrackleft}}OF\ {\isadigit{1}}{\isaliteral{5D}{\isacharbrackright}}\isanewline
\ \ \isacommand{note}\isamarkupfalse%
\ {\isadigit{1}}\ {\isaliteral{5B}{\isacharbrackleft}}THEN\ {\isadigit{2}}{\isaliteral{5D}{\isacharbrackright}}%
\begin{isamarkuptxt}%
Symmetric results:%
\end{isamarkuptxt}%
\isamarkuptrue%
\ \ \isacommand{assume}\isamarkupfalse%
\ {\isaliteral{22}{\isachardoublequoteopen}}x\ {\isaliteral{3D}{\isacharequal}}\ y{\isaliteral{22}{\isachardoublequoteclose}}\isanewline
\ \ \isacommand{note}\isamarkupfalse%
\ this\ {\isaliteral{5B}{\isacharbrackleft}}symmetric{\isaliteral{5D}{\isacharbrackright}}\isanewline
\isanewline
\ \ \isacommand{assume}\isamarkupfalse%
\ {\isaliteral{22}{\isachardoublequoteopen}}x\ {\isaliteral{5C3C6E6F7465713E}{\isasymnoteq}}\ y{\isaliteral{22}{\isachardoublequoteclose}}\isanewline
\ \ \isacommand{note}\isamarkupfalse%
\ this\ {\isaliteral{5B}{\isacharbrackleft}}symmetric{\isaliteral{5D}{\isacharbrackright}}%
\begin{isamarkuptxt}%
Adhoc-simplication (take care!):%
\end{isamarkuptxt}%
\isamarkuptrue%
\ \ \isacommand{assume}\isamarkupfalse%
\ {\isaliteral{22}{\isachardoublequoteopen}}P\ {\isaliteral{28}{\isacharparenleft}}{\isaliteral{5B}{\isacharbrackleft}}{\isaliteral{5D}{\isacharbrackright}}\ {\isaliteral{40}{\isacharat}}\ xs{\isaliteral{29}{\isacharparenright}}{\isaliteral{22}{\isachardoublequoteclose}}\isanewline
\ \ \isacommand{note}\isamarkupfalse%
\ this\ {\isaliteral{5B}{\isacharbrackleft}}simplified{\isaliteral{5D}{\isacharbrackright}}%
\endisatagproof
{\isafoldproof}%
%
\isadelimproof
%
\endisadelimproof
\isanewline
\isacommand{end}\isamarkupfalse%
%
\isamarkupsubsubsection{Projections%
}
\isamarkuptrue%
%
\begin{isamarkuptext}%
Isar facts consist of multiple theorems.  There is notation to project
  interval ranges.%
\end{isamarkuptext}%
\isamarkuptrue%
\isacommand{notepad}\isamarkupfalse%
\isanewline
\isakeyword{begin}\isanewline
%
\isadelimproof
\ \ %
\endisadelimproof
%
\isatagproof
\isacommand{assume}\isamarkupfalse%
\ stuff{\isaliteral{3A}{\isacharcolon}}\ A\ B\ C\ D\isanewline
\ \ \isacommand{note}\isamarkupfalse%
\ stuff{\isaliteral{28}{\isacharparenleft}}{\isadigit{1}}{\isaliteral{29}{\isacharparenright}}\isanewline
\ \ \isacommand{note}\isamarkupfalse%
\ stuff{\isaliteral{28}{\isacharparenleft}}{\isadigit{2}}{\isaliteral{2D}{\isacharminus}}{\isadigit{3}}{\isaliteral{29}{\isacharparenright}}\isanewline
\ \ \isacommand{note}\isamarkupfalse%
\ stuff{\isaliteral{28}{\isacharparenleft}}{\isadigit{2}}{\isaliteral{2D}{\isacharminus}}{\isaliteral{29}{\isacharparenright}}%
\endisatagproof
{\isafoldproof}%
%
\isadelimproof
\isanewline
%
\endisadelimproof
\isacommand{end}\isamarkupfalse%
%
\isamarkupsubsubsection{Naming conventions%
}
\isamarkuptrue%
%
\begin{isamarkuptext}%
\begin{itemize}

  \item Lower-case identifiers are usually preferred.

  \item Facts can be named after the main term within the proposition.

  \item Facts should \emph{not} be named after the command that
  introduced them (\hyperlink{command.assume}{\mbox{\isa{\isacommand{assume}}}}, \hyperlink{command.have}{\mbox{\isa{\isacommand{have}}}}).  This is
  misleading and hard to maintain.

  \item Natural numbers can be used as ``meaningless'' names (more
  appropriate than \isa{{\isaliteral{22}{\isachardoublequote}}a{\isadigit{1}}{\isaliteral{22}{\isachardoublequote}}}, \isa{{\isaliteral{22}{\isachardoublequote}}a{\isadigit{2}}{\isaliteral{22}{\isachardoublequote}}} etc.)

  \item Symbolic identifiers are supported (e.g. \isa{{\isaliteral{22}{\isachardoublequote}}{\isaliteral{2A}{\isacharasterisk}}{\isaliteral{22}{\isachardoublequote}}}, \isa{{\isaliteral{22}{\isachardoublequote}}{\isaliteral{2A}{\isacharasterisk}}{\isaliteral{2A}{\isacharasterisk}}{\isaliteral{22}{\isachardoublequote}}}, \isa{{\isaliteral{22}{\isachardoublequote}}{\isaliteral{2A}{\isacharasterisk}}{\isaliteral{2A}{\isacharasterisk}}{\isaliteral{2A}{\isacharasterisk}}{\isaliteral{22}{\isachardoublequote}}}).

  \end{itemize}%
\end{isamarkuptext}%
\isamarkuptrue%
%
\isamarkupsubsection{Block structure%
}
\isamarkuptrue%
%
\begin{isamarkuptext}%
The formal notepad is block structured.  The fact produced by the last
  entry of a block is exported into the outer context.%
\end{isamarkuptext}%
\isamarkuptrue%
\isacommand{notepad}\isamarkupfalse%
\isanewline
\isakeyword{begin}\isanewline
%
\isadelimproof
\ \ %
\endisadelimproof
%
\isatagproof
\isacommand{{\isaliteral{7B}{\isacharbraceleft}}}\isamarkupfalse%
\isanewline
\ \ \ \ \isacommand{have}\isamarkupfalse%
\ a{\isaliteral{3A}{\isacharcolon}}\ A\ \isacommand{sorry}\isamarkupfalse%
\isanewline
\ \ \ \ \isacommand{have}\isamarkupfalse%
\ b{\isaliteral{3A}{\isacharcolon}}\ B\ \isacommand{sorry}\isamarkupfalse%
\isanewline
\ \ \ \ \isacommand{note}\isamarkupfalse%
\ a\ b\isanewline
\ \ \isacommand{{\isaliteral{7D}{\isacharbraceright}}}\isamarkupfalse%
\isanewline
\ \ \isacommand{note}\isamarkupfalse%
\ this\isanewline
\ \ \isacommand{note}\isamarkupfalse%
\ {\isaliteral{60}{\isacharbackquoteopen}}A{\isaliteral{60}{\isacharbackquoteclose}}\isanewline
\ \ \isacommand{note}\isamarkupfalse%
\ {\isaliteral{60}{\isacharbackquoteopen}}B{\isaliteral{60}{\isacharbackquoteclose}}%
\endisatagproof
{\isafoldproof}%
%
\isadelimproof
\isanewline
%
\endisadelimproof
\isacommand{end}\isamarkupfalse%
%
\begin{isamarkuptext}%
Explicit blocks as well as implicit blocks of nested goal
  statements (e.g.\ \hyperlink{command.have}{\mbox{\isa{\isacommand{have}}}}) automatically introduce one extra
  pair of parentheses in reserve.  The \hyperlink{command.next}{\mbox{\isa{\isacommand{next}}}} command allows
  to ``jump'' between these sub-blocks.%
\end{isamarkuptext}%
\isamarkuptrue%
\isacommand{notepad}\isamarkupfalse%
\isanewline
\isakeyword{begin}\isanewline
%
\isadelimproof
\isanewline
\ \ %
\endisadelimproof
%
\isatagproof
\isacommand{{\isaliteral{7B}{\isacharbraceleft}}}\isamarkupfalse%
\isanewline
\ \ \ \ \isacommand{have}\isamarkupfalse%
\ a{\isaliteral{3A}{\isacharcolon}}\ A\ \isacommand{sorry}\isamarkupfalse%
\isanewline
\ \ \isacommand{next}\isamarkupfalse%
\isanewline
\ \ \ \ \isacommand{have}\isamarkupfalse%
\ b{\isaliteral{3A}{\isacharcolon}}\ B\isanewline
\ \ \ \ \isacommand{proof}\isamarkupfalse%
\ {\isaliteral{2D}{\isacharminus}}\isanewline
\ \ \ \ \ \ \isacommand{show}\isamarkupfalse%
\ B\ \isacommand{sorry}\isamarkupfalse%
\isanewline
\ \ \ \ \isacommand{next}\isamarkupfalse%
\isanewline
\ \ \ \ \ \ \isacommand{have}\isamarkupfalse%
\ c{\isaliteral{3A}{\isacharcolon}}\ C\ \isacommand{sorry}\isamarkupfalse%
\isanewline
\ \ \ \ \isacommand{next}\isamarkupfalse%
\isanewline
\ \ \ \ \ \ \isacommand{have}\isamarkupfalse%
\ d{\isaliteral{3A}{\isacharcolon}}\ D\ \isacommand{sorry}\isamarkupfalse%
\isanewline
\ \ \ \ \isacommand{qed}\isamarkupfalse%
\isanewline
\ \ \isacommand{{\isaliteral{7D}{\isacharbraceright}}}\isamarkupfalse%
%
\begin{isamarkuptxt}%
Alternative version with explicit parentheses everywhere:%
\end{isamarkuptxt}%
\isamarkuptrue%
\ \ \isacommand{{\isaliteral{7B}{\isacharbraceleft}}}\isamarkupfalse%
\isanewline
\ \ \ \ \isacommand{{\isaliteral{7B}{\isacharbraceleft}}}\isamarkupfalse%
\isanewline
\ \ \ \ \ \ \isacommand{have}\isamarkupfalse%
\ a{\isaliteral{3A}{\isacharcolon}}\ A\ \isacommand{sorry}\isamarkupfalse%
\isanewline
\ \ \ \ \isacommand{{\isaliteral{7D}{\isacharbraceright}}}\isamarkupfalse%
\isanewline
\ \ \ \ \isacommand{{\isaliteral{7B}{\isacharbraceleft}}}\isamarkupfalse%
\isanewline
\ \ \ \ \ \ \isacommand{have}\isamarkupfalse%
\ b{\isaliteral{3A}{\isacharcolon}}\ B\isanewline
\ \ \ \ \ \ \isacommand{proof}\isamarkupfalse%
\ {\isaliteral{2D}{\isacharminus}}\isanewline
\ \ \ \ \ \ \ \ \isacommand{{\isaliteral{7B}{\isacharbraceleft}}}\isamarkupfalse%
\isanewline
\ \ \ \ \ \ \ \ \ \ \isacommand{show}\isamarkupfalse%
\ B\ \isacommand{sorry}\isamarkupfalse%
\isanewline
\ \ \ \ \ \ \ \ \isacommand{{\isaliteral{7D}{\isacharbraceright}}}\isamarkupfalse%
\isanewline
\ \ \ \ \ \ \ \ \isacommand{{\isaliteral{7B}{\isacharbraceleft}}}\isamarkupfalse%
\isanewline
\ \ \ \ \ \ \ \ \ \ \isacommand{have}\isamarkupfalse%
\ c{\isaliteral{3A}{\isacharcolon}}\ C\ \isacommand{sorry}\isamarkupfalse%
\isanewline
\ \ \ \ \ \ \ \ \isacommand{{\isaliteral{7D}{\isacharbraceright}}}\isamarkupfalse%
\isanewline
\ \ \ \ \ \ \ \ \isacommand{{\isaliteral{7B}{\isacharbraceleft}}}\isamarkupfalse%
\isanewline
\ \ \ \ \ \ \ \ \ \ \isacommand{have}\isamarkupfalse%
\ d{\isaliteral{3A}{\isacharcolon}}\ D\ \isacommand{sorry}\isamarkupfalse%
\isanewline
\ \ \ \ \ \ \ \ \isacommand{{\isaliteral{7D}{\isacharbraceright}}}\isamarkupfalse%
\isanewline
\ \ \ \ \ \ \isacommand{qed}\isamarkupfalse%
\isanewline
\ \ \ \ \isacommand{{\isaliteral{7D}{\isacharbraceright}}}\isamarkupfalse%
\isanewline
\ \ \isacommand{{\isaliteral{7D}{\isacharbraceright}}}\isamarkupfalse%
%
\endisatagproof
{\isafoldproof}%
%
\isadelimproof
\isanewline
%
\endisadelimproof
\isanewline
\isacommand{end}\isamarkupfalse%
%
\isamarkupsection{Calculational reasoning%
}
\isamarkuptrue%
%
\begin{isamarkuptext}%
For example, see \verb|~~/src/HOL/Isar_Examples/Group.thy|.%
\end{isamarkuptext}%
\isamarkuptrue%
%
\isamarkupsubsection{Special names in Isar proofs%
}
\isamarkuptrue%
%
\begin{isamarkuptext}%
\begin{itemize}

  \item term \isa{{\isaliteral{22}{\isachardoublequote}}{\isaliteral{3F}{\isacharquery}}thesis{\isaliteral{22}{\isachardoublequote}}} --- the main conclusion of the
  innermost pending claim

  \item term \isa{{\isaliteral{22}{\isachardoublequote}}{\isaliteral{5C3C646F74733E}{\isasymdots}}{\isaliteral{22}{\isachardoublequote}}} --- the argument of the last explicitly
    stated result (for infix application this is the right-hand side)

  \item fact \isa{{\isaliteral{22}{\isachardoublequote}}this{\isaliteral{22}{\isachardoublequote}}} --- the last result produced in the text

  \end{itemize}%
\end{isamarkuptext}%
\isamarkuptrue%
\isacommand{notepad}\isamarkupfalse%
\isanewline
\isakeyword{begin}\isanewline
%
\isadelimproof
\ \ %
\endisadelimproof
%
\isatagproof
\isacommand{have}\isamarkupfalse%
\ {\isaliteral{22}{\isachardoublequoteopen}}x\ {\isaliteral{3D}{\isacharequal}}\ y{\isaliteral{22}{\isachardoublequoteclose}}\isanewline
\ \ \isacommand{proof}\isamarkupfalse%
\ {\isaliteral{2D}{\isacharminus}}%
\endisatagproof
{\isafoldproof}%
%
\isadelimproof
\isanewline
%
\endisadelimproof
\ \ \ \ \isacommand{term}\isamarkupfalse%
\ {\isaliteral{3F}{\isacharquery}}thesis\isanewline
%
\isadelimproof
\ \ \ \ %
\endisadelimproof
%
\isatagproof
\isacommand{show}\isamarkupfalse%
\ {\isaliteral{3F}{\isacharquery}}thesis\ \isacommand{sorry}\isamarkupfalse%
%
\endisatagproof
{\isafoldproof}%
%
\isadelimproof
\isanewline
%
\endisadelimproof
\ \ \ \ \isacommand{term}\isamarkupfalse%
\ {\isaliteral{3F}{\isacharquery}}thesis\ \ %
\isamarkupcmt{static!%
}
\isanewline
%
\isadelimproof
\ \ %
\endisadelimproof
%
\isatagproof
\isacommand{qed}\isamarkupfalse%
%
\endisatagproof
{\isafoldproof}%
%
\isadelimproof
\isanewline
%
\endisadelimproof
\ \ \isacommand{term}\isamarkupfalse%
\ {\isaliteral{22}{\isachardoublequoteopen}}{\isaliteral{5C3C646F74733E}{\isasymdots}}{\isaliteral{22}{\isachardoublequoteclose}}\isanewline
\ \ \isacommand{thm}\isamarkupfalse%
\ this\isanewline
\isacommand{end}\isamarkupfalse%
%
\begin{isamarkuptext}%
Calculational reasoning maintains the special fact called
  ``\isa{calculation}'' in the background.  Certain language
  elements combine primary \isa{this} with secondary \isa{calculation}.%
\end{isamarkuptext}%
\isamarkuptrue%
%
\isamarkupsubsection{Transitive chains%
}
\isamarkuptrue%
%
\begin{isamarkuptext}%
The Idea is to combine \isa{this} and \isa{calculation}
  via typical \isa{trans} rules (see also \hyperlink{command.print-trans-rules}{\mbox{\isa{\isacommand{print{\isaliteral{5F}{\isacharunderscore}}trans{\isaliteral{5F}{\isacharunderscore}}rules}}}}):%
\end{isamarkuptext}%
\isamarkuptrue%
\isacommand{thm}\isamarkupfalse%
\ trans\isanewline
\isacommand{thm}\isamarkupfalse%
\ less{\isaliteral{5F}{\isacharunderscore}}trans\isanewline
\isacommand{thm}\isamarkupfalse%
\ less{\isaliteral{5F}{\isacharunderscore}}le{\isaliteral{5F}{\isacharunderscore}}trans\isanewline
\isanewline
\isacommand{notepad}\isamarkupfalse%
\isanewline
\isakeyword{begin}%
\isadelimproof
%
\endisadelimproof
%
\isatagproof
%
\begin{isamarkuptxt}%
Plain bottom-up calculation:%
\end{isamarkuptxt}%
\isamarkuptrue%
\ \ \isacommand{have}\isamarkupfalse%
\ {\isaliteral{22}{\isachardoublequoteopen}}a\ {\isaliteral{3D}{\isacharequal}}\ b{\isaliteral{22}{\isachardoublequoteclose}}\ \isacommand{sorry}\isamarkupfalse%
\isanewline
\ \ \isacommand{also}\isamarkupfalse%
\isanewline
\ \ \isacommand{have}\isamarkupfalse%
\ {\isaliteral{22}{\isachardoublequoteopen}}b\ {\isaliteral{3D}{\isacharequal}}\ c{\isaliteral{22}{\isachardoublequoteclose}}\ \isacommand{sorry}\isamarkupfalse%
\isanewline
\ \ \isacommand{also}\isamarkupfalse%
\isanewline
\ \ \isacommand{have}\isamarkupfalse%
\ {\isaliteral{22}{\isachardoublequoteopen}}c\ {\isaliteral{3D}{\isacharequal}}\ d{\isaliteral{22}{\isachardoublequoteclose}}\ \isacommand{sorry}\isamarkupfalse%
\isanewline
\ \ \isacommand{finally}\isamarkupfalse%
\isanewline
\ \ \isacommand{have}\isamarkupfalse%
\ {\isaliteral{22}{\isachardoublequoteopen}}a\ {\isaliteral{3D}{\isacharequal}}\ d{\isaliteral{22}{\isachardoublequoteclose}}\ \isacommand{{\isaliteral{2E}{\isachardot}}}\isamarkupfalse%
%
\begin{isamarkuptxt}%
Variant using the \isa{{\isaliteral{22}{\isachardoublequote}}{\isaliteral{5C3C646F74733E}{\isasymdots}}{\isaliteral{22}{\isachardoublequote}}} abbreviation:%
\end{isamarkuptxt}%
\isamarkuptrue%
\ \ \isacommand{have}\isamarkupfalse%
\ {\isaliteral{22}{\isachardoublequoteopen}}a\ {\isaliteral{3D}{\isacharequal}}\ b{\isaliteral{22}{\isachardoublequoteclose}}\ \isacommand{sorry}\isamarkupfalse%
\isanewline
\ \ \isacommand{also}\isamarkupfalse%
\isanewline
\ \ \isacommand{have}\isamarkupfalse%
\ {\isaliteral{22}{\isachardoublequoteopen}}{\isaliteral{5C3C646F74733E}{\isasymdots}}\ {\isaliteral{3D}{\isacharequal}}\ c{\isaliteral{22}{\isachardoublequoteclose}}\ \isacommand{sorry}\isamarkupfalse%
\isanewline
\ \ \isacommand{also}\isamarkupfalse%
\isanewline
\ \ \isacommand{have}\isamarkupfalse%
\ {\isaliteral{22}{\isachardoublequoteopen}}{\isaliteral{5C3C646F74733E}{\isasymdots}}\ {\isaliteral{3D}{\isacharequal}}\ d{\isaliteral{22}{\isachardoublequoteclose}}\ \isacommand{sorry}\isamarkupfalse%
\isanewline
\ \ \isacommand{finally}\isamarkupfalse%
\isanewline
\ \ \isacommand{have}\isamarkupfalse%
\ {\isaliteral{22}{\isachardoublequoteopen}}a\ {\isaliteral{3D}{\isacharequal}}\ d{\isaliteral{22}{\isachardoublequoteclose}}\ \isacommand{{\isaliteral{2E}{\isachardot}}}\isamarkupfalse%
%
\begin{isamarkuptxt}%
Top-down version with explicit claim at the head:%
\end{isamarkuptxt}%
\isamarkuptrue%
\ \ \isacommand{have}\isamarkupfalse%
\ {\isaliteral{22}{\isachardoublequoteopen}}a\ {\isaliteral{3D}{\isacharequal}}\ d{\isaliteral{22}{\isachardoublequoteclose}}\isanewline
\ \ \isacommand{proof}\isamarkupfalse%
\ {\isaliteral{2D}{\isacharminus}}\isanewline
\ \ \ \ \isacommand{have}\isamarkupfalse%
\ {\isaliteral{22}{\isachardoublequoteopen}}a\ {\isaliteral{3D}{\isacharequal}}\ b{\isaliteral{22}{\isachardoublequoteclose}}\ \isacommand{sorry}\isamarkupfalse%
\isanewline
\ \ \ \ \isacommand{also}\isamarkupfalse%
\isanewline
\ \ \ \ \isacommand{have}\isamarkupfalse%
\ {\isaliteral{22}{\isachardoublequoteopen}}{\isaliteral{5C3C646F74733E}{\isasymdots}}\ {\isaliteral{3D}{\isacharequal}}\ c{\isaliteral{22}{\isachardoublequoteclose}}\ \isacommand{sorry}\isamarkupfalse%
\isanewline
\ \ \ \ \isacommand{also}\isamarkupfalse%
\isanewline
\ \ \ \ \isacommand{have}\isamarkupfalse%
\ {\isaliteral{22}{\isachardoublequoteopen}}{\isaliteral{5C3C646F74733E}{\isasymdots}}\ {\isaliteral{3D}{\isacharequal}}\ d{\isaliteral{22}{\isachardoublequoteclose}}\ \isacommand{sorry}\isamarkupfalse%
\isanewline
\ \ \ \ \isacommand{finally}\isamarkupfalse%
\isanewline
\ \ \ \ \isacommand{show}\isamarkupfalse%
\ {\isaliteral{3F}{\isacharquery}}thesis\ \isacommand{{\isaliteral{2E}{\isachardot}}}\isamarkupfalse%
\isanewline
\ \ \isacommand{qed}\isamarkupfalse%
\isanewline
\isacommand{next}\isamarkupfalse%
%
\begin{isamarkuptxt}%
Mixed inequalities (require suitable base type):%
\end{isamarkuptxt}%
\isamarkuptrue%
\ \ \isacommand{fix}\isamarkupfalse%
\ a\ b\ c\ d\ {\isaliteral{3A}{\isacharcolon}}{\isaliteral{3A}{\isacharcolon}}\ nat\isanewline
\isanewline
\ \ \isacommand{have}\isamarkupfalse%
\ {\isaliteral{22}{\isachardoublequoteopen}}a\ {\isaliteral{3C}{\isacharless}}\ b{\isaliteral{22}{\isachardoublequoteclose}}\ \isacommand{sorry}\isamarkupfalse%
\isanewline
\ \ \isacommand{also}\isamarkupfalse%
\isanewline
\ \ \isacommand{have}\isamarkupfalse%
\ {\isaliteral{22}{\isachardoublequoteopen}}b{\isaliteral{5C3C6C653E}{\isasymle}}\ c{\isaliteral{22}{\isachardoublequoteclose}}\ \isacommand{sorry}\isamarkupfalse%
\isanewline
\ \ \isacommand{also}\isamarkupfalse%
\isanewline
\ \ \isacommand{have}\isamarkupfalse%
\ {\isaliteral{22}{\isachardoublequoteopen}}c\ {\isaliteral{3D}{\isacharequal}}\ d{\isaliteral{22}{\isachardoublequoteclose}}\ \isacommand{sorry}\isamarkupfalse%
\isanewline
\ \ \isacommand{finally}\isamarkupfalse%
\isanewline
\ \ \isacommand{have}\isamarkupfalse%
\ {\isaliteral{22}{\isachardoublequoteopen}}a\ {\isaliteral{3C}{\isacharless}}\ d{\isaliteral{22}{\isachardoublequoteclose}}\ \isacommand{{\isaliteral{2E}{\isachardot}}}\isamarkupfalse%
%
\endisatagproof
{\isafoldproof}%
%
\isadelimproof
%
\endisadelimproof
\isanewline
\isacommand{end}\isamarkupfalse%
%
\isamarkupsubsubsection{Notes%
}
\isamarkuptrue%
%
\begin{isamarkuptext}%
\begin{itemize}

  \item The notion of \isa{trans} rule is very general due to the
  flexibility of Isabelle/Pure rule composition.

  \item User applications may declare there own rules, with some care
  about the operational details of higher-order unification.

  \end{itemize}%
\end{isamarkuptext}%
\isamarkuptrue%
%
\isamarkupsubsection{Degenerate calculations and bigstep reasoning%
}
\isamarkuptrue%
%
\begin{isamarkuptext}%
The Idea is to append \isa{this} to \isa{calculation},
  without rule composition.%
\end{isamarkuptext}%
\isamarkuptrue%
\isacommand{notepad}\isamarkupfalse%
\isanewline
\isakeyword{begin}%
\isadelimproof
%
\endisadelimproof
%
\isatagproof
%
\begin{isamarkuptxt}%
A vacous proof:%
\end{isamarkuptxt}%
\isamarkuptrue%
\ \ \isacommand{have}\isamarkupfalse%
\ A\ \isacommand{sorry}\isamarkupfalse%
\isanewline
\ \ \isacommand{moreover}\isamarkupfalse%
\isanewline
\ \ \isacommand{have}\isamarkupfalse%
\ B\ \isacommand{sorry}\isamarkupfalse%
\isanewline
\ \ \isacommand{moreover}\isamarkupfalse%
\isanewline
\ \ \isacommand{have}\isamarkupfalse%
\ C\ \isacommand{sorry}\isamarkupfalse%
\isanewline
\ \ \isacommand{ultimately}\isamarkupfalse%
\isanewline
\ \ \isacommand{have}\isamarkupfalse%
\ A\ \isakeyword{and}\ B\ \isakeyword{and}\ C\ \isacommand{{\isaliteral{2E}{\isachardot}}}\isamarkupfalse%
\isanewline
\isacommand{next}\isamarkupfalse%
%
\begin{isamarkuptxt}%
Slightly more content (trivial bigstep reasoning):%
\end{isamarkuptxt}%
\isamarkuptrue%
\ \ \isacommand{have}\isamarkupfalse%
\ A\ \isacommand{sorry}\isamarkupfalse%
\isanewline
\ \ \isacommand{moreover}\isamarkupfalse%
\isanewline
\ \ \isacommand{have}\isamarkupfalse%
\ B\ \isacommand{sorry}\isamarkupfalse%
\isanewline
\ \ \isacommand{moreover}\isamarkupfalse%
\isanewline
\ \ \isacommand{have}\isamarkupfalse%
\ C\ \isacommand{sorry}\isamarkupfalse%
\isanewline
\ \ \isacommand{ultimately}\isamarkupfalse%
\isanewline
\ \ \isacommand{have}\isamarkupfalse%
\ {\isaliteral{22}{\isachardoublequoteopen}}A\ {\isaliteral{5C3C616E643E}{\isasymand}}\ B\ {\isaliteral{5C3C616E643E}{\isasymand}}\ C{\isaliteral{22}{\isachardoublequoteclose}}\ \isacommand{by}\isamarkupfalse%
\ blast\isanewline
\isacommand{next}\isamarkupfalse%
%
\begin{isamarkuptxt}%
More ambitous bigstep reasoning involving structured results:%
\end{isamarkuptxt}%
\isamarkuptrue%
\ \ \isacommand{have}\isamarkupfalse%
\ {\isaliteral{22}{\isachardoublequoteopen}}A\ {\isaliteral{5C3C6F723E}{\isasymor}}\ B\ {\isaliteral{5C3C6F723E}{\isasymor}}\ C{\isaliteral{22}{\isachardoublequoteclose}}\ \isacommand{sorry}\isamarkupfalse%
\isanewline
\ \ \isacommand{moreover}\isamarkupfalse%
\isanewline
\ \ \isacommand{{\isaliteral{7B}{\isacharbraceleft}}}\isamarkupfalse%
\ \isacommand{assume}\isamarkupfalse%
\ A\ \isacommand{have}\isamarkupfalse%
\ R\ \isacommand{sorry}\isamarkupfalse%
\ \isacommand{{\isaliteral{7D}{\isacharbraceright}}}\isamarkupfalse%
\isanewline
\ \ \isacommand{moreover}\isamarkupfalse%
\isanewline
\ \ \isacommand{{\isaliteral{7B}{\isacharbraceleft}}}\isamarkupfalse%
\ \isacommand{assume}\isamarkupfalse%
\ B\ \isacommand{have}\isamarkupfalse%
\ R\ \isacommand{sorry}\isamarkupfalse%
\ \isacommand{{\isaliteral{7D}{\isacharbraceright}}}\isamarkupfalse%
\isanewline
\ \ \isacommand{moreover}\isamarkupfalse%
\isanewline
\ \ \isacommand{{\isaliteral{7B}{\isacharbraceleft}}}\isamarkupfalse%
\ \isacommand{assume}\isamarkupfalse%
\ C\ \isacommand{have}\isamarkupfalse%
\ R\ \isacommand{sorry}\isamarkupfalse%
\ \isacommand{{\isaliteral{7D}{\isacharbraceright}}}\isamarkupfalse%
\isanewline
\ \ \isacommand{ultimately}\isamarkupfalse%
\isanewline
\ \ \isacommand{have}\isamarkupfalse%
\ R\ \isacommand{by}\isamarkupfalse%
\ blast\ \ %
\isamarkupcmt{``big-bang integration'' of proof blocks (occasionally fragile)%
}
%
\endisatagproof
{\isafoldproof}%
%
\isadelimproof
%
\endisadelimproof
\isanewline
\isacommand{end}\isamarkupfalse%
\isanewline
%
\isadelimtheory
\isanewline
%
\endisadelimtheory
%
\isatagtheory
\isacommand{end}\isamarkupfalse%
%
\endisatagtheory
{\isafoldtheory}%
%
\isadelimtheory
%
\endisadelimtheory
\end{isabellebody}%
%%% Local Variables:
%%% mode: latex
%%% TeX-master: "root"
%%% End:
