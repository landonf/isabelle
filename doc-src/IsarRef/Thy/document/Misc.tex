%
\begin{isabellebody}%
\def\isabellecontext{Misc}%
%
\isadelimtheory
\isanewline
\isanewline
%
\endisadelimtheory
%
\isatagtheory
\isacommand{theory}\isamarkupfalse%
\ Misc\isanewline
\isakeyword{imports}\ Main\isanewline
\isakeyword{begin}%
\endisatagtheory
{\isafoldtheory}%
%
\isadelimtheory
%
\endisadelimtheory
%
\isamarkupchapter{Other commands%
}
\isamarkuptrue%
%
\isamarkupsection{Inspecting the context%
}
\isamarkuptrue%
%
\begin{isamarkuptext}%
\begin{matharray}{rcl}
    \indexdef{}{command}{print\_commands}\hypertarget{command.print-commands}{\hyperlink{command.print-commands}{\mbox{\isa{\isacommand{print{\isacharunderscore}commands}}}}}\isa{{\isachardoublequote}\isactrlsup {\isacharasterisk}{\isachardoublequote}} & : & \isa{{\isachardoublequote}any\ {\isasymrightarrow}{\isachardoublequote}} \\
    \indexdef{}{command}{print\_theory}\hypertarget{command.print-theory}{\hyperlink{command.print-theory}{\mbox{\isa{\isacommand{print{\isacharunderscore}theory}}}}}\isa{{\isachardoublequote}\isactrlsup {\isacharasterisk}{\isachardoublequote}} & : & \isa{{\isachardoublequote}context\ {\isasymrightarrow}{\isachardoublequote}} \\
    \indexdef{}{command}{print\_methods}\hypertarget{command.print-methods}{\hyperlink{command.print-methods}{\mbox{\isa{\isacommand{print{\isacharunderscore}methods}}}}}\isa{{\isachardoublequote}\isactrlsup {\isacharasterisk}{\isachardoublequote}} & : & \isa{{\isachardoublequote}context\ {\isasymrightarrow}{\isachardoublequote}} \\
    \indexdef{}{command}{print\_attributes}\hypertarget{command.print-attributes}{\hyperlink{command.print-attributes}{\mbox{\isa{\isacommand{print{\isacharunderscore}attributes}}}}}\isa{{\isachardoublequote}\isactrlsup {\isacharasterisk}{\isachardoublequote}} & : & \isa{{\isachardoublequote}context\ {\isasymrightarrow}{\isachardoublequote}} \\
    \indexdef{}{command}{print\_theorems}\hypertarget{command.print-theorems}{\hyperlink{command.print-theorems}{\mbox{\isa{\isacommand{print{\isacharunderscore}theorems}}}}}\isa{{\isachardoublequote}\isactrlsup {\isacharasterisk}{\isachardoublequote}} & : & \isa{{\isachardoublequote}context\ {\isasymrightarrow}{\isachardoublequote}} \\
    \indexdef{}{command}{find\_theorems}\hypertarget{command.find-theorems}{\hyperlink{command.find-theorems}{\mbox{\isa{\isacommand{find{\isacharunderscore}theorems}}}}}\isa{{\isachardoublequote}\isactrlsup {\isacharasterisk}{\isachardoublequote}} & : & \isa{{\isachardoublequote}context\ {\isasymrightarrow}{\isachardoublequote}} \\
    \indexdef{}{command}{find\_consts}\hypertarget{command.find-consts}{\hyperlink{command.find-consts}{\mbox{\isa{\isacommand{find{\isacharunderscore}consts}}}}}\isa{{\isachardoublequote}\isactrlsup {\isacharasterisk}{\isachardoublequote}} & : & \isa{{\isachardoublequote}context\ {\isasymrightarrow}{\isachardoublequote}} \\
    \indexdef{}{command}{thm\_deps}\hypertarget{command.thm-deps}{\hyperlink{command.thm-deps}{\mbox{\isa{\isacommand{thm{\isacharunderscore}deps}}}}}\isa{{\isachardoublequote}\isactrlsup {\isacharasterisk}{\isachardoublequote}} & : & \isa{{\isachardoublequote}context\ {\isasymrightarrow}{\isachardoublequote}} \\
    \indexdef{}{command}{print\_facts}\hypertarget{command.print-facts}{\hyperlink{command.print-facts}{\mbox{\isa{\isacommand{print{\isacharunderscore}facts}}}}}\isa{{\isachardoublequote}\isactrlsup {\isacharasterisk}{\isachardoublequote}} & : & \isa{{\isachardoublequote}context\ {\isasymrightarrow}{\isachardoublequote}} \\
    \indexdef{}{command}{print\_binds}\hypertarget{command.print-binds}{\hyperlink{command.print-binds}{\mbox{\isa{\isacommand{print{\isacharunderscore}binds}}}}}\isa{{\isachardoublequote}\isactrlsup {\isacharasterisk}{\isachardoublequote}} & : & \isa{{\isachardoublequote}context\ {\isasymrightarrow}{\isachardoublequote}} \\
  \end{matharray}

  \begin{rail}
    'print\_theory' ( '!'?)
    ;

    'find\_theorems' (('(' (nat)? ('with\_dups')? ')')?) (thmcriterion *)
    ;
    thmcriterion: ('-'?) ('name' ':' nameref | 'intro' | 'elim' | 'dest' |
      'solves' | 'simp' ':' term | term)
    ;
    'find\_consts' (constcriterion +)
    ;
    constcriterion: ('-'?) ('name' ':' nameref | 'strict' ':' type | type)
    ;
    'thm\_deps' thmrefs
    ;
  \end{rail}

  These commands print certain parts of the theory and proof context.
  Note that there are some further ones available, such as for the set
  of rules declared for simplifications.

  \begin{description}
  
  \item \hyperlink{command.print-commands}{\mbox{\isa{\isacommand{print{\isacharunderscore}commands}}}} prints Isabelle's outer theory
  syntax, including keywords and command.
  
  \item \hyperlink{command.print-theory}{\mbox{\isa{\isacommand{print{\isacharunderscore}theory}}}} prints the main logical content of
  the theory context; the ``\isa{{\isachardoublequote}{\isacharbang}{\isachardoublequote}}'' option indicates extra
  verbosity.

  \item \hyperlink{command.print-methods}{\mbox{\isa{\isacommand{print{\isacharunderscore}methods}}}} prints all proof methods
  available in the current theory context.
  
  \item \hyperlink{command.print-attributes}{\mbox{\isa{\isacommand{print{\isacharunderscore}attributes}}}} prints all attributes
  available in the current theory context.
  
  \item \hyperlink{command.print-theorems}{\mbox{\isa{\isacommand{print{\isacharunderscore}theorems}}}} prints theorems resulting from
  the last command.
  
  \item \hyperlink{command.find-theorems}{\mbox{\isa{\isacommand{find{\isacharunderscore}theorems}}}}~\isa{criteria} retrieves facts
  from the theory or proof context matching all of given search
  criteria.  The criterion \isa{{\isachardoublequote}name{\isacharcolon}\ p{\isachardoublequote}} selects all theorems
  whose fully qualified name matches pattern \isa{p}, which may
  contain ``\isa{{\isachardoublequote}{\isacharasterisk}{\isachardoublequote}}'' wildcards.  The criteria \isa{intro},
  \isa{elim}, and \isa{dest} select theorems that match the
  current goal as introduction, elimination or destruction rules,
  respectively.  The criteria \isa{{\isachardoublequote}solves{\isachardoublequote}} returns all rules
  that would directly solve the current goal.  The criterion
  \isa{{\isachardoublequote}simp{\isacharcolon}\ t{\isachardoublequote}} selects all rewrite rules whose left-hand side
  matches the given term.  The criterion term \isa{t} selects all
  theorems that contain the pattern \isa{t} -- as usual, patterns
  may contain occurrences of the dummy ``\isa{{\isacharunderscore}}'', schematic
  variables, and type constraints.
  
  Criteria can be preceded by ``\isa{{\isachardoublequote}{\isacharminus}{\isachardoublequote}}'' to select theorems that
  do \emph{not} match. Note that giving the empty list of criteria
  yields \emph{all} currently known facts.  An optional limit for the
  number of printed facts may be given; the default is 40.  By
  default, duplicates are removed from the search result. Use
  \isa{with{\isacharunderscore}dups} to display duplicates.

  \item \hyperlink{command.find-consts}{\mbox{\isa{\isacommand{find{\isacharunderscore}consts}}}}~\isa{criteria} prints all constants
  whose type meets all of the given criteria. The criterion \isa{{\isachardoublequote}strict{\isacharcolon}\ ty{\isachardoublequote}} is met by any type that matches the type pattern
  \isa{ty}.  Patterns may contain both the dummy type ``\isa{{\isacharunderscore}}''
  and sort constraints. The criterion \isa{ty} is similar, but it
  also matches against subtypes. The criterion \isa{{\isachardoublequote}name{\isacharcolon}\ p{\isachardoublequote}} and
  the prefix ``\isa{{\isachardoublequote}{\isacharminus}{\isachardoublequote}}'' function as described for \hyperlink{command.find-theorems}{\mbox{\isa{\isacommand{find{\isacharunderscore}theorems}}}}.

  \item \hyperlink{command.thm-deps}{\mbox{\isa{\isacommand{thm{\isacharunderscore}deps}}}}~\isa{{\isachardoublequote}a\isactrlsub {\isadigit{1}}\ {\isasymdots}\ a\isactrlsub n{\isachardoublequote}}
  visualizes dependencies of facts, using Isabelle's graph browser
  tool (see also \cite{isabelle-sys}).
  
  \item \hyperlink{command.print-facts}{\mbox{\isa{\isacommand{print{\isacharunderscore}facts}}}} prints all local facts of the
  current context, both named and unnamed ones.
  
  \item \hyperlink{command.print-binds}{\mbox{\isa{\isacommand{print{\isacharunderscore}binds}}}} prints all term abbreviations
  present in the context.

  \end{description}%
\end{isamarkuptext}%
\isamarkuptrue%
%
\isamarkupsection{History commands \label{sec:history}%
}
\isamarkuptrue%
%
\begin{isamarkuptext}%
\begin{matharray}{rcl}
    \indexdef{}{command}{undo}\hypertarget{command.undo}{\hyperlink{command.undo}{\mbox{\isa{\isacommand{undo}}}}}^{{ * }{ * }} & : & \isa{{\isachardoublequote}any\ {\isasymrightarrow}\ any{\isachardoublequote}} \\
    \indexdef{}{command}{linear\_undo}\hypertarget{command.linear-undo}{\hyperlink{command.linear-undo}{\mbox{\isa{\isacommand{linear{\isacharunderscore}undo}}}}}^{{ * }{ * }} & : & \isa{{\isachardoublequote}any\ {\isasymrightarrow}\ any{\isachardoublequote}} \\
    \indexdef{}{command}{kill}\hypertarget{command.kill}{\hyperlink{command.kill}{\mbox{\isa{\isacommand{kill}}}}}^{{ * }{ * }} & : & \isa{{\isachardoublequote}any\ {\isasymrightarrow}\ any{\isachardoublequote}} \\
  \end{matharray}

  The Isabelle/Isar top-level maintains a two-stage history, for
  theory and proof state transformation.  Basically, any command can
  be undone using \hyperlink{command.undo}{\mbox{\isa{\isacommand{undo}}}}, excluding mere diagnostic
  elements.  Note that a theorem statement with a \emph{finished}
  proof is treated as a single unit by \hyperlink{command.undo}{\mbox{\isa{\isacommand{undo}}}}.  In
  contrast, the variant \hyperlink{command.linear-undo}{\mbox{\isa{\isacommand{linear{\isacharunderscore}undo}}}} admits to step back
  into the middle of a proof.  The \hyperlink{command.kill}{\mbox{\isa{\isacommand{kill}}}} command aborts
  the current history node altogether, discontinuing a proof or even
  the whole theory.  This operation is \emph{not} undo-able.

  \begin{warn}
    History commands should never be used with user interfaces such as
    Proof~General \cite{proofgeneral,Aspinall:TACAS:2000}, which takes
    care of stepping forth and back itself.  Interfering by manual
    \hyperlink{command.undo}{\mbox{\isa{\isacommand{undo}}}}, \hyperlink{command.linear-undo}{\mbox{\isa{\isacommand{linear{\isacharunderscore}undo}}}}, or even \hyperlink{command.kill}{\mbox{\isa{\isacommand{kill}}}} commands would quickly result in utter confusion.
  \end{warn}%
\end{isamarkuptext}%
\isamarkuptrue%
%
\isamarkupsection{System commands%
}
\isamarkuptrue%
%
\begin{isamarkuptext}%
\begin{matharray}{rcl}
    \indexdef{}{command}{cd}\hypertarget{command.cd}{\hyperlink{command.cd}{\mbox{\isa{\isacommand{cd}}}}}\isa{{\isachardoublequote}\isactrlsup {\isacharasterisk}{\isachardoublequote}} & : & \isa{{\isachardoublequote}any\ {\isasymrightarrow}{\isachardoublequote}} \\
    \indexdef{}{command}{pwd}\hypertarget{command.pwd}{\hyperlink{command.pwd}{\mbox{\isa{\isacommand{pwd}}}}}\isa{{\isachardoublequote}\isactrlsup {\isacharasterisk}{\isachardoublequote}} & : & \isa{{\isachardoublequote}any\ {\isasymrightarrow}{\isachardoublequote}} \\
    \indexdef{}{command}{use\_thy}\hypertarget{command.use-thy}{\hyperlink{command.use-thy}{\mbox{\isa{\isacommand{use{\isacharunderscore}thy}}}}}\isa{{\isachardoublequote}\isactrlsup {\isacharasterisk}{\isachardoublequote}} & : & \isa{{\isachardoublequote}any\ {\isasymrightarrow}{\isachardoublequote}} \\
  \end{matharray}

  \begin{rail}
    ('cd' | 'use\_thy' | 'update\_thy') name
    ;
  \end{rail}

  \begin{description}

  \item \hyperlink{command.cd}{\mbox{\isa{\isacommand{cd}}}}~\isa{path} changes the current directory
  of the Isabelle process.

  \item \hyperlink{command.pwd}{\mbox{\isa{\isacommand{pwd}}}} prints the current working directory.

  \item \hyperlink{command.use-thy}{\mbox{\isa{\isacommand{use{\isacharunderscore}thy}}}}~\isa{A} preload theory \isa{A}.
  These system commands are scarcely used when working interactively,
  since loading of theories is done automatically as required.

  \end{description}%
\end{isamarkuptext}%
\isamarkuptrue%
%
\isadelimtheory
%
\endisadelimtheory
%
\isatagtheory
\isacommand{end}\isamarkupfalse%
%
\endisatagtheory
{\isafoldtheory}%
%
\isadelimtheory
%
\endisadelimtheory
\isanewline
\end{isabellebody}%
%%% Local Variables:
%%% mode: latex
%%% TeX-master: "root"
%%% End:
