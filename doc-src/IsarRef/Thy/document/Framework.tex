%
\begin{isabellebody}%
\def\isabellecontext{Framework}%
%
\isadelimtheory
%
\endisadelimtheory
%
\isatagtheory
\isacommand{theory}\isamarkupfalse%
\ Framework\isanewline
\isakeyword{imports}\ Main\isanewline
\isakeyword{begin}%
\endisatagtheory
{\isafoldtheory}%
%
\isadelimtheory
%
\endisadelimtheory
%
\isamarkupchapter{The Isabelle/Isar Framework \label{ch:isar-framework}%
}
\isamarkuptrue%
%
\begin{isamarkuptext}%
Isabelle/Isar
  \cite{Wenzel:1999:TPHOL,Wenzel-PhD,Nipkow-TYPES02,Wenzel-Paulson:2006,Wenzel:2006:Festschrift}
  is intended as a generic framework for developing formal
  mathematical documents with full proof checking.  Definitions and
  proofs are organized as theories; an assembly of theory sources may
  be presented as a printed document; see also
  \chref{ch:document-prep}.

  The main objective of Isar is the design of a human-readable
  structured proof language, which is called the ``primary proof
  format'' in Isar terminology.  Such a primary proof language is
  somewhere in the middle between the extremes of primitive proof
  objects and actual natural language.  In this respect, Isar is a bit
  more formalistic than Mizar
  \cite{Trybulec:1993:MizarFeatures,Rudnicki:1992:MizarOverview,Wiedijk:1999:Mizar},
  using logical symbols for certain reasoning schemes where Mizar
  would prefer English words; see \cite{Wenzel-Wiedijk:2002} for
  further comparisons of these systems.

  So Isar challenges the traditional way of recording informal proofs
  in mathematical prose, as well as the common tendency to see fully
  formal proofs directly as objects of some logical calculus (e.g.\
  \isa{{\isachardoublequote}{\isasymlambda}{\isachardoublequote}}-terms in a version of type theory).  In fact, Isar is
  better understood as an interpreter of a simple block-structured
  language for describing data flow of local facts and goals,
  interspersed with occasional invocations of proof methods.
  Everything is reduced to logical inferences internally, but these
  steps are somewhat marginal compared to the overall bookkeeping of
  the interpretation process.  Thanks to careful design of the syntax
  and semantics of Isar language elements, a formal record of Isar
  instructions may later appear as an intelligible text to the
  attentive reader.

  The Isar proof language has emerged from careful analysis of some
  inherent virtues of the existing logical framework of Isabelle/Pure
  \cite{paulson-found,paulson700}, notably composition of higher-order
  natural deduction rules, which is a generalization of Gentzen's
  original calculus \cite{Gentzen:1935}.  The approach of generic
  inference systems in Pure is continued by Isar towards actual proof
  texts.

  Concrete applications require another intermediate layer: an
  object-logic.  Isabelle/HOL \cite{isa-tutorial} (simply-typed
  set-theory) is being used most of the time; Isabelle/ZF
  \cite{isabelle-ZF} is less extensively developed, although it would
  probably fit better for classical mathematics.%
\end{isamarkuptext}%
\isamarkuptrue%
%
\isadelimtheory
%
\endisadelimtheory
%
\isatagtheory
\isacommand{end}\isamarkupfalse%
%
\endisatagtheory
{\isafoldtheory}%
%
\isadelimtheory
%
\endisadelimtheory
\isanewline
\end{isabellebody}%
%%% Local Variables:
%%% mode: latex
%%% TeX-master: "root"
%%% End:
