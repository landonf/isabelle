
\chapter{Introduction}

\section{Quick start}

\subsection{Terminal sessions}

Isar is already part of Isabelle (as of version Isabelle99, or later).  The
\texttt{isabelle} binary provides option \texttt{-I} to run the Isar
interaction loop at startup, rather than the plain ML top-level.  Thus the
quickest way to do \emph{anything} with Isabelle/Isar is as follows:
\begin{ttbox}
isabelle -I HOL\medskip
\out{> Welcome to Isabelle/HOL (Isabelle99)}\medskip
theory Foo = Main:
constdefs foo :: nat  "foo == 1";
lemma "0 < foo" by (simp add: foo_def);
end
\end{ttbox}
Note that any Isabelle/Isar command may be retracted by \texttt{undo}.  The
\texttt{help} command prints a list of available language elements.


\subsection{The Proof~General interface}

Plain TTY-based interaction as above used to be quite feasible with
traditional tactic based theorem proving, but developing Isar documents really
demands some better user-interface support.  David Aspinall's
\emph{Proof~General}\index{Proof General} environment
\cite{proofgeneral,Aspinall:TACAS:2000} offers a generic Emacs interface for
interactive theorem provers that does all the cut-and-paste and
forward-backward walk through the text in a very neat way.  In Isabelle/Isar,
the current position within a partial proof document is equally important than
the actual proof state.  Thus Proof~General provides the canonical working
environment for Isabelle/Isar, both for getting acquainted (e.g.\ by replaying
existing Isar documents) and for production work.

\medskip

The easiest way to use Proof~General is to make it the default Isabelle user
interface (see also \cite{isabelle-sys}).  Just put something like this into
your Isabelle settings file:
\begin{ttbox}
ISABELLE_INTERFACE=\$ISABELLE_HOME/contrib/ProofGeneral/isar/interface
PROOFGENERAL_OPTIONS=""
\end{ttbox}
You may have to change \texttt{\$ISABELLE_HOME/contrib/ProofGeneral} to the
actual installation directory of Proof~General.  From now on, the capital
\texttt{Isabelle} executable refers to the \texttt{ProofGeneral/isar}
interface.\footnote{There is also a \texttt{ProofGeneral/isa} interface, for
  classic Isabelle tactic scripts.}

The interface script provides several options, just pass \verb,-?, to see its
usage.  Apart from the command line, the defaults for these options may be
overridden via the \texttt{PROOFGENERAL_OPTIONS} setting as well.  For
example, plain FSF Emacs (instead of the default XEmacs) may be configured in
Isabelle's settings via \texttt{PROOFGENERAL_OPTIONS="-p emacs"}.

Occasionally, the user's \verb,~/.emacs, file contains material that is
incompatible with the version of Emacs that Proof~General prefers.  Then
proper startup may be still achieved by using the \texttt{-u false} option.
Also note that any Emacs lisp file called \texttt{proofgeneral-settings.el}
occurring in \texttt{\$ISABELLE_HOME/etc} or \texttt{\$ISABELLE_HOME_USER/etc}
is automatically loaded by the Proof~General interface script as well.

\medskip

With the proper Isabelle interface setup, Isar documents may now be edited by
visiting appropriate theory files, e.g.\ 
\begin{ttbox}
Isabelle \({\langle}isabellehome{\rangle}\)/src/HOL/Isar_examples/BasicLogic.thy
\end{ttbox}
Users of XEmacs may note the tool bar for navigating forward and backward
through the text.  Consult the Proof~General documentation \cite{proofgeneral}
for further basic command sequences, such as ``\texttt{C-c C-return}'' or
``\texttt{C-c u}''.

\medskip

Proof~General also supports the Emacs X-Symbol package \cite{x-symbol}, which
provides a nice way to get proper mathematical symbols displayed on screen.
Just pass option \texttt{-x true} to the Isabelle interface script, or check
the appropriate menu setting by hand.  In any case, the X-Symbol package must
have been properly installed already.

Note that using actual mathematical symbols in the text easily makes the ASCII
sources hard to read.  For example, $\forall$ will appear as \verb,\\<forall>,
according the default set of Isabelle symbols.  On the other hand, the
Isabelle document preparation system \cite{isabelle-sys} will be happy to
print non-ASCII symbols properly.  It is even possible to invent additional
notation beyond the display capabilities of XEmacs and X-Symbol.


\section{Isabelle/Isar theories}

Isabelle/Isar offers the following main improvements over classic Isabelle.
\begin{enumerate}
\item A new \emph{theory format}, occasionally referred to as ``new-style
  theories'', supporting interactive development and unlimited undo operation.
\item A \emph{formal proof document language} designed to support intelligible
  semi-automated reasoning.  Instead of putting together unreadable tactic
  scripts, the author is enabled to express the reasoning in way that is close
  to usual mathematical practice.
\item A simple document preparation system, for typesetting formal
  developments together with informal text.  The resulting hyper-linked PDF
  documents are equally well suited for WWW presentation and as printed
  copies.
\end{enumerate}

The Isar proof language is embedded into the new theory format as a proper
sub-language.  Proof mode is entered by stating some $\THEOREMNAME$ or
$\LEMMANAME$ at the theory level, and left again with the final conclusion
(e.g.\ via $\QEDNAME$).  A few theory extension mechanisms require proof as
well, such as HOL's $\isarkeyword{typedef}$ which demands non-emptiness of the
representing sets.

New-style theory files may still be associated with separate ML files
consisting of plain old tactic scripts.  There is no longer any ML binding
generated for the theory and theorems, though.  ML functions \texttt{theory},
\texttt{thm}, and \texttt{thms} retrieve this information \cite{isabelle-ref}.
Nevertheless, migration between classic Isabelle and Isabelle/Isar is
relatively easy.  Thus users may start to benefit from interactive theory
development and document preparation, even before they have any idea of the
Isar proof language at all.

\begin{warn}
  Currently, Proof~General does \emph{not} support mixed interactive
  development of classic Isabelle theory files or tactic scripts, together
  with Isar documents.  The ``\texttt{isa}'' and ``\texttt{isar}'' versions of
  Proof~General are handled as two different theorem proving systems, only one
  of these may be active at the same time.
\end{warn}

Porting of existing tactic scripts is best done by running two separate
Proof~General sessions, one for replaying the old script and the other for the
emerging Isabelle/Isar document.


\subsection{Document preparation}

Isabelle/Isar provides a simple document preparation system based on current
(PDF) {\LaTeX} technology, with full support of hyper-links (both local
references and URLs), bookmarks, thumbnails etc.  Thus the results are equally
well suited for WWW browsing and as printed copies.

\medskip

Isabelle generates {\LaTeX} output as part of the run of a \emph{logic
  session} (see also \cite{isabelle-sys}).  Getting started with a working
configuration for common situations is quite easy by using the Isabelle
\texttt{mkdir} and \texttt{make} tools.  Just invoke
\begin{ttbox}
  isatool mkdir -d Foo
\end{ttbox}
to setup a separate directory for session \texttt{Foo}.\footnote{It is safe to
  experiment, since \texttt{isatool mkdir} never overwrites existing files.}
Ensure that \texttt{Foo/ROOT.ML} loads all theories required for this session.
Furthermore \texttt{Foo/document/root.tex} should include any special {\LaTeX}
macro packages required for your document (the default is usually sufficient
as a start).

The session is controlled by a separate \texttt{IsaMakefile} (with very crude
source dependencies only by default).  This file is located one level up from
the \texttt{Foo} directory location.  At that point just invoke
\begin{ttbox}
  isatool make Foo
\end{ttbox}
to run the \texttt{Foo} session, with browser information and document
preparation enabled.  Unless any errors are reported by Isabelle or {\LaTeX},
the output will appear inside the directory indicated by \texttt{isatool
  getenv ISABELLE_BROWSER_INFO}, with the logical session prefix added (e.g.\ 
\texttt{HOL/Foo}).  Note that the \texttt{index.html} located there provides a
link to the finished {\LaTeX} document, too.

Note that this really is batch processing --- better let Isabelle check your
theory and proof developments beforehand in interactive mode.

\medskip

You may also consider to tune the \texttt{usedir} options in
\texttt{IsaMakefile}, for example to change the output format from
\texttt{dvi} to \texttt{pdf}, or activate the \texttt{-D document} option in
order to preserve a copy of the generated {\LaTeX} sources.  The latter
feature is very useful for debugging {\LaTeX} errors, while avoiding repeated
runs of Isabelle.

\medskip

See \emph{The Isabelle System Manual} \cite{isabelle-sys} for further details
on Isabelle logic sessions and theory presentation.


\subsection{How to write Isar proofs anyway?}

This is one of the key questions, of course.  Isar offers a rather different
approach to formal proof documents than plain old tactic scripts.  Experienced
users of existing interactive theorem proving systems may have to learn
thinking differently in order to make effective use of Isabelle/Isar.  On the
other hand, Isabelle/Isar comes much closer to existing mathematical practice
of formal proof, so users with less experience in old-style tactical proving,
but a good understanding of mathematical proof, might cope with Isar even
better.  See also \cite{Wenzel:1999:TPHOL} for further background information
on Isar.
%FIXME cite [HahnBanach-in-Isar]

\medskip This really is a \emph{reference manual}.  Nevertheless, we will also
give some clues of how the concepts introduced here may be put into practice.
Appendix~\ref{ap:refcard} provides a quick reference card of the most common
Isabelle/Isar language elements.  There are several examples distributed with
Isabelle, and available via the Isabelle WWW library as well as the
Isabelle/Isar page:
\begin{center}\small
  \begin{tabular}{l}
    \url{http://www.cl.cam.ac.uk/Research/HVG/Isabelle/library/} \\
    \url{http://isabelle.in.tum.de/library/} \\[1ex]
    \url{http://isabelle.in.tum.de/Isar/} \\
  \end{tabular}
\end{center}

See \texttt{HOL/Isar_examples} for a collection of introductory examples, and
\texttt{HOL/HOL-Real/HahnBanach} for a big mathematics application.  Apart
from plain HTML sources, these sessions also provide actual documents (in
PDF).


%%% Local Variables: 
%%% mode: latex
%%% TeX-master: "isar-ref"
%%% End: 
