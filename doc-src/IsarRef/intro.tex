
\chapter{Introduction}

\section{Overview}

The \emph{Isabelle} system essentially provides a generic infrastructure for
building deductive systems (programmed in Standard ML), with a special focus
on interactive theorem proving in higher-order logics.  In the olden days even
end-users would refer to certain ML functions (goal commands, tactics,
tacticals etc.) to pursue their everyday theorem proving tasks
\cite{isabelle-intro,isabelle-ref}.
  
In contrast \emph{Isar} provides an interpreted language environment of its
own, which has been specifically tailored for the needs of theory and proof
development.  Compared to raw ML, the Isabelle/Isar top-level provides a more
robust and comfortable development platform, with proper support for theory
development graphs, single-step transactions with unlimited undo, etc.  The
Isabelle/Isar version of the \emph{Proof~General} user interface
\cite{proofgeneral,Aspinall:TACAS:2000} provides an adequate front-end for
interactive theory and proof development in this advanced theorem proving
environment.

\medskip Apart from the technical advances over bare-bones ML programming, the
main purpose of the Isar language is to provide a conceptually different view
on machine-checked proofs \cite{Wenzel:1999:TPHOL,Wenzel-PhD}.  ``Isar''
stands for ``Intelligible semi-automated reasoning''.  Drawing from both the
traditions of informal mathematical proof texts and high-level programming
languages, Isar offers a versatile environment for structured formal proof
documents.  Thus properly written Isar proofs become accessible to a broader
audience than unstructured tactic scripts (which typically only provide
operational information for the machine).  Writing human-readable proof texts
certainly requires some additional efforts by the writer to achieve a good
presentation, both of formal and informal parts of the text.  On the other
hand, human-readable formal texts gain some value in their own right,
independently of the mechanic proof-checking process.

Despite its grand design of structured proof texts, Isar is able to assimilate
the old tactical style as an ``improper'' sub-language.  This provides an easy
upgrade path for existing tactic scripts, as well as additional means for
interactive experimentation and debugging of structured proofs.  Isabelle/Isar
supports a broad range of proof styles, both readable and unreadable ones.

\medskip The Isabelle/Isar framework is generic and should work reasonably
well for any Isabelle object-logic that conforms to the natural deduction view
of the Isabelle/Pure framework.  Major Isabelle logics like HOL
\cite{isabelle-HOL}, HOLCF \cite{MuellerNvOS99}, FOL \cite{isabelle-logics},
and ZF \cite{isabelle-ZF} have already been set up for end-users.
Nonetheless, much of the existing body of theories still consist of old-style
theory files with accompanied ML code for proof scripts; this legacy will be
gradually converted in due time.


\section{Quick start}

\subsection{Terminal sessions}

Isar is already part of Isabelle.  The low-level \texttt{isabelle} binary
provides option \texttt{-I} to run the Isabelle/Isar interaction loop at
startup, rather than the raw ML top-level.  So the most basic way to do
anything with Isabelle/Isar is as follows:
\begin{ttbox}
isabelle -I HOL\medskip
\out{> Welcome to Isabelle/HOL (Isabelle2005)}\medskip
theory Foo imports Main begin;
constdefs foo :: nat  "foo == 1";
lemma "0 < foo" by (simp add: foo_def);
end;
\end{ttbox}
Note that any Isabelle/Isar command may be retracted by \texttt{undo}.  See
the Isabelle/Isar Quick Reference (appendix~\ref{ap:refcard}) for a
comprehensive overview of available commands and other language elements.


\subsection{Proof~General}

Plain TTY-based interaction as above used to be quite feasible with
traditional tactic based theorem proving, but developing Isar documents really
demands some better user-interface support.  The Proof~General environment by
David Aspinall \cite{proofgeneral,Aspinall:TACAS:2000} offers a generic Emacs
interface for interactive theorem provers that organizes all the cut-and-paste
and forward-backward walk through the text in a very neat way.  In
Isabelle/Isar, the current position within a partial proof document is equally
important than the actual proof state.  Thus Proof~General provides the
canonical working environment for Isabelle/Isar, both for getting acquainted
(e.g.\ by replaying existing Isar documents) and for production work.


\subsubsection{Proof~General as default Isabelle interface}

The Isabelle interface wrapper script provides an easy way to invoke
Proof~General (including XEmacs or GNU Emacs).  The default configuration of
Isabelle is smart enough to detect the Proof~General distribution in several
canonical places (e.g.\ \texttt{\$ISABELLE_HOME/contrib/ProofGeneral}).  Thus
the capital \texttt{Isabelle} executable would already refer to the
\texttt{ProofGeneral/isar} interface without further ado.\footnote{There is
  also a \texttt{ProofGeneral/isa} interface for old tactic scripts written in
  ML.} The Isabelle interface script provides several options; pass \verb,-?,
to see its usage.

With the proper Isabelle interface setup, Isar documents may now be edited by
visiting appropriate theory files, e.g.\ 
\begin{ttbox}
Isabelle \({\langle}isabellehome{\rangle}\)/src/HOL/Isar_examples/Summation.thy
\end{ttbox}
Beginners may note the tool bar for navigating forward and backward through
the text (this depends on the local Emacs installation).  Consult the
Proof~General documentation \cite{proofgeneral} for further basic command
sequences, in particular ``\texttt{C-c C-return}'' and ``\texttt{C-c u}''.

\medskip

Proof~General may be also configured manually by giving Isabelle settings like
this (see also \cite{isabelle-sys}):
\begin{ttbox}
ISABELLE_INTERFACE=\$ISABELLE_HOME/contrib/ProofGeneral/isar/interface
PROOFGENERAL_OPTIONS=""
\end{ttbox}
You may have to change \texttt{\$ISABELLE_HOME/contrib/ProofGeneral} to the
actual installation directory of Proof~General.

\medskip

Apart from the Isabelle command line, defaults for interface options may be
given by the \texttt{PROOFGENERAL_OPTIONS} setting.  For example, the Emacs
executable to be used may be configured in Isabelle's settings like this:
\begin{ttbox}
PROOFGENERAL_OPTIONS="-p xemacs-mule"  
\end{ttbox}

Occasionally, a user's \verb,~/.emacs, file contains code that is incompatible
with the (X)Emacs version used by Proof~General, causing the interface startup
to fail prematurely.  Here the \texttt{-u false} option helps to get the
interface process up and running.  Note that additional Lisp customization
code may reside in \texttt{proofgeneral-settings.el} of
\texttt{\$ISABELLE_HOME/etc} or \texttt{\$ISABELLE_HOME_USER/etc}.


\subsubsection{The X-Symbol package}

Proof~General provides native support for the Emacs X-Symbol package
\cite{x-symbol}, which handles proper mathematical symbols displayed on
screen.  Pass option \texttt{-x true} to the Isabelle interface script, or
check the appropriate Proof~General menu setting by hand.  In any case, the
X-Symbol package must have been properly installed already.

Contrary to what you may expect from the documentation of X-Symbol, the
package is very easy to install and configures itself automatically.  The
default configuration of Isabelle is smart enough to detect the X-Symbol
package in several canonical places (e.g.\ 
\texttt{\$ISABELLE_HOME/contrib/x-symbol}).

\medskip

Using proper mathematical symbols in Isabelle theories can be very convenient
for readability of large formulas.  On the other hand, the plain ASCII sources
easily become somewhat unintelligible.  For example, $\Longrightarrow$ would
appear as \verb,\<Longrightarrow>, according the default set of Isabelle
symbols.  Nevertheless, the Isabelle document preparation system (see
\S\ref{sec:document-prep}) will be happy to print non-ASCII symbols properly.
It is even possible to invent additional notation beyond the display
capabilities of Emacs and X-Symbol.


\section{Isabelle/Isar theories}

Isabelle/Isar offers the following main improvements over classic Isabelle.
\begin{enumerate}
  
\item A new \emph{theory format}, occasionally referred to as ``new-style
  theories'', supporting interactive development and unlimited undo operation.
  
\item A \emph{formal proof document language} designed to support intelligible
  semi-automated reasoning.  Instead of putting together unreadable tactic
  scripts, the author is enabled to express the reasoning in way that is close
  to usual mathematical practice.  The old tactical style has been assimilated
  as ``improper'' language elements.
  
\item A simple document preparation system, for typesetting formal
  developments together with informal text.  The resulting hyper-linked PDF
  documents are equally well suited for WWW presentation and as printed
  copies.

\end{enumerate}

The Isar proof language is embedded into the new theory format as a proper
sub-language.  Proof mode is entered by stating some $\THEOREMNAME$ or
$\LEMMANAME$ at the theory level, and left again with the final conclusion
(e.g.\ via $\QEDNAME$).  A few theory specification mechanisms also require
some proof, such as HOL's $\isarkeyword{typedef}$ which demands non-emptiness
of the representing sets.

New-style theory files may still be associated with separate ML files
consisting of plain old tactic scripts.  There is no longer any ML binding
generated for the theory and theorems, though.  ML functions \texttt{theory},
\texttt{thm}, and \texttt{thms} retrieve this information from the context
\cite{isabelle-ref}.  Nevertheless, migration between classic Isabelle and
Isabelle/Isar is relatively easy.  Thus users may start to benefit from
interactive theory development and document preparation, even before they have
any idea of the Isar proof language at all.

\begin{warn}
  Proof~General does \emph{not} support mixed interactive development of
  classic Isabelle theory files or tactic scripts, together with Isar
  documents.  The ``\texttt{isa}'' and ``\texttt{isar}'' versions of
  Proof~General are handled as two different theorem proving systems, only one
  of these may be active at the same time.
\end{warn}

Manual conversion of existing tactic scripts may be done by running two
separate Proof~General sessions, one for replaying the old script and the
other for the emerging Isabelle/Isar document.  Also note that Isar supports
emulation commands and methods that support traditional tactic scripts within
new-style theories, see appendix~\ref{ap:conv} for more information.


\subsection{Document preparation}\label{sec:document-prep}

Isabelle/Isar provides a simple document preparation system based on existing
{PDF-\LaTeX} technology, with full support of hyper-links (both local
references and URLs), bookmarks, and thumbnails.  Thus the results are equally
well suited for WWW browsing and as printed copies.

\medskip

Isabelle generates {\LaTeX} output as part of the run of a \emph{logic
  session} (see also \cite{isabelle-sys}).  Getting started with a working
configuration for common situations is quite easy by using the Isabelle
\texttt{mkdir} and \texttt{make} tools.  First invoke
\begin{ttbox}
  isatool mkdir Foo
\end{ttbox}
to initialize a separate directory for session \texttt{Foo} --- it is safe to
experiment, since \texttt{isatool mkdir} never overwrites existing files.
Ensure that \texttt{Foo/ROOT.ML} holds ML commands to load all theories
required for this session; furthermore \texttt{Foo/document/root.tex} should
include any special {\LaTeX} macro packages required for your document (the
default is usually sufficient as a start).

The session is controlled by a separate \texttt{IsaMakefile} (with crude
source dependencies by default).  This file is located one level up from the
\texttt{Foo} directory location.  Now invoke
\begin{ttbox}
  isatool make Foo
\end{ttbox}
to run the \texttt{Foo} session, with browser information and document
preparation enabled.  Unless any errors are reported by Isabelle or {\LaTeX},
the output will appear inside the directory \texttt{ISABELLE_BROWSER_INFO}, as
reported by the batch job in verbose mode.

\medskip

You may also consider to tune the \texttt{usedir} options in
\texttt{IsaMakefile}, for example to change the output format from
\texttt{pdf} to \texttt{dvi}, or activate the \texttt{-D} option to retain a
second copy of the generated {\LaTeX} sources.

\medskip

See \emph{The Isabelle System Manual} \cite{isabelle-sys} for further details
on Isabelle logic sessions and theory presentation.  The Isabelle/HOL tutorial
\cite{isabelle-hol-book} also covers theory presentation issues.


\subsection{How to write Isar proofs anyway?}\label{sec:isar-howto}

This is one of the key questions, of course.  First of all, the tactic script
emulation of Isabelle/Isar essentially provides a clarified version of the
very same unstructured proof style of classic Isabelle.  Old-time users should
quickly become acquainted with that (slightly degenerative) view of Isar.

Writing \emph{proper} Isar proof texts targeted at human readers is quite
different, though.  Experienced users of the unstructured style may even have
to unlearn some of their habits to master proof composition in Isar.  In
contrast, new users with less experience in old-style tactical proving, but a
good understanding of mathematical proof in general, often get started easier.

\medskip The present text really is only a reference manual on Isabelle/Isar,
not a tutorial.  Nevertheless, we will attempt to give some clues of how the
concepts introduced here may be put into practice.  Appendix~\ref{ap:refcard}
provides a quick reference card of the most common Isabelle/Isar language
elements.  Appendix~\ref{ap:conv} offers some practical hints on converting
existing Isabelle theories and proof scripts to the new format (without
restructuring proofs).

Further issues concerning the Isar concepts are covered in the literature
\cite{Wenzel:1999:TPHOL,Wiedijk:2000:MV,Bauer-Wenzel:2000:HB,Bauer-Wenzel:2001}.
The author's PhD thesis \cite{Wenzel-PhD} presently provides the most complete
exposition of Isar foundations, techniques, and applications.  A number of
example applications are distributed with Isabelle, and available via the
Isabelle WWW library (e.g.\ \url{http://isabelle.in.tum.de/library/}).  As a
general rule of thumb, more recent Isabelle applications that also include a
separate ``document'' (in PDF) are more likely to consist of proper
Isabelle/Isar theories and proofs.

%FIXME
% The following examples may be of particular interest.  Apart from the plain
% sources represented in HTML, these Isabelle sessions also provide actual
% documents (in PDF).
% \begin{itemize}
% \item \url{http://isabelle.in.tum.de/library/HOL/Isar_examples/} is a
%   collection of introductory examples.
% \item \url{http://isabelle.in.tum.de/library/HOL/Lattice/} is an example of
%   typical mathematics-style reasoning in ``axiomatic'' structures.
% \item \url{http://isabelle.in.tum.de/library/HOL/HOL-Real/HahnBanach/} is a
%   big mathematics application on infinitary vector spaces and functional
%   analysis.
% \item \url{http://isabelle.in.tum.de/library/HOL/Lambda/} develops fundamental
%   properties of $\lambda$-calculus (Church-Rosser and termination).
  
%   This may serve as a realistic example of porting of legacy proof scripts
%   into Isar tactic emulation scripts.
% \item \url{http://isabelle.in.tum.de/library/HOL/Unix/} gives a mathematical
%   model of the main aspects of the Unix file-system including its security
%   model, but ignoring processes.  A few odd effects caused by the general
%   ``worse-is-better'' approach followed in Unix are discussed within the
%   formal model.
  
%   This example represents a non-trivial verification task, with all proofs
%   carefully worked out using the proper part of the Isar proof language;
%   unstructured scripts are only used for symbolic evaluation.
% \item \url{http://isabelle.in.tum.de/library/HOL/MicroJava/} is a
%   formalization of a fragment of Java, together with a corresponding virtual
%   machine and a specification of its bytecode verifier and a lightweight
%   bytecode verifier, including proofs of type-safety.
  
%   This represents a very ``realistic'' example of large formalizations
%   performed in form of tactic emulation scripts and proper Isar proof texts.
% \end{itemize}


%%% Local Variables: 
%%% mode: latex
%%% TeX-master: "isar-ref"
%%% End: 
