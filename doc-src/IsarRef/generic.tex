
\chapter{Generic Tools and Packages}\label{ch:gen-tools}

\section{Theory specification commands}

\subsection{Axiomatic type classes}\label{sec:axclass}

\indexisarcmd{axclass}\indexisarcmd{instance}\indexisarmeth{intro-classes}
\begin{matharray}{rcl}
  \isarcmd{axclass} & : & \isartrans{theory}{theory} \\
  \isarcmd{instance} & : & \isartrans{theory}{proof(prove)} \\
  intro_classes & : & \isarmeth \\
\end{matharray}

Axiomatic type classes are provided by Isabelle/Pure as a \emph{definitional}
interface to type classes (cf.~\S\ref{sec:classes}).  Thus any object logic
may make use of this light-weight mechanism of abstract theories
\cite{Wenzel:1997:TPHOL}.  There is also a tutorial on using axiomatic type
classes in Isabelle \cite{isabelle-axclass} that is part of the standard
Isabelle documentation.

\begin{rail}
  'axclass' classdecl (axmdecl prop +)
  ;
  'instance' (nameref ('<' | subseteq) nameref | nameref '::' simplearity)
  ;
\end{rail}

\begin{descr}
  
\item [$\AXCLASS~c \subseteq \vec c~~axms$] defines an axiomatic type class as
  the intersection of existing classes, with additional axioms holding.  Class
  axioms may not contain more than one type variable.  The class axioms (with
  implicit sort constraints added) are bound to the given names.  Furthermore
  a class introduction rule is generated (being bound as $c{.}intro$); this
  rule is employed by method $intro_classes$ to support instantiation proofs
  of this class.
  
  The ``axioms'' are stored as theorems according to the given name
  specifications, adding the class name $c$ as name space prefix; the same
  facts are also stored collectively as $c{\dtt}axioms$.
  
\item [$\INSTANCE~c@1 \subseteq c@2$ and $\INSTANCE~t :: (\vec s)c$] setup a
  goal stating a class relation or type arity.  The proof would usually
  proceed by $intro_classes$, and then establish the characteristic theorems
  of the type classes involved.  After finishing the proof, the theory will be
  augmented by a type signature declaration corresponding to the resulting
  theorem.

\item [$intro_classes$] repeatedly expands all class introduction rules of
  this theory.  Note that this method usually needs not be named explicitly,
  as it is already included in the default proof step (of $\PROOFNAME$,
  $\BYNAME$, etc.).  In particular, instantiation of trivial (syntactic)
  classes may be performed by a single ``$\DDOT$'' proof step.

\end{descr}


\subsection{Locales and local contexts}\label{sec:locale}

Locales are named local contexts, consisting of a declaration elements that
are modeled after the Isar proof context (cf.\ \S\ref{sec:proof-context}).

\subsubsection{Localized commands}

Existing locales may be augmented later on by adding new facts.  Note that the
actual context definition may not be changed!  Several theory commands that
produce facts in some way are available in ``localized'' versions, referring
to a named locale instead of the global theory context.

\indexouternonterm{locale}
\begin{rail}
  locale: '(' 'in' name ')'
  ;
\end{rail}

Emerging facts of localized commands are stored in two versions, both in the
target locale and the theory (after export).  The latter view produces a
qualified binding, using the locale name as a name space prefix.

For example, ``$\LEMMAS~(\IN~loc)~a = \vec b$'' retrieves facts $\vec b$ from
the locale context of $loc$ and augments its body by an appropriate
``$\isarkeyword{notes}$'' element (see below).  The exported view of $a$,
after discharging the locale context, is stored as $loc{.}a$ within the global
theory.  A localized goal ``$\LEMMANAME~(\IN~loc)~a:~\phi$'' work similarly,
only that the fact emerges through the subsequent proof,
which may refer to the full infrastructure of the locale context (including
local parameters with typing and concrete syntax, assumptions, definitions
etc.).  Most notably, fact declarations of the locale are active during the
proof, too (e.g.\ local $simp$ rules).


\subsubsection{Locale specifications}

\indexisarcmd{locale}\indexisarcmd{print-locale}\indexisarcmd{print-locales}
\begin{matharray}{rcl}
  \isarcmd{locale} & : & \isarkeep{theory} \\
  \isarcmd{print_locale}^* & : & \isarkeep{theory~|~proof} \\
  \isarcmd{print_locales}^* & : & \isarkeep{theory~|~proof} \\
\end{matharray}

\indexouternonterm{contextexpr}\indexouternonterm{contextelem}

\railalias{printlocale}{print\_locale}
\railterm{printlocale}

\begin{rail}
  'locale' name ('=' localeexpr)?
  ;
  printlocale localeexpr
  ;
  localeexpr: ((contextexpr '+' (contextelem+)) | contextexpr | (contextelem+))
  ;

  contextexpr: nameref | '(' contextexpr ')' |
  (contextexpr (name+)) | (contextexpr + '+')
  ;
  contextelem: fixes | assumes | defines | notes | includes
  ;
  fixes: 'fixes' (name ('::' type)? structmixfix? + 'and')
  ;
  assumes: 'assumes' (thmdecl? props + 'and')
  ;
  defines: 'defines' (thmdecl? prop proppat? + 'and')
  ;
  notes: 'notes' (thmdef? thmrefs + 'and')
  ;
  includes: 'includes' contextexpr
  ;
\end{rail}

\begin{descr}
  
\item [$\LOCALE~loc~=~import~+~body$] defines new locale $loc$ as a context
  consisting of a certain view of existing locales ($import$) plus some
  additional elements ($body$).  Both $import$ and $body$ are optional; the
  degenerate form $\LOCALE~loc$ defines an empty locale, which may still be
  useful to collect declarations of facts later on.  Type-inference on locale
  expressions automatically takes care of the most general typing that the
  combined context elements may acquire.
  
  The $import$ consists of a structured context expression, consisting of
  references to existing locales, renamed contexts, or merged contexts.
  Renaming uses positional notation: $c~\vec x$ means that (a prefix) the
  fixed parameters of context $c$ are named according to $\vec x$; a
  ``\texttt{_}'' (underscore).\indexisarthm{_@\texttt{_}} means to skip that
  position.  Also note that concrete syntax only works with the original name.
  Merging proceeds from left-to-right, suppressing any duplicates emerging
  from different paths through an import hierarchy.
  
  The $body$ consists of basic context elements, further context expressions
  may be included as well.

  \begin{descr}
    
  \item [$\FIXES{~x::\tau~(mx)}$] declares a local parameter of type $\tau$
    and mixfix annotation $mx$ (both are optional).  The special syntax
    declaration ``$(structure)$'' means that $x$ may be referenced
    implicitly in this context.
    
  \item [$\ASSUMES{a}{\vec\phi}$] introduces local premises, similar to
    $\ASSUMENAME$ within a proof (cf.\ \S\ref{sec:proof-context}).
    
  \item [$\DEFINES{a}{x \equiv t}$] defines a previously declared parameter.
    This is close to $\DEFNAME$ within a proof (cf.\ 
    \S\ref{sec:proof-context}), but $\DEFINESNAME$ takes an equational
    proposition instead of variable-term.  The left-hand side of the equation
    may have additional arguments, e.g.\ $\DEFINES{}{f~\vec x \equiv t}$.
    
  \item [$\NOTES{a}{\vec b}$] reconsiders facts within a local context.  Most
    notably, this may include arbitrary declarations in any attribute
    specifications included here, e.g.\ a local $simp$ rule.
    
  \item [$\INCLUDES{c}$] copies the specified context in a statically scoped
    manner.
    
    In contrast, the initial $import$ specification of a locale expression
    maintains a dynamic relation to the locales being referenced (benefiting
    from any later fact declarations in the obvious manner).
  \end{descr}
  
  Note that $\IS{p}$ patterns given in the syntax of $\ASSUMESNAME$ and
  $\DEFINESNAME$ above is actually illegal in locale definitions.  In the long
  goal format of \S\ref{sec:goals}, term bindings may be included as expected.

\item [$\isarkeyword{print_locale}~import~+~body$] prints the specified locale
  expression in a flattened form.  The notable special case
  $\isarkeyword{print_locale}~loc$ just prints the contents of the named
  locale, but keep in mind that type-inference will normalize type variables
  according to the usual alphabetical order.
  
\item [$\isarkeyword{print_locales}$] prints the names of all locales of the
  current theory.

\end{descr}


\section{Derived proof schemes}

\subsection{Generalized elimination}\label{sec:obtain}

\indexisarcmd{obtain}
\begin{matharray}{rcl}
  \isarcmd{obtain} & : & \isartrans{proof(state)}{proof(prove)} \\
\end{matharray}

Generalized elimination means that additional elements with certain properties
may introduced in the current context, by virtue of a locally proven
``soundness statement''.  Technically speaking, the $\OBTAINNAME$ language
element is like a declaration of $\FIXNAME$ and $\ASSUMENAME$ (see also see
\S\ref{sec:proof-context}), together with a soundness proof of its additional
claim.  According to the nature of existential reasoning, assumptions get
eliminated from any result exported from the context later, provided that the
corresponding parameters do \emph{not} occur in the conclusion.

\begin{rail}
  'obtain' (vars + 'and') 'where' (props + 'and')
  ;
\end{rail}

$\OBTAINNAME$ is defined as a derived Isar command as follows, where $\vec b$
shall refer to (optional) facts indicated for forward chaining.
\begin{matharray}{l}
  \langle facts~\vec b\rangle \\
  \OBTAIN{\vec x}{a}{\vec \phi}~~\langle proof\rangle \equiv {} \\[1ex]
  \quad \BG \\
  \qquad \FIX{thesis} \\
  \qquad \ASSUME{that~[simp, intro]}{\All{\vec x} \vec\phi \Imp thesis} \\
  \qquad \FROM{\vec b}~\HAVE{}{thesis}~~\langle proof\rangle \\
  \quad \EN \\
  \quad \FIX{\vec x}~\ASSUMENAME^\ast~a\colon~\vec\phi \\
\end{matharray}

Typically, the soundness proof is relatively straight-forward, often just by
canonical automated tools such as $\BY{simp}$ (see \S\ref{sec:simp}) or
$\BY{blast}$ (see \S\ref{sec:classical-auto}).  Accordingly, the ``$that$''
reduction above is declared as simplification and introduction rule.

\medskip

In a sense, $\OBTAINNAME$ represents at the level of Isar proofs what would be
meta-logical existential quantifiers and conjunctions.  This concept has a
broad range of useful applications, ranging from plain elimination (or even
introduction) of object-level existentials and conjunctions, to elimination
over results of symbolic evaluation of recursive definitions, for example.
Also note that $\OBTAINNAME$ without parameters acts much like $\HAVENAME$,
where the result is treated as an assumption.


\subsection{Calculational reasoning}\label{sec:calculation}

\indexisarcmd{also}\indexisarcmd{finally}
\indexisarcmd{moreover}\indexisarcmd{ultimately}
\indexisarcmd{print-trans-rules}
\indexisaratt{trans}\indexisaratt{sym}\indexisaratt{symmetric}
\begin{matharray}{rcl}
  \isarcmd{also} & : & \isartrans{proof(state)}{proof(state)} \\
  \isarcmd{finally} & : & \isartrans{proof(state)}{proof(chain)} \\
  \isarcmd{moreover} & : & \isartrans{proof(state)}{proof(state)} \\
  \isarcmd{ultimately} & : & \isartrans{proof(state)}{proof(chain)} \\
  \isarcmd{print_trans_rules}^* & : & \isarkeep{theory~|~proof} \\
  trans & : & \isaratt \\
  sym & : & \isaratt \\
  symmetric & : & \isaratt \\
\end{matharray}

Calculational proof is forward reasoning with implicit application of
transitivity rules (such those of $=$, $\leq$, $<$).  Isabelle/Isar maintains
an auxiliary register $calculation$\indexisarthm{calculation} for accumulating
results obtained by transitivity composed with the current result.  Command
$\ALSO$ updates $calculation$ involving $this$, while $\FINALLY$ exhibits the
final $calculation$ by forward chaining towards the next goal statement.  Both
commands require valid current facts, i.e.\ may occur only after commands that
produce theorems such as $\ASSUMENAME$, $\NOTENAME$, or some finished proof of
$\HAVENAME$, $\SHOWNAME$ etc.  The $\MOREOVER$ and $\ULTIMATELY$ commands are
similar to $\ALSO$ and $\FINALLY$, but only collect further results in
$calculation$ without applying any rules yet.

Also note that the automatic term abbreviation ``$\dots$'' has its canonical
application with calculational proofs.  It refers to the argument\footnote{The
  argument of a curried infix expression is its right-hand side.} of the
preceding statement.

Isabelle/Isar calculations are implicitly subject to block structure in the
sense that new threads of calculational reasoning are commenced for any new
block (as opened by a local goal, for example).  This means that, apart from
being able to nest calculations, there is no separate \emph{begin-calculation}
command required.

\medskip

The Isar calculation proof commands may be defined as
follows:\footnote{Internal bookkeeping such as proper handling of
  block-structure has been suppressed.}
\begin{matharray}{rcl}
  \ALSO@0 & \equiv & \NOTE{calculation}{this} \\
  \ALSO@{n+1} & \equiv & \NOTE{calculation}{trans~[OF~calculation~this]} \\[0.5ex]
  \FINALLY & \equiv & \ALSO~\FROM{calculation} \\
  \MOREOVER & \equiv & \NOTE{calculation}{calculation~this} \\
  \ULTIMATELY & \equiv & \MOREOVER~\FROM{calculation} \\
\end{matharray}

\begin{rail}
  ('also' | 'finally') ('(' thmrefs ')')?
  ;
  'trans' (() | 'add' | 'del')
  ;
\end{rail}

\begin{descr}
  
\item [$\ALSO~(\vec a)$] maintains the auxiliary $calculation$ register as
  follows.  The first occurrence of $\ALSO$ in some calculational thread
  initializes $calculation$ by $this$. Any subsequent $\ALSO$ on the same
  level of block-structure updates $calculation$ by some transitivity rule
  applied to $calculation$ and $this$ (in that order).  Transitivity rules are
  picked from the current context, unless alternative rules are given as
  explicit arguments.

\item [$\FINALLY~(\vec a)$] maintaining $calculation$ in the same way as
  $\ALSO$, and concludes the current calculational thread.  The final result
  is exhibited as fact for forward chaining towards the next goal. Basically,
  $\FINALLY$ just abbreviates $\ALSO~\FROM{calculation}$.  Note that
  ``$\FINALLY~\SHOW{}{\Var{thesis}}~\DOT$'' and
  ``$\FINALLY~\HAVE{}{\phi}~\DOT$'' are typical idioms for concluding
  calculational proofs.

\item [$\MOREOVER$ and $\ULTIMATELY$] are analogous to $\ALSO$ and $\FINALLY$,
  but collect results only, without applying rules.
  
\item [$\isarkeyword{print_trans_rules}$] prints the list of transitivity
  rules (for calculational commands $\ALSO$ and $\FINALLY$) and symmetry rules
  (for the $symmetric$ operation and single step elimination patters) of the
  current context.
  
\item [$trans$] declares theorems as transitivity rules.
  
\item [$sym$] declares symmetry rules.
  
\item [$symmetric$] resolves a theorem with some rule declared as $sym$ in the
  current context.  For example, ``$\ASSUME{[symmetric]}{x = y}$'' produces a
  swapped fact derived from that assumption.
  
  In structured proof texts it is often more appropriate to use an explicit
  single-step elimination proof, such as ``$\ASSUME{}{x = y}~\HENCE{}{y =
    x}~\DDOT$''.  Note that the very same rules known to $symmetric$ are
  declared as $elim$ at the same time.

\end{descr}


\section{Specific proof tools}

\subsection{Miscellaneous methods and attributes}\label{sec:misc-meth-att}

\indexisarmeth{unfold}\indexisarmeth{fold}\indexisarmeth{insert}
\indexisarmeth{erule}\indexisarmeth{drule}\indexisarmeth{frule}
\indexisarmeth{fail}\indexisarmeth{succeed}
\begin{matharray}{rcl}
  unfold & : & \isarmeth \\
  fold & : & \isarmeth \\
  insert & : & \isarmeth \\[0.5ex]
  erule^* & : & \isarmeth \\
  drule^* & : & \isarmeth \\
  frule^* & : & \isarmeth \\
  succeed & : & \isarmeth \\
  fail & : & \isarmeth \\
\end{matharray}

\begin{rail}
  ('fold' | 'unfold' | 'insert') thmrefs
  ;
  ('erule' | 'drule' | 'frule') ('('nat')')? thmrefs
  ;
\end{rail}

\begin{descr}
  
\item [$unfold~\vec a$ and $fold~\vec a$] expand (or fold back again) the
  given meta-level definitions throughout all goals; any chained facts
  provided are inserted into the goal and subject to rewriting as well.
  
\item [$insert~\vec a$] inserts theorems as facts into all goals of the proof
  state.  Note that current facts indicated for forward chaining are ignored.

\item [$erule~\vec a$, $drule~\vec a$, and $frule~\vec a$] are similar to the
  basic $rule$ method (see \S\ref{sec:pure-meth-att}), but apply rules by
  elim-resolution, destruct-resolution, and forward-resolution, respectively
  \cite{isabelle-ref}.  The optional natural number argument (default $0$)
  specifies additional assumption steps to be performed.
  
  Note that these methods are improper ones, mainly serving for
  experimentation and tactic script emulation.  Different modes of basic rule
  application are usually expressed in Isar at the proof language level,
  rather than via implicit proof state manipulations.  For example, a proper
  single-step elimination would be done using the basic $rule$ method, with
  forward chaining of current facts.

\item [$succeed$] yields a single (unchanged) result; it is the identity of
  the ``\texttt{,}'' method combinator (cf.\ \S\ref{sec:syn-meth}).

\item [$fail$] yields an empty result sequence; it is the identity of the
  ``\texttt{|}'' method combinator (cf.\ \S\ref{sec:syn-meth}).

\end{descr}

\indexisaratt{tagged}\indexisaratt{untagged}
\indexisaratt{THEN}\indexisaratt{COMP}
\indexisaratt{where}\indexisaratt{unfolded}\indexisaratt{folded}
\indexisaratt{standard}\indexisarattof{Pure}{elim-format}
\indexisaratt{no-vars}
\begin{matharray}{rcl}
  tagged & : & \isaratt \\
  untagged & : & \isaratt \\[0.5ex]
  THEN & : & \isaratt \\
  COMP & : & \isaratt \\[0.5ex]
  where & : & \isaratt \\[0.5ex]
  unfolded & : & \isaratt \\
  folded & : & \isaratt \\[0.5ex]
  standard & : & \isaratt \\
  elim_format & : & \isaratt \\
  no_vars^* & : & \isaratt \\
\end{matharray}

\begin{rail}
  'tagged' (nameref+)
  ;
  'untagged' name
  ;
  ('THEN' | 'COMP') ('[' nat ']')? thmref
  ;
  'where' (name '=' term * 'and')
  ;
  ('unfolded' | 'folded') thmrefs
  ;
\end{rail}

\begin{descr}
  
\item [$tagged~name~args$ and $untagged~name$] add and remove $tags$ of some
  theorem.  Tags may be any list of strings that serve as comment for some
  tools (e.g.\ $\LEMMANAME$ causes the tag ``$lemma$'' to be added to the
  result).  The first string is considered the tag name, the rest its
  arguments.  Note that untag removes any tags of the same name.
  
\item [$THEN~n~a$ and $COMP~n~a$] compose rules.  $THEN$ resolves with the
  $n$-th premise of $a$; the $COMP$ version skips the automatic lifting
  process that is normally intended (cf.\ \texttt{RS} and \texttt{COMP} in
  \cite[\S5]{isabelle-ref}).
  
\item [$where~\vec x = \vec t$] perform named instantiation of schematic
  variables occurring in a theorem.  Unlike instantiation tactics such as
  $rule_tac$ (see \S\ref{sec:tactic-commands}), actual schematic variables
  have to be specified (e.g.\ $\Var{x@3}$).
  
\item [$unfolded~\vec a$ and $folded~\vec a$] expand and fold back again the
  given meta-level definitions throughout a rule.
  
\item [$standard$] puts a theorem into the standard form of object-rules, just
  as the ML function \texttt{standard} (see \cite[\S5]{isabelle-ref}).
  
\item [$elim_format$] turns a destruction rule into elimination rule format,
  by resolving with the rule $\PROP A \Imp (\PROP A \Imp \PROP B) \Imp \PROP
  B$.
  
  Note that the Classical Reasoner (\S\ref{sec:classical-att}) provides its
  own version of this operation.
  
\item [$no_vars$] replaces schematic variables by free ones; this is mainly
  for tuning output of pretty printed theorems.

\end{descr}


\subsection{Further tactic emulations}\label{sec:tactics}

The following improper proof methods emulate traditional tactics.  These admit
direct access to the goal state, which is normally considered harmful!  In
particular, this may involve both numbered goal addressing (default 1), and
dynamic instantiation within the scope of some subgoal.

\begin{warn}
  Dynamic instantiations are read and type-checked according to a subgoal of
  the current dynamic goal state, rather than the static proof context!  In
  particular, locally fixed variables and term abbreviations may not be
  included in the term specifications.  Thus schematic variables are left to
  be solved by unification with certain parts of the subgoal involved.
\end{warn}

Note that the tactic emulation proof methods in Isabelle/Isar are consistently
named $foo_tac$.

\indexisarmeth{rule-tac}\indexisarmeth{erule-tac}
\indexisarmeth{drule-tac}\indexisarmeth{frule-tac}
\indexisarmeth{cut-tac}\indexisarmeth{thin-tac}
\indexisarmeth{subgoal-tac}\indexisarmeth{rename-tac}
\indexisarmeth{rotate-tac}\indexisarmeth{tactic}
\begin{matharray}{rcl}
  rule_tac^* & : & \isarmeth \\
  erule_tac^* & : & \isarmeth \\
  drule_tac^* & : & \isarmeth \\
  frule_tac^* & : & \isarmeth \\
  cut_tac^* & : & \isarmeth \\
  thin_tac^* & : & \isarmeth \\
  subgoal_tac^* & : & \isarmeth \\
  rename_tac^* & : & \isarmeth \\
  rotate_tac^* & : & \isarmeth \\
  tactic^* & : & \isarmeth \\
\end{matharray}

\railalias{ruletac}{rule\_tac}
\railterm{ruletac}

\railalias{eruletac}{erule\_tac}
\railterm{eruletac}

\railalias{druletac}{drule\_tac}
\railterm{druletac}

\railalias{fruletac}{frule\_tac}
\railterm{fruletac}

\railalias{cuttac}{cut\_tac}
\railterm{cuttac}

\railalias{thintac}{thin\_tac}
\railterm{thintac}

\railalias{subgoaltac}{subgoal\_tac}
\railterm{subgoaltac}

\railalias{renametac}{rename\_tac}
\railterm{renametac}

\railalias{rotatetac}{rotate\_tac}
\railterm{rotatetac}

\begin{rail}
  ( ruletac | eruletac | druletac | fruletac | cuttac | thintac ) goalspec?
  ( insts thmref | thmrefs )
  ;
  subgoaltac goalspec? (prop +)
  ;
  renametac goalspec? (name +)
  ;
  rotatetac goalspec? int?
  ;
  'tactic' text
  ;

  insts: ((name '=' term) + 'and') 'in'
  ;
\end{rail}

\begin{descr}
  
\item [$rule_tac$ etc.] do resolution of rules with explicit instantiation.
  This works the same way as the ML tactics \texttt{res_inst_tac} etc. (see
  \cite[\S3]{isabelle-ref}).
  
  Note that multiple rules may be only given there is no instantiation.  Then
  $rule_tac$ is the same as \texttt{resolve_tac} in ML (see
  \cite[\S3]{isabelle-ref}).
  
\item [$cut_tac$] inserts facts into the proof state as assumption of a
  subgoal, see also \texttt{cut_facts_tac} in \cite[\S3]{isabelle-ref}.  Note
  that the scope of schematic variables is spread over the main goal
  statement.  Instantiations may be given as well, see also ML tactic
  \texttt{cut_inst_tac} in \cite[\S3]{isabelle-ref}.
  
\item [$thin_tac~\phi$] deletes the specified assumption from a subgoal; note
  that $\phi$ may contain schematic variables.  See also \texttt{thin_tac} in
  \cite[\S3]{isabelle-ref}.
  
\item [$subgoal_tac~\phi$] adds $\phi$ as an assumption to a subgoal.  See
  also \texttt{subgoal_tac} and \texttt{subgoals_tac} in
  \cite[\S3]{isabelle-ref}.
  
\item [$rename_tac~\vec x$] renames parameters of a goal according to the list
  $\vec x$, which refers to the \emph{suffix} of variables.
  
\item [$rotate_tac~n$] rotates the assumptions of a goal by $n$ positions:
  from right to left if $n$ is positive, and from left to right if $n$ is
  negative; the default value is $1$.  See also \texttt{rotate_tac} in
  \cite[\S3]{isabelle-ref}.
  
\item [$tactic~text$] produces a proof method from any ML text of type
  \texttt{tactic}.  Apart from the usual ML environment and the current
  implicit theory context, the ML code may refer to the following locally
  bound values:

{\footnotesize\begin{verbatim}
val ctxt  : Proof.context
val facts : thm list
val thm   : string -> thm
val thms  : string -> thm list
\end{verbatim}}
  Here \texttt{ctxt} refers to the current proof context, \texttt{facts}
  indicates any current facts for forward-chaining, and
  \texttt{thm}~/~\texttt{thms} retrieve named facts (including global
  theorems) from the context.
\end{descr}


\subsection{The Simplifier}\label{sec:simplifier}

\subsubsection{Simplification methods}\label{sec:simp}

\indexisarmeth{simp}\indexisarmeth{simp-all}
\begin{matharray}{rcl}
  simp & : & \isarmeth \\
  simp_all & : & \isarmeth \\
\end{matharray}

\railalias{simpall}{simp\_all}
\railterm{simpall}

\railalias{noasm}{no\_asm}
\railterm{noasm}

\railalias{noasmsimp}{no\_asm\_simp}
\railterm{noasmsimp}

\railalias{noasmuse}{no\_asm\_use}
\railterm{noasmuse}

\indexouternonterm{simpmod}
\begin{rail}
  ('simp' | simpall) ('!' ?) opt? (simpmod *)
  ;

  opt: '(' (noasm | noasmsimp | noasmuse) ')'
  ;
  simpmod: ('add' | 'del' | 'only' | 'cong' (() | 'add' | 'del') |
    'split' (() | 'add' | 'del')) ':' thmrefs
  ;
\end{rail}

\begin{descr}

\item [$simp$] invokes Isabelle's simplifier, after declaring additional rules
  according to the arguments given.  Note that the \railtterm{only} modifier
  first removes all other rewrite rules, congruences, and looper tactics
  (including splits), and then behaves like \railtterm{add}.
  
  \medskip The \railtterm{cong} modifiers add or delete Simplifier congruence
  rules (see also \cite{isabelle-ref}), the default is to add.
  
  \medskip The \railtterm{split} modifiers add or delete rules for the
  Splitter (see also \cite{isabelle-ref}), the default is to add.  This works
  only if the Simplifier method has been properly setup to include the
  Splitter (all major object logics such HOL, HOLCF, FOL, ZF do this already).
  
\item [$simp_all$] is similar to $simp$, but acts on all goals (backwards from
  the last to the first one).

\end{descr}

By default the Simplifier methods take local assumptions fully into account,
using equational assumptions in the subsequent normalization process, or
simplifying assumptions themselves (cf.\ \texttt{asm_full_simp_tac} in
\cite[\S10]{isabelle-ref}).  In structured proofs this is usually quite well
behaved in practice: just the local premises of the actual goal are involved,
additional facts may inserted via explicit forward-chaining (using $\THEN$,
$\FROMNAME$ etc.).  The full context of assumptions is only included if the
``$!$'' (bang) argument is given, which should be used with some care, though.

Additional Simplifier options may be specified to tune the behavior further
(mostly for unstructured scripts with many accidental local facts): $(no_asm)$
means assumptions are ignored completely (cf.\ \texttt{simp_tac}),
$(no_asm_simp)$ means assumptions are used in the simplification of the
conclusion but are not themselves simplified (cf.\ \texttt{asm_simp_tac}), and
$(no_asm_use)$ means assumptions are simplified but are not used in the
simplification of each other or the conclusion (cf.  \texttt{full_simp_tac}).

\medskip

The Splitter package is usually configured to work as part of the Simplifier.
The effect of repeatedly applying \texttt{split_tac} can be simulated by
$(simp~only\colon~split\colon~\vec a)$.  There is also a separate $split$
method available for single-step case splitting, see \S\ref{sec:basic-eq}.


\subsubsection{Declaring rules}

\indexisarcmd{print-simpset}
\indexisaratt{simp}\indexisaratt{split}\indexisaratt{cong}
\begin{matharray}{rcl}
  \isarcmd{print_simpset}^* & : & \isarkeep{theory~|~proof} \\
  simp & : & \isaratt \\
  cong & : & \isaratt \\
  split & : & \isaratt \\
\end{matharray}

\begin{rail}
  ('simp' | 'cong' | 'split') (() | 'add' | 'del')
  ;
\end{rail}

\begin{descr}

\item [$\isarcmd{print_simpset}$] prints the collection of rules declared to
  the Simplifier, which is also known as ``simpset'' internally
  \cite{isabelle-ref}.  This is a diagnostic command; $undo$ does not apply.

\item [$simp$] declares simplification rules.

\item [$cong$] declares congruence rules.

\item [$split$] declares case split rules.

\end{descr}


\subsubsection{Forward simplification}

\indexisaratt{simplified}
\begin{matharray}{rcl}
  simplified & : & \isaratt \\
\end{matharray}

\begin{rail}
  'simplified' opt? thmrefs?
  ;

  opt: '(' (noasm | noasmsimp | noasmuse) ')'
  ;
\end{rail}

\begin{descr}
  
\item [$simplified~\vec a$] causes a theorem to be simplified, either by
  exactly the specified rules $\vec a$, or the implicit Simplifier context if
  no arguments are given.  The result is fully simplified by default,
  including assumptions and conclusion; the options $no_asm$ etc.\ tune the
  Simplifier in the same way as the for the $simp$ method (see
  \S\ref{sec:simp}).
  
  Note that forward simplification restricts the simplifier to its most basic
  operation of term rewriting; solver and looper tactics \cite{isabelle-ref}
  are \emph{not} involved here.  The $simplified$ attribute should be only
  rarely required under normal circumstances.

\end{descr}


\subsubsection{Low-level equational reasoning}\label{sec:basic-eq}

\indexisarmeth{subst}\indexisarmeth{hypsubst}\indexisarmeth{split}
\begin{matharray}{rcl}
  subst^* & : & \isarmeth \\
  hypsubst^* & : & \isarmeth \\
  split^* & : & \isarmeth \\
\end{matharray}

\begin{rail}
  'subst' thmref
  ;
  'split' ('(' 'asm' ')')? thmrefs
  ;
\end{rail}

These methods provide low-level facilities for equational reasoning that are
intended for specialized applications only.  Normally, single step
calculations would be performed in a structured text (see also
\S\ref{sec:calculation}), while the Simplifier methods provide the canonical
way for automated normalization (see \S\ref{sec:simplifier}).

\begin{descr}
  
\item [$subst~thm$] performs a single substitution step using rule $thm$,
  which may be either a meta or object equality.
  
\item [$hypsubst$] performs substitution using some assumption.  Note that
  this only works for equations of the form $x = t$ where $x$ is a free or
  bound variable!
  
\item [$split~thms$] performs single-step case splitting using rules $thms$.
  By default, splitting is performed in the conclusion of a goal; the $asm$
  option indicates to operate on assumptions instead.
  
  Note that the $simp$ method already involves repeated application of split
  rules as declared in the current context (see \S\ref{sec:simp}).
\end{descr}


\subsection{The Classical Reasoner}\label{sec:classical}

\subsubsection{Basic methods}\label{sec:classical-basic}

\indexisarmeth{rule}\indexisarmeth{default}\indexisarmeth{contradiction}
\indexisarmeth{intro}\indexisarmeth{elim}
\begin{matharray}{rcl}
  rule & : & \isarmeth \\
  contradiction & : & \isarmeth \\
  intro & : & \isarmeth \\
  elim & : & \isarmeth \\
\end{matharray}

\begin{rail}
  ('rule' | 'intro' | 'elim') thmrefs?
  ;
\end{rail}

\begin{descr}
  
\item [$rule$] as offered by the classical reasoner is a refinement over the
  primitive one (see \S\ref{sec:pure-meth-att}).  Both versions essentially
  work the same, but the classical version observes the classical rule context
  in addition to the Isabelle/Pure one.
  
  The library of common object logics (HOL, ZF, etc.) usually declare a rich
  collection of classical rules (even if these perfectly OK from the
  intuitionistic viewpoint), but only few declarations to the rule context of
  Isabelle/Pure (\S\ref{sec:pure-meth-att}).
  
\item [$contradiction$] solves some goal by contradiction, deriving any result
  from both $\neg A$ and $A$.  Facts, which are guaranteed to participate, may
  appear in either order.

\item [$intro$ and $elim$] repeatedly refine some goal by intro- or
  elim-resolution, after having inserted any facts.  Omitting the arguments
  refers to any suitable rules declared in the context, otherwise only the
  explicitly given ones may be applied.  The latter form admits better control
  of what actually happens, thus it is very appropriate as an initial method
  for $\PROOFNAME$ that splits up certain connectives of the goal, before
  entering the actual sub-proof.
  
\end{descr}


\subsubsection{Automated methods}\label{sec:classical-auto}

\indexisarmeth{blast}\indexisarmeth{fast}\indexisarmeth{slow}
\indexisarmeth{best}\indexisarmeth{safe}\indexisarmeth{clarify}
\begin{matharray}{rcl}
  blast & : & \isarmeth \\
  fast & : & \isarmeth \\
  slow & : & \isarmeth \\
  best & : & \isarmeth \\
  safe & : & \isarmeth \\
  clarify & : & \isarmeth \\
\end{matharray}

\indexouternonterm{clamod}
\begin{rail}
  'blast' ('!' ?) nat? (clamod *)
  ;
  ('fast' | 'slow' | 'best' | 'safe' | 'clarify') ('!' ?) (clamod *)
  ;

  clamod: (('intro' | 'elim' | 'dest') ('!' | () | '?') | 'del') ':' thmrefs
  ;
\end{rail}

\begin{descr}
\item [$blast$] refers to the classical tableau prover (see \texttt{blast_tac}
  in \cite[\S11]{isabelle-ref}).  The optional argument specifies a
  user-supplied search bound (default 20).
\item [$fast$, $slow$, $best$, $safe$, and $clarify$] refer to the generic
  classical reasoner.  See \texttt{fast_tac}, \texttt{slow_tac},
  \texttt{best_tac}, \texttt{safe_tac}, and \texttt{clarify_tac} in
  \cite[\S11]{isabelle-ref} for more information.
\end{descr}

Any of above methods support additional modifiers of the context of classical
rules.  Their semantics is analogous to the attributes given in
\S\ref{sec:classical-mod}.  Facts provided by forward chaining are
inserted\footnote{These methods usually cannot make proper use of actual rules
  inserted that way, though.} into the goal before doing the search.  The
``!''~argument causes the full context of assumptions to be included as well.
This is slightly less hazardous than for the Simplifier (see
\S\ref{sec:simp}).


\subsubsection{Combined automated methods}\label{sec:clasimp}

\indexisarmeth{auto}\indexisarmeth{force}\indexisarmeth{clarsimp}
\indexisarmeth{fastsimp}\indexisarmeth{slowsimp}\indexisarmeth{bestsimp}
\begin{matharray}{rcl}
  auto & : & \isarmeth \\
  force & : & \isarmeth \\
  clarsimp & : & \isarmeth \\
  fastsimp & : & \isarmeth \\
  slowsimp & : & \isarmeth \\
  bestsimp & : & \isarmeth \\
\end{matharray}

\indexouternonterm{clasimpmod}
\begin{rail}
  'auto' '!'? (nat nat)? (clasimpmod *)
  ;
  ('force' | 'clarsimp' | 'fastsimp' | 'slowsimp' | 'bestsimp') '!'? (clasimpmod *)
  ;

  clasimpmod: ('simp' (() | 'add' | 'del' | 'only') |
    ('cong' | 'split') (() | 'add' | 'del') |
    'iff' (((() | 'add') '?'?) | 'del') |
    (('intro' | 'elim' | 'dest') ('!' | () | '?') | 'del')) ':' thmrefs
\end{rail}

\begin{descr}
\item [$auto$, $force$, $clarsimp$, $fastsimp$, $slowsimp$, and $bestsimp$]
  provide access to Isabelle's combined simplification and classical reasoning
  tactics.  These correspond to \texttt{auto_tac}, \texttt{force_tac},
  \texttt{clarsimp_tac}, and Classical Reasoner tactics with the Simplifier
  added as wrapper, see \cite[\S11]{isabelle-ref} for more information.  The
  modifier arguments correspond to those given in \S\ref{sec:simp} and
  \S\ref{sec:classical-auto}.  Just note that the ones related to the
  Simplifier are prefixed by \railtterm{simp} here.

  Facts provided by forward chaining are inserted into the goal before doing
  the search.  The ``!''~argument causes the full context of assumptions to be
  included as well.
\end{descr}


\subsubsection{Declaring rules}\label{sec:classical-mod}

\indexisarcmd{print-claset}
\indexisaratt{intro}\indexisaratt{elim}\indexisaratt{dest}
\indexisaratt{iff}\indexisaratt{rule}
\begin{matharray}{rcl}
  \isarcmd{print_claset}^* & : & \isarkeep{theory~|~proof} \\
  intro & : & \isaratt \\
  elim & : & \isaratt \\
  dest & : & \isaratt \\
  rule & : & \isaratt \\
  iff & : & \isaratt \\
\end{matharray}

\begin{rail}
  ('intro' | 'elim' | 'dest') ('!' | () | '?')
  ;
  'rule' 'del'
  ;
  'iff' (((() | 'add') '?'?) | 'del')
  ;
\end{rail}

\begin{descr}

\item [$\isarcmd{print_claset}$] prints the collection of rules declared to
  the Classical Reasoner, which is also known as ``simpset'' internally
  \cite{isabelle-ref}.  This is a diagnostic command; $undo$ does not apply.

\item [$intro$, $elim$, and $dest$] declare introduction, elimination, and
  destruction rules, respectively.  By default, rules are considered as
  \emph{unsafe} (i.e.\ not applied blindly without backtracking), while a
  single ``!'' classifies as \emph{safe}, and ``?'' as \emph{extra} (i.e.\ not
  applied in the search-oriented automated methods, but only in single-step
  methods such as $rule$).

\item [$rule~del$] deletes introduction, elimination, or destruction rules from
  the context.
  
\item [$iff$] declares a logical equivalences to the Simplifier and the
  Classical reasoner at the same time.  Non-conditional rules result in a
  ``safe'' introduction and elimination pair; conditional ones are considered
  ``unsafe''.  Rules with negative conclusion are automatically inverted
  (using $\neg$-elimination).
  
  The ``?'' version of $iff$ declares rules to the Pure context only, and
  omits the Simplifier declaration.  Thus the declaration does not have any
  effect on automated proof tools, but only on the single-step $rule$ method
  (see \S\ref{sec:misc-meth-att}).
\end{descr}


\subsubsection{Classical operations}\label{sec:classical-att}

\indexisaratt{elim-format}\indexisaratt{swapped}

\begin{matharray}{rcl}
  elim_format & : & \isaratt \\
  swapped & : & \isaratt \\
\end{matharray}

\begin{descr}
  
\item [$elim_format$] turns a destruction rule into elimination rule format;
  this operation is similar to the the intuitionistic version
  (\S\ref{sec:misc-meth-att}), but each premise of the resulting rule acquires
  an additional local fact of the negated main thesis -- according to the
  classical principle $(\neg A \Imp A) \Imp A$.
  
\item [$swapped$] turns an introduction rule into an elimination, by resolving
  with the classical swap principle $(\neg B \Imp A) \Imp (\neg A \Imp B)$.

\end{descr}


\subsection{Proof by cases and induction}\label{sec:cases-induct}

\subsubsection{Rule contexts}\label{sec:rule-cases}

\indexisarcmd{case}\indexisarcmd{print-cases}
\indexisaratt{case-names}\indexisaratt{params}\indexisaratt{consumes}
\begin{matharray}{rcl}
  \isarcmd{case} & : & \isartrans{proof(state)}{proof(state)} \\
  \isarcmd{print_cases}^* & : & \isarkeep{proof} \\
  case_names & : & \isaratt \\
  params & : & \isaratt \\
  consumes & : & \isaratt \\
\end{matharray}

Basically, Isar proof contexts are built up explicitly using commands like
$\FIXNAME$, $\ASSUMENAME$ etc.\ (see \S\ref{sec:proof-context}).  In typical
verification tasks this can become hard to manage, though.  In particular, a
large number of local contexts may emerge from case analysis or induction over
inductive sets and types.

\medskip

The $\CASENAME$ command provides a shorthand to refer to certain parts of
logical context symbolically.  Proof methods may provide an environment of
named ``cases'' of the form $c\colon \vec x, \vec \phi$.  Then the effect of
``$\CASE{c}$'' is that of ``$\FIX{\vec x}~\ASSUME{c}{\vec\phi}$''; term
bindings may be covered as well, such as $\Var{case}$.

Normally the ``terminology'' of a case value (i.e.\ the parameters $\vec x$)
are marked as hidden.  Using the alternative form ``$(\CASE{c}~\vec x)$''
enables proof writers to choose their own naming for the subsequent proof
text.

\medskip

It is important to note that $\CASENAME$ does \emph{not} provide direct means
to peek at the current goal state, which is generally considered
non-observable in Isar.  The text of the cases basically emerge from standard
elimination or induction rules, which in turn are derived from previous theory
specifications in a canonical way (say via $\isarkeyword{inductive}$).

Named cases may be exhibited in the current proof context only if both the
proof method and the rules involved support this.  Case names and parameters
of basic rules may be declared by hand as well, by using appropriate
attributes.  Thus variant versions of rules that have been derived manually
may be used in advanced case analysis later.

\railalias{casenames}{case\_names}
\railterm{casenames}

\begin{rail}
  'case' caseref | ('(' caseref ((name | underscore) +) ')')
  ;
  caseref: nameref attributes?
  ;

  casenames (name +)
  ;
  'params' ((name *) + 'and')
  ;
  'consumes' nat?
  ;
\end{rail}

\begin{descr}
  
\item [$\CASE{c}$] invokes a named local context $c\colon \vec x, \vec \phi$,
  as provided by an appropriate proof method (such as $cases$ and $induct$,
  see \S\ref{sec:cases-induct-meth}).  The command $\CASE{c}$ abbreviates
  $\FIX{\vec x}~\ASSUME{c}{\vec\phi}$.
  
\item [$\isarkeyword{print_cases}$] prints all local contexts of the current
  state, using Isar proof language notation.  This is a diagnostic command;
  $undo$ does not apply.
  
\item [$case_names~\vec c$] declares names for the local contexts of premises
  of some theorem; $\vec c$ refers to the \emph{suffix} of the list of
  premises.
  
\item [$params~\vec p@1 \dots \vec p@n$] renames the innermost parameters of
  premises $1, \dots, n$ of some theorem.  An empty list of names may be given
  to skip positions, leaving the present parameters unchanged.
  
  Note that the default usage of case rules does \emph{not} directly expose
  parameters to the proof context (see also \S\ref{sec:cases-induct-meth}).
  
\item [$consumes~n$] declares the number of ``major premises'' of a rule,
  i.e.\ the number of facts to be consumed when it is applied by an
  appropriate proof method (cf.\ \S\ref{sec:cases-induct-meth}).  The default
  value of $consumes$ is $n = 1$, which is appropriate for the usual kind of
  cases and induction rules for inductive sets (cf.\ 
  \S\ref{sec:hol-inductive}).  Rules without any $consumes$ declaration given
  are treated as if $consumes~0$ had been specified.
  
  Note that explicit $consumes$ declarations are only rarely needed; this is
  already taken care of automatically by the higher-level $cases$ and $induct$
  declarations, see also \S\ref{sec:cases-induct-att}.

\end{descr}


\subsubsection{Proof methods}\label{sec:cases-induct-meth}

\indexisarmeth{cases}\indexisarmeth{induct}
\begin{matharray}{rcl}
  cases & : & \isarmeth \\
  induct & : & \isarmeth \\
\end{matharray}

The $cases$ and $induct$ methods provide a uniform interface to case analysis
and induction over datatypes, inductive sets, and recursive functions.  The
corresponding rules may be specified and instantiated in a casual manner.
Furthermore, these methods provide named local contexts that may be invoked
via the $\CASENAME$ proof command within the subsequent proof text (cf.\ 
\S\ref{sec:rule-cases}).  This accommodates compact proof texts even when
reasoning about large specifications.

Note that the full spectrum of this generic functionality is currently only
supported by Isabelle/HOL, when used in conjunction with advanced definitional
packages (see especially \S\ref{sec:hol-datatype} and
\S\ref{sec:hol-inductive}).

\begin{rail}
  'cases' spec
  ;
  'induct' spec
  ;

  spec: open? args rule? params?
  ;
  open: '(' 'open' ')'
  ;
  args: (insts * 'and') 
  ;
  rule: ('type' | 'set') ':' nameref | 'rule' ':' thmref
  ;
  params: 'of' ':' insts
  ;
\end{rail}

\begin{descr}
  
\item [$cases~insts~R~ps$] applies method $rule$ with an appropriate case
  distinction theorem, instantiated to the subjects $insts$.  Symbolic case
  names are bound according to the rule's local contexts.
  
  The rule is determined as follows, according to the facts and arguments
  passed to the $cases$ method:
  \begin{matharray}{llll}
    \Text{facts}    &       & \Text{arguments} & \Text{rule} \\\hline
                    & cases &           & \Text{classical case split} \\
                    & cases & t         & \Text{datatype exhaustion (type of $t$)} \\
    \edrv a \in A   & cases & \dots     & \Text{inductive set elimination (of $A$)} \\
    \dots           & cases & \dots ~ R & \Text{explicit rule $R$} \\
  \end{matharray}
  
  Several instantiations may be given, referring to the \emph{suffix} of
  premises of the case rule; within each premise, the \emph{prefix} of
  variables is instantiated.  In most situations, only a single term needs to
  be specified; this refers to the first variable of the last premise (it is
  usually the same for all cases).
  
  Additional parameters may be specified as $ps$; these are applied after the
  primary instantiation in the same manner as by the $of$ attribute (cf.\ 
  \S\ref{sec:pure-meth-att}).  This feature is rarely needed in practice; a
  typical application would be to specify additional arguments for rules
  stemming from parameterized inductive definitions (see also
  \S\ref{sec:hol-inductive}).
  
  The $open$ option causes the parameters of the new local contexts to be
  exposed to the current proof context.  Thus local variables stemming from
  distant parts of the theory development may be introduced in an implicit
  manner, which can be quite confusing to the reader.  Furthermore, this
  option may cause unwanted hiding of existing local variables, resulting in
  less robust proof texts.
  
\item [$induct~insts~R~ps$] is analogous to the $cases$ method, but refers to
  induction rules, which are determined as follows:
  \begin{matharray}{llll}
    \Text{facts}    &        & \Text{arguments} & \Text{rule} \\\hline
                    & induct & P ~ x ~ \dots & \Text{datatype induction (type of $x$)} \\
    \edrv x \in A   & induct & \dots         & \Text{set induction (of $A$)} \\
    \dots           & induct & \dots ~ R     & \Text{explicit rule $R$} \\
  \end{matharray}
  
  Several instantiations may be given, each referring to some part of a mutual
  inductive definition or datatype --- only related partial induction rules
  may be used together, though.  Any of the lists of terms $P, x, \dots$
  refers to the \emph{suffix} of variables present in the induction rule.
  This enables the writer to specify only induction variables, or both
  predicates and variables, for example.
  
  Additional parameters (including the $open$ option) may be given in the same
  way as for $cases$, see above.

\end{descr}

Above methods produce named local contexts (cf.\ \S\ref{sec:rule-cases}), as
determined by the instantiated rule \emph{before} it has been applied to the
internal proof state.\footnote{As a general principle, Isar proof text may
  never refer to parts of proof states directly.} Thus proper use of symbolic
cases usually require the rule to be instantiated fully, as far as the
emerging local contexts and subgoals are concerned.  In particular, for
induction both the predicates and variables have to be specified.  Otherwise
the $\CASENAME$ command would refuse to invoke cases containing schematic
variables.  Furthermore the resulting local goal statement is bound to the
term variable $\Var{case}$\indexisarvar{case} --- for each case where it is
fully specified.

The $\isarkeyword{print_cases}$ command (\S\ref{sec:rule-cases}) prints all
named cases present in the current proof state.

\medskip

It is important to note that there is a fundamental difference of the $cases$
and $induct$ methods in handling of non-atomic goal statements: $cases$ just
applies a certain rule in backward fashion, splitting the result into new
goals with the local contexts being augmented in a purely monotonic manner.

In contrast, $induct$ passes the full goal statement through the ``recursive''
course involved in the induction.  Thus the original statement is basically
replaced by separate copies, corresponding to the induction hypotheses and
conclusion; the original goal context is no longer available.  This behavior
allows \emph{strengthened induction predicates} to be expressed concisely as
meta-level rule statements, i.e.\ $\All{\vec x} \vec\phi \Imp \psi$ to
indicate ``variable'' parameters $\vec x$ and ``recursive'' assumptions
$\vec\phi$.  Also note that local definitions may be expressed as $\All{\vec
  x} n \equiv t[\vec x] \Imp \phi[n]$, with induction over $n$.

\medskip

Facts presented to either method are consumed according to the number of
``major premises'' of the rule involved (see also \S\ref{sec:cases-induct}),
which is usually $0$ for plain cases and induction rules of datatypes etc.\ 
and $1$ for rules of inductive sets and the like.  The remaining facts are
inserted into the goal verbatim before the actual $cases$ or $induct$ rule is
applied (thus facts may be even passed through an induction).

Note that whenever facts are present, the default rule selection scheme would
provide a ``set'' rule only, with the first fact consumed and the rest
inserted into the goal.  In order to pass all facts into a ``type'' rule
instead, one would have to specify this explicitly, e.g.\ by appending
``$type: name$'' to the method argument.


\subsubsection{Declaring rules}\label{sec:cases-induct-att}

\indexisarcmd{print-induct-rules}\indexisaratt{cases}\indexisaratt{induct}
\begin{matharray}{rcl}
  \isarcmd{print_induct_rules}^* & : & \isarkeep{theory~|~proof} \\
  cases & : & \isaratt \\
  induct & : & \isaratt \\
\end{matharray}

\begin{rail}
  'cases' spec
  ;
  'induct' spec
  ;

  spec: ('type' | 'set') ':' nameref
  ;
\end{rail}

\begin{descr}
  
\item [$\isarkeyword{print_induct_rules}$] prints cases and induct rules for
  sets and types of the current context.
  
\item [$cases$ and $induct$] (as attributes) augment the corresponding context
  of rules for reasoning about inductive sets and types, using the
  corresponding methods of the same name.  Certain definitional packages of
  object-logics usually declare emerging cases and induction rules as
  expected, so users rarely need to intervene.
  
  Manual rule declarations usually include the the $case_names$ and $ps$
  attributes to adjust names of cases and parameters of a rule (see
  \S\ref{sec:rule-cases}); the $consumes$ declaration is taken care of
  automatically: $consumes~0$ is specified for ``type'' rules and $consumes~1$
  for ``set'' rules.
  
\end{descr}

%%% Local Variables:
%%% mode: latex
%%% TeX-master: "isar-ref"
%%% End:
