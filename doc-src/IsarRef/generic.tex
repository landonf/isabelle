
\chapter{Generic tools and packages}\label{ch:gen-tools}

\section{Theory specification commands}

\subsection{Derived specifications}

\indexisarcmd{axiomatization}
\indexisarcmd{definition}\indexisaratt{defn}
\indexisarcmd{abbreviation}
\indexisarcmd{const-syntax}
\begin{matharray}{rcll}
  \isarcmd{axiomatization} & : & \isarkeep{local{\dsh}theory} & (axiomatic!)\\
  \isarcmd{definition} & : & \isarkeep{local{\dsh}theory} \\
  defn & : & \isaratt \\
  \isarcmd{abbreviation} & : & \isarkeep{local{\dsh}theory} \\
  \isarcmd{const_syntax} & : & \isarkeep{local{\dsh}theory} \\
\end{matharray}

These specification mechanisms provide a slightly more abstract view
than the underlying primitives of $\CONSTS$, $\DEFS$ (see
\S\ref{sec:consts}), and $\isarkeyword{axioms}$ (see
\S\ref{sec:axms-thms}).  In particular, type-inference is commonly
available, and result names need not be given.

\begin{rail}
  'axiomatization' locale? consts? ('where' specs)?
  ;
  'definition' locale? (constdecl? constdef +)
  ;
  'abbreviation' locale? mode? (constdecl? prop +)
  ;
  'const\_syntax' locale? mode? (nameref mixfix +)
  ;

  consts: ((name ('::' type)? structmixfix? | vars) + 'and')
  ;
  specs: (thmdecl? props + 'and')
  ;
\end{rail}

\begin{descr}
  
\item $\isarkeyword{axiomatization} ~ c@1 \dots c@n ~
  \isarkeyword{where} ~ A@1 \dots A@m$ introduces several constants
  simultaneously and states axiomatic properties for these.  The
  constants are marked as being specified once and for all, which
  prevents additional specifications being issued later on.
  
  Note that axiomatic specifications are only appropriate when
  declaring a new logical system.  Normal applications should only use
  definitional mechanisms!

\item $\isarkeyword{definition}~c~\isarkeyword{where}~eq$ produces an
  internal definition $c \equiv t$ according to the specification
  given as $eq$, which is then turned into a proven fact.  The given
  proposition may deviate from internal meta-level equality according
  to the rewrite rules declared as $defn$ by the object-logic.  This
  typically covers object-level equality $x = t$ and equivalence $A
  \leftrightarrow B$.  Users normally need not change the $defn$
  setup.
  
  Definitions may be presented with explicit arguments on the LHS, as
  well as additional conditions, e.g.\ $f\;x\;y = t$ instead of $f
  \equiv \lambda x\;y. t$ and $y \not= 0 \Imp g\;x\;y = u$ instead of
  an unguarded $g \equiv \lambda x\;y. u$.
  
  Multiple definitions are processed consecutively; no overloading is
  supported here.
  
\item $\isarkeyword{abbreviation}~c~\isarkeyword{where}~eq$ introduces
  a syntactic constant which is associated with a certain term
  according to the meta-level equality $eq$.
  
  Abbreviations participate in the usual type-inference process, but
  are expanded before the logic ever sees them.  Pretty printing of
  terms involves higher-order rewriting with rules stemming from
  reverted abbreviations.  This needs some care to avoid overlapping
  or looping syntactic replacements!
  
  The optional $mode$ specification restricts output to a particular
  print mode; using ``$input$'' here achieves the effect of one-way
  abbreviations.  The mode may also include an ``$output$'' qualifier
  that affects the concrete syntax declared for abbreviations, cf.\ 
  $\isarkeyword{syntax}$ in \S\ref{sec:syn-trans}.
  
\item $\isarkeyword{const_syntax}~c~mx$ associates mixfix syntax with
  an existing constant $c$.  This is a robust interface to the
  underlying $\isarkeyword{syntax}$ primitive (\S\ref{sec:syn-trans}).
  Type declaration and internal syntactic representation of given
  constants is retrieved from the context.
  
\end{descr}

Any of these specifications support an optional target locale context
(cf.\ \S\ref{sec:locale}).  In the latter case, constants being
introduced depend on certain fixed parameters of the locale context;
the constant name is qualified by the locale base name.  A syntactic
abbreviation takes care for convenient input and output of such terms,
making the parameters implicit and using the original short name.
Outside the locale context, the specified entities are available in
generalized form, with the parameters being open to explicit
instantiation.


\subsection{Axiomatic type classes}\label{sec:axclass}

\indexisarcmd{axclass}\indexisarmeth{intro-classes}
\begin{matharray}{rcl}
  \isarcmd{axclass} & : & \isartrans{theory}{theory} \\
  \isarcmd{instance} & : & \isartrans{theory}{proof(prove)} \\
  intro_classes & : & \isarmeth \\
\end{matharray}

Axiomatic type classes are provided by Isabelle/Pure as a \emph{definitional}
interface to type classes (cf.~\S\ref{sec:classes}).  Thus any object logic
may make use of this light-weight mechanism of abstract theories
\cite{Wenzel:1997:TPHOL}.  There is also a tutorial on using axiomatic type
classes in Isabelle \cite{isabelle-axclass} that is part of the standard
Isabelle documentation.

\begin{rail}
  'axclass' classdecl (axmdecl prop +)
  ;
  'instance' (nameref ('<' | subseteq) nameref | nameref '::' arity)
  ;
\end{rail}

\begin{descr}
  
\item [$\AXCLASS~c \subseteq \vec c~~axms$] defines an axiomatic type class as
  the intersection of existing classes, with additional axioms holding.  Class
  axioms may not contain more than one type variable.  The class axioms (with
  implicit sort constraints added) are bound to the given names.  Furthermore
  a class introduction rule is generated (being bound as
  $c_class{\dtt}intro$); this rule is employed by method $intro_classes$ to
  support instantiation proofs of this class.
  
  The ``axioms'' are stored as theorems according to the given name
  specifications, adding the class name $c$ as name space prefix; the same
  facts are also stored collectively as $c_class{\dtt}axioms$.
  
\item [$\INSTANCE~c@1 \subseteq c@2$ and $\INSTANCE~t :: (\vec s)s$] setup a
  goal stating a class relation or type arity.  The proof would usually
  proceed by $intro_classes$, and then establish the characteristic theorems
  of the type classes involved.  After finishing the proof, the theory will be
  augmented by a type signature declaration corresponding to the resulting
  theorem.

\item [$intro_classes$] repeatedly expands all class introduction rules of
  this theory.  Note that this method usually needs not be named explicitly,
  as it is already included in the default proof step (of $\PROOFNAME$ etc.).
  In particular, instantiation of trivial (syntactic) classes may be performed
  by a single ``$\DDOT$'' proof step.

\end{descr}


\subsection{Locales and local contexts}\label{sec:locale}

Locales are named local contexts, consisting of a list of declaration elements
that are modeled after the Isar proof context commands (cf.\
\S\ref{sec:proof-context}).


\subsubsection{Localized commands}

Existing locales may be augmented later on by adding new facts.  Note that the
actual context definition may not be changed!  Several theory commands that
produce facts in some way are available in ``localized'' versions, referring
to a named locale instead of the global theory context.

\indexouternonterm{locale}
\begin{rail}
  locale: '(' 'in' name ')'
  ;
\end{rail}

Emerging facts of localized commands are stored in two versions, both in the
target locale and the theory (after export).  The latter view produces a
qualified binding, using the locale name as a name space prefix.

For example, ``$\LEMMAS~(\IN~loc)~a = \vec b$'' retrieves facts $\vec b$ from
the locale context of $loc$ and augments its body by an appropriate
``$\isarkeyword{notes}$'' element (see below).  The exported view of $a$,
after discharging the locale context, is stored as $loc{.}a$ within the global
theory.  A localized goal ``$\LEMMANAME~(\IN~loc)~a:~\phi$'' works similarly,
only that the fact emerges through the subsequent proof, which may refer to
the full infrastructure of the locale context (covering local parameters with
typing and concrete syntax, assumptions, definitions etc.).  Most notably,
fact declarations of the locale are active during the proof as well (e.g.\ 
local $simp$ rules).

As a general principle, results exported from a locale context acquire
additional premises according to the specification.  Usually this is only a
single predicate according to the standard ``closed'' view of locale
specifications.


\subsubsection{Locale specifications}

\indexisarcmd{locale}\indexisarcmd{print-locale}\indexisarcmd{print-locales}
\begin{matharray}{rcl}
  \isarcmd{locale} & : & \isartrans{theory}{local{\dsh}theory} \\
  \isarcmd{print_locale}^* & : & \isarkeep{theory~|~proof} \\
  \isarcmd{print_locales}^* & : & \isarkeep{theory~|~proof} \\
\end{matharray}

\indexouternonterm{contextexpr}\indexouternonterm{contextelem}
\indexisarelem{fixes}\indexisarelem{constrains}\indexisarelem{assumes}
\indexisarelem{defines}\indexisarelem{notes}\indexisarelem{includes}

\begin{rail}
  'locale' ('(open)')? name ('=' localeexpr)?
  ;
  'print\_locale' '!'? localeexpr
  ;
  localeexpr: ((contextexpr '+' (contextelem+)) | contextexpr | (contextelem+))
  ;

  contextexpr: nameref | '(' contextexpr ')' |
  (contextexpr (name mixfix? +)) | (contextexpr + '+')
  ;
  contextelem: fixes | constrains | assumes | defines | notes | includes
  ;
  fixes: 'fixes' ((name ('::' type)? structmixfix? | vars) + 'and')
  ;
  constrains: 'constrains' (name '::' type + 'and')
  ;
  assumes: 'assumes' (thmdecl? props + 'and')
  ;
  defines: 'defines' (thmdecl? prop proppat? + 'and')
  ;
  notes: 'notes' (thmdef? thmrefs + 'and')
  ;
  includes: 'includes' contextexpr
  ;
\end{rail}

\begin{descr}
  
\item [$\LOCALE~loc~=~import~+~body$] defines a new locale $loc$ as a context
  consisting of a certain view of existing locales ($import$) plus some
  additional elements ($body$).  Both $import$ and $body$ are optional; the
  degenerate form $\LOCALE~loc$ defines an empty locale, which may still be
  useful to collect declarations of facts later on.  Type-inference on locale
  expressions automatically takes care of the most general typing that the
  combined context elements may acquire.

  The $import$ consists of a structured context expression, consisting of
  references to existing locales, renamed contexts, or merged contexts.
  Renaming uses positional notation: $c~\vec x$ means that (a prefix of) the
  fixed parameters of context $c$ are named according to $\vec x$; a
  ``\texttt{_}'' (underscore) \indexisarthm{_@\texttt{_}} means to skip that
  position.  Renaming by default deletes existing syntax.  Optionally,
  new syntax may by specified with a mixfix annotation.  Note that the
  special syntax declared with ``$(structure)$'' (see below) is
  neither deleted nor can it be changed.
  Merging proceeds from left-to-right, suppressing any duplicates stemming
  from different paths through the import hierarchy.

  The $body$ consists of basic context elements, further context expressions
  may be included as well.

  \begin{descr}

  \item [$\FIXES{~x::\tau~(mx)}$] declares a local parameter of type $\tau$
    and mixfix annotation $mx$ (both are optional).  The special syntax
    declaration ``$(structure)$'' means that $x$ may be referenced
    implicitly in this context.

  \item [$\CONSTRAINS{~x::\tau}$] introduces a type constraint $\tau$
    on the local parameter $x$.

  \item [$\ASSUMES{a}{\vec\phi}$] introduces local premises, similar to
    $\ASSUMENAME$ within a proof (cf.\ \S\ref{sec:proof-context}).

  \item [$\DEFINES{a}{x \equiv t}$] defines a previously declared parameter.
    This is close to $\DEFNAME$ within a proof (cf.\
    \S\ref{sec:proof-context}), but $\DEFINESNAME$ takes an equational
    proposition instead of variable-term pair.  The left-hand side of the
    equation may have additional arguments, e.g.\ ``$\DEFINES{}{f~\vec x
      \equiv t}$''.

  \item [$\NOTES{a}{\vec b}$] reconsiders facts within a local context.  Most
    notably, this may include arbitrary declarations in any attribute
    specifications included here, e.g.\ a local $simp$ rule.

  \item [$\INCLUDES{c}$] copies the specified context in a statically scoped
    manner.  Only available in the long goal format of \S\ref{sec:goals}.

    In contrast, the initial $import$ specification of a locale expression
    maintains a dynamic relation to the locales being referenced (benefiting
    from any later fact declarations in the obvious manner).
  \end{descr}
  
  Note that ``$\IS{p}$'' patterns given in the syntax of $\ASSUMESNAME$ and
  $\DEFINESNAME$ above are illegal in locale definitions.  In the long goal
  format of \S\ref{sec:goals}, term bindings may be included as expected,
  though.
  
  \medskip By default, locale specifications are ``closed up'' by turning the
  given text into a predicate definition $loc_axioms$ and deriving the
  original assumptions as local lemmas (modulo local definitions).  The
  predicate statement covers only the newly specified assumptions, omitting
  the content of included locale expressions.  The full cumulative view is
  only provided on export, involving another predicate $loc$ that refers to
  the complete specification text.
  
  In any case, the predicate arguments are those locale parameters that
  actually occur in the respective piece of text.  Also note that these
  predicates operate at the meta-level in theory, but the locale packages
  attempts to internalize statements according to the object-logic setup
  (e.g.\ replacing $\Forall$ by $\forall$, and $\Imp$ by $\imp$ in HOL; see
  also \S\ref{sec:object-logic}).  Separate introduction rules
  $loc_axioms.intro$ and $loc.intro$ are declared as well.
  
  The $(open)$ option of a locale specification prevents both the current
  $loc_axioms$ and cumulative $loc$ predicate constructions.  Predicates are
  also omitted for empty specification texts.

\item [$\isarkeyword{print_locale}~import~+~body$] prints the specified locale
  expression in a flattened form.  The notable special case
  $\isarkeyword{print_locale}~loc$ just prints the contents of the named
  locale, but keep in mind that type-inference will normalize type variables
  according to the usual alphabetical order.  The command omits
  $\isarkeyword{notes}$ elements by default.  Use
  $\isarkeyword{print_locale}!$ to get them included.

\item [$\isarkeyword{print_locales}$] prints the names of all locales of the
  current theory.

\end{descr}


\subsubsection{Interpretation of locales}

Locale expressions (more precisely, \emph{context expressions}) may be
instantiated, and the instantiated facts added to the current context.
This requires a proof of the instantiated specification and is called
\emph{locale interpretation}.  Interpretation is possible in theories
and locales
(command $\isarcmd{interpretation}$) and also in proof contexts
($\isarcmd{interpret}$).

\indexisarcmd{interpretation}\indexisarcmd{interpret}
\indexisarcmd{print-interps}
\begin{matharray}{rcl}
  \isarcmd{interpretation} & : & \isartrans{theory}{proof(prove)} \\
  \isarcmd{interpret} & : & \isartrans{proof(state) ~|~ proof(chain)}{proof(prove)} \\
  \isarcmd{print_interps}^* & : &  \isarkeep{theory~|~proof} \\
\end{matharray}

\indexouternonterm{interp}

\railalias{printinterps}{print\_interps}
\railterm{printinterps}

\begin{rail}
  'interpretation' (interp | name ('<' | subseteq) contextexp)
  ;
  'interpret' interp
  ;
  printinterps '!'? name
  ;
  interp: thmdecl? contextexpr ('[' (inst+) ']')?
  ;
\end{rail}


\begin{descr}

\item [$\isarcmd{interpretation}~expr~insts$]

  The first form of $\isarcmd{interpretation}$ interprets $expr$
  in the theory.  The instantiation is given as a list of
  terms $insts$ and is positional.
  All parameters must receive an instantiation term --- with the
  exception of defined parameters.  These are, if omitted, derived
  from the defining equation and other instantiations.  Use ``\_'' to
  omit an instantiation term.  Free variables are automatically
  generalized.

  The command generates proof obligations for the instantiated
  specifications (assumes and defines elements).  Once these are
  discharged by the user, instantiated facts are added to the theory in
  a post-processing phase.

  The command is aware of interpretations already active in the
  theory.  No proof obligations are generated for those, neither is
  post-processing applied to their facts.  This avoids duplication of
  interpreted facts, in particular.  Note that, in the case of a
  locale with import, parts of the interpretation may already be
  active.  The command will only generate proof obligations and add
  facts for new parts.

  The context expression may be preceded by a name and/or attributes.
  These take effect in the post-processing of facts.  The name is used
  to prefix fact names, for example to avoid accidental hiding of
  other facts.  Attributes are applied after attributes of the
  interpreted facts.

  Adding facts to locales has the
  effect of adding interpreted facts to the theory for all active
  interpretations also.  That is, interpretations dynamically
  participate in any facts added to locales.

\item [$\isarcmd{interpretation}~name~\subseteq~expr$]

  This form of the command interprets $expr$ in the locale $name$.  It
  requires a proof that the specification of $name$ implies the
  specification of $expr$.  As in the localized version of the theorem
  command, the proof is in the context of $name$.  After the proof
  obligation has been dischared, the facts of $expr$
  become part of locale $name$ as \emph{derived} context elements and
  are available when the context $name$ is subsequently entered.
  Note that, like import, this is dynamic: facts added to a locale
  part of $expr$ after interpretation become also available in
  $name$.  Like facts
  of renamed context elements, facts obtained by interpretation may be
  accessed by prefixing with the parameter renaming (where the parameters
  are separated by `\_').

  Unlike interpretation in theories, instantiation is confined to the
  renaming of parameters, which may be specified as part of the context
  expression $expr$.  Using defined parameters in $name$ one may
  achieve an effect similar to instantiation, though.

  Only specification fragments of $expr$ that are not already part of
  $name$ (be it imported, derived or a derived fragment of the import)
  are considered by interpretation.  This enables circular
  interpretations.

  If interpretations of $name$ exist in the current theory, the
  command adds interpretations for $expr$ as well, with the same
  prefix and attributes, although only for fragments of $expr$ that
  are not interpreted in the theory already.

\item [$\isarcmd{interpret}~expr~insts$]
  interprets $expr$ in the proof context and is otherwise similar to
  interpretation in theories.  Free variables in instantiations are not
  generalized, however.

\item [$\isarcmd{print_interps}~loc$]
  prints the interpretations of a particular locale $loc$ that are
  active in the current context, either theory or proof context.  The
  exclamation point argument triggers printing of
  \emph{witness} theorems justifying interpretations.  These are
  normally omitted from the output.

  
\end{descr}

\begin{warn}
  Since attributes are applied to interpreted theorems, interpretation
  may modify the current simpset and claset.  Take this into
  account when choosing attributes for local theorems.
\end{warn}

\begin{warn}
  An interpretation in a theory may subsume previous interpretations.
  This happens if the same specification fragment is interpreted twice
  and the instantiation of the second interpretation is more general
  than the interpretation of the first.  A warning
  is issued, since it is likely that these could have been generalized
  in the first place.  The locale package does not attempt to remove
  subsumed interpretations.  This situation is normally harmless, but
  note that $blast$ gets confused by the presence of multiple axclass
  instances of a rule.
\end{warn}


\subsubsection{Constructive type classes}

A special case of locales are constructive type classes.
Constructive type classes
consist of a locale with \emph{exactly one} type variable
and an corresponding axclass.

\indexisarcmd{instance}\indexisarcmd{class}\indexisarcmd{print-classes}
\begin{matharray}{rcl}
  \isarcmd{class} & : & \isartrans{theory}{local{\dsh}theory} \\
  \isarcmd{instance} & : & \isartrans{theory}{proof(prove)} \\
  \isarcmd{print_classes}^* & : & \isarkeep{theory~|~proof} \\
\end{matharray}

\begin{rail}
  'class' name '=' classexpr
  ;
  'instance' (instarity | instsubsort)
  ;
  'print\_classes'
  ;

  classexpr: ((superclassexpr '+' (contextelem+)) | superclassexpr | (contextelem+))
  ;
  instarity: (axmdecl)? (nameref '::' arity + 'and') (axmdecl prop +)?
  ;
  instsubsort: nameref ('<' | subseteq) sort
  ;
  superclassexpr: nameref | (nameref '+' superclassexpr)
  ;
\end{rail}

\begin{descr}

\item [$\CLASS~c = superclasses~+~body$] defines a new class $c$,
  inheriting from $superclasses$. Simultaneously, a locale
  named $c$ is introduced, inheriting from the locales
  corresponding to $superclasses$; also, an axclass
  named $c$, inheriting from the axclasses corresponding to
  $superclasses$. $\FIXESNAME$ in $body$ are lifted
  to the theory toplevel, constraining
  the free type variable to sort $c$ and stripping local syntax.
  $\ASSUMESNAME$ in $body$ are also lifted, 
  constraining
  the free type variable to sort $c$.

\item [$\INSTANCE~a: \vec{arity}~\vec{defs}$]
  sets up a goal stating type arities.  The proof would usually
  proceed by $intro_classes$, and then establish the characteristic theorems
  of the type classes involved.
  The $defs$, if given, must correspond to the class parameters
  involved in the $arities$ and are introduces in the theory
  before proof. Name and attributes given after the $\INSTANCE$
  command refer to \emph{all} definitions as a whole.
  After finishing the proof, the theory will be
  augmented by a type signature declaration corresponding to the
  resulting theorems.
  Note that this $\isarcmd{instance}$ command is different
  from primitive axclass $\isarcmd{instance}$ (see \ref{sec:axclass}).
  
\item [$\INSTANCE~c \subseteq \vec{c}$] sets up a
  goal stating 
  the interpretation of the locale corresponding to $c$
  in the merge of all locales corresponding to $\vec{c}$.
  After finishing the proof, it is automatically lifted to
  prove the additional class relation $c \subseteq \vec{c}$.

\item [$\isarkeyword{print_classes}$] prints the names of all classes
  in the current theory.

\end{descr}


\section{Derived proof schemes}

\subsection{Generalized elimination}\label{sec:obtain}

\indexisarcmd{obtain}\indexisarcmd{guess}
\begin{matharray}{rcl}
  \isarcmd{obtain} & : & \isartrans{proof(state)}{proof(prove)} \\
  \isarcmd{guess}^* & : & \isartrans{proof(state)}{proof(prove)} \\
\end{matharray}

Generalized elimination means that additional elements with certain properties
may be introduced in the current context, by virtue of a locally proven
``soundness statement''.  Technically speaking, the $\OBTAINNAME$ language
element is like a declaration of $\FIXNAME$ and $\ASSUMENAME$ (see also see
\S\ref{sec:proof-context}), together with a soundness proof of its additional
claim.  According to the nature of existential reasoning, assumptions get
eliminated from any result exported from the context later, provided that the
corresponding parameters do \emph{not} occur in the conclusion.

\begin{rail}
  'obtain' parname? (vars + 'and') 'where' (props + 'and')
  ;
  'guess' (vars + 'and')
  ;
\end{rail}

$\OBTAINNAME$ is defined as a derived Isar command as follows, where $\vec b$
shall refer to (optional) facts indicated for forward chaining.
\begin{matharray}{l}
  \langle facts~\vec b\rangle \\
  \OBTAIN{\vec x}{a}{\vec \phi}~~\langle proof\rangle \equiv {} \\[1ex]
  \quad \HAVE{}{\All{thesis} (\All{\vec x} \vec\phi \Imp thesis) \Imp thesis} \\
  \quad \PROOF{succeed} \\
  \qquad \FIX{thesis} \\
  \qquad \ASSUME{that~[intro?]}{\All{\vec x} \vec\phi \Imp thesis} \\
  \qquad \THUS{}{thesis} \\
  \quad\qquad \APPLY{-} \\
  \quad\qquad \USING{\vec b}~~\langle proof\rangle \\
  \quad \QED{} \\
  \quad \FIX{\vec x}~\ASSUMENAME^\ast~a\colon~\vec\phi \\
\end{matharray}

Typically, the soundness proof is relatively straight-forward, often just by
canonical automated tools such as ``$\BY{simp}$'' or ``$\BY{blast}$''.
Accordingly, the ``$that$'' reduction above is declared as simplification and
introduction rule.

In a sense, $\OBTAINNAME$ represents at the level of Isar proofs what would be
meta-logical existential quantifiers and conjunctions.  This concept has a
broad range of useful applications, ranging from plain elimination (or
introduction) of object-level existential and conjunctions, to elimination
over results of symbolic evaluation of recursive definitions, for example.
Also note that $\OBTAINNAME$ without parameters acts much like $\HAVENAME$,
where the result is treated as a genuine assumption.

An alternative name to be used instead of ``$that$'' above may be
given in parentheses.

\medskip

The improper variant $\isarkeyword{guess}$ is similar to $\OBTAINNAME$, but
derives the obtained statement from the course of reasoning!  The proof starts
with a fixed goal $thesis$.  The subsequent proof may refine this to anything
of the form like $\All{\vec x} \vec\phi \Imp thesis$, but must not introduce
new subgoals.  The final goal state is then used as reduction rule for the
obtain scheme described above.  Obtained parameters $\vec x$ are marked as
internal by default, which prevents the proof context from being polluted by
ad-hoc variables.  The variable names and type constraints given as arguments
for $\isarkeyword{guess}$ specify a prefix of obtained parameters explicitly
in the text.

It is important to note that the facts introduced by $\OBTAINNAME$ and
$\isarkeyword{guess}$ may not be polymorphic: any type-variables occurring
here are fixed in the present context!


\subsection{Calculational reasoning}\label{sec:calculation}

\indexisarcmd{also}\indexisarcmd{finally}
\indexisarcmd{moreover}\indexisarcmd{ultimately}
\indexisarcmd{print-trans-rules}
\indexisaratt{trans}\indexisaratt{sym}\indexisaratt{symmetric}
\begin{matharray}{rcl}
  \isarcmd{also} & : & \isartrans{proof(state)}{proof(state)} \\
  \isarcmd{finally} & : & \isartrans{proof(state)}{proof(chain)} \\
  \isarcmd{moreover} & : & \isartrans{proof(state)}{proof(state)} \\
  \isarcmd{ultimately} & : & \isartrans{proof(state)}{proof(chain)} \\
  \isarcmd{print_trans_rules}^* & : & \isarkeep{theory~|~proof} \\
  trans & : & \isaratt \\
  sym & : & \isaratt \\
  symmetric & : & \isaratt \\
\end{matharray}

Calculational proof is forward reasoning with implicit application of
transitivity rules (such those of $=$, $\leq$, $<$).  Isabelle/Isar maintains
an auxiliary register $calculation$\indexisarthm{calculation} for accumulating
results obtained by transitivity composed with the current result.  Command
$\ALSO$ updates $calculation$ involving $this$, while $\FINALLY$ exhibits the
final $calculation$ by forward chaining towards the next goal statement.  Both
commands require valid current facts, i.e.\ may occur only after commands that
produce theorems such as $\ASSUMENAME$, $\NOTENAME$, or some finished proof of
$\HAVENAME$, $\SHOWNAME$ etc.  The $\MOREOVER$ and $\ULTIMATELY$ commands are
similar to $\ALSO$ and $\FINALLY$, but only collect further results in
$calculation$ without applying any rules yet.

Also note that the implicit term abbreviation ``$\dots$'' has its canonical
application with calculational proofs.  It refers to the argument of the
preceding statement. (The argument of a curried infix expression happens to be
its right-hand side.)

Isabelle/Isar calculations are implicitly subject to block structure in the
sense that new threads of calculational reasoning are commenced for any new
block (as opened by a local goal, for example).  This means that, apart from
being able to nest calculations, there is no separate \emph{begin-calculation}
command required.

\medskip

The Isar calculation proof commands may be defined as follows:\footnote{We
  suppress internal bookkeeping such as proper handling of block-structure.}
\begin{matharray}{rcl}
  \ALSO@0 & \equiv & \NOTE{calculation}{this} \\
  \ALSO@{n+1} & \equiv & \NOTE{calculation}{trans~[OF~calculation~this]} \\[0.5ex]
  \FINALLY & \equiv & \ALSO~\FROM{calculation} \\
  \MOREOVER & \equiv & \NOTE{calculation}{calculation~this} \\
  \ULTIMATELY & \equiv & \MOREOVER~\FROM{calculation} \\
\end{matharray}

\begin{rail}
  ('also' | 'finally') ('(' thmrefs ')')?
  ;
  'trans' (() | 'add' | 'del')
  ;
\end{rail}

\begin{descr}

\item [$\ALSO~(\vec a)$] maintains the auxiliary $calculation$ register as
  follows.  The first occurrence of $\ALSO$ in some calculational thread
  initializes $calculation$ by $this$. Any subsequent $\ALSO$ on the same
  level of block-structure updates $calculation$ by some transitivity rule
  applied to $calculation$ and $this$ (in that order).  Transitivity rules are
  picked from the current context, unless alternative rules are given as
  explicit arguments.

\item [$\FINALLY~(\vec a)$] maintaining $calculation$ in the same way as
  $\ALSO$, and concludes the current calculational thread.  The final result
  is exhibited as fact for forward chaining towards the next goal. Basically,
  $\FINALLY$ just abbreviates $\ALSO~\FROM{calculation}$.  Note that
  ``$\FINALLY~\SHOW{}{\Var{thesis}}~\DOT$'' and
  ``$\FINALLY~\HAVE{}{\phi}~\DOT$'' are typical idioms for concluding
  calculational proofs.

\item [$\MOREOVER$ and $\ULTIMATELY$] are analogous to $\ALSO$ and $\FINALLY$,
  but collect results only, without applying rules.

\item [$\isarkeyword{print_trans_rules}$] prints the list of transitivity
  rules (for calculational commands $\ALSO$ and $\FINALLY$) and symmetry rules
  (for the $symmetric$ operation and single step elimination patters) of the
  current context.

\item [$trans$] declares theorems as transitivity rules.

\item [$sym$] declares symmetry rules.

\item [$symmetric$] resolves a theorem with some rule declared as $sym$ in the
  current context.  For example, ``$\ASSUME{[symmetric]}{x = y}$'' produces a
  swapped fact derived from that assumption.

  In structured proof texts it is often more appropriate to use an explicit
  single-step elimination proof, such as ``$\ASSUME{}{x = y}~\HENCE{}{y =
    x}~\DDOT$''.  The very same rules known to $symmetric$ are declared as
  $elim?$ as well.

\end{descr}


\section{Proof tools}

\subsection{Miscellaneous methods and attributes}\label{sec:misc-meth-att}

\indexisarmeth{unfold}\indexisarmeth{fold}\indexisarmeth{insert}
\indexisarmeth{erule}\indexisarmeth{drule}\indexisarmeth{frule}
\indexisarmeth{fail}\indexisarmeth{succeed}
\begin{matharray}{rcl}
  unfold & : & \isarmeth \\
  fold & : & \isarmeth \\
  insert & : & \isarmeth \\[0.5ex]
  erule^* & : & \isarmeth \\
  drule^* & : & \isarmeth \\
  frule^* & : & \isarmeth \\
  succeed & : & \isarmeth \\
  fail & : & \isarmeth \\
\end{matharray}

\begin{rail}
  ('fold' | 'unfold' | 'insert') thmrefs
  ;
  ('erule' | 'drule' | 'frule') ('('nat')')? thmrefs
  ;
\end{rail}

\begin{descr}
  
\item [$unfold~\vec a$ and $fold~\vec a$] expand (or fold back again)
  the given definitions throughout all goals; any chained facts
  provided are inserted into the goal and subject to rewriting as
  well.

\item [$insert~\vec a$] inserts theorems as facts into all goals of the proof
  state.  Note that current facts indicated for forward chaining are ignored.

\item [$erule~\vec a$, $drule~\vec a$, and $frule~\vec a$] are similar to the
  basic $rule$ method (see \S\ref{sec:pure-meth-att}), but apply rules by
  elim-resolution, destruct-resolution, and forward-resolution, respectively
  \cite{isabelle-ref}.  The optional natural number argument (default $0$)
  specifies additional assumption steps to be performed here.

  Note that these methods are improper ones, mainly serving for
  experimentation and tactic script emulation.  Different modes of basic rule
  application are usually expressed in Isar at the proof language level,
  rather than via implicit proof state manipulations.  For example, a proper
  single-step elimination would be done using the plain $rule$ method, with
  forward chaining of current facts.

\item [$succeed$] yields a single (unchanged) result; it is the identity of
  the ``\texttt{,}'' method combinator (cf.\ \S\ref{sec:syn-meth}).

\item [$fail$] yields an empty result sequence; it is the identity of the
  ``\texttt{|}'' method combinator (cf.\ \S\ref{sec:syn-meth}).

\end{descr}

\indexisaratt{tagged}\indexisaratt{untagged}
\indexisaratt{THEN}\indexisaratt{COMP}
\indexisaratt{unfolded}\indexisaratt{folded}
\indexisaratt{standard}\indexisarattof{Pure}{elim-format}
\indexisaratt{no-vars}
\begin{matharray}{rcl}
  tagged & : & \isaratt \\
  untagged & : & \isaratt \\[0.5ex]
  THEN & : & \isaratt \\
  COMP & : & \isaratt \\[0.5ex]
  unfolded & : & \isaratt \\
  folded & : & \isaratt \\[0.5ex]
  elim_format & : & \isaratt \\
  standard^* & : & \isaratt \\
  no_vars^* & : & \isaratt \\
\end{matharray}

\begin{rail}
  'tagged' (nameref+)
  ;
  'untagged' name
  ;
  ('THEN' | 'COMP') ('[' nat ']')? thmref
  ;
  ('unfolded' | 'folded') thmrefs
  ;
\end{rail}

\begin{descr}

\item [$tagged~name~args$ and $untagged~name$] add and remove $tags$ of some
  theorem.  Tags may be any list of strings that serve as comment for some
  tools (e.g.\ $\LEMMANAME$ causes the tag ``$lemma$'' to be added to the
  result).  The first string is considered the tag name, the rest its
  arguments.  Note that untag removes any tags of the same name.

\item [$THEN~a$ and $COMP~a$] compose rules by resolution.  $THEN$ resolves
  with the first premise of $a$ (an alternative position may be also
  specified); the $COMP$ version skips the automatic lifting process that is
  normally intended (cf.\ \texttt{RS} and \texttt{COMP} in
  \cite[\S5]{isabelle-ref}).
  
\item [$unfolded~\vec a$ and $folded~\vec a$] expand and fold back
  again the given definitions throughout a rule.

\item [$elim_format$] turns a destruction rule into elimination rule format,
  by resolving with the rule $\PROP A \Imp (\PROP A \Imp \PROP B) \Imp \PROP
  B$.
  
  Note that the Classical Reasoner (\S\ref{sec:classical}) provides its own
  version of this operation.

\item [$standard$] puts a theorem into the standard form of object-rules at
  the outermost theory level.  Note that this operation violates the local
  proof context (including active locales).

\item [$no_vars$] replaces schematic variables by free ones; this is mainly
  for tuning output of pretty printed theorems.

\end{descr}


\subsection{Further tactic emulations}\label{sec:tactics}

The following improper proof methods emulate traditional tactics.  These admit
direct access to the goal state, which is normally considered harmful!  In
particular, this may involve both numbered goal addressing (default 1), and
dynamic instantiation within the scope of some subgoal.

\begin{warn}
  Dynamic instantiations refer to universally quantified parameters of
  a subgoal (the dynamic context) rather than fixed variables and term
  abbreviations of a (static) Isar context.
\end{warn}

Tactic emulation methods, unlike their ML counterparts, admit
simultaneous instantiation from both dynamic and static contexts.  If
names occur in both contexts goal parameters hide locally fixed
variables.  Likewise, schematic variables refer to term abbreviations,
if present in the static context.  Otherwise the schematic variable is
interpreted as a schematic variable and left to be solved by unification
with certain parts of the subgoal.

Note that the tactic emulation proof methods in Isabelle/Isar are consistently
named $foo_tac$.  Note also that variable names occurring on left hand sides
of instantiations must be preceded by a question mark if they coincide with
a keyword or contain dots.
This is consistent with the attribute $where$ (see \S\ref{sec:pure-meth-att}).

\indexisarmeth{rule-tac}\indexisarmeth{erule-tac}
\indexisarmeth{drule-tac}\indexisarmeth{frule-tac}
\indexisarmeth{cut-tac}\indexisarmeth{thin-tac}
\indexisarmeth{subgoal-tac}\indexisarmeth{rename-tac}
\indexisarmeth{rotate-tac}\indexisarmeth{tactic}
\begin{matharray}{rcl}
  rule_tac^* & : & \isarmeth \\
  erule_tac^* & : & \isarmeth \\
  drule_tac^* & : & \isarmeth \\
  frule_tac^* & : & \isarmeth \\
  cut_tac^* & : & \isarmeth \\
  thin_tac^* & : & \isarmeth \\
  subgoal_tac^* & : & \isarmeth \\
  rename_tac^* & : & \isarmeth \\
  rotate_tac^* & : & \isarmeth \\
  tactic^* & : & \isarmeth \\
\end{matharray}

\railalias{ruletac}{rule\_tac}
\railterm{ruletac}

\railalias{eruletac}{erule\_tac}
\railterm{eruletac}

\railalias{druletac}{drule\_tac}
\railterm{druletac}

\railalias{fruletac}{frule\_tac}
\railterm{fruletac}

\railalias{cuttac}{cut\_tac}
\railterm{cuttac}

\railalias{thintac}{thin\_tac}
\railterm{thintac}

\railalias{subgoaltac}{subgoal\_tac}
\railterm{subgoaltac}

\railalias{renametac}{rename\_tac}
\railterm{renametac}

\railalias{rotatetac}{rotate\_tac}
\railterm{rotatetac}

\begin{rail}
  ( ruletac | eruletac | druletac | fruletac | cuttac | thintac ) goalspec?
  ( insts thmref | thmrefs )
  ;
  subgoaltac goalspec? (prop +)
  ;
  renametac goalspec? (name +)
  ;
  rotatetac goalspec? int?
  ;
  'tactic' text
  ;

  insts: ((name '=' term) + 'and') 'in'
  ;
\end{rail}

\begin{descr}

\item [$rule_tac$ etc.] do resolution of rules with explicit instantiation.
  This works the same way as the ML tactics \texttt{res_inst_tac} etc. (see
  \cite[\S3]{isabelle-ref}).

  Multiple rules may be only given if there is no instantiation; then
  $rule_tac$ is the same as \texttt{resolve_tac} in ML (see
  \cite[\S3]{isabelle-ref}).

\item [$cut_tac$] inserts facts into the proof state as assumption of a
  subgoal, see also \texttt{cut_facts_tac} in \cite[\S3]{isabelle-ref}.  Note
  that the scope of schematic variables is spread over the main goal
  statement.  Instantiations may be given as well, see also ML tactic
  \texttt{cut_inst_tac} in \cite[\S3]{isabelle-ref}.

\item [$thin_tac~\phi$] deletes the specified assumption from a subgoal; note
  that $\phi$ may contain schematic variables.  See also \texttt{thin_tac} in
  \cite[\S3]{isabelle-ref}.

\item [$subgoal_tac~\phi$] adds $\phi$ as an assumption to a subgoal.  See
  also \texttt{subgoal_tac} and \texttt{subgoals_tac} in
  \cite[\S3]{isabelle-ref}.

\item [$rename_tac~\vec x$] renames parameters of a goal according to the list
  $\vec x$, which refers to the \emph{suffix} of variables.

\item [$rotate_tac~n$] rotates the assumptions of a goal by $n$ positions:
  from right to left if $n$ is positive, and from left to right if $n$ is
  negative; the default value is $1$.  See also \texttt{rotate_tac} in
  \cite[\S3]{isabelle-ref}.

\item [$tactic~text$] produces a proof method from any ML text of type
  \texttt{tactic}.  Apart from the usual ML environment and the current
  implicit theory context, the ML code may refer to the following locally
  bound values:

{\footnotesize\begin{verbatim}
val ctxt  : Proof.context
val facts : thm list
val thm   : string -> thm
val thms  : string -> thm list
\end{verbatim}}
  Here \texttt{ctxt} refers to the current proof context, \texttt{facts}
  indicates any current facts for forward-chaining, and
  \texttt{thm}~/~\texttt{thms} retrieve named facts (including global
  theorems) from the context.
\end{descr}


\subsection{The Simplifier}\label{sec:simplifier}

\subsubsection{Simplification methods}

\indexisarmeth{simp}\indexisarmeth{simp-all}
\begin{matharray}{rcl}
  simp & : & \isarmeth \\
  simp_all & : & \isarmeth \\
\end{matharray}

\indexouternonterm{simpmod}
\begin{rail}
  ('simp' | 'simp\_all') ('!' ?) opt? (simpmod *)
  ;

  opt: '(' ('no\_asm' | 'no\_asm\_simp' | 'no\_asm\_use' | 'asm\_lr' | 'depth\_limit' ':' nat) ')'
  ;
  simpmod: ('add' | 'del' | 'only' | 'cong' (() | 'add' | 'del') |
    'split' (() | 'add' | 'del')) ':' thmrefs
  ;
\end{rail}

\begin{descr}

\item [$simp$] invokes Isabelle's simplifier, after declaring additional rules
  according to the arguments given.  Note that the \railtterm{only} modifier
  first removes all other rewrite rules, congruences, and looper tactics
  (including splits), and then behaves like \railtterm{add}.

  \medskip The \railtterm{cong} modifiers add or delete Simplifier congruence
  rules (see also \cite{isabelle-ref}), the default is to add.

  \medskip The \railtterm{split} modifiers add or delete rules for the
  Splitter (see also \cite{isabelle-ref}), the default is to add.  This works
  only if the Simplifier method has been properly setup to include the
  Splitter (all major object logics such HOL, HOLCF, FOL, ZF do this already).

\item [$simp_all$] is similar to $simp$, but acts on all goals (backwards from
  the last to the first one).

\end{descr}

By default the Simplifier methods take local assumptions fully into account,
using equational assumptions in the subsequent normalization process, or
simplifying assumptions themselves (cf.\ \texttt{asm_full_simp_tac} in
\cite[\S10]{isabelle-ref}).  In structured proofs this is usually quite well
behaved in practice: just the local premises of the actual goal are involved,
additional facts may be inserted via explicit forward-chaining (using $\THEN$,
$\FROMNAME$ etc.).  The full context of assumptions is only included if the
``$!$'' (bang) argument is given, which should be used with some care, though.

Additional Simplifier options may be specified to tune the behavior further
(mostly for unstructured scripts with many accidental local facts):
``$(no_asm)$'' means assumptions are ignored completely (cf.\ 
\texttt{simp_tac}), ``$(no_asm_simp)$'' means assumptions are used in the
simplification of the conclusion but are not themselves simplified (cf.\ 
\texttt{asm_simp_tac}), and ``$(no_asm_use)$'' means assumptions are
simplified but are not used in the simplification of each other or the
conclusion (cf.\ \texttt{full_simp_tac}).  For compatibility reasons, there is
also an option ``$(asm_lr)$'', which means that an assumption is only used for
simplifying assumptions which are to the right of it (cf.\ 
\texttt{asm_lr_simp_tac}).  Giving an option ``$(depth_limit: n)$'' limits the
number of recursive invocations of the simplifier during conditional
rewriting.

\medskip

The Splitter package is usually configured to work as part of the Simplifier.
The effect of repeatedly applying \texttt{split_tac} can be simulated by
``$(simp~only\colon~split\colon~\vec a)$''.  There is also a separate $split$
method available for single-step case splitting.


\subsubsection{Declaring rules}

\indexisarcmd{print-simpset}
\indexisaratt{simp}\indexisaratt{split}\indexisaratt{cong}
\begin{matharray}{rcl}
  \isarcmd{print_simpset}^* & : & \isarkeep{theory~|~proof} \\
  simp & : & \isaratt \\
  cong & : & \isaratt \\
  split & : & \isaratt \\
\end{matharray}

\begin{rail}
  ('simp' | 'cong' | 'split') (() | 'add' | 'del')
  ;
\end{rail}

\begin{descr}

\item [$\isarcmd{print_simpset}$] prints the collection of rules declared to
  the Simplifier, which is also known as ``simpset'' internally
  \cite{isabelle-ref}.  This is a diagnostic command; $undo$ does not apply.

\item [$simp$] declares simplification rules.

\item [$cong$] declares congruence rules.

\item [$split$] declares case split rules.

\end{descr}


\subsubsection{Forward simplification}

\indexisaratt{simplified}
\begin{matharray}{rcl}
  simplified & : & \isaratt \\
\end{matharray}

\begin{rail}
  'simplified' opt? thmrefs?
  ;

  opt: '(' (noasm | noasmsimp | noasmuse) ')'
  ;
\end{rail}

\begin{descr}
  
\item [$simplified~\vec a$] causes a theorem to be simplified, either by
  exactly the specified rules $\vec a$, or the implicit Simplifier context if
  no arguments are given.  The result is fully simplified by default,
  including assumptions and conclusion; the options $no_asm$ etc.\ tune the
  Simplifier in the same way as the for the $simp$ method.

  Note that forward simplification restricts the simplifier to its most basic
  operation of term rewriting; solver and looper tactics \cite{isabelle-ref}
  are \emph{not} involved here.  The $simplified$ attribute should be only
  rarely required under normal circumstances.

\end{descr}


\subsubsection{Low-level equational reasoning}

\indexisarmeth{subst}\indexisarmeth{hypsubst}\indexisarmeth{split}
\begin{matharray}{rcl}
  subst^* & : & \isarmeth \\
  hypsubst^* & : & \isarmeth \\
  split^* & : & \isarmeth \\
\end{matharray}

\begin{rail}
  'subst' ('(' 'asm' ')')? ('(' (nat+) ')')? thmref
  ;
  'split' ('(' 'asm' ')')? thmrefs
  ;
\end{rail}

These methods provide low-level facilities for equational reasoning that are
intended for specialized applications only.  Normally, single step
calculations would be performed in a structured text (see also
\S\ref{sec:calculation}), while the Simplifier methods provide the canonical
way for automated normalization (see \S\ref{sec:simplifier}).

\begin{descr}

\item [$subst~eq$] performs a single substitution step using rule $eq$, which
  may be either a meta or object equality.

\item [$subst~(asm)~eq$] substitutes in an assumption.

\item [$subst~(i \dots j)~eq$] performs several substitutions in the
conclusion. The numbers $i$ to $j$ indicate the positions to substitute at.
Positions are ordered from the top of the term tree moving down from left to
right. For example, in $(a+b)+(c+d)$ there are three positions where
commutativity of $+$ is applicable: 1 refers to the whole term, 2 to $a+b$
and 3 to $c+d$. If the positions in the list $(i \dots j)$ are
non-overlapping (e.g. $(2~3)$ in $(a+b)+(c+d)$) you may assume all
substitutions are performed simultaneously. Otherwise the behaviour of
$subst$ is not specified.

\item [$subst~(asm)~(i \dots j)~eq$] performs the substitutions in the
assumptions. Positions $1 \dots i@1$ refer
to assumption 1, positions $i@1+1 \dots i@2$ to assumption 2, and so on.

\item [$hypsubst$] performs substitution using some assumption; this only
  works for equations of the form $x = t$ where $x$ is a free or bound
  variable.

\item [$split~\vec a$] performs single-step case splitting using rules $thms$.
  By default, splitting is performed in the conclusion of a goal; the $asm$
  option indicates to operate on assumptions instead.
  
  Note that the $simp$ method already involves repeated application of split
  rules as declared in the current context.
\end{descr}


\subsection{The Classical Reasoner}\label{sec:classical}

\subsubsection{Basic methods}

\indexisarmeth{rule}\indexisarmeth{default}\indexisarmeth{contradiction}
\indexisarmeth{intro}\indexisarmeth{elim}
\begin{matharray}{rcl}
  rule & : & \isarmeth \\
  contradiction & : & \isarmeth \\
  intro & : & \isarmeth \\
  elim & : & \isarmeth \\
\end{matharray}

\begin{rail}
  ('rule' | 'intro' | 'elim') thmrefs?
  ;
\end{rail}

\begin{descr}

\item [$rule$] as offered by the classical reasoner is a refinement over the
  primitive one (see \S\ref{sec:pure-meth-att}).  Both versions essentially
  work the same, but the classical version observes the classical rule context
  in addition to that of Isabelle/Pure.

  Common object logics (HOL, ZF, etc.) declare a rich collection of classical
  rules (even if these would qualify as intuitionistic ones), but only few
  declarations to the rule context of Isabelle/Pure
  (\S\ref{sec:pure-meth-att}).

\item [$contradiction$] solves some goal by contradiction, deriving any result
  from both $\lnot A$ and $A$.  Chained facts, which are guaranteed to
  participate, may appear in either order.

\item [$intro$ and $elim$] repeatedly refine some goal by intro- or
  elim-resolution, after having inserted any chained facts.  Exactly the rules
  given as arguments are taken into account; this allows fine-tuned
  decomposition of a proof problem, in contrast to common automated tools.

\end{descr}


\subsubsection{Automated methods}

\indexisarmeth{blast}\indexisarmeth{fast}\indexisarmeth{slow}
\indexisarmeth{best}\indexisarmeth{safe}\indexisarmeth{clarify}
\begin{matharray}{rcl}
  blast & : & \isarmeth \\
  fast & : & \isarmeth \\
  slow & : & \isarmeth \\
  best & : & \isarmeth \\
  safe & : & \isarmeth \\
  clarify & : & \isarmeth \\
\end{matharray}

\indexouternonterm{clamod}
\begin{rail}
  'blast' ('!' ?) nat? (clamod *)
  ;
  ('fast' | 'slow' | 'best' | 'safe' | 'clarify') ('!' ?) (clamod *)
  ;

  clamod: (('intro' | 'elim' | 'dest') ('!' | () | '?') | 'del') ':' thmrefs
  ;
\end{rail}

\begin{descr}
\item [$blast$] refers to the classical tableau prover (see \texttt{blast_tac}
  in \cite[\S11]{isabelle-ref}).  The optional argument specifies a
  user-supplied search bound (default 20).
\item [$fast$, $slow$, $best$, $safe$, and $clarify$] refer to the generic
  classical reasoner.  See \texttt{fast_tac}, \texttt{slow_tac},
  \texttt{best_tac}, \texttt{safe_tac}, and \texttt{clarify_tac} in
  \cite[\S11]{isabelle-ref} for more information.
\end{descr}

Any of the above methods support additional modifiers of the context of
classical rules.  Their semantics is analogous to the attributes given before.
Facts provided by forward chaining are inserted into the goal before
commencing proof search.  The ``!''~argument causes the full context of
assumptions to be included as well.


\subsubsection{Combined automated methods}\label{sec:clasimp}

\indexisarmeth{auto}\indexisarmeth{force}\indexisarmeth{clarsimp}
\indexisarmeth{fastsimp}\indexisarmeth{slowsimp}\indexisarmeth{bestsimp}
\begin{matharray}{rcl}
  auto & : & \isarmeth \\
  force & : & \isarmeth \\
  clarsimp & : & \isarmeth \\
  fastsimp & : & \isarmeth \\
  slowsimp & : & \isarmeth \\
  bestsimp & : & \isarmeth \\
\end{matharray}

\indexouternonterm{clasimpmod}
\begin{rail}
  'auto' '!'? (nat nat)? (clasimpmod *)
  ;
  ('force' | 'clarsimp' | 'fastsimp' | 'slowsimp' | 'bestsimp') '!'? (clasimpmod *)
  ;

  clasimpmod: ('simp' (() | 'add' | 'del' | 'only') |
    ('cong' | 'split') (() | 'add' | 'del') |
    'iff' (((() | 'add') '?'?) | 'del') |
    (('intro' | 'elim' | 'dest') ('!' | () | '?') | 'del')) ':' thmrefs
\end{rail}

\begin{descr}
\item [$auto$, $force$, $clarsimp$, $fastsimp$, $slowsimp$, and $bestsimp$]
  provide access to Isabelle's combined simplification and classical reasoning
  tactics.  These correspond to \texttt{auto_tac}, \texttt{force_tac},
  \texttt{clarsimp_tac}, and Classical Reasoner tactics with the Simplifier
  added as wrapper, see \cite[\S11]{isabelle-ref} for more information.  The
  modifier arguments correspond to those given in \S\ref{sec:simplifier} and
  \S\ref{sec:classical}.  Just note that the ones related to the Simplifier
  are prefixed by \railtterm{simp} here.

  Facts provided by forward chaining are inserted into the goal before doing
  the search.  The ``!''~argument causes the full context of assumptions to be
  included as well.
\end{descr}


\subsubsection{Declaring rules}

\indexisarcmd{print-claset}
\indexisaratt{intro}\indexisaratt{elim}\indexisaratt{dest}
\indexisaratt{iff}\indexisaratt{rule}
\begin{matharray}{rcl}
  \isarcmd{print_claset}^* & : & \isarkeep{theory~|~proof} \\
  intro & : & \isaratt \\
  elim & : & \isaratt \\
  dest & : & \isaratt \\
  rule & : & \isaratt \\
  iff & : & \isaratt \\
\end{matharray}

\begin{rail}
  ('intro' | 'elim' | 'dest') ('!' | () | '?') nat?
  ;
  'rule' 'del'
  ;
  'iff' (((() | 'add') '?'?) | 'del')
  ;
\end{rail}

\begin{descr}

\item [$\isarcmd{print_claset}$] prints the collection of rules declared to
  the Classical Reasoner, which is also known as ``simpset'' internally
  \cite{isabelle-ref}.  This is a diagnostic command; $undo$ does not apply.
  
\item [$intro$, $elim$, and $dest$] declare introduction, elimination, and
  destruction rules, respectively.  By default, rules are considered as
  \emph{unsafe} (i.e.\ not applied blindly without backtracking), while a
  single ``!'' classifies as \emph{safe}.  Rule declarations marked by ``?''
  coincide with those of Isabelle/Pure, cf.\ \S\ref{sec:pure-meth-att} (i.e.\ 
  are only applied in single steps of the $rule$ method).  The optional
  natural number specifies an explicit weight argument, which is ignored by
  automated tools, but determines the search order of single rule steps.

\item [$rule~del$] deletes introduction, elimination, or destruction rules from
  the context.

\item [$iff$] declares logical equivalences to the Simplifier and the
  Classical reasoner at the same time.  Non-conditional rules result in a
  ``safe'' introduction and elimination pair; conditional ones are considered
  ``unsafe''.  Rules with negative conclusion are automatically inverted
  (using $\lnot$ elimination internally).

  The ``?'' version of $iff$ declares rules to the Isabelle/Pure context only,
  and omits the Simplifier declaration.

\end{descr}


\subsubsection{Classical operations}

\indexisaratt{swapped}

\begin{matharray}{rcl}
  swapped & : & \isaratt \\
\end{matharray}

\begin{descr}

\item [$swapped$] turns an introduction rule into an elimination, by resolving
  with the classical swap principle $(\lnot B \Imp A) \Imp (\lnot A \Imp B)$.

\end{descr}


\subsection{Proof by cases and induction}\label{sec:cases-induct}

\subsubsection{Rule contexts}

\indexisarcmd{case}\indexisarcmd{print-cases}
\indexisaratt{case-names}\indexisaratt{case-conclusion}
\indexisaratt{params}\indexisaratt{consumes}
\begin{matharray}{rcl}
  \isarcmd{case} & : & \isartrans{proof(state)}{proof(state)} \\
  \isarcmd{print_cases}^* & : & \isarkeep{proof} \\
  case_names & : & \isaratt \\
  case_conclusion & : & \isaratt \\
  params & : & \isaratt \\
  consumes & : & \isaratt \\
\end{matharray}

The puristic way to build up Isar proof contexts is by explicit language
elements like $\FIXNAME$, $\ASSUMENAME$, $\LET$ (see
\S\ref{sec:proof-context}).  This is adequate for plain natural deduction, but
easily becomes unwieldy in concrete verification tasks, which typically
involve big induction rules with several cases.

The $\CASENAME$ command provides a shorthand to refer to a local context
symbolically: certain proof methods provide an environment of named ``cases''
of the form $c\colon \vec x, \vec \phi$; the effect of ``$\CASE{c}$'' is then
equivalent to ``$\FIX{\vec x}~\ASSUME{c}{\vec\phi}$''.  Term bindings may be
covered as well, notably $\Var{case}$ for the main conclusion.

By default, the ``terminology'' $\vec x$ of a case value is marked as hidden,
i.e.\ there is no way to refer to such parameters in the subsequent proof
text.  After all, original rule parameters stem from somewhere outside of the
current proof text.  By using the explicit form ``$\CASE{(c~\vec y)}$''
instead, the proof author is able to chose local names that fit nicely into
the current context.

\medskip

It is important to note that proper use of $\CASENAME$ does not provide means
to peek at the current goal state, which is not directly observable in Isar!
Nonetheless, goal refinement commands do provide named cases $goal@i$ for each
subgoal $i = 1, \dots, n$ of the resulting goal state.  Using this feature
requires great care, because some bits of the internal tactical machinery
intrude the proof text.  In particular, parameter names stemming from the
left-over of automated reasoning tools are usually quite unpredictable.

Under normal circumstances, the text of cases emerge from standard elimination
or induction rules, which in turn are derived from previous theory
specifications in a canonical way (say from $\isarkeyword{inductive}$
definitions).

\medskip Proper cases are only available if both the proof method and the
rules involved support this.  By using appropriate attributes, case names,
conclusions, and parameters may be also declared by hand.  Thus variant
versions of rules that have been derived manually become ready to use in
advanced case analysis later.

\begin{rail}
  'case' (caseref | '(' caseref ((name | underscore) +) ')')
  ;
  caseref: nameref attributes?
  ;

  'case\_names' (name +)
  ;
  'case\_conclusion' name (name *)
  ;
  'params' ((name *) + 'and')
  ;
  'consumes' nat?
  ;
\end{rail}

\begin{descr}
  
\item [$\CASE{(c~\vec x)}$] invokes a named local context $c\colon \vec x,
  \vec \phi$, as provided by an appropriate proof method (such as $cases$ and
  $induct$).  The command ``$\CASE{(c~\vec x)}$'' abbreviates ``$\FIX{\vec
    x}~\ASSUME{c}{\vec\phi}$''.

\item [$\isarkeyword{print_cases}$] prints all local contexts of the current
  state, using Isar proof language notation.  This is a diagnostic command;
  $undo$ does not apply.
  
\item [$case_names~\vec c$] declares names for the local contexts of premises
  of a theorem; $\vec c$ refers to the \emph{suffix} of the list of premises.
  
\item [$case_conclusion~c~\vec d$] declares names for the conclusions of a
  named premise $c$; here $\vec d$ refers to the prefix of arguments of a
  logical formula built by nesting a binary connective (e.g.\ $\lor$).
  
  Note that proof methods such as $induct$ and $coinduct$ already provide a
  default name for the conclusion as a whole.  The need to name subformulas
  only arises with cases that split into several sub-cases, as in common
  co-induction rules.

\item [$params~\vec p@1 \dots \vec p@n$] renames the innermost parameters of
  premises $1, \dots, n$ of some theorem.  An empty list of names may be given
  to skip positions, leaving the present parameters unchanged.
  
  Note that the default usage of case rules does \emph{not} directly expose
  parameters to the proof context.
  
\item [$consumes~n$] declares the number of ``major premises'' of a rule,
  i.e.\ the number of facts to be consumed when it is applied by an
  appropriate proof method.  The default value of $consumes$ is $n = 1$, which
  is appropriate for the usual kind of cases and induction rules for inductive
  sets (cf.\ \S\ref{sec:hol-inductive}).  Rules without any $consumes$
  declaration given are treated as if $consumes~0$ had been specified.
  
  Note that explicit $consumes$ declarations are only rarely needed; this is
  already taken care of automatically by the higher-level $cases$, $induct$,
  and $coinduct$ declarations.

\end{descr}


\subsubsection{Proof methods}

\indexisarmeth{cases}\indexisarmeth{induct}\indexisarmeth{coinduct}
\begin{matharray}{rcl}
  cases & : & \isarmeth \\
  induct & : & \isarmeth \\
  coinduct & : & \isarmeth \\
\end{matharray}

The $cases$, $induct$, and $coinduct$ methods provide a uniform interface to
common proof techniques over datatypes, inductive sets, recursive functions
etc.  The corresponding rules may be specified and instantiated in a casual
manner.  Furthermore, these methods provide named local contexts that may be
invoked via the $\CASENAME$ proof command within the subsequent proof text.
This accommodates compact proof texts even when reasoning about large
specifications.

The $induct$ method also provides some additional infrastructure in order to
be applicable to structure statements (either using explicit meta-level
connectives, or including facts and parameters separately).  This avoids
cumbersome encoding of ``strengthened'' inductive statements within the
object-logic.

\begin{rail}
  'cases' open? (insts * 'and') rule?
  ;
  'induct' open? (definsts * 'and') \\ fixing? taking? rule?
  ;
  'coinduct' open? insts taking rule?
  ;

  open: '(' 'open' ')'
  ;
  rule: ('type' | 'set') ':' (nameref +) | 'rule' ':' (thmref +)
  ;
  definst: name ('==' | equiv) term | inst
  ;
  definsts: ( definst *)
  ;
  fixing: 'fixing' ':' ((term *) 'and' +)
  ;
  taking: 'taking' ':' insts
  ;
\end{rail}

\begin{descr}

\item [$cases~insts~R$] applies method $rule$ with an appropriate case
  distinction theorem, instantiated to the subjects $insts$.  Symbolic case
  names are bound according to the rule's local contexts.

  The rule is determined as follows, according to the facts and arguments
  passed to the $cases$ method:
  \begin{matharray}{llll}
    \Text{facts}    &       & \Text{arguments} & \Text{rule} \\\hline
                    & cases &           & \Text{classical case split} \\
                    & cases & t         & \Text{datatype exhaustion (type of $t$)} \\
    \edrv a \in A   & cases & \dots     & \Text{inductive set elimination (of $A$)} \\
    \dots           & cases & \dots ~ R & \Text{explicit rule $R$} \\
  \end{matharray}

  Several instantiations may be given, referring to the \emph{suffix} of
  premises of the case rule; within each premise, the \emph{prefix} of
  variables is instantiated.  In most situations, only a single term needs to
  be specified; this refers to the first variable of the last premise (it is
  usually the same for all cases).

  The ``$(open)$'' option causes the parameters of the new local contexts to
  be exposed to the current proof context.  Thus local variables stemming from
  distant parts of the theory development may be introduced in an implicit
  manner, which can be quite confusing to the reader.  Furthermore, this
  option may cause unwanted hiding of existing local variables, resulting in
  less robust proof texts.

\item [$induct~insts~R$] is analogous to the $cases$ method, but refers to
  induction rules, which are determined as follows:
  \begin{matharray}{llll}
    \Text{facts}    &        & \Text{arguments} & \Text{rule} \\\hline
                    & induct & P ~ x ~ \dots & \Text{datatype induction (type of $x$)} \\
    \edrv x \in A   & induct & \dots         & \Text{set induction (of $A$)} \\
    \dots           & induct & \dots ~ R     & \Text{explicit rule $R$} \\
  \end{matharray}
  
  Several instantiations may be given, each referring to some part of
  a mutual inductive definition or datatype --- only related partial
  induction rules may be used together, though.  Any of the lists of
  terms $P, x, \dots$ refers to the \emph{suffix} of variables present
  in the induction rule.  This enables the writer to specify only
  induction variables, or both predicates and variables, for example.
  
  Instantiations may be definitional: equations $x \equiv t$ introduce local
  definitions, which are inserted into the claim and discharged after applying
  the induction rule.  Equalities reappear in the inductive cases, but have
  been transformed according to the induction principle being involved here.
  In order to achieve practically useful induction hypotheses, some variables
  occurring in $t$ need to be fixed (see below).
  
  The optional ``$fixing\colon \vec x$'' specification generalizes variables
  $\vec x$ of the original goal before applying induction.  Thus induction
  hypotheses may become sufficiently general to get the proof through.
  Together with definitional instantiations, one may effectively perform
  induction over expressions of a certain structure.
  
  The optional ``$taking\colon \vec t$'' specification provides additional
  instantiations of a prefix of pending variables in the rule.  Such schematic
  induction rules rarely occur in practice, though.

  The ``$(open)$'' option works the same way as for $cases$.

\item [$coinduct~inst~R$] is analogous to the $induct$ method, but refers to
  coinduction rules, which are determined as follows:
  \begin{matharray}{llll}
    \Text{goal}     &          & \Text{arguments} & \Text{rule} \\\hline
                    & coinduct & x ~ \dots        & \Text{type coinduction (type of $x$)} \\
    x \in A         & coinduct & \dots            & \Text{set coinduction (of $A$)} \\
    \dots           & coinduct & \dots ~ R        & \Text{explicit rule $R$} \\
  \end{matharray}
  
  Coinduction is the dual of induction.  Induction essentially eliminates $x
  \in A$ towards a generic result $P ~ x$, while coinduction introduces $x \in
  A$ starting with $x \in B$, for a suitable ``bisimulation'' $B$.  The cases
  of a coinduct rule are typically named after the sets being covered, while
  the conclusions consist of several alternatives being named after the
  individual destructor patterns.
  
  The given instantiation refers to the \emph{prefix} of variables occurring
  in the rule's conclusion.  An additional ``$taking: \vec t$'' specification
  may be required in order to specify the bisimulation to be used in the
  coinduction step.

  The ``$(open)$'' option works the same way as for $cases$.

\end{descr}

Above methods produce named local contexts, as determined by the instantiated
rule as given in the text.  Beyond that, the $induct$ and $coinduct$ methods
guess further instantiations from the goal specification itself.  Any
persisting unresolved schematic variables of the resulting rule will render
the the corresponding case invalid.  The term binding
$\Var{case}$\indexisarvar{case} for the conclusion will be provided with each
case, provided that term is fully specified.

The $\isarkeyword{print_cases}$ command prints all named cases present in the
current proof state.

\medskip

Despite the additional infrastructure, both $cases$ and $coinduct$ merely
apply a certain rule, after instantiation, while conforming due to the usual
way of monotonic natural deduction: the context of a structured statement
$\All{\vec x} \vec\phi \Imp \dots$ reappears unchanged after the case split.

The $induct$ method is significantly different in this respect: the meta-level
structure is passed through the ``recursive'' course involved in the
induction.  Thus the original statement is basically replaced by separate
copies, corresponding to the induction hypotheses and conclusion; the original
goal context is no longer available.  Thus local assumptions, fixed parameters
and definitions effectively participate in the inductive rephrasing of the
original statement.

In induction proofs, local assumptions introduced by cases are split into two
different kinds: $hyps$ stemming from the rule and $prems$ from the goal
statement.  This is reflected in the extracted cases accordingly, so invoking
``$\isarcmd{case}~c$'' will provide separate facts $c\mathord.hyps$ and
$c\mathord.prems$, as well as fact $c$ to hold the all-inclusive list.

\medskip

Facts presented to either method are consumed according to the number of
``major premises'' of the rule involved, which is usually $0$ for plain cases
and induction rules of datatypes etc.\ and $1$ for rules of inductive sets and
the like.  The remaining facts are inserted into the goal verbatim before the
actual $cases$, $induct$, or $coinduct$ rule is applied.


\subsubsection{Declaring rules}

\indexisarcmd{print-induct-rules}\indexisaratt{cases}\indexisaratt{induct}\indexisaratt{coinduct}
\begin{matharray}{rcl}
  \isarcmd{print_induct_rules}^* & : & \isarkeep{theory~|~proof} \\
  cases & : & \isaratt \\
  induct & : & \isaratt \\
  coinduct & : & \isaratt \\
\end{matharray}

\begin{rail}
  'cases' spec
  ;
  'induct' spec
  ;
  'coinduct' spec
  ;

  spec: ('type' | 'set') ':' nameref
  ;
\end{rail}

\begin{descr}

\item [$\isarkeyword{print_induct_rules}$] prints cases and induct rules for
  sets and types of the current context.
  
\item [$cases$, $induct$, and $coinduct$] (as attributes) augment the
  corresponding context of rules for reasoning about (co)inductive sets and
  types, using the corresponding methods of the same name.  Certain
  definitional packages of object-logics usually declare emerging cases and
  induction rules as expected, so users rarely need to intervene.
  
  Manual rule declarations usually refer to the $case_names$ and $params$
  attributes to adjust names of cases and parameters of a rule; the $consumes$
  declaration is taken care of automatically: $consumes~0$ is specified for
  ``type'' rules and $consumes~1$ for ``set'' rules.

\end{descr}

%%% Local Variables:
%%% mode: latex
%%% TeX-master: "isar-ref"
%%% End:
