
\chapter{Isar Syntax Primitives}

We give a complete reference of all basic syntactic entities underlying the
Isabelle/Isar document syntax.  Actual theory and proof commands will be
introduced later on.

\medskip

In order to get started with writing well-formed Isabelle/Isar documents, the
most important aspect to be noted is the difference of \emph{inner} versus
\emph{outer} syntax.  Inner syntax is that of Isabelle types and terms of the
logic, while outer syntax is that of Isabelle/Isar theories (including
proofs).  As a general rule, inner syntax entities may occur only as
\emph{atomic entities} within outer syntax.  For example, the string
\texttt{"x + y"} and identifier \texttt{z} are legal term specifications
within a theory, while \texttt{x + y} is not.

\begin{warn}
  Note that classic Isabelle theories used to fake parts of the inner syntax
  of types, with rather complicated rules when quotes may be omitted.  Despite
  the minor drawback of requiring quotes more often, the syntax of
  Isabelle/Isar is much simpler and more robust in that respect.
\end{warn}

\medskip

Isabelle/Isar input may contain any number of input termination characters
``\texttt{;}'' (semicolon) to separate commands explicitly.  This may be
particularly useful in interactive shell sessions to make clear where the
current command is intended to end.  Otherwise, the interactive loop will
continue until end-of-command is clearly indicated from the input syntax,
e.g.\ encounter of the next command keyword.

Advanced interfaces such as Proof~General \cite{proofgeneral} do not require
explicit semicolons, the amount of input text is determined automatically by
inspecting the Emacs text buffer.  Also note that in the presentation of
Isabelle/Isar documents, semicolons are omitted in any case to gain
readability.


\section{Lexical matters}\label{sec:lex-syntax}

The Isabelle/Isar outer syntax provides token classes as presented below.
Note that some of these coincide (by full intention) with the inner lexical
syntax as presented in \cite{isabelle-ref}.  These different levels of syntax
should not be confused, though.

%FIXME keyword, command
\indexoutertoken{ident}\indexoutertoken{longident}\indexoutertoken{symident}
\indexoutertoken{nat}\indexoutertoken{var}\indexoutertoken{typefree}
\indexoutertoken{typevar}\indexoutertoken{string}\indexoutertoken{verbatim}
\begin{matharray}{rcl}
  ident & = & letter~quasiletter^* \\
  longident & = & ident\verb,.,ident~\dots~ident \\
  symident & = & sym^+ ~|~ symbol \\
  nat & = & digit^+ \\
  var & = & \verb,?,ident ~|~ \verb,?,ident\verb,.,nat \\
  typefree & = & \verb,',ident \\
  typevar & = & \verb,?,typefree ~|~ \verb,?,typefree\verb,.,nat \\
  string & = & \verb,", ~\dots~ \verb,", \\
  verbatim & = & \verb,{*, ~\dots~ \verb,*}, \\
\end{matharray}
\begin{matharray}{rcl}
  letter & = & \verb,a, ~|~ \dots ~|~ \verb,z, ~|~ \verb,A, ~|~ \dots ~|~ \verb,Z, \\
  digit & = & \verb,0, ~|~ \dots ~|~ \verb,9, \\
  quasiletter & = & letter ~|~ digit ~|~ \verb,_, ~|~ \verb,', \\
  sym & = & \verb,!, ~|~ \verb,#, ~|~ \verb,$, ~|~ \verb,%, ~|~ \verb,&, ~|~  %$
   \verb,*, ~|~ \verb,+, ~|~ \verb,-, ~|~ \verb,/, ~|~ \verb,:, ~|~ \\
  & & \verb,<, ~|~ \verb,=, ~|~ \verb,>, ~|~ \verb,?, ~|~ \texttt{\at} ~|~
  \verb,^, ~|~ \verb,_, ~|~ \verb,`, ~|~ \verb,|, ~|~ \verb,~, \\
  symbol & = & {\forall} ~|~ {\exists} ~|~ \dots
\end{matharray}

The syntax of \texttt{string} admits any characters, including newlines;
``\verb|"|'' (double-quote) and ``\verb|\|'' (backslash) have to be escaped by
a backslash; newlines need not be escaped.  Note that ML-style control
characters are \emph{not} supported.  The body of \texttt{verbatim} may
consist of any text not containing ``\verb|*}|''.

Comments take the form \texttt{(*~\dots~*)} and may be
nested\footnote{Proof~General may occasionally get confused by nested
  comments.}, just as in ML. Note that these are \emph{source} comments only,
which are stripped after lexical analysis of the input.  The Isar document
syntax also provides \emph{formal comments} that are actually part of the text
(see \S\ref{sec:comments}).

Mathematical symbols such as ``$\forall$'' are represented in plain ASCII as
``\verb,\<forall>,''.


\section{Common syntax entities}

Subsequently, we introduce several basic syntactic entities, such as names,
terms, and theorem specifications, which have been factored out of the actual
Isar language elements to be described later.

Note that some of the basic syntactic entities introduced below (e.g.\ 
\railqtoken{name}) act much like tokens rather than plain nonterminals (e.g.\ 
\railnonterm{sort}), especially for the sake of error messages.  E.g.\ syntax
elements such as $\CONSTS$ referring to \railqtoken{name} or \railqtoken{type}
would really report a missing name or type rather than any of the constituent
primitive tokens such as \railtoken{ident} or \railtoken{string}.


\subsection{Names}

Entity \railqtoken{name} usually refers to any name of types, constants,
theorems etc.\ that are to be \emph{declared} or \emph{defined} (so qualified
identifiers are excluded here).  Quoted strings provide an escape for
non-identifier names or those ruled out by outer syntax keywords (e.g.\ 
\verb|"let"|).  Already existing objects are usually referenced by
\railqtoken{nameref}.

\indexoutertoken{name}\indexoutertoken{parname}\indexoutertoken{nameref}
\indexoutertoken{int}
\begin{rail}
  name: ident | symident | string | nat
  ;
  parname: '(' name ')'
  ;
  nameref: name | longident
  ;
  int: nat | '-' nat
  ;
\end{rail}


\subsection{Comments}\label{sec:comments}

Large chunks of plain \railqtoken{text} are usually given
\railtoken{verbatim}, i.e.\ enclosed in \verb|{*|~\dots~\verb|*}|.  For
convenience, any of the smaller text units conforming to \railqtoken{nameref}
are admitted as well.  Almost any of the Isar commands may be annotated by
marginal \railnonterm{comment} of the form \texttt{--} \railqtoken{text}.
Note that the latter kind of comment is actually part of the language, while
source level comments \verb|(*|~\dots~\verb|*)| are stripped at the lexical
level.  A few commands such as $\PROOFNAME$ admit additional markup with a
``level of interest'': \texttt{\%} followed by an optional number $n$ (default
$n = 1$) indicates that the respective part of the document becomes $n$ levels
more obscure; \texttt{\%\%} means that interest drops by $\infty$ --- abandon
every hope, who enter here.

\indexoutertoken{text}\indexouternonterm{comment}\indexouternonterm{interest}
\begin{rail}
  text: verbatim | nameref
  ;
  comment: ('--' text +)
  ;
  interest: percent nat? | ppercent
  ;
\end{rail}


\subsection{Type classes, sorts and arities}

Classes are specified by plain names.  Sorts have a very simple inner syntax,
which is either a single class name $c$ or a list $\{c@1, \dots, c@n\}$
referring to the intersection of these classes.  The syntax of type arities is
given directly at the outer level.

\railalias{subseteq}{\isasymsubseteq}
\railterm{subseteq}

\indexouternonterm{sort}\indexouternonterm{arity}\indexouternonterm{simplearity}
\indexouternonterm{classdecl}
\begin{rail}
  classdecl: name (('<' | subseteq) (nameref + ','))?
  ;
  sort: nameref
  ;
  arity: ('(' (sort + ',') ')')? sort
  ;
  simplearity: ('(' (sort + ',') ')')? nameref
  ;
\end{rail}


\subsection{Types and terms}\label{sec:types-terms}

The actual inner Isabelle syntax, that of types and terms of the logic, is far
too sophisticated in order to be modelled explicitly at the outer theory
level.  Basically, any such entity has to be quoted to turn it into a single
token (the parsing and type-checking is performed internally later).  For
convenience, a slightly more liberal convention is adopted: quotes may be
omitted for any type or term that is already \emph{atomic} at the outer level.
For example, one may write just \texttt{x} instead of \texttt{"x"}.  Note that
symbolic identifiers (e.g.\ \texttt{++} or $\forall$) are available as well,
provided these are not superseded by commands or keywords (e.g.\ \texttt{+}).

\indexoutertoken{type}\indexoutertoken{term}\indexoutertoken{prop}
\begin{rail}
  type: nameref | typefree | typevar
  ;
  term: nameref | var
  ;
  prop: term
  ;
\end{rail}

Positional instantiations are indicated by giving a sequence of terms, or the
placeholder ``$\_$'' (underscore), which means to skip a position.

\indexoutertoken{inst}\indexoutertoken{insts}
\begin{rail}
  inst: underscore | term
  ;
  insts: (inst *)
  ;
\end{rail}

Type declarations and definitions usually refer to \railnonterm{typespec} on
the left-hand side.  This models basic type constructor application at the
outer syntax level.  Note that only plain postfix notation is available here,
but no infixes.

\indexouternonterm{typespec}
\begin{rail}
  typespec: (() | typefree | '(' ( typefree + ',' ) ')') name
  ;
\end{rail}


\subsection{Term patterns}\label{sec:term-pats}

Assumptions and goal statements usually admit casual binding of schematic term
variables by giving (optional) patterns of the form $\ISS{p@1\;\dots}{p@n}$.
There are separate versions available for \railqtoken{term}s and
\railqtoken{prop}s.  The latter provides a $\CONCLNAME$ part with patterns
referring the (atomic) conclusion of a rule.

\indexouternonterm{termpat}\indexouternonterm{proppat}
\begin{rail}
  termpat: '(' ('is' term +) ')'
  ;
  proppat: '(' (('is' prop +) | 'concl' ('is' prop +) | ('is' prop +) 'concl' ('is' prop +)) ')'
  ;
\end{rail}


\subsection{Mixfix annotations}

Mixfix annotations specify concrete \emph{inner} syntax of Isabelle types and
terms (see also \cite{isabelle-ref}).  Some commands such as $\TYPES$ (see
\S\ref{sec:types-pure}) admit infixes only, while $\CONSTS$ (see
\S\ref{sec:consts}) and $\isarkeyword{syntax}$ (see \S\ref{sec:syn-trans})
support the full range of general mixfixes and binders.

\indexouternonterm{infix}\indexouternonterm{mixfix}
\begin{rail}
  infix: '(' ('infix' | 'infixl' | 'infixr') string? nat ')'
  ;
  mixfix: infix | '(' string prios? nat? ')' | '(' 'binder' string prios? nat ')'
  ;

  prios: '[' (nat + ',') ']'
  ;
\end{rail}


\subsection{Attributes and theorems}\label{sec:syn-att}

Attributes (and proof methods, see \S\ref{sec:syn-meth}) have their own
``semi-inner'' syntax, in the sense that input conforming to
\railnonterm{args} below is parsed by the attribute a second time.  The
attribute argument specifications may be any sequence of atomic entities
(identifiers, strings etc.), or properly bracketed argument lists.  Below
\railqtoken{atom} refers to any atomic entity, including any
\railtoken{keyword} conforming to \railtoken{symident}.

\indexoutertoken{atom}\indexouternonterm{args}\indexouternonterm{attributes}
\begin{rail}
  atom: nameref | typefree | typevar | var | nat | keyword
  ;
  arg: atom | '(' args ')' | '[' args ']'
  ;
  args: arg *
  ;
  attributes: '[' (nameref args * ',') ']'
  ;
\end{rail}

Theorem specifications come in several flavors: \railnonterm{axmdecl} and
\railnonterm{thmdecl} usually refer to axioms, assumptions or results of goal
statements, while \railnonterm{thmdef} collects lists of existing theorems.
Existing theorems are given by \railnonterm{thmref} and \railnonterm{thmrefs},
the former requires an actual singleton result.  Any of these theorem
specifications may include lists of attributes both on the left and right hand
sides; attributes are applied to any immediately preceding theorem.  If names
are omitted, the theorems are not stored within the theorem database of the
theory or proof context; any given attributes are still applied, though.

\indexouternonterm{thmdecl}\indexouternonterm{axmdecl}
\indexouternonterm{thmdef}\indexouternonterm{thmrefs}
\begin{rail}
  axmdecl: name attributes? ':'
  ;
  thmdecl: thmbind ':'
  ;
  thmdef: thmbind '='
  ;
  thmref: nameref attributes?
  ;
  thmrefs: thmref +
  ;

  thmbind: name attributes | name | attributes
  ;
\end{rail}


\subsection{Proof methods}\label{sec:syn-meth}

Proof methods are either basic ones, or expressions composed of methods via
``\texttt{,}'' (sequential composition), ``\texttt{|}'' (alternative choices),
``\texttt{?}'' (try), ``\texttt{+}'' (repeat at least once).  In practice,
proof methods are usually just a comma separated list of
\railqtoken{nameref}~\railnonterm{args} specifications.  Note that parentheses
may be dropped for single method specifications (with no arguments).

\indexouternonterm{method}
\begin{rail}
  method: (nameref | '(' methods ')') (() | '?' | '+')
  ;
  methods: (nameref args | method) + (',' | '|')
  ;
\end{rail}

Proper use of Isar proof methods does \emph{not} involve goal addressing.
Nevertheless, specifying goal ranges may occasionally come in handy in
emulating tactic scripts.  Note that $[n-]$ refers to all goals, starting from
$n$.  All goals may be specified by $[!]$, which is the same as $[1-]$.

\indexouternonterm{goalspec}
\begin{rail}
  goalspec: '[' (nat '-' nat | nat '-' | nat | '!' ) ']'
  ;
\end{rail}


\subsection{Antiquotations}\label{sec:antiq}

\begin{matharray}{rcl}
  thm & : & \isarantiq \\
  prop & : & \isarantiq \\
  term & : & \isarantiq \\
  typ & : & \isarantiq \\
  text & : & \isarantiq \\
  goals & : & \isarantiq \\
  subgoals & : & \isarantiq \\
\end{matharray}

The text body of formal comments (see also \S\ref{sec:comments}) may contain
antiquotations of logical entities, such as theorems, terms and types, which
are to be presented in the final output produced by the Isabelle document
preparation system (see also \S\ref{sec:document-prep}).

Thus embedding of
\texttt{{\at}{\ttlbrace}term~[show_types]~"f(x)~=~a~+~x"{\ttrbrace}} within a
text block would cause
\isa{(f{\isasymColon}'a~{\isasymRightarrow}~'a)~(x{\isasymColon}'a)~=~(a{\isasymColon}'a)~+~x}
to appear in the final {\LaTeX} document.  Also note that theorem
antiquotations may involve attributes as well.  For example,
\texttt{{\at}{\ttlbrace}thm~sym~[no_vars]{\ttrbrace}} would print the
statement where all schematic variables have been replaced by fixed ones,
which are better readable.

\indexisarant{thm}\indexisarant{prop}\indexisarant{term}
\indexisarant{typ}\indexisarant{text}\indexisarant{goals}\indexisarant{subgoals}
\begin{rail}
  atsign lbrace antiquotation rbrace
  ;

  antiquotation:
    'thm' options thmrefs |
    'prop' options prop |
    'term' options term |
    'typ' options type |
    'text' options name |
    'goals' options |
    'subgoals' options
  ;
  options: '[' (option * ',') ']'
  ;
  option: name | name '=' name
  ;
\end{rail}

Note that the syntax of antiquotations may \emph{not} include source comments
\texttt{(*~\dots~*)} or verbatim text \verb|{*|~\dots~\verb|*}|.

\begin{descr}
\item [$\at\{thm~\vec a\}$] prints theorems $\vec a$. Note that attribute
  specifications may be included as well (see also \S\ref{sec:syn-att}); the
  $no_vars$ operation (see \S\ref{sec:misc-methods}) would be particularly
  useful to suppress printing of schematic variables.
\item [$\at\{prop~\phi\}$] prints a well-typed proposition $\phi$.
\item [$\at\{term~t\}$] prints a well-typed term $t$.
\item [$\at\{typ~\tau\}$] prints a well-formed type $\tau$.
\item [$\at\{text~s\}$] prints uninterpreted source text $s$.  This is
  particularly useful to print portions of text according to the Isabelle
  {\LaTeX} output style, without demanding well-formedness (e.g.\ small pieces
  of terms that cannot be parsed or type-checked yet).
\item [$\at\{goals\}$] prints the current \emph{dynamic} goal state.  This is
  only for support of tactic-emulation scripts within Isar --- presentation of
  goal states does not conform to actual human-readable proof documents.
  
  Please do not include goal states into document output unless you really
  know what you are doing!
\item [$\at\{subgoals\}$] behaves almost like $goals$, except that it does not
  print the main goal.
\end{descr}

\medskip

The following options are available to tune the output.  Note that most of
these coincide with ML flags of the same names (see also \cite{isabelle-ref}).
\begin{descr}
\item[$show_types = bool$ and $show_sorts = bool$] control printing of
  explicit type and sort constraints.
\item[$long_names = bool$] forces names of types and constants etc.\ to be
  printed in their fully qualified internal form.
\item[$eta_contract = bool$] prints terms in $\eta$-contracted form.
\item[$display = bool$] indicates if the text is to be output as multi-line
  ``display material'', rather than a small piece of text without line breaks
  (which is the default).
\item[$quotes = bool$] indicates if the output should be enclosed in double
  quotes.
\item[$mode = name$] adds $name$ to the print mode to be used for presentation
  (see also \cite{isabelle-ref}).  Note that the standard setup for {\LaTeX}
  output is already present by default, including the modes ``$latex$'',
  ``$xsymbols$'', ``$symbols$''.
\item[$margin = nat$ and $indent = nat$] change the margin or indentation for
  pretty printing of display material.
\item[$source = bool$] prints the source text of the antiquotation arguments,
  rather than the actual value.  Note that this does not affect
  well-formedness checks of $thm$, $term$, etc. (only the $text$ antiquotation
  admits arbitrary output).
\item[$goals_limit = nat$] determines the maximum number of goals to be
  printed.
\end{descr}

For boolean flags, ``$name = true$'' may be abbreviated as ``$name$''.  All of
the above flags are disabled by default, unless changed from ML.

\medskip Note that antiquotations do not only spare the author from tedious
typing, but also achieve some degree of consistency-checking of informal
explanations with formal developments, since well-formedness of terms and
types with respect to the current theory or proof context can be ensured.

%%% Local Variables: 
%%% mode: latex
%%% TeX-master: "isar-ref"
%%% End: 
