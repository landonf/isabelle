%
\begin{isabellebody}%
\def\isabellecontext{Induction}%
%
\isadelimtheory
%
\endisadelimtheory
%
\isatagtheory
%
\endisatagtheory
{\isafoldtheory}%
%
\isadelimtheory
%
\endisadelimtheory
%
\isamarkupsection{Case distinction and induction \label{sec:Induct}%
}
\isamarkuptrue%
%
\begin{isamarkuptext}%
Computer science applications abound with inductively defined
structures, which is why we treat them in more detail. HOL already
comes with a datatype of lists with the two constructors \isa{Nil}
and \isa{Cons}. \isa{Nil} is written \isa{{\isaliteral{5B}{\isacharbrackleft}}{\isaliteral{5D}{\isacharbrackright}}} and \isa{Cons\ x\ xs} is written \isa{x\ {\isaliteral{23}{\isacharhash}}\ xs}.%
\end{isamarkuptext}%
\isamarkuptrue%
%
\isamarkupsubsection{Case distinction\label{sec:CaseDistinction}%
}
\isamarkuptrue%
%
\begin{isamarkuptext}%
We have already met the \isa{cases} method for performing
binary case splits. Here is another example:%
\end{isamarkuptext}%
\isamarkuptrue%
\isacommand{lemma}\isamarkupfalse%
\ {\isaliteral{22}{\isachardoublequoteopen}}{\isaliteral{5C3C6E6F743E}{\isasymnot}}\ A\ {\isaliteral{5C3C6F723E}{\isasymor}}\ A{\isaliteral{22}{\isachardoublequoteclose}}\isanewline
%
\isadelimproof
%
\endisadelimproof
%
\isatagproof
\isacommand{proof}\isamarkupfalse%
\ cases\isanewline
\ \ \isacommand{assume}\isamarkupfalse%
\ {\isaliteral{22}{\isachardoublequoteopen}}A{\isaliteral{22}{\isachardoublequoteclose}}\ \isacommand{thus}\isamarkupfalse%
\ {\isaliteral{3F}{\isacharquery}}thesis\ \isacommand{{\isaliteral{2E}{\isachardot}}{\isaliteral{2E}{\isachardot}}}\isamarkupfalse%
\isanewline
\isacommand{next}\isamarkupfalse%
\isanewline
\ \ \isacommand{assume}\isamarkupfalse%
\ {\isaliteral{22}{\isachardoublequoteopen}}{\isaliteral{5C3C6E6F743E}{\isasymnot}}\ A{\isaliteral{22}{\isachardoublequoteclose}}\ \isacommand{thus}\isamarkupfalse%
\ {\isaliteral{3F}{\isacharquery}}thesis\ \isacommand{{\isaliteral{2E}{\isachardot}}{\isaliteral{2E}{\isachardot}}}\isamarkupfalse%
\isanewline
\isacommand{qed}\isamarkupfalse%
%
\endisatagproof
{\isafoldproof}%
%
\isadelimproof
%
\endisadelimproof
%
\begin{isamarkuptext}%
\noindent The two cases must come in this order because \isa{cases} merely abbreviates \isa{{\isaliteral{28}{\isacharparenleft}}rule\ case{\isaliteral{5F}{\isacharunderscore}}split{\isaliteral{29}{\isacharparenright}}} where
\isa{case{\isaliteral{5F}{\isacharunderscore}}split} is \isa{{\isaliteral{5C3C6C6272616B6B3E}{\isasymlbrakk}}{\isaliteral{3F}{\isacharquery}}P\ {\isaliteral{5C3C4C6F6E6772696768746172726F773E}{\isasymLongrightarrow}}\ {\isaliteral{3F}{\isacharquery}}Q{\isaliteral{3B}{\isacharsemicolon}}\ {\isaliteral{5C3C6E6F743E}{\isasymnot}}\ {\isaliteral{3F}{\isacharquery}}P\ {\isaliteral{5C3C4C6F6E6772696768746172726F773E}{\isasymLongrightarrow}}\ {\isaliteral{3F}{\isacharquery}}Q{\isaliteral{5C3C726272616B6B3E}{\isasymrbrakk}}\ {\isaliteral{5C3C4C6F6E6772696768746172726F773E}{\isasymLongrightarrow}}\ {\isaliteral{3F}{\isacharquery}}Q}. If we reverse
the order of the two cases in the proof, the first case would prove
\isa{{\isaliteral{5C3C6E6F743E}{\isasymnot}}\ A\ {\isaliteral{5C3C4C6F6E6772696768746172726F773E}{\isasymLongrightarrow}}\ {\isaliteral{5C3C6E6F743E}{\isasymnot}}\ A\ {\isaliteral{5C3C6F723E}{\isasymor}}\ A} which would solve the first premise of
\isa{case{\isaliteral{5F}{\isacharunderscore}}split}, instantiating \isa{{\isaliteral{3F}{\isacharquery}}P} with \isa{{\isaliteral{5C3C6E6F743E}{\isasymnot}}\ A}, thus making the second premise \isa{{\isaliteral{5C3C6E6F743E}{\isasymnot}}\ {\isaliteral{5C3C6E6F743E}{\isasymnot}}\ A\ {\isaliteral{5C3C4C6F6E6772696768746172726F773E}{\isasymLongrightarrow}}\ {\isaliteral{5C3C6E6F743E}{\isasymnot}}\ A\ {\isaliteral{5C3C6F723E}{\isasymor}}\ A}.
Therefore the order of subgoals is not always completely arbitrary.

The above proof is appropriate if \isa{A} is textually small.
However, if \isa{A} is large, we do not want to repeat it. This can
be avoided by the following idiom%
\end{isamarkuptext}%
\isamarkuptrue%
\isacommand{lemma}\isamarkupfalse%
\ {\isaliteral{22}{\isachardoublequoteopen}}{\isaliteral{5C3C6E6F743E}{\isasymnot}}\ A\ {\isaliteral{5C3C6F723E}{\isasymor}}\ A{\isaliteral{22}{\isachardoublequoteclose}}\isanewline
%
\isadelimproof
%
\endisadelimproof
%
\isatagproof
\isacommand{proof}\isamarkupfalse%
\ {\isaliteral{28}{\isacharparenleft}}cases\ {\isaliteral{22}{\isachardoublequoteopen}}A{\isaliteral{22}{\isachardoublequoteclose}}{\isaliteral{29}{\isacharparenright}}\isanewline
\ \ \isacommand{case}\isamarkupfalse%
\ True\ \isacommand{thus}\isamarkupfalse%
\ {\isaliteral{3F}{\isacharquery}}thesis\ \isacommand{{\isaliteral{2E}{\isachardot}}{\isaliteral{2E}{\isachardot}}}\isamarkupfalse%
\isanewline
\isacommand{next}\isamarkupfalse%
\isanewline
\ \ \isacommand{case}\isamarkupfalse%
\ False\ \isacommand{thus}\isamarkupfalse%
\ {\isaliteral{3F}{\isacharquery}}thesis\ \isacommand{{\isaliteral{2E}{\isachardot}}{\isaliteral{2E}{\isachardot}}}\isamarkupfalse%
\isanewline
\isacommand{qed}\isamarkupfalse%
%
\endisatagproof
{\isafoldproof}%
%
\isadelimproof
%
\endisadelimproof
%
\begin{isamarkuptext}%
\noindent which is like the previous proof but instantiates
\isa{{\isaliteral{3F}{\isacharquery}}P} right away with \isa{A}. Thus we could prove the two
cases in any order. The phrase \isakeyword{case}~\isa{True}
abbreviates \isakeyword{assume}~\isa{True{\isaliteral{3A}{\isacharcolon}}\ A} and analogously for
\isa{False} and \isa{{\isaliteral{5C3C6E6F743E}{\isasymnot}}\ A}.

The same game can be played with other datatypes, for example lists,
where \isa{tl} is the tail of a list, and \isa{length} returns a
natural number (remember: $0-1=0$):%
\end{isamarkuptext}%
\isamarkuptrue%
\isacommand{lemma}\isamarkupfalse%
\ {\isaliteral{22}{\isachardoublequoteopen}}length{\isaliteral{28}{\isacharparenleft}}tl\ xs{\isaliteral{29}{\isacharparenright}}\ {\isaliteral{3D}{\isacharequal}}\ length\ xs\ {\isaliteral{2D}{\isacharminus}}\ {\isadigit{1}}{\isaliteral{22}{\isachardoublequoteclose}}\isanewline
%
\isadelimproof
%
\endisadelimproof
%
\isatagproof
\isacommand{proof}\isamarkupfalse%
\ {\isaliteral{28}{\isacharparenleft}}cases\ xs{\isaliteral{29}{\isacharparenright}}\isanewline
\ \ \isacommand{case}\isamarkupfalse%
\ Nil\ \isacommand{thus}\isamarkupfalse%
\ {\isaliteral{3F}{\isacharquery}}thesis\ \isacommand{by}\isamarkupfalse%
\ simp\isanewline
\isacommand{next}\isamarkupfalse%
\isanewline
\ \ \isacommand{case}\isamarkupfalse%
\ Cons\ \isacommand{thus}\isamarkupfalse%
\ {\isaliteral{3F}{\isacharquery}}thesis\ \isacommand{by}\isamarkupfalse%
\ simp\isanewline
\isacommand{qed}\isamarkupfalse%
%
\endisatagproof
{\isafoldproof}%
%
\isadelimproof
%
\endisadelimproof
%
\begin{isamarkuptext}%
\noindent Here \isakeyword{case}~\isa{Nil} abbreviates
\isakeyword{assume}~\isa{Nil{\isaliteral{3A}{\isacharcolon}}}~\isa{xs\ {\isaliteral{3D}{\isacharequal}}\ {\isaliteral{5B}{\isacharbrackleft}}{\isaliteral{5D}{\isacharbrackright}}} and
\isakeyword{case}~\isa{Cons} abbreviates \isakeyword{fix}~\isa{{\isaliteral{3F}{\isacharquery}}\ {\isaliteral{3F}{\isacharquery}}{\isaliteral{3F}{\isacharquery}}}
\isakeyword{assume}~\isa{Cons{\isaliteral{3A}{\isacharcolon}}}~\isa{xs\ {\isaliteral{3D}{\isacharequal}}\ {\isaliteral{3F}{\isacharquery}}\ {\isaliteral{23}{\isacharhash}}\ {\isaliteral{3F}{\isacharquery}}{\isaliteral{3F}{\isacharquery}}},
where \isa{{\isaliteral{3F}{\isacharquery}}} and \isa{{\isaliteral{3F}{\isacharquery}}{\isaliteral{3F}{\isacharquery}}}
stand for variable names that have been chosen by the system.
Therefore we cannot refer to them.
Luckily, this proof is simple enough we do not need to refer to them.
However, sometimes one may have to. Hence Isar offers a simple scheme for
naming those variables: replace the anonymous \isa{Cons} by
\isa{{\isaliteral{28}{\isacharparenleft}}Cons\ y\ ys{\isaliteral{29}{\isacharparenright}}}, which abbreviates \isakeyword{fix}~\isa{y\ ys}
\isakeyword{assume}~\isa{Cons{\isaliteral{3A}{\isacharcolon}}}~\isa{xs\ {\isaliteral{3D}{\isacharequal}}\ y\ {\isaliteral{23}{\isacharhash}}\ ys}.
In each \isakeyword{case} the assumption can be
referred to inside the proof by the name of the constructor. In
Section~\ref{sec:full-Ind} below we will come across an example
of this.

\subsection{Structural induction}

We start with an inductive proof where both cases are proved automatically:%
\end{isamarkuptext}%
\isamarkuptrue%
\isacommand{lemma}\isamarkupfalse%
\ {\isaliteral{22}{\isachardoublequoteopen}}{\isadigit{2}}\ {\isaliteral{2A}{\isacharasterisk}}\ {\isaliteral{28}{\isacharparenleft}}{\isaliteral{5C3C53756D3E}{\isasymSum}}i{\isaliteral{3A}{\isacharcolon}}{\isaliteral{3A}{\isacharcolon}}nat{\isaliteral{5C3C6C653E}{\isasymle}}n{\isaliteral{2E}{\isachardot}}\ i{\isaliteral{29}{\isacharparenright}}\ {\isaliteral{3D}{\isacharequal}}\ n{\isaliteral{2A}{\isacharasterisk}}{\isaliteral{28}{\isacharparenleft}}n{\isaliteral{2B}{\isacharplus}}{\isadigit{1}}{\isaliteral{29}{\isacharparenright}}{\isaliteral{22}{\isachardoublequoteclose}}\isanewline
%
\isadelimproof
%
\endisadelimproof
%
\isatagproof
\isacommand{by}\isamarkupfalse%
\ {\isaliteral{28}{\isacharparenleft}}induct\ n{\isaliteral{29}{\isacharparenright}}\ simp{\isaliteral{5F}{\isacharunderscore}}all%
\endisatagproof
{\isafoldproof}%
%
\isadelimproof
%
\endisadelimproof
%
\begin{isamarkuptext}%
\noindent The constraint \isa{{\isaliteral{3A}{\isacharcolon}}{\isaliteral{3A}{\isacharcolon}}nat} is needed because all of
the operations involved are overloaded.
This proof also demonstrates that \isakeyword{by} can take two arguments,
one to start and one to finish the proof --- the latter is optional.

If we want to expose more of the structure of the
proof, we can use pattern matching to avoid having to repeat the goal
statement:%
\end{isamarkuptext}%
\isamarkuptrue%
\isacommand{lemma}\isamarkupfalse%
\ {\isaliteral{22}{\isachardoublequoteopen}}{\isadigit{2}}\ {\isaliteral{2A}{\isacharasterisk}}\ {\isaliteral{28}{\isacharparenleft}}{\isaliteral{5C3C53756D3E}{\isasymSum}}i{\isaliteral{3A}{\isacharcolon}}{\isaliteral{3A}{\isacharcolon}}nat{\isaliteral{5C3C6C653E}{\isasymle}}n{\isaliteral{2E}{\isachardot}}\ i{\isaliteral{29}{\isacharparenright}}\ {\isaliteral{3D}{\isacharequal}}\ n{\isaliteral{2A}{\isacharasterisk}}{\isaliteral{28}{\isacharparenleft}}n{\isaliteral{2B}{\isacharplus}}{\isadigit{1}}{\isaliteral{29}{\isacharparenright}}{\isaliteral{22}{\isachardoublequoteclose}}\ {\isaliteral{28}{\isacharparenleft}}\isakeyword{is}\ {\isaliteral{22}{\isachardoublequoteopen}}{\isaliteral{3F}{\isacharquery}}P\ n{\isaliteral{22}{\isachardoublequoteclose}}{\isaliteral{29}{\isacharparenright}}\isanewline
%
\isadelimproof
%
\endisadelimproof
%
\isatagproof
\isacommand{proof}\isamarkupfalse%
\ {\isaliteral{28}{\isacharparenleft}}induct\ n{\isaliteral{29}{\isacharparenright}}\isanewline
\ \ \isacommand{show}\isamarkupfalse%
\ {\isaliteral{22}{\isachardoublequoteopen}}{\isaliteral{3F}{\isacharquery}}P\ {\isadigit{0}}{\isaliteral{22}{\isachardoublequoteclose}}\ \isacommand{by}\isamarkupfalse%
\ simp\isanewline
\isacommand{next}\isamarkupfalse%
\isanewline
\ \ \isacommand{fix}\isamarkupfalse%
\ n\ \isacommand{assume}\isamarkupfalse%
\ {\isaliteral{22}{\isachardoublequoteopen}}{\isaliteral{3F}{\isacharquery}}P\ n{\isaliteral{22}{\isachardoublequoteclose}}\isanewline
\ \ \isacommand{thus}\isamarkupfalse%
\ {\isaliteral{22}{\isachardoublequoteopen}}{\isaliteral{3F}{\isacharquery}}P{\isaliteral{28}{\isacharparenleft}}Suc\ n{\isaliteral{29}{\isacharparenright}}{\isaliteral{22}{\isachardoublequoteclose}}\ \isacommand{by}\isamarkupfalse%
\ simp\isanewline
\isacommand{qed}\isamarkupfalse%
%
\endisatagproof
{\isafoldproof}%
%
\isadelimproof
%
\endisadelimproof
%
\begin{isamarkuptext}%
\noindent We could refine this further to show more of the equational
proof. Instead we explore the same avenue as for case distinctions:
introducing context via the \isakeyword{case} command:%
\end{isamarkuptext}%
\isamarkuptrue%
\isacommand{lemma}\isamarkupfalse%
\ {\isaliteral{22}{\isachardoublequoteopen}}{\isadigit{2}}\ {\isaliteral{2A}{\isacharasterisk}}\ {\isaliteral{28}{\isacharparenleft}}{\isaliteral{5C3C53756D3E}{\isasymSum}}i{\isaliteral{3A}{\isacharcolon}}{\isaliteral{3A}{\isacharcolon}}nat\ {\isaliteral{5C3C6C653E}{\isasymle}}\ n{\isaliteral{2E}{\isachardot}}\ i{\isaliteral{29}{\isacharparenright}}\ {\isaliteral{3D}{\isacharequal}}\ n{\isaliteral{2A}{\isacharasterisk}}{\isaliteral{28}{\isacharparenleft}}n{\isaliteral{2B}{\isacharplus}}{\isadigit{1}}{\isaliteral{29}{\isacharparenright}}{\isaliteral{22}{\isachardoublequoteclose}}\isanewline
%
\isadelimproof
%
\endisadelimproof
%
\isatagproof
\isacommand{proof}\isamarkupfalse%
\ {\isaliteral{28}{\isacharparenleft}}induct\ n{\isaliteral{29}{\isacharparenright}}\isanewline
\ \ \isacommand{case}\isamarkupfalse%
\ {\isadigit{0}}\ \isacommand{show}\isamarkupfalse%
\ {\isaliteral{3F}{\isacharquery}}case\ \isacommand{by}\isamarkupfalse%
\ simp\isanewline
\isacommand{next}\isamarkupfalse%
\isanewline
\ \ \isacommand{case}\isamarkupfalse%
\ Suc\ \isacommand{thus}\isamarkupfalse%
\ {\isaliteral{3F}{\isacharquery}}case\ \isacommand{by}\isamarkupfalse%
\ simp\isanewline
\isacommand{qed}\isamarkupfalse%
%
\endisatagproof
{\isafoldproof}%
%
\isadelimproof
%
\endisadelimproof
%
\begin{isamarkuptext}%
\noindent The implicitly defined \isa{{\isaliteral{3F}{\isacharquery}}case} refers to the
corresponding case to be proved, i.e.\ \isa{{\isaliteral{3F}{\isacharquery}}P\ {\isadigit{0}}} in the first case and
\isa{{\isaliteral{3F}{\isacharquery}}P{\isaliteral{28}{\isacharparenleft}}Suc\ n{\isaliteral{29}{\isacharparenright}}} in the second case. Context \isakeyword{case}~\isa{{\isadigit{0}}} is
empty whereas \isakeyword{case}~\isa{Suc} assumes \isa{{\isaliteral{3F}{\isacharquery}}P\ n}. Again we
have the same problem as with case distinctions: we cannot refer to an anonymous \isa{n}
in the induction step because it has not been introduced via \isakeyword{fix}
(in contrast to the previous proof). The solution is the one outlined for
\isa{Cons} above: replace \isa{Suc} by \isa{{\isaliteral{28}{\isacharparenleft}}Suc\ i{\isaliteral{29}{\isacharparenright}}}:%
\end{isamarkuptext}%
\isamarkuptrue%
\isacommand{lemma}\isamarkupfalse%
\ \isakeyword{fixes}\ n{\isaliteral{3A}{\isacharcolon}}{\isaliteral{3A}{\isacharcolon}}nat\ \isakeyword{shows}\ {\isaliteral{22}{\isachardoublequoteopen}}n\ {\isaliteral{3C}{\isacharless}}\ n{\isaliteral{2A}{\isacharasterisk}}n\ {\isaliteral{2B}{\isacharplus}}\ {\isadigit{1}}{\isaliteral{22}{\isachardoublequoteclose}}\isanewline
%
\isadelimproof
%
\endisadelimproof
%
\isatagproof
\isacommand{proof}\isamarkupfalse%
\ {\isaliteral{28}{\isacharparenleft}}induct\ n{\isaliteral{29}{\isacharparenright}}\isanewline
\ \ \isacommand{case}\isamarkupfalse%
\ {\isadigit{0}}\ \isacommand{show}\isamarkupfalse%
\ {\isaliteral{3F}{\isacharquery}}case\ \isacommand{by}\isamarkupfalse%
\ simp\isanewline
\isacommand{next}\isamarkupfalse%
\isanewline
\ \ \isacommand{case}\isamarkupfalse%
\ {\isaliteral{28}{\isacharparenleft}}Suc\ i{\isaliteral{29}{\isacharparenright}}\ \isacommand{thus}\isamarkupfalse%
\ {\isaliteral{22}{\isachardoublequoteopen}}Suc\ i\ {\isaliteral{3C}{\isacharless}}\ Suc\ i\ {\isaliteral{2A}{\isacharasterisk}}\ Suc\ i\ {\isaliteral{2B}{\isacharplus}}\ {\isadigit{1}}{\isaliteral{22}{\isachardoublequoteclose}}\ \isacommand{by}\isamarkupfalse%
\ simp\isanewline
\isacommand{qed}\isamarkupfalse%
%
\endisatagproof
{\isafoldproof}%
%
\isadelimproof
%
\endisadelimproof
%
\begin{isamarkuptext}%
\noindent Of course we could again have written
\isakeyword{thus}~\isa{{\isaliteral{3F}{\isacharquery}}case} instead of giving the term explicitly
but we wanted to use \isa{i} somewhere.

\subsection{Generalization via \isa{arbitrary}}

It is frequently necessary to generalize a claim before it becomes
provable by induction. The tutorial~\cite{LNCS2283} demonstrates this
with \isa{itrev\ xs\ ys\ {\isaliteral{3D}{\isacharequal}}\ rev\ xs\ {\isaliteral{40}{\isacharat}}\ ys}, where \isa{ys}
needs to be universally quantified before induction succeeds.\footnote{\isa{rev\ {\isaliteral{5B}{\isacharbrackleft}}{\isaliteral{5D}{\isacharbrackright}}\ {\isaliteral{3D}{\isacharequal}}\ {\isaliteral{5B}{\isacharbrackleft}}{\isaliteral{5D}{\isacharbrackright}}},\quad \isa{rev\ {\isaliteral{28}{\isacharparenleft}}x\ {\isaliteral{23}{\isacharhash}}\ xs{\isaliteral{29}{\isacharparenright}}\ {\isaliteral{3D}{\isacharequal}}\ rev\ xs\ {\isaliteral{40}{\isacharat}}\ {\isaliteral{5B}{\isacharbrackleft}}x{\isaliteral{5D}{\isacharbrackright}}},\\ \isa{itrev\ {\isaliteral{5B}{\isacharbrackleft}}{\isaliteral{5D}{\isacharbrackright}}\ ys\ {\isaliteral{3D}{\isacharequal}}\ ys},\quad \isa{itrev\ {\isaliteral{28}{\isacharparenleft}}x\ {\isaliteral{23}{\isacharhash}}\ xs{\isaliteral{29}{\isacharparenright}}\ ys\ {\isaliteral{3D}{\isacharequal}}\ itrev\ xs\ {\isaliteral{28}{\isacharparenleft}}x\ {\isaliteral{23}{\isacharhash}}\ ys{\isaliteral{29}{\isacharparenright}}}} But
strictly speaking, this quantification step is already part of the
proof and the quantifiers should not clutter the original claim. This
is how the quantification step can be combined with induction:%
\end{isamarkuptext}%
\isamarkuptrue%
\isacommand{lemma}\isamarkupfalse%
\ {\isaliteral{22}{\isachardoublequoteopen}}itrev\ xs\ ys\ {\isaliteral{3D}{\isacharequal}}\ rev\ xs\ {\isaliteral{40}{\isacharat}}\ ys{\isaliteral{22}{\isachardoublequoteclose}}\isanewline
%
\isadelimproof
%
\endisadelimproof
%
\isatagproof
\isacommand{by}\isamarkupfalse%
\ {\isaliteral{28}{\isacharparenleft}}induct\ xs\ arbitrary{\isaliteral{3A}{\isacharcolon}}\ ys{\isaliteral{29}{\isacharparenright}}\ simp{\isaliteral{5F}{\isacharunderscore}}all%
\endisatagproof
{\isafoldproof}%
%
\isadelimproof
%
\endisadelimproof
%
\begin{isamarkuptext}%
\noindent The annotation \isa{arbitrary{\isaliteral{3A}{\isacharcolon}}}~\emph{vars}
universally quantifies all \emph{vars} before the induction.  Hence
they can be replaced by \emph{arbitrary} values in the proof.

Generalization via \isa{arbitrary} is particularly convenient
if the induction step is a structured proof as opposed to the automatic
example above. Then the claim is available in unquantified form but
with the generalized variables replaced by \isa{{\isaliteral{3F}{\isacharquery}}}-variables, ready
for instantiation. In the above example, in the \isa{Cons} case the
induction hypothesis is \isa{itrev\ xs\ {\isaliteral{3F}{\isacharquery}}ys\ {\isaliteral{3D}{\isacharequal}}\ rev\ xs\ {\isaliteral{40}{\isacharat}}\ {\isaliteral{3F}{\isacharquery}}ys} (available
under the name \isa{Cons}).


\subsection{Inductive proofs of conditional formulae}
\label{sec:full-Ind}

Induction also copes well with formulae involving \isa{{\isaliteral{5C3C4C6F6E6772696768746172726F773E}{\isasymLongrightarrow}}}, for example%
\end{isamarkuptext}%
\isamarkuptrue%
\isacommand{lemma}\isamarkupfalse%
\ {\isaliteral{22}{\isachardoublequoteopen}}xs\ {\isaliteral{5C3C6E6F7465713E}{\isasymnoteq}}\ {\isaliteral{5B}{\isacharbrackleft}}{\isaliteral{5D}{\isacharbrackright}}\ {\isaliteral{5C3C4C6F6E6772696768746172726F773E}{\isasymLongrightarrow}}\ hd{\isaliteral{28}{\isacharparenleft}}rev\ xs{\isaliteral{29}{\isacharparenright}}\ {\isaliteral{3D}{\isacharequal}}\ last\ xs{\isaliteral{22}{\isachardoublequoteclose}}\isanewline
%
\isadelimproof
%
\endisadelimproof
%
\isatagproof
\isacommand{by}\isamarkupfalse%
\ {\isaliteral{28}{\isacharparenleft}}induct\ xs{\isaliteral{29}{\isacharparenright}}\ simp{\isaliteral{5F}{\isacharunderscore}}all%
\endisatagproof
{\isafoldproof}%
%
\isadelimproof
%
\endisadelimproof
%
\begin{isamarkuptext}%
\noindent This is an improvement over that style the
tutorial~\cite{LNCS2283} advises, which requires \isa{{\isaliteral{5C3C6C6F6E6772696768746172726F773E}{\isasymlongrightarrow}}}.
A further improvement is shown in the following proof:%
\end{isamarkuptext}%
\isamarkuptrue%
\isacommand{lemma}\isamarkupfalse%
\ \ {\isaliteral{22}{\isachardoublequoteopen}}map\ f\ xs\ {\isaliteral{3D}{\isacharequal}}\ map\ f\ ys\ {\isaliteral{5C3C4C6F6E6772696768746172726F773E}{\isasymLongrightarrow}}\ length\ xs\ {\isaliteral{3D}{\isacharequal}}\ length\ ys{\isaliteral{22}{\isachardoublequoteclose}}\isanewline
%
\isadelimproof
%
\endisadelimproof
%
\isatagproof
\isacommand{proof}\isamarkupfalse%
\ {\isaliteral{28}{\isacharparenleft}}induct\ ys\ arbitrary{\isaliteral{3A}{\isacharcolon}}\ xs{\isaliteral{29}{\isacharparenright}}\isanewline
\ \ \isacommand{case}\isamarkupfalse%
\ Nil\ \isacommand{thus}\isamarkupfalse%
\ {\isaliteral{3F}{\isacharquery}}case\ \isacommand{by}\isamarkupfalse%
\ simp\isanewline
\isacommand{next}\isamarkupfalse%
\isanewline
\ \ \isacommand{case}\isamarkupfalse%
\ {\isaliteral{28}{\isacharparenleft}}Cons\ y\ ys{\isaliteral{29}{\isacharparenright}}\ \ \isacommand{note}\isamarkupfalse%
\ Asm\ {\isaliteral{3D}{\isacharequal}}\ Cons\isanewline
\ \ \isacommand{show}\isamarkupfalse%
\ {\isaliteral{3F}{\isacharquery}}case\isanewline
\ \ \isacommand{proof}\isamarkupfalse%
\ {\isaliteral{28}{\isacharparenleft}}cases\ xs{\isaliteral{29}{\isacharparenright}}\isanewline
\ \ \ \ \isacommand{case}\isamarkupfalse%
\ Nil\isanewline
\ \ \ \ \isacommand{hence}\isamarkupfalse%
\ False\ \isacommand{using}\isamarkupfalse%
\ Asm{\isaliteral{28}{\isacharparenleft}}{\isadigit{2}}{\isaliteral{29}{\isacharparenright}}\ \isacommand{by}\isamarkupfalse%
\ simp\isanewline
\ \ \ \ \isacommand{thus}\isamarkupfalse%
\ {\isaliteral{3F}{\isacharquery}}thesis\ \isacommand{{\isaliteral{2E}{\isachardot}}{\isaliteral{2E}{\isachardot}}}\isamarkupfalse%
\isanewline
\ \ \isacommand{next}\isamarkupfalse%
\isanewline
\ \ \ \ \isacommand{case}\isamarkupfalse%
\ {\isaliteral{28}{\isacharparenleft}}Cons\ x\ xs{\isaliteral{27}{\isacharprime}}{\isaliteral{29}{\isacharparenright}}\isanewline
\ \ \ \ \isacommand{with}\isamarkupfalse%
\ Asm{\isaliteral{28}{\isacharparenleft}}{\isadigit{2}}{\isaliteral{29}{\isacharparenright}}\ \isacommand{have}\isamarkupfalse%
\ {\isaliteral{22}{\isachardoublequoteopen}}map\ f\ xs{\isaliteral{27}{\isacharprime}}\ {\isaliteral{3D}{\isacharequal}}\ map\ f\ ys{\isaliteral{22}{\isachardoublequoteclose}}\ \isacommand{by}\isamarkupfalse%
\ simp\isanewline
\ \ \ \ \isacommand{from}\isamarkupfalse%
\ Asm{\isaliteral{28}{\isacharparenleft}}{\isadigit{1}}{\isaliteral{29}{\isacharparenright}}{\isaliteral{5B}{\isacharbrackleft}}OF\ this{\isaliteral{5D}{\isacharbrackright}}\ {\isaliteral{60}{\isacharbackquoteopen}}xs\ {\isaliteral{3D}{\isacharequal}}\ x{\isaliteral{23}{\isacharhash}}xs{\isaliteral{27}{\isacharprime}}{\isaliteral{60}{\isacharbackquoteclose}}\ \isacommand{show}\isamarkupfalse%
\ {\isaliteral{3F}{\isacharquery}}thesis\ \isacommand{by}\isamarkupfalse%
\ simp\isanewline
\ \ \isacommand{qed}\isamarkupfalse%
\isanewline
\isacommand{qed}\isamarkupfalse%
%
\endisatagproof
{\isafoldproof}%
%
\isadelimproof
%
\endisadelimproof
%
\begin{isamarkuptext}%
\noindent
The base case is trivial. In the step case Isar assumes
(under the name \isa{Cons}) two propositions:
\begin{center}
\begin{tabular}{l}
\isa{map\ f\ {\isaliteral{3F}{\isacharquery}}xs\ {\isaliteral{3D}{\isacharequal}}\ map\ f\ ys\ {\isaliteral{5C3C4C6F6E6772696768746172726F773E}{\isasymLongrightarrow}}\ length\ {\isaliteral{3F}{\isacharquery}}xs\ {\isaliteral{3D}{\isacharequal}}\ length\ ys}\\
\isa{map\ f\ xs\ {\isaliteral{3D}{\isacharequal}}\ map\ f\ {\isaliteral{28}{\isacharparenleft}}y\ {\isaliteral{23}{\isacharhash}}\ ys{\isaliteral{29}{\isacharparenright}}}
\end{tabular}
\end{center}
The first is the induction hypothesis, the second, and this is new,
is the premise of the induction step. The actual goal at this point is merely
\isa{length\ xs\ {\isaliteral{3D}{\isacharequal}}\ length\ {\isaliteral{28}{\isacharparenleft}}y\ {\isaliteral{23}{\isacharhash}}\ ys{\isaliteral{29}{\isacharparenright}}}. The assumptions are given the new name
\isa{Asm} to avoid a name clash further down. The proof procedes with a case distinction on \isa{xs}. In the case \isa{xs\ {\isaliteral{3D}{\isacharequal}}\ {\isaliteral{5B}{\isacharbrackleft}}{\isaliteral{5D}{\isacharbrackright}}}, the second of our two
assumptions (\isa{Asm{\isaliteral{28}{\isacharparenleft}}{\isadigit{2}}{\isaliteral{29}{\isacharparenright}}}) implies the contradiction \isa{{\isadigit{0}}\ {\isaliteral{3D}{\isacharequal}}\ Suc{\isaliteral{28}{\isacharparenleft}}{\isaliteral{5C3C646F74733E}{\isasymdots}}{\isaliteral{29}{\isacharparenright}}}.
 In the case \isa{xs\ {\isaliteral{3D}{\isacharequal}}\ x\ {\isaliteral{23}{\isacharhash}}\ xs{\isaliteral{27}{\isacharprime}}}, we first obtain
\isa{map\ f\ xs{\isaliteral{27}{\isacharprime}}\ {\isaliteral{3D}{\isacharequal}}\ map\ f\ ys}, from which a forward step with the first assumption (\isa{Asm{\isaliteral{28}{\isacharparenleft}}{\isadigit{1}}{\isaliteral{29}{\isacharparenright}}{\isaliteral{5B}{\isacharbrackleft}}OF\ this{\isaliteral{5D}{\isacharbrackright}}}) yields \isa{length\ xs{\isaliteral{27}{\isacharprime}}\ {\isaliteral{3D}{\isacharequal}}\ length\ ys}. Together
with \isa{xs\ {\isaliteral{3D}{\isacharequal}}\ x\ {\isaliteral{23}{\isacharhash}}\ xs} this yields the goal
\isa{length\ xs\ {\isaliteral{3D}{\isacharequal}}\ length\ {\isaliteral{28}{\isacharparenleft}}y\ {\isaliteral{23}{\isacharhash}}\ ys{\isaliteral{29}{\isacharparenright}}}.


\subsection{Induction formulae involving \isa{{\isaliteral{5C3C416E643E}{\isasymAnd}}} or \isa{{\isaliteral{5C3C4C6F6E6772696768746172726F773E}{\isasymLongrightarrow}}}}

Let us now consider abstractly the situation where the goal to be proved
contains both \isa{{\isaliteral{5C3C416E643E}{\isasymAnd}}} and \isa{{\isaliteral{5C3C4C6F6E6772696768746172726F773E}{\isasymLongrightarrow}}}, say \isa{{\isaliteral{5C3C416E643E}{\isasymAnd}}x{\isaliteral{2E}{\isachardot}}\ P\ x\ {\isaliteral{5C3C4C6F6E6772696768746172726F773E}{\isasymLongrightarrow}}\ Q\ x}.
This means that in each case of the induction,
\isa{{\isaliteral{3F}{\isacharquery}}case} would be of the form \isa{{\isaliteral{5C3C416E643E}{\isasymAnd}}x{\isaliteral{2E}{\isachardot}}\ P{\isaliteral{27}{\isacharprime}}\ x\ {\isaliteral{5C3C4C6F6E6772696768746172726F773E}{\isasymLongrightarrow}}\ Q{\isaliteral{27}{\isacharprime}}\ x}.  Thus the
first proof steps will be the canonical ones, fixing \isa{x} and assuming
\isa{P{\isaliteral{27}{\isacharprime}}\ x}. To avoid this tedium, induction performs the canonical steps
automatically: in each step case, the assumptions contain both the
usual induction hypothesis and \isa{P{\isaliteral{27}{\isacharprime}}\ x}, whereas \isa{{\isaliteral{3F}{\isacharquery}}case} is only
\isa{Q{\isaliteral{27}{\isacharprime}}\ x}.

\subsection{Rule induction}

HOL also supports inductively defined sets. See \cite{LNCS2283}
for details. As an example we define our own version of the reflexive
transitive closure of a relation --- HOL provides a predefined one as well.%
\end{isamarkuptext}%
\isamarkuptrue%
\isacommand{inductive{\isaliteral{5F}{\isacharunderscore}}set}\isamarkupfalse%
\isanewline
\ \ rtc\ {\isaliteral{3A}{\isacharcolon}}{\isaliteral{3A}{\isacharcolon}}\ {\isaliteral{22}{\isachardoublequoteopen}}{\isaliteral{28}{\isacharparenleft}}{\isaliteral{27}{\isacharprime}}a\ {\isaliteral{5C3C74696D65733E}{\isasymtimes}}\ {\isaliteral{27}{\isacharprime}}a{\isaliteral{29}{\isacharparenright}}set\ {\isaliteral{5C3C52696768746172726F773E}{\isasymRightarrow}}\ {\isaliteral{28}{\isacharparenleft}}{\isaliteral{27}{\isacharprime}}a\ {\isaliteral{5C3C74696D65733E}{\isasymtimes}}\ {\isaliteral{27}{\isacharprime}}a{\isaliteral{29}{\isacharparenright}}set{\isaliteral{22}{\isachardoublequoteclose}}\ \ \ {\isaliteral{28}{\isacharparenleft}}{\isaliteral{22}{\isachardoublequoteopen}}{\isaliteral{5F}{\isacharunderscore}}{\isaliteral{2A}{\isacharasterisk}}{\isaliteral{22}{\isachardoublequoteclose}}\ {\isaliteral{5B}{\isacharbrackleft}}{\isadigit{1}}{\isadigit{0}}{\isadigit{0}}{\isadigit{0}}{\isaliteral{5D}{\isacharbrackright}}\ {\isadigit{9}}{\isadigit{9}}{\isadigit{9}}{\isaliteral{29}{\isacharparenright}}\isanewline
\ \ \isakeyword{for}\ r\ {\isaliteral{3A}{\isacharcolon}}{\isaliteral{3A}{\isacharcolon}}\ {\isaliteral{22}{\isachardoublequoteopen}}{\isaliteral{28}{\isacharparenleft}}{\isaliteral{27}{\isacharprime}}a\ {\isaliteral{5C3C74696D65733E}{\isasymtimes}}\ {\isaliteral{27}{\isacharprime}}a{\isaliteral{29}{\isacharparenright}}set{\isaliteral{22}{\isachardoublequoteclose}}\isanewline
\isakeyword{where}\isanewline
\ \ refl{\isaliteral{3A}{\isacharcolon}}\ \ {\isaliteral{22}{\isachardoublequoteopen}}{\isaliteral{28}{\isacharparenleft}}x{\isaliteral{2C}{\isacharcomma}}x{\isaliteral{29}{\isacharparenright}}\ {\isaliteral{5C3C696E3E}{\isasymin}}\ r{\isaliteral{2A}{\isacharasterisk}}{\isaliteral{22}{\isachardoublequoteclose}}\isanewline
{\isaliteral{7C}{\isacharbar}}\ step{\isaliteral{3A}{\isacharcolon}}\ \ {\isaliteral{22}{\isachardoublequoteopen}}{\isaliteral{5C3C6C6272616B6B3E}{\isasymlbrakk}}\ {\isaliteral{28}{\isacharparenleft}}x{\isaliteral{2C}{\isacharcomma}}y{\isaliteral{29}{\isacharparenright}}\ {\isaliteral{5C3C696E3E}{\isasymin}}\ r{\isaliteral{3B}{\isacharsemicolon}}\ {\isaliteral{28}{\isacharparenleft}}y{\isaliteral{2C}{\isacharcomma}}z{\isaliteral{29}{\isacharparenright}}\ {\isaliteral{5C3C696E3E}{\isasymin}}\ r{\isaliteral{2A}{\isacharasterisk}}\ {\isaliteral{5C3C726272616B6B3E}{\isasymrbrakk}}\ {\isaliteral{5C3C4C6F6E6772696768746172726F773E}{\isasymLongrightarrow}}\ {\isaliteral{28}{\isacharparenleft}}x{\isaliteral{2C}{\isacharcomma}}z{\isaliteral{29}{\isacharparenright}}\ {\isaliteral{5C3C696E3E}{\isasymin}}\ r{\isaliteral{2A}{\isacharasterisk}}{\isaliteral{22}{\isachardoublequoteclose}}%
\begin{isamarkuptext}%
\noindent
First the constant is declared as a function on binary
relations (with concrete syntax \isa{r{\isaliteral{2A}{\isacharasterisk}}} instead of \isa{rtc\ r}), then the defining clauses are given. We will now prove that
\isa{r{\isaliteral{2A}{\isacharasterisk}}} is indeed transitive:%
\end{isamarkuptext}%
\isamarkuptrue%
\isacommand{lemma}\isamarkupfalse%
\ \isakeyword{assumes}\ A{\isaliteral{3A}{\isacharcolon}}\ {\isaliteral{22}{\isachardoublequoteopen}}{\isaliteral{28}{\isacharparenleft}}x{\isaliteral{2C}{\isacharcomma}}y{\isaliteral{29}{\isacharparenright}}\ {\isaliteral{5C3C696E3E}{\isasymin}}\ r{\isaliteral{2A}{\isacharasterisk}}{\isaliteral{22}{\isachardoublequoteclose}}\ \isakeyword{shows}\ {\isaliteral{22}{\isachardoublequoteopen}}{\isaliteral{28}{\isacharparenleft}}y{\isaliteral{2C}{\isacharcomma}}z{\isaliteral{29}{\isacharparenright}}\ {\isaliteral{5C3C696E3E}{\isasymin}}\ r{\isaliteral{2A}{\isacharasterisk}}\ {\isaliteral{5C3C4C6F6E6772696768746172726F773E}{\isasymLongrightarrow}}\ {\isaliteral{28}{\isacharparenleft}}x{\isaliteral{2C}{\isacharcomma}}z{\isaliteral{29}{\isacharparenright}}\ {\isaliteral{5C3C696E3E}{\isasymin}}\ r{\isaliteral{2A}{\isacharasterisk}}{\isaliteral{22}{\isachardoublequoteclose}}\isanewline
%
\isadelimproof
%
\endisadelimproof
%
\isatagproof
\isacommand{using}\isamarkupfalse%
\ A\isanewline
\isacommand{proof}\isamarkupfalse%
\ induct\isanewline
\ \ \isacommand{case}\isamarkupfalse%
\ refl\ \isacommand{thus}\isamarkupfalse%
\ {\isaliteral{3F}{\isacharquery}}case\ \isacommand{{\isaliteral{2E}{\isachardot}}}\isamarkupfalse%
\isanewline
\isacommand{next}\isamarkupfalse%
\isanewline
\ \ \isacommand{case}\isamarkupfalse%
\ step\ \isacommand{thus}\isamarkupfalse%
\ {\isaliteral{3F}{\isacharquery}}case\ \isacommand{by}\isamarkupfalse%
{\isaliteral{28}{\isacharparenleft}}blast\ intro{\isaliteral{3A}{\isacharcolon}}\ rtc{\isaliteral{2E}{\isachardot}}step{\isaliteral{29}{\isacharparenright}}\isanewline
\isacommand{qed}\isamarkupfalse%
%
\endisatagproof
{\isafoldproof}%
%
\isadelimproof
%
\endisadelimproof
%
\begin{isamarkuptext}%
\noindent Rule induction is triggered by a fact $(x_1,\dots,x_n)
\in R$ piped into the proof, here \isakeyword{using}~\isa{A}. The
proof itself follows the inductive definition very
closely: there is one case for each rule, and it has the same name as
the rule, analogous to structural induction.

However, this proof is rather terse. Here is a more readable version:%
\end{isamarkuptext}%
\isamarkuptrue%
\isacommand{lemma}\isamarkupfalse%
\ \isakeyword{assumes}\ {\isaliteral{22}{\isachardoublequoteopen}}{\isaliteral{28}{\isacharparenleft}}x{\isaliteral{2C}{\isacharcomma}}y{\isaliteral{29}{\isacharparenright}}\ {\isaliteral{5C3C696E3E}{\isasymin}}\ r{\isaliteral{2A}{\isacharasterisk}}{\isaliteral{22}{\isachardoublequoteclose}}\ \isakeyword{and}\ {\isaliteral{22}{\isachardoublequoteopen}}{\isaliteral{28}{\isacharparenleft}}y{\isaliteral{2C}{\isacharcomma}}z{\isaliteral{29}{\isacharparenright}}\ {\isaliteral{5C3C696E3E}{\isasymin}}\ r{\isaliteral{2A}{\isacharasterisk}}{\isaliteral{22}{\isachardoublequoteclose}}\ \isakeyword{shows}\ {\isaliteral{22}{\isachardoublequoteopen}}{\isaliteral{28}{\isacharparenleft}}x{\isaliteral{2C}{\isacharcomma}}z{\isaliteral{29}{\isacharparenright}}\ {\isaliteral{5C3C696E3E}{\isasymin}}\ r{\isaliteral{2A}{\isacharasterisk}}{\isaliteral{22}{\isachardoublequoteclose}}\isanewline
%
\isadelimproof
%
\endisadelimproof
%
\isatagproof
\isacommand{using}\isamarkupfalse%
\ assms\isanewline
\isacommand{proof}\isamarkupfalse%
\ induct\isanewline
\ \ \isacommand{fix}\isamarkupfalse%
\ x\ \isacommand{assume}\isamarkupfalse%
\ {\isaliteral{22}{\isachardoublequoteopen}}{\isaliteral{28}{\isacharparenleft}}x{\isaliteral{2C}{\isacharcomma}}z{\isaliteral{29}{\isacharparenright}}\ {\isaliteral{5C3C696E3E}{\isasymin}}\ r{\isaliteral{2A}{\isacharasterisk}}{\isaliteral{22}{\isachardoublequoteclose}}\ \ %
\isamarkupcmt{\isa{B}[\isa{y} := \isa{x}]%
}
\isanewline
\ \ \isacommand{thus}\isamarkupfalse%
\ {\isaliteral{22}{\isachardoublequoteopen}}{\isaliteral{28}{\isacharparenleft}}x{\isaliteral{2C}{\isacharcomma}}z{\isaliteral{29}{\isacharparenright}}\ {\isaliteral{5C3C696E3E}{\isasymin}}\ r{\isaliteral{2A}{\isacharasterisk}}{\isaliteral{22}{\isachardoublequoteclose}}\ \isacommand{{\isaliteral{2E}{\isachardot}}}\isamarkupfalse%
\isanewline
\isacommand{next}\isamarkupfalse%
\isanewline
\ \ \isacommand{fix}\isamarkupfalse%
\ x{\isaliteral{27}{\isacharprime}}\ x\ y\isanewline
\ \ \isacommand{assume}\isamarkupfalse%
\ {\isadigit{1}}{\isaliteral{3A}{\isacharcolon}}\ {\isaliteral{22}{\isachardoublequoteopen}}{\isaliteral{28}{\isacharparenleft}}x{\isaliteral{27}{\isacharprime}}{\isaliteral{2C}{\isacharcomma}}x{\isaliteral{29}{\isacharparenright}}\ {\isaliteral{5C3C696E3E}{\isasymin}}\ r{\isaliteral{22}{\isachardoublequoteclose}}\ \isakeyword{and}\isanewline
\ \ \ \ \ \ \ \ \ IH{\isaliteral{3A}{\isacharcolon}}\ {\isaliteral{22}{\isachardoublequoteopen}}{\isaliteral{28}{\isacharparenleft}}y{\isaliteral{2C}{\isacharcomma}}z{\isaliteral{29}{\isacharparenright}}\ {\isaliteral{5C3C696E3E}{\isasymin}}\ r{\isaliteral{2A}{\isacharasterisk}}\ {\isaliteral{5C3C4C6F6E6772696768746172726F773E}{\isasymLongrightarrow}}\ {\isaliteral{28}{\isacharparenleft}}x{\isaliteral{2C}{\isacharcomma}}z{\isaliteral{29}{\isacharparenright}}\ {\isaliteral{5C3C696E3E}{\isasymin}}\ r{\isaliteral{2A}{\isacharasterisk}}{\isaliteral{22}{\isachardoublequoteclose}}\ \isakeyword{and}\isanewline
\ \ \ \ \ \ \ \ \ B{\isaliteral{3A}{\isacharcolon}}\ \ {\isaliteral{22}{\isachardoublequoteopen}}{\isaliteral{28}{\isacharparenleft}}y{\isaliteral{2C}{\isacharcomma}}z{\isaliteral{29}{\isacharparenright}}\ {\isaliteral{5C3C696E3E}{\isasymin}}\ r{\isaliteral{2A}{\isacharasterisk}}{\isaliteral{22}{\isachardoublequoteclose}}\isanewline
\ \ \isacommand{from}\isamarkupfalse%
\ {\isadigit{1}}\ IH{\isaliteral{5B}{\isacharbrackleft}}OF\ B{\isaliteral{5D}{\isacharbrackright}}\ \isacommand{show}\isamarkupfalse%
\ {\isaliteral{22}{\isachardoublequoteopen}}{\isaliteral{28}{\isacharparenleft}}x{\isaliteral{27}{\isacharprime}}{\isaliteral{2C}{\isacharcomma}}z{\isaliteral{29}{\isacharparenright}}\ {\isaliteral{5C3C696E3E}{\isasymin}}\ r{\isaliteral{2A}{\isacharasterisk}}{\isaliteral{22}{\isachardoublequoteclose}}\ \isacommand{by}\isamarkupfalse%
{\isaliteral{28}{\isacharparenleft}}rule\ rtc{\isaliteral{2E}{\isachardot}}step{\isaliteral{29}{\isacharparenright}}\isanewline
\isacommand{qed}\isamarkupfalse%
%
\endisatagproof
{\isafoldproof}%
%
\isadelimproof
%
\endisadelimproof
%
\begin{isamarkuptext}%
\noindent
This time, merely for a change, we start the proof with by feeding both
assumptions into the inductive proof. Only the first assumption is
``consumed'' by the induction.
Since the second one is left over we don't just prove \isa{{\isaliteral{3F}{\isacharquery}}thesis} but
\isa{{\isaliteral{28}{\isacharparenleft}}y{\isaliteral{2C}{\isacharcomma}}z{\isaliteral{29}{\isacharparenright}}\ {\isaliteral{5C3C696E3E}{\isasymin}}\ r{\isaliteral{2A}{\isacharasterisk}}\ {\isaliteral{5C3C4C6F6E6772696768746172726F773E}{\isasymLongrightarrow}}\ {\isaliteral{3F}{\isacharquery}}thesis}, just as in the previous proof.
The base case is trivial. In the assumptions for the induction step we can
see very clearly how things fit together and permit ourselves the
obvious forward step \isa{IH{\isaliteral{5B}{\isacharbrackleft}}OF\ B{\isaliteral{5D}{\isacharbrackright}}}.

The notation \isakeyword{case}~\isa{(}\emph{constructor} \emph{vars}\isa{)}
is also supported for inductive definitions. The \emph{constructor} is the
name of the rule and the \emph{vars} fix the free variables in the
rule; the order of the \emph{vars} must correspond to the
left-to-right order of the variables as they appear in the rule.
For example, we could start the above detailed proof of the induction
with \isakeyword{case}~\isa{(step x' x y)}. In that case we don't need
to spell out the assumptions but can refer to them by \isa{step{\isaliteral{28}{\isacharparenleft}}{\isaliteral{2E}{\isachardot}}{\isaliteral{29}{\isacharparenright}}},
although the resulting text will be quite cryptic.

\subsection{More induction}

We close the section by demonstrating how arbitrary induction
rules are applied. As a simple example we have chosen recursion
induction, i.e.\ induction based on a recursive function
definition. However, most of what we show works for induction in
general.

The example is an unusual definition of rotation:%
\end{isamarkuptext}%
\isamarkuptrue%
\isacommand{fun}\isamarkupfalse%
\ rot\ {\isaliteral{3A}{\isacharcolon}}{\isaliteral{3A}{\isacharcolon}}\ {\isaliteral{22}{\isachardoublequoteopen}}{\isaliteral{27}{\isacharprime}}a\ list\ {\isaliteral{5C3C52696768746172726F773E}{\isasymRightarrow}}\ {\isaliteral{27}{\isacharprime}}a\ list{\isaliteral{22}{\isachardoublequoteclose}}\ \isakeyword{where}\isanewline
{\isaliteral{22}{\isachardoublequoteopen}}rot\ {\isaliteral{5B}{\isacharbrackleft}}{\isaliteral{5D}{\isacharbrackright}}\ {\isaliteral{3D}{\isacharequal}}\ {\isaliteral{5B}{\isacharbrackleft}}{\isaliteral{5D}{\isacharbrackright}}{\isaliteral{22}{\isachardoublequoteclose}}\ {\isaliteral{7C}{\isacharbar}}\isanewline
{\isaliteral{22}{\isachardoublequoteopen}}rot\ {\isaliteral{5B}{\isacharbrackleft}}x{\isaliteral{5D}{\isacharbrackright}}\ {\isaliteral{3D}{\isacharequal}}\ {\isaliteral{5B}{\isacharbrackleft}}x{\isaliteral{5D}{\isacharbrackright}}{\isaliteral{22}{\isachardoublequoteclose}}\ {\isaliteral{7C}{\isacharbar}}\isanewline
{\isaliteral{22}{\isachardoublequoteopen}}rot\ {\isaliteral{28}{\isacharparenleft}}x{\isaliteral{23}{\isacharhash}}y{\isaliteral{23}{\isacharhash}}zs{\isaliteral{29}{\isacharparenright}}\ {\isaliteral{3D}{\isacharequal}}\ y\ {\isaliteral{23}{\isacharhash}}\ rot{\isaliteral{28}{\isacharparenleft}}x{\isaliteral{23}{\isacharhash}}zs{\isaliteral{29}{\isacharparenright}}{\isaliteral{22}{\isachardoublequoteclose}}%
\begin{isamarkuptext}%
\noindent This yields, among other things, the induction rule
\isa{rot{\isaliteral{2E}{\isachardot}}induct}: \begin{isabelle}%
{\isaliteral{5C3C6C6272616B6B3E}{\isasymlbrakk}}P\ {\isaliteral{5B}{\isacharbrackleft}}{\isaliteral{5D}{\isacharbrackright}}{\isaliteral{3B}{\isacharsemicolon}}\ {\isaliteral{5C3C416E643E}{\isasymAnd}}x{\isaliteral{2E}{\isachardot}}\ P\ {\isaliteral{5B}{\isacharbrackleft}}x{\isaliteral{5D}{\isacharbrackright}}{\isaliteral{3B}{\isacharsemicolon}}\ {\isaliteral{5C3C416E643E}{\isasymAnd}}x\ y\ zs{\isaliteral{2E}{\isachardot}}\ P\ {\isaliteral{28}{\isacharparenleft}}x\ {\isaliteral{23}{\isacharhash}}\ zs{\isaliteral{29}{\isacharparenright}}\ {\isaliteral{5C3C4C6F6E6772696768746172726F773E}{\isasymLongrightarrow}}\ P\ {\isaliteral{28}{\isacharparenleft}}x\ {\isaliteral{23}{\isacharhash}}\ y\ {\isaliteral{23}{\isacharhash}}\ zs{\isaliteral{29}{\isacharparenright}}{\isaliteral{5C3C726272616B6B3E}{\isasymrbrakk}}\ {\isaliteral{5C3C4C6F6E6772696768746172726F773E}{\isasymLongrightarrow}}\ P\ a{\isadigit{0}}%
\end{isabelle}
The following proof relies on a default naming scheme for cases: they are
called 1, 2, etc, unless they have been named explicitly. The latter happens
only with datatypes and inductively defined sets, but (usually)
not with recursive functions.%
\end{isamarkuptext}%
\isamarkuptrue%
\isacommand{lemma}\isamarkupfalse%
\ {\isaliteral{22}{\isachardoublequoteopen}}xs\ {\isaliteral{5C3C6E6F7465713E}{\isasymnoteq}}\ {\isaliteral{5B}{\isacharbrackleft}}{\isaliteral{5D}{\isacharbrackright}}\ {\isaliteral{5C3C4C6F6E6772696768746172726F773E}{\isasymLongrightarrow}}\ rot\ xs\ {\isaliteral{3D}{\isacharequal}}\ tl\ xs\ {\isaliteral{40}{\isacharat}}\ {\isaliteral{5B}{\isacharbrackleft}}hd\ xs{\isaliteral{5D}{\isacharbrackright}}{\isaliteral{22}{\isachardoublequoteclose}}\isanewline
%
\isadelimproof
%
\endisadelimproof
%
\isatagproof
\isacommand{proof}\isamarkupfalse%
\ {\isaliteral{28}{\isacharparenleft}}induct\ xs\ rule{\isaliteral{3A}{\isacharcolon}}\ rot{\isaliteral{2E}{\isachardot}}induct{\isaliteral{29}{\isacharparenright}}\isanewline
\ \ \isacommand{case}\isamarkupfalse%
\ {\isadigit{1}}\ \isacommand{thus}\isamarkupfalse%
\ {\isaliteral{3F}{\isacharquery}}case\ \isacommand{by}\isamarkupfalse%
\ simp\isanewline
\isacommand{next}\isamarkupfalse%
\isanewline
\ \ \isacommand{case}\isamarkupfalse%
\ {\isadigit{2}}\ \isacommand{show}\isamarkupfalse%
\ {\isaliteral{3F}{\isacharquery}}case\ \isacommand{by}\isamarkupfalse%
\ simp\isanewline
\isacommand{next}\isamarkupfalse%
\isanewline
\ \ \isacommand{case}\isamarkupfalse%
\ {\isaliteral{28}{\isacharparenleft}}{\isadigit{3}}\ a\ b\ cs{\isaliteral{29}{\isacharparenright}}\isanewline
\ \ \isacommand{have}\isamarkupfalse%
\ {\isaliteral{22}{\isachardoublequoteopen}}rot\ {\isaliteral{28}{\isacharparenleft}}a\ {\isaliteral{23}{\isacharhash}}\ b\ {\isaliteral{23}{\isacharhash}}\ cs{\isaliteral{29}{\isacharparenright}}\ {\isaliteral{3D}{\isacharequal}}\ b\ {\isaliteral{23}{\isacharhash}}\ rot{\isaliteral{28}{\isacharparenleft}}a\ {\isaliteral{23}{\isacharhash}}\ cs{\isaliteral{29}{\isacharparenright}}{\isaliteral{22}{\isachardoublequoteclose}}\ \isacommand{by}\isamarkupfalse%
\ simp\isanewline
\ \ \isacommand{also}\isamarkupfalse%
\ \isacommand{have}\isamarkupfalse%
\ {\isaliteral{22}{\isachardoublequoteopen}}{\isaliteral{5C3C646F74733E}{\isasymdots}}\ {\isaliteral{3D}{\isacharequal}}\ b\ {\isaliteral{23}{\isacharhash}}\ tl{\isaliteral{28}{\isacharparenleft}}a\ {\isaliteral{23}{\isacharhash}}\ cs{\isaliteral{29}{\isacharparenright}}\ {\isaliteral{40}{\isacharat}}\ {\isaliteral{5B}{\isacharbrackleft}}hd{\isaliteral{28}{\isacharparenleft}}a\ {\isaliteral{23}{\isacharhash}}\ cs{\isaliteral{29}{\isacharparenright}}{\isaliteral{5D}{\isacharbrackright}}{\isaliteral{22}{\isachardoublequoteclose}}\ \isacommand{by}\isamarkupfalse%
{\isaliteral{28}{\isacharparenleft}}simp\ add{\isaliteral{3A}{\isacharcolon}}{\isadigit{3}}{\isaliteral{29}{\isacharparenright}}\isanewline
\ \ \isacommand{also}\isamarkupfalse%
\ \isacommand{have}\isamarkupfalse%
\ {\isaliteral{22}{\isachardoublequoteopen}}{\isaliteral{5C3C646F74733E}{\isasymdots}}\ {\isaliteral{3D}{\isacharequal}}\ tl\ {\isaliteral{28}{\isacharparenleft}}a\ {\isaliteral{23}{\isacharhash}}\ b\ {\isaliteral{23}{\isacharhash}}\ cs{\isaliteral{29}{\isacharparenright}}\ {\isaliteral{40}{\isacharat}}\ {\isaliteral{5B}{\isacharbrackleft}}hd\ {\isaliteral{28}{\isacharparenleft}}a\ {\isaliteral{23}{\isacharhash}}\ b\ {\isaliteral{23}{\isacharhash}}\ cs{\isaliteral{29}{\isacharparenright}}{\isaliteral{5D}{\isacharbrackright}}{\isaliteral{22}{\isachardoublequoteclose}}\ \isacommand{by}\isamarkupfalse%
\ simp\isanewline
\ \ \isacommand{finally}\isamarkupfalse%
\ \isacommand{show}\isamarkupfalse%
\ {\isaliteral{3F}{\isacharquery}}case\ \isacommand{{\isaliteral{2E}{\isachardot}}}\isamarkupfalse%
\isanewline
\isacommand{qed}\isamarkupfalse%
%
\endisatagproof
{\isafoldproof}%
%
\isadelimproof
%
\endisadelimproof
%
\begin{isamarkuptext}%
\noindent
The third case is only shown in gory detail (see \cite{BauerW-TPHOLs01}
for how to reason with chains of equations) to demonstrate that the
\isakeyword{case}~\isa{(}\emph{constructor} \emph{vars}\isa{)} notation also
works for arbitrary induction theorems with numbered cases. The order
of the \emph{vars} corresponds to the order of the
\isa{{\isaliteral{5C3C416E643E}{\isasymAnd}}}-quantified variables in each case of the induction
theorem. For induction theorems produced by \isakeyword{fun} it is
the order in which the variables appear on the left-hand side of the
equation.

The proof is so simple that it can be condensed to%
\end{isamarkuptext}%
\isamarkuptrue%
%
\isadelimproof
%
\endisadelimproof
%
\isatagproof
\isacommand{by}\isamarkupfalse%
\ {\isaliteral{28}{\isacharparenleft}}induct\ xs\ rule{\isaliteral{3A}{\isacharcolon}}\ rot{\isaliteral{2E}{\isachardot}}induct{\isaliteral{29}{\isacharparenright}}\ simp{\isaliteral{5F}{\isacharunderscore}}all\isanewline
%
\endisatagproof
{\isafoldproof}%
%
\isadelimproof
%
\endisadelimproof
%
\isadelimtheory
%
\endisadelimtheory
%
\isatagtheory
%
\endisatagtheory
{\isafoldtheory}%
%
\isadelimtheory
%
\endisadelimtheory
\end{isabellebody}%
%%% Local Variables:
%%% mode: latex
%%% TeX-master: "root"
%%% End:
