%
\begin{isabellebody}%
\def\isabellecontext{Logic}%
\isamarkupfalse%
%
\isamarkupsection{Logic \label{sec:Logic}%
}
\isamarkuptrue%
%
\isamarkupsubsection{Propositional logic%
}
\isamarkuptrue%
%
\isamarkupsubsubsection{Introduction rules%
}
\isamarkuptrue%
%
\begin{isamarkuptext}%
We start with a really trivial toy proof to introduce the basic
features of structured proofs.%
\end{isamarkuptext}%
\isamarkuptrue%
\isacommand{lemma}\ {\isachardoublequote}A\ {\isasymlongrightarrow}\ A{\isachardoublequote}\isanewline
\isamarkupfalse%
\isacommand{proof}\ {\isacharparenleft}rule\ impI{\isacharparenright}\isanewline
\ \ \isamarkupfalse%
\isacommand{assume}\ a{\isacharcolon}\ {\isachardoublequote}A{\isachardoublequote}\isanewline
\ \ \isamarkupfalse%
\isacommand{show}\ {\isachardoublequote}A{\isachardoublequote}\ \isamarkupfalse%
\isacommand{by}{\isacharparenleft}rule\ a{\isacharparenright}\isanewline
\isamarkupfalse%
\isacommand{qed}\isamarkupfalse%
%
\begin{isamarkuptext}%
\noindent
The operational reading: the \isakeyword{assume}-\isakeyword{show}
block proves \isa{A\ {\isasymLongrightarrow}\ A} (\isa{a} is a degenerate rule (no
assumptions) that proves \isa{A} outright), which rule
\isa{impI} (\isa{{\isacharparenleft}{\isacharquery}P\ {\isasymLongrightarrow}\ {\isacharquery}Q{\isacharparenright}\ {\isasymLongrightarrow}\ {\isacharquery}P\ {\isasymlongrightarrow}\ {\isacharquery}Q}) turns into the desired \isa{A\ {\isasymlongrightarrow}\ A}.  However, this text is much too detailed for comfort. Therefore
Isar implements the following principle: \begin{quote}\em Command
\isakeyword{proof} automatically tries to select an introduction rule
based on the goal and a predefined list of rules.  \end{quote} Here
\isa{impI} is applied automatically:%
\end{isamarkuptext}%
\isamarkuptrue%
\isacommand{lemma}\ {\isachardoublequote}A\ {\isasymlongrightarrow}\ A{\isachardoublequote}\isanewline
\isamarkupfalse%
\isacommand{proof}\isanewline
\ \ \isamarkupfalse%
\isacommand{assume}\ a{\isacharcolon}\ A\isanewline
\ \ \isamarkupfalse%
\isacommand{show}\ A\ \isamarkupfalse%
\isacommand{by}{\isacharparenleft}rule\ a{\isacharparenright}\isanewline
\isamarkupfalse%
\isacommand{qed}\isamarkupfalse%
%
\begin{isamarkuptext}%
\noindent Single-identifier formulae such as \isa{A} need not
be enclosed in double quotes. However, we will continue to do so for
uniformity.

Trivial proofs, in particular those by assumption, should be trivial
to perform. Proof ``.'' does just that (and a bit more). Thus
naming of assumptions is often superfluous:%
\end{isamarkuptext}%
\isamarkuptrue%
\isacommand{lemma}\ {\isachardoublequote}A\ {\isasymlongrightarrow}\ A{\isachardoublequote}\isanewline
\isamarkupfalse%
\isacommand{proof}\isanewline
\ \ \isamarkupfalse%
\isacommand{assume}\ {\isachardoublequote}A{\isachardoublequote}\isanewline
\ \ \isamarkupfalse%
\isacommand{show}\ {\isachardoublequote}A{\isachardoublequote}\ \isamarkupfalse%
\isacommand{{\isachardot}}\isanewline
\isamarkupfalse%
\isacommand{qed}\isamarkupfalse%
%
\begin{isamarkuptext}%
To hide proofs by assumption further, \isakeyword{by}\isa{{\isacharparenleft}method{\isacharparenright}}
first applies \isa{method} and then tries to solve all remaining subgoals
by assumption:%
\end{isamarkuptext}%
\isamarkuptrue%
\isacommand{lemma}\ {\isachardoublequote}A\ {\isasymlongrightarrow}\ A\ {\isasymand}\ A{\isachardoublequote}\isanewline
\isamarkupfalse%
\isacommand{proof}\isanewline
\ \ \isamarkupfalse%
\isacommand{assume}\ {\isachardoublequote}A{\isachardoublequote}\isanewline
\ \ \isamarkupfalse%
\isacommand{show}\ {\isachardoublequote}A\ {\isasymand}\ A{\isachardoublequote}\ \isamarkupfalse%
\isacommand{by}{\isacharparenleft}rule\ conjI{\isacharparenright}\isanewline
\isamarkupfalse%
\isacommand{qed}\isamarkupfalse%
%
\begin{isamarkuptext}%
\noindent Rule \isa{conjI} is of course \isa{{\isasymlbrakk}{\isacharquery}P{\isacharsemicolon}\ {\isacharquery}Q{\isasymrbrakk}\ {\isasymLongrightarrow}\ {\isacharquery}P\ {\isasymand}\ {\isacharquery}Q}.
A drawback of implicit proofs by assumption is that it
is no longer obvious where an assumption is used.

Proofs of the form \isakeyword{by}\isa{{\isacharparenleft}rule}~\emph{name}\isa{{\isacharparenright}}
can be abbreviated to ``..''  if \emph{name} refers to one of the
predefined introduction rules (or elimination rules, see below):%
\end{isamarkuptext}%
\isamarkuptrue%
\isacommand{lemma}\ {\isachardoublequote}A\ {\isasymlongrightarrow}\ A\ {\isasymand}\ A{\isachardoublequote}\isanewline
\isamarkupfalse%
\isacommand{proof}\isanewline
\ \ \isamarkupfalse%
\isacommand{assume}\ {\isachardoublequote}A{\isachardoublequote}\isanewline
\ \ \isamarkupfalse%
\isacommand{show}\ {\isachardoublequote}A\ {\isasymand}\ A{\isachardoublequote}\ \isamarkupfalse%
\isacommand{{\isachardot}{\isachardot}}\isanewline
\isamarkupfalse%
\isacommand{qed}\isamarkupfalse%
%
\begin{isamarkuptext}%
\noindent
This is what happens: first the matching introduction rule \isa{conjI}
is applied (first ``.''), then the two subgoals are solved by assumption
(second ``.'').%
\end{isamarkuptext}%
\isamarkuptrue%
%
\isamarkupsubsubsection{Elimination rules%
}
\isamarkuptrue%
%
\begin{isamarkuptext}%
A typical elimination rule is \isa{conjE}, $\land$-elimination:
\begin{isabelle}%
\ \ \ \ \ {\isasymlbrakk}{\isacharquery}P\ {\isasymand}\ {\isacharquery}Q{\isacharsemicolon}\ {\isasymlbrakk}{\isacharquery}P{\isacharsemicolon}\ {\isacharquery}Q{\isasymrbrakk}\ {\isasymLongrightarrow}\ {\isacharquery}R{\isasymrbrakk}\ {\isasymLongrightarrow}\ {\isacharquery}R%
\end{isabelle}  In the following proof it is applied
by hand, after its first (\emph{major}) premise has been eliminated via
\isa{{\isacharbrackleft}OF\ AB{\isacharbrackright}}:%
\end{isamarkuptext}%
\isamarkuptrue%
\isacommand{lemma}\ {\isachardoublequote}A\ {\isasymand}\ B\ {\isasymlongrightarrow}\ B\ {\isasymand}\ A{\isachardoublequote}\isanewline
\isamarkupfalse%
\isacommand{proof}\isanewline
\ \ \isamarkupfalse%
\isacommand{assume}\ AB{\isacharcolon}\ {\isachardoublequote}A\ {\isasymand}\ B{\isachardoublequote}\isanewline
\ \ \isamarkupfalse%
\isacommand{show}\ {\isachardoublequote}B\ {\isasymand}\ A{\isachardoublequote}\isanewline
\ \ \isamarkupfalse%
\isacommand{proof}\ {\isacharparenleft}rule\ conjE{\isacharbrackleft}OF\ AB{\isacharbrackright}{\isacharparenright}\ \ %
\isamarkupcmt{\isa{conjE{\isacharbrackleft}OF\ AB{\isacharbrackright}}: \isa{{\isacharparenleft}{\isasymlbrakk}A{\isacharsemicolon}\ B{\isasymrbrakk}\ {\isasymLongrightarrow}\ {\isacharquery}R{\isacharparenright}\ {\isasymLongrightarrow}\ {\isacharquery}R}%
}
\isanewline
\ \ \ \ \isamarkupfalse%
\isacommand{assume}\ {\isachardoublequote}A{\isachardoublequote}\ {\isachardoublequote}B{\isachardoublequote}\isanewline
\ \ \ \ \isamarkupfalse%
\isacommand{show}\ {\isacharquery}thesis\ \isamarkupfalse%
\isacommand{{\isachardot}{\isachardot}}\isanewline
\ \ \isamarkupfalse%
\isacommand{qed}\isanewline
\isamarkupfalse%
\isacommand{qed}\isamarkupfalse%
%
\begin{isamarkuptext}%
\noindent Note that the term \isa{{\isacharquery}thesis} always stands for the
``current goal'', i.e.\ the enclosing \isakeyword{show} (or
\isakeyword{have}) statement.

This is too much proof text. Elimination rules should be selected
automatically based on their major premise, the formula or rather connective
to be eliminated. In Isar they are triggered by facts being fed
\emph{into} a proof. Syntax:
\begin{center}
\isakeyword{from} \emph{fact} \isakeyword{show} \emph{proposition} \emph{proof}
\end{center}
where \emph{fact} stands for the name of a previously proved
proposition, e.g.\ an assumption, an intermediate result or some global
theorem, which may also be modified with \isa{OF} etc.
The \emph{fact} is ``piped'' into the \emph{proof}, which can deal with it
how it chooses. If the \emph{proof} starts with a plain \isakeyword{proof},
an elimination rule (from a predefined list) is applied
whose first premise is solved by the \emph{fact}. Thus the proof above
is equivalent to the following one:%
\end{isamarkuptext}%
\isamarkuptrue%
\isacommand{lemma}\ {\isachardoublequote}A\ {\isasymand}\ B\ {\isasymlongrightarrow}\ B\ {\isasymand}\ A{\isachardoublequote}\isanewline
\isamarkupfalse%
\isacommand{proof}\isanewline
\ \ \isamarkupfalse%
\isacommand{assume}\ AB{\isacharcolon}\ {\isachardoublequote}A\ {\isasymand}\ B{\isachardoublequote}\isanewline
\ \ \isamarkupfalse%
\isacommand{from}\ AB\ \isamarkupfalse%
\isacommand{show}\ {\isachardoublequote}B\ {\isasymand}\ A{\isachardoublequote}\isanewline
\ \ \isamarkupfalse%
\isacommand{proof}\isanewline
\ \ \ \ \isamarkupfalse%
\isacommand{assume}\ {\isachardoublequote}A{\isachardoublequote}\ {\isachardoublequote}B{\isachardoublequote}\isanewline
\ \ \ \ \isamarkupfalse%
\isacommand{show}\ {\isacharquery}thesis\ \isamarkupfalse%
\isacommand{{\isachardot}{\isachardot}}\isanewline
\ \ \isamarkupfalse%
\isacommand{qed}\isanewline
\isamarkupfalse%
\isacommand{qed}\isamarkupfalse%
%
\begin{isamarkuptext}%
Now we come to a second important principle:
\begin{quote}\em
Try to arrange the sequence of propositions in a UNIX-like pipe,
such that the proof of each proposition builds on the previous proposition.
\end{quote}
The previous proposition can be referred to via the fact \isa{this}.
This greatly reduces the need for explicit naming of propositions:%
\end{isamarkuptext}%
\isamarkuptrue%
\isacommand{lemma}\ {\isachardoublequote}A\ {\isasymand}\ B\ {\isasymlongrightarrow}\ B\ {\isasymand}\ A{\isachardoublequote}\isanewline
\isamarkupfalse%
\isacommand{proof}\isanewline
\ \ \isamarkupfalse%
\isacommand{assume}\ {\isachardoublequote}A\ {\isasymand}\ B{\isachardoublequote}\isanewline
\ \ \isamarkupfalse%
\isacommand{from}\ this\ \isamarkupfalse%
\isacommand{show}\ {\isachardoublequote}B\ {\isasymand}\ A{\isachardoublequote}\isanewline
\ \ \isamarkupfalse%
\isacommand{proof}\isanewline
\ \ \ \ \isamarkupfalse%
\isacommand{assume}\ {\isachardoublequote}A{\isachardoublequote}\ {\isachardoublequote}B{\isachardoublequote}\isanewline
\ \ \ \ \isamarkupfalse%
\isacommand{show}\ {\isacharquery}thesis\ \isamarkupfalse%
\isacommand{{\isachardot}{\isachardot}}\isanewline
\ \ \isamarkupfalse%
\isacommand{qed}\isanewline
\isamarkupfalse%
\isacommand{qed}\isamarkupfalse%
%
\begin{isamarkuptext}%
\noindent Because of the frequency of \isakeyword{from}~\isa{this}, Isar provides two abbreviations:
\begin{center}
\begin{tabular}{r@ {\quad=\quad}l}
\isakeyword{then} & \isakeyword{from} \isa{this} \\
\isakeyword{thus} & \isakeyword{then} \isakeyword{show}
\end{tabular}
\end{center}

Here is an alternative proof that operates purely by forward reasoning:%
\end{isamarkuptext}%
\isamarkuptrue%
\isacommand{lemma}\ {\isachardoublequote}A\ {\isasymand}\ B\ {\isasymlongrightarrow}\ B\ {\isasymand}\ A{\isachardoublequote}\isanewline
\isamarkupfalse%
\isacommand{proof}\isanewline
\ \ \isamarkupfalse%
\isacommand{assume}\ ab{\isacharcolon}\ {\isachardoublequote}A\ {\isasymand}\ B{\isachardoublequote}\isanewline
\ \ \isamarkupfalse%
\isacommand{from}\ ab\ \isamarkupfalse%
\isacommand{have}\ a{\isacharcolon}\ {\isachardoublequote}A{\isachardoublequote}\ \isamarkupfalse%
\isacommand{{\isachardot}{\isachardot}}\isanewline
\ \ \isamarkupfalse%
\isacommand{from}\ ab\ \isamarkupfalse%
\isacommand{have}\ b{\isacharcolon}\ {\isachardoublequote}B{\isachardoublequote}\ \isamarkupfalse%
\isacommand{{\isachardot}{\isachardot}}\isanewline
\ \ \isamarkupfalse%
\isacommand{from}\ b\ a\ \isamarkupfalse%
\isacommand{show}\ {\isachardoublequote}B\ {\isasymand}\ A{\isachardoublequote}\ \isamarkupfalse%
\isacommand{{\isachardot}{\isachardot}}\isanewline
\isamarkupfalse%
\isacommand{qed}\isamarkupfalse%
%
\begin{isamarkuptext}%
\noindent It is worth examining this text in detail because it
exhibits a number of new concepts.  For a start, it is the first time
we have proved intermediate propositions (\isakeyword{have}) on the
way to the final \isakeyword{show}. This is the norm in nontrivial
proofs where one cannot bridge the gap between the assumptions and the
conclusion in one step. To understand how the proof works we need to
explain more Isar details.

Method \isa{rule} can be given a list of rules, in which case
\isa{{\isacharparenleft}rule}~\textit{rules}\isa{{\isacharparenright}} applies the first matching
rule in the list \textit{rules}. Command \isakeyword{from} can be
followed by any number of facts.  Given \isakeyword{from}~\isa{f}$_1$~\dots~\isa{f}$_n$, the proof step
\isa{{\isacharparenleft}rule}~\textit{rules}\isa{{\isacharparenright}} following a \isakeyword{have}
or \isakeyword{show} searches \textit{rules} for a rule whose first
$n$ premises can be proved by \isa{f}$_1$~\dots~\isa{f}$_n$ in the
given order. Finally one needs to know that ``..'' is short for
\isa{by{\isacharparenleft}rule}~\textit{elim-rules intro-rules}\isa{{\isacharparenright}} (or
\isa{by{\isacharparenleft}rule}~\textit{intro-rules}\isa{{\isacharparenright}} if there are no facts
fed into the proof), i.e.\ elimination rules are tried before
introduction rules.

Thus in the above proof both \isakeyword{have}s are proved via
\isa{conjE} triggered by \isakeyword{from}~\isa{ab} whereas
in the \isakeyword{show} step no elimination rule is applicable and
the proof succeeds with \isa{conjI}. The latter would fail had
we written \isakeyword{from}~\isa{a\ b} instead of
\isakeyword{from}~\isa{b\ a}.

Proofs starting with a plain \isa{proof} behave the same because the
latter is short for \isa{proof\ {\isacharparenleft}rule}~\textit{elim-rules
intro-rules}\isa{{\isacharparenright}} (or \isa{proof\ {\isacharparenleft}rule}~\textit{intro-rules}\isa{{\isacharparenright}} if there are no facts fed into
the proof).%
\end{isamarkuptext}%
\isamarkuptrue%
%
\isamarkupsubsection{More constructs%
}
\isamarkuptrue%
%
\begin{isamarkuptext}%
In the previous proof of \isa{A\ {\isasymand}\ B\ {\isasymlongrightarrow}\ B\ {\isasymand}\ A} we needed to feed
more than one fact into a proof step, a frequent situation. Then the
UNIX-pipe model appears to break down and we need to name the different
facts to refer to them. But this can be avoided:%
\end{isamarkuptext}%
\isamarkuptrue%
\isacommand{lemma}\ {\isachardoublequote}A\ {\isasymand}\ B\ {\isasymlongrightarrow}\ B\ {\isasymand}\ A{\isachardoublequote}\isanewline
\isamarkupfalse%
\isacommand{proof}\isanewline
\ \ \isamarkupfalse%
\isacommand{assume}\ ab{\isacharcolon}\ {\isachardoublequote}A\ {\isasymand}\ B{\isachardoublequote}\isanewline
\ \ \isamarkupfalse%
\isacommand{from}\ ab\ \isamarkupfalse%
\isacommand{have}\ {\isachardoublequote}B{\isachardoublequote}\ \isamarkupfalse%
\isacommand{{\isachardot}{\isachardot}}\isanewline
\ \ \isamarkupfalse%
\isacommand{moreover}\isanewline
\ \ \isamarkupfalse%
\isacommand{from}\ ab\ \isamarkupfalse%
\isacommand{have}\ {\isachardoublequote}A{\isachardoublequote}\ \isamarkupfalse%
\isacommand{{\isachardot}{\isachardot}}\isanewline
\ \ \isamarkupfalse%
\isacommand{ultimately}\ \isamarkupfalse%
\isacommand{show}\ {\isachardoublequote}B\ {\isasymand}\ A{\isachardoublequote}\ \isamarkupfalse%
\isacommand{{\isachardot}{\isachardot}}\isanewline
\isamarkupfalse%
\isacommand{qed}\isamarkupfalse%
%
\begin{isamarkuptext}%
\noindent You can combine any number of facts \isa{A{\isadigit{1}}} \dots\ \isa{An} into a sequence by separating their proofs with
\isakeyword{moreover}. After the final fact, \isakeyword{ultimately} stands
for \isakeyword{from}~\isa{A{\isadigit{1}}}~\dots~\isa{An}.  This avoids having to
introduce names for all of the sequence elements.%
\end{isamarkuptext}%
\isamarkuptrue%
%
\begin{isamarkuptext}%
Although we have only seen a few introduction and elimination rules so
far, Isar's predefined rules include all the usual natural deduction
rules. We conclude our exposition of propositional logic with an extended
example --- which rules are used implicitly where?%
\end{isamarkuptext}%
\isamarkuptrue%
\isacommand{lemma}\ {\isachardoublequote}{\isasymnot}\ {\isacharparenleft}A\ {\isasymand}\ B{\isacharparenright}\ {\isasymlongrightarrow}\ {\isasymnot}\ A\ {\isasymor}\ {\isasymnot}\ B{\isachardoublequote}\isanewline
\isamarkupfalse%
\isacommand{proof}\isanewline
\ \ \isamarkupfalse%
\isacommand{assume}\ n{\isacharcolon}\ {\isachardoublequote}{\isasymnot}\ {\isacharparenleft}A\ {\isasymand}\ B{\isacharparenright}{\isachardoublequote}\isanewline
\ \ \isamarkupfalse%
\isacommand{show}\ {\isachardoublequote}{\isasymnot}\ A\ {\isasymor}\ {\isasymnot}\ B{\isachardoublequote}\isanewline
\ \ \isamarkupfalse%
\isacommand{proof}\ {\isacharparenleft}rule\ ccontr{\isacharparenright}\isanewline
\ \ \ \ \isamarkupfalse%
\isacommand{assume}\ nn{\isacharcolon}\ {\isachardoublequote}{\isasymnot}\ {\isacharparenleft}{\isasymnot}\ A\ {\isasymor}\ {\isasymnot}\ B{\isacharparenright}{\isachardoublequote}\isanewline
\ \ \ \ \isamarkupfalse%
\isacommand{have}\ {\isachardoublequote}{\isasymnot}\ A{\isachardoublequote}\isanewline
\ \ \ \ \isamarkupfalse%
\isacommand{proof}\isanewline
\ \ \ \ \ \ \isamarkupfalse%
\isacommand{assume}\ {\isachardoublequote}A{\isachardoublequote}\isanewline
\ \ \ \ \ \ \isamarkupfalse%
\isacommand{have}\ {\isachardoublequote}{\isasymnot}\ B{\isachardoublequote}\isanewline
\ \ \ \ \ \ \isamarkupfalse%
\isacommand{proof}\isanewline
\ \ \ \ \ \ \ \ \isamarkupfalse%
\isacommand{assume}\ {\isachardoublequote}B{\isachardoublequote}\isanewline
\ \ \ \ \ \ \ \ \isamarkupfalse%
\isacommand{have}\ {\isachardoublequote}A\ {\isasymand}\ B{\isachardoublequote}\ \isamarkupfalse%
\isacommand{{\isachardot}{\isachardot}}\isanewline
\ \ \ \ \ \ \ \ \isamarkupfalse%
\isacommand{with}\ n\ \isamarkupfalse%
\isacommand{show}\ False\ \isamarkupfalse%
\isacommand{{\isachardot}{\isachardot}}\isanewline
\ \ \ \ \ \ \isamarkupfalse%
\isacommand{qed}\isanewline
\ \ \ \ \ \ \isamarkupfalse%
\isacommand{hence}\ {\isachardoublequote}{\isasymnot}\ A\ {\isasymor}\ {\isasymnot}\ B{\isachardoublequote}\ \isamarkupfalse%
\isacommand{{\isachardot}{\isachardot}}\isanewline
\ \ \ \ \ \ \isamarkupfalse%
\isacommand{with}\ nn\ \isamarkupfalse%
\isacommand{show}\ False\ \isamarkupfalse%
\isacommand{{\isachardot}{\isachardot}}\isanewline
\ \ \ \ \isamarkupfalse%
\isacommand{qed}\isanewline
\ \ \ \ \isamarkupfalse%
\isacommand{hence}\ {\isachardoublequote}{\isasymnot}\ A\ {\isasymor}\ {\isasymnot}\ B{\isachardoublequote}\ \isamarkupfalse%
\isacommand{{\isachardot}{\isachardot}}\isanewline
\ \ \ \ \isamarkupfalse%
\isacommand{with}\ nn\ \isamarkupfalse%
\isacommand{show}\ False\ \isamarkupfalse%
\isacommand{{\isachardot}{\isachardot}}\isanewline
\ \ \isamarkupfalse%
\isacommand{qed}\isanewline
\isamarkupfalse%
\isacommand{qed}\isamarkupfalse%
%
\begin{isamarkuptext}%
\noindent
Rule \isa{ccontr} (``classical contradiction'') is
\isa{{\isacharparenleft}{\isasymnot}\ P\ {\isasymLongrightarrow}\ False{\isacharparenright}\ {\isasymLongrightarrow}\ P}.
Apart from demonstrating the strangeness of classical
arguments by contradiction, this example also introduces two new
abbreviations:
\begin{center}
\begin{tabular}{l@ {\quad=\quad}l}
\isakeyword{hence} & \isakeyword{then} \isakeyword{have} \\
\isakeyword{with}~\emph{facts} &
\isakeyword{from}~\emph{facts} \isa{this}
\end{tabular}
\end{center}%
\end{isamarkuptext}%
\isamarkuptrue%
%
\isamarkupsubsection{Avoiding duplication%
}
\isamarkuptrue%
%
\begin{isamarkuptext}%
So far our examples have been a bit unnatural: normally we want to
prove rules expressed with \isa{{\isasymLongrightarrow}}, not \isa{{\isasymlongrightarrow}}. Here is an example:%
\end{isamarkuptext}%
\isamarkuptrue%
\isacommand{lemma}\ {\isachardoublequote}A\ {\isasymand}\ B\ {\isasymLongrightarrow}\ B\ {\isasymand}\ A{\isachardoublequote}\isanewline
\isamarkupfalse%
\isacommand{proof}\isanewline
\ \ \isamarkupfalse%
\isacommand{assume}\ {\isachardoublequote}A\ {\isasymand}\ B{\isachardoublequote}\ \isamarkupfalse%
\isacommand{thus}\ {\isachardoublequote}B{\isachardoublequote}\ \isamarkupfalse%
\isacommand{{\isachardot}{\isachardot}}\isanewline
\isamarkupfalse%
\isacommand{next}\isanewline
\ \ \isamarkupfalse%
\isacommand{assume}\ {\isachardoublequote}A\ {\isasymand}\ B{\isachardoublequote}\ \isamarkupfalse%
\isacommand{thus}\ {\isachardoublequote}A{\isachardoublequote}\ \isamarkupfalse%
\isacommand{{\isachardot}{\isachardot}}\isanewline
\isamarkupfalse%
\isacommand{qed}\isamarkupfalse%
%
\begin{isamarkuptext}%
\noindent The \isakeyword{proof} always works on the conclusion,
\isa{B\ {\isasymand}\ A} in our case, thus selecting $\land$-introduction. Hence
we must show \isa{B} and \isa{A}; both are proved by
$\land$-elimination and the proofs are separated by \isakeyword{next}:
\begin{description}
\item[\isakeyword{next}] deals with multiple subgoals. For example,
when showing \isa{A\ {\isasymand}\ B} we need to show both \isa{A} and \isa{B}.  Each subgoal is proved separately, in \emph{any} order. The
individual proofs are separated by \isakeyword{next}.  \footnote{Each
\isakeyword{show} must prove one of the pending subgoals.  If a
\isakeyword{show} matches multiple subgoals, e.g.\ if the subgoals
contain ?-variables, the first one is proved. Thus the order in which
the subgoals are proved can matter --- see
\S\ref{sec:CaseDistinction} for an example.}

Strictly speaking \isakeyword{next} is only required if the subgoals
are proved in different assumption contexts which need to be
separated, which is not the case above. For clarity we
have employed \isakeyword{next} anyway and will continue to do so.
\end{description}

This is all very well as long as formulae are small. Let us now look at some
devices to avoid repeating (possibly large) formulae. A very general method
is pattern matching:%
\end{isamarkuptext}%
\isamarkuptrue%
\isacommand{lemma}\ {\isachardoublequote}large{\isacharunderscore}A\ {\isasymand}\ large{\isacharunderscore}B\ {\isasymLongrightarrow}\ large{\isacharunderscore}B\ {\isasymand}\ large{\isacharunderscore}A{\isachardoublequote}\isanewline
\ \ \ \ \ \ {\isacharparenleft}\isakeyword{is}\ {\isachardoublequote}{\isacharquery}AB\ {\isasymLongrightarrow}\ {\isacharquery}B\ {\isasymand}\ {\isacharquery}A{\isachardoublequote}{\isacharparenright}\isanewline
\isamarkupfalse%
\isacommand{proof}\isanewline
\ \ \isamarkupfalse%
\isacommand{assume}\ {\isachardoublequote}{\isacharquery}AB{\isachardoublequote}\ \isamarkupfalse%
\isacommand{thus}\ {\isachardoublequote}{\isacharquery}B{\isachardoublequote}\ \isamarkupfalse%
\isacommand{{\isachardot}{\isachardot}}\isanewline
\isamarkupfalse%
\isacommand{next}\isanewline
\ \ \isamarkupfalse%
\isacommand{assume}\ {\isachardoublequote}{\isacharquery}AB{\isachardoublequote}\ \isamarkupfalse%
\isacommand{thus}\ {\isachardoublequote}{\isacharquery}A{\isachardoublequote}\ \isamarkupfalse%
\isacommand{{\isachardot}{\isachardot}}\isanewline
\isamarkupfalse%
\isacommand{qed}\isamarkupfalse%
%
\begin{isamarkuptext}%
\noindent Any formula may be followed by
\isa{{\isacharparenleft}}\isakeyword{is}~\emph{pattern}\isa{{\isacharparenright}} which causes the pattern
to be matched against the formula, instantiating the \isa{{\isacharquery}}-variables in
the pattern. Subsequent uses of these variables in other terms causes
them to be replaced by the terms they stand for.

We can simplify things even more by stating the theorem by means of the
\isakeyword{assumes} and \isakeyword{shows} elements which allow direct
naming of assumptions:%
\end{isamarkuptext}%
\isamarkuptrue%
\isacommand{lemma}\ \isakeyword{assumes}\ AB{\isacharcolon}\ {\isachardoublequote}large{\isacharunderscore}A\ {\isasymand}\ large{\isacharunderscore}B{\isachardoublequote}\isanewline
\ \ \isakeyword{shows}\ {\isachardoublequote}large{\isacharunderscore}B\ {\isasymand}\ large{\isacharunderscore}A{\isachardoublequote}\ {\isacharparenleft}\isakeyword{is}\ {\isachardoublequote}{\isacharquery}B\ {\isasymand}\ {\isacharquery}A{\isachardoublequote}{\isacharparenright}\isanewline
\isamarkupfalse%
\isacommand{proof}\isanewline
\ \ \isamarkupfalse%
\isacommand{from}\ AB\ \isamarkupfalse%
\isacommand{show}\ {\isachardoublequote}{\isacharquery}B{\isachardoublequote}\ \isamarkupfalse%
\isacommand{{\isachardot}{\isachardot}}\isanewline
\isamarkupfalse%
\isacommand{next}\isanewline
\ \ \isamarkupfalse%
\isacommand{from}\ AB\ \isamarkupfalse%
\isacommand{show}\ {\isachardoublequote}{\isacharquery}A{\isachardoublequote}\ \isamarkupfalse%
\isacommand{{\isachardot}{\isachardot}}\isanewline
\isamarkupfalse%
\isacommand{qed}\isamarkupfalse%
%
\begin{isamarkuptext}%
\noindent Note the difference between \isa{{\isacharquery}AB}, a term, and
\isa{AB}, a fact.

Finally we want to start the proof with $\land$-elimination so we
don't have to perform it twice, as above. Here is a slick way to
achieve this:%
\end{isamarkuptext}%
\isamarkuptrue%
\isacommand{lemma}\ \isakeyword{assumes}\ AB{\isacharcolon}\ {\isachardoublequote}large{\isacharunderscore}A\ {\isasymand}\ large{\isacharunderscore}B{\isachardoublequote}\isanewline
\ \ \isakeyword{shows}\ {\isachardoublequote}large{\isacharunderscore}B\ {\isasymand}\ large{\isacharunderscore}A{\isachardoublequote}\ {\isacharparenleft}\isakeyword{is}\ {\isachardoublequote}{\isacharquery}B\ {\isasymand}\ {\isacharquery}A{\isachardoublequote}{\isacharparenright}\isanewline
\isamarkupfalse%
\isacommand{using}\ AB\isanewline
\isamarkupfalse%
\isacommand{proof}\isanewline
\ \ \isamarkupfalse%
\isacommand{assume}\ {\isachardoublequote}{\isacharquery}A{\isachardoublequote}\ {\isachardoublequote}{\isacharquery}B{\isachardoublequote}\ \isamarkupfalse%
\isacommand{show}\ {\isacharquery}thesis\ \isamarkupfalse%
\isacommand{{\isachardot}{\isachardot}}\isanewline
\isamarkupfalse%
\isacommand{qed}\isamarkupfalse%
%
\begin{isamarkuptext}%
\noindent Command \isakeyword{using} can appear before a proof
and adds further facts to those piped into the proof. Here \isa{AB}
is the only such fact and it triggers $\land$-elimination. Another
frequent idiom is as follows:
\begin{center}
\isakeyword{from} \emph{major-facts}~
\isakeyword{show} \emph{proposition}~
\isakeyword{using} \emph{minor-facts}~
\emph{proof}
\end{center}

Sometimes it is necessary to suppress the implicit application of rules in a
\isakeyword{proof}. For example \isakeyword{show}~\isa{A\ {\isasymor}\ B} would
trigger $\lor$-introduction, requiring us to prove \isa{A}. A simple
``\isa{{\isacharminus}}'' prevents this \emph{faux pas}:%
\end{isamarkuptext}%
\isamarkuptrue%
\isacommand{lemma}\ \isakeyword{assumes}\ AB{\isacharcolon}\ {\isachardoublequote}A\ {\isasymor}\ B{\isachardoublequote}\ \isakeyword{shows}\ {\isachardoublequote}B\ {\isasymor}\ A{\isachardoublequote}\isanewline
\isamarkupfalse%
\isacommand{proof}\ {\isacharminus}\isanewline
\ \ \isamarkupfalse%
\isacommand{from}\ AB\ \isamarkupfalse%
\isacommand{show}\ {\isacharquery}thesis\isanewline
\ \ \isamarkupfalse%
\isacommand{proof}\isanewline
\ \ \ \ \isamarkupfalse%
\isacommand{assume}\ A\ \isamarkupfalse%
\isacommand{show}\ {\isacharquery}thesis\ \isamarkupfalse%
\isacommand{{\isachardot}{\isachardot}}\isanewline
\ \ \isamarkupfalse%
\isacommand{next}\isanewline
\ \ \ \ \isamarkupfalse%
\isacommand{assume}\ B\ \isamarkupfalse%
\isacommand{show}\ {\isacharquery}thesis\ \isamarkupfalse%
\isacommand{{\isachardot}{\isachardot}}\isanewline
\ \ \isamarkupfalse%
\isacommand{qed}\isanewline
\isamarkupfalse%
\isacommand{qed}\isamarkupfalse%
%
\isamarkupsubsection{Predicate calculus%
}
\isamarkuptrue%
%
\begin{isamarkuptext}%
Command \isakeyword{fix} introduces new local variables into a
proof. The pair \isakeyword{fix}-\isakeyword{show} corresponds to \isa{{\isasymAnd}}
(the universal quantifier at the
meta-level) just like \isakeyword{assume}-\isakeyword{show} corresponds to
\isa{{\isasymLongrightarrow}}. Here is a sample proof, annotated with the rules that are
applied implicitly:%
\end{isamarkuptext}%
\isamarkuptrue%
\isacommand{lemma}\ \isakeyword{assumes}\ P{\isacharcolon}\ {\isachardoublequote}{\isasymforall}x{\isachardot}\ P\ x{\isachardoublequote}\ \isakeyword{shows}\ {\isachardoublequote}{\isasymforall}x{\isachardot}\ P{\isacharparenleft}f\ x{\isacharparenright}{\isachardoublequote}\isanewline
\isamarkupfalse%
\isacommand{proof}\ \ \ \ \ \ \ \ \ \ \ \ \ \ \ \ \ \ \ \ \ \ \ %
\isamarkupcmt{\isa{allI}: \isa{{\isacharparenleft}{\isasymAnd}x{\isachardot}\ {\isacharquery}P\ x{\isacharparenright}\ {\isasymLongrightarrow}\ {\isasymforall}x{\isachardot}\ {\isacharquery}P\ x}%
}
\isanewline
\ \ \isamarkupfalse%
\isacommand{fix}\ a\isanewline
\ \ \isamarkupfalse%
\isacommand{from}\ P\ \isamarkupfalse%
\isacommand{show}\ {\isachardoublequote}P{\isacharparenleft}f\ a{\isacharparenright}{\isachardoublequote}\ \isamarkupfalse%
\isacommand{{\isachardot}{\isachardot}}\ \ %
\isamarkupcmt{\isa{allE}: \isa{{\isasymlbrakk}{\isasymforall}x{\isachardot}\ {\isacharquery}P\ x{\isacharsemicolon}\ {\isacharquery}P\ {\isacharquery}x\ {\isasymLongrightarrow}\ {\isacharquery}R{\isasymrbrakk}\ {\isasymLongrightarrow}\ {\isacharquery}R}%
}
\isanewline
\isamarkupfalse%
\isacommand{qed}\isamarkupfalse%
%
\begin{isamarkuptext}%
\noindent Note that in the proof we have chosen to call the bound
variable \isa{a} instead of \isa{x} merely to show that the choice of
local names is irrelevant.

Next we look at \isa{{\isasymexists}} which is a bit more tricky.%
\end{isamarkuptext}%
\isamarkuptrue%
\isacommand{lemma}\ \isakeyword{assumes}\ Pf{\isacharcolon}\ {\isachardoublequote}{\isasymexists}x{\isachardot}\ P{\isacharparenleft}f\ x{\isacharparenright}{\isachardoublequote}\ \isakeyword{shows}\ {\isachardoublequote}{\isasymexists}y{\isachardot}\ P\ y{\isachardoublequote}\isanewline
\isamarkupfalse%
\isacommand{proof}\ {\isacharminus}\isanewline
\ \ \isamarkupfalse%
\isacommand{from}\ Pf\ \isamarkupfalse%
\isacommand{show}\ {\isacharquery}thesis\isanewline
\ \ \isamarkupfalse%
\isacommand{proof}\ \ \ \ \ \ \ \ \ \ \ \ \ \ %
\isamarkupcmt{\isa{exE}: \isa{{\isasymlbrakk}{\isasymexists}x{\isachardot}\ {\isacharquery}P\ x{\isacharsemicolon}\ {\isasymAnd}x{\isachardot}\ {\isacharquery}P\ x\ {\isasymLongrightarrow}\ {\isacharquery}Q{\isasymrbrakk}\ {\isasymLongrightarrow}\ {\isacharquery}Q}%
}
\isanewline
\ \ \ \ \isamarkupfalse%
\isacommand{fix}\ x\isanewline
\ \ \ \ \isamarkupfalse%
\isacommand{assume}\ {\isachardoublequote}P{\isacharparenleft}f\ x{\isacharparenright}{\isachardoublequote}\isanewline
\ \ \ \ \isamarkupfalse%
\isacommand{show}\ {\isacharquery}thesis\ \isamarkupfalse%
\isacommand{{\isachardot}{\isachardot}}\ \ %
\isamarkupcmt{\isa{exI}: \isa{{\isacharquery}P\ {\isacharquery}x\ {\isasymLongrightarrow}\ {\isasymexists}x{\isachardot}\ {\isacharquery}P\ x}%
}
\isanewline
\ \ \isamarkupfalse%
\isacommand{qed}\isanewline
\isamarkupfalse%
\isacommand{qed}\isamarkupfalse%
%
\begin{isamarkuptext}%
\noindent Explicit $\exists$-elimination as seen above can become
cumbersome in practice.  The derived Isar language element
\isakeyword{obtain} provides a more appealing form of generalised
existence reasoning:%
\end{isamarkuptext}%
\isamarkuptrue%
\isacommand{lemma}\ \isakeyword{assumes}\ Pf{\isacharcolon}\ {\isachardoublequote}{\isasymexists}x{\isachardot}\ P{\isacharparenleft}f\ x{\isacharparenright}{\isachardoublequote}\ \isakeyword{shows}\ {\isachardoublequote}{\isasymexists}y{\isachardot}\ P\ y{\isachardoublequote}\isanewline
\isamarkupfalse%
\isacommand{proof}\ {\isacharminus}\isanewline
\ \ \isamarkupfalse%
\isacommand{from}\ Pf\ \isamarkupfalse%
\isacommand{obtain}\ x\ \isakeyword{where}\ {\isachardoublequote}P{\isacharparenleft}f\ x{\isacharparenright}{\isachardoublequote}\ \isamarkupfalse%
\isacommand{{\isachardot}{\isachardot}}\isanewline
\ \ \isamarkupfalse%
\isacommand{thus}\ {\isachardoublequote}{\isasymexists}y{\isachardot}\ P\ y{\isachardoublequote}\ \isamarkupfalse%
\isacommand{{\isachardot}{\isachardot}}\isanewline
\isamarkupfalse%
\isacommand{qed}\isamarkupfalse%
%
\begin{isamarkuptext}%
\noindent Note how the proof text follows the usual mathematical style
of concluding $P(x)$ from $\exists x. P(x)$, while carefully introducing $x$
as a new local variable.  Technically, \isakeyword{obtain} is similar to
\isakeyword{fix} and \isakeyword{assume} together with a soundness proof of
the elimination involved.

Here is a proof of a well known tautology.
Which rule is used where?%
\end{isamarkuptext}%
\isamarkuptrue%
\isacommand{lemma}\ \isakeyword{assumes}\ ex{\isacharcolon}\ {\isachardoublequote}{\isasymexists}x{\isachardot}\ {\isasymforall}y{\isachardot}\ P\ x\ y{\isachardoublequote}\ \isakeyword{shows}\ {\isachardoublequote}{\isasymforall}y{\isachardot}\ {\isasymexists}x{\isachardot}\ P\ x\ y{\isachardoublequote}\isanewline
\isamarkupfalse%
\isacommand{proof}\isanewline
\ \ \isamarkupfalse%
\isacommand{fix}\ y\isanewline
\ \ \isamarkupfalse%
\isacommand{from}\ ex\ \isamarkupfalse%
\isacommand{obtain}\ x\ \isakeyword{where}\ {\isachardoublequote}{\isasymforall}y{\isachardot}\ P\ x\ y{\isachardoublequote}\ \isamarkupfalse%
\isacommand{{\isachardot}{\isachardot}}\isanewline
\ \ \isamarkupfalse%
\isacommand{hence}\ {\isachardoublequote}P\ x\ y{\isachardoublequote}\ \isamarkupfalse%
\isacommand{{\isachardot}{\isachardot}}\isanewline
\ \ \isamarkupfalse%
\isacommand{thus}\ {\isachardoublequote}{\isasymexists}x{\isachardot}\ P\ x\ y{\isachardoublequote}\ \isamarkupfalse%
\isacommand{{\isachardot}{\isachardot}}\isanewline
\isamarkupfalse%
\isacommand{qed}\isamarkupfalse%
%
\isamarkupsubsection{Making bigger steps%
}
\isamarkuptrue%
%
\begin{isamarkuptext}%
So far we have confined ourselves to single step proofs. Of course
powerful automatic methods can be used just as well. Here is an example,
Cantor's theorem that there is no surjective function from a set to its
powerset:%
\end{isamarkuptext}%
\isamarkuptrue%
\isacommand{theorem}\ {\isachardoublequote}{\isasymexists}S{\isachardot}\ S\ {\isasymnotin}\ range\ {\isacharparenleft}f\ {\isacharcolon}{\isacharcolon}\ {\isacharprime}a\ {\isasymRightarrow}\ {\isacharprime}a\ set{\isacharparenright}{\isachardoublequote}\isanewline
\isamarkupfalse%
\isacommand{proof}\isanewline
\ \ \isamarkupfalse%
\isacommand{let}\ {\isacharquery}S\ {\isacharequal}\ {\isachardoublequote}{\isacharbraceleft}x{\isachardot}\ x\ {\isasymnotin}\ f\ x{\isacharbraceright}{\isachardoublequote}\isanewline
\ \ \isamarkupfalse%
\isacommand{show}\ {\isachardoublequote}{\isacharquery}S\ {\isasymnotin}\ range\ f{\isachardoublequote}\isanewline
\ \ \isamarkupfalse%
\isacommand{proof}\isanewline
\ \ \ \ \isamarkupfalse%
\isacommand{assume}\ {\isachardoublequote}{\isacharquery}S\ {\isasymin}\ range\ f{\isachardoublequote}\isanewline
\ \ \ \ \isamarkupfalse%
\isacommand{then}\ \isamarkupfalse%
\isacommand{obtain}\ y\ \isakeyword{where}\ fy{\isacharcolon}\ {\isachardoublequote}{\isacharquery}S\ {\isacharequal}\ f\ y{\isachardoublequote}\ \isamarkupfalse%
\isacommand{{\isachardot}{\isachardot}}\isanewline
\ \ \ \ \isamarkupfalse%
\isacommand{show}\ False\isanewline
\ \ \ \ \isamarkupfalse%
\isacommand{proof}\ cases\isanewline
\ \ \ \ \ \ \isamarkupfalse%
\isacommand{assume}\ {\isachardoublequote}y\ {\isasymin}\ {\isacharquery}S{\isachardoublequote}\isanewline
\ \ \ \ \ \ \isamarkupfalse%
\isacommand{with}\ fy\ \isamarkupfalse%
\isacommand{show}\ False\ \isamarkupfalse%
\isacommand{by}\ blast\isanewline
\ \ \ \ \isamarkupfalse%
\isacommand{next}\isanewline
\ \ \ \ \ \ \isamarkupfalse%
\isacommand{assume}\ {\isachardoublequote}y\ {\isasymnotin}\ {\isacharquery}S{\isachardoublequote}\isanewline
\ \ \ \ \ \ \isamarkupfalse%
\isacommand{with}\ fy\ \isamarkupfalse%
\isacommand{show}\ False\ \isamarkupfalse%
\isacommand{by}\ blast\isanewline
\ \ \ \ \isamarkupfalse%
\isacommand{qed}\isanewline
\ \ \isamarkupfalse%
\isacommand{qed}\isanewline
\isamarkupfalse%
\isacommand{qed}\isamarkupfalse%
%
\begin{isamarkuptext}%
\noindent
For a start, the example demonstrates two new constructs:
\begin{itemize}
\item \isakeyword{let} introduces an abbreviation for a term, in our case
the witness for the claim.
\item Proof by \isa{cases} starts a proof by cases. Note that it remains
implicit what the two cases are: it is merely expected that the two subproofs
prove \isa{P\ {\isasymLongrightarrow}\ {\isacharquery}thesis} and \isa{{\isasymnot}P\ {\isasymLongrightarrow}\ {\isacharquery}thesis} (in that order)
for some \isa{P}.
\end{itemize}
If you wonder how to \isakeyword{obtain} \isa{y}:
via the predefined elimination rule \isa{{\isasymlbrakk}b\ {\isasymin}\ range\ f{\isacharsemicolon}\ {\isasymAnd}x{\isachardot}\ b\ {\isacharequal}\ f\ x\ {\isasymLongrightarrow}\ P{\isasymrbrakk}\ {\isasymLongrightarrow}\ P}.

Method \isa{blast} is used because the contradiction does not follow easily
by just a single rule. If you find the proof too cryptic for human
consumption, here is a more detailed version; the beginning up to
\isakeyword{obtain} stays unchanged.%
\end{isamarkuptext}%
\isamarkuptrue%
\isacommand{theorem}\ {\isachardoublequote}{\isasymexists}S{\isachardot}\ S\ {\isasymnotin}\ range\ {\isacharparenleft}f\ {\isacharcolon}{\isacharcolon}\ {\isacharprime}a\ {\isasymRightarrow}\ {\isacharprime}a\ set{\isacharparenright}{\isachardoublequote}\isanewline
\isamarkupfalse%
\isacommand{proof}\isanewline
\ \ \isamarkupfalse%
\isacommand{let}\ {\isacharquery}S\ {\isacharequal}\ {\isachardoublequote}{\isacharbraceleft}x{\isachardot}\ x\ {\isasymnotin}\ f\ x{\isacharbraceright}{\isachardoublequote}\isanewline
\ \ \isamarkupfalse%
\isacommand{show}\ {\isachardoublequote}{\isacharquery}S\ {\isasymnotin}\ range\ f{\isachardoublequote}\isanewline
\ \ \isamarkupfalse%
\isacommand{proof}\isanewline
\ \ \ \ \isamarkupfalse%
\isacommand{assume}\ {\isachardoublequote}{\isacharquery}S\ {\isasymin}\ range\ f{\isachardoublequote}\isanewline
\ \ \ \ \isamarkupfalse%
\isacommand{then}\ \isamarkupfalse%
\isacommand{obtain}\ y\ \isakeyword{where}\ fy{\isacharcolon}\ {\isachardoublequote}{\isacharquery}S\ {\isacharequal}\ f\ y{\isachardoublequote}\ \isamarkupfalse%
\isacommand{{\isachardot}{\isachardot}}\isanewline
\ \ \ \ \isamarkupfalse%
\isacommand{show}\ False\isanewline
\ \ \ \ \isamarkupfalse%
\isacommand{proof}\ cases\isanewline
\ \ \ \ \ \ \isamarkupfalse%
\isacommand{assume}\ {\isachardoublequote}y\ {\isasymin}\ {\isacharquery}S{\isachardoublequote}\isanewline
\ \ \ \ \ \ \isamarkupfalse%
\isacommand{hence}\ {\isachardoublequote}y\ {\isasymnotin}\ f\ y{\isachardoublequote}\ \ \ \isamarkupfalse%
\isacommand{by}\ simp\isanewline
\ \ \ \ \ \ \isamarkupfalse%
\isacommand{hence}\ {\isachardoublequote}y\ {\isasymnotin}\ {\isacharquery}S{\isachardoublequote}\ \ \ \ \isamarkupfalse%
\isacommand{by}{\isacharparenleft}simp\ add{\isacharcolon}fy{\isacharparenright}\isanewline
\ \ \ \ \ \ \isamarkupfalse%
\isacommand{thus}\ False\ \ \ \ \ \ \ \ \ \isamarkupfalse%
\isacommand{by}\ contradiction\isanewline
\ \ \ \ \isamarkupfalse%
\isacommand{next}\isanewline
\ \ \ \ \ \ \isamarkupfalse%
\isacommand{assume}\ {\isachardoublequote}y\ {\isasymnotin}\ {\isacharquery}S{\isachardoublequote}\isanewline
\ \ \ \ \ \ \isamarkupfalse%
\isacommand{hence}\ {\isachardoublequote}y\ {\isasymin}\ f\ y{\isachardoublequote}\ \ \ \isamarkupfalse%
\isacommand{by}\ simp\isanewline
\ \ \ \ \ \ \isamarkupfalse%
\isacommand{hence}\ {\isachardoublequote}y\ {\isasymin}\ {\isacharquery}S{\isachardoublequote}\ \ \ \ \isamarkupfalse%
\isacommand{by}{\isacharparenleft}simp\ add{\isacharcolon}fy{\isacharparenright}\isanewline
\ \ \ \ \ \ \isamarkupfalse%
\isacommand{thus}\ False\ \ \ \ \ \ \ \ \ \isamarkupfalse%
\isacommand{by}\ contradiction\isanewline
\ \ \ \ \isamarkupfalse%
\isacommand{qed}\isanewline
\ \ \isamarkupfalse%
\isacommand{qed}\isanewline
\isamarkupfalse%
\isacommand{qed}\isamarkupfalse%
%
\begin{isamarkuptext}%
\noindent Method \isa{contradiction} succeeds if both $P$ and
$\neg P$ are among the assumptions and the facts fed into that step, in any order.

As it happens, Cantor's theorem can be proved automatically by best-first
search. Depth-first search would diverge, but best-first search successfully
navigates through the large search space:%
\end{isamarkuptext}%
\isamarkuptrue%
\isacommand{theorem}\ {\isachardoublequote}{\isasymexists}S{\isachardot}\ S\ {\isasymnotin}\ range\ {\isacharparenleft}f\ {\isacharcolon}{\isacharcolon}\ {\isacharprime}a\ {\isasymRightarrow}\ {\isacharprime}a\ set{\isacharparenright}{\isachardoublequote}\isanewline
\isamarkupfalse%
\isacommand{by}\ best\isamarkupfalse%
%
\isamarkupsubsection{Raw proof blocks%
}
\isamarkuptrue%
%
\begin{isamarkuptext}%
Although we have shown how to employ powerful automatic methods like
\isa{blast} to achieve bigger proof steps, there may still be the
tendency to use the default introduction and elimination rules to
decompose goals and facts. This can lead to very tedious proofs:%
\end{isamarkuptext}%
\isamarkuptrue%
\isamarkupfalse%
\isacommand{lemma}\ {\isachardoublequote}{\isasymforall}x\ y{\isachardot}\ A\ x\ y\ {\isasymand}\ B\ x\ y\ {\isasymlongrightarrow}\ C\ x\ y{\isachardoublequote}\isanewline
\isamarkupfalse%
\isacommand{proof}\isanewline
\ \ \isamarkupfalse%
\isacommand{fix}\ x\ \isamarkupfalse%
\isacommand{show}\ {\isachardoublequote}{\isasymforall}y{\isachardot}\ A\ x\ y\ {\isasymand}\ B\ x\ y\ {\isasymlongrightarrow}\ C\ x\ y{\isachardoublequote}\isanewline
\ \ \isamarkupfalse%
\isacommand{proof}\isanewline
\ \ \ \ \isamarkupfalse%
\isacommand{fix}\ y\ \isamarkupfalse%
\isacommand{show}\ {\isachardoublequote}A\ x\ y\ {\isasymand}\ B\ x\ y\ {\isasymlongrightarrow}\ C\ x\ y{\isachardoublequote}\isanewline
\ \ \ \ \isamarkupfalse%
\isacommand{proof}\isanewline
\ \ \ \ \ \ \isamarkupfalse%
\isacommand{assume}\ {\isachardoublequote}A\ x\ y\ {\isasymand}\ B\ x\ y{\isachardoublequote}\isanewline
\ \ \ \ \ \ \isamarkupfalse%
\isacommand{show}\ {\isachardoublequote}C\ x\ y{\isachardoublequote}\ \isamarkupfalse%
\isacommand{sorry}\isanewline
\ \ \ \ \isamarkupfalse%
\isacommand{qed}\isanewline
\ \ \isamarkupfalse%
\isacommand{qed}\isanewline
\isamarkupfalse%
\isacommand{qed}\isamarkupfalse%
%
\begin{isamarkuptext}%
\noindent Since we are only interested in the decomposition and not the
actual proof, the latter has been replaced by
\isakeyword{sorry}. Command \isakeyword{sorry} proves anything but is
only allowed in quick and dirty mode, the default interactive mode. It
is very convenient for top down proof development.

Luckily we can avoid this step by step decomposition very easily:%
\end{isamarkuptext}%
\isamarkuptrue%
\isacommand{lemma}\ {\isachardoublequote}{\isasymforall}x\ y{\isachardot}\ A\ x\ y\ {\isasymand}\ B\ x\ y\ {\isasymlongrightarrow}\ C\ x\ y{\isachardoublequote}\isanewline
\isamarkupfalse%
\isacommand{proof}\ {\isacharminus}\isanewline
\ \ \isamarkupfalse%
\isacommand{have}\ {\isachardoublequote}{\isasymAnd}x\ y{\isachardot}\ {\isasymlbrakk}\ A\ x\ y{\isacharsemicolon}\ B\ x\ y\ {\isasymrbrakk}\ {\isasymLongrightarrow}\ C\ x\ y{\isachardoublequote}\isanewline
\ \ \isamarkupfalse%
\isacommand{proof}\ {\isacharminus}\isanewline
\ \ \ \ \isamarkupfalse%
\isacommand{fix}\ x\ y\ \isamarkupfalse%
\isacommand{assume}\ {\isachardoublequote}A\ x\ y{\isachardoublequote}\ {\isachardoublequote}B\ x\ y{\isachardoublequote}\isanewline
\ \ \ \ \isamarkupfalse%
\isacommand{show}\ {\isachardoublequote}C\ x\ y{\isachardoublequote}\ \isamarkupfalse%
\isacommand{sorry}\isanewline
\ \ \isamarkupfalse%
\isacommand{qed}\isanewline
\ \ \isamarkupfalse%
\isacommand{thus}\ {\isacharquery}thesis\ \isamarkupfalse%
\isacommand{by}\ blast\isanewline
\isamarkupfalse%
\isacommand{qed}\isamarkupfalse%
%
\begin{isamarkuptext}%
\noindent
This can be simplified further by \emph{raw proof blocks}, i.e.\
proofs enclosed in braces:%
\end{isamarkuptext}%
\isamarkuptrue%
\isacommand{lemma}\ {\isachardoublequote}{\isasymforall}x\ y{\isachardot}\ A\ x\ y\ {\isasymand}\ B\ x\ y\ {\isasymlongrightarrow}\ C\ x\ y{\isachardoublequote}\isanewline
\isamarkupfalse%
\isacommand{proof}\ {\isacharminus}\isanewline
\ \ \isamarkupfalse%
\isacommand{{\isacharbraceleft}}\ \isamarkupfalse%
\isacommand{fix}\ x\ y\ \isamarkupfalse%
\isacommand{assume}\ {\isachardoublequote}A\ x\ y{\isachardoublequote}\ {\isachardoublequote}B\ x\ y{\isachardoublequote}\isanewline
\ \ \ \ \isamarkupfalse%
\isacommand{have}\ {\isachardoublequote}C\ x\ y{\isachardoublequote}\ \isamarkupfalse%
\isacommand{sorry}\ \isamarkupfalse%
\isacommand{{\isacharbraceright}}\isanewline
\ \ \isamarkupfalse%
\isacommand{thus}\ {\isacharquery}thesis\ \isamarkupfalse%
\isacommand{by}\ blast\isanewline
\isamarkupfalse%
\isacommand{qed}\isamarkupfalse%
%
\begin{isamarkuptext}%
\noindent The result of the raw proof block is the same theorem
as above, namely \isa{{\isasymAnd}x\ y{\isachardot}\ {\isasymlbrakk}A\ x\ y{\isacharsemicolon}\ B\ x\ y{\isasymrbrakk}\ {\isasymLongrightarrow}\ C\ x\ y}.  Raw
proof blocks are like ordinary proofs except that they do not prove
some explicitly stated property but that the property emerges directly
out of the \isakeyword{fixe}s, \isakeyword{assume}s and
\isakeyword{have} in the block. Thus they again serve to avoid
duplication. Note that the conclusion of a raw proof block is stated with
\isakeyword{have} rather than \isakeyword{show} because it is not the
conclusion of some pending goal but some independent claim.

The general idea demonstrated in this subsection is very
important in Isar and distinguishes it from tactic-style proofs:
\begin{quote}\em
Do not manipulate the proof state into a particular form by applying
tactics but state the desired form explicitly and let the tactic verify
that from this form the original goal follows.
\end{quote}
This yields more readable and also more robust proofs.

\subsubsection{General case distinctions}

As an important application of raw proof blocks we show how to deal
with general case distinctions --- more specific kinds are treated in
\S\ref{sec:CaseDistinction}. Imagine that you would like to prove some
goal by distinguishing $n$ cases $P_1$, \dots, $P_n$. You show that
the $n$ cases are exhaustive (i.e.\ $P_1 \lor \dots \lor P_n$) and
that each case $P_i$ implies the goal. Taken together, this proves the
goal. The corresponding Isar proof pattern (for $n = 3$) is very handy:%
\end{isamarkuptext}%
\isamarkuptrue%
%
\renewcommand{\isamarkupcmt}[1]{#1}
\isamarkupfalse%
\isacommand{proof}\ {\isacharminus}\isanewline
\ \ \isamarkupfalse%
\isacommand{have}\ {\isachardoublequote}P\isactrlisub {\isadigit{1}}\ {\isasymor}\ P\isactrlisub {\isadigit{2}}\ {\isasymor}\ P\isactrlisub {\isadigit{3}}{\isachardoublequote}\ \isamarkupfalse%
\ %
\isamarkupcmt{\dots%
}
\isanewline
\ \ \isamarkupfalse%
\isacommand{moreover}\isanewline
\ \ \isamarkupfalse%
\isacommand{{\isacharbraceleft}}\ \isamarkupfalse%
\isacommand{assume}\ P\isactrlisub {\isadigit{1}}\isanewline
\ \ \ \ %
\isamarkupcmt{\dots%
}
\isanewline
\ \ \ \ \isamarkupfalse%
\isacommand{have}\ {\isacharquery}thesis\ \isamarkupfalse%
\ %
\isamarkupcmt{\dots%
}
\ \isamarkupfalse%
\isacommand{{\isacharbraceright}}\isanewline
\ \ \isamarkupfalse%
\isacommand{moreover}\isanewline
\ \ \isamarkupfalse%
\isacommand{{\isacharbraceleft}}\ \isamarkupfalse%
\isacommand{assume}\ P\isactrlisub {\isadigit{2}}\isanewline
\ \ \ \ %
\isamarkupcmt{\dots%
}
\isanewline
\ \ \ \ \isamarkupfalse%
\isacommand{have}\ {\isacharquery}thesis\ \isamarkupfalse%
\ %
\isamarkupcmt{\dots%
}
\ \isamarkupfalse%
\isacommand{{\isacharbraceright}}\isanewline
\ \ \isamarkupfalse%
\isacommand{moreover}\isanewline
\ \ \isamarkupfalse%
\isacommand{{\isacharbraceleft}}\ \isamarkupfalse%
\isacommand{assume}\ P\isactrlisub {\isadigit{3}}\isanewline
\ \ \ \ %
\isamarkupcmt{\dots%
}
\isanewline
\ \ \ \ \isamarkupfalse%
\isacommand{have}\ {\isacharquery}thesis\ \isamarkupfalse%
\ %
\isamarkupcmt{\dots%
}
\ \isamarkupfalse%
\isacommand{{\isacharbraceright}}\isanewline
\ \ \isamarkupfalse%
\isacommand{ultimately}\ \isamarkupfalse%
\isacommand{show}\ {\isacharquery}thesis\ \isamarkupfalse%
\isacommand{by}\ blast\isanewline
\isamarkupfalse%
\isacommand{qed}\isamarkupfalse%
%
\renewcommand{\isamarkupcmt}[1]{{\isastylecmt--- #1}}
%
\isamarkupsubsection{Further refinements%
}
\isamarkuptrue%
%
\begin{isamarkuptext}%
This subsection discusses some further tricks that can make
life easier although they are not essential.%
\end{isamarkuptext}%
\isamarkuptrue%
%
\isamarkupsubsubsection{\isakeyword{and}%
}
\isamarkuptrue%
%
\begin{isamarkuptext}%
Propositions (following \isakeyword{assume} etc) may but need not be
separated by \isakeyword{and}. This is not just for readability
(\isakeyword{from} \isa{A} \isakeyword{and} \isa{B} looks nicer than
\isakeyword{from} \isa{A} \isa{B}) but for structuring lists of propositions
into possibly named blocks. In
\begin{center}
\isakeyword{assume} \isa{A:} $A_1$ $A_2$ \isakeyword{and} \isa{B:} $A_3$
\isakeyword{and} $A_4$
\end{center}
label \isa{A} refers to the list of propositions $A_1$ $A_2$ and
label \isa{B} to $A_3$.%
\end{isamarkuptext}%
\isamarkuptrue%
%
\isamarkupsubsubsection{\isakeyword{note}%
}
\isamarkuptrue%
%
\begin{isamarkuptext}%
If you want to remember intermediate fact(s) that cannot be
named directly, use \isakeyword{note}. For example the result of raw
proof block can be named by following it with
\isakeyword{note}~\isa{some{\isacharunderscore}name\ {\isacharequal}\ this}.  As a side effect,
\isa{this} is set to the list of facts on the right-hand side. You
can also say \isa{note\ some{\isacharunderscore}fact}, which simply sets \isa{this},
i.e.\ recalls \isa{some{\isacharunderscore}fact}, e.g.\ in a \isakeyword{moreover} sequence.%
\end{isamarkuptext}%
\isamarkuptrue%
%
\isamarkupsubsubsection{\isakeyword{fixes}%
}
\isamarkuptrue%
%
\begin{isamarkuptext}%
Sometimes it is necessary to decorate a proposition with type
constraints, as in Cantor's theorem above. These type constraints tend
to make the theorem less readable. The situation can be improved a
little by combining the type constraint with an outer \isa{{\isasymAnd}}:%
\end{isamarkuptext}%
\isamarkuptrue%
\isacommand{theorem}\ {\isachardoublequote}{\isasymAnd}f\ {\isacharcolon}{\isacharcolon}\ {\isacharprime}a\ {\isasymRightarrow}\ {\isacharprime}a\ set{\isachardot}\ {\isasymexists}S{\isachardot}\ S\ {\isasymnotin}\ range\ f{\isachardoublequote}\isamarkupfalse%
\isamarkupfalse%
%
\begin{isamarkuptext}%
\noindent However, now \isa{f} is bound and we need a
\isakeyword{fix}~\isa{f} in the proof before we can refer to \isa{f}.
This is avoided by \isakeyword{fixes}:%
\end{isamarkuptext}%
\isamarkuptrue%
\isacommand{theorem}\ \isakeyword{fixes}\ f\ {\isacharcolon}{\isacharcolon}\ {\isachardoublequote}{\isacharprime}a\ {\isasymRightarrow}\ {\isacharprime}a\ set{\isachardoublequote}\ \isakeyword{shows}\ {\isachardoublequote}{\isasymexists}S{\isachardot}\ S\ {\isasymnotin}\ range\ f{\isachardoublequote}\isamarkupfalse%
\isamarkupfalse%
%
\begin{isamarkuptext}%
\noindent
Even better, \isakeyword{fixes} allows to introduce concrete syntax locally:%
\end{isamarkuptext}%
\isamarkuptrue%
\isacommand{lemma}\ comm{\isacharunderscore}mono{\isacharcolon}\isanewline
\ \ \isakeyword{fixes}\ r\ {\isacharcolon}{\isacharcolon}\ {\isachardoublequote}{\isacharprime}a\ {\isasymRightarrow}\ {\isacharprime}a\ {\isasymRightarrow}\ bool{\isachardoublequote}\ {\isacharparenleft}\isakeyword{infix}\ {\isachardoublequote}{\isachargreater}{\isachardoublequote}\ {\isadigit{6}}{\isadigit{0}}{\isacharparenright}\ \isakeyword{and}\isanewline
\ \ \ \ \ \ \ f\ {\isacharcolon}{\isacharcolon}\ {\isachardoublequote}{\isacharprime}a\ {\isasymRightarrow}\ {\isacharprime}a\ {\isasymRightarrow}\ {\isacharprime}a{\isachardoublequote}\ \ \ {\isacharparenleft}\isakeyword{infixl}\ {\isachardoublequote}{\isacharplus}{\isacharplus}{\isachardoublequote}\ {\isadigit{7}}{\isadigit{0}}{\isacharparenright}\isanewline
\ \ \isakeyword{assumes}\ comm{\isacharcolon}\ {\isachardoublequote}{\isasymAnd}x\ y{\isacharcolon}{\isacharcolon}{\isacharprime}a{\isachardot}\ x\ {\isacharplus}{\isacharplus}\ y\ {\isacharequal}\ y\ {\isacharplus}{\isacharplus}\ x{\isachardoublequote}\ \isakeyword{and}\isanewline
\ \ \ \ \ \ \ \ \ \ mono{\isacharcolon}\ {\isachardoublequote}{\isasymAnd}x\ y\ z{\isacharcolon}{\isacharcolon}{\isacharprime}a{\isachardot}\ x\ {\isachargreater}\ y\ {\isasymLongrightarrow}\ x\ {\isacharplus}{\isacharplus}\ z\ {\isachargreater}\ y\ {\isacharplus}{\isacharplus}\ z{\isachardoublequote}\isanewline
\ \ \isakeyword{shows}\ {\isachardoublequote}x\ {\isachargreater}\ y\ {\isasymLongrightarrow}\ z\ {\isacharplus}{\isacharplus}\ x\ {\isachargreater}\ z\ {\isacharplus}{\isacharplus}\ y{\isachardoublequote}\isanewline
\isamarkupfalse%
\isacommand{by}{\isacharparenleft}simp\ add{\isacharcolon}\ comm\ mono{\isacharparenright}\isamarkupfalse%
%
\begin{isamarkuptext}%
\noindent The concrete syntax is dropped at the end of the proof and the
theorem becomes \begin{isabelle}%
{\isasymlbrakk}{\isasymAnd}x\ y{\isachardot}\ {\isacharquery}f\ x\ y\ {\isacharequal}\ {\isacharquery}f\ y\ x{\isacharsemicolon}\isanewline
\isaindent{\ }{\isasymAnd}x\ y\ z{\isachardot}\ {\isacharquery}r\ x\ y\ {\isasymLongrightarrow}\ {\isacharquery}r\ {\isacharparenleft}{\isacharquery}f\ x\ z{\isacharparenright}\ {\isacharparenleft}{\isacharquery}f\ y\ z{\isacharparenright}{\isacharsemicolon}\ {\isacharquery}r\ {\isacharquery}x\ {\isacharquery}y{\isasymrbrakk}\isanewline
{\isasymLongrightarrow}\ {\isacharquery}r\ {\isacharparenleft}{\isacharquery}f\ {\isacharquery}z\ {\isacharquery}x{\isacharparenright}\ {\isacharparenleft}{\isacharquery}f\ {\isacharquery}z\ {\isacharquery}y{\isacharparenright}%
\end{isabelle}
\tweakskip%
\end{isamarkuptext}%
\isamarkuptrue%
%
\isamarkupsubsubsection{\isakeyword{obtain}%
}
\isamarkuptrue%
%
\begin{isamarkuptext}%
The \isakeyword{obtain} construct can introduce multiple
witnesses and propositions as in the following proof fragment:%
\end{isamarkuptext}%
\isamarkuptrue%
\isacommand{lemma}\ \isakeyword{assumes}\ A{\isacharcolon}\ {\isachardoublequote}{\isasymexists}x\ y{\isachardot}\ P\ x\ y\ {\isasymand}\ Q\ x\ y{\isachardoublequote}\ \isakeyword{shows}\ {\isachardoublequote}R{\isachardoublequote}\isanewline
\isamarkupfalse%
\isacommand{proof}\ {\isacharminus}\isanewline
\ \ \isamarkupfalse%
\isacommand{from}\ A\ \isamarkupfalse%
\isacommand{obtain}\ x\ y\ \isakeyword{where}\ P{\isacharcolon}\ {\isachardoublequote}P\ x\ y{\isachardoublequote}\ \isakeyword{and}\ Q{\isacharcolon}\ {\isachardoublequote}Q\ x\ y{\isachardoublequote}\ \ \isamarkupfalse%
\isacommand{by}\ blast\isamarkupfalse%
\isamarkupfalse%
%
\begin{isamarkuptext}%
Remember also that one does not even need to start with a formula
containing \isa{{\isasymexists}} as we saw in the proof of Cantor's theorem.%
\end{isamarkuptext}%
\isamarkuptrue%
%
\isamarkupsubsubsection{Combining proof styles%
}
\isamarkuptrue%
%
\begin{isamarkuptext}%
Finally, whole ``scripts'' (tactic-based proofs in the style of
\cite{LNCS2283}) may appear in the leaves of the proof tree, although this is
best avoided.  Here is a contrived example:%
\end{isamarkuptext}%
\isamarkuptrue%
\isacommand{lemma}\ {\isachardoublequote}A\ {\isasymlongrightarrow}\ {\isacharparenleft}A\ {\isasymlongrightarrow}\ B{\isacharparenright}\ {\isasymlongrightarrow}\ B{\isachardoublequote}\isanewline
\isamarkupfalse%
\isacommand{proof}\isanewline
\ \ \isamarkupfalse%
\isacommand{assume}\ a{\isacharcolon}\ {\isachardoublequote}A{\isachardoublequote}\isanewline
\ \ \isamarkupfalse%
\isacommand{show}\ {\isachardoublequote}{\isacharparenleft}A\ {\isasymlongrightarrow}B{\isacharparenright}\ {\isasymlongrightarrow}\ B{\isachardoublequote}\isanewline
\ \ \ \ \isamarkupfalse%
\isacommand{apply}{\isacharparenleft}rule\ impI{\isacharparenright}\isanewline
\ \ \ \ \isamarkupfalse%
\isacommand{apply}{\isacharparenleft}erule\ impE{\isacharparenright}\isanewline
\ \ \ \ \isamarkupfalse%
\isacommand{apply}{\isacharparenleft}rule\ a{\isacharparenright}\isanewline
\ \ \ \ \isamarkupfalse%
\isacommand{apply}\ assumption\isanewline
\ \ \ \ \isamarkupfalse%
\isacommand{done}\isanewline
\isamarkupfalse%
\isacommand{qed}\isamarkupfalse%
%
\begin{isamarkuptext}%
\noindent You may need to resort to this technique if an
automatic step fails to prove the desired proposition.

When converting a proof from tactic-style into Isar you can proceed
in a top-down manner: parts of the proof can be left in script form
while the outer structure is already expressed in Isar.%
\end{isamarkuptext}%
\isamarkuptrue%
\isamarkupfalse%
\end{isabellebody}%
%%% Local Variables:
%%% mode: latex
%%% TeX-master: "root"
%%% End:
