
\chapter{Introduction}

A Haskell-style type-system \cite{haskell-report} with ordered type-classes
has been present in Isabelle since 1991 \cite{nipkow-sorts93}.  Initially,
classes have mainly served as a \emph{purely syntactic} tool to formulate
polymorphic object-logics in a clean way, such as the standard Isabelle
formulation of many-sorted FOL \cite{paulson-isa-book}.

Applying classes at the \emph{logical level} to provide a simple notion of
abstract theories and instantiations to concrete ones, has been long proposed
as well \cite{nipkow-types93,nipkow-sorts93}).  At that time, Isabelle still
lacked built-in support for these \emph{axiomatic type classes}. More
importantly, their semantics was not yet fully fleshed out (and unnecessarily
complicated, too).

Since the Isabelle94 releases, actual axiomatic type classes have been an
integral part of Isabelle's meta-logic.  This very simple implementation is
based on a straight-forward extension of traditional simple-typed Higher-Order
Logic, including types qualified by logical predicates and overloaded constant
definitions; see \cite{Wenzel:1997:TPHOL} for further details.

Yet until Isabelle99, there used to be still a fundamental methodological
problem in using axiomatic type classes conveniently, due to the traditional
distinction of Isabelle theory files vs.\ ML proof scripts.  This has been
finally overcome with the advent of Isabelle/Isar theories
\cite{isabelle-isar-ref}: now definitions and proofs may be freely intermixed.
This nicely accommodates the usual procedure of defining axiomatic type
classes, proving abstract properties, defining operations on concrete types,
proving concrete properties for instantiation of classes etc.

\medskip

So to cut a long story short, the present version of axiomatic type classes
(Isabelle99 or later) now provides an even more useful and convenient
mechanism for light-weight abstract theories, without any special provisions
to be observed by the user.


\chapter{Examples}\label{sec:ex}

Axiomatic type classes are a concept of Isabelle's meta-logic
\cite{paulson-isa-book,Wenzel:1997:TPHOL}.  They may be applied to any
object-logic that directly uses the meta type system, such as Isabelle/HOL
\cite{isabelle-HOL}.  Subsequently, we present various examples that are all
formulated within HOL, except the one of \secref{sec:ex-natclass} which is in
FOL.

\section{Semigroups}

An axiomatic type class is simply a class of types that all meet certain
\emph{axioms}. Thus, type classes may be also understood as type predicates
--- i.e.\ abstractions over a single type argument $\alpha$.  Class axioms
typically contain polymorphic constants that depend on this type $\alpha$.
These \emph{characteristic constants} behave like operations associated with
the ``carrier'' type $\alpha$.

We illustrate these basic concepts by the following theory of semigroups.

\bigskip
\begin{isabelle}%
\isacommand{theory}~Semigroup~=~Main:\isanewline
\isanewline
\isacommand{consts}\isanewline
~~times~::~{"}'a~{\isasymRightarrow}~'a~{\isasymRightarrow}~'a{"}~~~~(\isakeyword{infixl}~{"}{\isasymOtimes}{"}~70)\isanewline
\isacommand{axclass}\isanewline
~~semigroup~<~{"}term{"}\isanewline
~~assoc:~{"}(x~{\isasymOtimes}~y)~{\isasymOtimes}~z~=~x~{\isasymOtimes}~(y~{\isasymOtimes}~z){"}\isanewline
\isanewline
\isacommand{end}\end{isabelle}%

\bigskip

\noindent
Above we have first declared a polymorphic constant $\TIMES :: \alpha \To
\alpha \To \alpha$ and then defined the class $semigroup$ of all types $\tau$
such that $\TIMES :: \tau \To \tau \To \tau$ is indeed an associative
operator.  The $assoc$ axiom contains exactly one type variable, which is
invisible in the above presentation, though.  Also note that free term
variables (like $x$, $y$, $z$) are allowed for user convenience ---
conceptually all of these are bound by outermost universal quantifiers.

\medskip

In general, type classes may be used to describe \emph{structures} with
exactly one carrier $\alpha$ and a fixed \emph{signature}.  Different
signatures require different classes. In the following theory, class
$plus_semigroup$ represents semigroups of the form $(\tau, \PLUS^\tau)$ while
$times_semigroup$ represents semigroups $(\tau, \TIMES^\tau)$.

\bigskip
\begin{isabelle}%
%
\isamarkupheader{Semigroups}
\isacommand{theory}\ Semigroups\ =\ Main:%
\begin{isamarkuptext}%
\medskip\noindent An axiomatic type class is simply a class of types
 that all meet certain properties, which are also called \emph{class
 axioms}. Thus, type classes may be also understood as type predicates
 --- i.e.\ abstractions over a single type argument $\alpha$.  Class
 axioms typically contain polymorphic constants that depend on this
 type $\alpha$.  These \emph{characteristic constants} behave like
 operations associated with the ``carrier'' type $\alpha$.

 We illustrate these basic concepts by the following formulation of
 semigroups.%
\end{isamarkuptext}%
\isacommand{consts}\isanewline
\ \ times\ ::\ {\isachardoublequote}{\isacharprime}a\ {\isasymRightarrow}\ {\isacharprime}a\ {\isasymRightarrow}\ {\isacharprime}a{\isachardoublequote}\ \ \ \ {\isacharparenleft}\isakeyword{infixl}\ {\isachardoublequote}{\isasymOtimes}{\isachardoublequote}\ 70{\isacharparenright}\isanewline
\isacommand{axclass}\isanewline
\ \ semigroup\ {\isacharless}\ {\isachardoublequote}term{\isachardoublequote}\isanewline
\ \ assoc:\ {\isachardoublequote}{\isacharparenleft}x\ {\isasymOtimes}\ y{\isacharparenright}\ {\isasymOtimes}\ z\ =\ x\ {\isasymOtimes}\ {\isacharparenleft}y\ {\isasymOtimes}\ z{\isacharparenright}{\isachardoublequote}%
\begin{isamarkuptext}%
\noindent Above we have first declared a polymorphic constant $\TIMES
 :: \alpha \To \alpha \To \alpha$ and then defined the class
 $semigroup$ of all types $\tau$ such that $\TIMES :: \tau \To \tau
 \To \tau$ is indeed an associative operator.  The $assoc$ axiom
 contains exactly one type variable, which is invisible in the above
 presentation, though.  Also note that free term variables (like $x$,
 $y$, $z$) are allowed for user convenience --- conceptually all of
 these are bound by outermost universal quantifiers.

 \medskip In general, type classes may be used to describe
 \emph{structures} with exactly one carrier $\alpha$ and a fixed
 \emph{signature}.  Different signatures require different classes.
 Below, class $plus_semigroup$ represents semigroups of the form
 $(\tau, \PLUS^\tau)$, while the original $semigroup$ would correspond
 to semigroups $(\tau, \TIMES^\tau)$.%
\end{isamarkuptext}%
\isacommand{consts}\isanewline
\ \ plus\ ::\ {\isachardoublequote}{\isacharprime}a\ {\isasymRightarrow}\ {\isacharprime}a\ {\isasymRightarrow}\ {\isacharprime}a{\isachardoublequote}\ \ \ \ {\isacharparenleft}\isakeyword{infixl}\ {\isachardoublequote}{\isasymOplus}{\isachardoublequote}\ 70{\isacharparenright}\isanewline
\isacommand{axclass}\isanewline
\ \ plus{\isacharunderscore}semigroup\ {\isacharless}\ {\isachardoublequote}term{\isachardoublequote}\isanewline
\ \ assoc:\ {\isachardoublequote}{\isacharparenleft}x\ {\isasymOplus}\ y{\isacharparenright}\ {\isasymOplus}\ z\ =\ x\ {\isasymOplus}\ {\isacharparenleft}y\ {\isasymOplus}\ z{\isacharparenright}{\isachardoublequote}%
\begin{isamarkuptext}%
\noindent Even if classes $plus_semigroup$ and $semigroup$ both
 represent semigroups in a sense, they are certainly not quite the
 same.%
\end{isamarkuptext}%
\isacommand{end}\end{isabelle}%
%%% Local Variables:
%%% mode: latex
%%% TeX-master: "root"
%%% End:

\bigskip

\noindent Even if classes $plus_semigroup$ and $times_semigroup$ both represent
semigroups in a sense, they are not the same!


\begin{isabelle}%
%
\isamarkupheader{Basic group theory}
\isacommand{theory}~Group~=~Main:%
\begin{isamarkuptext}%
\medskip\noindent The meta-type system of Isabelle supports
 \emph{intersections} and \emph{inclusions} of type classes. These
 directly correspond to intersections and inclusions of type
 predicates in a purely set theoretic sense. This is sufficient as a
 means to describe simple hierarchies of structures.  As an
 illustration, we use the well-known example of semigroups, monoids,
 general groups and Abelian groups.%
\end{isamarkuptext}%
%
\isamarkupsubsection{Monoids and Groups}
%
\begin{isamarkuptext}%
First we declare some polymorphic constants required later for the
 signature parts of our structures.%
\end{isamarkuptext}%
\isacommand{consts}\isanewline
~~times~::~{"}'a~=>~'a~=>~'a{"}~~~~(\isakeyword{infixl}~{"}{\isasymOtimes}{"}~70)\isanewline
~~inverse~::~{"}'a~=>~'a{"}~~~~~~~~({"}(\_{\isasyminv}){"}~[1000]~999)\isanewline
~~one~::~'a~~~~~~~~~~~~~~~~~~~~({"}{\isasymunit}{"})%
\begin{isamarkuptext}%
\noindent Next we define class $monoid$ of monoids with operations
 $\TIMES$ and $1$.  Note that multiple class axioms are allowed for
 user convenience --- they simply represent the conjunction of their
 respective universal closures.%
\end{isamarkuptext}%
\isacommand{axclass}\isanewline
~~monoid~<~{"}term{"}\isanewline
~~assoc:~~~~~~{"}(x~{\isasymOtimes}~y)~{\isasymOtimes}~z~=~x~{\isasymOtimes}~(y~{\isasymOtimes}~z){"}\isanewline
~~left\_unit:~~{"}{\isasymunit}~{\isasymOtimes}~x~=~x{"}\isanewline
~~right\_unit:~{"}x~{\isasymOtimes}~{\isasymunit}~=~x{"}%
\begin{isamarkuptext}%
\noindent So class $monoid$ contains exactly those types $\tau$ where
 $\TIMES :: \tau \To \tau \To \tau$ and $1 :: \tau$ are specified
 appropriately, such that $\TIMES$ is associative and $1$ is a left
 and right unit element for $\TIMES$.%
\end{isamarkuptext}%
%
\begin{isamarkuptext}%
\medskip Independently of $monoid$, we now define a linear hierarchy
 of semigroups, general groups and Abelian groups.  Note that the
 names of class axioms are automatically qualified with each class
 name, so we may re-use common names such as $assoc$.%
\end{isamarkuptext}%
\isacommand{axclass}\isanewline
~~semigroup~<~{"}term{"}\isanewline
~~assoc:~{"}(x~{\isasymOtimes}~y)~{\isasymOtimes}~z~=~x~{\isasymOtimes}~(y~{\isasymOtimes}~z){"}\isanewline
\isanewline
\isacommand{axclass}\isanewline
~~group~<~semigroup\isanewline
~~left\_unit:~~~~{"}{\isasymunit}~{\isasymOtimes}~x~=~x{"}\isanewline
~~left\_inverse:~{"}x{\isasyminv}~{\isasymOtimes}~x~=~{\isasymunit}{"}\isanewline
\isanewline
\isacommand{axclass}\isanewline
~~agroup~<~group\isanewline
~~commute:~{"}x~{\isasymOtimes}~y~=~y~{\isasymOtimes}~x{"}%
\begin{isamarkuptext}%
\noindent Class $group$ inherits associativity of $\TIMES$ from
 $semigroup$ and adds two further group axioms. Similarly, $agroup$
 is defined as the subset of $group$ such that for all of its elements
 $\tau$, the operation $\TIMES :: \tau \To \tau \To \tau$ is even
 commutative.%
\end{isamarkuptext}%
%
\isamarkupsubsection{Abstract reasoning}
%
\begin{isamarkuptext}%
In a sense, axiomatic type classes may be viewed as \emph{abstract
 theories}.  Above class definitions gives rise to abstract axioms
 $assoc$, $left_unit$, $left_inverse$, $commute$, where any of these
 contain a type variable $\alpha :: c$ that is restricted to types of
 the corresponding class $c$.  \emph{Sort constraints} like this
 express a logical precondition for the whole formula.  For example,
 $assoc$ states that for all $\tau$, provided that $\tau ::
 semigroup$, the operation $\TIMES :: \tau \To \tau \To \tau$ is
 associative.

 \medskip From a technical point of view, abstract axioms are just
 ordinary Isabelle theorems, which may be used in proofs without
 special treatment.  Such ``abstract proofs'' usually yield new
 ``abstract theorems''.  For example, we may now derive the following
 well-known laws of general groups.%
\end{isamarkuptext}%
\isacommand{theorem}~group\_right\_inverse:~{"}x~{\isasymOtimes}~x{\isasyminv}~=~({\isasymunit}{\isasymColon}'a{\isasymColon}group){"}\isanewline
\isacommand{proof}~-\isanewline
~~\isacommand{have}~{"}x~{\isasymOtimes}~x{\isasyminv}~=~{\isasymunit}~{\isasymOtimes}~(x~{\isasymOtimes}~x{\isasyminv}){"}\isanewline
~~~~\isacommand{by}~(simp~only:~group.left\_unit)\isanewline
~~\isacommand{also}~\isacommand{have}~{"}...~=~{\isasymunit}~{\isasymOtimes}~x~{\isasymOtimes}~x{\isasyminv}{"}\isanewline
~~~~\isacommand{by}~(simp~only:~semigroup.assoc)\isanewline
~~\isacommand{also}~\isacommand{have}~{"}...~=~(x{\isasyminv}){\isasyminv}~{\isasymOtimes}~x{\isasyminv}~{\isasymOtimes}~x~{\isasymOtimes}~x{\isasyminv}{"}\isanewline
~~~~\isacommand{by}~(simp~only:~group.left\_inverse)\isanewline
~~\isacommand{also}~\isacommand{have}~{"}...~=~(x{\isasyminv}){\isasyminv}~{\isasymOtimes}~(x{\isasyminv}~{\isasymOtimes}~x)~{\isasymOtimes}~x{\isasyminv}{"}\isanewline
~~~~\isacommand{by}~(simp~only:~semigroup.assoc)\isanewline
~~\isacommand{also}~\isacommand{have}~{"}...~=~(x{\isasyminv}){\isasyminv}~{\isasymOtimes}~{\isasymunit}~{\isasymOtimes}~x{\isasyminv}{"}\isanewline
~~~~\isacommand{by}~(simp~only:~group.left\_inverse)\isanewline
~~\isacommand{also}~\isacommand{have}~{"}...~=~(x{\isasyminv}){\isasyminv}~{\isasymOtimes}~({\isasymunit}~{\isasymOtimes}~x{\isasyminv}){"}\isanewline
~~~~\isacommand{by}~(simp~only:~semigroup.assoc)\isanewline
~~\isacommand{also}~\isacommand{have}~{"}...~=~(x{\isasyminv}){\isasyminv}~{\isasymOtimes}~x{\isasyminv}{"}\isanewline
~~~~\isacommand{by}~(simp~only:~group.left\_unit)\isanewline
~~\isacommand{also}~\isacommand{have}~{"}...~=~{\isasymunit}{"}\isanewline
~~~~\isacommand{by}~(simp~only:~group.left\_inverse)\isanewline
~~\isacommand{finally}~\isacommand{show}~?thesis~\isacommand{.}\isanewline
\isacommand{qed}%
\begin{isamarkuptext}%
\noindent With $group_right_inverse$ already available,
 $group_right_unit$\label{thm:group-right-unit} is now established
 much easier.%
\end{isamarkuptext}%
\isacommand{theorem}~group\_right\_unit:~{"}x~{\isasymOtimes}~{\isasymunit}~=~(x{\isasymColon}'a{\isasymColon}group){"}\isanewline
\isacommand{proof}~-\isanewline
~~\isacommand{have}~{"}x~{\isasymOtimes}~{\isasymunit}~=~x~{\isasymOtimes}~(x{\isasyminv}~{\isasymOtimes}~x){"}\isanewline
~~~~\isacommand{by}~(simp~only:~group.left\_inverse)\isanewline
~~\isacommand{also}~\isacommand{have}~{"}...~=~x~{\isasymOtimes}~x{\isasyminv}~{\isasymOtimes}~x{"}\isanewline
~~~~\isacommand{by}~(simp~only:~semigroup.assoc)\isanewline
~~\isacommand{also}~\isacommand{have}~{"}...~=~{\isasymunit}~{\isasymOtimes}~x{"}\isanewline
~~~~\isacommand{by}~(simp~only:~group\_right\_inverse)\isanewline
~~\isacommand{also}~\isacommand{have}~{"}...~=~x{"}\isanewline
~~~~\isacommand{by}~(simp~only:~group.left\_unit)\isanewline
~~\isacommand{finally}~\isacommand{show}~?thesis~\isacommand{.}\isanewline
\isacommand{qed}%
\begin{isamarkuptext}%
\medskip Abstract theorems may be instantiated to only those types
 $\tau$ where the appropriate class membership $\tau :: c$ is known at
 Isabelle's type signature level.  Since we have $agroup \subseteq
 group \subseteq semigroup$ by definition, all theorems of $semigroup$
 and $group$ are automatically inherited by $group$ and $agroup$.%
\end{isamarkuptext}%
%
\isamarkupsubsection{Abstract instantiation}
%
\begin{isamarkuptext}%
From the definition, the $monoid$ and $group$ classes have been
 independent.  Note that for monoids, $right_unit$ had to be included
 as an axiom, but for groups both $right_unit$ and $right_inverse$ are
 derivable from the other axioms.  With $group_right_unit$ derived as
 a theorem of group theory (see page~\pageref{thm:group-right-unit}),
 we may now instantiate $monoid \subseteq semigroup$ and $group
 \subseteq monoid$ properly as follows
 (cf.\ \figref{fig:monoid-group}).

 \begin{figure}[htbp]
   \begin{center}
     \small
     \unitlength 0.6mm
     \begin{picture}(65,90)(0,-10)
       \put(15,10){\line(0,1){10}} \put(15,30){\line(0,1){10}}
       \put(15,50){\line(1,1){10}} \put(35,60){\line(1,-1){10}}
       \put(15,5){\makebox(0,0){$agroup$}}
       \put(15,25){\makebox(0,0){$group$}}
       \put(15,45){\makebox(0,0){$semigroup$}}
       \put(30,65){\makebox(0,0){$term$}} \put(50,45){\makebox(0,0){$monoid$}}
     \end{picture}
     \hspace{4em}
     \begin{picture}(30,90)(0,0)
       \put(15,10){\line(0,1){10}} \put(15,30){\line(0,1){10}}
       \put(15,50){\line(0,1){10}} \put(15,70){\line(0,1){10}}
       \put(15,5){\makebox(0,0){$agroup$}}
       \put(15,25){\makebox(0,0){$group$}}
       \put(15,45){\makebox(0,0){$monoid$}}
       \put(15,65){\makebox(0,0){$semigroup$}}
       \put(15,85){\makebox(0,0){$term$}}
     \end{picture}
     \caption{Monoids and groups: according to definition, and by proof}
     \label{fig:monoid-group}
   \end{center}
 \end{figure}%
\end{isamarkuptext}%
\isacommand{instance}~monoid~<~semigroup\isanewline
\isacommand{proof}~intro\_classes\isanewline
~~\isacommand{fix}~x~y~z~::~{"}'a{\isasymColon}monoid{"}\isanewline
~~\isacommand{show}~{"}x~{\isasymOtimes}~y~{\isasymOtimes}~z~=~x~{\isasymOtimes}~(y~{\isasymOtimes}~z){"}\isanewline
~~~~\isacommand{by}~(rule~monoid.assoc)\isanewline
\isacommand{qed}\isanewline
\isanewline
\isacommand{instance}~group~<~monoid\isanewline
\isacommand{proof}~intro\_classes\isanewline
~~\isacommand{fix}~x~y~z~::~{"}'a{\isasymColon}group{"}\isanewline
~~\isacommand{show}~{"}x~{\isasymOtimes}~y~{\isasymOtimes}~z~=~x~{\isasymOtimes}~(y~{\isasymOtimes}~z){"}\isanewline
~~~~\isacommand{by}~(rule~semigroup.assoc)\isanewline
~~\isacommand{show}~{"}{\isasymunit}~{\isasymOtimes}~x~=~x{"}\isanewline
~~~~\isacommand{by}~(rule~group.left\_unit)\isanewline
~~\isacommand{show}~{"}x~{\isasymOtimes}~{\isasymunit}~=~x{"}\isanewline
~~~~\isacommand{by}~(rule~group\_right\_unit)\isanewline
\isacommand{qed}%
\begin{isamarkuptext}%
\medskip The $\isakeyword{instance}$ command sets up an appropriate
 goal that represents the class inclusion (or type arity, see
 \secref{sec:inst-arity}) to be proven
 (see also \cite{isabelle-isar-ref}).  The $intro_classes$ proof
 method does back-chaining of class membership statements wrt.\ the
 hierarchy of any classes defined in the current theory; the effect is
 to reduce to the initial statement to a number of goals that directly
 correspond to any class axioms encountered on the path upwards
 through the class hierarchy.%
\end{isamarkuptext}%
%
\isamarkupsubsection{Concrete instantiation \label{sec:inst-arity}}
%
\begin{isamarkuptext}%
So far we have covered the case of the form
 $\isakeyword{instance}~c@1 < c@2$, namely \emph{abstract
 instantiation} --- $c@1$ is more special than $c@2$ and thus an
 instance of $c@2$.  Even more interesting for practical applications
 are \emph{concrete instantiations} of axiomatic type classes.  That
 is, certain simple schemes $(\alpha@1, \ldots, \alpha@n)t :: c$ of
 class membership may be established at the logical level and then
 transferred to Isabelle's type signature level.

 \medskip As a typical example, we show that type $bool$ with
 exclusive-or as operation $\TIMES$, identity as $\isasyminv$, and
 $False$ as $1$ forms an Abelian group.%
\end{isamarkuptext}%
\isacommand{defs}~(\isakeyword{overloaded})\isanewline
~~times\_bool\_def:~~~{"}x~{\isasymOtimes}~y~{\isasymequiv}~x~{\isasymnoteq}~(y{\isasymColon}bool){"}\isanewline
~~inverse\_bool\_def:~{"}x{\isasyminv}~{\isasymequiv}~x{\isasymColon}bool{"}\isanewline
~~unit\_bool\_def:~~~~{"}{\isasymunit}~{\isasymequiv}~False{"}%
\begin{isamarkuptext}%
\medskip It is important to note that above $\DEFS$ are just
 overloaded meta-level constant definitions, where type classes are
 not yet involved at all.  This form of constant definition with
 overloading (and optional recursion over the syntactic structure of
 simple types) are admissible as definitional extensions of plain HOL
 \cite{Wenzel:1997:TPHOL}.  The Haskell-style type system is not
 required for overloading.  Nevertheless, overloaded definitions are
 best applied in the context of type classes.

 \medskip Since we have chosen above $\DEFS$ of the generic group
 operations on type $bool$ appropriately, the class membership $bool
 :: agroup$ may be now derived as follows.%
\end{isamarkuptext}%
\isacommand{instance}~bool~::~agroup\isanewline
\isacommand{proof}~(intro\_classes,\isanewline
~~~~unfold~times\_bool\_def~inverse\_bool\_def~unit\_bool\_def)\isanewline
~~\isacommand{fix}~x~y~z\isanewline
~~\isacommand{show}~{"}((x~{\isasymnoteq}~y)~{\isasymnoteq}~z)~=~(x~{\isasymnoteq}~(y~{\isasymnoteq}~z)){"}~\isacommand{by}~blast\isanewline
~~\isacommand{show}~{"}(False~{\isasymnoteq}~x)~=~x{"}~\isacommand{by}~blast\isanewline
~~\isacommand{show}~{"}(x~{\isasymnoteq}~x)~=~False{"}~\isacommand{by}~blast\isanewline
~~\isacommand{show}~{"}(x~{\isasymnoteq}~y)~=~(y~{\isasymnoteq}~x){"}~\isacommand{by}~blast\isanewline
\isacommand{qed}%
\begin{isamarkuptext}%
The result of an $\isakeyword{instance}$ statement is both expressed
 as a theorem of Isabelle's meta-logic, and as a type arity of the
 type signature.  The latter enables type-inference system to take
 care of this new instance automatically.

 \medskip We could now also instantiate our group theory classes to
 many other concrete types.  For example, $int :: agroup$ (e.g.\ by
 defining $\TIMES$ as addition, $\isasyminv$ as negation and $1$ as
 zero) or $list :: (term)semigroup$ (e.g.\ if $\TIMES$ is defined as
 list append).  Thus, the characteristic constants $\TIMES$,
 $\isasyminv$, $1$ really become overloaded, i.e.\ have different
 meanings on different types.%
\end{isamarkuptext}%
%
\isamarkupsubsection{Lifting and Functors}
%
\begin{isamarkuptext}%
As already mentioned above, overloading in the simply-typed HOL
 systems may include recursion over the syntactic structure of types.
 That is, definitional equations $c^\tau \equiv t$ may also contain
 constants of name $c$ on the right-hand side --- if these have types
 that are structurally simpler than $\tau$.

 This feature enables us to \emph{lift operations}, say to Cartesian
 products, direct sums or function spaces.  Subsequently we lift
 $\TIMES$ component-wise to binary products $\alpha \times \beta$.%
\end{isamarkuptext}%
\isacommand{defs}~(\isakeyword{overloaded})\isanewline
~~times\_prod\_def:~{"}p~{\isasymOtimes}~q~{\isasymequiv}~(fst~p~{\isasymOtimes}~fst~q,~snd~p~{\isasymOtimes}~snd~q){"}%
\begin{isamarkuptext}%
It is very easy to see that associativity of $\TIMES^\alpha$ and
 $\TIMES^\beta$ transfers to ${\TIMES}^{\alpha \times \beta}$.  Hence
 the binary type constructor $\times$ maps semigroups to semigroups.
 This may be established formally as follows.%
\end{isamarkuptext}%
\isacommand{instance}~*~::~(semigroup,~semigroup)~semigroup\isanewline
\isacommand{proof}~(intro\_classes,~unfold~times\_prod\_def)\isanewline
~~\isacommand{fix}~p~q~r~::~{"}'a{\isasymColon}semigroup~{\isasymtimes}~'b{\isasymColon}semigroup{"}\isanewline
~~\isacommand{show}\isanewline
~~~~{"}(fst~(fst~p~{\isasymOtimes}~fst~q,~snd~p~{\isasymOtimes}~snd~q)~{\isasymOtimes}~fst~r,\isanewline
~~~~~~snd~(fst~p~{\isasymOtimes}~fst~q,~snd~p~{\isasymOtimes}~snd~q)~{\isasymOtimes}~snd~r)~=\isanewline
~~~~~~~(fst~p~{\isasymOtimes}~fst~(fst~q~{\isasymOtimes}~fst~r,~snd~q~{\isasymOtimes}~snd~r),\isanewline
~~~~~~~~snd~p~{\isasymOtimes}~snd~(fst~q~{\isasymOtimes}~fst~r,~snd~q~{\isasymOtimes}~snd~r)){"}\isanewline
~~~~\isacommand{by}~(simp~add:~semigroup.assoc)\isanewline
\isacommand{qed}%
\begin{isamarkuptext}%
Thus, if we view class instances as ``structures'', then overloaded
 constant definitions with recursion over types indirectly provide
 some kind of ``functors'' --- i.e.\ mappings between abstract
 theories.%
\end{isamarkuptext}%
\isacommand{end}\end{isabelle}%
%%% Local Variables:
%%% mode: latex
%%% TeX-master: "root"
%%% End:


\begin{isabelle}%
%
\isamarkupheader{Syntactic classes}
\isacommand{theory}~Product~=~Main:%
\begin{isamarkuptext}%
\medskip\noindent There is still a feature of Isabelle's type system
 left that we have not yet discussed.  When declaring polymorphic
 constants $c :: \sigma$, the type variables occurring in $\sigma$ may
 be constrained by type classes (or even general sorts) in an
 arbitrary way.  Note that by default, in Isabelle/HOL the declaration
 $\TIMES :: \alpha \To \alpha \To \alpha$ is actually an abbreviation
 for $\TIMES :: (\alpha::term) \To \alpha \To \alpha$.  Since class
 $term$ is the universal class of HOL, this is not really a constraint
 at all.

 The $product$ class below provides a less degenerate example of
 syntactic type classes.%
\end{isamarkuptext}%
\isacommand{axclass}\isanewline
~~product~<~{"}term{"}\isanewline
\isacommand{consts}\isanewline
~~product~::~{"}'a::product~{\isasymRightarrow}~'a~{\isasymRightarrow}~'a{"}~~~~(\isakeyword{infixl}~{"}{\isasymOtimes}{"}~70)%
\begin{isamarkuptext}%
Here class $product$ is defined as subclass of $term$ without any
 additional axioms.  This effects in logical equivalence of $product$
 and $term$, as is reflected by the trivial introduction rule
 generated for this definition.

 \medskip So what is the difference of declaring $\TIMES :: (\alpha ::
 product) \To \alpha \To \alpha$ vs.\ declaring $\TIMES :: (\alpha ::
 term) \To \alpha \To \alpha$ anyway?  In this particular case where
 $product \equiv term$, it should be obvious that both declarations
 are the same from the logic's point of view.  It even makes the most
 sense to remove sort constraints from constant declarations, as far
 as the purely logical meaning is concerned \cite{Wenzel:1997:TPHOL}.

 On the other hand there are syntactic differences, of course.
 Constants $\TIMES^\tau$ are rejected by the type-checker, unless the
 arity $\tau :: product$ is part of the type signature.  In our
 example, this arity may be always added when required by means of an
 $\isarkeyword{instance}$ with the trivial proof $\BY{intro_classes}$.

 \medskip Thus, we may observe the following discipline of using
 syntactic classes.  Overloaded polymorphic constants have their type
 arguments restricted to an associated (logically trivial) class $c$.
 Only immediately before \emph{specifying} these constants on a
 certain type $\tau$ do we instantiate $\tau :: c$.

 This is done for class $product$ and type $bool$ as follows.%
\end{isamarkuptext}%
\isacommand{instance}~bool~::~product\isanewline
~~\isacommand{by}~intro\_classes\isanewline
\isacommand{defs}\isanewline
~~product\_bool\_def:~{"}x~{\isasymOtimes}~y~{\isasymequiv}~x~{\isasymand}~y{"}%
\begin{isamarkuptext}%
The definition $prod_bool_def$ becomes syntactically well-formed only
 after the arity $bool :: product$ is made known to the type checker.

 \medskip It is very important to see that above $\DEFS$ are not
 directly connected with $\isarkeyword{instance}$ at all!  We were
 just following our convention to specify $\TIMES$ on $bool$ after
 having instantiated $bool :: product$.  Isabelle does not require
 these definitions, which is in contrast to programming languages like
 Haskell \cite{haskell-report}.

 \medskip While Isabelle type classes and those of Haskell are almost
 the same as far as type-checking and type inference are concerned,
 there are important semantic differences.  Haskell classes require
 their instances to \emph{provide operations} of certain \emph{names}.
 Therefore, its \texttt{instance} has a \texttt{where} part that tells
 the system what these ``member functions'' should be.

 This style of \texttt{instance} won't make much sense in Isabelle's
 meta-logic, because there is no internal notion of ``providing
 operations'' or even ``names of functions''.%
\end{isamarkuptext}%
\isacommand{end}\end{isabelle}%



\section{Defining natural numbers in FOL}\label{sec:ex-natclass}

Axiomatic type classes abstract over exactly one type argument. Thus, any
\emph{axiomatic} theory extension where each axiom refers to at most one type
variable, may be trivially turned into a \emph{definitional} one.

We illustrate this with the natural numbers in Isabelle/FOL.

\bigskip
%
\begin{isabellebody}%
\def\isabellecontext{NatClass}%
\isamarkuptrue%
%
\isamarkupheader{Defining natural numbers in FOL \label{sec:ex-natclass}%
}
%
\isadelimtheory
%
\endisadelimtheory
%
\isatagtheory
\isamarkupfalse%
\isacommand{theory}\ NatClass\ \isakeyword{imports}\ FOL\ \isakeyword{begin}%
\endisatagtheory
{\isafoldtheory}%
%
\isadelimtheory
%
\endisadelimtheory
\isamarkuptrue%
%
\begin{isamarkuptext}%
\medskip\noindent Axiomatic type classes abstract over exactly one
 type argument. Thus, any \emph{axiomatic} theory extension where each
 axiom refers to at most one type variable, may be trivially turned
 into a \emph{definitional} one.

 We illustrate this with the natural numbers in
 Isabelle/FOL.\footnote{See also
 \url{http://isabelle.in.tum.de/library/FOL/ex/NatClass.html}}%
\end{isamarkuptext}%
\isamarkupfalse%
\isacommand{consts}\isanewline
\ \ zero\ {\isacharcolon}{\isacharcolon}\ {\isacharprime}a\ \ \ \ {\isacharparenleft}{\isachardoublequote}{\isasymzero}{\isachardoublequote}{\isacharparenright}\isanewline
\ \ Suc\ {\isacharcolon}{\isacharcolon}\ {\isachardoublequote}{\isacharprime}a\ {\isasymRightarrow}\ {\isacharprime}a{\isachardoublequote}\isanewline
\ \ rec\ {\isacharcolon}{\isacharcolon}\ {\isachardoublequote}{\isacharprime}a\ {\isasymRightarrow}\ {\isacharprime}a\ {\isasymRightarrow}\ {\isacharparenleft}{\isacharprime}a\ {\isasymRightarrow}\ {\isacharprime}a\ {\isasymRightarrow}\ {\isacharprime}a{\isacharparenright}\ {\isasymRightarrow}\ {\isacharprime}a{\isachardoublequote}\isanewline
\isanewline
\isamarkupfalse%
\isacommand{axclass}\ nat\ {\isasymsubseteq}\ {\isachardoublequote}term{\isachardoublequote}\isanewline
\ \ induct{\isacharcolon}\ {\isachardoublequote}P{\isacharparenleft}{\isasymzero}{\isacharparenright}\ {\isasymLongrightarrow}\ {\isacharparenleft}{\isasymAnd}x{\isachardot}\ P{\isacharparenleft}x{\isacharparenright}\ {\isasymLongrightarrow}\ P{\isacharparenleft}Suc{\isacharparenleft}x{\isacharparenright}{\isacharparenright}{\isacharparenright}\ {\isasymLongrightarrow}\ P{\isacharparenleft}n{\isacharparenright}{\isachardoublequote}\isanewline
\ \ Suc{\isacharunderscore}inject{\isacharcolon}\ {\isachardoublequote}Suc{\isacharparenleft}m{\isacharparenright}\ {\isacharequal}\ Suc{\isacharparenleft}n{\isacharparenright}\ {\isasymLongrightarrow}\ m\ {\isacharequal}\ n{\isachardoublequote}\isanewline
\ \ Suc{\isacharunderscore}neq{\isacharunderscore}{\isadigit{0}}{\isacharcolon}\ {\isachardoublequote}Suc{\isacharparenleft}m{\isacharparenright}\ {\isacharequal}\ {\isasymzero}\ {\isasymLongrightarrow}\ R{\isachardoublequote}\isanewline
\ \ rec{\isacharunderscore}{\isadigit{0}}{\isacharcolon}\ {\isachardoublequote}rec{\isacharparenleft}{\isasymzero}{\isacharcomma}\ a{\isacharcomma}\ f{\isacharparenright}\ {\isacharequal}\ a{\isachardoublequote}\isanewline
\ \ rec{\isacharunderscore}Suc{\isacharcolon}\ {\isachardoublequote}rec{\isacharparenleft}Suc{\isacharparenleft}m{\isacharparenright}{\isacharcomma}\ a{\isacharcomma}\ f{\isacharparenright}\ {\isacharequal}\ f{\isacharparenleft}m{\isacharcomma}\ rec{\isacharparenleft}m{\isacharcomma}\ a{\isacharcomma}\ f{\isacharparenright}{\isacharparenright}{\isachardoublequote}\isanewline
\isanewline
\isamarkupfalse%
\isacommand{constdefs}\isanewline
\ \ add\ {\isacharcolon}{\isacharcolon}\ {\isachardoublequote}{\isacharprime}a{\isacharcolon}{\isacharcolon}nat\ {\isasymRightarrow}\ {\isacharprime}a\ {\isasymRightarrow}\ {\isacharprime}a{\isachardoublequote}\ \ \ \ {\isacharparenleft}\isakeyword{infixl}\ {\isachardoublequote}{\isacharplus}{\isachardoublequote}\ {\isadigit{6}}{\isadigit{0}}{\isacharparenright}\isanewline
\ \ {\isachardoublequote}m\ {\isacharplus}\ n\ {\isasymequiv}\ rec{\isacharparenleft}m{\isacharcomma}\ n{\isacharcomma}\ {\isasymlambda}x\ y{\isachardot}\ Suc{\isacharparenleft}y{\isacharparenright}{\isacharparenright}{\isachardoublequote}\isamarkuptrue%
%
\begin{isamarkuptext}%
This is an abstract version of the plain \isa{Nat} theory in
 FOL.\footnote{See
 \url{http://isabelle.in.tum.de/library/FOL/ex/Nat.html}} Basically,
 we have just replaced all occurrences of type \isa{nat} by \isa{{\isacharprime}a} and used the natural number axioms to define class \isa{nat}.
 There is only a minor snag, that the original recursion operator
 \isa{rec} had to be made monomorphic.

 Thus class \isa{nat} contains exactly those types \isa{{\isasymtau}} that
 are isomorphic to ``the'' natural numbers (with signature \isa{{\isasymzero}}, \isa{Suc}, \isa{rec}).

 \medskip What we have done here can be also viewed as \emph{type
 specification}.  Of course, it still remains open if there is some
 type at all that meets the class axioms.  Now a very nice property of
 axiomatic type classes is that abstract reasoning is always possible
 --- independent of satisfiability.  The meta-logic won't break, even
 if some classes (or general sorts) turns out to be empty later ---
 ``inconsistent'' class definitions may be useless, but do not cause
 any harm.

 Theorems of the abstract natural numbers may be derived in the same
 way as for the concrete version.  The original proof scripts may be
 re-used with some trivial changes only (mostly adding some type
 constraints).%
\end{isamarkuptext}%
%
\isadelimtheory
%
\endisadelimtheory
%
\isatagtheory
\isamarkupfalse%
\isacommand{end}%
\endisatagtheory
{\isafoldtheory}%
%
\isadelimtheory
%
\endisadelimtheory
\end{isabellebody}%
%%% Local Variables:
%%% mode: latex
%%% TeX-master: "root"
%%% End:

\bigskip

This is an abstract version of the plain $Nat$ theory in
FOL.\footnote{See \url{http://isabelle.in.tum.de/library/FOL/ex/Nat.html}}

Basically, we have just replaced all occurrences of type $nat$ by $\alpha$ and
used the natural number axioms to define class $nat$.  There is only a minor
snag, that the original recursion operator $rec$ had to be made monomorphic,
in a sense.  Thus class $nat$ contains exactly those types $\tau$ that are
isomorphic to ``the'' natural numbers (with signature $0$, $Suc$, $rec$).

\medskip

What we have done here can be also viewed as \emph{type specification}.  Of
course, it still remains open if there is some type at all that meets the
class axioms.  Now a very nice property of axiomatic type classes is, that
abstract reasoning is always possible --- independent of satisfiability.  The
meta-logic won't break, even if some classes (or general sorts) turns out to
be empty (``inconsistent'') later.

Theorems of the abstract natural numbers may be derived in the same way as for
the concrete version.  The original proof scripts may be used with some
trivial changes only (mostly adding some type constraints).


%% FIXME move some parts to ref or isar-ref manual (!?);

% \chapter{The user interface of Isabelle's axclass package}

% The actual axiomatic type class package of Isabelle/Pure mainly consists
% of two new theory sections: \texttt{axclass} and \texttt{instance}.  Some
% typical applications of these have already been demonstrated in
% \secref{sec:ex}, below their syntax and semantics are presented more
% completely.


% \section{The axclass section}

% Within theory files, \texttt{axclass} introduces an axiomatic type class
% definition. Its concrete syntax is:

% \begin{matharray}{l}
%   \texttt{axclass} \\
%   \ \ c \texttt{ < } c@1\texttt, \ldots\texttt, c@n \\
%   \ \ id@1\ axm@1 \\
%   \ \ \vdots \\
%   \ \ id@m\ axm@m
% \emphnd{matharray}

% Where $c, c@1, \ldots, c@n$ are classes (category $id$ or
% $string$) and $axm@1, \ldots, axm@m$ (with $m \ge
% 0$) are formulas (category $string$).

% Class $c$ has to be new, and sort $\{c@1, \ldots, c@n\}$ a subsort of
% \texttt{logic}. Each class axiom $axm@j$ may contain any term
% variables, but at most one type variable (which need not be the same
% for all axioms). The sort of this type variable has to be a supersort
% of $\{c@1, \ldots, c@n\}$.

% \medskip

% The \texttt{axclass} section declares $c$ as subclass of $c@1, \ldots,
% c@n$ to the type signature.

% Furthermore, $axm@1, \ldots, axm@m$ are turned into the
% ``abstract axioms'' of $c$ with names $id@1, \ldots,
% id@m$.  This is done by replacing all occurring type variables
% by $\alpha :: c$. Original axioms that do not contain any type
% variable will be prefixed by the logical precondition
% $\texttt{OFCLASS}(\alpha :: \texttt{logic}, c\texttt{_class})$.

% Another axiom of name $c\texttt{I}$ --- the ``class $c$ introduction
% rule'' --- is built from the respective universal closures of
% $axm@1, \ldots, axm@m$ appropriately.


% \section{The instance section}

% Section \texttt{instance} proves class inclusions or type arities at the
% logical level and then transfers these into the type signature.

% Its concrete syntax is:

% \begin{matharray}{l}
%   \texttt{instance} \\
%   \ \ [\ c@1 \texttt{ < } c@2 \ |\
%       t \texttt{ ::\ (}sort@1\texttt, \ldots \texttt, sort@n\texttt) sort\ ] \\
%   \ \ [\ \texttt(name@1 \texttt, \ldots\texttt, name@m\texttt)\ ] \\
%   \ \ [\ \texttt{\{|} text \texttt{|\}}\ ]
% \emphnd{matharray}

% Where $c@1, c@2$ are classes and $t$ is an $n$-place type constructor
% (all of category $id$ or $string)$. Furthermore,
% $sort@i$ are sorts in the usual Isabelle-syntax.

% \medskip

% Internally, \texttt{instance} first sets up an appropriate goal that
% expresses the class inclusion or type arity as a meta-proposition.
% Then tactic \texttt{AxClass.axclass_tac} is applied with all preceding
% meta-definitions of the current theory file and the user-supplied
% witnesses. The latter are $name@1, \ldots, name@m$, where
% $id$ refers to an \ML-name of a theorem, and $string$ to an
% axiom of the current theory node\footnote{Thus, the user may reference
%   axioms from above this \texttt{instance} in the theory file. Note
%   that new axioms appear at the \ML-toplevel only after the file is
%   processed completely.}.

% Tactic \texttt{AxClass.axclass_tac} first unfolds the class definition by
% resolving with rule $c\texttt\texttt{I}$, and then applies the witnesses
% according to their form: Meta-definitions are unfolded, all other
% formulas are repeatedly resolved\footnote{This is done in a way that
%   enables proper object-\emph{rules} to be used as witnesses for
%   corresponding class axioms.} with.

% The final optional argument $text$ is \ML-code of an arbitrary
% user tactic which is applied last to any remaining goals.

% \medskip

% Because of the complexity of \texttt{instance}'s witnessing mechanisms,
% new users of the axclass package are advised to only use the simple
% form $\texttt{instance}\ \ldots\ (id@1, \ldots, id@!m)$, where
% the identifiers refer to theorems that are appropriate type instances
% of the class axioms. This typically requires an auxiliary theory,
% though, which defines some constants and then proves these witnesses.


%%% Local Variables: 
%%% mode: latex
%%% TeX-master: "axclass"
%%% End: 
% LocalWords:  Isabelle FOL
