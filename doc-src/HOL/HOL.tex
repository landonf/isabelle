\chapter{Higher-Order Logic}
\index{higher-order logic|(}
\index{HOL system@{\sc hol} system}

The theory~\thydx{HOL} implements higher-order logic.  It is based on
Gordon's~{\sc hol} system~\cite{mgordon-hol}, which itself is based on
Church's original paper~\cite{church40}.  Andrews's book~\cite{andrews86} is a
full description of the original Church-style higher-order logic.  Experience
with the {\sc hol} system has demonstrated that higher-order logic is widely
applicable in many areas of mathematics and computer science, not just
hardware verification, {\sc hol}'s original \textit{raison d'{\^e}tre\/}.  It
is weaker than ZF set theory but for most applications this does not matter.
If you prefer {\ML} to Lisp, you will probably prefer HOL to~ZF.

The syntax of HOL\footnote{Earlier versions of Isabelle's HOL used a different
  syntax.  Ancient releases of Isabelle included still another version of~HOL,
  with explicit type inference rules~\cite{paulson-COLOG}.  This version no
  longer exists, but \thydx{ZF} supports a similar style of reasoning.}
follows $\lambda$-calculus and functional programming.  Function application
is curried.  To apply the function~$f$ of type $\tau@1\To\tau@2\To\tau@3$ to
the arguments~$a$ and~$b$ in HOL, you simply write $f\,a\,b$.  There is no
`apply' operator as in \thydx{ZF}.  Note that $f(a,b)$ means ``$f$ applied to
the pair $(a,b)$'' in HOL.  We write ordered pairs as $(a,b)$, not $\langle
a,b\rangle$ as in ZF.

HOL has a distinct feel, compared with ZF and CTT.  It identifies object-level
types with meta-level types, taking advantage of Isabelle's built-in
type-checker.  It identifies object-level functions with meta-level functions,
so it uses Isabelle's operations for abstraction and application.

These identifications allow Isabelle to support HOL particularly nicely, but
they also mean that HOL requires more sophistication from the user --- in
particular, an understanding of Isabelle's type system.  Beginners should work
with \texttt{show_types} (or even \texttt{show_sorts}) set to \texttt{true}.


\begin{figure}
\begin{constants}
  \it name      &\it meta-type  & \it description \\
  \cdx{Trueprop}& $bool\To prop$                & coercion to $prop$\\
  \cdx{Not}     & $bool\To bool$                & negation ($\lnot$) \\
  \cdx{True}    & $bool$                        & tautology ($\top$) \\
  \cdx{False}   & $bool$                        & absurdity ($\bot$) \\
  \cdx{If}      & $[bool,\alpha,\alpha]\To\alpha$ & conditional \\
  \cdx{Let}     & $[\alpha,\alpha\To\beta]\To\beta$ & let binder
\end{constants}
\subcaption{Constants}

\begin{constants}
\index{"@@{\tt\at} symbol}
\index{*"! symbol}\index{*"? symbol}
\index{*"?"! symbol}\index{*"E"X"! symbol}
  \it symbol &\it name     &\it meta-type & \it description \\
  \sdx{SOME} or \tt\at & \cdx{Eps}  & $(\alpha\To bool)\To\alpha$ & 
        Hilbert description ($\varepsilon$) \\
  \sdx{ALL} or {\tt!~} & \cdx{All}  & $(\alpha\To bool)\To bool$ & 
        universal quantifier ($\forall$) \\
  \sdx{EX} or {\tt?~}  & \cdx{Ex}   & $(\alpha\To bool)\To bool$ & 
        existential quantifier ($\exists$) \\
  \texttt{EX!} or {\tt?!} & \cdx{Ex1}  & $(\alpha\To bool)\To bool$ & 
        unique existence ($\exists!$)\\
  \texttt{LEAST}  & \cdx{Least}  & $(\alpha::ord \To bool)\To\alpha$ & 
        least element
\end{constants}
\subcaption{Binders} 

\begin{constants}
\index{*"= symbol}
\index{&@{\tt\&} symbol}
\index{*"| symbol}
\index{*"-"-"> symbol}
  \it symbol    & \it meta-type & \it priority & \it description \\ 
  \sdx{o}       & $[\beta\To\gamma,\alpha\To\beta]\To (\alpha\To\gamma)$ & 
        Left 55 & composition ($\circ$) \\
  \tt =         & $[\alpha,\alpha]\To bool$ & Left 50 & equality ($=$) \\
  \tt <         & $[\alpha::ord,\alpha]\To bool$ & Left 50 & less than ($<$) \\
  \tt <=        & $[\alpha::ord,\alpha]\To bool$ & Left 50 & 
                less than or equals ($\leq$)\\
  \tt \&        & $[bool,bool]\To bool$ & Right 35 & conjunction ($\conj$) \\
  \tt |         & $[bool,bool]\To bool$ & Right 30 & disjunction ($\disj$) \\
  \tt -->       & $[bool,bool]\To bool$ & Right 25 & implication ($\imp$)
\end{constants}
\subcaption{Infixes}
\caption{Syntax of \texttt{HOL}} \label{hol-constants}
\end{figure}


\begin{figure}
\index{*let symbol}
\index{*in symbol}
\dquotes
\[\begin{array}{rclcl}
    term & = & \hbox{expression of class~$term$} \\
         & | & "SOME~" id " . " formula
         & | & "\at~" id " . " formula \\
         & | & 
    \multicolumn{3}{l}{"let"~id~"="~term";"\dots";"~id~"="~term~"in"~term} \\
         & | & 
    \multicolumn{3}{l}{"if"~formula~"then"~term~"else"~term} \\
         & | & "LEAST"~ id " . " formula \\[2ex]
 formula & = & \hbox{expression of type~$bool$} \\
         & | & term " = " term \\
         & | & term " \ttilde= " term \\
         & | & term " < " term \\
         & | & term " <= " term \\
         & | & "\ttilde\ " formula \\
         & | & formula " \& " formula \\
         & | & formula " | " formula \\
         & | & formula " --> " formula \\
         & | & "ALL~" id~id^* " . " formula
         & | & "!~~~" id~id^* " . " formula \\
         & | & "EX~~" id~id^* " . " formula 
         & | & "?~~~" id~id^* " . " formula \\
         & | & "EX!~" id~id^* " . " formula
         & | & "?!~~" id~id^* " . " formula \\
  \end{array}
\]
\caption{Full grammar for HOL} \label{hol-grammar}
\end{figure} 


\section{Syntax}

Figure~\ref{hol-constants} lists the constants (including infixes and
binders), while Fig.\ts\ref{hol-grammar} presents the grammar of
higher-order logic.  Note that $a$\verb|~=|$b$ is translated to
$\lnot(a=b)$.

\begin{warn}
  HOL has no if-and-only-if connective; logical equivalence is expressed using
  equality.  But equality has a high priority, as befitting a relation, while
  if-and-only-if typically has the lowest priority.  Thus, $\lnot\lnot P=P$
  abbreviates $\lnot\lnot (P=P)$ and not $(\lnot\lnot P)=P$.  When using $=$
  to mean logical equivalence, enclose both operands in parentheses.
\end{warn}

\subsection{Types and overloading}
The universal type class of higher-order terms is called~\cldx{term}.
By default, explicit type variables have class \cldx{term}.  In
particular the equality symbol and quantifiers are polymorphic over
class \texttt{term}.

The type of formulae, \tydx{bool}, belongs to class \cldx{term}; thus,
formulae are terms.  The built-in type~\tydx{fun}, which constructs
function types, is overloaded with arity {\tt(term,\thinspace
  term)\thinspace term}.  Thus, $\sigma\To\tau$ belongs to class~{\tt
  term} if $\sigma$ and~$\tau$ do, allowing quantification over
functions.

HOL allows new types to be declared as subsets of existing types,
either using the primitive \texttt{typedef} or the more convenient
\texttt{datatype} (see~{\S}\ref{sec:HOL:datatype}).

Several syntactic type classes --- \cldx{plus}, \cldx{minus},
\cldx{times} and
\cldx{power} --- permit overloading of the operators {\tt+},\index{*"+
  symbol} {\tt-}\index{*"- symbol}, {\tt*}.\index{*"* symbol} 
and \verb|^|.\index{^@\verb.^. symbol} 
%
They are overloaded to denote the obvious arithmetic operations on types
\tdx{nat}, \tdx{int} and~\tdx{real}. (With the \verb|^| operator, the
exponent always has type~\tdx{nat}.)  Non-arithmetic overloadings are also
done: the operator {\tt-} can denote set difference, while \verb|^| can
denote exponentiation of relations (iterated composition).  Unary minus is
also written as~{\tt-} and is overloaded like its 2-place counterpart; it even
can stand for set complement.

The constant \cdx{0} is also overloaded.  It serves as the zero element of
several types, of which the most important is \tdx{nat} (the natural
numbers).  The type class \cldx{plus_ac0} comprises all types for which 0
and~+ satisfy the laws $x+y=y+x$, $(x+y)+z = x+(y+z)$ and $0+x = x$.  These
types include the numeric ones \tdx{nat}, \tdx{int} and~\tdx{real} and also
multisets.  The summation operator \cdx{setsum} is available for all types in
this class. 

Theory \thydx{Ord} defines the syntactic class \cldx{ord} of order
signatures.  The relations $<$ and $\leq$ are polymorphic over this
class, as are the functions \cdx{mono}, \cdx{min} and \cdx{max}, and
the \cdx{LEAST} operator. \thydx{Ord} also defines a subclass
\cldx{order} of \cldx{ord} which axiomatizes the types that are partially
ordered with respect to~$\leq$.  A further subclass \cldx{linorder} of
\cldx{order} axiomatizes linear orderings.
For details, see the file \texttt{Ord.thy}.
                                          
If you state a goal containing overloaded functions, you may need to include
type constraints.  Type inference may otherwise make the goal more
polymorphic than you intended, with confusing results.  For example, the
variables $i$, $j$ and $k$ in the goal $i \leq j \Imp i \leq j+k$ have type
$\alpha::\{ord,plus\}$, although you may have expected them to have some
numeric type, e.g. $nat$.  Instead you should have stated the goal as
$(i::nat) \leq j \Imp i \leq j+k$, which causes all three variables to have
type $nat$.

\begin{warn}
  If resolution fails for no obvious reason, try setting
  \ttindex{show_types} to \texttt{true}, causing Isabelle to display
  types of terms.  Possibly set \ttindex{show_sorts} to \texttt{true} as
  well, causing Isabelle to display type classes and sorts.

  \index{unification!incompleteness of}
  Where function types are involved, Isabelle's unification code does not
  guarantee to find instantiations for type variables automatically.  Be
  prepared to use \ttindex{res_inst_tac} instead of \texttt{resolve_tac},
  possibly instantiating type variables.  Setting
  \ttindex{Unify.trace_types} to \texttt{true} causes Isabelle to report
  omitted search paths during unification.\index{tracing!of unification}
\end{warn}


\subsection{Binders}

Hilbert's {\bf description} operator~$\varepsilon x. P[x]$ stands for some~$x$
satisfying~$P$, if such exists.  Since all terms in HOL denote something, a
description is always meaningful, but we do not know its value unless $P$
defines it uniquely.  We may write descriptions as \cdx{Eps}($\lambda x.
P[x]$) or use the syntax \hbox{\tt SOME~$x$.~$P[x]$}.

Existential quantification is defined by
\[ \exists x. P~x \;\equiv\; P(\varepsilon x. P~x). \]
The unique existence quantifier, $\exists!x. P$, is defined in terms
of~$\exists$ and~$\forall$.  An Isabelle binder, it admits nested
quantifications.  For instance, $\exists!x\,y. P\,x\,y$ abbreviates
$\exists!x. \exists!y. P\,x\,y$; note that this does not mean that there
exists a unique pair $(x,y)$ satisfying~$P\,x\,y$.

\medskip

\index{*"! symbol}\index{*"? symbol}\index{HOL system@{\sc hol} system} The
basic Isabelle/HOL binders have two notations.  Apart from the usual
\texttt{ALL} and \texttt{EX} for $\forall$ and $\exists$, Isabelle/HOL also
supports the original notation of Gordon's {\sc hol} system: \texttt{!}\ 
and~\texttt{?}.  In the latter case, the existential quantifier \emph{must} be
followed by a space; thus {\tt?x} is an unknown, while \verb'? x. f x=y' is a
quantification.  Both notations are accepted for input.  The print mode
``\ttindexbold{HOL}'' governs the output notation.  If enabled (e.g.\ by
passing option \texttt{-m HOL} to the \texttt{isabelle} executable),
then~{\tt!}\ and~{\tt?}\ are displayed.

\medskip

If $\tau$ is a type of class \cldx{ord}, $P$ a formula and $x$ a
variable of type $\tau$, then the term \cdx{LEAST}~$x. P[x]$ is defined
to be the least (w.r.t.\ $\leq$) $x$ such that $P~x$ holds (see
Fig.~\ref{hol-defs}).  The definition uses Hilbert's $\varepsilon$
choice operator, so \texttt{Least} is always meaningful, but may yield
nothing useful in case there is not a unique least element satisfying
$P$.\footnote{Class $ord$ does not require much of its instances, so
  $\leq$ need not be a well-ordering, not even an order at all!}

\medskip All these binders have priority 10.

\begin{warn}
The low priority of binders means that they need to be enclosed in
parenthesis when they occur in the context of other operations.  For example,
instead of $P \land \forall x. Q$ you need to write $P \land (\forall x. Q)$.
\end{warn}


\subsection{The let and case constructions}
Local abbreviations can be introduced by a \texttt{let} construct whose
syntax appears in Fig.\ts\ref{hol-grammar}.  Internally it is translated into
the constant~\cdx{Let}.  It can be expanded by rewriting with its
definition, \tdx{Let_def}.

HOL also defines the basic syntax
\[\dquotes"case"~e~"of"~c@1~"=>"~e@1~"|" \dots "|"~c@n~"=>"~e@n\] 
as a uniform means of expressing \texttt{case} constructs.  Therefore \texttt{case}
and \sdx{of} are reserved words.  Initially, this is mere syntax and has no
logical meaning.  By declaring translations, you can cause instances of the
\texttt{case} construct to denote applications of particular case operators.
This is what happens automatically for each \texttt{datatype} definition
(see~{\S}\ref{sec:HOL:datatype}).

\begin{warn}
Both \texttt{if} and \texttt{case} constructs have as low a priority as
quantifiers, which requires additional enclosing parentheses in the context
of most other operations.  For example, instead of $f~x = {\tt if\dots
then\dots else}\dots$ you need to write $f~x = ({\tt if\dots then\dots
else\dots})$.
\end{warn}

\section{Rules of inference}

\begin{figure}
\begin{ttbox}\makeatother
\tdx{refl}          t = (t::'a)
\tdx{subst}         [| s = t; P s |] ==> P (t::'a)
\tdx{ext}           (!!x::'a. (f x :: 'b) = g x) ==> (\%x. f x) = (\%x. g x)
\tdx{impI}          (P ==> Q) ==> P-->Q
\tdx{mp}            [| P-->Q;  P |] ==> Q
\tdx{iff}           (P-->Q) --> (Q-->P) --> (P=Q)
\tdx{someI}         P(x::'a) ==> P(@x. P x)
\tdx{True_or_False} (P=True) | (P=False)
\end{ttbox}
\caption{The \texttt{HOL} rules} \label{hol-rules}
\end{figure}

Figure~\ref{hol-rules} shows the primitive inference rules of~HOL, with
their~{\ML} names.  Some of the rules deserve additional comments:
\begin{ttdescription}
\item[\tdx{ext}] expresses extensionality of functions.
\item[\tdx{iff}] asserts that logically equivalent formulae are
  equal.
\item[\tdx{someI}] gives the defining property of the Hilbert
  $\varepsilon$-operator.  It is a form of the Axiom of Choice.  The derived rule
  \tdx{some_equality} (see below) is often easier to use.
\item[\tdx{True_or_False}] makes the logic classical.\footnote{In
    fact, the $\varepsilon$-operator already makes the logic classical, as
    shown by Diaconescu; see Paulson~\cite{paulson-COLOG} for details.}
\end{ttdescription}


\begin{figure}\hfuzz=4pt%suppress "Overfull \hbox" message
\begin{ttbox}\makeatother
\tdx{True_def}   True     == ((\%x::bool. x)=(\%x. x))
\tdx{All_def}    All      == (\%P. P = (\%x. True))
\tdx{Ex_def}     Ex       == (\%P. P(@x. P x))
\tdx{False_def}  False    == (!P. P)
\tdx{not_def}    not      == (\%P. P-->False)
\tdx{and_def}    op &     == (\%P Q. !R. (P-->Q-->R) --> R)
\tdx{or_def}     op |     == (\%P Q. !R. (P-->R) --> (Q-->R) --> R)
\tdx{Ex1_def}    Ex1      == (\%P. ? x. P x & (! y. P y --> y=x))

\tdx{o_def}      op o     == (\%(f::'b=>'c) g x::'a. f(g x))
\tdx{if_def}     If P x y ==
              (\%P x y. @z::'a.(P=True --> z=x) & (P=False --> z=y))
\tdx{Let_def}    Let s f  == f s
\tdx{Least_def}  Least P  == @x. P(x) & (ALL y. P(y) --> x <= y)"
\end{ttbox}
\caption{The \texttt{HOL} definitions} \label{hol-defs}
\end{figure}


HOL follows standard practice in higher-order logic: only a few connectives
are taken as primitive, with the remainder defined obscurely
(Fig.\ts\ref{hol-defs}).  Gordon's {\sc hol} system expresses the
corresponding definitions \cite[page~270]{mgordon-hol} using
object-equality~({\tt=}), which is possible because equality in higher-order
logic may equate formulae and even functions over formulae.  But theory~HOL,
like all other Isabelle theories, uses meta-equality~({\tt==}) for
definitions.
\begin{warn}
The definitions above should never be expanded and are shown for completeness
only.  Instead users should reason in terms of the derived rules shown below
or, better still, using high-level tactics
(see~{\S}\ref{sec:HOL:generic-packages}).
\end{warn}

Some of the rules mention type variables; for example, \texttt{refl}
mentions the type variable~{\tt'a}.  This allows you to instantiate
type variables explicitly by calling \texttt{res_inst_tac}.


\begin{figure}
\begin{ttbox}
\tdx{sym}         s=t ==> t=s
\tdx{trans}       [| r=s; s=t |] ==> r=t
\tdx{ssubst}      [| t=s; P s |] ==> P t
\tdx{box_equals}  [| a=b;  a=c;  b=d |] ==> c=d  
\tdx{arg_cong}    x = y ==> f x = f y
\tdx{fun_cong}    f = g ==> f x = g x
\tdx{cong}        [| f = g; x = y |] ==> f x = g y
\tdx{not_sym}     t ~= s ==> s ~= t
\subcaption{Equality}

\tdx{TrueI}       True 
\tdx{FalseE}      False ==> P

\tdx{conjI}       [| P; Q |] ==> P&Q
\tdx{conjunct1}   [| P&Q |] ==> P
\tdx{conjunct2}   [| P&Q |] ==> Q 
\tdx{conjE}       [| P&Q;  [| P; Q |] ==> R |] ==> R

\tdx{disjI1}      P ==> P|Q
\tdx{disjI2}      Q ==> P|Q
\tdx{disjE}       [| P | Q; P ==> R; Q ==> R |] ==> R

\tdx{notI}        (P ==> False) ==> ~ P
\tdx{notE}        [| ~ P;  P |] ==> R
\tdx{impE}        [| P-->Q;  P;  Q ==> R |] ==> R
\subcaption{Propositional logic}

\tdx{iffI}        [| P ==> Q;  Q ==> P |] ==> P=Q
\tdx{iffD1}       [| P=Q; P |] ==> Q
\tdx{iffD2}       [| P=Q; Q |] ==> P
\tdx{iffE}        [| P=Q; [| P --> Q; Q --> P |] ==> R |] ==> R
\subcaption{Logical equivalence}

\end{ttbox}
\caption{Derived rules for HOL} \label{hol-lemmas1}
\end{figure}
%
%\tdx{eqTrueI}     P ==> P=True 
%\tdx{eqTrueE}     P=True ==> P 


\begin{figure}
\begin{ttbox}\makeatother
\tdx{allI}      (!!x. P x) ==> !x. P x
\tdx{spec}      !x. P x ==> P x
\tdx{allE}      [| !x. P x;  P x ==> R |] ==> R
\tdx{all_dupE}  [| !x. P x;  [| P x; !x. P x |] ==> R |] ==> R

\tdx{exI}       P x ==> ? x. P x
\tdx{exE}       [| ? x. P x; !!x. P x ==> Q |] ==> Q

\tdx{ex1I}      [| P a;  !!x. P x ==> x=a |] ==> ?! x. P x
\tdx{ex1E}      [| ?! x. P x;  !!x. [| P x;  ! y. P y --> y=x |] ==> R 
          |] ==> R

\tdx{some_equality}   [| P a;  !!x. P x ==> x=a |] ==> (@x. P x) = a
\subcaption{Quantifiers and descriptions}

\tdx{ccontr}          (~P ==> False) ==> P
\tdx{classical}       (~P ==> P) ==> P
\tdx{excluded_middle} ~P | P

\tdx{disjCI}       (~Q ==> P) ==> P|Q
\tdx{exCI}         (! x. ~ P x ==> P a) ==> ? x. P x
\tdx{impCE}        [| P-->Q; ~ P ==> R; Q ==> R |] ==> R
\tdx{iffCE}        [| P=Q;  [| P;Q |] ==> R;  [| ~P; ~Q |] ==> R |] ==> R
\tdx{notnotD}      ~~P ==> P
\tdx{swap}         ~P ==> (~Q ==> P) ==> Q
\subcaption{Classical logic}

\tdx{if_P}         P ==> (if P then x else y) = x
\tdx{if_not_P}     ~ P ==> (if P then x else y) = y
\tdx{split_if}     P(if Q then x else y) = ((Q --> P x) & (~Q --> P y))
\subcaption{Conditionals}
\end{ttbox}
\caption{More derived rules} \label{hol-lemmas2}
\end{figure}

Some derived rules are shown in Figures~\ref{hol-lemmas1}
and~\ref{hol-lemmas2}, with their {\ML} names.  These include natural rules
for the logical connectives, as well as sequent-style elimination rules for
conjunctions, implications, and universal quantifiers.  

Note the equality rules: \tdx{ssubst} performs substitution in
backward proofs, while \tdx{box_equals} supports reasoning by
simplifying both sides of an equation.

The following simple tactics are occasionally useful:
\begin{ttdescription}
\item[\ttindexbold{strip_tac} $i$] applies \texttt{allI} and \texttt{impI}
  repeatedly to remove all outermost universal quantifiers and implications
  from subgoal $i$.
\item[\ttindexbold{case_tac} {\tt"}$P${\tt"} $i$] performs case distinction on
  $P$ for subgoal $i$: the latter is replaced by two identical subgoals with
  the added assumptions $P$ and $\lnot P$, respectively.
\item[\ttindexbold{smp_tac} $j$ $i$] applies $j$ times \texttt{spec} and then
  \texttt{mp} in subgoal $i$, which is typically useful when forward-chaining 
  from an induction hypothesis. As a generalization of \texttt{mp_tac}, 
  if there are assumptions $\forall \vec{x}. P \vec{x} \imp Q \vec{x}$ and 
  $P \vec{a}$, ($\vec{x}$ being a vector of $j$ variables)
  then it replaces the universally quantified implication by $Q \vec{a}$. 
  It may instantiate unknowns. It fails if it can do nothing.
\end{ttdescription}


\begin{figure} 
\begin{center}
\begin{tabular}{rrr}
  \it name      &\it meta-type  & \it description \\ 
\index{{}@\verb'{}' symbol}
  \verb|{}|     & $\alpha\,set$         & the empty set \\
  \cdx{insert}  & $[\alpha,\alpha\,set]\To \alpha\,set$
        & insertion of element \\
  \cdx{Collect} & $(\alpha\To bool)\To\alpha\,set$
        & comprehension \\
  \cdx{INTER} & $[\alpha\,set,\alpha\To\beta\,set]\To\beta\,set$
        & intersection over a set\\
  \cdx{UNION} & $[\alpha\,set,\alpha\To\beta\,set]\To\beta\,set$
        & union over a set\\
  \cdx{Inter} & $(\alpha\,set)set\To\alpha\,set$
        &set of sets intersection \\
  \cdx{Union} & $(\alpha\,set)set\To\alpha\,set$
        &set of sets union \\
  \cdx{Pow}   & $\alpha\,set \To (\alpha\,set)set$
        & powerset \\[1ex]
  \cdx{range}   & $(\alpha\To\beta )\To\beta\,set$
        & range of a function \\[1ex]
  \cdx{Ball}~~\cdx{Bex} & $[\alpha\,set,\alpha\To bool]\To bool$
        & bounded quantifiers
\end{tabular}
\end{center}
\subcaption{Constants}

\begin{center}
\begin{tabular}{llrrr} 
  \it symbol &\it name     &\it meta-type & \it priority & \it description \\
  \sdx{INT}  & \cdx{INTER1}  & $(\alpha\To\beta\,set)\To\beta\,set$ & 10 & 
        intersection\\
  \sdx{UN}  & \cdx{UNION1}  & $(\alpha\To\beta\,set)\To\beta\,set$ & 10 & 
        union 
\end{tabular}
\end{center}
\subcaption{Binders} 

\begin{center}
\index{*"`"` symbol}
\index{*": symbol}
\index{*"<"= symbol}
\begin{tabular}{rrrr} 
  \it symbol    & \it meta-type & \it priority & \it description \\ 
  \tt ``        & $[\alpha\To\beta ,\alpha\,set]\To  \beta\,set$
        & Left 90 & image \\
  \sdx{Int}     & $[\alpha\,set,\alpha\,set]\To\alpha\,set$
        & Left 70 & intersection ($\int$) \\
  \sdx{Un}      & $[\alpha\,set,\alpha\,set]\To\alpha\,set$
        & Left 65 & union ($\un$) \\
  \tt:          & $[\alpha ,\alpha\,set]\To bool$       
        & Left 50 & membership ($\in$) \\
  \tt <=        & $[\alpha\,set,\alpha\,set]\To bool$
        & Left 50 & subset ($\subseteq$) 
\end{tabular}
\end{center}
\subcaption{Infixes}
\caption{Syntax of the theory \texttt{Set}} \label{hol-set-syntax}
\end{figure} 


\begin{figure} 
\begin{center} \tt\frenchspacing
\index{*"! symbol}
\begin{tabular}{rrr} 
  \it external          & \it internal  & \it description \\ 
  $a$ \ttilde: $b$      & \ttilde($a$ : $b$)    & \rm not in\\
  {\ttlbrace}$a@1$, $\ldots${\ttrbrace}  &  insert $a@1$ $\ldots$ {\ttlbrace}{\ttrbrace} & \rm finite set \\
  {\ttlbrace}$x$. $P[x]${\ttrbrace}        &  Collect($\lambda x. P[x]$) &
        \rm comprehension \\
  \sdx{INT} $x$:$A$. $B[x]$      & INTER $A$ $\lambda x. B[x]$ &
        \rm intersection \\
  \sdx{UN}{\tt\ }  $x$:$A$. $B[x]$      & UNION $A$ $\lambda x. B[x]$ &
        \rm union \\
  \sdx{ALL} $x$:$A$.\ $P[x]$ or \texttt{!} $x$:$A$.\ $P[x]$ &
        Ball $A$ $\lambda x.\ P[x]$ & 
        \rm bounded $\forall$ \\
  \sdx{EX}{\tt\ } $x$:$A$.\ $P[x]$ or \texttt{?} $x$:$A$.\ $P[x]$ & 
        Bex $A$ $\lambda x.\ P[x]$ & \rm bounded $\exists$
\end{tabular}
\end{center}
\subcaption{Translations}

\dquotes
\[\begin{array}{rclcl}
    term & = & \hbox{other terms\ldots} \\
         & | & "{\ttlbrace}{\ttrbrace}" \\
         & | & "{\ttlbrace} " term\; ("," term)^* " {\ttrbrace}" \\
         & | & "{\ttlbrace} " id " . " formula " {\ttrbrace}" \\
         & | & term " `` " term \\
         & | & term " Int " term \\
         & | & term " Un " term \\
         & | & "INT~~"  id ":" term " . " term \\
         & | & "UN~~~"  id ":" term " . " term \\
         & | & "INT~~"  id~id^* " . " term \\
         & | & "UN~~~"  id~id^* " . " term \\[2ex]
 formula & = & \hbox{other formulae\ldots} \\
         & | & term " : " term \\
         & | & term " \ttilde: " term \\
         & | & term " <= " term \\
         & | & "ALL " id ":" term " . " formula
         & | & "!~" id ":" term " . " formula \\
         & | & "EX~~" id ":" term " . " formula
         & | & "?~" id ":" term " . " formula \\
  \end{array}
\]
\subcaption{Full Grammar}
\caption{Syntax of the theory \texttt{Set} (continued)} \label{hol-set-syntax2}
\end{figure} 


\section{A formulation of set theory}
Historically, higher-order logic gives a foundation for Russell and
Whitehead's theory of classes.  Let us use modern terminology and call them
{\bf sets}, but note that these sets are distinct from those of ZF set theory,
and behave more like ZF classes.
\begin{itemize}
\item
Sets are given by predicates over some type~$\sigma$.  Types serve to
define universes for sets, but type-checking is still significant.
\item
There is a universal set (for each type).  Thus, sets have complements, and
may be defined by absolute comprehension.
\item
Although sets may contain other sets as elements, the containing set must
have a more complex type.
\end{itemize}
Finite unions and intersections have the same behaviour in HOL as they do
in~ZF.  In HOL the intersection of the empty set is well-defined, denoting the
universal set for the given type.

\subsection{Syntax of set theory}\index{*set type}
HOL's set theory is called \thydx{Set}.  The type $\alpha\,set$ is essentially
the same as $\alpha\To bool$.  The new type is defined for clarity and to
avoid complications involving function types in unification.  The isomorphisms
between the two types are declared explicitly.  They are very natural:
\texttt{Collect} maps $\alpha\To bool$ to $\alpha\,set$, while \hbox{\tt op :}
maps in the other direction (ignoring argument order).

Figure~\ref{hol-set-syntax} lists the constants, infixes, and syntax
translations.  Figure~\ref{hol-set-syntax2} presents the grammar of the new
constructs.  Infix operators include union and intersection ($A\un B$
and $A\int B$), the subset and membership relations, and the image
operator~{\tt``}\@.  Note that $a$\verb|~:|$b$ is translated to
$\lnot(a\in b)$.  

The $\{a@1,\ldots\}$ notation abbreviates finite sets constructed in
the obvious manner using~\texttt{insert} and~$\{\}$:
\begin{eqnarray*}
  \{a, b, c\} & \equiv &
  \texttt{insert} \, a \, ({\tt insert} \, b \, ({\tt insert} \, c \, \{\}))
\end{eqnarray*}

The set \hbox{\tt{\ttlbrace}$x$.\ $P[x]${\ttrbrace}} consists of all $x$ (of
suitable type) that satisfy~$P[x]$, where $P[x]$ is a formula that may contain
free occurrences of~$x$.  This syntax expands to \cdx{Collect}$(\lambda x.
P[x])$.  It defines sets by absolute comprehension, which is impossible in~ZF;
the type of~$x$ implicitly restricts the comprehension.

The set theory defines two {\bf bounded quantifiers}:
\begin{eqnarray*}
   \forall x\in A. P[x] &\hbox{abbreviates}& \forall x. x\in A\imp P[x] \\
   \exists x\in A. P[x] &\hbox{abbreviates}& \exists x. x\in A\conj P[x]
\end{eqnarray*}
The constants~\cdx{Ball} and~\cdx{Bex} are defined
accordingly.  Instead of \texttt{Ball $A$ $P$} and \texttt{Bex $A$ $P$} we may
write\index{*"! symbol}\index{*"? symbol}
\index{*ALL symbol}\index{*EX symbol} 
%
\hbox{\tt ALL~$x$:$A$.\ $P[x]$} and \hbox{\tt EX~$x$:$A$.\ $P[x]$}.  The
original notation of Gordon's {\sc hol} system is supported as well:
\texttt{!}\ and \texttt{?}.

Unions and intersections over sets, namely $\bigcup@{x\in A}B[x]$ and
$\bigcap@{x\in A}B[x]$, are written 
\sdx{UN}~\hbox{\tt$x$:$A$.\ $B[x]$} and
\sdx{INT}~\hbox{\tt$x$:$A$.\ $B[x]$}.  

Unions and intersections over types, namely $\bigcup@x B[x]$ and $\bigcap@x
B[x]$, are written \sdx{UN}~\hbox{\tt$x$.\ $B[x]$} and
\sdx{INT}~\hbox{\tt$x$.\ $B[x]$}.  They are equivalent to the previous
union and intersection operators when $A$ is the universal set.

The operators $\bigcup A$ and $\bigcap A$ act upon sets of sets.  They are
not binders, but are equal to $\bigcup@{x\in A}x$ and $\bigcap@{x\in A}x$,
respectively.



\begin{figure} \underscoreon
\begin{ttbox}
\tdx{mem_Collect_eq}    (a : {\ttlbrace}x. P x{\ttrbrace}) = P a
\tdx{Collect_mem_eq}    {\ttlbrace}x. x:A{\ttrbrace} = A

\tdx{empty_def}         {\ttlbrace}{\ttrbrace}          == {\ttlbrace}x. False{\ttrbrace}
\tdx{insert_def}        insert a B  == {\ttlbrace}x. x=a{\ttrbrace} Un B
\tdx{Ball_def}          Ball A P    == ! x. x:A --> P x
\tdx{Bex_def}           Bex A P     == ? x. x:A & P x
\tdx{subset_def}        A <= B      == ! x:A. x:B
\tdx{Un_def}            A Un B      == {\ttlbrace}x. x:A | x:B{\ttrbrace}
\tdx{Int_def}           A Int B     == {\ttlbrace}x. x:A & x:B{\ttrbrace}
\tdx{set_diff_def}      A - B       == {\ttlbrace}x. x:A & x~:B{\ttrbrace}
\tdx{Compl_def}         -A          == {\ttlbrace}x. ~ x:A{\ttrbrace}
\tdx{INTER_def}         INTER A B   == {\ttlbrace}y. ! x:A. y: B x{\ttrbrace}
\tdx{UNION_def}         UNION A B   == {\ttlbrace}y. ? x:A. y: B x{\ttrbrace}
\tdx{INTER1_def}        INTER1 B    == INTER {\ttlbrace}x. True{\ttrbrace} B 
\tdx{UNION1_def}        UNION1 B    == UNION {\ttlbrace}x. True{\ttrbrace} B 
\tdx{Inter_def}         Inter S     == (INT x:S. x)
\tdx{Union_def}         Union S     == (UN  x:S. x)
\tdx{Pow_def}           Pow A       == {\ttlbrace}B. B <= A{\ttrbrace}
\tdx{image_def}         f``A        == {\ttlbrace}y. ? x:A. y=f x{\ttrbrace}
\tdx{range_def}         range f     == {\ttlbrace}y. ? x. y=f x{\ttrbrace}
\end{ttbox}
\caption{Rules of the theory \texttt{Set}} \label{hol-set-rules}
\end{figure}


\begin{figure} \underscoreon
\begin{ttbox}
\tdx{CollectI}        [| P a |] ==> a : {\ttlbrace}x. P x{\ttrbrace}
\tdx{CollectD}        [| a : {\ttlbrace}x. P x{\ttrbrace} |] ==> P a
\tdx{CollectE}        [| a : {\ttlbrace}x. P x{\ttrbrace};  P a ==> W |] ==> W

\tdx{ballI}           [| !!x. x:A ==> P x |] ==> ! x:A. P x
\tdx{bspec}           [| ! x:A. P x;  x:A |] ==> P x
\tdx{ballE}           [| ! x:A. P x;  P x ==> Q;  ~ x:A ==> Q |] ==> Q

\tdx{bexI}            [| P x;  x:A |] ==> ? x:A. P x
\tdx{bexCI}           [| ! x:A. ~ P x ==> P a;  a:A |] ==> ? x:A. P x
\tdx{bexE}            [| ? x:A. P x;  !!x. [| x:A; P x |] ==> Q  |] ==> Q
\subcaption{Comprehension and Bounded quantifiers}

\tdx{subsetI}         (!!x. x:A ==> x:B) ==> A <= B
\tdx{subsetD}         [| A <= B;  c:A |] ==> c:B
\tdx{subsetCE}        [| A <= B;  ~ (c:A) ==> P;  c:B ==> P |] ==> P

\tdx{subset_refl}     A <= A
\tdx{subset_trans}    [| A<=B;  B<=C |] ==> A<=C

\tdx{equalityI}       [| A <= B;  B <= A |] ==> A = B
\tdx{equalityD1}      A = B ==> A<=B
\tdx{equalityD2}      A = B ==> B<=A
\tdx{equalityE}       [| A = B;  [| A<=B; B<=A |] ==> P |]  ==>  P

\tdx{equalityCE}      [| A = B;  [| c:A; c:B |] ==> P;  
                           [| ~ c:A; ~ c:B |] ==> P 
                |]  ==>  P
\subcaption{The subset and equality relations}
\end{ttbox}
\caption{Derived rules for set theory} \label{hol-set1}
\end{figure}


\begin{figure} \underscoreon
\begin{ttbox}
\tdx{emptyE}   a : {\ttlbrace}{\ttrbrace} ==> P

\tdx{insertI1} a : insert a B
\tdx{insertI2} a : B ==> a : insert b B
\tdx{insertE}  [| a : insert b A;  a=b ==> P;  a:A ==> P |] ==> P

\tdx{ComplI}   [| c:A ==> False |] ==> c : -A
\tdx{ComplD}   [| c : -A |] ==> ~ c:A

\tdx{UnI1}     c:A ==> c : A Un B
\tdx{UnI2}     c:B ==> c : A Un B
\tdx{UnCI}     (~c:B ==> c:A) ==> c : A Un B
\tdx{UnE}      [| c : A Un B;  c:A ==> P;  c:B ==> P |] ==> P

\tdx{IntI}     [| c:A;  c:B |] ==> c : A Int B
\tdx{IntD1}    c : A Int B ==> c:A
\tdx{IntD2}    c : A Int B ==> c:B
\tdx{IntE}     [| c : A Int B;  [| c:A; c:B |] ==> P |] ==> P

\tdx{UN_I}     [| a:A;  b: B a |] ==> b: (UN x:A. B x)
\tdx{UN_E}     [| b: (UN x:A. B x);  !!x.[| x:A;  b:B x |] ==> R |] ==> R

\tdx{INT_I}    (!!x. x:A ==> b: B x) ==> b : (INT x:A. B x)
\tdx{INT_D}    [| b: (INT x:A. B x);  a:A |] ==> b: B a
\tdx{INT_E}    [| b: (INT x:A. B x);  b: B a ==> R;  ~ a:A ==> R |] ==> R

\tdx{UnionI}   [| X:C;  A:X |] ==> A : Union C
\tdx{UnionE}   [| A : Union C;  !!X.[| A:X;  X:C |] ==> R |] ==> R

\tdx{InterI}   [| !!X. X:C ==> A:X |] ==> A : Inter C
\tdx{InterD}   [| A : Inter C;  X:C |] ==> A:X
\tdx{InterE}   [| A : Inter C;  A:X ==> R;  ~ X:C ==> R |] ==> R

\tdx{PowI}     A<=B ==> A: Pow B
\tdx{PowD}     A: Pow B ==> A<=B

\tdx{imageI}   [| x:A |] ==> f x : f``A
\tdx{imageE}   [| b : f``A;  !!x.[| b=f x;  x:A |] ==> P |] ==> P

\tdx{rangeI}   f x : range f
\tdx{rangeE}   [| b : range f;  !!x.[| b=f x |] ==> P |] ==> P
\end{ttbox}
\caption{Further derived rules for set theory} \label{hol-set2}
\end{figure}


\subsection{Axioms and rules of set theory}
Figure~\ref{hol-set-rules} presents the rules of theory \thydx{Set}.  The
axioms \tdx{mem_Collect_eq} and \tdx{Collect_mem_eq} assert
that the functions \texttt{Collect} and \hbox{\tt op :} are isomorphisms.  Of
course, \hbox{\tt op :} also serves as the membership relation.

All the other axioms are definitions.  They include the empty set, bounded
quantifiers, unions, intersections, complements and the subset relation.
They also include straightforward constructions on functions: image~({\tt``})
and \texttt{range}.

%The predicate \cdx{inj_on} is used for simulating type definitions.
%The statement ${\tt inj_on}~f~A$ asserts that $f$ is injective on the
%set~$A$, which specifies a subset of its domain type.  In a type
%definition, $f$ is the abstraction function and $A$ is the set of valid
%representations; we should not expect $f$ to be injective outside of~$A$.

%\begin{figure} \underscoreon
%\begin{ttbox}
%\tdx{Inv_f_f}    inj f ==> Inv f (f x) = x
%\tdx{f_Inv_f}    y : range f ==> f(Inv f y) = y
%
%\tdx{Inv_injective}
%    [| Inv f x=Inv f y; x: range f;  y: range f |] ==> x=y
%
%
%\tdx{monoI}      [| !!A B. A <= B ==> f A <= f B |] ==> mono f
%\tdx{monoD}      [| mono f;  A <= B |] ==> f A <= f B
%
%\tdx{injI}       [| !! x y. f x = f y ==> x=y |] ==> inj f
%\tdx{inj_inverseI}              (!!x. g(f x) = x) ==> inj f
%\tdx{injD}       [| inj f; f x = f y |] ==> x=y
%
%\tdx{inj_onI}  (!!x y. [| f x=f y; x:A; y:A |] ==> x=y) ==> inj_on f A
%\tdx{inj_onD}  [| inj_on f A;  f x=f y;  x:A;  y:A |] ==> x=y
%
%\tdx{inj_on_inverseI}
%    (!!x. x:A ==> g(f x) = x) ==> inj_on f A
%\tdx{inj_on_contraD}
%    [| inj_on f A;  x~=y;  x:A;  y:A |] ==> ~ f x=f y
%\end{ttbox}
%\caption{Derived rules involving functions} \label{hol-fun}
%\end{figure}


\begin{figure} \underscoreon
\begin{ttbox}
\tdx{Union_upper}     B:A ==> B <= Union A
\tdx{Union_least}     [| !!X. X:A ==> X<=C |] ==> Union A <= C

\tdx{Inter_lower}     B:A ==> Inter A <= B
\tdx{Inter_greatest}  [| !!X. X:A ==> C<=X |] ==> C <= Inter A

\tdx{Un_upper1}       A <= A Un B
\tdx{Un_upper2}       B <= A Un B
\tdx{Un_least}        [| A<=C;  B<=C |] ==> A Un B <= C

\tdx{Int_lower1}      A Int B <= A
\tdx{Int_lower2}      A Int B <= B
\tdx{Int_greatest}    [| C<=A;  C<=B |] ==> C <= A Int B
\end{ttbox}
\caption{Derived rules involving subsets} \label{hol-subset}
\end{figure}


\begin{figure} \underscoreon   \hfuzz=4pt%suppress "Overfull \hbox" message
\begin{ttbox}
\tdx{Int_absorb}        A Int A = A
\tdx{Int_commute}       A Int B = B Int A
\tdx{Int_assoc}         (A Int B) Int C  =  A Int (B Int C)
\tdx{Int_Un_distrib}    (A Un B)  Int C  =  (A Int C) Un (B Int C)

\tdx{Un_absorb}         A Un A = A
\tdx{Un_commute}        A Un B = B Un A
\tdx{Un_assoc}          (A Un B)  Un C  =  A Un (B Un C)
\tdx{Un_Int_distrib}    (A Int B) Un C  =  (A Un C) Int (B Un C)

\tdx{Compl_disjoint}    A Int (-A) = {\ttlbrace}x. False{\ttrbrace}
\tdx{Compl_partition}   A Un  (-A) = {\ttlbrace}x. True{\ttrbrace}
\tdx{double_complement} -(-A) = A
\tdx{Compl_Un}          -(A Un B)  = (-A) Int (-B)
\tdx{Compl_Int}         -(A Int B) = (-A) Un (-B)

\tdx{Union_Un_distrib}  Union(A Un B) = (Union A) Un (Union B)
\tdx{Int_Union}         A Int (Union B) = (UN C:B. A Int C)

\tdx{Inter_Un_distrib}  Inter(A Un B) = (Inter A) Int (Inter B)
\tdx{Un_Inter}          A Un (Inter B) = (INT C:B. A Un C)

\end{ttbox}
\caption{Set equalities} \label{hol-equalities}
\end{figure}
%\tdx{Un_Union_image}    (UN x:C.(A x) Un (B x)) = Union(A``C) Un Union(B``C)
%\tdx{Int_Inter_image}   (INT x:C.(A x) Int (B x)) = Inter(A``C) Int Inter(B``C)

Figures~\ref{hol-set1} and~\ref{hol-set2} present derived rules.  Most are
obvious and resemble rules of Isabelle's ZF set theory.  Certain rules, such
as \tdx{subsetCE}, \tdx{bexCI} and \tdx{UnCI}, are designed for classical
reasoning; the rules \tdx{subsetD}, \tdx{bexI}, \tdx{Un1} and~\tdx{Un2} are
not strictly necessary but yield more natural proofs.  Similarly,
\tdx{equalityCE} supports classical reasoning about extensionality, after the
fashion of \tdx{iffCE}.  See the file \texttt{HOL/Set.ML} for proofs
pertaining to set theory.

Figure~\ref{hol-subset} presents lattice properties of the subset relation.
Unions form least upper bounds; non-empty intersections form greatest lower
bounds.  Reasoning directly about subsets often yields clearer proofs than
reasoning about the membership relation.  See the file \texttt{HOL/subset.ML}.

Figure~\ref{hol-equalities} presents many common set equalities.  They
include commutative, associative and distributive laws involving unions,
intersections and complements.  For a complete listing see the file {\tt
HOL/equalities.ML}.

\begin{warn}
\texttt{Blast_tac} proves many set-theoretic theorems automatically.
Hence you seldom need to refer to the theorems above.
\end{warn}

\begin{figure}
\begin{center}
\begin{tabular}{rrr}
  \it name      &\it meta-type  & \it description \\ 
  \cdx{inj}~~\cdx{surj}& $(\alpha\To\beta )\To bool$
        & injective/surjective \\
  \cdx{inj_on}        & $[\alpha\To\beta ,\alpha\,set]\To bool$
        & injective over subset\\
  \cdx{inv} & $(\alpha\To\beta)\To(\beta\To\alpha)$ & inverse function
\end{tabular}
\end{center}

\underscoreon
\begin{ttbox}
\tdx{inj_def}         inj f      == ! x y. f x=f y --> x=y
\tdx{surj_def}        surj f     == ! y. ? x. y=f x
\tdx{inj_on_def}      inj_on f A == !x:A. !y:A. f x=f y --> x=y
\tdx{inv_def}         inv f      == (\%y. @x. f(x)=y)
\end{ttbox}
\caption{Theory \thydx{Fun}} \label{fig:HOL:Fun}
\end{figure}

\subsection{Properties of functions}\nopagebreak
Figure~\ref{fig:HOL:Fun} presents a theory of simple properties of functions.
Note that ${\tt inv}~f$ uses Hilbert's $\varepsilon$ to yield an inverse
of~$f$.  See the file \texttt{HOL/Fun.ML} for a complete listing of the derived
rules.  Reasoning about function composition (the operator~\sdx{o}) and the
predicate~\cdx{surj} is done simply by expanding the definitions.

There is also a large collection of monotonicity theorems for constructions
on sets in the file \texttt{HOL/mono.ML}.


\section{Generic packages}
\label{sec:HOL:generic-packages}

HOL instantiates most of Isabelle's generic packages, making available the
simplifier and the classical reasoner.

\subsection{Simplification and substitution}

Simplification tactics tactics such as \texttt{Asm_simp_tac} and \texttt{Full_simp_tac} use the default simpset
(\texttt{simpset()}), which works for most purposes.  A quite minimal
simplification set for higher-order logic is~\ttindexbold{HOL_ss};
even more frugal is \ttindexbold{HOL_basic_ss}.  Equality~($=$), which
also expresses logical equivalence, may be used for rewriting.  See
the file \texttt{HOL/simpdata.ML} for a complete listing of the basic
simplification rules.

See \iflabelundefined{chap:classical}{the {\em Reference Manual\/}}%
{Chaps.\ts\ref{substitution} and~\ref{simp-chap}} for details of substitution
and simplification.

\begin{warn}\index{simplification!of conjunctions}%
  Reducing $a=b\conj P(a)$ to $a=b\conj P(b)$ is sometimes advantageous.  The
  left part of a conjunction helps in simplifying the right part.  This effect
  is not available by default: it can be slow.  It can be obtained by
  including \ttindex{conj_cong} in a simpset, \verb$addcongs [conj_cong]$.
\end{warn}

\begin{warn}\index{simplification!of \texttt{if}}\label{if-simp}%
  By default only the condition of an \ttindex{if} is simplified but not the
  \texttt{then} and \texttt{else} parts. Of course the latter are simplified
  once the condition simplifies to \texttt{True} or \texttt{False}. To ensure
  full simplification of all parts of a conditional you must remove
  \ttindex{if_weak_cong} from the simpset, \verb$delcongs [if_weak_cong]$.
\end{warn}

If the simplifier cannot use a certain rewrite rule --- either because
of nontermination or because its left-hand side is too flexible ---
then you might try \texttt{stac}:
\begin{ttdescription}
\item[\ttindexbold{stac} $thm$ $i,$] where $thm$ is of the form $lhs = rhs$,
  replaces in subgoal $i$ instances of $lhs$ by corresponding instances of
  $rhs$.  In case of multiple instances of $lhs$ in subgoal $i$, backtracking
  may be necessary to select the desired ones.

If $thm$ is a conditional equality, the instantiated condition becomes an
additional (first) subgoal.
\end{ttdescription}

HOL provides the tactic \ttindex{hyp_subst_tac}, which substitutes for an
equality throughout a subgoal and its hypotheses.  This tactic uses HOL's
general substitution rule.

\subsubsection{Case splitting}
\label{subsec:HOL:case:splitting}

HOL also provides convenient means for case splitting during rewriting. Goals
containing a subterm of the form \texttt{if}~$b$~{\tt then\dots else\dots}
often require a case distinction on $b$. This is expressed by the theorem
\tdx{split_if}:
$$
\Var{P}(\mbox{\tt if}~\Var{b}~{\tt then}~\Var{x}~\mbox{\tt else}~\Var{y})~=~
((\Var{b} \to \Var{P}(\Var{x})) \land (\lnot \Var{b} \to \Var{P}(\Var{y})))
\eqno{(*)}
$$
For example, a simple instance of $(*)$ is
\[
x \in (\mbox{\tt if}~x \in A~{\tt then}~A~\mbox{\tt else}~\{x\})~=~
((x \in A \to x \in A) \land (x \notin A \to x \in \{x\}))
\]
Because $(*)$ is too general as a rewrite rule for the simplifier (the
left-hand side is not a higher-order pattern in the sense of
\iflabelundefined{chap:simplification}{the {\em Reference Manual\/}}%
{Chap.\ts\ref{chap:simplification}}), there is a special infix function 
\ttindexbold{addsplits} of type \texttt{simpset * thm list -> simpset}
(analogous to \texttt{addsimps}) that adds rules such as $(*)$ to a
simpset, as in
\begin{ttbox}
by(simp_tac (simpset() addsplits [split_if]) 1);
\end{ttbox}
The effect is that after each round of simplification, one occurrence of
\texttt{if} is split acording to \texttt{split_if}, until all occurences of
\texttt{if} have been eliminated.

It turns out that using \texttt{split_if} is almost always the right thing to
do. Hence \texttt{split_if} is already included in the default simpset. If
you want to delete it from a simpset, use \ttindexbold{delsplits}, which is
the inverse of \texttt{addsplits}:
\begin{ttbox}
by(simp_tac (simpset() delsplits [split_if]) 1);
\end{ttbox}

In general, \texttt{addsplits} accepts rules of the form
\[
\Var{P}(c~\Var{x@1}~\dots~\Var{x@n})~=~ rhs
\]
where $c$ is a constant and $rhs$ is arbitrary. Note that $(*)$ is of the
right form because internally the left-hand side is
$\Var{P}(\mathtt{If}~\Var{b}~\Var{x}~~\Var{y})$. Important further examples
are splitting rules for \texttt{case} expressions (see~{\S}\ref{subsec:list}
and~{\S}\ref{subsec:datatype:basics}).

Analogous to \texttt{Addsimps} and \texttt{Delsimps}, there are also
imperative versions of \texttt{addsplits} and \texttt{delsplits}
\begin{ttbox}
\ttindexbold{Addsplits}: thm list -> unit
\ttindexbold{Delsplits}: thm list -> unit
\end{ttbox}
for adding splitting rules to, and deleting them from the current simpset.

\subsection{Classical reasoning}

HOL derives classical introduction rules for $\disj$ and~$\exists$, as well as
classical elimination rules for~$\imp$ and~$\bimp$, and the swap rule; recall
Fig.\ts\ref{hol-lemmas2} above.

The classical reasoner is installed.  Tactics such as \texttt{Blast_tac} and
{\tt Best_tac} refer to the default claset (\texttt{claset()}), which works
for most purposes.  Named clasets include \ttindexbold{prop_cs}, which
includes the propositional rules, and \ttindexbold{HOL_cs}, which also
includes quantifier rules.  See the file \texttt{HOL/cladata.ML} for lists of
the classical rules,
and \iflabelundefined{chap:classical}{the {\em Reference Manual\/}}%
{Chap.\ts\ref{chap:classical}} for more discussion of classical proof methods.


%FIXME outdated, both from the Isabelle and SVC perspective
% \section{Calling the decision procedure SVC}\label{sec:HOL:SVC}

% \index{SVC decision procedure|(}

% The Stanford Validity Checker (SVC) is a tool that can check the validity of
% certain types of formulae.  If it is installed on your machine, then
% Isabelle/HOL can be configured to call it through the tactic
% \ttindex{svc_tac}.  It is ideal for large tautologies and complex problems in
% linear arithmetic.  Subexpressions that SVC cannot handle are automatically
% replaced by variables, so you can call the tactic on any subgoal.  See the
% file \texttt{HOL/ex/svc_test.ML} for examples.
% \begin{ttbox} 
% svc_tac   : int -> tactic
% Svc.trace : bool ref      \hfill{\bf initially false}
% \end{ttbox}

% \begin{ttdescription}
% \item[\ttindexbold{svc_tac} $i$] attempts to prove subgoal~$i$ by translating
%   it into a formula recognized by~SVC\@.  If it succeeds then the subgoal is
%   removed.  It fails if SVC is unable to prove the subgoal.  It crashes with
%   an error message if SVC appears not to be installed.  Numeric variables may
%   have types \texttt{nat}, \texttt{int} or \texttt{real}.

% \item[\ttindexbold{Svc.trace}] is a flag that, if set, causes \texttt{svc_tac}
%   to trace its operations: abstraction of the subgoal, translation to SVC
%   syntax, SVC's response.
% \end{ttdescription}

% Here is an example, with tracing turned on:
% \begin{ttbox}
% set Svc.trace;
% {\out val it : bool = true}
% Goal "(#3::nat)*a <= #2 + #4*b + #6*c  & #11 <= #2*a + b + #2*c & \ttback
% \ttback     a + #3*b <= #5 + #2*c  --> #2 + #3*b <= #2*a + #6*c";

% by (svc_tac 1);
% {\out Subgoal abstracted to}
% {\out #3 * a <= #2 + #4 * b + #6 * c &}
% {\out #11 <= #2 * a + b + #2 * c & a + #3 * b <= #5 + #2 * c -->}
% {\out #2 + #3 * b <= #2 * a + #6 * c}
% {\out Calling SVC:}
% {\out (=> (<= 0  (F_c) )  (=> (<= 0  (F_b) )  (=> (<= 0  (F_a) )}
% {\out   (=> (AND (<= {* 3  (F_a) }  {+ {+ 2  {* 4  (F_b) } }  }
% {\out {* 6  (F_c) } } )  (AND (<= 11  {+ {+ {* 2  (F_a) }  (F_b) }}
% {\out   {* 2  (F_c) } } )  (<= {+ (F_a)  {* 3  (F_b) } }  {+ 5  }
% {\out {* 2  (F_c) } } ) ) )  (< {+ 2  {* 3  (F_b) } }  {+ 1  {+ }
% {\out {* 2  (F_a) }  {* 6  (F_c) } } } ) ) ) ) ) }
% {\out SVC Returns:}
% {\out VALID}
% {\out Level 1}
% {\out #3 * a <= #2 + #4 * b + #6 * c &}
% {\out #11 <= #2 * a + b + #2 * c & a + #3 * b <= #5 + #2 * c -->}
% {\out #2 + #3 * b <= #2 * a + #6 * c}
% {\out No subgoals!}
% \end{ttbox}


% \begin{warn}
% Calling \ttindex{svc_tac} entails an above-average risk of
% unsoundness.  Isabelle does not check SVC's result independently.  Moreover,
% the tactic translates the submitted formula using code that lies outside
% Isabelle's inference core.  Theorems that depend upon results proved using SVC
% (and other oracles) are displayed with the annotation \texttt{[!]} attached.
% You can also use \texttt{\#der (rep_thm $th$)} to examine the proof object of
% theorem~$th$, as described in the \emph{Reference Manual}.  
% \end{warn}

% To start, first download SVC from the Internet at URL
% \begin{ttbox}
% http://agamemnon.stanford.edu/~levitt/vc/index.html
% \end{ttbox}
% and install it using the instructions supplied.  SVC requires two environment
% variables:
% \begin{ttdescription}
% \item[\ttindexbold{SVC_HOME}] is an absolute pathname to the SVC
%     distribution directory. 
    
%   \item[\ttindexbold{SVC_MACHINE}] identifies the type of computer and
%     operating system.  
% \end{ttdescription}
% You can set these environment variables either using the Unix shell or through
% an Isabelle \texttt{settings} file.  Isabelle assumes SVC to be installed if
% \texttt{SVC_HOME} is defined.

% \paragraph*{Acknowledgement.}  This interface uses code supplied by S{\o}ren
% Heilmann.


% \index{SVC decision procedure|)}




\section{Types}\label{sec:HOL:Types}
This section describes HOL's basic predefined types ($\alpha \times \beta$,
$\alpha + \beta$, $nat$ and $\alpha \; list$) and ways for introducing new
types in general.  The most important type construction, the
\texttt{datatype}, is treated separately in {\S}\ref{sec:HOL:datatype}.


\subsection{Product and sum types}\index{*"* type}\index{*"+ type}
\label{subsec:prod-sum}

\begin{figure}[htbp]
\begin{constants}
  \it symbol    & \it meta-type &           & \it description \\ 
  \cdx{Pair}    & $[\alpha,\beta]\To \alpha\times\beta$
        & & ordered pairs $(a,b)$ \\
  \cdx{fst}     & $\alpha\times\beta \To \alpha$        & & first projection\\
  \cdx{snd}     & $\alpha\times\beta \To \beta$         & & second projection\\
  \cdx{split}   & $[[\alpha,\beta]\To\gamma, \alpha\times\beta] \To \gamma$ 
        & & generalized projection\\
  \cdx{Sigma}  & 
        $[\alpha\,set, \alpha\To\beta\,set]\To(\alpha\times\beta)set$ &
        & general sum of sets
\end{constants}
%\tdx{fst_def}      fst p     == @a. ? b. p = (a,b)
%\tdx{snd_def}      snd p     == @b. ? a. p = (a,b)
%\tdx{split_def}    split c p == c (fst p) (snd p)
\begin{ttbox}\makeatletter
\tdx{Sigma_def}    Sigma A B == UN x:A. UN y:B x. {\ttlbrace}(x,y){\ttrbrace}

\tdx{Pair_eq}      ((a,b) = (a',b')) = (a=a' & b=b')
\tdx{Pair_inject}  [| (a, b) = (a',b');  [| a=a';  b=b' |] ==> R |] ==> R
\tdx{PairE}        [| !!x y. p = (x,y) ==> Q |] ==> Q

\tdx{fst_conv}     fst (a,b) = a
\tdx{snd_conv}     snd (a,b) = b
\tdx{surjective_pairing}  p = (fst p,snd p)

\tdx{split}        split c (a,b) = c a b
\tdx{split_split}  R(split c p) = (! x y. p = (x,y) --> R(c x y))

\tdx{SigmaI}    [| a:A;  b:B a |] ==> (a,b) : Sigma A B

\tdx{SigmaE}    [| c:Sigma A B; !!x y.[| x:A; y:B x; c=(x,y) |] ==> P 
          |] ==> P
\end{ttbox}
\caption{Type $\alpha\times\beta$}\label{hol-prod}
\end{figure} 

Theory \thydx{Prod} (Fig.\ts\ref{hol-prod}) defines the product type
$\alpha\times\beta$, with the ordered pair syntax $(a, b)$.  General
tuples are simulated by pairs nested to the right:
\begin{center}
\begin{tabular}{c|c}
external & internal \\
\hline
$\tau@1 \times \dots \times \tau@n$ & $\tau@1 \times (\dots (\tau@{n-1} \times \tau@n)\dots)$ \\
\hline
$(t@1,\dots,t@n)$ & $(t@1,(\dots,(t@{n-1},t@n)\dots)$ \\
\end{tabular}
\end{center}
In addition, it is possible to use tuples
as patterns in abstractions:
\begin{center}
{\tt\%($x$,$y$). $t$} \quad stands for\quad \texttt{split(\%$x$\thinspace$y$.\ $t$)} 
\end{center}
Nested patterns are also supported.  They are translated stepwise:
\begin{eqnarray*}
\hbox{\tt\%($x$,$y$,$z$).\ $t$} 
   & \leadsto & \hbox{\tt\%($x$,($y$,$z$)).\ $t$} \\
   & \leadsto & \hbox{\tt split(\%$x$.\%($y$,$z$).\ $t$)}\\
   & \leadsto & \hbox{\tt split(\%$x$.\ split(\%$y$ $z$.\ $t$))}
\end{eqnarray*}
The reverse translation is performed upon printing.
\begin{warn}
  The translation between patterns and \texttt{split} is performed automatically
  by the parser and printer.  Thus the internal and external form of a term
  may differ, which can affects proofs.  For example the term {\tt
  (\%(x,y).(y,x))(a,b)} requires the theorem \texttt{split} (which is in the
  default simpset) to rewrite to {\tt(b,a)}.
\end{warn}
In addition to explicit $\lambda$-abstractions, patterns can be used in any
variable binding construct which is internally described by a
$\lambda$-abstraction.  Some important examples are
\begin{description}
\item[Let:] \texttt{let {\it pattern} = $t$ in $u$}
\item[Quantifiers:] \texttt{ALL~{\it pattern}:$A$.~$P$}
\item[Choice:] {\underscoreon \tt SOME~{\it pattern}.~$P$}
\item[Set operations:] \texttt{UN~{\it pattern}:$A$.~$B$}
\item[Sets:] \texttt{{\ttlbrace}{\it pattern}.~$P${\ttrbrace}}
\end{description}

There is a simple tactic which supports reasoning about patterns:
\begin{ttdescription}
\item[\ttindexbold{split_all_tac} $i$] replaces in subgoal $i$ all
  {\tt!!}-quantified variables of product type by individual variables for
  each component.  A simple example:
\begin{ttbox}
{\out 1. !!p. (\%(x,y,z). (x, y, z)) p = p}
by(split_all_tac 1);
{\out 1. !!x xa ya. (\%(x,y,z). (x, y, z)) (x, xa, ya) = (x, xa, ya)}
\end{ttbox}
\end{ttdescription}

Theory \texttt{Prod} also introduces the degenerate product type \texttt{unit}
which contains only a single element named {\tt()} with the property
\begin{ttbox}
\tdx{unit_eq}       u = ()
\end{ttbox}
\bigskip

Theory \thydx{Sum} (Fig.~\ref{hol-sum}) defines the sum type $\alpha+\beta$
which associates to the right and has a lower priority than $*$: $\tau@1 +
\tau@2 + \tau@3*\tau@4$ means $\tau@1 + (\tau@2 + (\tau@3*\tau@4))$.

The definition of products and sums in terms of existing types is not
shown.  The constructions are fairly standard and can be found in the
respective theory files. Although the sum and product types are
constructed manually for foundational reasons, they are represented as
actual datatypes later.

\begin{figure}
\begin{constants}
  \it symbol    & \it meta-type &           & \it description \\ 
  \cdx{Inl}     & $\alpha \To \alpha+\beta$    & & first injection\\
  \cdx{Inr}     & $\beta \To \alpha+\beta$     & & second injection\\
  \cdx{sum_case} & $[\alpha\To\gamma, \beta\To\gamma, \alpha+\beta] \To\gamma$
        & & conditional
\end{constants}
\begin{ttbox}\makeatletter
\tdx{Inl_not_Inr}    Inl a ~= Inr b

\tdx{inj_Inl}        inj Inl
\tdx{inj_Inr}        inj Inr

\tdx{sumE}           [| !!x. P(Inl x);  !!y. P(Inr y) |] ==> P s

\tdx{sum_case_Inl}   sum_case f g (Inl x) = f x
\tdx{sum_case_Inr}   sum_case f g (Inr x) = g x

\tdx{surjective_sum} sum_case (\%x. f(Inl x)) (\%y. f(Inr y)) s = f s
\tdx{sum.split_case} R(sum_case f g s) = ((! x. s = Inl(x) --> R(f(x))) &
                                     (! y. s = Inr(y) --> R(g(y))))
\end{ttbox}
\caption{Type $\alpha+\beta$}\label{hol-sum}
\end{figure}

\begin{figure}
\index{*"< symbol}
\index{*"* symbol}
\index{*div symbol}
\index{*mod symbol}
\index{*dvd symbol}
\index{*"+ symbol}
\index{*"- symbol}
\begin{constants}
  \it symbol    & \it meta-type & \it priority & \it description \\ 
  \cdx{0}       & $\alpha$  & & zero \\
  \cdx{Suc}     & $nat \To nat$ & & successor function\\
  \tt *    & $[\alpha,\alpha]\To \alpha$    &  Left 70 & multiplication \\
  \tt div  & $[\alpha,\alpha]\To \alpha$    &  Left 70 & division\\
  \tt mod  & $[\alpha,\alpha]\To \alpha$    &  Left 70 & modulus\\
  \tt dvd  & $[\alpha,\alpha]\To bool$     &  Left 70 & ``divides'' relation\\
  \tt +    & $[\alpha,\alpha]\To \alpha$    &  Left 65 & addition\\
  \tt -    & $[\alpha,\alpha]\To \alpha$    &  Left 65 & subtraction
\end{constants}
\subcaption{Constants and infixes}

\begin{ttbox}\makeatother
\tdx{nat_induct}     [| P 0; !!n. P n ==> P(Suc n) |]  ==> P n

\tdx{Suc_not_Zero}   Suc m ~= 0
\tdx{inj_Suc}        inj Suc
\tdx{n_not_Suc_n}    n~=Suc n
\subcaption{Basic properties}
\end{ttbox}
\caption{The type of natural numbers, \tydx{nat}} \label{hol-nat1}
\end{figure}


\begin{figure}
\begin{ttbox}\makeatother
              0+n           = n
              (Suc m)+n     = Suc(m+n)

              m-0           = m
              0-n           = n
              Suc(m)-Suc(n) = m-n

              0*n           = 0
              Suc(m)*n      = n + m*n

\tdx{mod_less}      m<n ==> m mod n = m
\tdx{mod_geq}       [| 0<n;  ~m<n |] ==> m mod n = (m-n) mod n

\tdx{div_less}      m<n ==> m div n = 0
\tdx{div_geq}       [| 0<n;  ~m<n |] ==> m div n = Suc((m-n) div n)
\end{ttbox}
\caption{Recursion equations for the arithmetic operators} \label{hol-nat2}
\end{figure}

\subsection{The type of natural numbers, \textit{nat}}
\index{nat@{\textit{nat}} type|(}

The theory \thydx{Nat} defines the natural numbers in a roundabout but
traditional way.  The axiom of infinity postulates a type~\tydx{ind} of
individuals, which is non-empty and closed under an injective operation.  The
natural numbers are inductively generated by choosing an arbitrary individual
for~0 and using the injective operation to take successors.  This is a least
fixedpoint construction.  

Type~\tydx{nat} is an instance of class~\cldx{ord}, which makes the overloaded
functions of this class (especially \cdx{<} and \cdx{<=}, but also \cdx{min},
\cdx{max} and \cdx{LEAST}) available on \tydx{nat}.  Theory \thydx{Nat} 
also shows that {\tt<=} is a linear order, so \tydx{nat} is
also an instance of class \cldx{linorder}.

Theory \thydx{NatArith} develops arithmetic on the natural numbers.  It defines
addition, multiplication and subtraction.  Theory \thydx{Divides} defines
division, remainder and the ``divides'' relation.  The numerous theorems
proved include commutative, associative, distributive, identity and
cancellation laws.  See Figs.\ts\ref{hol-nat1} and~\ref{hol-nat2}.  The
recursion equations for the operators \texttt{+}, \texttt{-} and \texttt{*} on
\texttt{nat} are part of the default simpset.

Functions on \tydx{nat} can be defined by primitive or well-founded recursion;
see {\S}\ref{sec:HOL:recursive}.  A simple example is addition.
Here, \texttt{op +} is the name of the infix operator~\texttt{+}, following
the standard convention.
\begin{ttbox}
\sdx{primrec}
      "0 + n = n"
  "Suc m + n = Suc (m + n)"
\end{ttbox}
There is also a \sdx{case}-construct
of the form
\begin{ttbox}
case \(e\) of 0 => \(a\) | Suc \(m\) => \(b\)
\end{ttbox}
Note that Isabelle insists on precisely this format; you may not even change
the order of the two cases.
Both \texttt{primrec} and \texttt{case} are realized by a recursion operator
\cdx{nat_rec}, which is available because \textit{nat} is represented as
a datatype.

%The predecessor relation, \cdx{pred_nat}, is shown to be well-founded.
%Recursion along this relation resembles primitive recursion, but is
%stronger because we are in higher-order logic; using primitive recursion to
%define a higher-order function, we can easily Ackermann's function, which
%is not primitive recursive \cite[page~104]{thompson91}.
%The transitive closure of \cdx{pred_nat} is~$<$.  Many functions on the
%natural numbers are most easily expressed using recursion along~$<$.

Tactic {\tt\ttindex{induct_tac} "$n$" $i$} performs induction on variable~$n$
in subgoal~$i$ using theorem \texttt{nat_induct}.  There is also the derived
theorem \tdx{less_induct}:
\begin{ttbox}
[| !!n. [| ! m. m<n --> P m |] ==> P n |]  ==>  P n
\end{ttbox}


\subsection{Numerical types and numerical reasoning}

The integers (type \tdx{int}) are also available in HOL, and the reals (type
\tdx{real}) are available in the logic image \texttt{HOL-Complex}.  They support
the expected operations of addition (\texttt{+}), subtraction (\texttt{-}) and
multiplication (\texttt{*}), and much else.  Type \tdx{int} provides the
\texttt{div} and \texttt{mod} operators, while type \tdx{real} provides real
division and other operations.  Both types belong to class \cldx{linorder}, so
they inherit the relational operators and all the usual properties of linear
orderings.  For full details, please survey the theories in subdirectories
\texttt{Integ}, \texttt{Real}, and \texttt{Complex}.

All three numeric types admit numerals of the form \texttt{$sd\ldots d$},
where $s$ is an optional minus sign and $d\ldots d$ is a string of digits.
Numerals are represented internally by a datatype for binary notation, which
allows numerical calculations to be performed by rewriting.  For example, the
integer division of \texttt{54342339} by \texttt{3452} takes about five
seconds.  By default, the simplifier cancels like terms on the opposite sites
of relational operators (reducing \texttt{z+x<x+y} to \texttt{z<y}, for
instance.  The simplifier also collects like terms, replacing \texttt{x+y+x*3}
by \texttt{4*x+y}.

Sometimes numerals are not wanted, because for example \texttt{n+3} does not
match a pattern of the form \texttt{Suc $k$}.  You can re-arrange the form of
an arithmetic expression by proving (via \texttt{subgoal_tac}) a lemma such as
\texttt{n+3 = Suc (Suc (Suc n))}.  As an alternative, you can disable the
fancier simplifications by using a basic simpset such as \texttt{HOL_ss}
rather than the default one, \texttt{simpset()}.

Reasoning about arithmetic inequalities can be tedious.  Fortunately, HOL
provides a decision procedure for \emph{linear arithmetic}: formulae involving
addition and subtraction. The simplifier invokes a weak version of this
decision procedure automatically. If this is not sufficent, you can invoke the
full procedure \ttindex{Lin_Arith.tac} explicitly.  It copes with arbitrary
formulae involving {\tt=}, {\tt<}, {\tt<=}, {\tt+}, {\tt-}, {\tt Suc}, {\tt
  min}, {\tt max} and numerical constants. Other subterms are treated as
atomic, while subformulae not involving numerical types are ignored. Quantified
subformulae are ignored unless they are positive universal or negative
existential. The running time is exponential in the number of
occurrences of {\tt min}, {\tt max}, and {\tt-} because they require case
distinctions.
If {\tt k} is a numeral, then {\tt div k}, {\tt mod k} and
{\tt k dvd} are also supported. The former two are eliminated
by case distinctions, again blowing up the running time.
If the formula involves explicit quantifiers, \texttt{Lin_Arith.tac} may take
super-exponential time and space.

If \texttt{Lin_Arith.tac} fails, try to find relevant arithmetic results in
the library.  The theories \texttt{Nat} and \texttt{NatArith} contain
theorems about {\tt<}, {\tt<=}, \texttt{+}, \texttt{-} and \texttt{*}.
Theory \texttt{Divides} contains theorems about \texttt{div} and
\texttt{mod}.  Use Proof General's \emph{find} button (or other search
facilities) to locate them.

\index{nat@{\textit{nat}} type|)}


\begin{figure}
\index{#@{\tt[]} symbol}
\index{#@{\tt\#} symbol}
\index{"@@{\tt\at} symbol}
\index{*"! symbol}
\begin{constants}
  \it symbol & \it meta-type & \it priority & \it description \\
  \tt[]    & $\alpha\,list$ & & empty list\\
  \tt \#   & $[\alpha,\alpha\,list]\To \alpha\,list$ & Right 65 & 
        list constructor \\
  \cdx{null}    & $\alpha\,list \To bool$ & & emptiness test\\
  \cdx{hd}      & $\alpha\,list \To \alpha$ & & head \\
  \cdx{tl}      & $\alpha\,list \To \alpha\,list$ & & tail \\
  \cdx{last}    & $\alpha\,list \To \alpha$ & & last element \\
  \cdx{butlast} & $\alpha\,list \To \alpha\,list$ & & drop last element \\
  \tt\at  & $[\alpha\,list,\alpha\,list]\To \alpha\,list$ & Left 65 & append \\
  \cdx{map}     & $(\alpha\To\beta) \To (\alpha\,list \To \beta\,list)$
        & & apply to all\\
  \cdx{filter}  & $(\alpha \To bool) \To (\alpha\,list \To \alpha\,list)$
        & & filter functional\\
  \cdx{set}& $\alpha\,list \To \alpha\,set$ & & elements\\
  \sdx{mem}  & $\alpha \To \alpha\,list \To bool$  &  Left 55   & membership\\
  \cdx{foldl}   & $(\beta\To\alpha\To\beta) \To \beta \To \alpha\,list \To \beta$ &
  & iteration \\
  \cdx{concat}   & $(\alpha\,list)list\To \alpha\,list$ & & concatenation \\
  \cdx{rev}     & $\alpha\,list \To \alpha\,list$ & & reverse \\
  \cdx{length}  & $\alpha\,list \To nat$ & & length \\
  \tt! & $\alpha\,list \To nat \To \alpha$ & Left 100 & indexing \\
  \cdx{take}, \cdx{drop} & $nat \To \alpha\,list \To \alpha\,list$ &&
    take/drop a prefix \\
  \cdx{takeWhile},\\
  \cdx{dropWhile} &
    $(\alpha \To bool) \To \alpha\,list \To \alpha\,list$ &&
    take/drop a prefix
\end{constants}
\subcaption{Constants and infixes}

\begin{center} \tt\frenchspacing
\begin{tabular}{rrr} 
  \it external        & \it internal  & \it description \\{}
  [$x@1$, $\dots$, $x@n$]  &  $x@1$ \# $\cdots$ \# $x@n$ \# [] &
        \rm finite list \\{}
  [$x$:$l$. $P$]  & filter ($\lambda x{.}P$) $l$ & 
        \rm list comprehension
\end{tabular}
\end{center}
\subcaption{Translations}
\caption{The theory \thydx{List}} \label{hol-list}
\end{figure}


\begin{figure}
\begin{ttbox}\makeatother
null [] = True
null (x#xs) = False

hd (x#xs) = x

tl (x#xs) = xs
tl [] = []

[] @ ys = ys
(x#xs) @ ys = x # xs @ ys

set [] = \ttlbrace\ttrbrace
set (x#xs) = insert x (set xs)

x mem [] = False
x mem (y#ys) = (if y=x then True else x mem ys)

concat([]) = []
concat(x#xs) = x @ concat(xs)

rev([]) = []
rev(x#xs) = rev(xs) @ [x]

length([]) = 0
length(x#xs) = Suc(length(xs))

xs!0 = hd xs
xs!(Suc n) = (tl xs)!n
\end{ttbox}
\caption{Simple list processing functions}
\label{fig:HOL:list-simps}
\end{figure}

\begin{figure}
\begin{ttbox}\makeatother
map f [] = []
map f (x#xs) = f x # map f xs

filter P [] = []
filter P (x#xs) = (if P x then x#filter P xs else filter P xs)

foldl f a [] = a
foldl f a (x#xs) = foldl f (f a x) xs

take n [] = []
take n (x#xs) = (case n of 0 => [] | Suc(m) => x # take m xs)

drop n [] = []
drop n (x#xs) = (case n of 0 => x#xs | Suc(m) => drop m xs)

takeWhile P [] = []
takeWhile P (x#xs) = (if P x then x#takeWhile P xs else [])

dropWhile P [] = []
dropWhile P (x#xs) = (if P x then dropWhile P xs else xs)
\end{ttbox}
\caption{Further list processing functions}
\label{fig:HOL:list-simps2}
\end{figure}


\subsection{The type constructor for lists, \textit{list}}
\label{subsec:list}
\index{list@{\textit{list}} type|(}

Figure~\ref{hol-list} presents the theory \thydx{List}: the basic list
operations with their types and syntax.  Type $\alpha \; list$ is
defined as a \texttt{datatype} with the constructors {\tt[]} and {\tt\#}.
As a result the generic structural induction and case analysis tactics
\texttt{induct\_tac} and \texttt{cases\_tac} also become available for
lists.  A \sdx{case} construct of the form
\begin{center}\tt
case $e$ of [] => $a$  |  \(x\)\#\(xs\) => b
\end{center}
is defined by translation.  For details see~{\S}\ref{sec:HOL:datatype}. There
is also a case splitting rule \tdx{split_list_case}
\[
\begin{array}{l}
P(\mathtt{case}~e~\mathtt{of}~\texttt{[] =>}~a ~\texttt{|}~
               x\texttt{\#}xs~\texttt{=>}~f~x~xs) ~= \\
((e = \texttt{[]} \to P(a)) \land
 (\forall x~ xs. e = x\texttt{\#}xs \to P(f~x~xs)))
\end{array}
\]
which can be fed to \ttindex{addsplits} just like
\texttt{split_if} (see~{\S}\ref{subsec:HOL:case:splitting}).

\texttt{List} provides a basic library of list processing functions defined by
primitive recursion (see~{\S}\ref{sec:HOL:primrec}).  The recursion equations
are shown in Figs.\ts\ref{fig:HOL:list-simps} and~\ref{fig:HOL:list-simps2}.

\index{list@{\textit{list}} type|)}


\section{Datatype definitions}
\label{sec:HOL:datatype}
\index{*datatype|(}

Inductive datatypes, similar to those of \ML, frequently appear in
applications of Isabelle/HOL.  In principle, such types could be defined by
hand via \texttt{typedef}, but this would be far too
tedious.  The \ttindex{datatype} definition package of Isabelle/HOL (cf.\ 
\cite{Berghofer-Wenzel:1999:TPHOL}) automates such chores.  It generates an
appropriate \texttt{typedef} based on a least fixed-point construction, and
proves freeness theorems and induction rules, as well as theorems for
recursion and case combinators.  The user just has to give a simple
specification of new inductive types using a notation similar to {\ML} or
Haskell.

The current datatype package can handle both mutual and indirect recursion.
It also offers to represent existing types as datatypes giving the advantage
of a more uniform view on standard theories.


\subsection{Basics}
\label{subsec:datatype:basics}

A general \texttt{datatype} definition is of the following form:
\[
\begin{array}{llcl}
\mathtt{datatype} & (\vec{\alpha})t@1 & = &
  C^1@1~\tau^1@{1,1}~\ldots~\tau^1@{1,m^1@1} ~\mid~ \ldots ~\mid~
    C^1@{k@1}~\tau^1@{k@1,1}~\ldots~\tau^1@{k@1,m^1@{k@1}} \\
 & & \vdots \\
\mathtt{and} & (\vec{\alpha})t@n & = &
  C^n@1~\tau^n@{1,1}~\ldots~\tau^n@{1,m^n@1} ~\mid~ \ldots ~\mid~
    C^n@{k@n}~\tau^n@{k@n,1}~\ldots~\tau^n@{k@n,m^n@{k@n}}
\end{array}
\]
where $\vec{\alpha} = (\alpha@1,\ldots,\alpha@h)$ is a list of type variables,
$C^j@i$ are distinct constructor names and $\tau^j@{i,i'}$ are {\em
  admissible} types containing at most the type variables $\alpha@1, \ldots,
\alpha@h$. A type $\tau$ occurring in a \texttt{datatype} definition is {\em
  admissible} if and only if
\begin{itemize}
\item $\tau$ is non-recursive, i.e.\ $\tau$ does not contain any of the
newly defined type constructors $t@1,\ldots,t@n$, or
\item $\tau = (\vec{\alpha})t@{j'}$ where $1 \leq j' \leq n$, or
\item $\tau = (\tau'@1,\ldots,\tau'@{h'})t'$, where $t'$ is
the type constructor of an already existing datatype and $\tau'@1,\ldots,\tau'@{h'}$
are admissible types.
\item $\tau = \sigma \to \tau'$, where $\tau'$ is an admissible
type and $\sigma$ is non-recursive (i.e. the occurrences of the newly defined
types are {\em strictly positive})
\end{itemize}
If some $(\vec{\alpha})t@{j'}$ occurs in a type $\tau^j@{i,i'}$
of the form
\[
(\ldots,\ldots ~ (\vec{\alpha})t@{j'} ~ \ldots,\ldots)t'
\]
this is called a {\em nested} (or \emph{indirect}) occurrence. A very simple
example of a datatype is the type \texttt{list}, which can be defined by
\begin{ttbox}
datatype 'a list = Nil
                 | Cons 'a ('a list)
\end{ttbox}
Arithmetic expressions \texttt{aexp} and boolean expressions \texttt{bexp} can be modelled
by the mutually recursive datatype definition
\begin{ttbox}
datatype 'a aexp = If_then_else ('a bexp) ('a aexp) ('a aexp)
                 | Sum ('a aexp) ('a aexp)
                 | Diff ('a aexp) ('a aexp)
                 | Var 'a
                 | Num nat
and      'a bexp = Less ('a aexp) ('a aexp)
                 | And ('a bexp) ('a bexp)
                 | Or ('a bexp) ('a bexp)
\end{ttbox}
The datatype \texttt{term}, which is defined by
\begin{ttbox}
datatype ('a, 'b) term = Var 'a
                       | App 'b ((('a, 'b) term) list)
\end{ttbox}
is an example for a datatype with nested recursion. Using nested recursion
involving function spaces, we may also define infinitely branching datatypes, e.g.
\begin{ttbox}
datatype 'a tree = Atom 'a | Branch "nat => 'a tree"
\end{ttbox}

\medskip

Types in HOL must be non-empty. Each of the new datatypes
$(\vec{\alpha})t@j$ with $1 \leq j \leq n$ is non-empty if and only if it has a
constructor $C^j@i$ with the following property: for all argument types
$\tau^j@{i,i'}$ of the form $(\vec{\alpha})t@{j'}$ the datatype
$(\vec{\alpha})t@{j'}$ is non-empty.

If there are no nested occurrences of the newly defined datatypes, obviously
at least one of the newly defined datatypes $(\vec{\alpha})t@j$
must have a constructor $C^j@i$ without recursive arguments, a \emph{base
  case}, to ensure that the new types are non-empty. If there are nested
occurrences, a datatype can even be non-empty without having a base case
itself. Since \texttt{list} is a non-empty datatype, \texttt{datatype t = C (t
  list)} is non-empty as well.


\subsubsection{Freeness of the constructors}

The datatype constructors are automatically defined as functions of their
respective type:
\[ C^j@i :: [\tau^j@{i,1},\dots,\tau^j@{i,m^j@i}] \To (\alpha@1,\dots,\alpha@h)t@j \]
These functions have certain {\em freeness} properties.  They construct
distinct values:
\[
C^j@i~x@1~\dots~x@{m^j@i} \neq C^j@{i'}~y@1~\dots~y@{m^j@{i'}} \qquad
\mbox{for all}~ i \neq i'.
\]
The constructor functions are injective:
\[
(C^j@i~x@1~\dots~x@{m^j@i} = C^j@i~y@1~\dots~y@{m^j@i}) =
(x@1 = y@1 \land \dots \land x@{m^j@i} = y@{m^j@i})
\]
Since the number of distinctness inequalities is quadratic in the number of
constructors, the datatype package avoids proving them separately if there are
too many constructors. Instead, specific inequalities are proved by a suitable
simplification procedure on demand.\footnote{This procedure, which is already part
of the default simpset, may be referred to by the ML identifier
\texttt{DatatypePackage.distinct_simproc}.}

\subsubsection{Structural induction}

The datatype package also provides structural induction rules.  For
datatypes without nested recursion, this is of the following form:
\[
\infer{P@1~x@1 \land \dots \land P@n~x@n}
  {\begin{array}{lcl}
     \Forall x@1 \dots x@{m^1@1}.
       \List{P@{s^1@{1,1}}~x@{r^1@{1,1}}; \dots;
         P@{s^1@{1,l^1@1}}~x@{r^1@{1,l^1@1}}} & \Imp &
           P@1~\left(C^1@1~x@1~\dots~x@{m^1@1}\right) \\
     & \vdots \\
     \Forall x@1 \dots x@{m^1@{k@1}}.
       \List{P@{s^1@{k@1,1}}~x@{r^1@{k@1,1}}; \dots;
         P@{s^1@{k@1,l^1@{k@1}}}~x@{r^1@{k@1,l^1@{k@1}}}} & \Imp &
           P@1~\left(C^1@{k@1}~x@1~\ldots~x@{m^1@{k@1}}\right) \\
     & \vdots \\
     \Forall x@1 \dots x@{m^n@1}.
       \List{P@{s^n@{1,1}}~x@{r^n@{1,1}}; \dots;
         P@{s^n@{1,l^n@1}}~x@{r^n@{1,l^n@1}}} & \Imp &
           P@n~\left(C^n@1~x@1~\ldots~x@{m^n@1}\right) \\
     & \vdots \\
     \Forall x@1 \dots x@{m^n@{k@n}}.
       \List{P@{s^n@{k@n,1}}~x@{r^n@{k@n,1}}; \dots
         P@{s^n@{k@n,l^n@{k@n}}}~x@{r^n@{k@n,l^n@{k@n}}}} & \Imp &
           P@n~\left(C^n@{k@n}~x@1~\ldots~x@{m^n@{k@n}}\right)
   \end{array}}
\]
where
\[
\begin{array}{rcl}
Rec^j@i & := &
   \left\{\left(r^j@{i,1},s^j@{i,1}\right),\ldots,
     \left(r^j@{i,l^j@i},s^j@{i,l^j@i}\right)\right\} = \\[2ex]
&& \left\{(i',i'')~\left|~
     1\leq i' \leq m^j@i \land 1 \leq i'' \leq n \land
       \tau^j@{i,i'} = (\alpha@1,\ldots,\alpha@h)t@{i''}\right.\right\}
\end{array}
\]
i.e.\ the properties $P@j$ can be assumed for all recursive arguments.

For datatypes with nested recursion, such as the \texttt{term} example from
above, things are a bit more complicated.  Conceptually, Isabelle/HOL unfolds
a definition like
\begin{ttbox}
datatype ('a,'b) term = Var 'a
                      | App 'b ((('a, 'b) term) list)
\end{ttbox}
to an equivalent definition without nesting:
\begin{ttbox}
datatype ('a,'b) term      = Var
                           | App 'b (('a, 'b) term_list)
and      ('a,'b) term_list = Nil'
                           | Cons' (('a,'b) term) (('a,'b) term_list)
\end{ttbox}
Note however, that the type \texttt{('a,'b) term_list} and the constructors {\tt
  Nil'} and \texttt{Cons'} are not really introduced.  One can directly work with
the original (isomorphic) type \texttt{(('a, 'b) term) list} and its existing
constructors \texttt{Nil} and \texttt{Cons}. Thus, the structural induction rule for
\texttt{term} gets the form
\[
\infer{P@1~x@1 \land P@2~x@2}
  {\begin{array}{l}
     \Forall x.~P@1~(\mathtt{Var}~x) \\
     \Forall x@1~x@2.~P@2~x@2 \Imp P@1~(\mathtt{App}~x@1~x@2) \\
     P@2~\mathtt{Nil} \\
     \Forall x@1~x@2. \List{P@1~x@1; P@2~x@2} \Imp P@2~(\mathtt{Cons}~x@1~x@2)
   \end{array}}
\]
Note that there are two predicates $P@1$ and $P@2$, one for the type \texttt{('a,'b) term}
and one for the type \texttt{(('a, 'b) term) list}.

For a datatype with function types such as \texttt{'a tree}, the induction rule
is of the form
\[
\infer{P~t}
  {\Forall a.~P~(\mathtt{Atom}~a) &
   \Forall ts.~(\forall x.~P~(ts~x)) \Imp P~(\mathtt{Branch}~ts)}
\]

\medskip In principle, inductive types are already fully determined by
freeness and structural induction.  For convenience in applications,
the following derived constructions are automatically provided for any
datatype.

\subsubsection{The \sdx{case} construct}

The type comes with an \ML-like \texttt{case}-construct:
\[
\begin{array}{rrcl}
\mbox{\tt case}~e~\mbox{\tt of} & C^j@1~x@{1,1}~\dots~x@{1,m^j@1} & \To & e@1 \\
                           \vdots \\
                           \mid & C^j@{k@j}~x@{k@j,1}~\dots~x@{k@j,m^j@{k@j}} & \To & e@{k@j}
\end{array}
\]
where the $x@{i,j}$ are either identifiers or nested tuple patterns as in
{\S}\ref{subsec:prod-sum}.
\begin{warn}
  All constructors must be present, their order is fixed, and nested patterns
  are not supported (with the exception of tuples).  Violating this
  restriction results in strange error messages.
\end{warn}

To perform case distinction on a goal containing a \texttt{case}-construct,
the theorem $t@j.$\texttt{split} is provided:
\[
\begin{array}{@{}rcl@{}}
P(t@j_\mathtt{case}~f@1~\dots~f@{k@j}~e) &\!\!\!=&
\!\!\! ((\forall x@1 \dots x@{m^j@1}. e = C^j@1~x@1\dots x@{m^j@1} \to
                             P(f@1~x@1\dots x@{m^j@1})) \\
&&\!\!\! ~\land~ \dots ~\land \\
&&~\!\!\! (\forall x@1 \dots x@{m^j@{k@j}}. e = C^j@{k@j}~x@1\dots x@{m^j@{k@j}} \to
                             P(f@{k@j}~x@1\dots x@{m^j@{k@j}})))
\end{array}
\]
where $t@j$\texttt{_case} is the internal name of the \texttt{case}-construct.
This theorem can be added to a simpset via \ttindex{addsplits}
(see~{\S}\ref{subsec:HOL:case:splitting}).

Case splitting on assumption works as well, by using the rule
$t@j.$\texttt{split_asm} in the same manner.  Both rules are available under
$t@j.$\texttt{splits} (this name is \emph{not} bound in ML, though).

\begin{warn}\index{simplification!of \texttt{case}}%
  By default only the selector expression ($e$ above) in a
  \texttt{case}-construct is simplified, in analogy with \texttt{if} (see
  page~\pageref{if-simp}). Only if that reduces to a constructor is one of
  the arms of the \texttt{case}-construct exposed and simplified. To ensure
  full simplification of all parts of a \texttt{case}-construct for datatype
  $t$, remove $t$\texttt{.}\ttindexbold{case_weak_cong} from the simpset, for
  example by \texttt{delcongs [thm "$t$.weak_case_cong"]}.
\end{warn}

\subsubsection{The function \cdx{size}}\label{sec:HOL:size}

Theory \texttt{NatArith} declares a generic function \texttt{size} of type
$\alpha\To nat$.  Each datatype defines a particular instance of \texttt{size}
by overloading according to the following scheme:
%%% FIXME: This formula is too big and is completely unreadable
\[
size(C^j@i~x@1~\dots~x@{m^j@i}) = \!
\left\{
\begin{array}{ll}
0 & \!\mbox{if $Rec^j@i = \emptyset$} \\
1+\sum\limits@{h=1}^{l^j@i}size~x@{r^j@{i,h}} &
 \!\mbox{if $Rec^j@i = \left\{\left(r^j@{i,1},s^j@{i,1}\right),\ldots,
  \left(r^j@{i,l^j@i},s^j@{i,l^j@i}\right)\right\}$}
\end{array}
\right.
\]
where $Rec^j@i$ is defined above.  Viewing datatypes as generalised trees, the
size of a leaf is 0 and the size of a node is the sum of the sizes of its
subtrees ${}+1$.

\subsection{Defining datatypes}

The theory syntax for datatype definitions is given in the
Isabelle/Isar reference manual.  In order to be well-formed, a
datatype definition has to obey the rules stated in the previous
section.  As a result the theory is extended with the new types, the
constructors, and the theorems listed in the previous section.

Most of the theorems about datatypes become part of the default simpset and
you never need to see them again because the simplifier applies them
automatically.  Only induction or case distinction are usually invoked by hand.
\begin{ttdescription}
\item[\ttindexbold{induct_tac} {\tt"}$x${\tt"} $i$]
 applies structural induction on variable $x$ to subgoal $i$, provided the
 type of $x$ is a datatype.
\item[\texttt{induct_tac}
  {\tt"}$x@1$ $\ldots$ $x@n${\tt"} $i$] applies simultaneous
  structural induction on the variables $x@1,\ldots,x@n$ to subgoal $i$.  This
  is the canonical way to prove properties of mutually recursive datatypes
  such as \texttt{aexp} and \texttt{bexp}, or datatypes with nested recursion such as
  \texttt{term}.
\end{ttdescription}
In some cases, induction is overkill and a case distinction over all
constructors of the datatype suffices.
\begin{ttdescription}
\item[\ttindexbold{case_tac} {\tt"}$u${\tt"} $i$]
 performs a case analysis for the term $u$ whose type  must be a datatype.
 If the datatype has $k@j$ constructors  $C^j@1$, \dots $C^j@{k@j}$, subgoal
 $i$ is replaced by $k@j$ new subgoals which  contain the additional
 assumption $u = C^j@{i'}~x@1~\dots~x@{m^j@{i'}}$ for  $i'=1$, $\dots$,~$k@j$.
\end{ttdescription}

Note that induction is only allowed on free variables that should not occur
among the premises of the subgoal. Case distinction applies to arbitrary terms.

\bigskip


For the technically minded, we exhibit some more details.  Processing the
theory file produces an \ML\ structure which, in addition to the usual
components, contains a structure named $t$ for each datatype $t$ defined in
the file.  Each structure $t$ contains the following elements:
\begin{ttbox}
val distinct : thm list
val inject : thm list
val induct : thm
val exhaust : thm
val cases : thm list
val split : thm
val split_asm : thm
val recs : thm list
val size : thm list
val simps : thm list
\end{ttbox}
\texttt{distinct}, \texttt{inject}, \texttt{induct}, \texttt{size}
and \texttt{split} contain the theorems
described above.  For user convenience, \texttt{distinct} contains
inequalities in both directions.  The reduction rules of the {\tt
  case}-construct are in \texttt{cases}.  All theorems from {\tt
  distinct}, \texttt{inject} and \texttt{cases} are combined in \texttt{simps}.
In case of mutually recursive datatypes, \texttt{recs}, \texttt{size}, \texttt{induct}
and \texttt{simps} are contained in a separate structure named $t@1_\ldots_t@n$.


\subsection{Examples}

\subsubsection{The datatype $\alpha~mylist$}

We want to define a type $\alpha~mylist$. To do this we have to build a new
theory that contains the type definition.  We start from the theory
\texttt{Datatype} instead of \texttt{Main} in order to avoid clashes with the
\texttt{List} theory of Isabelle/HOL.
\begin{ttbox}
MyList = Datatype +
  datatype 'a mylist = Nil | Cons 'a ('a mylist)
end
\end{ttbox}
After loading the theory, we can prove $Cons~x~xs\neq xs$, for example.  To
ease the induction applied below, we state the goal with $x$ quantified at the
object-level.  This will be stripped later using \ttindex{qed_spec_mp}.
\begin{ttbox}
Goal "!x. Cons x xs ~= xs";
{\out Level 0}
{\out ! x. Cons x xs ~= xs}
{\out  1. ! x. Cons x xs ~= xs}
\end{ttbox}
This can be proved by the structural induction tactic:
\begin{ttbox}
by (induct_tac "xs" 1);
{\out Level 1}
{\out ! x. Cons x xs ~= xs}
{\out  1. ! x. Cons x Nil ~= Nil}
{\out  2. !!a mylist.}
{\out        ! x. Cons x mylist ~= mylist ==>}
{\out        ! x. Cons x (Cons a mylist) ~= Cons a mylist}
\end{ttbox}
The first subgoal can be proved using the simplifier.  Isabelle/HOL has
already added the freeness properties of lists to the default simplification
set.
\begin{ttbox}
by (Simp_tac 1);
{\out Level 2}
{\out ! x. Cons x xs ~= xs}
{\out  1. !!a mylist.}
{\out        ! x. Cons x mylist ~= mylist ==>}
{\out        ! x. Cons x (Cons a mylist) ~= Cons a mylist}
\end{ttbox}
Similarly, we prove the remaining goal.
\begin{ttbox}
by (Asm_simp_tac 1);
{\out Level 3}
{\out ! x. Cons x xs ~= xs}
{\out No subgoals!}
\ttbreak
qed_spec_mp "not_Cons_self";
{\out val not_Cons_self = "Cons x xs ~= xs" : thm}
\end{ttbox}
Because both subgoals could have been proved by \texttt{Asm_simp_tac}
we could have done that in one step:
\begin{ttbox}
by (ALLGOALS Asm_simp_tac);
\end{ttbox}


\subsubsection{The datatype $\alpha~mylist$ with mixfix syntax}

In this example we define the type $\alpha~mylist$ again but this time
we want to write \texttt{[]} for \texttt{Nil} and we want to use infix
notation \verb|#| for \texttt{Cons}.  To do this we simply add mixfix
annotations after the constructor declarations as follows:
\begin{ttbox}
MyList = Datatype +
  datatype 'a mylist =
    Nil ("[]")  |
    Cons 'a ('a mylist)  (infixr "#" 70)
end
\end{ttbox}
Now the theorem in the previous example can be written \verb|x#xs ~= xs|.


\subsubsection{A datatype for weekdays}

This example shows a datatype that consists of 7 constructors:
\begin{ttbox}
Days = Main +
  datatype days = Mon | Tue | Wed | Thu | Fri | Sat | Sun
end
\end{ttbox}
Because there are more than 6 constructors, inequality is expressed via a function
\verb|days_ord|.  The theorem \verb|Mon ~= Tue| is not directly
contained among the distinctness theorems, but the simplifier can
prove it thanks to rewrite rules inherited from theory \texttt{NatArith}:
\begin{ttbox}
Goal "Mon ~= Tue";
by (Simp_tac 1);
\end{ttbox}
You need not derive such inequalities explicitly: the simplifier will dispose
of them automatically.
\index{*datatype|)}


\section{Recursive function definitions}\label{sec:HOL:recursive}
\index{recursive functions|see{recursion}}

Isabelle/HOL provides two main mechanisms of defining recursive functions.
\begin{enumerate}
\item \textbf{Primitive recursion} is available only for datatypes, and it is
  somewhat restrictive.  Recursive calls are only allowed on the argument's
  immediate constituents.  On the other hand, it is the form of recursion most
  often wanted, and it is easy to use.
  
\item \textbf{Well-founded recursion} requires that you supply a well-founded
  relation that governs the recursion.  Recursive calls are only allowed if
  they make the argument decrease under the relation.  Complicated recursion
  forms, such as nested recursion, can be dealt with.  Termination can even be
  proved at a later time, though having unsolved termination conditions around
  can make work difficult.%
  \footnote{This facility is based on Konrad Slind's TFL
    package~\cite{slind-tfl}.  Thanks are due to Konrad for implementing TFL
    and assisting with its installation.}
\end{enumerate}

Following good HOL tradition, these declarations do not assert arbitrary
axioms.  Instead, they define the function using a recursion operator.  Both
HOL and ZF derive the theory of well-founded recursion from first
principles~\cite{paulson-set-II}.  Primitive recursion over some datatype
relies on the recursion operator provided by the datatype package.  With
either form of function definition, Isabelle proves the desired recursion
equations as theorems.


\subsection{Primitive recursive functions}
\label{sec:HOL:primrec}
\index{recursion!primitive|(}
\index{*primrec|(}

Datatypes come with a uniform way of defining functions, {\bf primitive
  recursion}.  In principle, one could introduce primitive recursive functions
by asserting their reduction rules as new axioms, but this is not recommended:
\begin{ttbox}\slshape
Append = Main +
consts app :: ['a list, 'a list] => 'a list
rules 
   app_Nil   "app [] ys = ys"
   app_Cons  "app (x#xs) ys = x#app xs ys"
end
\end{ttbox}
Asserting axioms brings the danger of accidentally asserting nonsense, as
in \verb$app [] ys = us$.

The \ttindex{primrec} declaration is a safe means of defining primitive
recursive functions on datatypes:
\begin{ttbox}
Append = Main +
consts app :: ['a list, 'a list] => 'a list
primrec
   "app [] ys = ys"
   "app (x#xs) ys = x#app xs ys"
end
\end{ttbox}
Isabelle will now check that the two rules do indeed form a primitive
recursive definition.  For example
\begin{ttbox}
primrec
    "app [] ys = us"
\end{ttbox}
is rejected with an error message ``\texttt{Extra variables on rhs}''.

\bigskip

The general form of a primitive recursive definition is
\begin{ttbox}
primrec
    {\it reduction rules}
\end{ttbox}
where \textit{reduction rules} specify one or more equations of the form
\[ f \, x@1 \, \dots \, x@m \, (C \, y@1 \, \dots \, y@k) \, z@1 \,
\dots \, z@n = r \] such that $C$ is a constructor of the datatype, $r$
contains only the free variables on the left-hand side, and all recursive
calls in $r$ are of the form $f \, \dots \, y@i \, \dots$ for some $i$.  There
must be at most one reduction rule for each constructor.  The order is
immaterial.  For missing constructors, the function is defined to return a
default value.  

If you would like to refer to some rule by name, then you must prefix
the rule with an identifier.  These identifiers, like those in the
\texttt{rules} section of a theory, will be visible at the \ML\ level.

The primitive recursive function can have infix or mixfix syntax:
\begin{ttbox}\underscoreon
consts "@"  :: ['a list, 'a list] => 'a list  (infixr 60)
primrec
   "[] @ ys = ys"
   "(x#xs) @ ys = x#(xs @ ys)"
\end{ttbox}

The reduction rules become part of the default simpset, which
leads to short proof scripts:
\begin{ttbox}\underscoreon
Goal "(xs @ ys) @ zs = xs @ (ys @ zs)";
by (induct\_tac "xs" 1);
by (ALLGOALS Asm\_simp\_tac);
\end{ttbox}

\subsubsection{Example: Evaluation of expressions}
Using mutual primitive recursion, we can define evaluation functions \texttt{evala}
and \texttt{eval_bexp} for the datatypes of arithmetic and boolean expressions mentioned in
{\S}\ref{subsec:datatype:basics}:
\begin{ttbox}
consts
  evala :: "['a => nat, 'a aexp] => nat"
  evalb :: "['a => nat, 'a bexp] => bool"

primrec
  "evala env (If_then_else b a1 a2) =
     (if evalb env b then evala env a1 else evala env a2)"
  "evala env (Sum a1 a2) = evala env a1 + evala env a2"
  "evala env (Diff a1 a2) = evala env a1 - evala env a2"
  "evala env (Var v) = env v"
  "evala env (Num n) = n"

  "evalb env (Less a1 a2) = (evala env a1 < evala env a2)"
  "evalb env (And b1 b2) = (evalb env b1 & evalb env b2)"
  "evalb env (Or b1 b2) = (evalb env b1 & evalb env b2)"
\end{ttbox}
Since the value of an expression depends on the value of its variables,
the functions \texttt{evala} and \texttt{evalb} take an additional
parameter, an {\em environment} of type \texttt{'a => nat}, which maps
variables to their values.

Similarly, we may define substitution functions \texttt{substa}
and \texttt{substb} for expressions: The mapping \texttt{f} of type
\texttt{'a => 'a aexp} given as a parameter is lifted canonically
on the types \texttt{'a aexp} and \texttt{'a bexp}:
\begin{ttbox}
consts
  substa :: "['a => 'b aexp, 'a aexp] => 'b aexp"
  substb :: "['a => 'b aexp, 'a bexp] => 'b bexp"

primrec
  "substa f (If_then_else b a1 a2) =
     If_then_else (substb f b) (substa f a1) (substa f a2)"
  "substa f (Sum a1 a2) = Sum (substa f a1) (substa f a2)"
  "substa f (Diff a1 a2) = Diff (substa f a1) (substa f a2)"
  "substa f (Var v) = f v"
  "substa f (Num n) = Num n"

  "substb f (Less a1 a2) = Less (substa f a1) (substa f a2)"
  "substb f (And b1 b2) = And (substb f b1) (substb f b2)"
  "substb f (Or b1 b2) = Or (substb f b1) (substb f b2)"
\end{ttbox}
In textbooks about semantics one often finds {\em substitution theorems},
which express the relationship between substitution and evaluation. For
\texttt{'a aexp} and \texttt{'a bexp}, we can prove such a theorem by mutual
induction, followed by simplification:
\begin{ttbox}
Goal
  "evala env (substa (Var(v := a')) a) =
     evala (env(v := evala env a')) a &
   evalb env (substb (Var(v := a')) b) =
     evalb (env(v := evala env a')) b";
by (induct_tac "a b" 1);
by (ALLGOALS Asm_full_simp_tac);
\end{ttbox}

\subsubsection{Example: A substitution function for terms}
Functions on datatypes with nested recursion, such as the type
\texttt{term} mentioned in {\S}\ref{subsec:datatype:basics}, are
also defined by mutual primitive recursion. A substitution
function \texttt{subst_term} on type \texttt{term}, similar to the functions
\texttt{substa} and \texttt{substb} described above, can
be defined as follows:
\begin{ttbox}
consts
  subst_term :: "['a => ('a,'b) term, ('a,'b) term] => ('a,'b) term"
  subst_term_list ::
    "['a => ('a,'b) term, ('a,'b) term list] => ('a,'b) term list"

primrec
  "subst_term f (Var a) = f a"
  "subst_term f (App b ts) = App b (subst_term_list f ts)"

  "subst_term_list f [] = []"
  "subst_term_list f (t # ts) =
     subst_term f t # subst_term_list f ts"
\end{ttbox}
The recursion scheme follows the structure of the unfolded definition of type
\texttt{term} shown in {\S}\ref{subsec:datatype:basics}. To prove properties of
this substitution function, mutual induction is needed:
\begin{ttbox}
Goal
  "(subst_term ((subst_term f1) o f2) t) =
     (subst_term f1 (subst_term f2 t)) &
   (subst_term_list ((subst_term f1) o f2) ts) =
     (subst_term_list f1 (subst_term_list f2 ts))";
by (induct_tac "t ts" 1);
by (ALLGOALS Asm_full_simp_tac);
\end{ttbox}

\subsubsection{Example: A map function for infinitely branching trees}
Defining functions on infinitely branching datatypes by primitive
recursion is just as easy. For example, we can define a function
\texttt{map_tree} on \texttt{'a tree} as follows:
\begin{ttbox}
consts
  map_tree :: "('a => 'b) => 'a tree => 'b tree"

primrec
  "map_tree f (Atom a) = Atom (f a)"
  "map_tree f (Branch ts) = Branch (\%x. map_tree f (ts x))"
\end{ttbox}
Note that all occurrences of functions such as \texttt{ts} in the
\texttt{primrec} clauses must be applied to an argument. In particular,
\texttt{map_tree f o ts} is not allowed.

\index{recursion!primitive|)}
\index{*primrec|)}


\subsection{General recursive functions}
\label{sec:HOL:recdef}
\index{recursion!general|(}
\index{*recdef|(}

Using \texttt{recdef}, you can declare functions involving nested recursion
and pattern-matching.  Recursion need not involve datatypes and there are few
syntactic restrictions.  Termination is proved by showing that each recursive
call makes the argument smaller in a suitable sense, which you specify by
supplying a well-founded relation.

Here is a simple example, the Fibonacci function.  The first line declares
\texttt{fib} to be a constant.  The well-founded relation is simply~$<$ (on
the natural numbers).  Pattern-matching is used here: \texttt{1} is a
macro for \texttt{Suc~0}.
\begin{ttbox}
consts fib  :: "nat => nat"
recdef fib "less_than"
    "fib 0 = 0"
    "fib 1 = 1"
    "fib (Suc(Suc x)) = (fib x + fib (Suc x))"
\end{ttbox}

With \texttt{recdef}, function definitions may be incomplete, and patterns may
overlap, as in functional programming.  The \texttt{recdef} package
disambiguates overlapping patterns by taking the order of rules into account.
For missing patterns, the function is defined to return a default value.

%For example, here is a declaration of the list function \cdx{hd}:
%\begin{ttbox}
%consts hd :: 'a list => 'a
%recdef hd "\{\}"
%    "hd (x#l) = x"
%\end{ttbox}
%Because this function is not recursive, we may supply the empty well-founded
%relation, $\{\}$.

The well-founded relation defines a notion of ``smaller'' for the function's
argument type.  The relation $\prec$ is \textbf{well-founded} provided it
admits no infinitely decreasing chains
\[ \cdots\prec x@n\prec\cdots\prec x@1. \]
If the function's argument has type~$\tau$, then $\prec$ has to be a relation
over~$\tau$: it must have type $(\tau\times\tau)set$.

Proving well-foundedness can be tricky, so Isabelle/HOL provides a collection
of operators for building well-founded relations.  The package recognises
these operators and automatically proves that the constructed relation is
well-founded.  Here are those operators, in order of importance:
\begin{itemize}
\item \texttt{less_than} is ``less than'' on the natural numbers.
  (It has type $(nat\times nat)set$, while $<$ has type $[nat,nat]\To bool$.
  
\item $\mathop{\mathtt{measure}} f$, where $f$ has type $\tau\To nat$, is the
  relation~$\prec$ on type~$\tau$ such that $x\prec y$ if and only if
  $f(x)<f(y)$.  
  Typically, $f$ takes the recursive function's arguments (as a tuple) and
  returns a result expressed in terms of the function \texttt{size}.  It is
  called a \textbf{measure function}.  Recall that \texttt{size} is overloaded
  and is defined on all datatypes (see {\S}\ref{sec:HOL:size}).
                                                    
\item $\mathop{\mathtt{inv_image}} R\;f$ is a generalisation of
  \texttt{measure}.  It specifies a relation such that $x\prec y$ if and only
  if $f(x)$ 
  is less than $f(y)$ according to~$R$, which must itself be a well-founded
  relation.

\item $R@1\texttt{<*lex*>}R@2$ is the lexicographic product of two relations.
  It 
  is a relation on pairs and satisfies $(x@1,x@2)\prec(y@1,y@2)$ if and only
  if $x@1$ 
  is less than $y@1$ according to~$R@1$ or $x@1=y@1$ and $x@2$
  is less than $y@2$ according to~$R@2$.

\item \texttt{finite_psubset} is the proper subset relation on finite sets.
\end{itemize}

We can use \texttt{measure} to declare Euclid's algorithm for the greatest
common divisor.  The measure function, $\lambda(m,n). n$, specifies that the
recursion terminates because argument~$n$ decreases.
\begin{ttbox}
recdef gcd "measure ((\%(m,n). n) ::nat*nat=>nat)"
    "gcd (m, n) = (if n=0 then m else gcd(n, m mod n))"
\end{ttbox}

The general form of a well-founded recursive definition is
\begin{ttbox}
recdef {\it function} {\it rel}
    congs   {\it congruence rules}      {\bf(optional)}
    simpset {\it simplification set}      {\bf(optional)}
   {\it reduction rules}
\end{ttbox}
where
\begin{itemize}
\item \textit{function} is the name of the function, either as an \textit{id}
  or a \textit{string}.  
  
\item \textit{rel} is a HOL expression for the well-founded termination
  relation.
  
\item \textit{congruence rules} are required only in highly exceptional
  circumstances.
  
\item The \textit{simplification set} is used to prove that the supplied
  relation is well-founded.  It is also used to prove the \textbf{termination
    conditions}: assertions that arguments of recursive calls decrease under
  \textit{rel}.  By default, simplification uses \texttt{simpset()}, which
  is sufficient to prove well-foundedness for the built-in relations listed
  above. 
  
\item \textit{reduction rules} specify one or more recursion equations.  Each
  left-hand side must have the form $f\,t$, where $f$ is the function and $t$
  is a tuple of distinct variables.  If more than one equation is present then
  $f$ is defined by pattern-matching on components of its argument whose type
  is a \texttt{datatype}.  

  The \ML\ identifier $f$\texttt{.simps} contains the reduction rules as
  a list of theorems.
\end{itemize}

With the definition of \texttt{gcd} shown above, Isabelle/HOL is unable to
prove one termination condition.  It remains as a precondition of the
recursion theorems:
\begin{ttbox}
gcd.simps;
{\out ["! m n. n ~= 0 --> m mod n < n}
{\out   ==> gcd (?m,?n) = (if ?n=0 then ?m else gcd (?n, ?m mod ?n))"] }
{\out : thm list}
\end{ttbox}
The theory \texttt{HOL/ex/Primes} illustrates how to prove termination
conditions afterwards.  The function \texttt{Tfl.tgoalw} is like the standard
function \texttt{goalw}, which sets up a goal to prove, but its argument
should be the identifier $f$\texttt{.simps} and its effect is to set up a
proof of the termination conditions:
\begin{ttbox}
Tfl.tgoalw thy [] gcd.simps;
{\out Level 0}
{\out ! m n. n ~= 0 --> m mod n < n}
{\out  1. ! m n. n ~= 0 --> m mod n < n}
\end{ttbox}
This subgoal has a one-step proof using \texttt{simp_tac}.  Once the theorem
is proved, it can be used to eliminate the termination conditions from
elements of \texttt{gcd.simps}.  Theory \texttt{HOL/Subst/Unify} is a much
more complicated example of this process, where the termination conditions can
only be proved by complicated reasoning involving the recursive function
itself.

Isabelle/HOL can prove the \texttt{gcd} function's termination condition
automatically if supplied with the right simpset.
\begin{ttbox}
recdef gcd "measure ((\%(m,n). n) ::nat*nat=>nat)"
  simpset "simpset() addsimps [mod_less_divisor, zero_less_eq]"
    "gcd (m, n) = (if n=0 then m else gcd(n, m mod n))"
\end{ttbox}

If all termination conditions were proved automatically, $f$\texttt{.simps}
is added to the simpset automatically, just as in \texttt{primrec}. 
The simplification rules corresponding to clause $i$ (where counting starts
at 0) are called $f$\texttt{.}$i$ and can be accessed as \texttt{thms
  "$f$.$i$"},
which returns a list of theorems. Thus you can, for example, remove specific
clauses from the simpset. Note that a single clause may give rise to a set of
simplification rules in order to capture the fact that if clauses overlap,
their order disambiguates them.

A \texttt{recdef} definition also returns an induction rule specialised for
the recursive function.  For the \texttt{gcd} function above, the induction
rule is
\begin{ttbox}
gcd.induct;
{\out "(!!m n. n ~= 0 --> ?P n (m mod n) ==> ?P m n) ==> ?P ?u ?v" : thm}
\end{ttbox}
This rule should be used to reason inductively about the \texttt{gcd}
function.  It usually makes the induction hypothesis available at all
recursive calls, leading to very direct proofs.  If any termination conditions
remain unproved, they will become additional premises of this rule.

\index{recursion!general|)}
\index{*recdef|)}


\section{Inductive and coinductive definitions}
\index{*inductive|(}
\index{*coinductive|(}

An {\bf inductive definition} specifies the least set~$R$ closed under given
rules.  (Applying a rule to elements of~$R$ yields a result within~$R$.)  For
example, a structural operational semantics is an inductive definition of an
evaluation relation.  Dually, a {\bf coinductive definition} specifies the
greatest set~$R$ consistent with given rules.  (Every element of~$R$ can be
seen as arising by applying a rule to elements of~$R$.)  An important example
is using bisimulation relations to formalise equivalence of processes and
infinite data structures.

A theory file may contain any number of inductive and coinductive
definitions.  They may be intermixed with other declarations; in
particular, the (co)inductive sets {\bf must} be declared separately as
constants, and may have mixfix syntax or be subject to syntax translations.

Each (co)inductive definition adds definitions to the theory and also
proves some theorems.  Each definition creates an \ML\ structure, which is a
substructure of the main theory structure.

This package is related to the ZF one, described in a separate
paper,%
\footnote{It appeared in CADE~\cite{paulson-CADE}; a longer version is
  distributed with Isabelle.}  %
which you should refer to in case of difficulties.  The package is simpler
than ZF's thanks to HOL's extra-logical automatic type-checking.  The types of
the (co)inductive sets determine the domain of the fixedpoint definition, and
the package does not have to use inference rules for type-checking.


\subsection{The result structure}
Many of the result structure's components have been discussed in the paper;
others are self-explanatory.
\begin{description}
\item[\tt defs] is the list of definitions of the recursive sets.

\item[\tt mono] is a monotonicity theorem for the fixedpoint operator.

\item[\tt unfold] is a fixedpoint equation for the recursive set (the union of
the recursive sets, in the case of mutual recursion).

\item[\tt intrs] is the list of introduction rules, now proved as theorems, for
the recursive sets.  The rules are also available individually, using the
names given them in the theory file. 

\item[\tt elims] is the list of elimination rule.  This is for compatibility
  with ML scripts; within the theory the name is \texttt{cases}.
  
\item[\tt elim] is the head of the list \texttt{elims}.  This is for
  compatibility only.
  
\item[\tt mk_cases] is a function to create simplified instances of {\tt
elim} using freeness reasoning on underlying datatypes.
\end{description}

For an inductive definition, the result structure contains the
rule \texttt{induct}.  For a
coinductive definition, it contains the rule \verb|coinduct|.

Figure~\ref{def-result-fig} summarises the two result signatures,
specifying the types of all these components.

\begin{figure}
\begin{ttbox}
sig
val defs         : thm list
val mono         : thm
val unfold       : thm
val intrs        : thm list
val elims        : thm list
val elim         : thm
val mk_cases     : string -> thm
{\it(Inductive definitions only)} 
val induct       : thm
{\it(coinductive definitions only)}
val coinduct     : thm
end
\end{ttbox}
\hrule
\caption{The {\ML} result of a (co)inductive definition} \label{def-result-fig}
\end{figure}

\subsection{The syntax of a (co)inductive definition}
An inductive definition has the form
\begin{ttbox}
inductive    {\it inductive sets}
  intrs      {\it introduction rules}
  monos      {\it monotonicity theorems}
\end{ttbox}
A coinductive definition is identical, except that it starts with the keyword
\texttt{coinductive}.  

The \texttt{monos} section is optional; if present it is specified by a list
of identifiers.

\begin{itemize}
\item The \textit{inductive sets} are specified by one or more strings.

\item The \textit{introduction rules} specify one or more introduction rules in
  the form \textit{ident\/}~\textit{string}, where the identifier gives the name of
  the rule in the result structure.

\item The \textit{monotonicity theorems} are required for each operator
  applied to a recursive set in the introduction rules.  There {\bf must}
  be a theorem of the form $A\subseteq B\Imp M(A)\subseteq M(B)$, for each
  premise $t\in M(R@i)$ in an introduction rule!

\item The \textit{constructor definitions} contain definitions of constants
  appearing in the introduction rules.  In most cases it can be omitted.
\end{itemize}


\subsection{*Monotonicity theorems}

Each theory contains a default set of theorems that are used in monotonicity
proofs. New rules can be added to this set via the \texttt{mono} attribute.
Theory \texttt{Inductive} shows how this is done. In general, the following
monotonicity theorems may be added:
\begin{itemize}
\item Theorems of the form $A\subseteq B\Imp M(A)\subseteq M(B)$, for proving
  monotonicity of inductive definitions whose introduction rules have premises
  involving terms such as $t\in M(R@i)$.
\item Monotonicity theorems for logical operators, which are of the general form
  $\List{\cdots \to \cdots;~\ldots;~\cdots \to \cdots} \Imp
    \cdots \to \cdots$.
  For example, in the case of the operator $\lor$, the corresponding theorem is
  \[
  \infer{P@1 \lor P@2 \to Q@1 \lor Q@2}
    {P@1 \to Q@1 & P@2 \to Q@2}
  \]
\item De Morgan style equations for reasoning about the ``polarity'' of expressions, e.g.
  \[
  (\lnot \lnot P) ~=~ P \qquad\qquad
  (\lnot (P \land Q)) ~=~ (\lnot P \lor \lnot Q)
  \]
\item Equations for reducing complex operators to more primitive ones whose
  monotonicity can easily be proved, e.g.
  \[
  (P \to Q) ~=~ (\lnot P \lor Q) \qquad\qquad
  \mathtt{Ball}~A~P ~\equiv~ \forall x.~x \in A \to P~x
  \]
\end{itemize}

\subsection{Example of an inductive definition}
Two declarations, included in a theory file, define the finite powerset
operator.  First we declare the constant~\texttt{Fin}.  Then we declare it
inductively, with two introduction rules:
\begin{ttbox}
consts Fin :: 'a set => 'a set set
inductive "Fin A"
  intrs
    emptyI  "{\ttlbrace}{\ttrbrace} : Fin A"
    insertI "[| a: A;  b: Fin A |] ==> insert a b : Fin A"
\end{ttbox}
The resulting theory structure contains a substructure, called~\texttt{Fin}.
It contains the \texttt{Fin}$~A$ introduction rules as the list \texttt{Fin.intrs},
and also individually as \texttt{Fin.emptyI} and \texttt{Fin.consI}.  The induction
rule is \texttt{Fin.induct}.

For another example, here is a theory file defining the accessible part of a
relation.  The paper \cite{paulson-CADE} discusses a ZF version of this
example in more detail.
\begin{ttbox}
Acc = WF + Inductive +

consts acc :: "('a * 'a)set => 'a set"   (* accessible part *)

inductive "acc r"
  intrs
    accI "ALL y. (y, x) : r --> y : acc r ==> x : acc r"

end
\end{ttbox}
The Isabelle distribution contains many other inductive definitions.

\index{*coinductive|)} \index{*inductive|)}


\section{Example: Cantor's Theorem}\label{sec:hol-cantor}
Cantor's Theorem states that every set has more subsets than it has
elements.  It has become a favourite example in higher-order logic since
it is so easily expressed:
\[  \forall f::\alpha \To \alpha \To bool. \exists S::\alpha\To bool.
    \forall x::\alpha. f~x \not= S 
\] 
%
Viewing types as sets, $\alpha\To bool$ represents the powerset
of~$\alpha$.  This version states that for every function from $\alpha$ to
its powerset, some subset is outside its range.  

The Isabelle proof uses HOL's set theory, with the type $\alpha\,set$ and
the operator \cdx{range}.
\begin{ttbox}
context Set.thy;
\end{ttbox}
The set~$S$ is given as an unknown instead of a
quantified variable so that we may inspect the subset found by the proof.
\begin{ttbox}
Goal "?S ~: range\thinspace(f :: 'a=>'a set)";
{\out Level 0}
{\out ?S ~: range f}
{\out  1. ?S ~: range f}
\end{ttbox}
The first two steps are routine.  The rule \tdx{rangeE} replaces
$\Var{S}\in \texttt{range} \, f$ by $\Var{S}=f~x$ for some~$x$.
\begin{ttbox}
by (resolve_tac [notI] 1);
{\out Level 1}
{\out ?S ~: range f}
{\out  1. ?S : range f ==> False}
\ttbreak
by (eresolve_tac [rangeE] 1);
{\out Level 2}
{\out ?S ~: range f}
{\out  1. !!x. ?S = f x ==> False}
\end{ttbox}
Next, we apply \tdx{equalityCE}, reasoning that since $\Var{S}=f~x$,
we have $\Var{c}\in \Var{S}$ if and only if $\Var{c}\in f~x$ for
any~$\Var{c}$.
\begin{ttbox}
by (eresolve_tac [equalityCE] 1);
{\out Level 3}
{\out ?S ~: range f}
{\out  1. !!x. [| ?c3 x : ?S; ?c3 x : f x |] ==> False}
{\out  2. !!x. [| ?c3 x ~: ?S; ?c3 x ~: f x |] ==> False}
\end{ttbox}
Now we use a bit of creativity.  Suppose that~$\Var{S}$ has the form of a
comprehension.  Then $\Var{c}\in\{x.\Var{P}~x\}$ implies
$\Var{P}~\Var{c}$.   Destruct-resolution using \tdx{CollectD}
instantiates~$\Var{S}$ and creates the new assumption.
\begin{ttbox}
by (dresolve_tac [CollectD] 1);
{\out Level 4}
{\out {\ttlbrace}x. ?P7 x{\ttrbrace} ~: range f}
{\out  1. !!x. [| ?c3 x : f x; ?P7(?c3 x) |] ==> False}
{\out  2. !!x. [| ?c3 x ~: {\ttlbrace}x. ?P7 x{\ttrbrace}; ?c3 x ~: f x |] ==> False}
\end{ttbox}
Forcing a contradiction between the two assumptions of subgoal~1
completes the instantiation of~$S$.  It is now the set $\{x. x\not\in
f~x\}$, which is the standard diagonal construction.
\begin{ttbox}
by (contr_tac 1);
{\out Level 5}
{\out {\ttlbrace}x. x ~: f x{\ttrbrace} ~: range f}
{\out  1. !!x. [| x ~: {\ttlbrace}x. x ~: f x{\ttrbrace}; x ~: f x |] ==> False}
\end{ttbox}
The rest should be easy.  To apply \tdx{CollectI} to the negated
assumption, we employ \ttindex{swap_res_tac}:
\begin{ttbox}
by (swap_res_tac [CollectI] 1);
{\out Level 6}
{\out {\ttlbrace}x. x ~: f x{\ttrbrace} ~: range f}
{\out  1. !!x. [| x ~: f x; ~ False |] ==> x ~: f x}
\ttbreak
by (assume_tac 1);
{\out Level 7}
{\out {\ttlbrace}x. x ~: f x{\ttrbrace} ~: range f}
{\out No subgoals!}
\end{ttbox}
How much creativity is required?  As it happens, Isabelle can prove this
theorem automatically.  The default classical set \texttt{claset()} contains
rules for most of the constructs of HOL's set theory.  We must augment it with
\tdx{equalityCE} to break up set equalities, and then apply best-first search.
Depth-first search would diverge, but best-first search successfully navigates
through the large search space.  \index{search!best-first}
\begin{ttbox}
choplev 0;
{\out Level 0}
{\out ?S ~: range f}
{\out  1. ?S ~: range f}
\ttbreak
by (best_tac (claset() addSEs [equalityCE]) 1);
{\out Level 1}
{\out {\ttlbrace}x. x ~: f x{\ttrbrace} ~: range f}
{\out No subgoals!}
\end{ttbox}
If you run this example interactively, make sure your current theory contains
theory \texttt{Set}, for example by executing \ttindex{context}~{\tt Set.thy}.
Otherwise the default claset may not contain the rules for set theory.
\index{higher-order logic|)}

%%% Local Variables: 
%%% mode: latex
%%% TeX-master: "logics-HOL"
%%% End: 
