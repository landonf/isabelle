%
\begin{isabellebody}%
\def\isabellecontext{Classes}%
%
\isadelimtheory
%
\endisadelimtheory
%
\isatagtheory
\isacommand{theory}\isamarkupfalse%
\ Classes\isanewline
\isakeyword{imports}\ Main\ Setup\isanewline
\isakeyword{begin}%
\endisatagtheory
{\isafoldtheory}%
%
\isadelimtheory
%
\endisadelimtheory
%
\isamarkupsection{Introduction%
}
\isamarkuptrue%
%
\begin{isamarkuptext}%
Type classes were introduced by Wadler and Blott \cite{wadler89how}
  into the Haskell language to allow for a reasonable implementation
  of overloading\footnote{throughout this tutorial, we are referring
  to classical Haskell 1.0 type classes, not considering later
  additions in expressiveness}.  As a canonical example, a polymorphic
  equality function \isa{eq\ {\isaliteral{5C3C436F6C6F6E3E}{\isasymColon}}\ {\isaliteral{5C3C616C7068613E}{\isasymalpha}}\ {\isaliteral{5C3C52696768746172726F773E}{\isasymRightarrow}}\ {\isaliteral{5C3C616C7068613E}{\isasymalpha}}\ {\isaliteral{5C3C52696768746172726F773E}{\isasymRightarrow}}\ bool} which is overloaded on
  different types for \isa{{\isaliteral{5C3C616C7068613E}{\isasymalpha}}}, which is achieved by splitting
  introduction of the \isa{eq} function from its overloaded
  definitions by means of \isa{class} and \isa{instance}
  declarations: \footnote{syntax here is a kind of isabellized
  Haskell}

  \begin{quote}

  \noindent\isa{class\ eq\ where} \\
  \hspace*{2ex}\isa{eq\ {\isaliteral{5C3C436F6C6F6E3E}{\isasymColon}}\ {\isaliteral{5C3C616C7068613E}{\isasymalpha}}\ {\isaliteral{5C3C52696768746172726F773E}{\isasymRightarrow}}\ {\isaliteral{5C3C616C7068613E}{\isasymalpha}}\ {\isaliteral{5C3C52696768746172726F773E}{\isasymRightarrow}}\ bool}

  \medskip\noindent\isa{instance\ nat\ {\isaliteral{5C3C436F6C6F6E3E}{\isasymColon}}\ eq\ where} \\
  \hspace*{2ex}\isa{eq\ {\isadigit{0}}\ {\isadigit{0}}\ {\isaliteral{3D}{\isacharequal}}\ True} \\
  \hspace*{2ex}\isa{eq\ {\isadigit{0}}\ {\isaliteral{5F}{\isacharunderscore}}\ {\isaliteral{3D}{\isacharequal}}\ False} \\
  \hspace*{2ex}\isa{eq\ {\isaliteral{5F}{\isacharunderscore}}\ {\isadigit{0}}\ {\isaliteral{3D}{\isacharequal}}\ False} \\
  \hspace*{2ex}\isa{eq\ {\isaliteral{28}{\isacharparenleft}}Suc\ n{\isaliteral{29}{\isacharparenright}}\ {\isaliteral{28}{\isacharparenleft}}Suc\ m{\isaliteral{29}{\isacharparenright}}\ {\isaliteral{3D}{\isacharequal}}\ eq\ n\ m}

  \medskip\noindent\isa{instance\ {\isaliteral{28}{\isacharparenleft}}{\isaliteral{5C3C616C7068613E}{\isasymalpha}}{\isaliteral{5C3C436F6C6F6E3E}{\isasymColon}}eq{\isaliteral{2C}{\isacharcomma}}\ {\isaliteral{5C3C626574613E}{\isasymbeta}}{\isaliteral{5C3C436F6C6F6E3E}{\isasymColon}}eq{\isaliteral{29}{\isacharparenright}}\ pair\ {\isaliteral{5C3C436F6C6F6E3E}{\isasymColon}}\ eq\ where} \\
  \hspace*{2ex}\isa{eq\ {\isaliteral{28}{\isacharparenleft}}x{\isadigit{1}}{\isaliteral{2C}{\isacharcomma}}\ y{\isadigit{1}}{\isaliteral{29}{\isacharparenright}}\ {\isaliteral{28}{\isacharparenleft}}x{\isadigit{2}}{\isaliteral{2C}{\isacharcomma}}\ y{\isadigit{2}}{\isaliteral{29}{\isacharparenright}}\ {\isaliteral{3D}{\isacharequal}}\ eq\ x{\isadigit{1}}\ x{\isadigit{2}}\ {\isaliteral{5C3C616E643E}{\isasymand}}\ eq\ y{\isadigit{1}}\ y{\isadigit{2}}}

  \medskip\noindent\isa{class\ ord\ extends\ eq\ where} \\
  \hspace*{2ex}\isa{less{\isaliteral{5F}{\isacharunderscore}}eq\ {\isaliteral{5C3C436F6C6F6E3E}{\isasymColon}}\ {\isaliteral{5C3C616C7068613E}{\isasymalpha}}\ {\isaliteral{5C3C52696768746172726F773E}{\isasymRightarrow}}\ {\isaliteral{5C3C616C7068613E}{\isasymalpha}}\ {\isaliteral{5C3C52696768746172726F773E}{\isasymRightarrow}}\ bool} \\
  \hspace*{2ex}\isa{less\ {\isaliteral{5C3C436F6C6F6E3E}{\isasymColon}}\ {\isaliteral{5C3C616C7068613E}{\isasymalpha}}\ {\isaliteral{5C3C52696768746172726F773E}{\isasymRightarrow}}\ {\isaliteral{5C3C616C7068613E}{\isasymalpha}}\ {\isaliteral{5C3C52696768746172726F773E}{\isasymRightarrow}}\ bool}

  \end{quote}

  \noindent Type variables are annotated with (finitely many) classes;
  these annotations are assertions that a particular polymorphic type
  provides definitions for overloaded functions.

  Indeed, type classes not only allow for simple overloading but form
  a generic calculus, an instance of order-sorted algebra
  \cite{nipkow-sorts93,Nipkow-Prehofer:1993,Wenzel:1997:TPHOL}.

  From a software engineering point of view, type classes roughly
  correspond to interfaces in object-oriented languages like Java; so,
  it is naturally desirable that type classes do not only provide
  functions (class parameters) but also state specifications
  implementations must obey.  For example, the \isa{class\ eq}
  above could be given the following specification, demanding that
  \isa{class\ eq} is an equivalence relation obeying reflexivity,
  symmetry and transitivity:

  \begin{quote}

  \noindent\isa{class\ eq\ where} \\
  \hspace*{2ex}\isa{eq\ {\isaliteral{5C3C436F6C6F6E3E}{\isasymColon}}\ {\isaliteral{5C3C616C7068613E}{\isasymalpha}}\ {\isaliteral{5C3C52696768746172726F773E}{\isasymRightarrow}}\ {\isaliteral{5C3C616C7068613E}{\isasymalpha}}\ {\isaliteral{5C3C52696768746172726F773E}{\isasymRightarrow}}\ bool} \\
  \isa{satisfying} \\
  \hspace*{2ex}\isa{refl{\isaliteral{3A}{\isacharcolon}}\ eq\ x\ x} \\
  \hspace*{2ex}\isa{sym{\isaliteral{3A}{\isacharcolon}}\ eq\ x\ y\ {\isaliteral{5C3C6C6F6E676C65667472696768746172726F773E}{\isasymlongleftrightarrow}}\ eq\ x\ y} \\
  \hspace*{2ex}\isa{trans{\isaliteral{3A}{\isacharcolon}}\ eq\ x\ y\ {\isaliteral{5C3C616E643E}{\isasymand}}\ eq\ y\ z\ {\isaliteral{5C3C6C6F6E6772696768746172726F773E}{\isasymlongrightarrow}}\ eq\ x\ z}

  \end{quote}

  \noindent From a theoretical point of view, type classes are
  lightweight modules; Haskell type classes may be emulated by SML
  functors \cite{classes_modules}.  Isabelle/Isar offers a discipline
  of type classes which brings all those aspects together:

  \begin{enumerate}
    \item specifying abstract parameters together with
       corresponding specifications,
    \item instantiating those abstract parameters by a particular
       type
    \item in connection with a ``less ad-hoc'' approach to overloading,
    \item with a direct link to the Isabelle module system:
      locales \cite{kammueller-locales}.
  \end{enumerate}

  \noindent Isar type classes also directly support code generation in
  a Haskell like fashion. Internally, they are mapped to more
  primitive Isabelle concepts \cite{Haftmann-Wenzel:2006:classes}.

  This tutorial demonstrates common elements of structured
  specifications and abstract reasoning with type classes by the
  algebraic hierarchy of semigroups, monoids and groups.  Our
  background theory is that of Isabelle/HOL \cite{isa-tutorial}, for
  which some familiarity is assumed.%
\end{isamarkuptext}%
\isamarkuptrue%
%
\isamarkupsection{A simple algebra example \label{sec:example}%
}
\isamarkuptrue%
%
\isamarkupsubsection{Class definition%
}
\isamarkuptrue%
%
\begin{isamarkuptext}%
Depending on an arbitrary type \isa{{\isaliteral{5C3C616C7068613E}{\isasymalpha}}}, class \isa{semigroup} introduces a binary operator \isa{{\isaliteral{28}{\isacharparenleft}}{\isaliteral{5C3C6F74696D65733E}{\isasymotimes}}{\isaliteral{29}{\isacharparenright}}} that is
  assumed to be associative:%
\end{isamarkuptext}%
\isamarkuptrue%
%
\isadelimquote
%
\endisadelimquote
%
\isatagquote
\isacommand{class}\isamarkupfalse%
\ semigroup\ {\isaliteral{3D}{\isacharequal}}\isanewline
\ \ \isakeyword{fixes}\ mult\ {\isaliteral{3A}{\isacharcolon}}{\isaliteral{3A}{\isacharcolon}}\ {\isaliteral{22}{\isachardoublequoteopen}}{\isaliteral{5C3C616C7068613E}{\isasymalpha}}\ {\isaliteral{5C3C52696768746172726F773E}{\isasymRightarrow}}\ {\isaliteral{5C3C616C7068613E}{\isasymalpha}}\ {\isaliteral{5C3C52696768746172726F773E}{\isasymRightarrow}}\ {\isaliteral{5C3C616C7068613E}{\isasymalpha}}{\isaliteral{22}{\isachardoublequoteclose}}\ \ \ \ {\isaliteral{28}{\isacharparenleft}}\isakeyword{infixl}\ {\isaliteral{22}{\isachardoublequoteopen}}{\isaliteral{5C3C6F74696D65733E}{\isasymotimes}}{\isaliteral{22}{\isachardoublequoteclose}}\ {\isadigit{7}}{\isadigit{0}}{\isaliteral{29}{\isacharparenright}}\isanewline
\ \ \isakeyword{assumes}\ assoc{\isaliteral{3A}{\isacharcolon}}\ {\isaliteral{22}{\isachardoublequoteopen}}{\isaliteral{28}{\isacharparenleft}}x\ {\isaliteral{5C3C6F74696D65733E}{\isasymotimes}}\ y{\isaliteral{29}{\isacharparenright}}\ {\isaliteral{5C3C6F74696D65733E}{\isasymotimes}}\ z\ {\isaliteral{3D}{\isacharequal}}\ x\ {\isaliteral{5C3C6F74696D65733E}{\isasymotimes}}\ {\isaliteral{28}{\isacharparenleft}}y\ {\isaliteral{5C3C6F74696D65733E}{\isasymotimes}}\ z{\isaliteral{29}{\isacharparenright}}{\isaliteral{22}{\isachardoublequoteclose}}%
\endisatagquote
{\isafoldquote}%
%
\isadelimquote
%
\endisadelimquote
%
\begin{isamarkuptext}%
\noindent This \hyperlink{command.class}{\mbox{\isa{\isacommand{class}}}} specification consists of two parts:
  the \qn{operational} part names the class parameter (\hyperlink{element.fixes}{\mbox{\isa{\isakeyword{fixes}}}}), the \qn{logical} part specifies properties on them
  (\hyperlink{element.assumes}{\mbox{\isa{\isakeyword{assumes}}}}).  The local \hyperlink{element.fixes}{\mbox{\isa{\isakeyword{fixes}}}} and \hyperlink{element.assumes}{\mbox{\isa{\isakeyword{assumes}}}} are lifted to the theory toplevel, yielding the global
  parameter \isa{{\isaliteral{22}{\isachardoublequote}}mult\ {\isaliteral{5C3C436F6C6F6E3E}{\isasymColon}}\ {\isaliteral{5C3C616C7068613E}{\isasymalpha}}{\isaliteral{5C3C436F6C6F6E3E}{\isasymColon}}semigroup\ {\isaliteral{5C3C52696768746172726F773E}{\isasymRightarrow}}\ {\isaliteral{5C3C616C7068613E}{\isasymalpha}}\ {\isaliteral{5C3C52696768746172726F773E}{\isasymRightarrow}}\ {\isaliteral{5C3C616C7068613E}{\isasymalpha}}{\isaliteral{22}{\isachardoublequote}}} and the
  global theorem \hyperlink{fact.semigroup.assoc:}{\mbox{\isa{semigroup{\isaliteral{2E}{\isachardot}}assoc{\isaliteral{3A}{\isacharcolon}}}}}~\isa{{\isaliteral{22}{\isachardoublequote}}{\isaliteral{5C3C416E643E}{\isasymAnd}}x\ y\ z\ {\isaliteral{5C3C436F6C6F6E3E}{\isasymColon}}\ {\isaliteral{5C3C616C7068613E}{\isasymalpha}}{\isaliteral{5C3C436F6C6F6E3E}{\isasymColon}}semigroup{\isaliteral{2E}{\isachardot}}\ {\isaliteral{28}{\isacharparenleft}}x\ {\isaliteral{5C3C6F74696D65733E}{\isasymotimes}}\ y{\isaliteral{29}{\isacharparenright}}\ {\isaliteral{5C3C6F74696D65733E}{\isasymotimes}}\ z\ {\isaliteral{3D}{\isacharequal}}\ x\ {\isaliteral{5C3C6F74696D65733E}{\isasymotimes}}\ {\isaliteral{28}{\isacharparenleft}}y\ {\isaliteral{5C3C6F74696D65733E}{\isasymotimes}}\ z{\isaliteral{29}{\isacharparenright}}{\isaliteral{22}{\isachardoublequote}}}.%
\end{isamarkuptext}%
\isamarkuptrue%
%
\isamarkupsubsection{Class instantiation \label{sec:class_inst}%
}
\isamarkuptrue%
%
\begin{isamarkuptext}%
The concrete type \isa{int} is made a \isa{semigroup} instance
  by providing a suitable definition for the class parameter \isa{{\isaliteral{28}{\isacharparenleft}}{\isaliteral{5C3C6F74696D65733E}{\isasymotimes}}{\isaliteral{29}{\isacharparenright}}} and a proof for the specification of \hyperlink{fact.assoc}{\mbox{\isa{assoc}}}.  This is
  accomplished by the \hyperlink{command.instantiation}{\mbox{\isa{\isacommand{instantiation}}}} target:%
\end{isamarkuptext}%
\isamarkuptrue%
%
\isadelimquote
%
\endisadelimquote
%
\isatagquote
\isacommand{instantiation}\isamarkupfalse%
\ int\ {\isaliteral{3A}{\isacharcolon}}{\isaliteral{3A}{\isacharcolon}}\ semigroup\isanewline
\isakeyword{begin}\isanewline
\isanewline
\isacommand{definition}\isamarkupfalse%
\isanewline
\ \ mult{\isaliteral{5F}{\isacharunderscore}}int{\isaliteral{5F}{\isacharunderscore}}def{\isaliteral{3A}{\isacharcolon}}\ {\isaliteral{22}{\isachardoublequoteopen}}i\ {\isaliteral{5C3C6F74696D65733E}{\isasymotimes}}\ j\ {\isaliteral{3D}{\isacharequal}}\ i\ {\isaliteral{2B}{\isacharplus}}\ {\isaliteral{28}{\isacharparenleft}}j{\isaliteral{5C3C436F6C6F6E3E}{\isasymColon}}int{\isaliteral{29}{\isacharparenright}}{\isaliteral{22}{\isachardoublequoteclose}}\isanewline
\isanewline
\isacommand{instance}\isamarkupfalse%
\ \isacommand{proof}\isamarkupfalse%
\isanewline
\ \ \isacommand{fix}\isamarkupfalse%
\ i\ j\ k\ {\isaliteral{3A}{\isacharcolon}}{\isaliteral{3A}{\isacharcolon}}\ int\ \isacommand{have}\isamarkupfalse%
\ {\isaliteral{22}{\isachardoublequoteopen}}{\isaliteral{28}{\isacharparenleft}}i\ {\isaliteral{2B}{\isacharplus}}\ j{\isaliteral{29}{\isacharparenright}}\ {\isaliteral{2B}{\isacharplus}}\ k\ {\isaliteral{3D}{\isacharequal}}\ i\ {\isaliteral{2B}{\isacharplus}}\ {\isaliteral{28}{\isacharparenleft}}j\ {\isaliteral{2B}{\isacharplus}}\ k{\isaliteral{29}{\isacharparenright}}{\isaliteral{22}{\isachardoublequoteclose}}\ \isacommand{by}\isamarkupfalse%
\ simp\isanewline
\ \ \isacommand{then}\isamarkupfalse%
\ \isacommand{show}\isamarkupfalse%
\ {\isaliteral{22}{\isachardoublequoteopen}}{\isaliteral{28}{\isacharparenleft}}i\ {\isaliteral{5C3C6F74696D65733E}{\isasymotimes}}\ j{\isaliteral{29}{\isacharparenright}}\ {\isaliteral{5C3C6F74696D65733E}{\isasymotimes}}\ k\ {\isaliteral{3D}{\isacharequal}}\ i\ {\isaliteral{5C3C6F74696D65733E}{\isasymotimes}}\ {\isaliteral{28}{\isacharparenleft}}j\ {\isaliteral{5C3C6F74696D65733E}{\isasymotimes}}\ k{\isaliteral{29}{\isacharparenright}}{\isaliteral{22}{\isachardoublequoteclose}}\isanewline
\ \ \ \ \isacommand{unfolding}\isamarkupfalse%
\ mult{\isaliteral{5F}{\isacharunderscore}}int{\isaliteral{5F}{\isacharunderscore}}def\ \isacommand{{\isaliteral{2E}{\isachardot}}}\isamarkupfalse%
\isanewline
\isacommand{qed}\isamarkupfalse%
\isanewline
\isanewline
\isacommand{end}\isamarkupfalse%
%
\endisatagquote
{\isafoldquote}%
%
\isadelimquote
%
\endisadelimquote
%
\begin{isamarkuptext}%
\noindent \hyperlink{command.instantiation}{\mbox{\isa{\isacommand{instantiation}}}} defines class parameters at a
  particular instance using common specification tools (here,
  \hyperlink{command.definition}{\mbox{\isa{\isacommand{definition}}}}).  The concluding \hyperlink{command.instance}{\mbox{\isa{\isacommand{instance}}}} opens a
  proof that the given parameters actually conform to the class
  specification.  Note that the first proof step is the \hyperlink{method.default}{\mbox{\isa{default}}} method, which for such instance proofs maps to the \hyperlink{method.intro-classes}{\mbox{\isa{intro{\isaliteral{5F}{\isacharunderscore}}classes}}} method.  This reduces an instance judgement to the
  relevant primitive proof goals; typically it is the first method
  applied in an instantiation proof.

  From now on, the type-checker will consider \isa{int} as a \isa{semigroup} automatically, i.e.\ any general results are immediately
  available on concrete instances.

  \medskip Another instance of \isa{semigroup} yields the natural
  numbers:%
\end{isamarkuptext}%
\isamarkuptrue%
%
\isadelimquote
%
\endisadelimquote
%
\isatagquote
\isacommand{instantiation}\isamarkupfalse%
\ nat\ {\isaliteral{3A}{\isacharcolon}}{\isaliteral{3A}{\isacharcolon}}\ semigroup\isanewline
\isakeyword{begin}\isanewline
\isanewline
\isacommand{primrec}\isamarkupfalse%
\ mult{\isaliteral{5F}{\isacharunderscore}}nat\ \isakeyword{where}\isanewline
\ \ {\isaliteral{22}{\isachardoublequoteopen}}{\isaliteral{28}{\isacharparenleft}}{\isadigit{0}}{\isaliteral{5C3C436F6C6F6E3E}{\isasymColon}}nat{\isaliteral{29}{\isacharparenright}}\ {\isaliteral{5C3C6F74696D65733E}{\isasymotimes}}\ n\ {\isaliteral{3D}{\isacharequal}}\ n{\isaliteral{22}{\isachardoublequoteclose}}\isanewline
\ \ {\isaliteral{7C}{\isacharbar}}\ {\isaliteral{22}{\isachardoublequoteopen}}Suc\ m\ {\isaliteral{5C3C6F74696D65733E}{\isasymotimes}}\ n\ {\isaliteral{3D}{\isacharequal}}\ Suc\ {\isaliteral{28}{\isacharparenleft}}m\ {\isaliteral{5C3C6F74696D65733E}{\isasymotimes}}\ n{\isaliteral{29}{\isacharparenright}}{\isaliteral{22}{\isachardoublequoteclose}}\isanewline
\isanewline
\isacommand{instance}\isamarkupfalse%
\ \isacommand{proof}\isamarkupfalse%
\isanewline
\ \ \isacommand{fix}\isamarkupfalse%
\ m\ n\ q\ {\isaliteral{3A}{\isacharcolon}}{\isaliteral{3A}{\isacharcolon}}\ nat\ \isanewline
\ \ \isacommand{show}\isamarkupfalse%
\ {\isaliteral{22}{\isachardoublequoteopen}}m\ {\isaliteral{5C3C6F74696D65733E}{\isasymotimes}}\ n\ {\isaliteral{5C3C6F74696D65733E}{\isasymotimes}}\ q\ {\isaliteral{3D}{\isacharequal}}\ m\ {\isaliteral{5C3C6F74696D65733E}{\isasymotimes}}\ {\isaliteral{28}{\isacharparenleft}}n\ {\isaliteral{5C3C6F74696D65733E}{\isasymotimes}}\ q{\isaliteral{29}{\isacharparenright}}{\isaliteral{22}{\isachardoublequoteclose}}\isanewline
\ \ \ \ \isacommand{by}\isamarkupfalse%
\ {\isaliteral{28}{\isacharparenleft}}induct\ m{\isaliteral{29}{\isacharparenright}}\ auto\isanewline
\isacommand{qed}\isamarkupfalse%
\isanewline
\isanewline
\isacommand{end}\isamarkupfalse%
%
\endisatagquote
{\isafoldquote}%
%
\isadelimquote
%
\endisadelimquote
%
\begin{isamarkuptext}%
\noindent Note the occurence of the name \isa{mult{\isaliteral{5F}{\isacharunderscore}}nat} in the
  primrec declaration; by default, the local name of a class operation
  \isa{f} to be instantiated on type constructor \isa{{\isaliteral{5C3C6B617070613E}{\isasymkappa}}} is
  mangled as \isa{f{\isaliteral{5F}{\isacharunderscore}}{\isaliteral{5C3C6B617070613E}{\isasymkappa}}}.  In case of uncertainty, these names may be
  inspected using the \hyperlink{command.print-context}{\mbox{\isa{\isacommand{print{\isaliteral{5F}{\isacharunderscore}}context}}}} command or the
  corresponding ProofGeneral button.%
\end{isamarkuptext}%
\isamarkuptrue%
%
\isamarkupsubsection{Lifting and parametric types%
}
\isamarkuptrue%
%
\begin{isamarkuptext}%
Overloaded definitions given at a class instantiation may include
  recursion over the syntactic structure of types.  As a canonical
  example, we model product semigroups using our simple algebra:%
\end{isamarkuptext}%
\isamarkuptrue%
%
\isadelimquote
%
\endisadelimquote
%
\isatagquote
\isacommand{instantiation}\isamarkupfalse%
\ prod\ {\isaliteral{3A}{\isacharcolon}}{\isaliteral{3A}{\isacharcolon}}\ {\isaliteral{28}{\isacharparenleft}}semigroup{\isaliteral{2C}{\isacharcomma}}\ semigroup{\isaliteral{29}{\isacharparenright}}\ semigroup\isanewline
\isakeyword{begin}\isanewline
\isanewline
\isacommand{definition}\isamarkupfalse%
\isanewline
\ \ mult{\isaliteral{5F}{\isacharunderscore}}prod{\isaliteral{5F}{\isacharunderscore}}def{\isaliteral{3A}{\isacharcolon}}\ {\isaliteral{22}{\isachardoublequoteopen}}p\isaliteral{5C3C5E697375623E}{}\isactrlisub {\isadigit{1}}\ {\isaliteral{5C3C6F74696D65733E}{\isasymotimes}}\ p\isaliteral{5C3C5E697375623E}{}\isactrlisub {\isadigit{2}}\ {\isaliteral{3D}{\isacharequal}}\ {\isaliteral{28}{\isacharparenleft}}fst\ p\isaliteral{5C3C5E697375623E}{}\isactrlisub {\isadigit{1}}\ {\isaliteral{5C3C6F74696D65733E}{\isasymotimes}}\ fst\ p\isaliteral{5C3C5E697375623E}{}\isactrlisub {\isadigit{2}}{\isaliteral{2C}{\isacharcomma}}\ snd\ p\isaliteral{5C3C5E697375623E}{}\isactrlisub {\isadigit{1}}\ {\isaliteral{5C3C6F74696D65733E}{\isasymotimes}}\ snd\ p\isaliteral{5C3C5E697375623E}{}\isactrlisub {\isadigit{2}}{\isaliteral{29}{\isacharparenright}}{\isaliteral{22}{\isachardoublequoteclose}}\isanewline
\isanewline
\isacommand{instance}\isamarkupfalse%
\ \isacommand{proof}\isamarkupfalse%
\isanewline
\ \ \isacommand{fix}\isamarkupfalse%
\ p\isaliteral{5C3C5E697375623E}{}\isactrlisub {\isadigit{1}}\ p\isaliteral{5C3C5E697375623E}{}\isactrlisub {\isadigit{2}}\ p\isaliteral{5C3C5E697375623E}{}\isactrlisub {\isadigit{3}}\ {\isaliteral{3A}{\isacharcolon}}{\isaliteral{3A}{\isacharcolon}}\ {\isaliteral{22}{\isachardoublequoteopen}}{\isaliteral{5C3C616C7068613E}{\isasymalpha}}{\isaliteral{5C3C436F6C6F6E3E}{\isasymColon}}semigroup\ {\isaliteral{5C3C74696D65733E}{\isasymtimes}}\ {\isaliteral{5C3C626574613E}{\isasymbeta}}{\isaliteral{5C3C436F6C6F6E3E}{\isasymColon}}semigroup{\isaliteral{22}{\isachardoublequoteclose}}\isanewline
\ \ \isacommand{show}\isamarkupfalse%
\ {\isaliteral{22}{\isachardoublequoteopen}}p\isaliteral{5C3C5E697375623E}{}\isactrlisub {\isadigit{1}}\ {\isaliteral{5C3C6F74696D65733E}{\isasymotimes}}\ p\isaliteral{5C3C5E697375623E}{}\isactrlisub {\isadigit{2}}\ {\isaliteral{5C3C6F74696D65733E}{\isasymotimes}}\ p\isaliteral{5C3C5E697375623E}{}\isactrlisub {\isadigit{3}}\ {\isaliteral{3D}{\isacharequal}}\ p\isaliteral{5C3C5E697375623E}{}\isactrlisub {\isadigit{1}}\ {\isaliteral{5C3C6F74696D65733E}{\isasymotimes}}\ {\isaliteral{28}{\isacharparenleft}}p\isaliteral{5C3C5E697375623E}{}\isactrlisub {\isadigit{2}}\ {\isaliteral{5C3C6F74696D65733E}{\isasymotimes}}\ p\isaliteral{5C3C5E697375623E}{}\isactrlisub {\isadigit{3}}{\isaliteral{29}{\isacharparenright}}{\isaliteral{22}{\isachardoublequoteclose}}\isanewline
\ \ \ \ \isacommand{unfolding}\isamarkupfalse%
\ mult{\isaliteral{5F}{\isacharunderscore}}prod{\isaliteral{5F}{\isacharunderscore}}def\ \isacommand{by}\isamarkupfalse%
\ {\isaliteral{28}{\isacharparenleft}}simp\ add{\isaliteral{3A}{\isacharcolon}}\ assoc{\isaliteral{29}{\isacharparenright}}\isanewline
\isacommand{qed}\isamarkupfalse%
\ \ \ \ \ \ \isanewline
\isanewline
\isacommand{end}\isamarkupfalse%
%
\endisatagquote
{\isafoldquote}%
%
\isadelimquote
%
\endisadelimquote
%
\begin{isamarkuptext}%
\noindent Associativity of product semigroups is established using
  the definition of \isa{{\isaliteral{28}{\isacharparenleft}}{\isaliteral{5C3C6F74696D65733E}{\isasymotimes}}{\isaliteral{29}{\isacharparenright}}} on products and the hypothetical
  associativity of the type components; these hypotheses are
  legitimate due to the \isa{semigroup} constraints imposed on the
  type components by the \hyperlink{command.instance}{\mbox{\isa{\isacommand{instance}}}} proposition.  Indeed,
  this pattern often occurs with parametric types and type classes.%
\end{isamarkuptext}%
\isamarkuptrue%
%
\isamarkupsubsection{Subclassing%
}
\isamarkuptrue%
%
\begin{isamarkuptext}%
We define a subclass \isa{monoidl} (a semigroup with a left-hand
  neutral) by extending \isa{semigroup} with one additional
  parameter \isa{neutral} together with its characteristic property:%
\end{isamarkuptext}%
\isamarkuptrue%
%
\isadelimquote
%
\endisadelimquote
%
\isatagquote
\isacommand{class}\isamarkupfalse%
\ monoidl\ {\isaliteral{3D}{\isacharequal}}\ semigroup\ {\isaliteral{2B}{\isacharplus}}\isanewline
\ \ \isakeyword{fixes}\ neutral\ {\isaliteral{3A}{\isacharcolon}}{\isaliteral{3A}{\isacharcolon}}\ {\isaliteral{22}{\isachardoublequoteopen}}{\isaliteral{5C3C616C7068613E}{\isasymalpha}}{\isaliteral{22}{\isachardoublequoteclose}}\ {\isaliteral{28}{\isacharparenleft}}{\isaliteral{22}{\isachardoublequoteopen}}{\isaliteral{5C3C6F6E653E}{\isasymone}}{\isaliteral{22}{\isachardoublequoteclose}}{\isaliteral{29}{\isacharparenright}}\isanewline
\ \ \isakeyword{assumes}\ neutl{\isaliteral{3A}{\isacharcolon}}\ {\isaliteral{22}{\isachardoublequoteopen}}{\isaliteral{5C3C6F6E653E}{\isasymone}}\ {\isaliteral{5C3C6F74696D65733E}{\isasymotimes}}\ x\ {\isaliteral{3D}{\isacharequal}}\ x{\isaliteral{22}{\isachardoublequoteclose}}%
\endisatagquote
{\isafoldquote}%
%
\isadelimquote
%
\endisadelimquote
%
\begin{isamarkuptext}%
\noindent Again, we prove some instances, by providing suitable
  parameter definitions and proofs for the additional specifications.
  Observe that instantiations for types with the same arity may be
  simultaneous:%
\end{isamarkuptext}%
\isamarkuptrue%
%
\isadelimquote
%
\endisadelimquote
%
\isatagquote
\isacommand{instantiation}\isamarkupfalse%
\ nat\ \isakeyword{and}\ int\ {\isaliteral{3A}{\isacharcolon}}{\isaliteral{3A}{\isacharcolon}}\ monoidl\isanewline
\isakeyword{begin}\isanewline
\isanewline
\isacommand{definition}\isamarkupfalse%
\isanewline
\ \ neutral{\isaliteral{5F}{\isacharunderscore}}nat{\isaliteral{5F}{\isacharunderscore}}def{\isaliteral{3A}{\isacharcolon}}\ {\isaliteral{22}{\isachardoublequoteopen}}{\isaliteral{5C3C6F6E653E}{\isasymone}}\ {\isaliteral{3D}{\isacharequal}}\ {\isaliteral{28}{\isacharparenleft}}{\isadigit{0}}{\isaliteral{5C3C436F6C6F6E3E}{\isasymColon}}nat{\isaliteral{29}{\isacharparenright}}{\isaliteral{22}{\isachardoublequoteclose}}\isanewline
\isanewline
\isacommand{definition}\isamarkupfalse%
\isanewline
\ \ neutral{\isaliteral{5F}{\isacharunderscore}}int{\isaliteral{5F}{\isacharunderscore}}def{\isaliteral{3A}{\isacharcolon}}\ {\isaliteral{22}{\isachardoublequoteopen}}{\isaliteral{5C3C6F6E653E}{\isasymone}}\ {\isaliteral{3D}{\isacharequal}}\ {\isaliteral{28}{\isacharparenleft}}{\isadigit{0}}{\isaliteral{5C3C436F6C6F6E3E}{\isasymColon}}int{\isaliteral{29}{\isacharparenright}}{\isaliteral{22}{\isachardoublequoteclose}}\isanewline
\isanewline
\isacommand{instance}\isamarkupfalse%
\ \isacommand{proof}\isamarkupfalse%
\isanewline
\ \ \isacommand{fix}\isamarkupfalse%
\ n\ {\isaliteral{3A}{\isacharcolon}}{\isaliteral{3A}{\isacharcolon}}\ nat\isanewline
\ \ \isacommand{show}\isamarkupfalse%
\ {\isaliteral{22}{\isachardoublequoteopen}}{\isaliteral{5C3C6F6E653E}{\isasymone}}\ {\isaliteral{5C3C6F74696D65733E}{\isasymotimes}}\ n\ {\isaliteral{3D}{\isacharequal}}\ n{\isaliteral{22}{\isachardoublequoteclose}}\isanewline
\ \ \ \ \isacommand{unfolding}\isamarkupfalse%
\ neutral{\isaliteral{5F}{\isacharunderscore}}nat{\isaliteral{5F}{\isacharunderscore}}def\ \isacommand{by}\isamarkupfalse%
\ simp\isanewline
\isacommand{next}\isamarkupfalse%
\isanewline
\ \ \isacommand{fix}\isamarkupfalse%
\ k\ {\isaliteral{3A}{\isacharcolon}}{\isaliteral{3A}{\isacharcolon}}\ int\isanewline
\ \ \isacommand{show}\isamarkupfalse%
\ {\isaliteral{22}{\isachardoublequoteopen}}{\isaliteral{5C3C6F6E653E}{\isasymone}}\ {\isaliteral{5C3C6F74696D65733E}{\isasymotimes}}\ k\ {\isaliteral{3D}{\isacharequal}}\ k{\isaliteral{22}{\isachardoublequoteclose}}\isanewline
\ \ \ \ \isacommand{unfolding}\isamarkupfalse%
\ neutral{\isaliteral{5F}{\isacharunderscore}}int{\isaliteral{5F}{\isacharunderscore}}def\ mult{\isaliteral{5F}{\isacharunderscore}}int{\isaliteral{5F}{\isacharunderscore}}def\ \isacommand{by}\isamarkupfalse%
\ simp\isanewline
\isacommand{qed}\isamarkupfalse%
\isanewline
\isanewline
\isacommand{end}\isamarkupfalse%
\isanewline
\isanewline
\isacommand{instantiation}\isamarkupfalse%
\ prod\ {\isaliteral{3A}{\isacharcolon}}{\isaliteral{3A}{\isacharcolon}}\ {\isaliteral{28}{\isacharparenleft}}monoidl{\isaliteral{2C}{\isacharcomma}}\ monoidl{\isaliteral{29}{\isacharparenright}}\ monoidl\isanewline
\isakeyword{begin}\isanewline
\isanewline
\isacommand{definition}\isamarkupfalse%
\isanewline
\ \ neutral{\isaliteral{5F}{\isacharunderscore}}prod{\isaliteral{5F}{\isacharunderscore}}def{\isaliteral{3A}{\isacharcolon}}\ {\isaliteral{22}{\isachardoublequoteopen}}{\isaliteral{5C3C6F6E653E}{\isasymone}}\ {\isaliteral{3D}{\isacharequal}}\ {\isaliteral{28}{\isacharparenleft}}{\isaliteral{5C3C6F6E653E}{\isasymone}}{\isaliteral{2C}{\isacharcomma}}\ {\isaliteral{5C3C6F6E653E}{\isasymone}}{\isaliteral{29}{\isacharparenright}}{\isaliteral{22}{\isachardoublequoteclose}}\isanewline
\isanewline
\isacommand{instance}\isamarkupfalse%
\ \isacommand{proof}\isamarkupfalse%
\isanewline
\ \ \isacommand{fix}\isamarkupfalse%
\ p\ {\isaliteral{3A}{\isacharcolon}}{\isaliteral{3A}{\isacharcolon}}\ {\isaliteral{22}{\isachardoublequoteopen}}{\isaliteral{5C3C616C7068613E}{\isasymalpha}}{\isaliteral{5C3C436F6C6F6E3E}{\isasymColon}}monoidl\ {\isaliteral{5C3C74696D65733E}{\isasymtimes}}\ {\isaliteral{5C3C626574613E}{\isasymbeta}}{\isaliteral{5C3C436F6C6F6E3E}{\isasymColon}}monoidl{\isaliteral{22}{\isachardoublequoteclose}}\isanewline
\ \ \isacommand{show}\isamarkupfalse%
\ {\isaliteral{22}{\isachardoublequoteopen}}{\isaliteral{5C3C6F6E653E}{\isasymone}}\ {\isaliteral{5C3C6F74696D65733E}{\isasymotimes}}\ p\ {\isaliteral{3D}{\isacharequal}}\ p{\isaliteral{22}{\isachardoublequoteclose}}\isanewline
\ \ \ \ \isacommand{unfolding}\isamarkupfalse%
\ neutral{\isaliteral{5F}{\isacharunderscore}}prod{\isaliteral{5F}{\isacharunderscore}}def\ mult{\isaliteral{5F}{\isacharunderscore}}prod{\isaliteral{5F}{\isacharunderscore}}def\ \isacommand{by}\isamarkupfalse%
\ {\isaliteral{28}{\isacharparenleft}}simp\ add{\isaliteral{3A}{\isacharcolon}}\ neutl{\isaliteral{29}{\isacharparenright}}\isanewline
\isacommand{qed}\isamarkupfalse%
\isanewline
\isanewline
\isacommand{end}\isamarkupfalse%
%
\endisatagquote
{\isafoldquote}%
%
\isadelimquote
%
\endisadelimquote
%
\begin{isamarkuptext}%
\noindent Fully-fledged monoids are modelled by another subclass,
  which does not add new parameters but tightens the specification:%
\end{isamarkuptext}%
\isamarkuptrue%
%
\isadelimquote
%
\endisadelimquote
%
\isatagquote
\isacommand{class}\isamarkupfalse%
\ monoid\ {\isaliteral{3D}{\isacharequal}}\ monoidl\ {\isaliteral{2B}{\isacharplus}}\isanewline
\ \ \isakeyword{assumes}\ neutr{\isaliteral{3A}{\isacharcolon}}\ {\isaliteral{22}{\isachardoublequoteopen}}x\ {\isaliteral{5C3C6F74696D65733E}{\isasymotimes}}\ {\isaliteral{5C3C6F6E653E}{\isasymone}}\ {\isaliteral{3D}{\isacharequal}}\ x{\isaliteral{22}{\isachardoublequoteclose}}\isanewline
\isanewline
\isacommand{instantiation}\isamarkupfalse%
\ nat\ \isakeyword{and}\ int\ {\isaliteral{3A}{\isacharcolon}}{\isaliteral{3A}{\isacharcolon}}\ monoid\ \isanewline
\isakeyword{begin}\isanewline
\isanewline
\isacommand{instance}\isamarkupfalse%
\ \isacommand{proof}\isamarkupfalse%
\isanewline
\ \ \isacommand{fix}\isamarkupfalse%
\ n\ {\isaliteral{3A}{\isacharcolon}}{\isaliteral{3A}{\isacharcolon}}\ nat\isanewline
\ \ \isacommand{show}\isamarkupfalse%
\ {\isaliteral{22}{\isachardoublequoteopen}}n\ {\isaliteral{5C3C6F74696D65733E}{\isasymotimes}}\ {\isaliteral{5C3C6F6E653E}{\isasymone}}\ {\isaliteral{3D}{\isacharequal}}\ n{\isaliteral{22}{\isachardoublequoteclose}}\isanewline
\ \ \ \ \isacommand{unfolding}\isamarkupfalse%
\ neutral{\isaliteral{5F}{\isacharunderscore}}nat{\isaliteral{5F}{\isacharunderscore}}def\ \isacommand{by}\isamarkupfalse%
\ {\isaliteral{28}{\isacharparenleft}}induct\ n{\isaliteral{29}{\isacharparenright}}\ simp{\isaliteral{5F}{\isacharunderscore}}all\isanewline
\isacommand{next}\isamarkupfalse%
\isanewline
\ \ \isacommand{fix}\isamarkupfalse%
\ k\ {\isaliteral{3A}{\isacharcolon}}{\isaliteral{3A}{\isacharcolon}}\ int\isanewline
\ \ \isacommand{show}\isamarkupfalse%
\ {\isaliteral{22}{\isachardoublequoteopen}}k\ {\isaliteral{5C3C6F74696D65733E}{\isasymotimes}}\ {\isaliteral{5C3C6F6E653E}{\isasymone}}\ {\isaliteral{3D}{\isacharequal}}\ k{\isaliteral{22}{\isachardoublequoteclose}}\isanewline
\ \ \ \ \isacommand{unfolding}\isamarkupfalse%
\ neutral{\isaliteral{5F}{\isacharunderscore}}int{\isaliteral{5F}{\isacharunderscore}}def\ mult{\isaliteral{5F}{\isacharunderscore}}int{\isaliteral{5F}{\isacharunderscore}}def\ \isacommand{by}\isamarkupfalse%
\ simp\isanewline
\isacommand{qed}\isamarkupfalse%
\isanewline
\isanewline
\isacommand{end}\isamarkupfalse%
\isanewline
\isanewline
\isacommand{instantiation}\isamarkupfalse%
\ prod\ {\isaliteral{3A}{\isacharcolon}}{\isaliteral{3A}{\isacharcolon}}\ {\isaliteral{28}{\isacharparenleft}}monoid{\isaliteral{2C}{\isacharcomma}}\ monoid{\isaliteral{29}{\isacharparenright}}\ monoid\isanewline
\isakeyword{begin}\isanewline
\isanewline
\isacommand{instance}\isamarkupfalse%
\ \isacommand{proof}\isamarkupfalse%
\ \isanewline
\ \ \isacommand{fix}\isamarkupfalse%
\ p\ {\isaliteral{3A}{\isacharcolon}}{\isaliteral{3A}{\isacharcolon}}\ {\isaliteral{22}{\isachardoublequoteopen}}{\isaliteral{5C3C616C7068613E}{\isasymalpha}}{\isaliteral{5C3C436F6C6F6E3E}{\isasymColon}}monoid\ {\isaliteral{5C3C74696D65733E}{\isasymtimes}}\ {\isaliteral{5C3C626574613E}{\isasymbeta}}{\isaliteral{5C3C436F6C6F6E3E}{\isasymColon}}monoid{\isaliteral{22}{\isachardoublequoteclose}}\isanewline
\ \ \isacommand{show}\isamarkupfalse%
\ {\isaliteral{22}{\isachardoublequoteopen}}p\ {\isaliteral{5C3C6F74696D65733E}{\isasymotimes}}\ {\isaliteral{5C3C6F6E653E}{\isasymone}}\ {\isaliteral{3D}{\isacharequal}}\ p{\isaliteral{22}{\isachardoublequoteclose}}\isanewline
\ \ \ \ \isacommand{unfolding}\isamarkupfalse%
\ neutral{\isaliteral{5F}{\isacharunderscore}}prod{\isaliteral{5F}{\isacharunderscore}}def\ mult{\isaliteral{5F}{\isacharunderscore}}prod{\isaliteral{5F}{\isacharunderscore}}def\ \isacommand{by}\isamarkupfalse%
\ {\isaliteral{28}{\isacharparenleft}}simp\ add{\isaliteral{3A}{\isacharcolon}}\ neutr{\isaliteral{29}{\isacharparenright}}\isanewline
\isacommand{qed}\isamarkupfalse%
\isanewline
\isanewline
\isacommand{end}\isamarkupfalse%
%
\endisatagquote
{\isafoldquote}%
%
\isadelimquote
%
\endisadelimquote
%
\begin{isamarkuptext}%
\noindent To finish our small algebra example, we add a \isa{group} class with a corresponding instance:%
\end{isamarkuptext}%
\isamarkuptrue%
%
\isadelimquote
%
\endisadelimquote
%
\isatagquote
\isacommand{class}\isamarkupfalse%
\ group\ {\isaliteral{3D}{\isacharequal}}\ monoidl\ {\isaliteral{2B}{\isacharplus}}\isanewline
\ \ \isakeyword{fixes}\ inverse\ {\isaliteral{3A}{\isacharcolon}}{\isaliteral{3A}{\isacharcolon}}\ {\isaliteral{22}{\isachardoublequoteopen}}{\isaliteral{5C3C616C7068613E}{\isasymalpha}}\ {\isaliteral{5C3C52696768746172726F773E}{\isasymRightarrow}}\ {\isaliteral{5C3C616C7068613E}{\isasymalpha}}{\isaliteral{22}{\isachardoublequoteclose}}\ \ \ \ {\isaliteral{28}{\isacharparenleft}}{\isaliteral{22}{\isachardoublequoteopen}}{\isaliteral{28}{\isacharparenleft}}{\isaliteral{5F}{\isacharunderscore}}{\isaliteral{5C3C6469763E}{\isasymdiv}}{\isaliteral{29}{\isacharparenright}}{\isaliteral{22}{\isachardoublequoteclose}}\ {\isaliteral{5B}{\isacharbrackleft}}{\isadigit{1}}{\isadigit{0}}{\isadigit{0}}{\isadigit{0}}{\isaliteral{5D}{\isacharbrackright}}\ {\isadigit{9}}{\isadigit{9}}{\isadigit{9}}{\isaliteral{29}{\isacharparenright}}\isanewline
\ \ \isakeyword{assumes}\ invl{\isaliteral{3A}{\isacharcolon}}\ {\isaliteral{22}{\isachardoublequoteopen}}x{\isaliteral{5C3C6469763E}{\isasymdiv}}\ {\isaliteral{5C3C6F74696D65733E}{\isasymotimes}}\ x\ {\isaliteral{3D}{\isacharequal}}\ {\isaliteral{5C3C6F6E653E}{\isasymone}}{\isaliteral{22}{\isachardoublequoteclose}}\isanewline
\isanewline
\isacommand{instantiation}\isamarkupfalse%
\ int\ {\isaliteral{3A}{\isacharcolon}}{\isaliteral{3A}{\isacharcolon}}\ group\isanewline
\isakeyword{begin}\isanewline
\isanewline
\isacommand{definition}\isamarkupfalse%
\isanewline
\ \ inverse{\isaliteral{5F}{\isacharunderscore}}int{\isaliteral{5F}{\isacharunderscore}}def{\isaliteral{3A}{\isacharcolon}}\ {\isaliteral{22}{\isachardoublequoteopen}}i{\isaliteral{5C3C6469763E}{\isasymdiv}}\ {\isaliteral{3D}{\isacharequal}}\ {\isaliteral{2D}{\isacharminus}}\ {\isaliteral{28}{\isacharparenleft}}i{\isaliteral{5C3C436F6C6F6E3E}{\isasymColon}}int{\isaliteral{29}{\isacharparenright}}{\isaliteral{22}{\isachardoublequoteclose}}\isanewline
\isanewline
\isacommand{instance}\isamarkupfalse%
\ \isacommand{proof}\isamarkupfalse%
\isanewline
\ \ \isacommand{fix}\isamarkupfalse%
\ i\ {\isaliteral{3A}{\isacharcolon}}{\isaliteral{3A}{\isacharcolon}}\ int\isanewline
\ \ \isacommand{have}\isamarkupfalse%
\ {\isaliteral{22}{\isachardoublequoteopen}}{\isaliteral{2D}{\isacharminus}}i\ {\isaliteral{2B}{\isacharplus}}\ i\ {\isaliteral{3D}{\isacharequal}}\ {\isadigit{0}}{\isaliteral{22}{\isachardoublequoteclose}}\ \isacommand{by}\isamarkupfalse%
\ simp\isanewline
\ \ \isacommand{then}\isamarkupfalse%
\ \isacommand{show}\isamarkupfalse%
\ {\isaliteral{22}{\isachardoublequoteopen}}i{\isaliteral{5C3C6469763E}{\isasymdiv}}\ {\isaliteral{5C3C6F74696D65733E}{\isasymotimes}}\ i\ {\isaliteral{3D}{\isacharequal}}\ {\isaliteral{5C3C6F6E653E}{\isasymone}}{\isaliteral{22}{\isachardoublequoteclose}}\isanewline
\ \ \ \ \isacommand{unfolding}\isamarkupfalse%
\ mult{\isaliteral{5F}{\isacharunderscore}}int{\isaliteral{5F}{\isacharunderscore}}def\ neutral{\isaliteral{5F}{\isacharunderscore}}int{\isaliteral{5F}{\isacharunderscore}}def\ inverse{\isaliteral{5F}{\isacharunderscore}}int{\isaliteral{5F}{\isacharunderscore}}def\ \isacommand{{\isaliteral{2E}{\isachardot}}}\isamarkupfalse%
\isanewline
\isacommand{qed}\isamarkupfalse%
\isanewline
\isanewline
\isacommand{end}\isamarkupfalse%
%
\endisatagquote
{\isafoldquote}%
%
\isadelimquote
%
\endisadelimquote
%
\isamarkupsection{Type classes as locales%
}
\isamarkuptrue%
%
\isamarkupsubsection{A look behind the scenes%
}
\isamarkuptrue%
%
\begin{isamarkuptext}%
The example above gives an impression how Isar type classes work in
  practice.  As stated in the introduction, classes also provide a
  link to Isar's locale system.  Indeed, the logical core of a class
  is nothing other than a locale:%
\end{isamarkuptext}%
\isamarkuptrue%
%
\isadelimquote
%
\endisadelimquote
%
\isatagquote
\isacommand{class}\isamarkupfalse%
\ idem\ {\isaliteral{3D}{\isacharequal}}\isanewline
\ \ \isakeyword{fixes}\ f\ {\isaliteral{3A}{\isacharcolon}}{\isaliteral{3A}{\isacharcolon}}\ {\isaliteral{22}{\isachardoublequoteopen}}{\isaliteral{5C3C616C7068613E}{\isasymalpha}}\ {\isaliteral{5C3C52696768746172726F773E}{\isasymRightarrow}}\ {\isaliteral{5C3C616C7068613E}{\isasymalpha}}{\isaliteral{22}{\isachardoublequoteclose}}\isanewline
\ \ \isakeyword{assumes}\ idem{\isaliteral{3A}{\isacharcolon}}\ {\isaliteral{22}{\isachardoublequoteopen}}f\ {\isaliteral{28}{\isacharparenleft}}f\ x{\isaliteral{29}{\isacharparenright}}\ {\isaliteral{3D}{\isacharequal}}\ f\ x{\isaliteral{22}{\isachardoublequoteclose}}%
\endisatagquote
{\isafoldquote}%
%
\isadelimquote
%
\endisadelimquote
%
\begin{isamarkuptext}%
\noindent essentially introduces the locale%
\end{isamarkuptext}%
\isamarkuptrue%
\ %
\isadeliminvisible
%
\endisadeliminvisible
%
\isataginvisible
%
\endisataginvisible
{\isafoldinvisible}%
%
\isadeliminvisible
%
\endisadeliminvisible
%
\isadelimquote
%
\endisadelimquote
%
\isatagquote
\isacommand{locale}\isamarkupfalse%
\ idem\ {\isaliteral{3D}{\isacharequal}}\isanewline
\ \ \isakeyword{fixes}\ f\ {\isaliteral{3A}{\isacharcolon}}{\isaliteral{3A}{\isacharcolon}}\ {\isaliteral{22}{\isachardoublequoteopen}}{\isaliteral{5C3C616C7068613E}{\isasymalpha}}\ {\isaliteral{5C3C52696768746172726F773E}{\isasymRightarrow}}\ {\isaliteral{5C3C616C7068613E}{\isasymalpha}}{\isaliteral{22}{\isachardoublequoteclose}}\isanewline
\ \ \isakeyword{assumes}\ idem{\isaliteral{3A}{\isacharcolon}}\ {\isaliteral{22}{\isachardoublequoteopen}}f\ {\isaliteral{28}{\isacharparenleft}}f\ x{\isaliteral{29}{\isacharparenright}}\ {\isaliteral{3D}{\isacharequal}}\ f\ x{\isaliteral{22}{\isachardoublequoteclose}}%
\endisatagquote
{\isafoldquote}%
%
\isadelimquote
%
\endisadelimquote
%
\begin{isamarkuptext}%
\noindent together with corresponding constant(s):%
\end{isamarkuptext}%
\isamarkuptrue%
%
\isadelimquote
%
\endisadelimquote
%
\isatagquote
\isacommand{consts}\isamarkupfalse%
\ f\ {\isaliteral{3A}{\isacharcolon}}{\isaliteral{3A}{\isacharcolon}}\ {\isaliteral{22}{\isachardoublequoteopen}}{\isaliteral{5C3C616C7068613E}{\isasymalpha}}\ {\isaliteral{5C3C52696768746172726F773E}{\isasymRightarrow}}\ {\isaliteral{5C3C616C7068613E}{\isasymalpha}}{\isaliteral{22}{\isachardoublequoteclose}}%
\endisatagquote
{\isafoldquote}%
%
\isadelimquote
%
\endisadelimquote
%
\begin{isamarkuptext}%
\noindent The connection to the type system is done by means
  of a primitive type class%
\end{isamarkuptext}%
\isamarkuptrue%
\ %
\isadeliminvisible
%
\endisadeliminvisible
%
\isataginvisible
%
\endisataginvisible
{\isafoldinvisible}%
%
\isadeliminvisible
%
\endisadeliminvisible
%
\isadelimquote
%
\endisadelimquote
%
\isatagquote
\isacommand{classes}\isamarkupfalse%
\ idem\ {\isaliteral{3C}{\isacharless}}\ type%
\endisatagquote
{\isafoldquote}%
%
\isadelimquote
%
\endisadelimquote
%
\isadeliminvisible
%
\endisadeliminvisible
%
\isataginvisible
%
\endisataginvisible
{\isafoldinvisible}%
%
\isadeliminvisible
%
\endisadeliminvisible
%
\begin{isamarkuptext}%
\noindent together with a corresponding interpretation:%
\end{isamarkuptext}%
\isamarkuptrue%
%
\isadelimquote
%
\endisadelimquote
%
\isatagquote
\isacommand{interpretation}\isamarkupfalse%
\ idem{\isaliteral{5F}{\isacharunderscore}}class{\isaliteral{3A}{\isacharcolon}}\isanewline
\ \ idem\ {\isaliteral{22}{\isachardoublequoteopen}}f\ {\isaliteral{5C3C436F6C6F6E3E}{\isasymColon}}\ {\isaliteral{28}{\isacharparenleft}}{\isaliteral{5C3C616C7068613E}{\isasymalpha}}{\isaliteral{5C3C436F6C6F6E3E}{\isasymColon}}idem{\isaliteral{29}{\isacharparenright}}\ {\isaliteral{5C3C52696768746172726F773E}{\isasymRightarrow}}\ {\isaliteral{5C3C616C7068613E}{\isasymalpha}}{\isaliteral{22}{\isachardoublequoteclose}}%
\endisatagquote
{\isafoldquote}%
%
\isadelimquote
%
\endisadelimquote
%
\begin{isamarkuptext}%
\noindent This gives you the full power of the Isabelle module system;
  conclusions in locale \isa{idem} are implicitly propagated
  to class \isa{idem}.%
\end{isamarkuptext}%
\isamarkuptrue%
\ %
\isadeliminvisible
%
\endisadeliminvisible
%
\isataginvisible
%
\endisataginvisible
{\isafoldinvisible}%
%
\isadeliminvisible
%
\endisadeliminvisible
%
\isamarkupsubsection{Abstract reasoning%
}
\isamarkuptrue%
%
\begin{isamarkuptext}%
Isabelle locales enable reasoning at a general level, while results
  are implicitly transferred to all instances.  For example, we can
  now establish the \isa{left{\isaliteral{5F}{\isacharunderscore}}cancel} lemma for groups, which
  states that the function \isa{{\isaliteral{28}{\isacharparenleft}}x\ {\isaliteral{5C3C6F74696D65733E}{\isasymotimes}}{\isaliteral{29}{\isacharparenright}}} is injective:%
\end{isamarkuptext}%
\isamarkuptrue%
%
\isadelimquote
%
\endisadelimquote
%
\isatagquote
\isacommand{lemma}\isamarkupfalse%
\ {\isaliteral{28}{\isacharparenleft}}\isakeyword{in}\ group{\isaliteral{29}{\isacharparenright}}\ left{\isaliteral{5F}{\isacharunderscore}}cancel{\isaliteral{3A}{\isacharcolon}}\ {\isaliteral{22}{\isachardoublequoteopen}}x\ {\isaliteral{5C3C6F74696D65733E}{\isasymotimes}}\ y\ {\isaliteral{3D}{\isacharequal}}\ x\ {\isaliteral{5C3C6F74696D65733E}{\isasymotimes}}\ z\ {\isaliteral{5C3C6C6F6E676C65667472696768746172726F773E}{\isasymlongleftrightarrow}}\ y\ {\isaliteral{3D}{\isacharequal}}\ z{\isaliteral{22}{\isachardoublequoteclose}}\isanewline
\isacommand{proof}\isamarkupfalse%
\isanewline
\ \ \isacommand{assume}\isamarkupfalse%
\ {\isaliteral{22}{\isachardoublequoteopen}}x\ {\isaliteral{5C3C6F74696D65733E}{\isasymotimes}}\ y\ {\isaliteral{3D}{\isacharequal}}\ x\ {\isaliteral{5C3C6F74696D65733E}{\isasymotimes}}\ z{\isaliteral{22}{\isachardoublequoteclose}}\isanewline
\ \ \isacommand{then}\isamarkupfalse%
\ \isacommand{have}\isamarkupfalse%
\ {\isaliteral{22}{\isachardoublequoteopen}}x{\isaliteral{5C3C6469763E}{\isasymdiv}}\ {\isaliteral{5C3C6F74696D65733E}{\isasymotimes}}\ {\isaliteral{28}{\isacharparenleft}}x\ {\isaliteral{5C3C6F74696D65733E}{\isasymotimes}}\ y{\isaliteral{29}{\isacharparenright}}\ {\isaliteral{3D}{\isacharequal}}\ x{\isaliteral{5C3C6469763E}{\isasymdiv}}\ {\isaliteral{5C3C6F74696D65733E}{\isasymotimes}}\ {\isaliteral{28}{\isacharparenleft}}x\ {\isaliteral{5C3C6F74696D65733E}{\isasymotimes}}\ z{\isaliteral{29}{\isacharparenright}}{\isaliteral{22}{\isachardoublequoteclose}}\ \isacommand{by}\isamarkupfalse%
\ simp\isanewline
\ \ \isacommand{then}\isamarkupfalse%
\ \isacommand{have}\isamarkupfalse%
\ {\isaliteral{22}{\isachardoublequoteopen}}{\isaliteral{28}{\isacharparenleft}}x{\isaliteral{5C3C6469763E}{\isasymdiv}}\ {\isaliteral{5C3C6F74696D65733E}{\isasymotimes}}\ x{\isaliteral{29}{\isacharparenright}}\ {\isaliteral{5C3C6F74696D65733E}{\isasymotimes}}\ y\ {\isaliteral{3D}{\isacharequal}}\ {\isaliteral{28}{\isacharparenleft}}x{\isaliteral{5C3C6469763E}{\isasymdiv}}\ {\isaliteral{5C3C6F74696D65733E}{\isasymotimes}}\ x{\isaliteral{29}{\isacharparenright}}\ {\isaliteral{5C3C6F74696D65733E}{\isasymotimes}}\ z{\isaliteral{22}{\isachardoublequoteclose}}\ \isacommand{using}\isamarkupfalse%
\ assoc\ \isacommand{by}\isamarkupfalse%
\ simp\isanewline
\ \ \isacommand{then}\isamarkupfalse%
\ \isacommand{show}\isamarkupfalse%
\ {\isaliteral{22}{\isachardoublequoteopen}}y\ {\isaliteral{3D}{\isacharequal}}\ z{\isaliteral{22}{\isachardoublequoteclose}}\ \isacommand{using}\isamarkupfalse%
\ neutl\ \isakeyword{and}\ invl\ \isacommand{by}\isamarkupfalse%
\ simp\isanewline
\isacommand{next}\isamarkupfalse%
\isanewline
\ \ \isacommand{assume}\isamarkupfalse%
\ {\isaliteral{22}{\isachardoublequoteopen}}y\ {\isaliteral{3D}{\isacharequal}}\ z{\isaliteral{22}{\isachardoublequoteclose}}\isanewline
\ \ \isacommand{then}\isamarkupfalse%
\ \isacommand{show}\isamarkupfalse%
\ {\isaliteral{22}{\isachardoublequoteopen}}x\ {\isaliteral{5C3C6F74696D65733E}{\isasymotimes}}\ y\ {\isaliteral{3D}{\isacharequal}}\ x\ {\isaliteral{5C3C6F74696D65733E}{\isasymotimes}}\ z{\isaliteral{22}{\isachardoublequoteclose}}\ \isacommand{by}\isamarkupfalse%
\ simp\isanewline
\isacommand{qed}\isamarkupfalse%
%
\endisatagquote
{\isafoldquote}%
%
\isadelimquote
%
\endisadelimquote
%
\begin{isamarkuptext}%
\noindent Here the \qt{\hyperlink{keyword.in}{\mbox{\isa{\isakeyword{in}}}} \isa{group}} target
  specification indicates that the result is recorded within that
  context for later use.  This local theorem is also lifted to the
  global one \hyperlink{fact.group.left-cancel:}{\mbox{\isa{group{\isaliteral{2E}{\isachardot}}left{\isaliteral{5F}{\isacharunderscore}}cancel{\isaliteral{3A}{\isacharcolon}}}}} \isa{{\isaliteral{22}{\isachardoublequote}}{\isaliteral{5C3C416E643E}{\isasymAnd}}x\ y\ z\ {\isaliteral{5C3C436F6C6F6E3E}{\isasymColon}}\ {\isaliteral{5C3C616C7068613E}{\isasymalpha}}{\isaliteral{5C3C436F6C6F6E3E}{\isasymColon}}group{\isaliteral{2E}{\isachardot}}\ x\ {\isaliteral{5C3C6F74696D65733E}{\isasymotimes}}\ y\ {\isaliteral{3D}{\isacharequal}}\ x\ {\isaliteral{5C3C6F74696D65733E}{\isasymotimes}}\ z\ {\isaliteral{5C3C6C6F6E676C65667472696768746172726F773E}{\isasymlongleftrightarrow}}\ y\ {\isaliteral{3D}{\isacharequal}}\ z{\isaliteral{22}{\isachardoublequote}}}.  Since type \isa{int} has been
  made an instance of \isa{group} before, we may refer to that
  fact as well: \isa{{\isaliteral{22}{\isachardoublequote}}{\isaliteral{5C3C416E643E}{\isasymAnd}}x\ y\ z\ {\isaliteral{5C3C436F6C6F6E3E}{\isasymColon}}\ int{\isaliteral{2E}{\isachardot}}\ x\ {\isaliteral{5C3C6F74696D65733E}{\isasymotimes}}\ y\ {\isaliteral{3D}{\isacharequal}}\ x\ {\isaliteral{5C3C6F74696D65733E}{\isasymotimes}}\ z\ {\isaliteral{5C3C6C6F6E676C65667472696768746172726F773E}{\isasymlongleftrightarrow}}\ y\ {\isaliteral{3D}{\isacharequal}}\ z{\isaliteral{22}{\isachardoublequote}}}.%
\end{isamarkuptext}%
\isamarkuptrue%
%
\isamarkupsubsection{Derived definitions%
}
\isamarkuptrue%
%
\begin{isamarkuptext}%
Isabelle locales are targets which support local definitions:%
\end{isamarkuptext}%
\isamarkuptrue%
%
\isadelimquote
%
\endisadelimquote
%
\isatagquote
\isacommand{primrec}\isamarkupfalse%
\ {\isaliteral{28}{\isacharparenleft}}\isakeyword{in}\ monoid{\isaliteral{29}{\isacharparenright}}\ pow{\isaliteral{5F}{\isacharunderscore}}nat\ {\isaliteral{3A}{\isacharcolon}}{\isaliteral{3A}{\isacharcolon}}\ {\isaliteral{22}{\isachardoublequoteopen}}nat\ {\isaliteral{5C3C52696768746172726F773E}{\isasymRightarrow}}\ {\isaliteral{5C3C616C7068613E}{\isasymalpha}}\ {\isaliteral{5C3C52696768746172726F773E}{\isasymRightarrow}}\ {\isaliteral{5C3C616C7068613E}{\isasymalpha}}{\isaliteral{22}{\isachardoublequoteclose}}\ \isakeyword{where}\isanewline
\ \ {\isaliteral{22}{\isachardoublequoteopen}}pow{\isaliteral{5F}{\isacharunderscore}}nat\ {\isadigit{0}}\ x\ {\isaliteral{3D}{\isacharequal}}\ {\isaliteral{5C3C6F6E653E}{\isasymone}}{\isaliteral{22}{\isachardoublequoteclose}}\isanewline
\ \ {\isaliteral{7C}{\isacharbar}}\ {\isaliteral{22}{\isachardoublequoteopen}}pow{\isaliteral{5F}{\isacharunderscore}}nat\ {\isaliteral{28}{\isacharparenleft}}Suc\ n{\isaliteral{29}{\isacharparenright}}\ x\ {\isaliteral{3D}{\isacharequal}}\ x\ {\isaliteral{5C3C6F74696D65733E}{\isasymotimes}}\ pow{\isaliteral{5F}{\isacharunderscore}}nat\ n\ x{\isaliteral{22}{\isachardoublequoteclose}}%
\endisatagquote
{\isafoldquote}%
%
\isadelimquote
%
\endisadelimquote
%
\begin{isamarkuptext}%
\noindent If the locale \isa{group} is also a class, this local
  definition is propagated onto a global definition of \isa{{\isaliteral{22}{\isachardoublequote}}pow{\isaliteral{5F}{\isacharunderscore}}nat\ {\isaliteral{5C3C436F6C6F6E3E}{\isasymColon}}\ nat\ {\isaliteral{5C3C52696768746172726F773E}{\isasymRightarrow}}\ {\isaliteral{5C3C616C7068613E}{\isasymalpha}}{\isaliteral{5C3C436F6C6F6E3E}{\isasymColon}}monoid\ {\isaliteral{5C3C52696768746172726F773E}{\isasymRightarrow}}\ {\isaliteral{5C3C616C7068613E}{\isasymalpha}}{\isaliteral{5C3C436F6C6F6E3E}{\isasymColon}}monoid{\isaliteral{22}{\isachardoublequote}}} with corresponding theorems

  \isa{pow{\isaliteral{5F}{\isacharunderscore}}nat\ {\isadigit{0}}\ x\ {\isaliteral{3D}{\isacharequal}}\ {\isaliteral{5C3C6F6E653E}{\isasymone}}\isasep\isanewline%
pow{\isaliteral{5F}{\isacharunderscore}}nat\ {\isaliteral{28}{\isacharparenleft}}Suc\ n{\isaliteral{29}{\isacharparenright}}\ x\ {\isaliteral{3D}{\isacharequal}}\ x\ {\isaliteral{5C3C6F74696D65733E}{\isasymotimes}}\ pow{\isaliteral{5F}{\isacharunderscore}}nat\ n\ x}.

  \noindent As you can see from this example, for local definitions
  you may use any specification tool which works together with
  locales, such as Krauss's recursive function package
  \cite{krauss2006}.%
\end{isamarkuptext}%
\isamarkuptrue%
%
\isamarkupsubsection{A functor analogy%
}
\isamarkuptrue%
%
\begin{isamarkuptext}%
We introduced Isar classes by analogy to type classes in functional
  programming; if we reconsider this in the context of what has been
  said about type classes and locales, we can drive this analogy
  further by stating that type classes essentially correspond to
  functors that have a canonical interpretation as type classes.
  There is also the possibility of other interpretations.  For
  example, \isa{list}s also form a monoid with \isa{append} and
  \isa{{\isaliteral{5B}{\isacharbrackleft}}{\isaliteral{5D}{\isacharbrackright}}} as operations, but it seems inappropriate to apply to
  lists the same operations as for genuinely algebraic types.  In such
  a case, we can simply make a particular interpretation of monoids
  for lists:%
\end{isamarkuptext}%
\isamarkuptrue%
%
\isadelimquote
%
\endisadelimquote
%
\isatagquote
\isacommand{interpretation}\isamarkupfalse%
\ list{\isaliteral{5F}{\isacharunderscore}}monoid{\isaliteral{3A}{\isacharcolon}}\ monoid\ append\ {\isaliteral{22}{\isachardoublequoteopen}}{\isaliteral{5B}{\isacharbrackleft}}{\isaliteral{5D}{\isacharbrackright}}{\isaliteral{22}{\isachardoublequoteclose}}\isanewline
\ \ \isacommand{proof}\isamarkupfalse%
\ \isacommand{qed}\isamarkupfalse%
\ auto%
\endisatagquote
{\isafoldquote}%
%
\isadelimquote
%
\endisadelimquote
%
\begin{isamarkuptext}%
\noindent This enables us to apply facts on monoids
  to lists, e.g. \isa{{\isaliteral{5B}{\isacharbrackleft}}{\isaliteral{5D}{\isacharbrackright}}\ {\isaliteral{40}{\isacharat}}\ x\ {\isaliteral{3D}{\isacharequal}}\ x}.

  When using this interpretation pattern, it may also
  be appropriate to map derived definitions accordingly:%
\end{isamarkuptext}%
\isamarkuptrue%
%
\isadelimquote
%
\endisadelimquote
%
\isatagquote
\isacommand{primrec}\isamarkupfalse%
\ replicate\ {\isaliteral{3A}{\isacharcolon}}{\isaliteral{3A}{\isacharcolon}}\ {\isaliteral{22}{\isachardoublequoteopen}}nat\ {\isaliteral{5C3C52696768746172726F773E}{\isasymRightarrow}}\ {\isaliteral{5C3C616C7068613E}{\isasymalpha}}\ list\ {\isaliteral{5C3C52696768746172726F773E}{\isasymRightarrow}}\ {\isaliteral{5C3C616C7068613E}{\isasymalpha}}\ list{\isaliteral{22}{\isachardoublequoteclose}}\ \isakeyword{where}\isanewline
\ \ {\isaliteral{22}{\isachardoublequoteopen}}replicate\ {\isadigit{0}}\ {\isaliteral{5F}{\isacharunderscore}}\ {\isaliteral{3D}{\isacharequal}}\ {\isaliteral{5B}{\isacharbrackleft}}{\isaliteral{5D}{\isacharbrackright}}{\isaliteral{22}{\isachardoublequoteclose}}\isanewline
\ \ {\isaliteral{7C}{\isacharbar}}\ {\isaliteral{22}{\isachardoublequoteopen}}replicate\ {\isaliteral{28}{\isacharparenleft}}Suc\ n{\isaliteral{29}{\isacharparenright}}\ xs\ {\isaliteral{3D}{\isacharequal}}\ xs\ {\isaliteral{40}{\isacharat}}\ replicate\ n\ xs{\isaliteral{22}{\isachardoublequoteclose}}\isanewline
\isanewline
\isacommand{interpretation}\isamarkupfalse%
\ list{\isaliteral{5F}{\isacharunderscore}}monoid{\isaliteral{3A}{\isacharcolon}}\ monoid\ append\ {\isaliteral{22}{\isachardoublequoteopen}}{\isaliteral{5B}{\isacharbrackleft}}{\isaliteral{5D}{\isacharbrackright}}{\isaliteral{22}{\isachardoublequoteclose}}\ \isakeyword{where}\isanewline
\ \ {\isaliteral{22}{\isachardoublequoteopen}}monoid{\isaliteral{2E}{\isachardot}}pow{\isaliteral{5F}{\isacharunderscore}}nat\ append\ {\isaliteral{5B}{\isacharbrackleft}}{\isaliteral{5D}{\isacharbrackright}}\ {\isaliteral{3D}{\isacharequal}}\ replicate{\isaliteral{22}{\isachardoublequoteclose}}\isanewline
\isacommand{proof}\isamarkupfalse%
\ {\isaliteral{2D}{\isacharminus}}\isanewline
\ \ \isacommand{interpret}\isamarkupfalse%
\ monoid\ append\ {\isaliteral{22}{\isachardoublequoteopen}}{\isaliteral{5B}{\isacharbrackleft}}{\isaliteral{5D}{\isacharbrackright}}{\isaliteral{22}{\isachardoublequoteclose}}\ \isacommand{{\isaliteral{2E}{\isachardot}}{\isaliteral{2E}{\isachardot}}}\isamarkupfalse%
\isanewline
\ \ \isacommand{show}\isamarkupfalse%
\ {\isaliteral{22}{\isachardoublequoteopen}}monoid{\isaliteral{2E}{\isachardot}}pow{\isaliteral{5F}{\isacharunderscore}}nat\ append\ {\isaliteral{5B}{\isacharbrackleft}}{\isaliteral{5D}{\isacharbrackright}}\ {\isaliteral{3D}{\isacharequal}}\ replicate{\isaliteral{22}{\isachardoublequoteclose}}\isanewline
\ \ \isacommand{proof}\isamarkupfalse%
\isanewline
\ \ \ \ \isacommand{fix}\isamarkupfalse%
\ n\isanewline
\ \ \ \ \isacommand{show}\isamarkupfalse%
\ {\isaliteral{22}{\isachardoublequoteopen}}monoid{\isaliteral{2E}{\isachardot}}pow{\isaliteral{5F}{\isacharunderscore}}nat\ append\ {\isaliteral{5B}{\isacharbrackleft}}{\isaliteral{5D}{\isacharbrackright}}\ n\ {\isaliteral{3D}{\isacharequal}}\ replicate\ n{\isaliteral{22}{\isachardoublequoteclose}}\isanewline
\ \ \ \ \ \ \isacommand{by}\isamarkupfalse%
\ {\isaliteral{28}{\isacharparenleft}}induct\ n{\isaliteral{29}{\isacharparenright}}\ auto\isanewline
\ \ \isacommand{qed}\isamarkupfalse%
\isanewline
\isacommand{qed}\isamarkupfalse%
\ intro{\isaliteral{5F}{\isacharunderscore}}locales%
\endisatagquote
{\isafoldquote}%
%
\isadelimquote
%
\endisadelimquote
%
\begin{isamarkuptext}%
\noindent This pattern is also helpful to reuse abstract
  specifications on the \emph{same} type.  For example, think of a
  class \isa{preorder}; for type \isa{nat}, there are at least two
  possible instances: the natural order or the order induced by the
  divides relation.  But only one of these instances can be used for
  \hyperlink{command.instantiation}{\mbox{\isa{\isacommand{instantiation}}}}; using the locale behind the class \isa{preorder}, it is still possible to utilise the same abstract
  specification again using \hyperlink{command.interpretation}{\mbox{\isa{\isacommand{interpretation}}}}.%
\end{isamarkuptext}%
\isamarkuptrue%
%
\isamarkupsubsection{Additional subclass relations%
}
\isamarkuptrue%
%
\begin{isamarkuptext}%
Any \isa{group} is also a \isa{monoid}; this can be made
  explicit by claiming an additional subclass relation, together with
  a proof of the logical difference:%
\end{isamarkuptext}%
\isamarkuptrue%
%
\isadelimquote
%
\endisadelimquote
%
\isatagquote
\isacommand{subclass}\isamarkupfalse%
\ {\isaliteral{28}{\isacharparenleft}}\isakeyword{in}\ group{\isaliteral{29}{\isacharparenright}}\ monoid\isanewline
\isacommand{proof}\isamarkupfalse%
\isanewline
\ \ \isacommand{fix}\isamarkupfalse%
\ x\isanewline
\ \ \isacommand{from}\isamarkupfalse%
\ invl\ \isacommand{have}\isamarkupfalse%
\ {\isaliteral{22}{\isachardoublequoteopen}}x{\isaliteral{5C3C6469763E}{\isasymdiv}}\ {\isaliteral{5C3C6F74696D65733E}{\isasymotimes}}\ x\ {\isaliteral{3D}{\isacharequal}}\ {\isaliteral{5C3C6F6E653E}{\isasymone}}{\isaliteral{22}{\isachardoublequoteclose}}\ \isacommand{by}\isamarkupfalse%
\ simp\isanewline
\ \ \isacommand{with}\isamarkupfalse%
\ assoc\ {\isaliteral{5B}{\isacharbrackleft}}symmetric{\isaliteral{5D}{\isacharbrackright}}\ neutl\ invl\ \isacommand{have}\isamarkupfalse%
\ {\isaliteral{22}{\isachardoublequoteopen}}x{\isaliteral{5C3C6469763E}{\isasymdiv}}\ {\isaliteral{5C3C6F74696D65733E}{\isasymotimes}}\ {\isaliteral{28}{\isacharparenleft}}x\ {\isaliteral{5C3C6F74696D65733E}{\isasymotimes}}\ {\isaliteral{5C3C6F6E653E}{\isasymone}}{\isaliteral{29}{\isacharparenright}}\ {\isaliteral{3D}{\isacharequal}}\ x{\isaliteral{5C3C6469763E}{\isasymdiv}}\ {\isaliteral{5C3C6F74696D65733E}{\isasymotimes}}\ x{\isaliteral{22}{\isachardoublequoteclose}}\ \isacommand{by}\isamarkupfalse%
\ simp\isanewline
\ \ \isacommand{with}\isamarkupfalse%
\ left{\isaliteral{5F}{\isacharunderscore}}cancel\ \isacommand{show}\isamarkupfalse%
\ {\isaliteral{22}{\isachardoublequoteopen}}x\ {\isaliteral{5C3C6F74696D65733E}{\isasymotimes}}\ {\isaliteral{5C3C6F6E653E}{\isasymone}}\ {\isaliteral{3D}{\isacharequal}}\ x{\isaliteral{22}{\isachardoublequoteclose}}\ \isacommand{by}\isamarkupfalse%
\ simp\isanewline
\isacommand{qed}\isamarkupfalse%
%
\endisatagquote
{\isafoldquote}%
%
\isadelimquote
%
\endisadelimquote
%
\begin{isamarkuptext}%
The logical proof is carried out on the locale level.  Afterwards it
  is propagated to the type system, making \isa{group} an instance
  of \isa{monoid} by adding an additional edge to the graph of
  subclass relations (\figref{fig:subclass}).

  \begin{figure}[htbp]
   \begin{center}
     \small
     \unitlength 0.6mm
     \begin{picture}(40,60)(0,0)
       \put(20,60){\makebox(0,0){\isa{semigroup}}}
       \put(20,40){\makebox(0,0){\isa{monoidl}}}
       \put(00,20){\makebox(0,0){\isa{monoid}}}
       \put(40,00){\makebox(0,0){\isa{group}}}
       \put(20,55){\vector(0,-1){10}}
       \put(15,35){\vector(-1,-1){10}}
       \put(25,35){\vector(1,-3){10}}
     \end{picture}
     \hspace{8em}
     \begin{picture}(40,60)(0,0)
       \put(20,60){\makebox(0,0){\isa{semigroup}}}
       \put(20,40){\makebox(0,0){\isa{monoidl}}}
       \put(00,20){\makebox(0,0){\isa{monoid}}}
       \put(40,00){\makebox(0,0){\isa{group}}}
       \put(20,55){\vector(0,-1){10}}
       \put(15,35){\vector(-1,-1){10}}
       \put(05,15){\vector(3,-1){30}}
     \end{picture}
     \caption{Subclass relationship of monoids and groups:
        before and after establishing the relationship
        \isa{group\ {\isaliteral{5C3C73756273657465713E}{\isasymsubseteq}}\ monoid};  transitive edges are left out.}
     \label{fig:subclass}
   \end{center}
  \end{figure}

  For illustration, a derived definition in \isa{group} using \isa{pow{\isaliteral{5F}{\isacharunderscore}}nat}%
\end{isamarkuptext}%
\isamarkuptrue%
%
\isadelimquote
%
\endisadelimquote
%
\isatagquote
\isacommand{definition}\isamarkupfalse%
\ {\isaliteral{28}{\isacharparenleft}}\isakeyword{in}\ group{\isaliteral{29}{\isacharparenright}}\ pow{\isaliteral{5F}{\isacharunderscore}}int\ {\isaliteral{3A}{\isacharcolon}}{\isaliteral{3A}{\isacharcolon}}\ {\isaliteral{22}{\isachardoublequoteopen}}int\ {\isaliteral{5C3C52696768746172726F773E}{\isasymRightarrow}}\ {\isaliteral{5C3C616C7068613E}{\isasymalpha}}\ {\isaliteral{5C3C52696768746172726F773E}{\isasymRightarrow}}\ {\isaliteral{5C3C616C7068613E}{\isasymalpha}}{\isaliteral{22}{\isachardoublequoteclose}}\ \isakeyword{where}\isanewline
\ \ {\isaliteral{22}{\isachardoublequoteopen}}pow{\isaliteral{5F}{\isacharunderscore}}int\ k\ x\ {\isaliteral{3D}{\isacharequal}}\ {\isaliteral{28}{\isacharparenleft}}if\ k\ {\isaliteral{3E}{\isachargreater}}{\isaliteral{3D}{\isacharequal}}\ {\isadigit{0}}\isanewline
\ \ \ \ then\ pow{\isaliteral{5F}{\isacharunderscore}}nat\ {\isaliteral{28}{\isacharparenleft}}nat\ k{\isaliteral{29}{\isacharparenright}}\ x\isanewline
\ \ \ \ else\ {\isaliteral{28}{\isacharparenleft}}pow{\isaliteral{5F}{\isacharunderscore}}nat\ {\isaliteral{28}{\isacharparenleft}}nat\ {\isaliteral{28}{\isacharparenleft}}{\isaliteral{2D}{\isacharminus}}\ k{\isaliteral{29}{\isacharparenright}}{\isaliteral{29}{\isacharparenright}}\ x{\isaliteral{29}{\isacharparenright}}{\isaliteral{5C3C6469763E}{\isasymdiv}}{\isaliteral{29}{\isacharparenright}}{\isaliteral{22}{\isachardoublequoteclose}}%
\endisatagquote
{\isafoldquote}%
%
\isadelimquote
%
\endisadelimquote
%
\begin{isamarkuptext}%
\noindent yields the global definition of \isa{{\isaliteral{22}{\isachardoublequote}}pow{\isaliteral{5F}{\isacharunderscore}}int\ {\isaliteral{5C3C436F6C6F6E3E}{\isasymColon}}\ int\ {\isaliteral{5C3C52696768746172726F773E}{\isasymRightarrow}}\ {\isaliteral{5C3C616C7068613E}{\isasymalpha}}{\isaliteral{5C3C436F6C6F6E3E}{\isasymColon}}group\ {\isaliteral{5C3C52696768746172726F773E}{\isasymRightarrow}}\ {\isaliteral{5C3C616C7068613E}{\isasymalpha}}{\isaliteral{5C3C436F6C6F6E3E}{\isasymColon}}group{\isaliteral{22}{\isachardoublequote}}} with the corresponding theorem \isa{pow{\isaliteral{5F}{\isacharunderscore}}int\ k\ x\ {\isaliteral{3D}{\isacharequal}}\ {\isaliteral{28}{\isacharparenleft}}if\ {\isadigit{0}}\ {\isaliteral{5C3C6C653E}{\isasymle}}\ k\ then\ pow{\isaliteral{5F}{\isacharunderscore}}nat\ {\isaliteral{28}{\isacharparenleft}}nat\ k{\isaliteral{29}{\isacharparenright}}\ x\ else\ {\isaliteral{28}{\isacharparenleft}}pow{\isaliteral{5F}{\isacharunderscore}}nat\ {\isaliteral{28}{\isacharparenleft}}nat\ {\isaliteral{28}{\isacharparenleft}}{\isaliteral{2D}{\isacharminus}}\ k{\isaliteral{29}{\isacharparenright}}{\isaliteral{29}{\isacharparenright}}\ x{\isaliteral{29}{\isacharparenright}}{\isaliteral{5C3C6469763E}{\isasymdiv}}{\isaliteral{29}{\isacharparenright}}}.%
\end{isamarkuptext}%
\isamarkuptrue%
%
\isamarkupsubsection{A note on syntax%
}
\isamarkuptrue%
%
\begin{isamarkuptext}%
As a convenience, class context syntax allows references to local
  class operations and their global counterparts uniformly; type
  inference resolves ambiguities.  For example:%
\end{isamarkuptext}%
\isamarkuptrue%
%
\isadelimquote
%
\endisadelimquote
%
\isatagquote
\isacommand{context}\isamarkupfalse%
\ semigroup\isanewline
\isakeyword{begin}\isanewline
\isanewline
\isacommand{term}\isamarkupfalse%
\ {\isaliteral{22}{\isachardoublequoteopen}}x\ {\isaliteral{5C3C6F74696D65733E}{\isasymotimes}}\ y{\isaliteral{22}{\isachardoublequoteclose}}\ %
\isamarkupcmt{example 1%
}
\isanewline
\isacommand{term}\isamarkupfalse%
\ {\isaliteral{22}{\isachardoublequoteopen}}{\isaliteral{28}{\isacharparenleft}}x{\isaliteral{5C3C436F6C6F6E3E}{\isasymColon}}nat{\isaliteral{29}{\isacharparenright}}\ {\isaliteral{5C3C6F74696D65733E}{\isasymotimes}}\ y{\isaliteral{22}{\isachardoublequoteclose}}\ %
\isamarkupcmt{example 2%
}
\isanewline
\isanewline
\isacommand{end}\isamarkupfalse%
\isanewline
\isanewline
\isacommand{term}\isamarkupfalse%
\ {\isaliteral{22}{\isachardoublequoteopen}}x\ {\isaliteral{5C3C6F74696D65733E}{\isasymotimes}}\ y{\isaliteral{22}{\isachardoublequoteclose}}\ %
\isamarkupcmt{example 3%
}
%
\endisatagquote
{\isafoldquote}%
%
\isadelimquote
%
\endisadelimquote
%
\begin{isamarkuptext}%
\noindent Here in example 1, the term refers to the local class
  operation \isa{mult\ {\isaliteral{5B}{\isacharbrackleft}}{\isaliteral{5C3C616C7068613E}{\isasymalpha}}{\isaliteral{5D}{\isacharbrackright}}}, whereas in example 2 the type
  constraint enforces the global class operation \isa{mult\ {\isaliteral{5B}{\isacharbrackleft}}nat{\isaliteral{5D}{\isacharbrackright}}}.
  In the global context in example 3, the reference is to the
  polymorphic global class operation \isa{mult\ {\isaliteral{5B}{\isacharbrackleft}}{\isaliteral{3F}{\isacharquery}}{\isaliteral{5C3C616C7068613E}{\isasymalpha}}\ {\isaliteral{5C3C436F6C6F6E3E}{\isasymColon}}\ semigroup{\isaliteral{5D}{\isacharbrackright}}}.%
\end{isamarkuptext}%
\isamarkuptrue%
%
\isamarkupsection{Further issues%
}
\isamarkuptrue%
%
\isamarkupsubsection{Type classes and code generation%
}
\isamarkuptrue%
%
\begin{isamarkuptext}%
Turning back to the first motivation for type classes, namely
  overloading, it is obvious that overloading stemming from \hyperlink{command.class}{\mbox{\isa{\isacommand{class}}}} statements and \hyperlink{command.instantiation}{\mbox{\isa{\isacommand{instantiation}}}} targets naturally
  maps to Haskell type classes.  The code generator framework
  \cite{isabelle-codegen} takes this into account.  If the target
  language (e.g.~SML) lacks type classes, then they are implemented by
  an explicit dictionary construction.  As example, let's go back to
  the power function:%
\end{isamarkuptext}%
\isamarkuptrue%
%
\isadelimquote
%
\endisadelimquote
%
\isatagquote
\isacommand{definition}\isamarkupfalse%
\ example\ {\isaliteral{3A}{\isacharcolon}}{\isaliteral{3A}{\isacharcolon}}\ int\ \isakeyword{where}\isanewline
\ \ {\isaliteral{22}{\isachardoublequoteopen}}example\ {\isaliteral{3D}{\isacharequal}}\ pow{\isaliteral{5F}{\isacharunderscore}}int\ {\isadigit{1}}{\isadigit{0}}\ {\isaliteral{28}{\isacharparenleft}}{\isaliteral{2D}{\isacharminus}}{\isadigit{2}}{\isaliteral{29}{\isacharparenright}}{\isaliteral{22}{\isachardoublequoteclose}}%
\endisatagquote
{\isafoldquote}%
%
\isadelimquote
%
\endisadelimquote
%
\begin{isamarkuptext}%
\noindent This maps to Haskell as follows:%
\end{isamarkuptext}%
\isamarkuptrue%
%
\isadeliminvisible
%
\endisadeliminvisible
%
\isataginvisible
%
\endisataginvisible
{\isafoldinvisible}%
%
\isadeliminvisible
%
\endisadeliminvisible
%
\isadelimquotetypewriter
%
\endisadelimquotetypewriter
%
\isatagquotetypewriter
%
\begin{isamarkuptext}%
module\ Example\ where\ {\isaliteral{7B}{\isacharbraceleft}}\isanewline
\isanewline
data\ Nat\ {\isaliteral{3D}{\isacharequal}}\ Zero{\isaliteral{5F}{\isacharunderscore}}nat\ {\isaliteral{7C}{\isacharbar}}\ Suc\ Nat{\isaliteral{3B}{\isacharsemicolon}}\isanewline
\isanewline
nat{\isaliteral{5F}{\isacharunderscore}}aux\ {\isaliteral{3A}{\isacharcolon}}{\isaliteral{3A}{\isacharcolon}}\ Integer\ {\isaliteral{2D}{\isacharminus}}{\isaliteral{3E}{\isachargreater}}\ Nat\ {\isaliteral{2D}{\isacharminus}}{\isaliteral{3E}{\isachargreater}}\ Nat{\isaliteral{3B}{\isacharsemicolon}}\isanewline
nat{\isaliteral{5F}{\isacharunderscore}}aux\ i\ n\ {\isaliteral{3D}{\isacharequal}}\ {\isaliteral{28}{\isacharparenleft}}if\ i\ {\isaliteral{3C}{\isacharless}}{\isaliteral{3D}{\isacharequal}}\ {\isadigit{0}}\ then\ n\ else\ nat{\isaliteral{5F}{\isacharunderscore}}aux\ {\isaliteral{28}{\isacharparenleft}}i\ {\isaliteral{2D}{\isacharminus}}\ {\isadigit{1}}{\isaliteral{29}{\isacharparenright}}\ {\isaliteral{28}{\isacharparenleft}}Suc\ n{\isaliteral{29}{\isacharparenright}}{\isaliteral{29}{\isacharparenright}}{\isaliteral{3B}{\isacharsemicolon}}\isanewline
\isanewline
nat\ {\isaliteral{3A}{\isacharcolon}}{\isaliteral{3A}{\isacharcolon}}\ Integer\ {\isaliteral{2D}{\isacharminus}}{\isaliteral{3E}{\isachargreater}}\ Nat{\isaliteral{3B}{\isacharsemicolon}}\isanewline
nat\ i\ {\isaliteral{3D}{\isacharequal}}\ nat{\isaliteral{5F}{\isacharunderscore}}aux\ i\ Zero{\isaliteral{5F}{\isacharunderscore}}nat{\isaliteral{3B}{\isacharsemicolon}}\isanewline
\isanewline
class\ Semigroup\ a\ where\ {\isaliteral{7B}{\isacharbraceleft}}\isanewline
\ \ mult\ {\isaliteral{3A}{\isacharcolon}}{\isaliteral{3A}{\isacharcolon}}\ a\ {\isaliteral{2D}{\isacharminus}}{\isaliteral{3E}{\isachargreater}}\ a\ {\isaliteral{2D}{\isacharminus}}{\isaliteral{3E}{\isachargreater}}\ a{\isaliteral{3B}{\isacharsemicolon}}\isanewline
{\isaliteral{7D}{\isacharbraceright}}{\isaliteral{3B}{\isacharsemicolon}}\isanewline
\isanewline
class\ {\isaliteral{28}{\isacharparenleft}}Semigroup\ a{\isaliteral{29}{\isacharparenright}}\ {\isaliteral{3D}{\isacharequal}}{\isaliteral{3E}{\isachargreater}}\ Monoidl\ a\ where\ {\isaliteral{7B}{\isacharbraceleft}}\isanewline
\ \ neutral\ {\isaliteral{3A}{\isacharcolon}}{\isaliteral{3A}{\isacharcolon}}\ a{\isaliteral{3B}{\isacharsemicolon}}\isanewline
{\isaliteral{7D}{\isacharbraceright}}{\isaliteral{3B}{\isacharsemicolon}}\isanewline
\isanewline
class\ {\isaliteral{28}{\isacharparenleft}}Monoidl\ a{\isaliteral{29}{\isacharparenright}}\ {\isaliteral{3D}{\isacharequal}}{\isaliteral{3E}{\isachargreater}}\ Monoid\ a\ where\ {\isaliteral{7B}{\isacharbraceleft}}\isanewline
{\isaliteral{7D}{\isacharbraceright}}{\isaliteral{3B}{\isacharsemicolon}}\isanewline
\isanewline
class\ {\isaliteral{28}{\isacharparenleft}}Monoid\ a{\isaliteral{29}{\isacharparenright}}\ {\isaliteral{3D}{\isacharequal}}{\isaliteral{3E}{\isachargreater}}\ Group\ a\ where\ {\isaliteral{7B}{\isacharbraceleft}}\isanewline
\ \ inverse\ {\isaliteral{3A}{\isacharcolon}}{\isaliteral{3A}{\isacharcolon}}\ a\ {\isaliteral{2D}{\isacharminus}}{\isaliteral{3E}{\isachargreater}}\ a{\isaliteral{3B}{\isacharsemicolon}}\isanewline
{\isaliteral{7D}{\isacharbraceright}}{\isaliteral{3B}{\isacharsemicolon}}\isanewline
\isanewline
pow{\isaliteral{5F}{\isacharunderscore}}nat\ {\isaliteral{3A}{\isacharcolon}}{\isaliteral{3A}{\isacharcolon}}\ forall\ a{\isaliteral{2E}{\isachardot}}\ {\isaliteral{28}{\isacharparenleft}}Monoid\ a{\isaliteral{29}{\isacharparenright}}\ {\isaliteral{3D}{\isacharequal}}{\isaliteral{3E}{\isachargreater}}\ Nat\ {\isaliteral{2D}{\isacharminus}}{\isaliteral{3E}{\isachargreater}}\ a\ {\isaliteral{2D}{\isacharminus}}{\isaliteral{3E}{\isachargreater}}\ a{\isaliteral{3B}{\isacharsemicolon}}\isanewline
pow{\isaliteral{5F}{\isacharunderscore}}nat\ Zero{\isaliteral{5F}{\isacharunderscore}}nat\ x\ {\isaliteral{3D}{\isacharequal}}\ neutral{\isaliteral{3B}{\isacharsemicolon}}\isanewline
pow{\isaliteral{5F}{\isacharunderscore}}nat\ {\isaliteral{28}{\isacharparenleft}}Suc\ n{\isaliteral{29}{\isacharparenright}}\ x\ {\isaliteral{3D}{\isacharequal}}\ mult\ x\ {\isaliteral{28}{\isacharparenleft}}pow{\isaliteral{5F}{\isacharunderscore}}nat\ n\ x{\isaliteral{29}{\isacharparenright}}{\isaliteral{3B}{\isacharsemicolon}}\isanewline
\isanewline
pow{\isaliteral{5F}{\isacharunderscore}}int\ {\isaliteral{3A}{\isacharcolon}}{\isaliteral{3A}{\isacharcolon}}\ forall\ a{\isaliteral{2E}{\isachardot}}\ {\isaliteral{28}{\isacharparenleft}}Group\ a{\isaliteral{29}{\isacharparenright}}\ {\isaliteral{3D}{\isacharequal}}{\isaliteral{3E}{\isachargreater}}\ Integer\ {\isaliteral{2D}{\isacharminus}}{\isaliteral{3E}{\isachargreater}}\ a\ {\isaliteral{2D}{\isacharminus}}{\isaliteral{3E}{\isachargreater}}\ a{\isaliteral{3B}{\isacharsemicolon}}\isanewline
pow{\isaliteral{5F}{\isacharunderscore}}int\ k\ x\ {\isaliteral{3D}{\isacharequal}}\isanewline
\ \ {\isaliteral{28}{\isacharparenleft}}if\ {\isadigit{0}}\ {\isaliteral{3C}{\isacharless}}{\isaliteral{3D}{\isacharequal}}\ k\ then\ pow{\isaliteral{5F}{\isacharunderscore}}nat\ {\isaliteral{28}{\isacharparenleft}}nat\ k{\isaliteral{29}{\isacharparenright}}\ x\isanewline
\ \ \ \ else\ inverse\ {\isaliteral{28}{\isacharparenleft}}pow{\isaliteral{5F}{\isacharunderscore}}nat\ {\isaliteral{28}{\isacharparenleft}}nat\ {\isaliteral{28}{\isacharparenleft}}negate\ k{\isaliteral{29}{\isacharparenright}}{\isaliteral{29}{\isacharparenright}}\ x{\isaliteral{29}{\isacharparenright}}{\isaliteral{29}{\isacharparenright}}{\isaliteral{3B}{\isacharsemicolon}}\isanewline
\isanewline
mult{\isaliteral{5F}{\isacharunderscore}}int\ {\isaliteral{3A}{\isacharcolon}}{\isaliteral{3A}{\isacharcolon}}\ Integer\ {\isaliteral{2D}{\isacharminus}}{\isaliteral{3E}{\isachargreater}}\ Integer\ {\isaliteral{2D}{\isacharminus}}{\isaliteral{3E}{\isachargreater}}\ Integer{\isaliteral{3B}{\isacharsemicolon}}\isanewline
mult{\isaliteral{5F}{\isacharunderscore}}int\ i\ j\ {\isaliteral{3D}{\isacharequal}}\ i\ {\isaliteral{2B}{\isacharplus}}\ j{\isaliteral{3B}{\isacharsemicolon}}\isanewline
\isanewline
instance\ Semigroup\ Integer\ where\ {\isaliteral{7B}{\isacharbraceleft}}\isanewline
\ \ mult\ {\isaliteral{3D}{\isacharequal}}\ mult{\isaliteral{5F}{\isacharunderscore}}int{\isaliteral{3B}{\isacharsemicolon}}\isanewline
{\isaliteral{7D}{\isacharbraceright}}{\isaliteral{3B}{\isacharsemicolon}}\isanewline
\isanewline
neutral{\isaliteral{5F}{\isacharunderscore}}int\ {\isaliteral{3A}{\isacharcolon}}{\isaliteral{3A}{\isacharcolon}}\ Integer{\isaliteral{3B}{\isacharsemicolon}}\isanewline
neutral{\isaliteral{5F}{\isacharunderscore}}int\ {\isaliteral{3D}{\isacharequal}}\ {\isadigit{0}}{\isaliteral{3B}{\isacharsemicolon}}\isanewline
\isanewline
instance\ Monoidl\ Integer\ where\ {\isaliteral{7B}{\isacharbraceleft}}\isanewline
\ \ neutral\ {\isaliteral{3D}{\isacharequal}}\ neutral{\isaliteral{5F}{\isacharunderscore}}int{\isaliteral{3B}{\isacharsemicolon}}\isanewline
{\isaliteral{7D}{\isacharbraceright}}{\isaliteral{3B}{\isacharsemicolon}}\isanewline
\isanewline
instance\ Monoid\ Integer\ where\ {\isaliteral{7B}{\isacharbraceleft}}\isanewline
{\isaliteral{7D}{\isacharbraceright}}{\isaliteral{3B}{\isacharsemicolon}}\isanewline
\isanewline
inverse{\isaliteral{5F}{\isacharunderscore}}int\ {\isaliteral{3A}{\isacharcolon}}{\isaliteral{3A}{\isacharcolon}}\ Integer\ {\isaliteral{2D}{\isacharminus}}{\isaliteral{3E}{\isachargreater}}\ Integer{\isaliteral{3B}{\isacharsemicolon}}\isanewline
inverse{\isaliteral{5F}{\isacharunderscore}}int\ i\ {\isaliteral{3D}{\isacharequal}}\ negate\ i{\isaliteral{3B}{\isacharsemicolon}}\isanewline
\isanewline
instance\ Group\ Integer\ where\ {\isaliteral{7B}{\isacharbraceleft}}\isanewline
\ \ inverse\ {\isaliteral{3D}{\isacharequal}}\ inverse{\isaliteral{5F}{\isacharunderscore}}int{\isaliteral{3B}{\isacharsemicolon}}\isanewline
{\isaliteral{7D}{\isacharbraceright}}{\isaliteral{3B}{\isacharsemicolon}}\isanewline
\isanewline
example\ {\isaliteral{3A}{\isacharcolon}}{\isaliteral{3A}{\isacharcolon}}\ Integer{\isaliteral{3B}{\isacharsemicolon}}\isanewline
example\ {\isaliteral{3D}{\isacharequal}}\ pow{\isaliteral{5F}{\isacharunderscore}}int\ {\isadigit{1}}{\isadigit{0}}\ {\isaliteral{28}{\isacharparenleft}}{\isaliteral{2D}{\isacharminus}}{\isadigit{2}}{\isaliteral{29}{\isacharparenright}}{\isaliteral{3B}{\isacharsemicolon}}\isanewline
\isanewline
{\isaliteral{7D}{\isacharbraceright}}\isanewline%
\end{isamarkuptext}%
\isamarkuptrue%
%
\endisatagquotetypewriter
{\isafoldquotetypewriter}%
%
\isadelimquotetypewriter
%
\endisadelimquotetypewriter
%
\begin{isamarkuptext}%
\noindent The code in SML has explicit dictionary passing:%
\end{isamarkuptext}%
\isamarkuptrue%
%
\isadelimquotetypewriter
%
\endisadelimquotetypewriter
%
\isatagquotetypewriter
%
\begin{isamarkuptext}%
structure\ Example\ {\isaliteral{3A}{\isacharcolon}}\ sig\isanewline
\ \ datatype\ nat\ {\isaliteral{3D}{\isacharequal}}\ Zero{\isaliteral{5F}{\isacharunderscore}}nat\ {\isaliteral{7C}{\isacharbar}}\ Suc\ of\ nat\isanewline
\ \ val\ nat{\isaliteral{5F}{\isacharunderscore}}aux\ {\isaliteral{3A}{\isacharcolon}}\ IntInf{\isaliteral{2E}{\isachardot}}int\ {\isaliteral{2D}{\isacharminus}}{\isaliteral{3E}{\isachargreater}}\ nat\ {\isaliteral{2D}{\isacharminus}}{\isaliteral{3E}{\isachargreater}}\ nat\isanewline
\ \ val\ nat\ {\isaliteral{3A}{\isacharcolon}}\ IntInf{\isaliteral{2E}{\isachardot}}int\ {\isaliteral{2D}{\isacharminus}}{\isaliteral{3E}{\isachargreater}}\ nat\isanewline
\ \ type\ {\isaliteral{27}{\isacharprime}}a\ semigroup\isanewline
\ \ val\ mult\ {\isaliteral{3A}{\isacharcolon}}\ {\isaliteral{27}{\isacharprime}}a\ semigroup\ {\isaliteral{2D}{\isacharminus}}{\isaliteral{3E}{\isachargreater}}\ {\isaliteral{27}{\isacharprime}}a\ {\isaliteral{2D}{\isacharminus}}{\isaliteral{3E}{\isachargreater}}\ {\isaliteral{27}{\isacharprime}}a\ {\isaliteral{2D}{\isacharminus}}{\isaliteral{3E}{\isachargreater}}\ {\isaliteral{27}{\isacharprime}}a\isanewline
\ \ type\ {\isaliteral{27}{\isacharprime}}a\ monoidl\isanewline
\ \ val\ semigroup{\isaliteral{5F}{\isacharunderscore}}monoidl\ {\isaliteral{3A}{\isacharcolon}}\ {\isaliteral{27}{\isacharprime}}a\ monoidl\ {\isaliteral{2D}{\isacharminus}}{\isaliteral{3E}{\isachargreater}}\ {\isaliteral{27}{\isacharprime}}a\ semigroup\isanewline
\ \ val\ neutral\ {\isaliteral{3A}{\isacharcolon}}\ {\isaliteral{27}{\isacharprime}}a\ monoidl\ {\isaliteral{2D}{\isacharminus}}{\isaliteral{3E}{\isachargreater}}\ {\isaliteral{27}{\isacharprime}}a\isanewline
\ \ type\ {\isaliteral{27}{\isacharprime}}a\ monoid\isanewline
\ \ val\ monoidl{\isaliteral{5F}{\isacharunderscore}}monoid\ {\isaliteral{3A}{\isacharcolon}}\ {\isaliteral{27}{\isacharprime}}a\ monoid\ {\isaliteral{2D}{\isacharminus}}{\isaliteral{3E}{\isachargreater}}\ {\isaliteral{27}{\isacharprime}}a\ monoidl\isanewline
\ \ type\ {\isaliteral{27}{\isacharprime}}a\ group\isanewline
\ \ val\ monoid{\isaliteral{5F}{\isacharunderscore}}group\ {\isaliteral{3A}{\isacharcolon}}\ {\isaliteral{27}{\isacharprime}}a\ group\ {\isaliteral{2D}{\isacharminus}}{\isaliteral{3E}{\isachargreater}}\ {\isaliteral{27}{\isacharprime}}a\ monoid\isanewline
\ \ val\ inverse\ {\isaliteral{3A}{\isacharcolon}}\ {\isaliteral{27}{\isacharprime}}a\ group\ {\isaliteral{2D}{\isacharminus}}{\isaliteral{3E}{\isachargreater}}\ {\isaliteral{27}{\isacharprime}}a\ {\isaliteral{2D}{\isacharminus}}{\isaliteral{3E}{\isachargreater}}\ {\isaliteral{27}{\isacharprime}}a\isanewline
\ \ val\ pow{\isaliteral{5F}{\isacharunderscore}}nat\ {\isaliteral{3A}{\isacharcolon}}\ {\isaliteral{27}{\isacharprime}}a\ monoid\ {\isaliteral{2D}{\isacharminus}}{\isaliteral{3E}{\isachargreater}}\ nat\ {\isaliteral{2D}{\isacharminus}}{\isaliteral{3E}{\isachargreater}}\ {\isaliteral{27}{\isacharprime}}a\ {\isaliteral{2D}{\isacharminus}}{\isaliteral{3E}{\isachargreater}}\ {\isaliteral{27}{\isacharprime}}a\isanewline
\ \ val\ pow{\isaliteral{5F}{\isacharunderscore}}int\ {\isaliteral{3A}{\isacharcolon}}\ {\isaliteral{27}{\isacharprime}}a\ group\ {\isaliteral{2D}{\isacharminus}}{\isaliteral{3E}{\isachargreater}}\ IntInf{\isaliteral{2E}{\isachardot}}int\ {\isaliteral{2D}{\isacharminus}}{\isaliteral{3E}{\isachargreater}}\ {\isaliteral{27}{\isacharprime}}a\ {\isaliteral{2D}{\isacharminus}}{\isaliteral{3E}{\isachargreater}}\ {\isaliteral{27}{\isacharprime}}a\isanewline
\ \ val\ mult{\isaliteral{5F}{\isacharunderscore}}int\ {\isaliteral{3A}{\isacharcolon}}\ IntInf{\isaliteral{2E}{\isachardot}}int\ {\isaliteral{2D}{\isacharminus}}{\isaliteral{3E}{\isachargreater}}\ IntInf{\isaliteral{2E}{\isachardot}}int\ {\isaliteral{2D}{\isacharminus}}{\isaliteral{3E}{\isachargreater}}\ IntInf{\isaliteral{2E}{\isachardot}}int\isanewline
\ \ val\ semigroup{\isaliteral{5F}{\isacharunderscore}}int\ {\isaliteral{3A}{\isacharcolon}}\ IntInf{\isaliteral{2E}{\isachardot}}int\ semigroup\isanewline
\ \ val\ neutral{\isaliteral{5F}{\isacharunderscore}}int\ {\isaliteral{3A}{\isacharcolon}}\ IntInf{\isaliteral{2E}{\isachardot}}int\isanewline
\ \ val\ monoidl{\isaliteral{5F}{\isacharunderscore}}int\ {\isaliteral{3A}{\isacharcolon}}\ IntInf{\isaliteral{2E}{\isachardot}}int\ monoidl\isanewline
\ \ val\ monoid{\isaliteral{5F}{\isacharunderscore}}int\ {\isaliteral{3A}{\isacharcolon}}\ IntInf{\isaliteral{2E}{\isachardot}}int\ monoid\isanewline
\ \ val\ inverse{\isaliteral{5F}{\isacharunderscore}}int\ {\isaliteral{3A}{\isacharcolon}}\ IntInf{\isaliteral{2E}{\isachardot}}int\ {\isaliteral{2D}{\isacharminus}}{\isaliteral{3E}{\isachargreater}}\ IntInf{\isaliteral{2E}{\isachardot}}int\isanewline
\ \ val\ group{\isaliteral{5F}{\isacharunderscore}}int\ {\isaliteral{3A}{\isacharcolon}}\ IntInf{\isaliteral{2E}{\isachardot}}int\ group\isanewline
\ \ val\ example\ {\isaliteral{3A}{\isacharcolon}}\ IntInf{\isaliteral{2E}{\isachardot}}int\isanewline
end\ {\isaliteral{3D}{\isacharequal}}\ struct\isanewline
\isanewline
datatype\ nat\ {\isaliteral{3D}{\isacharequal}}\ Zero{\isaliteral{5F}{\isacharunderscore}}nat\ {\isaliteral{7C}{\isacharbar}}\ Suc\ of\ nat{\isaliteral{3B}{\isacharsemicolon}}\isanewline
\isanewline
fun\ nat{\isaliteral{5F}{\isacharunderscore}}aux\ i\ n\ {\isaliteral{3D}{\isacharequal}}\isanewline
\ \ {\isaliteral{28}{\isacharparenleft}}if\ IntInf{\isaliteral{2E}{\isachardot}}{\isaliteral{3C}{\isacharless}}{\isaliteral{3D}{\isacharequal}}\ {\isaliteral{28}{\isacharparenleft}}i{\isaliteral{2C}{\isacharcomma}}\ {\isaliteral{28}{\isacharparenleft}}{\isadigit{0}}\ {\isaliteral{3A}{\isacharcolon}}\ IntInf{\isaliteral{2E}{\isachardot}}int{\isaliteral{29}{\isacharparenright}}{\isaliteral{29}{\isacharparenright}}\ then\ n\isanewline
\ \ \ \ else\ nat{\isaliteral{5F}{\isacharunderscore}}aux\ {\isaliteral{28}{\isacharparenleft}}IntInf{\isaliteral{2E}{\isachardot}}{\isaliteral{2D}{\isacharminus}}\ {\isaliteral{28}{\isacharparenleft}}i{\isaliteral{2C}{\isacharcomma}}\ {\isaliteral{28}{\isacharparenleft}}{\isadigit{1}}\ {\isaliteral{3A}{\isacharcolon}}\ IntInf{\isaliteral{2E}{\isachardot}}int{\isaliteral{29}{\isacharparenright}}{\isaliteral{29}{\isacharparenright}}{\isaliteral{29}{\isacharparenright}}\ {\isaliteral{28}{\isacharparenleft}}Suc\ n{\isaliteral{29}{\isacharparenright}}{\isaliteral{29}{\isacharparenright}}{\isaliteral{3B}{\isacharsemicolon}}\isanewline
\isanewline
fun\ nat\ i\ {\isaliteral{3D}{\isacharequal}}\ nat{\isaliteral{5F}{\isacharunderscore}}aux\ i\ Zero{\isaliteral{5F}{\isacharunderscore}}nat{\isaliteral{3B}{\isacharsemicolon}}\isanewline
\isanewline
type\ {\isaliteral{27}{\isacharprime}}a\ semigroup\ {\isaliteral{3D}{\isacharequal}}\ {\isaliteral{7B}{\isacharbraceleft}}mult\ {\isaliteral{3A}{\isacharcolon}}\ {\isaliteral{27}{\isacharprime}}a\ {\isaliteral{2D}{\isacharminus}}{\isaliteral{3E}{\isachargreater}}\ {\isaliteral{27}{\isacharprime}}a\ {\isaliteral{2D}{\isacharminus}}{\isaliteral{3E}{\isachargreater}}\ {\isaliteral{27}{\isacharprime}}a{\isaliteral{7D}{\isacharbraceright}}{\isaliteral{3B}{\isacharsemicolon}}\isanewline
val\ mult\ {\isaliteral{3D}{\isacharequal}}\ {\isaliteral{23}{\isacharhash}}mult\ {\isaliteral{3A}{\isacharcolon}}\ {\isaliteral{27}{\isacharprime}}a\ semigroup\ {\isaliteral{2D}{\isacharminus}}{\isaliteral{3E}{\isachargreater}}\ {\isaliteral{27}{\isacharprime}}a\ {\isaliteral{2D}{\isacharminus}}{\isaliteral{3E}{\isachargreater}}\ {\isaliteral{27}{\isacharprime}}a\ {\isaliteral{2D}{\isacharminus}}{\isaliteral{3E}{\isachargreater}}\ {\isaliteral{27}{\isacharprime}}a{\isaliteral{3B}{\isacharsemicolon}}\isanewline
\isanewline
type\ {\isaliteral{27}{\isacharprime}}a\ monoidl\ {\isaliteral{3D}{\isacharequal}}\ {\isaliteral{7B}{\isacharbraceleft}}semigroup{\isaliteral{5F}{\isacharunderscore}}monoidl\ {\isaliteral{3A}{\isacharcolon}}\ {\isaliteral{27}{\isacharprime}}a\ semigroup{\isaliteral{2C}{\isacharcomma}}\ neutral\ {\isaliteral{3A}{\isacharcolon}}\ {\isaliteral{27}{\isacharprime}}a{\isaliteral{7D}{\isacharbraceright}}{\isaliteral{3B}{\isacharsemicolon}}\isanewline
val\ semigroup{\isaliteral{5F}{\isacharunderscore}}monoidl\ {\isaliteral{3D}{\isacharequal}}\ {\isaliteral{23}{\isacharhash}}semigroup{\isaliteral{5F}{\isacharunderscore}}monoidl\ {\isaliteral{3A}{\isacharcolon}}\ {\isaliteral{27}{\isacharprime}}a\ monoidl\ {\isaliteral{2D}{\isacharminus}}{\isaliteral{3E}{\isachargreater}}\ {\isaliteral{27}{\isacharprime}}a\ semigroup{\isaliteral{3B}{\isacharsemicolon}}\isanewline
val\ neutral\ {\isaliteral{3D}{\isacharequal}}\ {\isaliteral{23}{\isacharhash}}neutral\ {\isaliteral{3A}{\isacharcolon}}\ {\isaliteral{27}{\isacharprime}}a\ monoidl\ {\isaliteral{2D}{\isacharminus}}{\isaliteral{3E}{\isachargreater}}\ {\isaliteral{27}{\isacharprime}}a{\isaliteral{3B}{\isacharsemicolon}}\isanewline
\isanewline
type\ {\isaliteral{27}{\isacharprime}}a\ monoid\ {\isaliteral{3D}{\isacharequal}}\ {\isaliteral{7B}{\isacharbraceleft}}monoidl{\isaliteral{5F}{\isacharunderscore}}monoid\ {\isaliteral{3A}{\isacharcolon}}\ {\isaliteral{27}{\isacharprime}}a\ monoidl{\isaliteral{7D}{\isacharbraceright}}{\isaliteral{3B}{\isacharsemicolon}}\isanewline
val\ monoidl{\isaliteral{5F}{\isacharunderscore}}monoid\ {\isaliteral{3D}{\isacharequal}}\ {\isaliteral{23}{\isacharhash}}monoidl{\isaliteral{5F}{\isacharunderscore}}monoid\ {\isaliteral{3A}{\isacharcolon}}\ {\isaliteral{27}{\isacharprime}}a\ monoid\ {\isaliteral{2D}{\isacharminus}}{\isaliteral{3E}{\isachargreater}}\ {\isaliteral{27}{\isacharprime}}a\ monoidl{\isaliteral{3B}{\isacharsemicolon}}\isanewline
\isanewline
type\ {\isaliteral{27}{\isacharprime}}a\ group\ {\isaliteral{3D}{\isacharequal}}\ {\isaliteral{7B}{\isacharbraceleft}}monoid{\isaliteral{5F}{\isacharunderscore}}group\ {\isaliteral{3A}{\isacharcolon}}\ {\isaliteral{27}{\isacharprime}}a\ monoid{\isaliteral{2C}{\isacharcomma}}\ inverse\ {\isaliteral{3A}{\isacharcolon}}\ {\isaliteral{27}{\isacharprime}}a\ {\isaliteral{2D}{\isacharminus}}{\isaliteral{3E}{\isachargreater}}\ {\isaliteral{27}{\isacharprime}}a{\isaliteral{7D}{\isacharbraceright}}{\isaliteral{3B}{\isacharsemicolon}}\isanewline
val\ monoid{\isaliteral{5F}{\isacharunderscore}}group\ {\isaliteral{3D}{\isacharequal}}\ {\isaliteral{23}{\isacharhash}}monoid{\isaliteral{5F}{\isacharunderscore}}group\ {\isaliteral{3A}{\isacharcolon}}\ {\isaliteral{27}{\isacharprime}}a\ group\ {\isaliteral{2D}{\isacharminus}}{\isaliteral{3E}{\isachargreater}}\ {\isaliteral{27}{\isacharprime}}a\ monoid{\isaliteral{3B}{\isacharsemicolon}}\isanewline
val\ inverse\ {\isaliteral{3D}{\isacharequal}}\ {\isaliteral{23}{\isacharhash}}inverse\ {\isaliteral{3A}{\isacharcolon}}\ {\isaliteral{27}{\isacharprime}}a\ group\ {\isaliteral{2D}{\isacharminus}}{\isaliteral{3E}{\isachargreater}}\ {\isaliteral{27}{\isacharprime}}a\ {\isaliteral{2D}{\isacharminus}}{\isaliteral{3E}{\isachargreater}}\ {\isaliteral{27}{\isacharprime}}a{\isaliteral{3B}{\isacharsemicolon}}\isanewline
\isanewline
fun\ pow{\isaliteral{5F}{\isacharunderscore}}nat\ A{\isaliteral{5F}{\isacharunderscore}}\ Zero{\isaliteral{5F}{\isacharunderscore}}nat\ x\ {\isaliteral{3D}{\isacharequal}}\ neutral\ {\isaliteral{28}{\isacharparenleft}}monoidl{\isaliteral{5F}{\isacharunderscore}}monoid\ A{\isaliteral{5F}{\isacharunderscore}}{\isaliteral{29}{\isacharparenright}}\isanewline
\ \ {\isaliteral{7C}{\isacharbar}}\ pow{\isaliteral{5F}{\isacharunderscore}}nat\ A{\isaliteral{5F}{\isacharunderscore}}\ {\isaliteral{28}{\isacharparenleft}}Suc\ n{\isaliteral{29}{\isacharparenright}}\ x\ {\isaliteral{3D}{\isacharequal}}\isanewline
\ \ \ \ mult\ {\isaliteral{28}{\isacharparenleft}}{\isaliteral{28}{\isacharparenleft}}semigroup{\isaliteral{5F}{\isacharunderscore}}monoidl\ o\ monoidl{\isaliteral{5F}{\isacharunderscore}}monoid{\isaliteral{29}{\isacharparenright}}\ A{\isaliteral{5F}{\isacharunderscore}}{\isaliteral{29}{\isacharparenright}}\ x\ {\isaliteral{28}{\isacharparenleft}}pow{\isaliteral{5F}{\isacharunderscore}}nat\ A{\isaliteral{5F}{\isacharunderscore}}\ n\ x{\isaliteral{29}{\isacharparenright}}{\isaliteral{3B}{\isacharsemicolon}}\isanewline
\isanewline
fun\ pow{\isaliteral{5F}{\isacharunderscore}}int\ A{\isaliteral{5F}{\isacharunderscore}}\ k\ x\ {\isaliteral{3D}{\isacharequal}}\isanewline
\ \ {\isaliteral{28}{\isacharparenleft}}if\ IntInf{\isaliteral{2E}{\isachardot}}{\isaliteral{3C}{\isacharless}}{\isaliteral{3D}{\isacharequal}}\ {\isaliteral{28}{\isacharparenleft}}{\isaliteral{28}{\isacharparenleft}}{\isadigit{0}}\ {\isaliteral{3A}{\isacharcolon}}\ IntInf{\isaliteral{2E}{\isachardot}}int{\isaliteral{29}{\isacharparenright}}{\isaliteral{2C}{\isacharcomma}}\ k{\isaliteral{29}{\isacharparenright}}\isanewline
\ \ \ \ then\ pow{\isaliteral{5F}{\isacharunderscore}}nat\ {\isaliteral{28}{\isacharparenleft}}monoid{\isaliteral{5F}{\isacharunderscore}}group\ A{\isaliteral{5F}{\isacharunderscore}}{\isaliteral{29}{\isacharparenright}}\ {\isaliteral{28}{\isacharparenleft}}nat\ k{\isaliteral{29}{\isacharparenright}}\ x\isanewline
\ \ \ \ else\ inverse\ A{\isaliteral{5F}{\isacharunderscore}}\ {\isaliteral{28}{\isacharparenleft}}pow{\isaliteral{5F}{\isacharunderscore}}nat\ {\isaliteral{28}{\isacharparenleft}}monoid{\isaliteral{5F}{\isacharunderscore}}group\ A{\isaliteral{5F}{\isacharunderscore}}{\isaliteral{29}{\isacharparenright}}\ {\isaliteral{28}{\isacharparenleft}}nat\ {\isaliteral{28}{\isacharparenleft}}IntInf{\isaliteral{2E}{\isachardot}}{\isaliteral{7E}{\isachartilde}}\ k{\isaliteral{29}{\isacharparenright}}{\isaliteral{29}{\isacharparenright}}\ x{\isaliteral{29}{\isacharparenright}}{\isaliteral{29}{\isacharparenright}}{\isaliteral{3B}{\isacharsemicolon}}\isanewline
\isanewline
fun\ mult{\isaliteral{5F}{\isacharunderscore}}int\ i\ j\ {\isaliteral{3D}{\isacharequal}}\ IntInf{\isaliteral{2E}{\isachardot}}{\isaliteral{2B}{\isacharplus}}\ {\isaliteral{28}{\isacharparenleft}}i{\isaliteral{2C}{\isacharcomma}}\ j{\isaliteral{29}{\isacharparenright}}{\isaliteral{3B}{\isacharsemicolon}}\isanewline
\isanewline
val\ semigroup{\isaliteral{5F}{\isacharunderscore}}int\ {\isaliteral{3D}{\isacharequal}}\ {\isaliteral{7B}{\isacharbraceleft}}mult\ {\isaliteral{3D}{\isacharequal}}\ mult{\isaliteral{5F}{\isacharunderscore}}int{\isaliteral{7D}{\isacharbraceright}}\ {\isaliteral{3A}{\isacharcolon}}\ IntInf{\isaliteral{2E}{\isachardot}}int\ semigroup{\isaliteral{3B}{\isacharsemicolon}}\isanewline
\isanewline
val\ neutral{\isaliteral{5F}{\isacharunderscore}}int\ {\isaliteral{3A}{\isacharcolon}}\ IntInf{\isaliteral{2E}{\isachardot}}int\ {\isaliteral{3D}{\isacharequal}}\ {\isaliteral{28}{\isacharparenleft}}{\isadigit{0}}\ {\isaliteral{3A}{\isacharcolon}}\ IntInf{\isaliteral{2E}{\isachardot}}int{\isaliteral{29}{\isacharparenright}}{\isaliteral{3B}{\isacharsemicolon}}\isanewline
\isanewline
val\ monoidl{\isaliteral{5F}{\isacharunderscore}}int\ {\isaliteral{3D}{\isacharequal}}\isanewline
\ \ {\isaliteral{7B}{\isacharbraceleft}}semigroup{\isaliteral{5F}{\isacharunderscore}}monoidl\ {\isaliteral{3D}{\isacharequal}}\ semigroup{\isaliteral{5F}{\isacharunderscore}}int{\isaliteral{2C}{\isacharcomma}}\ neutral\ {\isaliteral{3D}{\isacharequal}}\ neutral{\isaliteral{5F}{\isacharunderscore}}int{\isaliteral{7D}{\isacharbraceright}}\ {\isaliteral{3A}{\isacharcolon}}\isanewline
\ \ IntInf{\isaliteral{2E}{\isachardot}}int\ monoidl{\isaliteral{3B}{\isacharsemicolon}}\isanewline
\isanewline
val\ monoid{\isaliteral{5F}{\isacharunderscore}}int\ {\isaliteral{3D}{\isacharequal}}\ {\isaliteral{7B}{\isacharbraceleft}}monoidl{\isaliteral{5F}{\isacharunderscore}}monoid\ {\isaliteral{3D}{\isacharequal}}\ monoidl{\isaliteral{5F}{\isacharunderscore}}int{\isaliteral{7D}{\isacharbraceright}}\ {\isaliteral{3A}{\isacharcolon}}\ IntInf{\isaliteral{2E}{\isachardot}}int\ monoid{\isaliteral{3B}{\isacharsemicolon}}\isanewline
\isanewline
fun\ inverse{\isaliteral{5F}{\isacharunderscore}}int\ i\ {\isaliteral{3D}{\isacharequal}}\ IntInf{\isaliteral{2E}{\isachardot}}{\isaliteral{7E}{\isachartilde}}\ i{\isaliteral{3B}{\isacharsemicolon}}\isanewline
\isanewline
val\ group{\isaliteral{5F}{\isacharunderscore}}int\ {\isaliteral{3D}{\isacharequal}}\ {\isaliteral{7B}{\isacharbraceleft}}monoid{\isaliteral{5F}{\isacharunderscore}}group\ {\isaliteral{3D}{\isacharequal}}\ monoid{\isaliteral{5F}{\isacharunderscore}}int{\isaliteral{2C}{\isacharcomma}}\ inverse\ {\isaliteral{3D}{\isacharequal}}\ inverse{\isaliteral{5F}{\isacharunderscore}}int{\isaliteral{7D}{\isacharbraceright}}\ {\isaliteral{3A}{\isacharcolon}}\isanewline
\ \ IntInf{\isaliteral{2E}{\isachardot}}int\ group{\isaliteral{3B}{\isacharsemicolon}}\isanewline
\isanewline
val\ example\ {\isaliteral{3A}{\isacharcolon}}\ IntInf{\isaliteral{2E}{\isachardot}}int\ {\isaliteral{3D}{\isacharequal}}\isanewline
\ \ pow{\isaliteral{5F}{\isacharunderscore}}int\ group{\isaliteral{5F}{\isacharunderscore}}int\ {\isaliteral{28}{\isacharparenleft}}{\isadigit{1}}{\isadigit{0}}\ {\isaliteral{3A}{\isacharcolon}}\ IntInf{\isaliteral{2E}{\isachardot}}int{\isaliteral{29}{\isacharparenright}}\ {\isaliteral{28}{\isacharparenleft}}{\isaliteral{7E}{\isachartilde}}{\isadigit{2}}\ {\isaliteral{3A}{\isacharcolon}}\ IntInf{\isaliteral{2E}{\isachardot}}int{\isaliteral{29}{\isacharparenright}}{\isaliteral{3B}{\isacharsemicolon}}\isanewline
\isanewline
end{\isaliteral{3B}{\isacharsemicolon}}\ {\isaliteral{28}{\isacharparenleft}}{\isaliteral{2A}{\isacharasterisk}}struct\ Example{\isaliteral{2A}{\isacharasterisk}}{\isaliteral{29}{\isacharparenright}}\isanewline%
\end{isamarkuptext}%
\isamarkuptrue%
%
\endisatagquotetypewriter
{\isafoldquotetypewriter}%
%
\isadelimquotetypewriter
%
\endisadelimquotetypewriter
%
\begin{isamarkuptext}%
\noindent In Scala, implicts are used as dictionaries:%
\end{isamarkuptext}%
\isamarkuptrue%
%
\isadeliminvisible
%
\endisadeliminvisible
%
\isataginvisible
%
\endisataginvisible
{\isafoldinvisible}%
%
\isadeliminvisible
%
\endisadeliminvisible
%
\isadelimquotetypewriter
%
\endisadelimquotetypewriter
%
\isatagquotetypewriter
%
\begin{isamarkuptext}%
object\ Example\ {\isaliteral{7B}{\isacharbraceleft}}\isanewline
\isanewline
abstract\ sealed\ class\ nat\isanewline
final\ case\ object\ Zero{\isaliteral{5F}{\isacharunderscore}}nat\ extends\ nat\isanewline
final\ case\ class\ Suc{\isaliteral{28}{\isacharparenleft}}a{\isaliteral{3A}{\isacharcolon}}\ nat{\isaliteral{29}{\isacharparenright}}\ extends\ nat\isanewline
\isanewline
def\ nat{\isaliteral{5F}{\isacharunderscore}}aux{\isaliteral{28}{\isacharparenleft}}i{\isaliteral{3A}{\isacharcolon}}\ BigInt{\isaliteral{2C}{\isacharcomma}}\ n{\isaliteral{3A}{\isacharcolon}}\ nat{\isaliteral{29}{\isacharparenright}}{\isaliteral{3A}{\isacharcolon}}\ nat\ {\isaliteral{3D}{\isacharequal}}\isanewline
\ \ {\isaliteral{28}{\isacharparenleft}}if\ {\isaliteral{28}{\isacharparenleft}}i\ {\isaliteral{3C}{\isacharless}}{\isaliteral{3D}{\isacharequal}}\ BigInt{\isaliteral{28}{\isacharparenleft}}{\isadigit{0}}{\isaliteral{29}{\isacharparenright}}{\isaliteral{29}{\isacharparenright}}\ n\ else\ nat{\isaliteral{5F}{\isacharunderscore}}aux{\isaliteral{28}{\isacharparenleft}}i\ {\isaliteral{2D}{\isacharminus}}\ BigInt{\isaliteral{28}{\isacharparenleft}}{\isadigit{1}}{\isaliteral{29}{\isacharparenright}}{\isaliteral{2C}{\isacharcomma}}\ Suc{\isaliteral{28}{\isacharparenleft}}n{\isaliteral{29}{\isacharparenright}}{\isaliteral{29}{\isacharparenright}}{\isaliteral{29}{\isacharparenright}}\isanewline
\isanewline
def\ nat{\isaliteral{28}{\isacharparenleft}}i{\isaliteral{3A}{\isacharcolon}}\ BigInt{\isaliteral{29}{\isacharparenright}}{\isaliteral{3A}{\isacharcolon}}\ nat\ {\isaliteral{3D}{\isacharequal}}\ nat{\isaliteral{5F}{\isacharunderscore}}aux{\isaliteral{28}{\isacharparenleft}}i{\isaliteral{2C}{\isacharcomma}}\ Zero{\isaliteral{5F}{\isacharunderscore}}nat{\isaliteral{29}{\isacharparenright}}\isanewline
\isanewline
trait\ semigroup{\isaliteral{5B}{\isacharbrackleft}}A{\isaliteral{5D}{\isacharbrackright}}\ {\isaliteral{7B}{\isacharbraceleft}}\isanewline
\ \ val\ {\isaliteral{60}{\isacharbackquote}}Example{\isaliteral{2E}{\isachardot}}mult{\isaliteral{60}{\isacharbackquote}}{\isaliteral{3A}{\isacharcolon}}\ {\isaliteral{28}{\isacharparenleft}}A{\isaliteral{2C}{\isacharcomma}}\ A{\isaliteral{29}{\isacharparenright}}\ {\isaliteral{3D}{\isacharequal}}{\isaliteral{3E}{\isachargreater}}\ A\isanewline
{\isaliteral{7D}{\isacharbraceright}}\isanewline
def\ mult{\isaliteral{5B}{\isacharbrackleft}}A{\isaliteral{5D}{\isacharbrackright}}{\isaliteral{28}{\isacharparenleft}}a{\isaliteral{3A}{\isacharcolon}}\ A{\isaliteral{2C}{\isacharcomma}}\ b{\isaliteral{3A}{\isacharcolon}}\ A{\isaliteral{29}{\isacharparenright}}{\isaliteral{28}{\isacharparenleft}}implicit\ A{\isaliteral{3A}{\isacharcolon}}\ semigroup{\isaliteral{5B}{\isacharbrackleft}}A{\isaliteral{5D}{\isacharbrackright}}{\isaliteral{29}{\isacharparenright}}{\isaliteral{3A}{\isacharcolon}}\ A\ {\isaliteral{3D}{\isacharequal}}\isanewline
\ \ A{\isaliteral{2E}{\isachardot}}{\isaliteral{60}{\isacharbackquote}}Example{\isaliteral{2E}{\isachardot}}mult{\isaliteral{60}{\isacharbackquote}}{\isaliteral{28}{\isacharparenleft}}a{\isaliteral{2C}{\isacharcomma}}\ b{\isaliteral{29}{\isacharparenright}}\isanewline
\isanewline
trait\ monoidl{\isaliteral{5B}{\isacharbrackleft}}A{\isaliteral{5D}{\isacharbrackright}}\ extends\ semigroup{\isaliteral{5B}{\isacharbrackleft}}A{\isaliteral{5D}{\isacharbrackright}}\ {\isaliteral{7B}{\isacharbraceleft}}\isanewline
\ \ val\ {\isaliteral{60}{\isacharbackquote}}Example{\isaliteral{2E}{\isachardot}}neutral{\isaliteral{60}{\isacharbackquote}}{\isaliteral{3A}{\isacharcolon}}\ A\isanewline
{\isaliteral{7D}{\isacharbraceright}}\isanewline
def\ neutral{\isaliteral{5B}{\isacharbrackleft}}A{\isaliteral{5D}{\isacharbrackright}}{\isaliteral{28}{\isacharparenleft}}implicit\ A{\isaliteral{3A}{\isacharcolon}}\ monoidl{\isaliteral{5B}{\isacharbrackleft}}A{\isaliteral{5D}{\isacharbrackright}}{\isaliteral{29}{\isacharparenright}}{\isaliteral{3A}{\isacharcolon}}\ A\ {\isaliteral{3D}{\isacharequal}}\ A{\isaliteral{2E}{\isachardot}}{\isaliteral{60}{\isacharbackquote}}Example{\isaliteral{2E}{\isachardot}}neutral{\isaliteral{60}{\isacharbackquote}}\isanewline
\isanewline
trait\ monoid{\isaliteral{5B}{\isacharbrackleft}}A{\isaliteral{5D}{\isacharbrackright}}\ extends\ monoidl{\isaliteral{5B}{\isacharbrackleft}}A{\isaliteral{5D}{\isacharbrackright}}\ {\isaliteral{7B}{\isacharbraceleft}}\isanewline
{\isaliteral{7D}{\isacharbraceright}}\isanewline
\isanewline
trait\ group{\isaliteral{5B}{\isacharbrackleft}}A{\isaliteral{5D}{\isacharbrackright}}\ extends\ monoid{\isaliteral{5B}{\isacharbrackleft}}A{\isaliteral{5D}{\isacharbrackright}}\ {\isaliteral{7B}{\isacharbraceleft}}\isanewline
\ \ val\ {\isaliteral{60}{\isacharbackquote}}Example{\isaliteral{2E}{\isachardot}}inverse{\isaliteral{60}{\isacharbackquote}}{\isaliteral{3A}{\isacharcolon}}\ A\ {\isaliteral{3D}{\isacharequal}}{\isaliteral{3E}{\isachargreater}}\ A\isanewline
{\isaliteral{7D}{\isacharbraceright}}\isanewline
def\ inverse{\isaliteral{5B}{\isacharbrackleft}}A{\isaliteral{5D}{\isacharbrackright}}{\isaliteral{28}{\isacharparenleft}}a{\isaliteral{3A}{\isacharcolon}}\ A{\isaliteral{29}{\isacharparenright}}{\isaliteral{28}{\isacharparenleft}}implicit\ A{\isaliteral{3A}{\isacharcolon}}\ group{\isaliteral{5B}{\isacharbrackleft}}A{\isaliteral{5D}{\isacharbrackright}}{\isaliteral{29}{\isacharparenright}}{\isaliteral{3A}{\isacharcolon}}\ A\ {\isaliteral{3D}{\isacharequal}}\ A{\isaliteral{2E}{\isachardot}}{\isaliteral{60}{\isacharbackquote}}Example{\isaliteral{2E}{\isachardot}}inverse{\isaliteral{60}{\isacharbackquote}}{\isaliteral{28}{\isacharparenleft}}a{\isaliteral{29}{\isacharparenright}}\isanewline
\isanewline
def\ pow{\isaliteral{5F}{\isacharunderscore}}nat{\isaliteral{5B}{\isacharbrackleft}}A{\isaliteral{3A}{\isacharcolon}}\ monoid{\isaliteral{5D}{\isacharbrackright}}{\isaliteral{28}{\isacharparenleft}}xa{\isadigit{0}}{\isaliteral{3A}{\isacharcolon}}\ nat{\isaliteral{2C}{\isacharcomma}}\ x{\isaliteral{3A}{\isacharcolon}}\ A{\isaliteral{29}{\isacharparenright}}{\isaliteral{3A}{\isacharcolon}}\ A\ {\isaliteral{3D}{\isacharequal}}\ {\isaliteral{28}{\isacharparenleft}}xa{\isadigit{0}}{\isaliteral{2C}{\isacharcomma}}\ x{\isaliteral{29}{\isacharparenright}}\ match\ {\isaliteral{7B}{\isacharbraceleft}}\isanewline
\ \ case\ {\isaliteral{28}{\isacharparenleft}}Zero{\isaliteral{5F}{\isacharunderscore}}nat{\isaliteral{2C}{\isacharcomma}}\ x{\isaliteral{29}{\isacharparenright}}\ {\isaliteral{3D}{\isacharequal}}{\isaliteral{3E}{\isachargreater}}\ neutral{\isaliteral{5B}{\isacharbrackleft}}A{\isaliteral{5D}{\isacharbrackright}}\isanewline
\ \ case\ {\isaliteral{28}{\isacharparenleft}}Suc{\isaliteral{28}{\isacharparenleft}}n{\isaliteral{29}{\isacharparenright}}{\isaliteral{2C}{\isacharcomma}}\ x{\isaliteral{29}{\isacharparenright}}\ {\isaliteral{3D}{\isacharequal}}{\isaliteral{3E}{\isachargreater}}\ mult{\isaliteral{5B}{\isacharbrackleft}}A{\isaliteral{5D}{\isacharbrackright}}{\isaliteral{28}{\isacharparenleft}}x{\isaliteral{2C}{\isacharcomma}}\ pow{\isaliteral{5F}{\isacharunderscore}}nat{\isaliteral{5B}{\isacharbrackleft}}A{\isaliteral{5D}{\isacharbrackright}}{\isaliteral{28}{\isacharparenleft}}n{\isaliteral{2C}{\isacharcomma}}\ x{\isaliteral{29}{\isacharparenright}}{\isaliteral{29}{\isacharparenright}}\isanewline
{\isaliteral{7D}{\isacharbraceright}}\isanewline
\isanewline
def\ pow{\isaliteral{5F}{\isacharunderscore}}int{\isaliteral{5B}{\isacharbrackleft}}A{\isaliteral{3A}{\isacharcolon}}\ group{\isaliteral{5D}{\isacharbrackright}}{\isaliteral{28}{\isacharparenleft}}k{\isaliteral{3A}{\isacharcolon}}\ BigInt{\isaliteral{2C}{\isacharcomma}}\ x{\isaliteral{3A}{\isacharcolon}}\ A{\isaliteral{29}{\isacharparenright}}{\isaliteral{3A}{\isacharcolon}}\ A\ {\isaliteral{3D}{\isacharequal}}\isanewline
\ \ {\isaliteral{28}{\isacharparenleft}}if\ {\isaliteral{28}{\isacharparenleft}}BigInt{\isaliteral{28}{\isacharparenleft}}{\isadigit{0}}{\isaliteral{29}{\isacharparenright}}\ {\isaliteral{3C}{\isacharless}}{\isaliteral{3D}{\isacharequal}}\ k{\isaliteral{29}{\isacharparenright}}\ pow{\isaliteral{5F}{\isacharunderscore}}nat{\isaliteral{5B}{\isacharbrackleft}}A{\isaliteral{5D}{\isacharbrackright}}{\isaliteral{28}{\isacharparenleft}}nat{\isaliteral{28}{\isacharparenleft}}k{\isaliteral{29}{\isacharparenright}}{\isaliteral{2C}{\isacharcomma}}\ x{\isaliteral{29}{\isacharparenright}}\isanewline
\ \ \ \ else\ inverse{\isaliteral{5B}{\isacharbrackleft}}A{\isaliteral{5D}{\isacharbrackright}}{\isaliteral{28}{\isacharparenleft}}pow{\isaliteral{5F}{\isacharunderscore}}nat{\isaliteral{5B}{\isacharbrackleft}}A{\isaliteral{5D}{\isacharbrackright}}{\isaliteral{28}{\isacharparenleft}}nat{\isaliteral{28}{\isacharparenleft}}{\isaliteral{28}{\isacharparenleft}}{\isaliteral{2D}{\isacharminus}}\ k{\isaliteral{29}{\isacharparenright}}{\isaliteral{29}{\isacharparenright}}{\isaliteral{2C}{\isacharcomma}}\ x{\isaliteral{29}{\isacharparenright}}{\isaliteral{29}{\isacharparenright}}{\isaliteral{29}{\isacharparenright}}\isanewline
\isanewline
def\ mult{\isaliteral{5F}{\isacharunderscore}}int{\isaliteral{28}{\isacharparenleft}}i{\isaliteral{3A}{\isacharcolon}}\ BigInt{\isaliteral{2C}{\isacharcomma}}\ j{\isaliteral{3A}{\isacharcolon}}\ BigInt{\isaliteral{29}{\isacharparenright}}{\isaliteral{3A}{\isacharcolon}}\ BigInt\ {\isaliteral{3D}{\isacharequal}}\ i\ {\isaliteral{2B}{\isacharplus}}\ j\isanewline
\isanewline
implicit\ def\ semigroup{\isaliteral{5F}{\isacharunderscore}}int{\isaliteral{3A}{\isacharcolon}}\ semigroup{\isaliteral{5B}{\isacharbrackleft}}BigInt{\isaliteral{5D}{\isacharbrackright}}\ {\isaliteral{3D}{\isacharequal}}\ new\ semigroup{\isaliteral{5B}{\isacharbrackleft}}BigInt{\isaliteral{5D}{\isacharbrackright}}\ {\isaliteral{7B}{\isacharbraceleft}}\isanewline
\ \ val\ {\isaliteral{60}{\isacharbackquote}}Example{\isaliteral{2E}{\isachardot}}mult{\isaliteral{60}{\isacharbackquote}}\ {\isaliteral{3D}{\isacharequal}}\ {\isaliteral{28}{\isacharparenleft}}a{\isaliteral{3A}{\isacharcolon}}\ BigInt{\isaliteral{2C}{\isacharcomma}}\ b{\isaliteral{3A}{\isacharcolon}}\ BigInt{\isaliteral{29}{\isacharparenright}}\ {\isaliteral{3D}{\isacharequal}}{\isaliteral{3E}{\isachargreater}}\ mult{\isaliteral{5F}{\isacharunderscore}}int{\isaliteral{28}{\isacharparenleft}}a{\isaliteral{2C}{\isacharcomma}}\ b{\isaliteral{29}{\isacharparenright}}\isanewline
{\isaliteral{7D}{\isacharbraceright}}\isanewline
\isanewline
def\ neutral{\isaliteral{5F}{\isacharunderscore}}int{\isaliteral{3A}{\isacharcolon}}\ BigInt\ {\isaliteral{3D}{\isacharequal}}\ BigInt{\isaliteral{28}{\isacharparenleft}}{\isadigit{0}}{\isaliteral{29}{\isacharparenright}}\isanewline
\isanewline
implicit\ def\ monoidl{\isaliteral{5F}{\isacharunderscore}}int{\isaliteral{3A}{\isacharcolon}}\ monoidl{\isaliteral{5B}{\isacharbrackleft}}BigInt{\isaliteral{5D}{\isacharbrackright}}\ {\isaliteral{3D}{\isacharequal}}\ new\ monoidl{\isaliteral{5B}{\isacharbrackleft}}BigInt{\isaliteral{5D}{\isacharbrackright}}\ {\isaliteral{7B}{\isacharbraceleft}}\isanewline
\ \ val\ {\isaliteral{60}{\isacharbackquote}}Example{\isaliteral{2E}{\isachardot}}neutral{\isaliteral{60}{\isacharbackquote}}\ {\isaliteral{3D}{\isacharequal}}\ neutral{\isaliteral{5F}{\isacharunderscore}}int\isanewline
\ \ val\ {\isaliteral{60}{\isacharbackquote}}Example{\isaliteral{2E}{\isachardot}}mult{\isaliteral{60}{\isacharbackquote}}\ {\isaliteral{3D}{\isacharequal}}\ {\isaliteral{28}{\isacharparenleft}}a{\isaliteral{3A}{\isacharcolon}}\ BigInt{\isaliteral{2C}{\isacharcomma}}\ b{\isaliteral{3A}{\isacharcolon}}\ BigInt{\isaliteral{29}{\isacharparenright}}\ {\isaliteral{3D}{\isacharequal}}{\isaliteral{3E}{\isachargreater}}\ mult{\isaliteral{5F}{\isacharunderscore}}int{\isaliteral{28}{\isacharparenleft}}a{\isaliteral{2C}{\isacharcomma}}\ b{\isaliteral{29}{\isacharparenright}}\isanewline
{\isaliteral{7D}{\isacharbraceright}}\isanewline
\isanewline
implicit\ def\ monoid{\isaliteral{5F}{\isacharunderscore}}int{\isaliteral{3A}{\isacharcolon}}\ monoid{\isaliteral{5B}{\isacharbrackleft}}BigInt{\isaliteral{5D}{\isacharbrackright}}\ {\isaliteral{3D}{\isacharequal}}\ new\ monoid{\isaliteral{5B}{\isacharbrackleft}}BigInt{\isaliteral{5D}{\isacharbrackright}}\ {\isaliteral{7B}{\isacharbraceleft}}\isanewline
\ \ val\ {\isaliteral{60}{\isacharbackquote}}Example{\isaliteral{2E}{\isachardot}}neutral{\isaliteral{60}{\isacharbackquote}}\ {\isaliteral{3D}{\isacharequal}}\ neutral{\isaliteral{5F}{\isacharunderscore}}int\isanewline
\ \ val\ {\isaliteral{60}{\isacharbackquote}}Example{\isaliteral{2E}{\isachardot}}mult{\isaliteral{60}{\isacharbackquote}}\ {\isaliteral{3D}{\isacharequal}}\ {\isaliteral{28}{\isacharparenleft}}a{\isaliteral{3A}{\isacharcolon}}\ BigInt{\isaliteral{2C}{\isacharcomma}}\ b{\isaliteral{3A}{\isacharcolon}}\ BigInt{\isaliteral{29}{\isacharparenright}}\ {\isaliteral{3D}{\isacharequal}}{\isaliteral{3E}{\isachargreater}}\ mult{\isaliteral{5F}{\isacharunderscore}}int{\isaliteral{28}{\isacharparenleft}}a{\isaliteral{2C}{\isacharcomma}}\ b{\isaliteral{29}{\isacharparenright}}\isanewline
{\isaliteral{7D}{\isacharbraceright}}\isanewline
\isanewline
def\ inverse{\isaliteral{5F}{\isacharunderscore}}int{\isaliteral{28}{\isacharparenleft}}i{\isaliteral{3A}{\isacharcolon}}\ BigInt{\isaliteral{29}{\isacharparenright}}{\isaliteral{3A}{\isacharcolon}}\ BigInt\ {\isaliteral{3D}{\isacharequal}}\ {\isaliteral{28}{\isacharparenleft}}{\isaliteral{2D}{\isacharminus}}\ i{\isaliteral{29}{\isacharparenright}}\isanewline
\isanewline
implicit\ def\ group{\isaliteral{5F}{\isacharunderscore}}int{\isaliteral{3A}{\isacharcolon}}\ group{\isaliteral{5B}{\isacharbrackleft}}BigInt{\isaliteral{5D}{\isacharbrackright}}\ {\isaliteral{3D}{\isacharequal}}\ new\ group{\isaliteral{5B}{\isacharbrackleft}}BigInt{\isaliteral{5D}{\isacharbrackright}}\ {\isaliteral{7B}{\isacharbraceleft}}\isanewline
\ \ val\ {\isaliteral{60}{\isacharbackquote}}Example{\isaliteral{2E}{\isachardot}}inverse{\isaliteral{60}{\isacharbackquote}}\ {\isaliteral{3D}{\isacharequal}}\ {\isaliteral{28}{\isacharparenleft}}a{\isaliteral{3A}{\isacharcolon}}\ BigInt{\isaliteral{29}{\isacharparenright}}\ {\isaliteral{3D}{\isacharequal}}{\isaliteral{3E}{\isachargreater}}\ inverse{\isaliteral{5F}{\isacharunderscore}}int{\isaliteral{28}{\isacharparenleft}}a{\isaliteral{29}{\isacharparenright}}\isanewline
\ \ val\ {\isaliteral{60}{\isacharbackquote}}Example{\isaliteral{2E}{\isachardot}}neutral{\isaliteral{60}{\isacharbackquote}}\ {\isaliteral{3D}{\isacharequal}}\ neutral{\isaliteral{5F}{\isacharunderscore}}int\isanewline
\ \ val\ {\isaliteral{60}{\isacharbackquote}}Example{\isaliteral{2E}{\isachardot}}mult{\isaliteral{60}{\isacharbackquote}}\ {\isaliteral{3D}{\isacharequal}}\ {\isaliteral{28}{\isacharparenleft}}a{\isaliteral{3A}{\isacharcolon}}\ BigInt{\isaliteral{2C}{\isacharcomma}}\ b{\isaliteral{3A}{\isacharcolon}}\ BigInt{\isaliteral{29}{\isacharparenright}}\ {\isaliteral{3D}{\isacharequal}}{\isaliteral{3E}{\isachargreater}}\ mult{\isaliteral{5F}{\isacharunderscore}}int{\isaliteral{28}{\isacharparenleft}}a{\isaliteral{2C}{\isacharcomma}}\ b{\isaliteral{29}{\isacharparenright}}\isanewline
{\isaliteral{7D}{\isacharbraceright}}\isanewline
\isanewline
def\ example{\isaliteral{3A}{\isacharcolon}}\ BigInt\ {\isaliteral{3D}{\isacharequal}}\ pow{\isaliteral{5F}{\isacharunderscore}}int{\isaliteral{5B}{\isacharbrackleft}}BigInt{\isaliteral{5D}{\isacharbrackright}}{\isaliteral{28}{\isacharparenleft}}BigInt{\isaliteral{28}{\isacharparenleft}}{\isadigit{1}}{\isadigit{0}}{\isaliteral{29}{\isacharparenright}}{\isaliteral{2C}{\isacharcomma}}\ BigInt{\isaliteral{28}{\isacharparenleft}}{\isaliteral{2D}{\isacharminus}}\ {\isadigit{2}}{\isaliteral{29}{\isacharparenright}}{\isaliteral{29}{\isacharparenright}}\isanewline
\isanewline
{\isaliteral{7D}{\isacharbraceright}}\ {\isaliteral{2F}{\isacharslash}}{\isaliteral{2A}{\isacharasterisk}}\ object\ Example\ {\isaliteral{2A}{\isacharasterisk}}{\isaliteral{2F}{\isacharslash}}\isanewline%
\end{isamarkuptext}%
\isamarkuptrue%
%
\endisatagquotetypewriter
{\isafoldquotetypewriter}%
%
\isadelimquotetypewriter
%
\endisadelimquotetypewriter
%
\isamarkupsubsection{Inspecting the type class universe%
}
\isamarkuptrue%
%
\begin{isamarkuptext}%
To facilitate orientation in complex subclass structures, two
  diagnostics commands are provided:

  \begin{description}

    \item[\hyperlink{command.print-classes}{\mbox{\isa{\isacommand{print{\isaliteral{5F}{\isacharunderscore}}classes}}}}] print a list of all classes
      together with associated operations etc.

    \item[\hyperlink{command.class-deps}{\mbox{\isa{\isacommand{class{\isaliteral{5F}{\isacharunderscore}}deps}}}}] visualizes the subclass relation
      between all classes as a Hasse diagram.

  \end{description}%
\end{isamarkuptext}%
\isamarkuptrue%
%
\isadelimtheory
%
\endisadelimtheory
%
\isatagtheory
\isacommand{end}\isamarkupfalse%
%
\endisatagtheory
{\isafoldtheory}%
%
\isadelimtheory
%
\endisadelimtheory
\isanewline
\end{isabellebody}%
%%% Local Variables:
%%% mode: latex
%%% TeX-master: "root"
%%% End:
