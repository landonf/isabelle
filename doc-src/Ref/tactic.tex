%% $Id$
\chapter{Tactics} \label{tactics}
\index{tactics|(} Tactics have type \mltydx{tactic}.  This is just an
abbreviation for functions from theorems to theorem sequences, where
the theorems represent states of a backward proof.  Tactics seldom
need to be coded from scratch, as functions; instead they are
expressed using basic tactics and tacticals.

This chapter only presents the primitive tactics.  Substantial proofs require
the power of simplification (Chapter~\ref{simp-chap}) and classical reasoning
(Chapter~\ref{chap:classical}).

\section{Resolution and assumption tactics}
{\bf Resolution} is Isabelle's basic mechanism for refining a subgoal using
a rule.  {\bf Elim-resolution} is particularly suited for elimination
rules, while {\bf destruct-resolution} is particularly suited for
destruction rules.  The {\tt r}, {\tt e}, {\tt d} naming convention is
maintained for several different kinds of resolution tactics, as well as
the shortcuts in the subgoal module.

All the tactics in this section act on a subgoal designated by a positive
integer~$i$.  They fail (by returning the empty sequence) if~$i$ is out of
range.

\subsection{Resolution tactics}
\index{resolution!tactics}
\index{tactics!resolution|bold}
\begin{ttbox} 
resolve_tac  : thm list -> int -> tactic
eresolve_tac : thm list -> int -> tactic
dresolve_tac : thm list -> int -> tactic
forward_tac  : thm list -> int -> tactic 
\end{ttbox}
These perform resolution on a list of theorems, $thms$, representing a list
of object-rules.  When generating next states, they take each of the rules
in the order given.  Each rule may yield several next states, or none:
higher-order resolution may yield multiple resolvents.
\begin{ttdescription}
\item[\ttindexbold{resolve_tac} {\it thms} {\it i}] 
  refines the proof state using the rules, which should normally be
  introduction rules.  It resolves a rule's conclusion with
  subgoal~$i$ of the proof state.

\item[\ttindexbold{eresolve_tac} {\it thms} {\it i}] 
  \index{elim-resolution}
  performs elim-resolution with the rules, which should normally be
  elimination rules.  It resolves with a rule, solves its first premise by
  assumption, and finally {\em deletes\/} that assumption from any new
  subgoals.

\item[\ttindexbold{dresolve_tac} {\it thms} {\it i}] 
  \index{forward proof}\index{destruct-resolution}
  performs destruct-resolution with the rules, which normally should
  be destruction rules.  This replaces an assumption by the result of
  applying one of the rules.

\item[\ttindexbold{forward_tac}]\index{forward proof}
  is like {\tt dresolve_tac} except that the selected assumption is not
  deleted.  It applies a rule to an assumption, adding the result as a new
  assumption.
\end{ttdescription}

\subsection{Assumption tactics}
\index{tactics!assumption|bold}\index{assumptions!tactics for}
\begin{ttbox} 
assume_tac    : int -> tactic
eq_assume_tac : int -> tactic
\end{ttbox} 
\begin{ttdescription}
\item[\ttindexbold{assume_tac} {\it i}] 
attempts to solve subgoal~$i$ by assumption.

\item[\ttindexbold{eq_assume_tac}] 
is like {\tt assume_tac} but does not use unification.  It succeeds (with a
{\em unique\/} next state) if one of the assumptions is identical to the
subgoal's conclusion.  Since it does not instantiate variables, it cannot
make other subgoals unprovable.  It is intended to be called from proof
strategies, not interactively.
\end{ttdescription}

\subsection{Matching tactics} \label{match_tac}
\index{tactics!matching}
\begin{ttbox} 
match_tac  : thm list -> int -> tactic
ematch_tac : thm list -> int -> tactic
dmatch_tac : thm list -> int -> tactic
\end{ttbox}
These are just like the resolution tactics except that they never
instantiate unknowns in the proof state.  Flexible subgoals are not updated
willy-nilly, but are left alone.  Matching --- strictly speaking --- means
treating the unknowns in the proof state as constants; these tactics merely
discard unifiers that would update the proof state.
\begin{ttdescription}
\item[\ttindexbold{match_tac} {\it thms} {\it i}] 
refines the proof state using the rules, matching a rule's
conclusion with subgoal~$i$ of the proof state.

\item[\ttindexbold{ematch_tac}] 
is like {\tt match_tac}, but performs elim-resolution.

\item[\ttindexbold{dmatch_tac}] 
is like {\tt match_tac}, but performs destruct-resolution.
\end{ttdescription}


\subsection{Resolution with instantiation} \label{res_inst_tac}
\index{tactics!instantiation}\index{instantiation}
\begin{ttbox} 
res_inst_tac  : (string*string)list -> thm -> int -> tactic
eres_inst_tac : (string*string)list -> thm -> int -> tactic
dres_inst_tac : (string*string)list -> thm -> int -> tactic
forw_inst_tac : (string*string)list -> thm -> int -> tactic
\end{ttbox}
These tactics are designed for applying rules such as substitution and
induction, which cause difficulties for higher-order unification.  The
tactics accept explicit instantiations for unknowns in the rule ---
typically, in the rule's conclusion.  Each instantiation is a pair
{\tt($v$,$e$)}, where $v$ is an unknown {\em without\/} its leading
question mark!
\begin{itemize}
\item If $v$ is the type unknown {\tt'a}, then
the rule must contain a type unknown \verb$?'a$ of some
sort~$s$, and $e$ should be a type of sort $s$.

\item If $v$ is the unknown {\tt P}, then
the rule must contain an unknown \verb$?P$ of some type~$\tau$,
and $e$ should be a term of some type~$\sigma$ such that $\tau$ and
$\sigma$ are unifiable.  If the unification of $\tau$ and $\sigma$
instantiates any type unknowns in $\tau$, these instantiations
are recorded for application to the rule.
\end{itemize}
Types are instantiated before terms.  Because type instantiations are
inferred from term instantiations, explicit type instantiations are seldom
necessary --- if \verb$?t$ has type \verb$?'a$, then the instantiation list
\verb$[("'a","bool"),("t","True")]$ may be simplified to
\verb$[("t","True")]$.  Type unknowns in the proof state may cause
failure because the tactics cannot instantiate them.

The instantiation tactics act on a given subgoal.  Terms in the
instantiations are type-checked in the context of that subgoal --- in
particular, they may refer to that subgoal's parameters.  Any unknowns in
the terms receive subscripts and are lifted over the parameters; thus, you
may not refer to unknowns in the subgoal.

\begin{ttdescription}
\item[\ttindexbold{res_inst_tac} {\it insts} {\it thm} {\it i}]
instantiates the rule {\it thm} with the instantiations {\it insts}, as
described above, and then performs resolution on subgoal~$i$.  Resolution
typically causes further instantiations; you need not give explicit
instantiations for every unknown in the rule.

\item[\ttindexbold{eres_inst_tac}] 
is like {\tt res_inst_tac}, but performs elim-resolution.

\item[\ttindexbold{dres_inst_tac}] 
is like {\tt res_inst_tac}, but performs destruct-resolution.

\item[\ttindexbold{forw_inst_tac}] 
is like {\tt dres_inst_tac} except that the selected assumption is not
deleted.  It applies the instantiated rule to an assumption, adding the
result as a new assumption.
\end{ttdescription}


\section{Other basic tactics}
\subsection{Tactic shortcuts}
\index{shortcuts!for tactics}
\index{tactics!resolution}\index{tactics!assumption}
\index{tactics!meta-rewriting}
\begin{ttbox} 
rtac     :      thm -> int -> tactic
etac     :      thm -> int -> tactic
dtac     :      thm -> int -> tactic
atac     :             int -> tactic
ares_tac : thm list -> int -> tactic
rewtac   :      thm ->        tactic
\end{ttbox}
These abbreviate common uses of tactics.
\begin{ttdescription}
\item[\ttindexbold{rtac} {\it thm} {\it i}] 
abbreviates \hbox{\tt resolve_tac [{\it thm}] {\it i}}, doing resolution.

\item[\ttindexbold{etac} {\it thm} {\it i}] 
abbreviates \hbox{\tt eresolve_tac [{\it thm}] {\it i}}, doing elim-resolution.

\item[\ttindexbold{dtac} {\it thm} {\it i}] 
abbreviates \hbox{\tt dresolve_tac [{\it thm}] {\it i}}, doing
destruct-resolution.

\item[\ttindexbold{atac} {\it i}] 
abbreviates \hbox{\tt assume_tac {\it i}}, doing proof by assumption.

\item[\ttindexbold{ares_tac} {\it thms} {\it i}] 
tries proof by assumption and resolution; it abbreviates
\begin{ttbox}
assume_tac {\it i} ORELSE resolve_tac {\it thms} {\it i}
\end{ttbox}

\item[\ttindexbold{rewtac} {\it def}] 
abbreviates \hbox{\tt rewrite_goals_tac [{\it def}]}, unfolding a definition.
\end{ttdescription}


\subsection{Inserting premises and facts}\label{cut_facts_tac}
\index{tactics!for inserting facts}\index{assumptions!inserting}
\begin{ttbox} 
cut_facts_tac : thm list -> int -> tactic
cut_inst_tac  : (string*string)list -> thm -> int -> tactic
subgoal_tac   : string -> int -> tactic
subgoal_tacs  : string list -> int -> tactic
\end{ttbox}
These tactics add assumptions to a subgoal.
\begin{ttdescription}
\item[\ttindexbold{cut_facts_tac} {\it thms} {\it i}] 
  adds the {\it thms} as new assumptions to subgoal~$i$.  Once they have
  been inserted as assumptions, they become subject to tactics such as {\tt
    eresolve_tac} and {\tt rewrite_goals_tac}.  Only rules with no premises
  are inserted: Isabelle cannot use assumptions that contain $\Imp$
  or~$\Forall$.  Sometimes the theorems are premises of a rule being
  derived, returned by~{\tt goal}; instead of calling this tactic, you
  could state the goal with an outermost meta-quantifier.

\item[\ttindexbold{cut_inst_tac} {\it insts} {\it thm} {\it i}]
  instantiates the {\it thm} with the instantiations {\it insts}, as
  described in \S\ref{res_inst_tac}.  It adds the resulting theorem as a
  new assumption to subgoal~$i$. 

\item[\ttindexbold{subgoal_tac} {\it formula} {\it i}] 
adds the {\it formula} as a assumption to subgoal~$i$, and inserts the same
{\it formula} as a new subgoal, $i+1$.

\item[\ttindexbold{subgoals_tac} {\it formulae} {\it i}] 
  uses {\tt subgoal_tac} to add the members of the list of {\it
    formulae} as assumptions to subgoal~$i$. 
\end{ttdescription}


\subsection{``Putting off'' a subgoal}
\begin{ttbox} 
defer_tac : int -> tactic
\end{ttbox}
\begin{ttdescription}
\item[\ttindexbold{defer_tac} {\it i}] 
  moves subgoal~$i$ to the last position in the proof state.  It can be
  useful when correcting a proof script: if the tactic given for subgoal~$i$
  fails, calling {\tt defer_tac} instead will let you continue with the rest
  of the script.

  The tactic fails if subgoal~$i$ does not exist or if the proof state
  contains type unknowns. 
\end{ttdescription}


\subsection{Definitions and meta-level rewriting}
\index{tactics!meta-rewriting|bold}\index{meta-rewriting|bold}
\index{definitions}

Definitions in Isabelle have the form $t\equiv u$, where $t$ is typically a
constant or a constant applied to a list of variables, for example $\it
sqr(n)\equiv n\times n$.  (Conditional definitions, $\phi\Imp t\equiv u$,
are not supported.)  {\bf Unfolding} the definition ${t\equiv u}$ means using
it as a rewrite rule, replacing~$t$ by~$u$ throughout a theorem.  {\bf
Folding} $t\equiv u$ means replacing~$u$ by~$t$.  Rewriting continues until
no rewrites are applicable to any subterm.

There are rules for unfolding and folding definitions; Isabelle does not do
this automatically.  The corresponding tactics rewrite the proof state,
yielding a single next state.  See also the {\tt goalw} command, which is the
easiest way of handling definitions.
\begin{ttbox} 
rewrite_goals_tac : thm list -> tactic
rewrite_tac       : thm list -> tactic
fold_goals_tac    : thm list -> tactic
fold_tac          : thm list -> tactic
\end{ttbox}
\begin{ttdescription}
\item[\ttindexbold{rewrite_goals_tac} {\it defs}]  
unfolds the {\it defs} throughout the subgoals of the proof state, while
leaving the main goal unchanged.  Use \ttindex{SELECT_GOAL} to restrict it to a
particular subgoal.

\item[\ttindexbold{rewrite_tac} {\it defs}]  
unfolds the {\it defs} throughout the proof state, including the main goal
--- not normally desirable!

\item[\ttindexbold{fold_goals_tac} {\it defs}]  
folds the {\it defs} throughout the subgoals of the proof state, while
leaving the main goal unchanged.

\item[\ttindexbold{fold_tac} {\it defs}]  
folds the {\it defs} throughout the proof state.
\end{ttdescription}


\subsection{Theorems useful with tactics}
\index{theorems!of pure theory}
\begin{ttbox} 
asm_rl: thm 
cut_rl: thm 
\end{ttbox}
\begin{ttdescription}
\item[\tdx{asm_rl}] 
is $\psi\Imp\psi$.  Under elim-resolution it does proof by assumption, and
\hbox{\tt eresolve_tac (asm_rl::{\it thms}) {\it i}} is equivalent to
\begin{ttbox} 
assume_tac {\it i}  ORELSE  eresolve_tac {\it thms} {\it i}
\end{ttbox}

\item[\tdx{cut_rl}] 
is $\List{\psi\Imp\theta,\psi}\Imp\theta$.  It is useful for inserting
assumptions; it underlies {\tt forward_tac}, {\tt cut_facts_tac}
and {\tt subgoal_tac}.
\end{ttdescription}


\section{Obscure tactics}

\subsection{Renaming parameters in a goal} \index{parameters!renaming}
\begin{ttbox} 
rename_tac        : string -> int -> tactic
rename_last_tac   : string -> string list -> int -> tactic
Logic.set_rename_prefix : string -> unit
Logic.auto_rename       : bool ref      \hfill{\bf initially false}
\end{ttbox}
When creating a parameter, Isabelle chooses its name by matching variable
names via the object-rule.  Given the rule $(\forall I)$ formalized as
$\left(\Forall x. P(x)\right) \Imp \forall x.P(x)$, Isabelle will note that
the $\Forall$-bound variable in the premise has the same name as the
$\forall$-bound variable in the conclusion.  

Sometimes there is insufficient information and Isabelle chooses an
arbitrary name.  The renaming tactics let you override Isabelle's choice.
Because renaming parameters has no logical effect on the proof state, the
{\tt by} command prints the message {\tt Warning:\ same as previous
level}.

Alternatively, you can suppress the naming mechanism described above and
have Isabelle generate uniform names for parameters.  These names have the
form $p${\tt a}, $p${\tt b}, $p${\tt c},~\ldots, where $p$ is any desired
prefix.  They are ugly but predictable.

\begin{ttdescription}
\item[\ttindexbold{rename_tac} {\it str} {\it i}] 
interprets the string {\it str} as a series of blank-separated variable
names, and uses them to rename the parameters of subgoal~$i$.  The names
must be distinct.  If there are fewer names than parameters, then the
tactic renames the innermost parameters and may modify the remaining ones
to ensure that all the parameters are distinct.

\item[\ttindexbold{rename_last_tac} {\it prefix} {\it suffixes} {\it i}] 
generates a list of names by attaching each of the {\it suffixes\/} to the 
{\it prefix}.  It is intended for coding structural induction tactics,
where several of the new parameters should have related names.

\item[\ttindexbold{Logic.set_rename_prefix} {\it prefix};] 
sets the prefix for uniform renaming to~{\it prefix}.  The default prefix
is {\tt"k"}.

\item[\ttindexbold{Logic.auto_rename} := true;] 
makes Isabelle generate uniform names for parameters. 
\end{ttdescription}


\subsection{Manipulating assumptions}
\index{assumptions!rotating}
\begin{ttbox} 
thin_tac   : string -> int -> tactic
rotate_tac : int -> int -> tactic
\end{ttbox}
\begin{ttdescription}
\item[\ttindexbold{thin_tac} {\it formula} $i$]  
\index{assumptions!deleting}
deletes the specified assumption from subgoal $i$.  Often the assumption
can be abbreviated, replacing subformul{\ae} by unknowns; the first matching
assumption will be deleted.  Removing useless assumptions from a subgoal
increases its readability and can make search tactics run faster.

\item[\ttindexbold{rotate_tac} $n$ $i$]  
\index{assumptions!rotating}
rotates the assumptions of subgoal $i$ by $n$ positions: from right to left
if $n$ is positive, and from left to right if $n$ is negative.  This is 
sometimes necessary in connection with \ttindex{asm_full_simp_tac}, which 
processes assumptions from left to right.
\end{ttdescription}


\subsection{Tidying the proof state}
\index{duplicate subgoals!removing}
\index{parameters!removing unused}
\index{flex-flex constraints}
\begin{ttbox} 
distinct_subgoals_tac : tactic
prune_params_tac      : tactic
flexflex_tac          : tactic
\end{ttbox}
\begin{ttdescription}
\item[\ttindexbold{distinct_subgoals_tac}]  
  removes duplicate subgoals from a proof state.  (These arise especially
  in \ZF{}, where the subgoals are essentially type constraints.)

\item[\ttindexbold{prune_params_tac}]  
  removes unused parameters from all subgoals of the proof state.  It works
  by rewriting with the theorem $(\Forall x. V)\equiv V$.  This tactic can
  make the proof state more readable.  It is used with
  \ttindex{rule_by_tactic} to simplify the resulting theorem.

\item[\ttindexbold{flexflex_tac}]  
  removes all flex-flex pairs from the proof state by applying the trivial
  unifier.  This drastic step loses information, and should only be done as
  the last step of a proof.

  Flex-flex constraints arise from difficult cases of higher-order
  unification.  To prevent this, use \ttindex{res_inst_tac} to instantiate
  some variables in a rule~(\S\ref{res_inst_tac}).  Normally flex-flex
  constraints can be ignored; they often disappear as unknowns get
  instantiated.
\end{ttdescription}


\subsection{Composition: resolution without lifting}
\index{tactics!for composition}
\begin{ttbox}
compose_tac: (bool * thm * int) -> int -> tactic
\end{ttbox}
{\bf Composing} two rules means resolving them without prior lifting or
renaming of unknowns.  This low-level operation, which underlies the
resolution tactics, may occasionally be useful for special effects.
A typical application is \ttindex{res_inst_tac}, which lifts and instantiates a
rule, then passes the result to {\tt compose_tac}.
\begin{ttdescription}
\item[\ttindexbold{compose_tac} ($flag$, $rule$, $m$) $i$] 
refines subgoal~$i$ using $rule$, without lifting.  The $rule$ is taken to
have the form $\List{\psi@1; \ldots; \psi@m} \Imp \psi$, where $\psi$ need
not be atomic; thus $m$ determines the number of new subgoals.  If
$flag$ is {\tt true} then it performs elim-resolution --- it solves the
first premise of~$rule$ by assumption and deletes that assumption.
\end{ttdescription}


\section{Managing lots of rules}
These operations are not intended for interactive use.  They are concerned
with the processing of large numbers of rules in automatic proof
strategies.  Higher-order resolution involving a long list of rules is
slow.  Filtering techniques can shorten the list of rules given to
resolution, and can also detect whether a subgoal is too flexible,
with too many rules applicable.

\subsection{Combined resolution and elim-resolution} \label{biresolve_tac}
\index{tactics!resolution}
\begin{ttbox} 
biresolve_tac   : (bool*thm)list -> int -> tactic
bimatch_tac     : (bool*thm)list -> int -> tactic
subgoals_of_brl : bool*thm -> int
lessb           : (bool*thm) * (bool*thm) -> bool
\end{ttbox}
{\bf Bi-resolution} takes a list of $\it (flag,rule)$ pairs.  For each
pair, it applies resolution if the flag is~{\tt false} and
elim-resolution if the flag is~{\tt true}.  A single tactic call handles a
mixture of introduction and elimination rules.

\begin{ttdescription}
\item[\ttindexbold{biresolve_tac} {\it brls} {\it i}] 
refines the proof state by resolution or elim-resolution on each rule, as
indicated by its flag.  It affects subgoal~$i$ of the proof state.

\item[\ttindexbold{bimatch_tac}] 
is like {\tt biresolve_tac}, but performs matching: unknowns in the
proof state are never updated (see~\S\ref{match_tac}).

\item[\ttindexbold{subgoals_of_brl}({\it flag},{\it rule})] 
returns the number of new subgoals that bi-resolution would yield for the
pair (if applied to a suitable subgoal).  This is $n$ if the flag is
{\tt false} and $n-1$ if the flag is {\tt true}, where $n$ is the number
of premises of the rule.  Elim-resolution yields one fewer subgoal than
ordinary resolution because it solves the major premise by assumption.

\item[\ttindexbold{lessb} ({\it brl1},{\it brl2})] 
returns the result of 
\begin{ttbox}
subgoals_of_brl{\it brl1} < subgoals_of_brl{\it brl2}
\end{ttbox}
\end{ttdescription}
Note that \hbox{\tt sort lessb {\it brls}} sorts a list of $\it
(flag,rule)$ pairs by the number of new subgoals they will yield.  Thus,
those that yield the fewest subgoals should be tried first.


\subsection{Discrimination nets for fast resolution}\label{filt_resolve_tac}
\index{discrimination nets|bold}
\index{tactics!resolution}
\begin{ttbox} 
net_resolve_tac  : thm list -> int -> tactic
net_match_tac    : thm list -> int -> tactic
net_biresolve_tac: (bool*thm) list -> int -> tactic
net_bimatch_tac  : (bool*thm) list -> int -> tactic
filt_resolve_tac : thm list -> int -> int -> tactic
could_unify      : term*term->bool
filter_thms      : (term*term->bool) -> int*term*thm list -> thm list
\end{ttbox}
The module {\tt Net} implements a discrimination net data structure for
fast selection of rules \cite[Chapter 14]{charniak80}.  A term is
classified by the symbol list obtained by flattening it in preorder.
The flattening takes account of function applications, constants, and free
and bound variables; it identifies all unknowns and also regards
\index{lambda abs@$\lambda$-abstractions}
$\lambda$-abstractions as unknowns, since they could $\eta$-contract to
anything.  

A discrimination net serves as a polymorphic dictionary indexed by terms.
The module provides various functions for inserting and removing items from
nets.  It provides functions for returning all items whose term could match
or unify with a target term.  The matching and unification tests are
overly lax (due to the identifications mentioned above) but they serve as
useful filters.

A net can store introduction rules indexed by their conclusion, and
elimination rules indexed by their major premise.  Isabelle provides
several functions for `compiling' long lists of rules into fast
resolution tactics.  When supplied with a list of theorems, these functions
build a discrimination net; the net is used when the tactic is applied to a
goal.  To avoid repeatedly constructing the nets, use currying: bind the
resulting tactics to \ML{} identifiers.

\begin{ttdescription}
\item[\ttindexbold{net_resolve_tac} {\it thms}] 
builds a discrimination net to obtain the effect of a similar call to {\tt
resolve_tac}.

\item[\ttindexbold{net_match_tac} {\it thms}] 
builds a discrimination net to obtain the effect of a similar call to {\tt
match_tac}.

\item[\ttindexbold{net_biresolve_tac} {\it brls}] 
builds a discrimination net to obtain the effect of a similar call to {\tt
biresolve_tac}.

\item[\ttindexbold{net_bimatch_tac} {\it brls}] 
builds a discrimination net to obtain the effect of a similar call to {\tt
bimatch_tac}.

\item[\ttindexbold{filt_resolve_tac} {\it thms} {\it maxr} {\it i}] 
uses discrimination nets to extract the {\it thms} that are applicable to
subgoal~$i$.  If more than {\it maxr\/} theorems are applicable then the
tactic fails.  Otherwise it calls {\tt resolve_tac}.  

This tactic helps avoid runaway instantiation of unknowns, for example in
type inference.

\item[\ttindexbold{could_unify} ({\it t},{\it u})] 
returns {\tt false} if~$t$ and~$u$ are `obviously' non-unifiable, and
otherwise returns~{\tt true}.  It assumes all variables are distinct,
reporting that {\tt ?a=?a} may unify with {\tt 0=1}.

\item[\ttindexbold{filter_thms} $could\; (limit,prem,thms)$] 
returns the list of potentially resolvable rules (in {\it thms\/}) for the
subgoal {\it prem}, using the predicate {\it could\/} to compare the
conclusion of the subgoal with the conclusion of each rule.  The resulting list
is no longer than {\it limit}.
\end{ttdescription}


\section{Programming tools for proof strategies}
Do not consider using the primitives discussed in this section unless you
really need to code tactics from scratch.

\subsection{Operations on type {\tt tactic}}
\index{tactics!primitives for coding} A tactic maps theorems to sequences of
theorems.  The type constructor for sequences (lazy lists) is called
\mltydx{Sequence.seq}.  To simplify the types of tactics and tacticals,
Isabelle defines a type abbreviation:
\begin{ttbox} 
type tactic = thm -> thm Sequence.seq
\end{ttbox} 
The following operations provide means for coding tactics in a clean style.
\begin{ttbox} 
PRIMITIVE :                  (thm -> thm) -> tactic  
SUBGOAL   : ((term*int) -> tactic) -> int -> tactic
\end{ttbox} 
\begin{ttdescription}
\item[\ttindexbold{PRIMITIVE} $f$] packages the meta-rule~$f$ as a tactic that
  applies $f$ to the proof state and returns the result as a one-element
  sequence.  If $f$ raises an exception, then the tactic's result is the empty
  sequence.

\item[\ttindexbold{SUBGOAL} $f$ $i$] 
extracts subgoal~$i$ from the proof state as a term~$t$, and computes a
tactic by calling~$f(t,i)$.  It applies the resulting tactic to the same
state.  The tactic body is expressed using tactics and tacticals, but may
peek at a particular subgoal:
\begin{ttbox} 
SUBGOAL (fn (t,i) => {\it tactic-valued expression})
\end{ttbox} 
\end{ttdescription}


\subsection{Tracing}
\index{tactics!tracing}
\index{tracing!of tactics}
\begin{ttbox} 
pause_tac: tactic
print_tac: tactic
\end{ttbox}
These tactics print tracing information when they are applied to a proof
state.  Their output may be difficult to interpret.  Note that certain of
the searching tacticals, such as {\tt REPEAT}, have built-in tracing
options.
\begin{ttdescription}
\item[\ttindexbold{pause_tac}] 
prints {\footnotesize\tt** Press RETURN to continue:} and then reads a line
from the terminal.  If this line is blank then it returns the proof state
unchanged; otherwise it fails (which may terminate a repetition).

\item[\ttindexbold{print_tac}] 
returns the proof state unchanged, with the side effect of printing it at
the terminal.
\end{ttdescription}


\section{Sequences}
\index{sequences (lazy lists)|bold}
The module {\tt Sequence} declares a type of lazy lists.  It uses
Isabelle's type \mltydx{option} to represent the possible presence
(\ttindexbold{Some}) or absence (\ttindexbold{None}) of
a value:
\begin{ttbox}
datatype 'a option = None  |  Some of 'a;
\end{ttbox}
For clarity, the module name {\tt Sequence} is omitted from the signature
specifications below; for instance, {\tt null} appears instead of {\tt
  Sequence.null}.

\subsection{Basic operations on sequences}
\begin{ttbox} 
null   : 'a seq
seqof  : (unit -> ('a * 'a seq) option) -> 'a seq
single : 'a -> 'a seq
pull   : 'a seq -> ('a * 'a seq) option
\end{ttbox}
\begin{ttdescription}
\item[Sequence.null] 
is the empty sequence.

\item[\tt Sequence.seqof (fn()=> Some($x$,$s$))] 
constructs the sequence with head~$x$ and tail~$s$, neither of which is
evaluated.

\item[Sequence.single $x$] 
constructs the sequence containing the single element~$x$.

\item[Sequence.pull $s$] 
returns {\tt None} if the sequence is empty and {\tt Some($x$,$s'$)} if the
sequence has head~$x$ and tail~$s'$.  Warning: calling \hbox{Sequence.pull
$s$} again will {\it recompute\/} the value of~$x$; it is not stored!
\end{ttdescription}


\subsection{Converting between sequences and lists}
\begin{ttbox} 
chop      : int * 'a seq -> 'a list * 'a seq
list_of_s : 'a seq -> 'a list
s_of_list : 'a list -> 'a seq
\end{ttbox}
\begin{ttdescription}
\item[Sequence.chop($n$,$s$)] 
returns the first~$n$ elements of~$s$ as a list, paired with the remaining
elements of~$s$.  If $s$ has fewer than~$n$ elements, then so will the
list.

\item[Sequence.list_of_s $s$] 
returns the elements of~$s$, which must be finite, as a list.

\item[Sequence.s_of_list $l$] 
creates a sequence containing the elements of~$l$.
\end{ttdescription}


\subsection{Combining sequences}
\begin{ttbox} 
append     : 'a seq * 'a seq -> 'a seq
interleave : 'a seq * 'a seq -> 'a seq
flats      : 'a seq seq -> 'a seq
maps       : ('a -> 'b) -> 'a seq -> 'b seq
filters    : ('a -> bool) -> 'a seq -> 'a seq
\end{ttbox} 
\begin{ttdescription}
\item[Sequence.append($s@1$,$s@2$)] 
concatenates $s@1$ to $s@2$.

\item[Sequence.interleave($s@1$,$s@2$)] 
joins $s@1$ with $s@2$ by interleaving their elements.  The result contains
all the elements of the sequences, even if both are infinite.

\item[Sequence.flats $ss$] 
concatenates a sequence of sequences.

\item[Sequence.maps $f$ $s$] 
applies $f$ to every element of~$s=x@1,x@2,\ldots$, yielding the sequence
$f(x@1),f(x@2),\ldots$.

\item[Sequence.filters $p$ $s$] 
returns the sequence consisting of all elements~$x$ of~$s$ such that $p(x)$
is {\tt true}.
\end{ttdescription}

\index{tactics|)}
