
\chapter{Tactics} \label{tactics}
\index{tactics|(}

\section{Other basic tactics}

\subsection{Inserting premises and facts}\label{cut_facts_tac}
\index{tactics!for inserting facts}\index{assumptions!inserting}
\begin{ttbox} 
cut_facts_tac : thm list -> int -> tactic
cut_inst_tac  : (string*string)list -> thm -> int -> tactic
\end{ttbox}
These tactics add assumptions to a subgoal.
\begin{ttdescription}
\item[\ttindexbold{cut_facts_tac} {\it thms} {\it i}] 
  adds the {\it thms} as new assumptions to subgoal~$i$.  Once they have
  been inserted as assumptions, they become subject to tactics such as {\tt
    eresolve_tac} and {\tt rewrite_goals_tac}.  Only rules with no premises
  are inserted: Isabelle cannot use assumptions that contain $\Imp$
  or~$\Forall$.  Sometimes the theorems are premises of a rule being
  derived, returned by~{\tt goal}; instead of calling this tactic, you
  could state the goal with an outermost meta-quantifier.

\item[\ttindexbold{cut_inst_tac} {\it insts} {\it thm} {\it i}]
  instantiates the {\it thm} with the instantiations {\it insts}, as
  described in {\S}\ref{res_inst_tac}.  It adds the resulting theorem as a
  new assumption to subgoal~$i$. 

\end{ttdescription}


\subsection{Definitions and meta-level rewriting} \label{sec:rewrite_goals}
\index{tactics!meta-rewriting|bold}\index{meta-rewriting|bold}
\index{definitions}

Definitions in Isabelle have the form $t\equiv u$, where $t$ is typically a
constant or a constant applied to a list of variables, for example $\it
sqr(n)\equiv n\times n$.  Conditional definitions, $\phi\Imp t\equiv u$,
are also supported.  {\bf Unfolding} the definition ${t\equiv u}$ means using
it as a rewrite rule, replacing~$t$ by~$u$ throughout a theorem.  {\bf
Folding} $t\equiv u$ means replacing~$u$ by~$t$.  Rewriting continues until
no rewrites are applicable to any subterm.

There are rules for unfolding and folding definitions; Isabelle does not do
this automatically.  The corresponding tactics rewrite the proof state,
yielding a single next state.
\begin{ttbox} 
rewrite_goals_tac : thm list -> tactic
fold_goals_tac    : thm list -> tactic
\end{ttbox}
\begin{ttdescription}
\item[\ttindexbold{rewrite_goals_tac} {\it defs}]  
unfolds the {\it defs} throughout the subgoals of the proof state, while
leaving the main goal unchanged.  Use \ttindex{SELECT_GOAL} to restrict it to a
particular subgoal.

\item[\ttindexbold{fold_goals_tac} {\it defs}]  
folds the {\it defs} throughout the subgoals of the proof state, while
leaving the main goal unchanged.
\end{ttdescription}

\begin{warn}
  These tactics only cope with definitions expressed as meta-level
  equalities ($\equiv$).  More general equivalences are handled by the
  simplifier, provided that it is set up appropriately for your logic
  (see Chapter~\ref{chap:simplification}).
\end{warn}

\subsection{Theorems useful with tactics}
\index{theorems!of pure theory}
\begin{ttbox} 
asm_rl: thm 
\end{ttbox}
\begin{ttdescription}
\item[\tdx{asm_rl}] 
is $\psi\Imp\psi$.  Under elim-resolution it does proof by assumption, and
\hbox{\tt eresolve_tac (asm_rl::{\it thms}) {\it i}} is equivalent to
\begin{ttbox} 
assume_tac {\it i}  ORELSE  eresolve_tac {\it thms} {\it i}
\end{ttbox}

\end{ttdescription}


\section{Obscure tactics}

\subsection{Manipulating assumptions}
\begin{ttbox} 
thin_tac   : string -> int -> tactic
\end{ttbox}
\begin{ttdescription}
\item[\ttindexbold{thin_tac} {\it formula} $i$]  
\index{assumptions!deleting}
deletes the specified assumption from subgoal $i$.  Often the assumption
can be abbreviated, replacing subformul{\ae} by unknowns; the first matching
assumption will be deleted.  Removing useless assumptions from a subgoal
increases its readability and can make search tactics run faster.

\end{ttdescription}


\subsection{Composition: resolution without lifting}
\index{tactics!for composition}
\begin{ttbox}
compose_tac: (bool * thm * int) -> int -> tactic
\end{ttbox}
{\bf Composing} two rules means resolving them without prior lifting or
renaming of unknowns.  This low-level operation, which underlies the
resolution tactics, may occasionally be useful for special effects.
A typical application is \ttindex{res_inst_tac}, which lifts and instantiates a
rule, then passes the result to {\tt compose_tac}.
\begin{ttdescription}
\item[\ttindexbold{compose_tac} ($flag$, $rule$, $m$) $i$] 
refines subgoal~$i$ using $rule$, without lifting.  The $rule$ is taken to
have the form $\List{\psi@1; \ldots; \psi@m} \Imp \psi$, where $\psi$ need
not be atomic; thus $m$ determines the number of new subgoals.  If
$flag$ is {\tt true} then it performs elim-resolution --- it solves the
first premise of~$rule$ by assumption and deletes that assumption.
\end{ttdescription}


\section{*Managing lots of rules}
These operations are not intended for interactive use.  They are concerned
with the processing of large numbers of rules in automatic proof
strategies.  Higher-order resolution involving a long list of rules is
slow.  Filtering techniques can shorten the list of rules given to
resolution, and can also detect whether a subgoal is too flexible,
with too many rules applicable.

\subsection{Combined resolution and elim-resolution} \label{biresolve_tac}
\index{tactics!resolution}
\begin{ttbox} 
biresolve_tac   : (bool*thm)list -> int -> tactic
bimatch_tac     : (bool*thm)list -> int -> tactic
subgoals_of_brl : bool*thm -> int
lessb           : (bool*thm) * (bool*thm) -> bool
\end{ttbox}
{\bf Bi-resolution} takes a list of $\it (flag,rule)$ pairs.  For each
pair, it applies resolution if the flag is~{\tt false} and
elim-resolution if the flag is~{\tt true}.  A single tactic call handles a
mixture of introduction and elimination rules.

\begin{ttdescription}
\item[\ttindexbold{biresolve_tac} {\it brls} {\it i}] 
refines the proof state by resolution or elim-resolution on each rule, as
indicated by its flag.  It affects subgoal~$i$ of the proof state.

\item[\ttindexbold{bimatch_tac}] 
is like {\tt biresolve_tac}, but performs matching: unknowns in the
proof state are never updated (see~{\S}\ref{match_tac}).

\item[\ttindexbold{subgoals_of_brl}({\it flag},{\it rule})] 
returns the number of new subgoals that bi-res\-o\-lu\-tion would yield for the
pair (if applied to a suitable subgoal).  This is $n$ if the flag is
{\tt false} and $n-1$ if the flag is {\tt true}, where $n$ is the number
of premises of the rule.  Elim-resolution yields one fewer subgoal than
ordinary resolution because it solves the major premise by assumption.

\item[\ttindexbold{lessb} ({\it brl1},{\it brl2})] 
returns the result of 
\begin{ttbox}
subgoals_of_brl{\it brl1} < subgoals_of_brl{\it brl2}
\end{ttbox}
\end{ttdescription}
Note that \hbox{\tt sort lessb {\it brls}} sorts a list of $\it
(flag,rule)$ pairs by the number of new subgoals they will yield.  Thus,
those that yield the fewest subgoals should be tried first.


\subsection{Discrimination nets for fast resolution}\label{filt_resolve_tac}
\index{discrimination nets|bold}
\index{tactics!resolution}
\begin{ttbox} 
net_resolve_tac  : thm list -> int -> tactic
net_match_tac    : thm list -> int -> tactic
net_biresolve_tac: (bool*thm) list -> int -> tactic
net_bimatch_tac  : (bool*thm) list -> int -> tactic
filt_resolve_tac : thm list -> int -> int -> tactic
could_unify      : term*term->bool
filter_thms      : (term*term->bool) -> int*term*thm list -> thm{\ts}list
\end{ttbox}
The module {\tt Net} implements a discrimination net data structure for
fast selection of rules \cite[Chapter 14]{charniak80}.  A term is
classified by the symbol list obtained by flattening it in preorder.
The flattening takes account of function applications, constants, and free
and bound variables; it identifies all unknowns and also regards
\index{lambda abs@$\lambda$-abstractions}
$\lambda$-abstractions as unknowns, since they could $\eta$-contract to
anything.  

A discrimination net serves as a polymorphic dictionary indexed by terms.
The module provides various functions for inserting and removing items from
nets.  It provides functions for returning all items whose term could match
or unify with a target term.  The matching and unification tests are
overly lax (due to the identifications mentioned above) but they serve as
useful filters.

A net can store introduction rules indexed by their conclusion, and
elimination rules indexed by their major premise.  Isabelle provides
several functions for `compiling' long lists of rules into fast
resolution tactics.  When supplied with a list of theorems, these functions
build a discrimination net; the net is used when the tactic is applied to a
goal.  To avoid repeatedly constructing the nets, use currying: bind the
resulting tactics to \ML{} identifiers.

\begin{ttdescription}
\item[\ttindexbold{net_resolve_tac} {\it thms}] 
builds a discrimination net to obtain the effect of a similar call to {\tt
resolve_tac}.

\item[\ttindexbold{net_match_tac} {\it thms}] 
builds a discrimination net to obtain the effect of a similar call to {\tt
match_tac}.

\item[\ttindexbold{net_biresolve_tac} {\it brls}] 
builds a discrimination net to obtain the effect of a similar call to {\tt
biresolve_tac}.

\item[\ttindexbold{net_bimatch_tac} {\it brls}] 
builds a discrimination net to obtain the effect of a similar call to {\tt
bimatch_tac}.

\item[\ttindexbold{filt_resolve_tac} {\it thms} {\it maxr} {\it i}] 
uses discrimination nets to extract the {\it thms} that are applicable to
subgoal~$i$.  If more than {\it maxr\/} theorems are applicable then the
tactic fails.  Otherwise it calls {\tt resolve_tac}.  

This tactic helps avoid runaway instantiation of unknowns, for example in
type inference.

\item[\ttindexbold{could_unify} ({\it t},{\it u})] 
returns {\tt false} if~$t$ and~$u$ are `obviously' non-unifiable, and
otherwise returns~{\tt true}.  It assumes all variables are distinct,
reporting that {\tt ?a=?a} may unify with {\tt 0=1}.

\item[\ttindexbold{filter_thms} $could\; (limit,prem,thms)$] 
returns the list of potentially resolvable rules (in {\it thms\/}) for the
subgoal {\it prem}, using the predicate {\it could\/} to compare the
conclusion of the subgoal with the conclusion of each rule.  The resulting list
is no longer than {\it limit}.
\end{ttdescription}

\index{tactics|)}


%%% Local Variables: 
%%% mode: latex
%%% TeX-master: "ref"
%%% End: 
