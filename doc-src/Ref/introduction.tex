
\chapter{Basic Use of Isabelle}\index{sessions|(} 

\section{Ending a session}
\begin{ttbox} 
quit    : unit -> unit
exit    : int -> unit
commit  : unit -> bool
\end{ttbox}
\begin{ttdescription}
\item[\ttindexbold{quit}();] ends the Isabelle session, without saving
  the state.
  
\item[\ttindexbold{exit} \(i\);] similar to {\tt quit}, passing return
  code \(i\) to the operating system.

\item[\ttindexbold{commit}();] saves the current state without ending
  the session, provided that the logic image is opened read-write;
  return value {\tt false} indicates an error.
\end{ttdescription}

Typing control-D also finishes the session in essentially the same way
as the sequence {\tt commit(); quit();} would.


\section{Reading ML files}
\index{files!reading}
\begin{ttbox} 
cd              : string -> unit
pwd             : unit -> string
use             : string -> unit
time_use        : string -> unit
\end{ttbox}
\begin{ttdescription}
\item[\ttindexbold{cd} "{\it dir}";] changes the current directory to
  {\it dir}.  This is the default directory for reading files.
  
\item[\ttindexbold{pwd}();] returns the full path of the current
  directory.

\item[\ttindexbold{use} "$file$";]  
reads the given {\it file} as input to the \ML{} session.  Reading a file
of Isabelle commands is the usual way of replaying a proof.

\item[\ttindexbold{time_use} "$file$";]  
performs {\tt use~"$file$"} and prints the total execution time.
\end{ttdescription}

The $dir$ and $file$ specifications of the \texttt{cd} and \texttt{use}
commands may contain path variables (e.g.\ \texttt{\$ISABELLE_HOME}) that are
expanded appropriately.  Note that \texttt{\~\relax} abbreviates
\texttt{\$HOME}, and \texttt{\~\relax\~\relax} abbreviates
\texttt{\$ISABELLE_HOME}\index{*\$ISABELLE_HOME}.  The syntax for path
specifications follows Unix conventions.


\section{Reading theories}\label{sec:intro-theories}
\index{theories!reading}

In Isabelle, any kind of declarations, definitions, etc.\ are organized around
named \emph{theory} objects.  Logical reasoning always takes place within a
certain theory context, which may be switched at any time.  Theory $name$ is
defined by a theory file $name$\texttt{.thy}, containing declarations of
\texttt{consts}, \texttt{types}, \texttt{defs}, etc.\ (see
\S\ref{sec:ref-defining-theories} for more details on concrete syntax).
Furthermore, there may be an associated {\ML} file $name$\texttt{.ML} with
proof scripts that are to be run in the context of the theory.

\begin{ttbox}
context      : theory -> unit
the_context  : unit -> theory
theory       : string -> theory
use_thy      : string -> unit
time_use_thy : string -> unit
update_thy   : string -> unit
\end{ttbox}

\begin{ttdescription}
  
\item[\ttindexbold{context} $thy$;] switches the current theory context.  Any
  subsequent command with ``implicit theory argument'' (e.g.\ \texttt{Goal})
  will refer to $thy$ as its theory.
  
\item[\ttindexbold{the_context}();] obtains the current theory context, or
  raises an error if absent.
  
\item[\ttindexbold{theory} "$name$";] retrieves the theory called $name$ from
  the internal data\-base of loaded theories, raising an error if absent.
  
\item[\ttindexbold{use_thy} "$name$";] reads theory $name$ from the file
  system, looking for $name$\texttt{.thy} and $name$\texttt{.ML} (the latter
  being optional).  It also ensures that all parent theories are loaded as
  well.  In case some older versions have already been present,
  \texttt{use_thy} only tries to reload $name$ itself, but is content with any
  version of its ancestors.
  
\item[\ttindexbold{time_use_thy} "$name$";] same as \texttt{use_thy}, but
  reports the time taken to process the actual theory parts and {\ML} files
  separately.
  
\item[\ttindexbold{update_thy} "$name$";] is similar to \texttt{use_thy}, but
  ensures that theory $name$ is fully up-to-date with respect to the file
  system --- apart from theory $name$ itself, any of its ancestors may be
  reloaded as well.
  
\end{ttdescription}

Note that theories of pre-built logic images (e.g.\ HOL) are marked as
\emph{finished} and cannot be updated any more.  See \S\ref{sec:more-theories}
for further information on Isabelle's theory loader.


\section{Timing}
\index{timing statistics}\index{proofs!timing}
\begin{ttbox} 
timing: bool ref \hfill{\bf initially false}
\end{ttbox}

\begin{ttdescription}
\item[set \ttindexbold{timing};] enables global timing in Isabelle.
  This information is compiler-dependent.
\end{ttdescription}

\index{sessions|)}


%%% Local Variables: 
%%% mode: latex
%%% TeX-master: "ref"
%%% End: 
