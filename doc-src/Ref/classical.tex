%% $Id$
\chapter{The Classical Reasoner}\label{chap:classical}
\index{classical reasoner|(}
\newcommand\ainfer[2]{\begin{array}{r@{\,}l}#2\\ \hline#1\end{array}}

Although Isabelle is generic, many users will be working in some
extension of classical first-order logic.  Isabelle's set theory~{\tt
  ZF} is built upon theory~\texttt{FOL}, while {\HOL}
conceptually contains first-order logic as a fragment.
Theorem-proving in predicate logic is undecidable, but many
researchers have developed strategies to assist in this task.

Isabelle's classical reasoner is an \ML{} functor that accepts certain
information about a logic and delivers a suite of automatic tactics.  Each
tactic takes a collection of rules and executes a simple, non-clausal proof
procedure.  They are slow and simplistic compared with resolution theorem
provers, but they can save considerable time and effort.  They can prove
theorems such as Pelletier's~\cite{pelletier86} problems~40 and~41 in
seconds:
\[ (\exists y. \forall x. J(y,x) \bimp \neg J(x,x))  
   \imp  \neg (\forall x. \exists y. \forall z. J(z,y) \bimp \neg J(z,x)) \]
\[ (\forall z. \exists y. \forall x. F(x,y) \bimp F(x,z) \conj \neg F(x,x))
   \imp \neg (\exists z. \forall x. F(x,z))  
\]
%
The tactics are generic.  They are not restricted to first-order logic, and
have been heavily used in the development of Isabelle's set theory.  Few
interactive proof assistants provide this much automation.  The tactics can
be traced, and their components can be called directly; in this manner,
any proof can be viewed interactively.

The simplest way to apply the classical reasoner (to subgoal~$i$) is to type
\begin{ttbox}
by (Blast_tac \(i\));
\end{ttbox}
This command quickly proves most simple formulas of the predicate calculus or
set theory.  To attempt to prove \emph{all} subgoals using a combination of
rewriting and classical reasoning, try
\begin{ttbox}
by Auto_tac;
\end{ttbox}
To do all obvious logical steps, even if they do not prove the
subgoal, type one of the following:
\begin{ttbox}
by (Clarify_tac \(i\));        \emph{\textrm{applies to one subgoal}}
by Safe_tac;               \emph{\textrm{applies to all subgoals}}
\end{ttbox}
You need to know how the classical reasoner works in order to use it
effectively.  There are many tactics to choose from, including {\tt
  Fast_tac} and \texttt{Best_tac}.

We shall first discuss the underlying principles, then present the
classical reasoner.  Finally, we shall see how to instantiate it for
new logics.  The logics \FOL, \ZF, {\HOL} and {\HOLCF} have it already
installed.


\section{The sequent calculus}
\index{sequent calculus}
Isabelle supports natural deduction, which is easy to use for interactive
proof.  But natural deduction does not easily lend itself to automation,
and has a bias towards intuitionism.  For certain proofs in classical
logic, it can not be called natural.  The {\bf sequent calculus}, a
generalization of natural deduction, is easier to automate.

A {\bf sequent} has the form $\Gamma\turn\Delta$, where $\Gamma$
and~$\Delta$ are sets of formulae.%
\footnote{For first-order logic, sequents can equivalently be made from
  lists or multisets of formulae.} The sequent
\[ P@1,\ldots,P@m\turn Q@1,\ldots,Q@n \]
is {\bf valid} if $P@1\conj\ldots\conj P@m$ implies $Q@1\disj\ldots\disj
Q@n$.  Thus $P@1,\ldots,P@m$ represent assumptions, each of which is true,
while $Q@1,\ldots,Q@n$ represent alternative goals.  A sequent is {\bf
basic} if its left and right sides have a common formula, as in $P,Q\turn
Q,R$; basic sequents are trivially valid.

Sequent rules are classified as {\bf right} or {\bf left}, indicating which
side of the $\turn$~symbol they operate on.  Rules that operate on the
right side are analogous to natural deduction's introduction rules, and
left rules are analogous to elimination rules.  
Recall the natural deduction rules for
  first-order logic, 
\iflabelundefined{fol-fig}{from {\it Introduction to Isabelle}}%
                          {Fig.\ts\ref{fol-fig}}.
The sequent calculus analogue of~$({\imp}I)$ is the rule
$$
\ainfer{\Gamma &\turn \Delta, P\imp Q}{P,\Gamma &\turn \Delta,Q}
\eqno({\imp}R)
$$
This breaks down some implication on the right side of a sequent; $\Gamma$
and $\Delta$ stand for the sets of formulae that are unaffected by the
inference.  The analogue of the pair~$({\disj}I1)$ and~$({\disj}I2)$ is the
single rule 
$$
\ainfer{\Gamma &\turn \Delta, P\disj Q}{\Gamma &\turn \Delta,P,Q}
\eqno({\disj}R)
$$
This breaks down some disjunction on the right side, replacing it by both
disjuncts.  Thus, the sequent calculus is a kind of multiple-conclusion logic.

To illustrate the use of multiple formulae on the right, let us prove
the classical theorem $(P\imp Q)\disj(Q\imp P)$.  Working backwards, we
reduce this formula to a basic sequent:
\[ \infer[(\disj)R]{\turn(P\imp Q)\disj(Q\imp P)}
   {\infer[(\imp)R]{\turn(P\imp Q), (Q\imp P)\;}
    {\infer[(\imp)R]{P \turn Q, (Q\imp P)\qquad}
                    {P, Q \turn Q, P\qquad\qquad}}}
\]
This example is typical of the sequent calculus: start with the desired
theorem and apply rules backwards in a fairly arbitrary manner.  This yields a
surprisingly effective proof procedure.  Quantifiers add few complications,
since Isabelle handles parameters and schematic variables.  See Chapter~10
of {\em ML for the Working Programmer}~\cite{paulson91} for further
discussion.


\section{Simulating sequents by natural deduction}
Isabelle can represent sequents directly, as in the object-logic~\texttt{LK}\@.
But natural deduction is easier to work with, and most object-logics employ
it.  Fortunately, we can simulate the sequent $P@1,\ldots,P@m\turn
Q@1,\ldots,Q@n$ by the Isabelle formula
\[ \List{P@1;\ldots;P@m; \neg Q@2;\ldots; \neg Q@n}\Imp Q@1, \]
where the order of the assumptions and the choice of~$Q@1$ are arbitrary.
Elim-resolution plays a key role in simulating sequent proofs.

We can easily handle reasoning on the left.
As discussed in
\iflabelundefined{destruct}{{\it Introduction to Isabelle}}{\S\ref{destruct}}, 
elim-resolution with the rules $(\disj E)$, $(\bot E)$ and $(\exists E)$
achieves a similar effect as the corresponding sequent rules.  For the
other connectives, we use sequent-style elimination rules instead of
destruction rules such as $({\conj}E1,2)$ and $(\forall E)$.  But note that
the rule $(\neg L)$ has no effect under our representation of sequents!
$$
\ainfer{\neg P,\Gamma &\turn \Delta}{\Gamma &\turn \Delta,P}\eqno({\neg}L)
$$
What about reasoning on the right?  Introduction rules can only affect the
formula in the conclusion, namely~$Q@1$.  The other right-side formulae are
represented as negated assumptions, $\neg Q@2$, \ldots,~$\neg Q@n$.  
\index{assumptions!negated}
In order to operate on one of these, it must first be exchanged with~$Q@1$.
Elim-resolution with the {\bf swap} rule has this effect:
$$ \List{\neg P; \; \neg R\Imp P} \Imp R   \eqno(swap)  $$
To ensure that swaps occur only when necessary, each introduction rule is
converted into a swapped form: it is resolved with the second premise
of~$(swap)$.  The swapped form of~$({\conj}I)$, which might be
called~$({\neg\conj}E)$, is
\[ \List{\neg(P\conj Q); \; \neg R\Imp P; \; \neg R\Imp Q} \Imp R. \]
Similarly, the swapped form of~$({\imp}I)$ is
\[ \List{\neg(P\imp Q); \; \List{\neg R;P}\Imp Q} \Imp R  \]
Swapped introduction rules are applied using elim-resolution, which deletes
the negated formula.  Our representation of sequents also requires the use
of ordinary introduction rules.  If we had no regard for readability, we
could treat the right side more uniformly by representing sequents as
\[ \List{P@1;\ldots;P@m; \neg Q@1;\ldots; \neg Q@n}\Imp \bot. \]


\section{Extra rules for the sequent calculus}
As mentioned, destruction rules such as $({\conj}E1,2)$ and $(\forall E)$
must be replaced by sequent-style elimination rules.  In addition, we need
rules to embody the classical equivalence between $P\imp Q$ and $\neg P\disj
Q$.  The introduction rules~$({\disj}I1,2)$ are replaced by a rule that
simulates $({\disj}R)$:
\[ (\neg Q\Imp P) \Imp P\disj Q \]
The destruction rule $({\imp}E)$ is replaced by
\[ \List{P\imp Q;\; \neg P\Imp R;\; Q\Imp R} \Imp R. \]
Quantifier replication also requires special rules.  In classical logic,
$\exists x{.}P$ is equivalent to $\neg\forall x{.}\neg P$; the rules
$(\exists R)$ and $(\forall L)$ are dual:
\[ \ainfer{\Gamma &\turn \Delta, \exists x{.}P}
          {\Gamma &\turn \Delta, \exists x{.}P, P[t/x]} \; (\exists R)
   \qquad
   \ainfer{\forall x{.}P, \Gamma &\turn \Delta}
          {P[t/x], \forall x{.}P, \Gamma &\turn \Delta} \; (\forall L)
\]
Thus both kinds of quantifier may be replicated.  Theorems requiring
multiple uses of a universal formula are easy to invent; consider 
\[ (\forall x.P(x)\imp P(f(x))) \conj P(a) \imp P(f^n(a)), \]
for any~$n>1$.  Natural examples of the multiple use of an existential
formula are rare; a standard one is $\exists x.\forall y. P(x)\imp P(y)$.

Forgoing quantifier replication loses completeness, but gains decidability,
since the search space becomes finite.  Many useful theorems can be proved
without replication, and the search generally delivers its verdict in a
reasonable time.  To adopt this approach, represent the sequent rules
$(\exists R)$, $(\exists L)$ and $(\forall R)$ by $(\exists I)$, $(\exists
E)$ and $(\forall I)$, respectively, and put $(\forall E)$ into elimination
form:
$$ \List{\forall x{.}P(x); P(t)\Imp Q} \Imp Q    \eqno(\forall E@2) $$
Elim-resolution with this rule will delete the universal formula after a
single use.  To replicate universal quantifiers, replace the rule by
$$
\List{\forall x{.}P(x);\; \List{P(t); \forall x{.}P(x)}\Imp Q} \Imp Q.
\eqno(\forall E@3)
$$
To replicate existential quantifiers, replace $(\exists I)$ by
\[ \List{\neg(\exists x{.}P(x)) \Imp P(t)} \Imp \exists x{.}P(x). \]
All introduction rules mentioned above are also useful in swapped form.

Replication makes the search space infinite; we must apply the rules with
care.  The classical reasoner distinguishes between safe and unsafe
rules, applying the latter only when there is no alternative.  Depth-first
search may well go down a blind alley; best-first search is better behaved
in an infinite search space.  However, quantifier replication is too
expensive to prove any but the simplest theorems.


\section{Classical rule sets}
\index{classical sets}
Each automatic tactic takes a {\bf classical set} --- a collection of
rules, classified as introduction or elimination and as {\bf safe} or {\bf
unsafe}.  In general, safe rules can be attempted blindly, while unsafe
rules must be used with care.  A safe rule must never reduce a provable
goal to an unprovable set of subgoals.  

The rule~$({\disj}I1)$ is unsafe because it reduces $P\disj Q$ to~$P$.  Any
rule is unsafe whose premises contain new unknowns.  The elimination
rule~$(\forall E@2)$ is unsafe, since it is applied via elim-resolution,
which discards the assumption $\forall x{.}P(x)$ and replaces it by the
weaker assumption~$P(\Var{t})$.  The rule $({\exists}I)$ is unsafe for
similar reasons.  The rule~$(\forall E@3)$ is unsafe in a different sense:
since it keeps the assumption $\forall x{.}P(x)$, it is prone to looping.
In classical first-order logic, all rules are safe except those mentioned
above.

The safe/unsafe distinction is vague, and may be regarded merely as a way
of giving some rules priority over others.  One could argue that
$({\disj}E)$ is unsafe, because repeated application of it could generate
exponentially many subgoals.  Induction rules are unsafe because inductive
proofs are difficult to set up automatically.  Any inference is unsafe that
instantiates an unknown in the proof state --- thus \ttindex{match_tac}
must be used, rather than \ttindex{resolve_tac}.  Even proof by assumption
is unsafe if it instantiates unknowns shared with other subgoals --- thus
\ttindex{eq_assume_tac} must be used, rather than \ttindex{assume_tac}.

\subsection{Adding rules to classical sets}
Classical rule sets belong to the abstract type \mltydx{claset}, which
supports the following operations (provided the classical reasoner is
installed!):
\begin{ttbox} 
empty_cs    : claset
print_cs    : claset -> unit
addSIs      : claset * thm list -> claset                 \hfill{\bf infix 4}
addSEs      : claset * thm list -> claset                 \hfill{\bf infix 4}
addSDs      : claset * thm list -> claset                 \hfill{\bf infix 4}
addIs       : claset * thm list -> claset                 \hfill{\bf infix 4}
addEs       : claset * thm list -> claset                 \hfill{\bf infix 4}
addDs       : claset * thm list -> claset                 \hfill{\bf infix 4}
delrules    : claset * thm list -> claset                 \hfill{\bf infix 4}
\end{ttbox}
The add operations ignore any rule already present in the claset with the same
classification (such as Safe Introduction).  They print a warning if the rule
has already been added with some other classification, but add the rule
anyway.  Calling \texttt{delrules} deletes all occurrences of a rule from the
claset, but see the warning below concerning destruction rules.
\begin{ttdescription}
\item[\ttindexbold{empty_cs}] is the empty classical set.

\item[\ttindexbold{print_cs} $cs$] prints the rules of~$cs$.

\item[$cs$ addSIs $rules$] \indexbold{*addSIs}
adds safe introduction~$rules$ to~$cs$.

\item[$cs$ addSEs $rules$] \indexbold{*addSEs}
adds safe elimination~$rules$ to~$cs$.

\item[$cs$ addSDs $rules$] \indexbold{*addSDs}
adds safe destruction~$rules$ to~$cs$.

\item[$cs$ addIs $rules$] \indexbold{*addIs}
adds unsafe introduction~$rules$ to~$cs$.

\item[$cs$ addEs $rules$] \indexbold{*addEs}
adds unsafe elimination~$rules$ to~$cs$.

\item[$cs$ addDs $rules$] \indexbold{*addDs}
adds unsafe destruction~$rules$ to~$cs$.

\item[$cs$ delrules $rules$] \indexbold{*delrules}
deletes~$rules$ from~$cs$.  It prints a warning for those rules that are not
in~$cs$.
\end{ttdescription}

\begin{warn}
  If you added $rule$ using \texttt{addSDs} or \texttt{addDs}, then you must delete
  it as follows:
\begin{ttbox}
\(cs\) delrules [make_elim \(rule\)]
\end{ttbox}
\par\noindent
This is necessary because the operators \texttt{addSDs} and \texttt{addDs} convert
the destruction rules to elimination rules by applying \ttindex{make_elim},
and then insert them using \texttt{addSEs} and \texttt{addEs}, respectively.
\end{warn}

Introduction rules are those that can be applied using ordinary resolution.
The classical set automatically generates their swapped forms, which will
be applied using elim-resolution.  Elimination rules are applied using
elim-resolution.  In a classical set, rules are sorted by the number of new
subgoals they will yield; rules that generate the fewest subgoals will be
tried first (see \S\ref{biresolve_tac}).


\subsection{Modifying the search step}
For a given classical set, the proof strategy is simple.  Perform as many safe
inferences as possible; or else, apply certain safe rules, allowing
instantiation of unknowns; or else, apply an unsafe rule.  The tactics also
eliminate assumptions of the form $x=t$ by substitution if they have been set
up to do so (see \texttt{hyp_subst_tacs} in~\S\ref{sec:classical-setup} below).
They may perform a form of Modus Ponens: if there are assumptions $P\imp Q$
and~$P$, then replace $P\imp Q$ by~$Q$.

The classical reasoning tactics --- except \texttt{blast_tac}! --- allow
you to modify this basic proof strategy by applying two arbitrary {\bf
  wrapper tacticals} to it.  This affects each step of the search.
Usually they are the identity tacticals, but they could apply another
tactic before or after the step tactic.  The first one, which is
considered to be safe, affects \ttindex{safe_step_tac} and all the
tactics that call it.  The the second one, which may be unsafe, affects
\ttindex{step_tac}, \ttindex{slow_step_tac} and the tactics that call
them.

\begin{ttbox} 
addss        : claset * simpset -> claset                 \hfill{\bf infix 4}
addSbefore   : claset *  (int -> tactic)  -> claset       \hfill{\bf infix 4}
addSaltern   : claset *  (int -> tactic)  -> claset       \hfill{\bf infix 4}
setSWrapper  : claset * ((int -> tactic) -> 
                         (int -> tactic)) -> claset       \hfill{\bf infix 4}
compSWrapper : claset * ((int -> tactic) -> 
                         (int -> tactic)) -> claset       \hfill{\bf infix 4}
addbefore    : claset *  (int -> tactic)  -> claset       \hfill{\bf infix 4}
addaltern    : claset *  (int -> tactic)  -> claset       \hfill{\bf infix 4}
setWrapper   : claset * ((int -> tactic) -> 
                         (int -> tactic)) -> claset       \hfill{\bf infix 4}
compWrapper  : claset * ((int -> tactic) -> 
                         (int -> tactic)) -> claset       \hfill{\bf infix 4}
\end{ttbox}
%
\index{simplification!from classical reasoner} The wrapper tacticals
underly the operator addss, which combines each search step by
simplification.  Strictly speaking, \texttt{addss} is not part of the
classical reasoner.  It should be defined (using \texttt{addSaltern
  (CHANGED o (safe_asm_more_full_simp_tac ss)}) when the simplifier is
installed.

\begin{ttdescription}
\item[$cs$ addss $ss$] \indexbold{*addss}
adds the simpset~$ss$ to the classical set.  The assumptions and goal will be
simplified, in a safe way, after the safe steps of the search.

\item[$cs$ addSbefore $tac$] \indexbold{*addSbefore}
changes the safe wrapper tactical to apply the given tactic {\em before}
each safe step of the search.

\item[$cs$ addSaltern $tac$] \indexbold{*addSaltern}
changes the safe wrapper tactical to apply the given tactic when a safe step 
of the search would fail.

\item[$cs$ setSWrapper $tactical$] \indexbold{*setSWrapper}
specifies a new safe wrapper tactical.  

\item[$cs$ compSWrapper $tactical$] \indexbold{*compSWrapper}
composes the $tactical$ with the existing safe wrapper tactical, 
to combine their effects. 

\item[$cs$ addbefore $tac$] \indexbold{*addbefore}
changes the (unsafe) wrapper tactical to apply the given tactic, which should
be safe, {\em before} each step of the search.

\item[$cs$ addaltern $tac$] \indexbold{*addaltern}
changes the (unsafe) wrapper tactical to apply the given tactic 
{\em alternatively} after each step of the search.

\item[$cs$ setWrapper $tactical$] \indexbold{*setWrapper}
specifies a new (unsafe) wrapper tactical.  

\item[$cs$ compWrapper $tactical$] \indexbold{*compWrapper}
composes the $tactical$ with the existing (unsafe) wrapper tactical, 
to combine their effects. 
\end{ttdescription}


\section{The classical tactics}
\index{classical reasoner!tactics} If installed, the classical module provides
powerful theorem-proving tactics.  Most of them have capitalized analogues
that use the default claset; see \S\ref{sec:current-claset}.

\subsection{Semi-automatic tactics}
\begin{ttbox} 
clarify_tac      : claset -> int -> tactic
clarify_step_tac : claset -> int -> tactic
\end{ttbox}
Use these when the automatic tactics fail.  They perform all the obvious
logical inferences that do not split the subgoal.  The result is a
simpler subgoal that can be tackled by other means, such as by
instantiating quantifiers yourself.
\begin{ttdescription}
\item[\ttindexbold{clarify_tac} $cs$ $i$] performs a series of safe steps on
subgoal~$i$, using \texttt{clarify_step_tac}.

\item[\ttindexbold{clarify_step_tac} $cs$ $i$] performs a safe step on
  subgoal~$i$.  No splitting step is applied; for example, the subgoal $A\conj
  B$ is left as a conjunction.  Proof by assumption, Modus Ponens, etc., may be
  performed provided they do not instantiate unknowns.  Assumptions of the
  form $x=t$ may be eliminated.  The user-supplied safe wrapper tactical is
  applied.
\end{ttdescription}


\subsection{The tableau prover}
The tactic \texttt{blast_tac} searches for a proof using a fast tableau prover,
coded directly in \ML.  It then reconstructs the proof using Isabelle
tactics.  It is faster and more powerful than the other classical
reasoning tactics, but has major limitations too.
\begin{itemize}
\item It does not use the wrapper tacticals described above, such as
  \ttindex{addss}.
\item It ignores types, which can cause problems in \HOL.  If it applies a rule
  whose types are inappropriate, then proof reconstruction will fail.
\item It does not perform higher-order unification, as needed by the rule {\tt
    rangeI} in {\HOL} and \texttt{RepFunI} in {\ZF}.  There are often
    alternatives to such rules, for example {\tt
    range_eqI} and \texttt{RepFun_eqI}.
\item The message {\small\tt Function Var's argument not a bound variable\ }
relates to the lack of higher-order unification.  Function variables
may only be applied to parameters of the subgoal.
\item Its proof strategy is more general than \texttt{fast_tac}'s but can be
  slower.  If \texttt{blast_tac} fails or seems to be running forever, try {\tt
  fast_tac} and the other tactics described below.
\end{itemize}
%
\begin{ttbox} 
blast_tac        : claset -> int -> tactic
Blast.depth_tac  : claset -> int -> int -> tactic
Blast.trace      : bool ref \hfill{\bf initially false}
\end{ttbox}
The two tactics differ on how they bound the number of unsafe steps used in a
proof.  While \texttt{blast_tac} starts with a bound of zero and increases it
successively to~20, \texttt{Blast.depth_tac} applies a user-supplied search bound.
\begin{ttdescription}
\item[\ttindexbold{blast_tac} $cs$ $i$] tries to prove
  subgoal~$i$ using iterative deepening to increase the search bound.
  
\item[\ttindexbold{Blast.depth_tac} $cs$ $lim$ $i$] tries
  to prove subgoal~$i$ using a search bound of $lim$.  Often a slow
  proof using \texttt{blast_tac} can be made much faster by supplying the
  successful search bound to this tactic instead.
  
\item[set \ttindexbold{Blast.trace};] \index{tracing!of classical prover}
  causes the tableau prover to print a trace of its search.  At each step it
  displays the formula currently being examined and reports whether the branch
  has been closed, extended or split.
\end{ttdescription}


\subsection{An automatic tactic}
\begin{ttbox} 
auto_tac      : claset * simpset -> tactic
auto          : unit -> unit
\end{ttbox}
The auto-tactic attempts to prove all subgoals using a combination of
simplification and classical reasoning.  It is intended for situations where
there are a lot of mostly trivial subgoals; it proves all the easy ones,
leaving the ones it cannot prove.  (Unfortunately, attempting to prove the
hard ones may take a long time.)  It must be supplied both a simpset and a
claset; therefore it is most easily called as \texttt{Auto_tac}, which uses
the default claset and simpset (see \S\ref{sec:current-claset} below).  For
interactive use, the shorthand \texttt{auto();} abbreviates 
\begin{ttbox}
by Auto_tac;
\end{ttbox}

\subsection{Other classical tactics}
\begin{ttbox} 
fast_tac      : claset -> int -> tactic
best_tac      : claset -> int -> tactic
slow_tac      : claset -> int -> tactic
slow_best_tac : claset -> int -> tactic
\end{ttbox}
These tactics attempt to prove a subgoal using sequent-style reasoning.
Unlike \texttt{blast_tac}, they construct proofs directly in Isabelle.  Their
effect is restricted (by \texttt{SELECT_GOAL}) to one subgoal; they either prove
this subgoal or fail.  The \texttt{slow_} versions conduct a broader
search.%
\footnote{They may, when backtracking from a failed proof attempt, undo even
  the step of proving a subgoal by assumption.}

The best-first tactics are guided by a heuristic function: typically, the
total size of the proof state.  This function is supplied in the functor call
that sets up the classical reasoner.
\begin{ttdescription}
\item[\ttindexbold{fast_tac} $cs$ $i$] applies \texttt{step_tac} using
depth-first search, to prove subgoal~$i$.

\item[\ttindexbold{best_tac} $cs$ $i$] applies \texttt{step_tac} using
best-first search, to prove subgoal~$i$.

\item[\ttindexbold{slow_tac} $cs$ $i$] applies \texttt{slow_step_tac} using
depth-first search, to prove subgoal~$i$.

\item[\ttindexbold{slow_best_tac} $cs$ $i$] applies \texttt{slow_step_tac} using
best-first search, to prove subgoal~$i$.
\end{ttdescription}


\subsection{Depth-limited automatic tactics}
\begin{ttbox} 
depth_tac  : claset -> int -> int -> tactic
deepen_tac : claset -> int -> int -> tactic
\end{ttbox}
These work by exhaustive search up to a specified depth.  Unsafe rules are
modified to preserve the formula they act on, so that it be used repeatedly.
They can prove more goals than \texttt{fast_tac} can but are much
slower, for example if the assumptions have many universal quantifiers.

The depth limits the number of unsafe steps.  If you can estimate the minimum
number of unsafe steps needed, supply this value as~$m$ to save time.
\begin{ttdescription}
\item[\ttindexbold{depth_tac} $cs$ $m$ $i$] 
tries to prove subgoal~$i$ by exhaustive search up to depth~$m$.

\item[\ttindexbold{deepen_tac} $cs$ $m$ $i$] 
tries to prove subgoal~$i$ by iterative deepening.  It calls \texttt{depth_tac}
repeatedly with increasing depths, starting with~$m$.
\end{ttdescription}


\subsection{Single-step tactics}
\begin{ttbox} 
safe_step_tac : claset -> int -> tactic
safe_tac      : claset        -> tactic
inst_step_tac : claset -> int -> tactic
step_tac      : claset -> int -> tactic
slow_step_tac : claset -> int -> tactic
\end{ttbox}
The automatic proof procedures call these tactics.  By calling them
yourself, you can execute these procedures one step at a time.
\begin{ttdescription}
\item[\ttindexbold{safe_step_tac} $cs$ $i$] performs a safe step on
  subgoal~$i$.  The safe wrapper tactical is applied to a tactic that may
  include proof by assumption or Modus Ponens (taking care not to instantiate
  unknowns), or substitution.

\item[\ttindexbold{safe_tac} $cs$] repeatedly performs safe steps on all 
subgoals.  It is deterministic, with at most one outcome.  

\item[\ttindexbold{inst_step_tac} $cs$ $i$] is like \texttt{safe_step_tac},
but allows unknowns to be instantiated.

\item[\ttindexbold{step_tac} $cs$ $i$] is the basic step of the proof
  procedure.  The (unsafe) wrapper tactical is applied to a tactic that tries
 \texttt{safe_tac}, \texttt{inst_step_tac}, or applies an unsafe rule from~$cs$.

\item[\ttindexbold{slow_step_tac}] 
  resembles \texttt{step_tac}, but allows backtracking between using safe
  rules with instantiation (\texttt{inst_step_tac}) and using unsafe rules.
  The resulting search space is larger.
\end{ttdescription}

\subsection{The current claset}\label{sec:current-claset}
Some logics (\FOL, {\HOL} and \ZF) support the concept of a current
claset\index{claset!current}.  This is a default set of classical rules.  The
underlying idea is quite similar to that of a current simpset described in
\S\ref{sec:simp-for-dummies}; please read that section, including its
warnings.  Just like simpsets, clasets can be associated with theories.  The
tactics
\begin{ttbox}
Blast_tac        : int -> tactic
Auto_tac         :        tactic
Fast_tac         : int -> tactic
Best_tac         : int -> tactic
Deepen_tac       : int -> int -> tactic
Clarify_tac      : int -> tactic
Clarify_step_tac : int -> tactic
Safe_tac         :        tactic
Safe_step_tac    : int -> tactic
Step_tac         : int -> tactic
\end{ttbox}
\indexbold{*Blast_tac}\indexbold{*Auto_tac}
\indexbold{*Best_tac}\indexbold{*Fast_tac}%
\indexbold{*Deepen_tac}
\indexbold{*Clarify_tac}\indexbold{*Clarify_step_tac}
\indexbold{*Safe_tac}\indexbold{*Safe_step_tac}
\indexbold{*Step_tac}
make use of the current claset.  For example, \texttt{Blast_tac} is defined as 
\begin{ttbox}
fun Blast_tac i st = blast_tac (!claset) i st;
\end{ttbox}
and gets the current claset, \ttindex{!claset}, only after it is applied to a
proof state.  The functions
\begin{ttbox}
AddSIs, AddSEs, AddSDs, AddIs, AddEs, AddDs: thm list -> unit
\end{ttbox}
\indexbold{*AddSIs} \indexbold{*AddSEs} \indexbold{*AddSDs}
\indexbold{*AddIs} \indexbold{*AddEs} \indexbold{*AddDs}
are used to add rules to the current claset.  They work exactly like their
lower case counterparts, such as \texttt{addSIs}.  Calling
\begin{ttbox}
Delrules : thm list -> unit
\end{ttbox}
deletes rules from the current claset. 

\subsection{Other useful tactics}
\index{tactics!for contradiction}
\index{tactics!for Modus Ponens}
\begin{ttbox} 
contr_tac    :             int -> tactic
mp_tac       :             int -> tactic
eq_mp_tac    :             int -> tactic
swap_res_tac : thm list -> int -> tactic
\end{ttbox}
These can be used in the body of a specialized search.
\begin{ttdescription}
\item[\ttindexbold{contr_tac} {\it i}]\index{assumptions!contradictory}
  solves subgoal~$i$ by detecting a contradiction among two assumptions of
  the form $P$ and~$\neg P$, or fail.  It may instantiate unknowns.  The
  tactic can produce multiple outcomes, enumerating all possible
  contradictions.

\item[\ttindexbold{mp_tac} {\it i}] 
is like \texttt{contr_tac}, but also attempts to perform Modus Ponens in
subgoal~$i$.  If there are assumptions $P\imp Q$ and~$P$, then it replaces
$P\imp Q$ by~$Q$.  It may instantiate unknowns.  It fails if it can do
nothing.

\item[\ttindexbold{eq_mp_tac} {\it i}] 
is like \texttt{mp_tac} {\it i}, but may not instantiate unknowns --- thus, it
is safe.

\item[\ttindexbold{swap_res_tac} {\it thms} {\it i}] refines subgoal~$i$ of
the proof state using {\it thms}, which should be a list of introduction
rules.  First, it attempts to prove the goal using \texttt{assume_tac} or
\texttt{contr_tac}.  It then attempts to apply each rule in turn, attempting
resolution and also elim-resolution with the swapped form.
\end{ttdescription}

\subsection{Creating swapped rules}
\begin{ttbox} 
swapify   : thm list -> thm list
joinrules : thm list * thm list -> (bool * thm) list
\end{ttbox}
\begin{ttdescription}
\item[\ttindexbold{swapify} {\it thms}] returns a list consisting of the
swapped versions of~{\it thms}, regarded as introduction rules.

\item[\ttindexbold{joinrules} ({\it intrs}, {\it elims})]
joins introduction rules, their swapped versions, and elimination rules for
use with \ttindex{biresolve_tac}.  Each rule is paired with~\texttt{false}
(indicating ordinary resolution) or~\texttt{true} (indicating
elim-resolution).
\end{ttdescription}


\section{Setting up the classical reasoner}\label{sec:classical-setup}
\index{classical reasoner!setting up}
Isabelle's classical object-logics, including \texttt{FOL} and \texttt{HOL}, have
the classical reasoner already set up.  When defining a new classical logic,
you should set up the reasoner yourself.  It consists of the \ML{} functor
\ttindex{ClassicalFun}, which takes the argument
signature \texttt{
                  CLASSICAL_DATA}:
\begin{ttbox} 
signature CLASSICAL_DATA =
  sig
  val mp             : thm
  val not_elim       : thm
  val swap           : thm
  val sizef          : thm -> int
  val hyp_subst_tacs : (int -> tactic) list
  end;
\end{ttbox}
Thus, the functor requires the following items:
\begin{ttdescription}
\item[\tdxbold{mp}] should be the Modus Ponens rule
$\List{\Var{P}\imp\Var{Q};\; \Var{P}} \Imp \Var{Q}$.

\item[\tdxbold{not_elim}] should be the contradiction rule
$\List{\neg\Var{P};\; \Var{P}} \Imp \Var{R}$.

\item[\tdxbold{swap}] should be the swap rule
$\List{\neg \Var{P}; \; \neg \Var{R}\Imp \Var{P}} \Imp \Var{R}$.

\item[\ttindexbold{sizef}] is the heuristic function used for best-first
search.  It should estimate the size of the remaining subgoals.  A good
heuristic function is \ttindex{size_of_thm}, which measures the size of the
proof state.  Another size function might ignore certain subgoals (say,
those concerned with type checking).  A heuristic function might simply
count the subgoals.

\item[\ttindexbold{hyp_subst_tacs}] is a list of tactics for substitution in
the hypotheses, typically created by \ttindex{HypsubstFun} (see
Chapter~\ref{substitution}).  This list can, of course, be empty.  The
tactics are assumed to be safe!
\end{ttdescription}
The functor is not at all sensitive to the formalization of the
object-logic.  It does not even examine the rules, but merely applies
them according to its fixed strategy.  The functor resides in {\tt
  Provers/classical.ML} in the Isabelle sources.

\index{classical reasoner|)}
