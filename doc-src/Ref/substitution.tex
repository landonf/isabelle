%% $Id$
\chapter{Substitution Tactics} \label{substitution}
\index{tactics!substitution|(}\index{equality|(}

Replacing equals by equals is a basic form of reasoning.  Isabelle supports
several kinds of equality reasoning.  {\bf Substitution} means replacing
free occurrences of~$t$ by~$u$ in a subgoal.  This is easily done, given an
equality $t=u$, provided the logic possesses the appropriate rule.  The
tactic \texttt{hyp_subst_tac} performs substitution even in the assumptions.
But it works via object-level implication, and therefore must be specially
set up for each suitable object-logic.

Substitution should not be confused with object-level {\bf rewriting}.
Given equalities of the form $t=u$, rewriting replaces instances of~$t$ by
corresponding instances of~$u$, and continues until it reaches a normal
form.  Substitution handles `one-off' replacements by particular
equalities while rewriting handles general equations.
Chapter~\ref{chap:simplification} discusses Isabelle's rewriting tactics.


\section{Substitution rules}
\index{substitution!rules}\index{*subst theorem}
Many logics include a substitution rule of the form
$$
\List{\Var{a}=\Var{b}; \Var{P}(\Var{a})} \Imp 
\Var{P}(\Var{b})  \eqno(subst)
$$
In backward proof, this may seem difficult to use: the conclusion
$\Var{P}(\Var{b})$ admits far too many unifiers.  But, if the theorem {\tt
eqth} asserts $t=u$, then \hbox{\tt eqth RS subst} is the derived rule
\[ \Var{P}(t) \Imp \Var{P}(u). \]
Provided $u$ is not an unknown, resolution with this rule is
well-behaved.\footnote{Unifying $\Var{P}(u)$ with a formula~$Q$
expresses~$Q$ in terms of its dependence upon~$u$.  There are still $2^k$
unifiers, if $Q$ has $k$ occurrences of~$u$, but Isabelle ensures that
the first unifier includes all the occurrences.}  To replace $u$ by~$t$ in
subgoal~$i$, use
\begin{ttbox} 
resolve_tac [eqth RS subst] \(i\){\it.}
\end{ttbox}
To replace $t$ by~$u$ in
subgoal~$i$, use
\begin{ttbox} 
resolve_tac [eqth RS ssubst] \(i\){\it,}
\end{ttbox}
where \tdxbold{ssubst} is the `swapped' substitution rule
$$
\List{\Var{a}=\Var{b}; \Var{P}(\Var{b})} \Imp 
\Var{P}(\Var{a}).  \eqno(ssubst)
$$
If \tdx{sym} denotes the symmetry rule
\(\Var{a}=\Var{b}\Imp\Var{b}=\Var{a}\), then \texttt{ssubst} is just
\hbox{\tt sym RS subst}.  Many logics with equality include the rules {\tt
subst} and \texttt{ssubst}, as well as \texttt{refl}, \texttt{sym} and \texttt{trans}
(for the usual equality laws).  Examples include \texttt{FOL} and \texttt{HOL},
but not \texttt{CTT} (Constructive Type Theory).

Elim-resolution is well-behaved with assumptions of the form $t=u$.
To replace $u$ by~$t$ or $t$ by~$u$ in subgoal~$i$, use
\begin{ttbox} 
eresolve_tac [subst] \(i\)    {\rm or}    eresolve_tac [ssubst] \(i\){\it.}
\end{ttbox}

Logics \HOL, {\FOL} and {\ZF} define the tactic \ttindexbold{stac} by
\begin{ttbox} 
fun stac eqth = CHANGED o rtac (eqth RS ssubst);
\end{ttbox}
Now \texttt{stac~eqth} is like \texttt{resolve_tac [eqth RS ssubst]} but with the
valuable property of failing if the substitution has no effect.


\section{Substitution in the hypotheses}
\index{assumptions!substitution in}
Substitution rules, like other rules of natural deduction, do not affect
the assumptions.  This can be inconvenient.  Consider proving the subgoal
\[ \List{c=a; c=b} \Imp a=b. \]
Calling \texttt{eresolve_tac\ts[ssubst]\ts\(i\)} simply discards the
assumption~$c=a$, since $c$ does not occur in~$a=b$.  Of course, we can
work out a solution.  First apply \texttt{eresolve_tac\ts[subst]\ts\(i\)},
replacing~$a$ by~$c$:
\[ c=b \Imp c=b \]
Equality reasoning can be difficult, but this trivial proof requires
nothing more sophisticated than substitution in the assumptions.
Object-logics that include the rule~$(subst)$ provide tactics for this
purpose:
\begin{ttbox} 
hyp_subst_tac       : int -> tactic
bound_hyp_subst_tac : int -> tactic
\end{ttbox}
\begin{ttdescription}
\item[\ttindexbold{hyp_subst_tac} {\it i}] 
  selects an equality assumption of the form $t=u$ or $u=t$, where $t$ is a
  free variable or parameter.  Deleting this assumption, it replaces $t$
  by~$u$ throughout subgoal~$i$, including the other assumptions.

\item[\ttindexbold{bound_hyp_subst_tac} {\it i}] 
  is similar but only substitutes for parameters (bound variables).
  Uses for this are discussed below.
\end{ttdescription}
The term being replaced must be a free variable or parameter.  Substitution
for constants is usually unhelpful, since they may appear in other
theorems.  For instance, the best way to use the assumption $0=1$ is to
contradict a theorem that states $0\not=1$, rather than to replace 0 by~1
in the subgoal!

Substitution for unknowns, such as $\Var{x}=0$, is a bad idea: we might prove
the subgoal more easily by instantiating~$\Var{x}$ to~1.
Substitution for free variables is unhelpful if they appear in the
premises of a rule being derived: the substitution affects object-level
assumptions, not meta-level assumptions.  For instance, replacing~$a$
by~$b$ could make the premise~$P(a)$ worthless.  To avoid this problem, use
\texttt{bound_hyp_subst_tac}; alternatively, call \ttindex{cut_facts_tac} to
insert the atomic premises as object-level assumptions.


\section{Setting up the package} 
Many Isabelle object-logics, such as \texttt{FOL}, \texttt{HOL} and their
descendants, come with \texttt{hyp_subst_tac} already defined.  A few others,
such as \texttt{CTT}, do not support this tactic because they lack the
rule~$(subst)$.  When defining a new logic that includes a substitution
rule and implication, you must set up \texttt{hyp_subst_tac} yourself.  It
is packaged as the \ML{} functor \ttindex{HypsubstFun}, which takes the
argument signature~\texttt{HYPSUBST_DATA}:
\begin{ttbox} 
signature HYPSUBST_DATA =
  sig
  structure Simplifier : SIMPLIFIER
  val dest_Trueprop    : term -> term
  val dest_eq          : term -> term*term*typ
  val dest_imp         : term -> term*term
  val eq_reflection    : thm         (* a=b ==> a==b *)
  val imp_intr         : thm         (*(P ==> Q) ==> P-->Q *)
  val rev_mp           : thm         (* [| P;  P-->Q |] ==> Q *)
  val subst            : thm         (* [| a=b;  P(a) |] ==> P(b) *)
  val sym              : thm         (* a=b ==> b=a *)
  val thin_refl        : thm         (* [|x=x; P|] ==> P *)
  end;
\end{ttbox}
Thus, the functor requires the following items:
\begin{ttdescription}
\item[Simplifier] should be an instance of the simplifier (see
  Chapter~\ref{chap:simplification}).
  
\item[\ttindexbold{dest_Trueprop}] should coerce a meta-level formula to the
  corresponding object-level one.  Typically, it should return $P$ when
  applied to the term $\texttt{Trueprop}\,P$ (see example below).
  
\item[\ttindexbold{dest_eq}] should return the triple~$(t,u,T)$, where $T$ is
  the type of~$t$ and~$u$, when applied to the \ML{} term that
  represents~$t=u$.  For other terms, it should raise an exception.
  
\item[\ttindexbold{dest_imp}] should return the pair~$(P,Q)$ when applied to
  the \ML{} term that represents the implication $P\imp Q$.  For other terms,
  it should raise an exception.

\item[\tdxbold{eq_reflection}] is the theorem discussed
  in~\S\ref{sec:setting-up-simp}. 

\item[\tdxbold{imp_intr}] should be the implies introduction
rule $(\Var{P}\Imp\Var{Q})\Imp \Var{P}\imp\Var{Q}$.

\item[\tdxbold{rev_mp}] should be the `reversed' implies elimination
rule $\List{\Var{P};  \;\Var{P}\imp\Var{Q}} \Imp \Var{Q}$.

\item[\tdxbold{subst}] should be the substitution rule
$\List{\Var{a}=\Var{b};\; \Var{P}(\Var{a})} \Imp \Var{P}(\Var{b})$.

\item[\tdxbold{sym}] should be the symmetry rule
$\Var{a}=\Var{b}\Imp\Var{b}=\Var{a}$.

\item[\tdxbold{thin_refl}] should be the rule
$\List{\Var{a}=\Var{a};\; \Var{P}} \Imp \Var{P}$, which is used to erase
trivial equalities.
\end{ttdescription}
%
The functor resides in file \texttt{Provers/hypsubst.ML} in the Isabelle
distribution directory.  It is not sensitive to the precise formalization
of the object-logic.  It is not concerned with the names of the equality
and implication symbols, or the types of formula and terms.  

Coding the functions \texttt{dest_Trueprop}, \texttt{dest_eq} and
\texttt{dest_imp} requires knowledge of Isabelle's representation of terms.
For \texttt{FOL}, they are declared by
\begin{ttbox} 
fun dest_Trueprop (Const ("Trueprop", _) $ P) = P
  | dest_Trueprop t = raise TERM ("dest_Trueprop", [t]);

fun dest_eq (Const("op =",T)  $ t $ u) = (t, u, domain_type T)

fun dest_imp (Const("op -->",_) $ A $ B) = (A, B)
  | dest_imp  t = raise TERM ("dest_imp", [t]);
\end{ttbox}
Recall that \texttt{Trueprop} is the coercion from type~$o$ to type~$prop$,
while \hbox{\tt op =} is the internal name of the infix operator~\texttt{=}.
Function \ttindexbold{domain_type}, given the function type $S\To T$, returns
the type~$S$.  Pattern-matching expresses the function concisely, using
wildcards~({\tt_}) for the types.

The tactic \texttt{hyp_subst_tac} works as follows.  First, it identifies a
suitable equality assumption, possibly re-orienting it using~\texttt{sym}.
Then it moves other assumptions into the conclusion of the goal, by repeatedly
calling \texttt{etac~rev_mp}.  Then, it uses \texttt{asm_full_simp_tac} or
\texttt{ssubst} to substitute throughout the subgoal.  (If the equality
involves unknowns then it must use \texttt{ssubst}.)  Then, it deletes the
equality.  Finally, it moves the assumptions back to their original positions
by calling \hbox{\tt resolve_tac\ts[imp_intr]}.

\index{equality|)}\index{tactics!substitution|)}


%%% Local Variables: 
%%% mode: latex
%%% TeX-master: "ref"
%%% End: 
