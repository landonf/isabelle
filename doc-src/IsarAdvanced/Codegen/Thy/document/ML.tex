%
\begin{isabellebody}%
\def\isabellecontext{ML}%
%
\isadelimtheory
%
\endisadelimtheory
%
\isatagtheory
\isacommand{theory}\isamarkupfalse%
\ {\isachardoublequoteopen}ML{\isachardoublequoteclose}\isanewline
\isakeyword{imports}\ Setup\isanewline
\isakeyword{begin}%
\endisatagtheory
{\isafoldtheory}%
%
\isadelimtheory
%
\endisadelimtheory
%
\isamarkupsection{ML system interfaces \label{sec:ml}%
}
\isamarkuptrue%
%
\begin{isamarkuptext}%
Since the code generator framework not only aims to provide
  a nice Isar interface but also to form a base for
  code-generation-based applications, here a short
  description of the most important ML interfaces.%
\end{isamarkuptext}%
\isamarkuptrue%
%
\isamarkupsubsection{Executable theory content: \isa{Code}%
}
\isamarkuptrue%
%
\begin{isamarkuptext}%
This Pure module implements the core notions of
  executable content of a theory.%
\end{isamarkuptext}%
\isamarkuptrue%
%
\isamarkupsubsubsection{Managing executable content%
}
\isamarkuptrue%
%
\isadelimmlref
%
\endisadelimmlref
%
\isatagmlref
%
\begin{isamarkuptext}%
\begin{mldecls}
  \indexdef{}{ML}{Code.add\_eqn}\verb|Code.add_eqn: thm -> theory -> theory| \\
  \indexdef{}{ML}{Code.del\_eqn}\verb|Code.del_eqn: thm -> theory -> theory| \\
  \indexdef{}{ML}{Code.add\_eqnl}\verb|Code.add_eqnl: string * (thm * bool) list lazy -> theory -> theory| \\
  \indexdef{}{ML}{Code.map\_pre}\verb|Code.map_pre: (simpset -> simpset) -> theory -> theory| \\
  \indexdef{}{ML}{Code.map\_post}\verb|Code.map_post: (simpset -> simpset) -> theory -> theory| \\
  \indexdef{}{ML}{Code.add\_functrans}\verb|Code.add_functrans: string * (theory -> (thm * bool) list -> (thm * bool) list option)|\isasep\isanewline%
\verb|    -> theory -> theory| \\
  \indexdef{}{ML}{Code.del\_functrans}\verb|Code.del_functrans: string -> theory -> theory| \\
  \indexdef{}{ML}{Code.add\_datatype}\verb|Code.add_datatype: (string * typ) list -> theory -> theory| \\
  \indexdef{}{ML}{Code.get\_datatype}\verb|Code.get_datatype: theory -> string|\isasep\isanewline%
\verb|    -> (string * sort) list * (string * typ list) list| \\
  \indexdef{}{ML}{Code.get\_datatype\_of\_constr}\verb|Code.get_datatype_of_constr: theory -> string -> string option|
  \end{mldecls}

  \begin{description}

  \item \verb|Code.add_eqn|~\isa{thm}~\isa{thy} adds function
     theorem \isa{thm} to executable content.

  \item \verb|Code.del_eqn|~\isa{thm}~\isa{thy} removes function
     theorem \isa{thm} from executable content, if present.

  \item \verb|Code.add_eqnl|~\isa{{\isacharparenleft}const{\isacharcomma}\ lthms{\isacharparenright}}~\isa{thy} adds
     suspended code equations \isa{lthms} for constant
     \isa{const} to executable content.

  \item \verb|Code.map_pre|~\isa{f}~\isa{thy} changes
     the preprocessor simpset.

  \item \verb|Code.add_functrans|~\isa{{\isacharparenleft}name{\isacharcomma}\ f{\isacharparenright}}~\isa{thy} adds
     function transformer \isa{f} (named \isa{name}) to executable content;
     \isa{f} is a transformer of the code equations belonging
     to a certain function definition, depending on the
     current theory context.  Returning \isa{NONE} indicates that no
     transformation took place;  otherwise, the whole process will be iterated
     with the new code equations.

  \item \verb|Code.del_functrans|~\isa{name}~\isa{thy} removes
     function transformer named \isa{name} from executable content.

  \item \verb|Code.add_datatype|~\isa{cs}~\isa{thy} adds
     a datatype to executable content, with generation
     set \isa{cs}.

  \item \verb|Code.get_datatype_of_constr|~\isa{thy}~\isa{const}
     returns type constructor corresponding to
     constructor \isa{const}; returns \isa{NONE}
     if \isa{const} is no constructor.

  \end{description}%
\end{isamarkuptext}%
\isamarkuptrue%
%
\endisatagmlref
{\isafoldmlref}%
%
\isadelimmlref
%
\endisadelimmlref
%
\isamarkupsubsection{Auxiliary%
}
\isamarkuptrue%
%
\isadelimmlref
%
\endisadelimmlref
%
\isatagmlref
%
\begin{isamarkuptext}%
\begin{mldecls}
  \indexdef{}{ML}{Code\_Unit.read\_const}\verb|Code_Unit.read_const: theory -> string -> string| \\
  \indexdef{}{ML}{Code\_Unit.head\_eqn}\verb|Code_Unit.head_eqn: theory -> thm -> string * ((string * sort) list * typ)| \\
  \indexdef{}{ML}{Code\_Unit.rewrite\_eqn}\verb|Code_Unit.rewrite_eqn: simpset -> thm -> thm| \\
  \end{mldecls}

  \begin{description}

  \item \verb|Code_Unit.read_const|~\isa{thy}~\isa{s}
     reads a constant as a concrete term expression \isa{s}.

  \item \verb|Code_Unit.head_eqn|~\isa{thy}~\isa{thm}
     extracts the constant and its type from a code equation \isa{thm}.

  \item \verb|Code_Unit.rewrite_eqn|~\isa{ss}~\isa{thm}
     rewrites a code equation \isa{thm} with a simpset \isa{ss};
     only arguments and right hand side are rewritten,
     not the head of the code equation.

  \end{description}%
\end{isamarkuptext}%
\isamarkuptrue%
%
\endisatagmlref
{\isafoldmlref}%
%
\isadelimmlref
%
\endisadelimmlref
%
\isamarkupsubsection{Implementing code generator applications%
}
\isamarkuptrue%
%
\begin{isamarkuptext}%
Implementing code generator applications on top
  of the framework set out so far usually not only
  involves using those primitive interfaces
  but also storing code-dependent data and various
  other things.%
\end{isamarkuptext}%
\isamarkuptrue%
%
\isamarkupsubsubsection{Data depending on the theory's executable content%
}
\isamarkuptrue%
%
\begin{isamarkuptext}%
Due to incrementality of code generation, changes in the
  theory's executable content have to be propagated in a
  certain fashion.  Additionally, such changes may occur
  not only during theory extension but also during theory
  merge, which is a little bit nasty from an implementation
  point of view.  The framework provides a solution
  to this technical challenge by providing a functorial
  data slot \verb|CodeDataFun|; on instantiation
  of this functor, the following types and operations
  are required:

  \medskip
  \begin{tabular}{l}
  \isa{type\ T} \\
  \isa{val\ empty{\isacharcolon}\ T} \\
  \isa{val\ purge{\isacharcolon}\ theory\ {\isasymrightarrow}\ string\ list\ option\ {\isasymrightarrow}\ T\ {\isasymrightarrow}\ T}
  \end{tabular}

  \begin{description}

  \item \isa{T} the type of data to store.

  \item \isa{empty} initial (empty) data.

  \item \isa{purge}~\isa{thy}~\isa{consts} propagates changes in executable content;
    \isa{consts} indicates the kind
    of change: \verb|NONE| stands for a fundamental change
    which invalidates any existing code, \isa{SOME\ consts}
    hints that executable content for constants \isa{consts}
    has changed.

  \end{description}

  \noindent An instance of \verb|CodeDataFun| provides the following
  interface:

  \medskip
  \begin{tabular}{l}
  \isa{get{\isacharcolon}\ theory\ {\isasymrightarrow}\ T} \\
  \isa{change{\isacharcolon}\ theory\ {\isasymrightarrow}\ {\isacharparenleft}T\ {\isasymrightarrow}\ T{\isacharparenright}\ {\isasymrightarrow}\ T} \\
  \isa{change{\isacharunderscore}yield{\isacharcolon}\ theory\ {\isasymrightarrow}\ {\isacharparenleft}T\ {\isasymrightarrow}\ {\isacharprime}a\ {\isacharasterisk}\ T{\isacharparenright}\ {\isasymrightarrow}\ {\isacharprime}a\ {\isacharasterisk}\ T}
  \end{tabular}

  \begin{description}

  \item \isa{get} retrieval of the current data.

  \item \isa{change} update of current data (cached!)
    by giving a continuation.

  \item \isa{change{\isacharunderscore}yield} update with side result.

  \end{description}%
\end{isamarkuptext}%
\isamarkuptrue%
%
\begin{isamarkuptext}%
\bigskip

  \emph{Happy proving, happy hacking!}%
\end{isamarkuptext}%
\isamarkuptrue%
%
\isadelimtheory
%
\endisadelimtheory
%
\isatagtheory
\isacommand{end}\isamarkupfalse%
%
\endisatagtheory
{\isafoldtheory}%
%
\isadelimtheory
%
\endisadelimtheory
\isanewline
\end{isabellebody}%
%%% Local Variables:
%%% mode: latex
%%% TeX-master: "root"
%%% End:
