%
\begin{isabellebody}%
\def\isabellecontext{prelim}%
%
\isadelimtheory
\isanewline
\isanewline
\isanewline
%
\endisadelimtheory
%
\isatagtheory
\isacommand{theory}\isamarkupfalse%
\ prelim\ \isakeyword{imports}\ base\ \isakeyword{begin}%
\endisatagtheory
{\isafoldtheory}%
%
\isadelimtheory
%
\endisadelimtheory
%
\isamarkupchapter{Preliminaries%
}
\isamarkuptrue%
%
\isamarkupsection{Contexts \label{sec:context}%
}
\isamarkuptrue%
%
\begin{isamarkuptext}%
A logical context represents the background that is required for
  formulating statements and composing proofs.  It acts as a medium to
  produce formal content, depending on earlier material (declarations,
  results etc.).

  For example, derivations within the Isabelle/Pure logic can be
  described as a judgment \isa{{\isasymGamma}\ {\isasymturnstile}\isactrlsub {\isasymTheta}\ {\isasymphi}}, which means that a
  proposition \isa{{\isasymphi}} is derivable from hypotheses \isa{{\isasymGamma}}
  within the theory \isa{{\isasymTheta}}.  There are logical reasons for
  keeping \isa{{\isasymTheta}} and \isa{{\isasymGamma}} separate: theories can be
  liberal about supporting type constructors and schematic
  polymorphism of constants and axioms, while the inner calculus of
  \isa{{\isasymGamma}\ {\isasymturnstile}\ {\isasymphi}} is strictly limited to Simple Type Theory (with
  fixed type variables in the assumptions).

  \medskip Contexts and derivations are linked by the following key
  principles:

  \begin{itemize}

  \item Transfer: monotonicity of derivations admits results to be
  transferred into a \emph{larger} context, i.e.\ \isa{{\isasymGamma}\ {\isasymturnstile}\isactrlsub {\isasymTheta}\ {\isasymphi}} implies \isa{{\isasymGamma}{\isacharprime}\ {\isasymturnstile}\isactrlsub {\isasymTheta}\isactrlsub {\isacharprime}\ {\isasymphi}} for contexts \isa{{\isasymTheta}{\isacharprime}\ {\isasymsupseteq}\ {\isasymTheta}} and \isa{{\isasymGamma}{\isacharprime}\ {\isasymsupseteq}\ {\isasymGamma}}.

  \item Export: discharge of hypotheses admits results to be exported
  into a \emph{smaller} context, i.e.\ \isa{{\isasymGamma}{\isacharprime}\ {\isasymturnstile}\isactrlsub {\isasymTheta}\ {\isasymphi}}
  implies \isa{{\isasymGamma}\ {\isasymturnstile}\isactrlsub {\isasymTheta}\ {\isasymDelta}\ {\isasymLongrightarrow}\ {\isasymphi}} where \isa{{\isasymGamma}{\isacharprime}\ {\isasymsupseteq}\ {\isasymGamma}} and
  \isa{{\isasymDelta}\ {\isacharequal}\ {\isasymGamma}{\isacharprime}\ {\isacharminus}\ {\isasymGamma}}.  Note that \isa{{\isasymTheta}} remains unchanged here,
  only the \isa{{\isasymGamma}} part is affected.

  \end{itemize}

  \medskip By modeling the main characteristics of the primitive
  \isa{{\isasymTheta}} and \isa{{\isasymGamma}} above, and abstracting over any
  particular logical content, we arrive at the fundamental notions of
  \emph{theory context} and \emph{proof context} in Isabelle/Isar.
  These implement a certain policy to manage arbitrary \emph{context
  data}.  There is a strongly-typed mechanism to declare new kinds of
  data at compile time.

  The internal bootstrap process of Isabelle/Pure eventually reaches a
  stage where certain data slots provide the logical content of \isa{{\isasymTheta}} and \isa{{\isasymGamma}} sketched above, but this does not stop there!
  Various additional data slots support all kinds of mechanisms that
  are not necessarily part of the core logic.

  For example, there would be data for canonical introduction and
  elimination rules for arbitrary operators (depending on the
  object-logic and application), which enables users to perform
  standard proof steps implicitly (cf.\ the \isa{rule} method
  \cite{isabelle-isar-ref}).

  \medskip Thus Isabelle/Isar is able to bring forth more and more
  concepts successively.  In particular, an object-logic like
  Isabelle/HOL continues the Isabelle/Pure setup by adding specific
  components for automated reasoning (classical reasoner, tableau
  prover, structured induction etc.) and derived specification
  mechanisms (inductive predicates, recursive functions etc.).  All of
  this is ultimately based on the generic data management by theory
  and proof contexts introduced here.%
\end{isamarkuptext}%
\isamarkuptrue%
%
\isamarkupsubsection{Theory context \label{sec:context-theory}%
}
\isamarkuptrue%
%
\begin{isamarkuptext}%
\glossary{Theory}{FIXME}

  A \emph{theory} is a data container with explicit named and unique
  identifier.  Theories are related by a (nominal) sub-theory
  relation, which corresponds to the dependency graph of the original
  construction; each theory is derived from a certain sub-graph of
  ancestor theories.

  The \isa{merge} operation produces the least upper bound of two
  theories, which actually degenerates into absorption of one theory
  into the other (due to the nominal sub-theory relation).

  The \isa{begin} operation starts a new theory by importing
  several parent theories and entering a special \isa{draft} mode,
  which is sustained until the final \isa{end} operation.  A draft
  theory acts like a linear type, where updates invalidate earlier
  versions.  An invalidated draft is called ``stale''.

  The \isa{checkpoint} operation produces an intermediate stepping
  stone that will survive the next update: both the original and the
  changed theory remain valid and are related by the sub-theory
  relation.  Checkpointing essentially recovers purely functional
  theory values, at the expense of some extra internal bookkeeping.

  The \isa{copy} operation produces an auxiliary version that has
  the same data content, but is unrelated to the original: updates of
  the copy do not affect the original, neither does the sub-theory
  relation hold.

  \medskip The example in \figref{fig:ex-theory} below shows a theory
  graph derived from \isa{Pure}, with theory \isa{Length}
  importing \isa{Nat} and \isa{List}.  The body of \isa{Length} consists of a sequence of updates, working mostly on
  drafts.  Intermediate checkpoints may occur as well, due to the
  history mechanism provided by the Isar top-level, cf.\
  \secref{sec:isar-toplevel}.

  \begin{figure}[htb]
  \begin{center}
  \begin{tabular}{rcccl}
        &            & \isa{Pure} \\
        &            & \isa{{\isasymdown}} \\
        &            & \isa{FOL} \\
        & $\swarrow$ &              & $\searrow$ & \\
  \isa{Nat} &    &              &            & \isa{List} \\
        & $\searrow$ &              & $\swarrow$ \\
        &            & \isa{Length} \\
        &            & \multicolumn{3}{l}{~~\mbox{\isa{\isakeyword{imports}}}} \\
        &            & \multicolumn{3}{l}{~~\mbox{\isa{\isakeyword{begin}}}} \\
        &            & $\vdots$~~ \\
        &            & \isa{{\isasymbullet}}~~ \\
        &            & $\vdots$~~ \\
        &            & \isa{{\isasymbullet}}~~ \\
        &            & $\vdots$~~ \\
        &            & \multicolumn{3}{l}{~~\mbox{\isa{\isacommand{end}}}} \\
  \end{tabular}
  \caption{A theory definition depending on ancestors}\label{fig:ex-theory}
  \end{center}
  \end{figure}

  \medskip There is a separate notion of \emph{theory reference} for
  maintaining a live link to an evolving theory context: updates on
  drafts are propagated automatically.  Dynamic updating stops after
  an explicit \isa{end} only.

  Derived entities may store a theory reference in order to indicate
  the context they belong to.  This implicitly assumes monotonic
  reasoning, because the referenced context may become larger without
  further notice.%
\end{isamarkuptext}%
\isamarkuptrue%
%
\isadelimmlref
%
\endisadelimmlref
%
\isatagmlref
%
\begin{isamarkuptext}%
\begin{mldecls}
  \indexmltype{theory}\verb|type theory| \\
  \indexml{Theory.subthy}\verb|Theory.subthy: theory * theory -> bool| \\
  \indexml{Theory.merge}\verb|Theory.merge: theory * theory -> theory| \\
  \indexml{Theory.checkpoint}\verb|Theory.checkpoint: theory -> theory| \\
  \indexml{Theory.copy}\verb|Theory.copy: theory -> theory| \\
  \end{mldecls}
  \begin{mldecls}
  \indexmltype{theory\_ref}\verb|type theory_ref| \\
  \indexml{Theory.deref}\verb|Theory.deref: theory_ref -> theory| \\
  \indexml{Theory.check\_thy}\verb|Theory.check_thy: theory -> theory_ref| \\
  \end{mldecls}

  \begin{description}

  \item \verb|theory| represents theory contexts.  This is
  essentially a linear type!  Most operations destroy the original
  version, which then becomes ``stale''.

  \item \verb|Theory.subthy|~\isa{{\isacharparenleft}thy\isactrlsub {\isadigit{1}}{\isacharcomma}\ thy\isactrlsub {\isadigit{2}}{\isacharparenright}}
  compares theories according to the inherent graph structure of the
  construction.  This sub-theory relation is a nominal approximation
  of inclusion (\isa{{\isasymsubseteq}}) of the corresponding content.

  \item \verb|Theory.merge|~\isa{{\isacharparenleft}thy\isactrlsub {\isadigit{1}}{\isacharcomma}\ thy\isactrlsub {\isadigit{2}}{\isacharparenright}}
  absorbs one theory into the other.  This fails for unrelated
  theories!

  \item \verb|Theory.checkpoint|~\isa{thy} produces a safe
  stepping stone in the linear development of \isa{thy}.  The next
  update will result in two related, valid theories.

  \item \verb|Theory.copy|~\isa{thy} produces a variant of \isa{thy} that holds a copy of the same data.  The result is not
  related to the original; the original is unchanched.

  \item \verb|theory_ref| represents a sliding reference to an
  always valid theory; updates on the original are propagated
  automatically.

  \item \verb|Theory.deref|~\isa{thy{\isacharunderscore}ref} turns a \verb|theory_ref| into an \verb|theory| value.  As the referenced
  theory evolves monotonically over time, later invocations of \verb|Theory.deref| may refer to a larger context.

  \item \verb|Theory.check_thy|~\isa{thy} produces a \verb|theory_ref| from a valid \verb|theory| value.

  \end{description}%
\end{isamarkuptext}%
\isamarkuptrue%
%
\endisatagmlref
{\isafoldmlref}%
%
\isadelimmlref
%
\endisadelimmlref
%
\isamarkupsubsection{Proof context \label{sec:context-proof}%
}
\isamarkuptrue%
%
\begin{isamarkuptext}%
\glossary{Proof context}{The static context of a structured proof,
  acts like a local ``theory'' of the current portion of Isar proof
  text, generalizes the idea of local hypotheses \isa{{\isasymGamma}} in
  judgments \isa{{\isasymGamma}\ {\isasymturnstile}\ {\isasymphi}} of natural deduction calculi.  There is a
  generic notion of introducing and discharging hypotheses.
  Arbritrary auxiliary context data may be adjoined.}

  A proof context is a container for pure data with a back-reference
  to the theory it belongs to.  The \isa{init} operation creates a
  proof context from a given theory.  Modifications to draft theories
  are propagated to the proof context as usual, but there is also an
  explicit \isa{transfer} operation to force resynchronization
  with more substantial updates to the underlying theory.  The actual
  context data does not require any special bookkeeping, thanks to the
  lack of destructive features.

  Entities derived in a proof context need to record inherent logical
  requirements explicitly, since there is no separate context
  identification as for theories.  For example, hypotheses used in
  primitive derivations (cf.\ \secref{sec:thms}) are recorded
  separately within the sequent \isa{{\isasymGamma}\ {\isasymturnstile}\ {\isasymphi}}, just to make double
  sure.  Results could still leak into an alien proof context do to
  programming errors, but Isabelle/Isar includes some extra validity
  checks in critical positions, notably at the end of a sub-proof.

  Proof contexts may be manipulated arbitrarily, although the common
  discipline is to follow block structure as a mental model: a given
  context is extended consecutively, and results are exported back
  into the original context.  Note that the Isar proof states model
  block-structured reasoning explicitly, using a stack of proof
  contexts internally, cf.\ \secref{sec:isar-proof-state}.%
\end{isamarkuptext}%
\isamarkuptrue%
%
\isadelimmlref
%
\endisadelimmlref
%
\isatagmlref
%
\begin{isamarkuptext}%
\begin{mldecls}
  \indexmltype{Proof.context}\verb|type Proof.context| \\
  \indexml{ProofContext.init}\verb|ProofContext.init: theory -> Proof.context| \\
  \indexml{ProofContext.theory\_of}\verb|ProofContext.theory_of: Proof.context -> theory| \\
  \indexml{ProofContext.transfer}\verb|ProofContext.transfer: theory -> Proof.context -> Proof.context| \\
  \end{mldecls}

  \begin{description}

  \item \verb|Proof.context| represents proof contexts.  Elements
  of this type are essentially pure values, with a sliding reference
  to the background theory.

  \item \verb|ProofContext.init|~\isa{thy} produces a proof context
  derived from \isa{thy}, initializing all data.

  \item \verb|ProofContext.theory_of|~\isa{ctxt} selects the
  background theory from \isa{ctxt}, dereferencing its internal
  \verb|theory_ref|.

  \item \verb|ProofContext.transfer|~\isa{thy\ ctxt} promotes the
  background theory of \isa{ctxt} to the super theory \isa{thy}.

  \end{description}%
\end{isamarkuptext}%
\isamarkuptrue%
%
\endisatagmlref
{\isafoldmlref}%
%
\isadelimmlref
%
\endisadelimmlref
%
\isamarkupsubsection{Generic contexts \label{sec:generic-context}%
}
\isamarkuptrue%
%
\begin{isamarkuptext}%
A generic context is the disjoint sum of either a theory or proof
  context.  Occasionally, this enables uniform treatment of generic
  context data, typically extra-logical information.  Operations on
  generic contexts include the usual injections, partial selections,
  and combinators for lifting operations on either component of the
  disjoint sum.

  Moreover, there are total operations \isa{theory{\isacharunderscore}of} and \isa{proof{\isacharunderscore}of} to convert a generic context into either kind: a theory
  can always be selected from the sum, while a proof context might
  have to be constructed by an ad-hoc \isa{init} operation.%
\end{isamarkuptext}%
\isamarkuptrue%
%
\isadelimmlref
%
\endisadelimmlref
%
\isatagmlref
%
\begin{isamarkuptext}%
\begin{mldecls}
  \indexmltype{Context.generic}\verb|type Context.generic| \\
  \indexml{Context.theory\_of}\verb|Context.theory_of: Context.generic -> theory| \\
  \indexml{Context.proof\_of}\verb|Context.proof_of: Context.generic -> Proof.context| \\
  \end{mldecls}

  \begin{description}

  \item \verb|Context.generic| is the direct sum of \verb|theory| and \verb|Proof.context|, with the datatype
  constructors \verb|Context.Theory| and \verb|Context.Proof|.

  \item \verb|Context.theory_of|~\isa{context} always produces a
  theory from the generic \isa{context}, using \verb|ProofContext.theory_of| as required.

  \item \verb|Context.proof_of|~\isa{context} always produces a
  proof context from the generic \isa{context}, using \verb|ProofContext.init| as required (note that this re-initializes the
  context data with each invocation).

  \end{description}%
\end{isamarkuptext}%
\isamarkuptrue%
%
\endisatagmlref
{\isafoldmlref}%
%
\isadelimmlref
%
\endisadelimmlref
%
\isamarkupsubsection{Context data \label{sec:context-data}%
}
\isamarkuptrue%
%
\begin{isamarkuptext}%
The main purpose of theory and proof contexts is to manage arbitrary
  data.  New data types can be declared incrementally at compile time.
  There are separate declaration mechanisms for any of the three kinds
  of contexts: theory, proof, generic.

  \paragraph{Theory data} may refer to destructive entities, which are
  maintained in direct correspondence to the linear evolution of
  theory values, including explicit copies.\footnote{Most existing
  instances of destructive theory data are merely historical relics
  (e.g.\ the destructive theorem storage, and destructive hints for
  the Simplifier and Classical rules).}  A theory data declaration
  needs to implement the following SML signature:

  \medskip
  \begin{tabular}{ll}
  \isa{{\isasymtype}\ T} & representing type \\
  \isa{{\isasymval}\ empty{\isacharcolon}\ T} & empty default value \\
  \isa{{\isasymval}\ copy{\isacharcolon}\ T\ {\isasymrightarrow}\ T} & refresh impure data \\
  \isa{{\isasymval}\ extend{\isacharcolon}\ T\ {\isasymrightarrow}\ T} & re-initialize on import \\
  \isa{{\isasymval}\ merge{\isacharcolon}\ T\ {\isasymtimes}\ T\ {\isasymrightarrow}\ T} & join on import \\
  \end{tabular}
  \medskip

  \noindent The \isa{empty} value acts as initial default for
  \emph{any} theory that does not declare actual data content; \isa{copy} maintains persistent integrity for impure data, it is just
  the identity for pure values; \isa{extend} is acts like a
  unitary version of \isa{merge}, both operations should also
  include the functionality of \isa{copy} for impure data.

  \paragraph{Proof context data} is purely functional.  A declaration
  needs to implement the following SML signature:

  \medskip
  \begin{tabular}{ll}
  \isa{{\isasymtype}\ T} & representing type \\
  \isa{{\isasymval}\ init{\isacharcolon}\ theory\ {\isasymrightarrow}\ T} & produce initial value \\
  \end{tabular}
  \medskip

  \noindent The \isa{init} operation is supposed to produce a pure
  value from the given background theory.

  \paragraph{Generic data} provides a hybrid interface for both theory
  and proof data.  The declaration is essentially the same as for
  (pure) theory data, without \isa{copy}.  The \isa{init}
  operation for proof contexts merely selects the current data value
  from the background theory.

  \bigskip A data declaration of type \isa{T} results in the
  following interface:

  \medskip
  \begin{tabular}{ll}
  \isa{init{\isacharcolon}\ theory\ {\isasymrightarrow}\ theory} \\
  \isa{get{\isacharcolon}\ context\ {\isasymrightarrow}\ T} \\
  \isa{put{\isacharcolon}\ T\ {\isasymrightarrow}\ context\ {\isasymrightarrow}\ context} \\
  \isa{map{\isacharcolon}\ {\isacharparenleft}T\ {\isasymrightarrow}\ T{\isacharparenright}\ {\isasymrightarrow}\ context\ {\isasymrightarrow}\ context} \\
  \end{tabular}
  \medskip

  \noindent Here \isa{init} is only applicable to impure theory
  data to install a fresh copy persistently (destructive update on
  uninitialized has no permanent effect).  The other operations provide
  access for the particular kind of context (theory, proof, or generic
  context).  Note that this is a safe interface: there is no other way
  to access the corresponding data slot of a context.  By keeping
  these operations private, a component may maintain abstract values
  authentically, without other components interfering.%
\end{isamarkuptext}%
\isamarkuptrue%
%
\isadelimmlref
%
\endisadelimmlref
%
\isatagmlref
%
\begin{isamarkuptext}%
\begin{mldecls}
  \indexmlfunctor{TheoryDataFun}\verb|functor TheoryDataFun| \\
  \indexmlfunctor{ProofDataFun}\verb|functor ProofDataFun| \\
  \indexmlfunctor{GenericDataFun}\verb|functor GenericDataFun| \\
  \end{mldecls}

  \begin{description}

  \item \verb|TheoryDataFun|\isa{{\isacharparenleft}spec{\isacharparenright}} declares data for
  type \verb|theory| according to the specification provided as
  argument structure.  The resulting structure provides data init and
  access operations as described above.

  \item \verb|ProofDataFun|\isa{{\isacharparenleft}spec{\isacharparenright}} is analogous to
  \verb|TheoryDataFun| for type \verb|Proof.context|.

  \item \verb|GenericDataFun|\isa{{\isacharparenleft}spec{\isacharparenright}} is analogous to
  \verb|TheoryDataFun| for type \verb|Context.generic|.

  \end{description}%
\end{isamarkuptext}%
\isamarkuptrue%
%
\endisatagmlref
{\isafoldmlref}%
%
\isadelimmlref
%
\endisadelimmlref
%
\isamarkupsection{Names%
}
\isamarkuptrue%
%
\begin{isamarkuptext}%
In principle, a name is just a string, but there are various
  convention for encoding additional structure.  For example, ``\isa{Foo{\isachardot}bar{\isachardot}baz}'' is considered as a qualified name consisting of
  three basic name components.  The individual constituents of a name
  may have further substructure, e.g.\ the string
  ``\verb,\,\verb,<alpha>,'' encodes as a single symbol.%
\end{isamarkuptext}%
\isamarkuptrue%
%
\isamarkupsubsection{Strings of symbols%
}
\isamarkuptrue%
%
\begin{isamarkuptext}%
\glossary{Symbol}{The smallest unit of text in Isabelle, subsumes
  plain ASCII characters as well as an infinite collection of named
  symbols (for greek, math etc.).}

  A \emph{symbol} constitutes the smallest textual unit in Isabelle
  --- raw characters are normally not encountered at all.  Isabelle
  strings consist of a sequence of symbols, represented as a packed
  string or a list of strings.  Each symbol is in itself a small
  string, which has either one of the following forms:

  \begin{enumerate}

  \item a single ASCII character ``\isa{c}'', for example
  ``\verb,a,'',

  \item a regular symbol ``\verb,\,\verb,<,\isa{ident}\verb,>,'',
  for example ``\verb,\,\verb,<alpha>,'',

  \item a control symbol ``\verb,\,\verb,<^,\isa{ident}\verb,>,'',
  for example ``\verb,\,\verb,<^bold>,'',

  \item a raw symbol ``\verb,\,\verb,<^raw:,\isa{text}\verb,>,''
  where \isa{text} constists of printable characters excluding
  ``\verb,.,'' and ``\verb,>,'', for example
  ``\verb,\,\verb,<^raw:$\sum_{i = 1}^n$>,'',

  \item a numbered raw control symbol ``\verb,\,\verb,<^raw,\isa{n}\verb,>, where \isa{n} consists of digits, for example
  ``\verb,\,\verb,<^raw42>,''.

  \end{enumerate}

  \noindent The \isa{ident} syntax for symbol names is \isa{letter\ {\isacharparenleft}letter\ {\isacharbar}\ digit{\isacharparenright}\isactrlsup {\isacharasterisk}}, where \isa{letter\ {\isacharequal}\ A{\isachardot}{\isachardot}Za{\isachardot}{\isachardot}z} and \isa{digit\ {\isacharequal}\ {\isadigit{0}}{\isachardot}{\isachardot}{\isadigit{9}}}.  There are infinitely many
  regular symbols and control symbols, but a fixed collection of
  standard symbols is treated specifically.  For example,
  ``\verb,\,\verb,<alpha>,'' is classified as a letter, which means it
  may occur within regular Isabelle identifiers.

  Since the character set underlying Isabelle symbols is 7-bit ASCII
  and 8-bit characters are passed through transparently, Isabelle may
  also process Unicode/UCS data in UTF-8 encoding.  Unicode provides
  its own collection of mathematical symbols, but there is no built-in
  link to the standard collection of Isabelle.

  \medskip Output of Isabelle symbols depends on the print mode
  (\secref{FIXME}).  For example, the standard {\LaTeX} setup of the
  Isabelle document preparation system would present
  ``\verb,\,\verb,<alpha>,'' as \isa{{\isasymalpha}}, and
  ``\verb,\,\verb,<^bold>,\verb,\,\verb,<alpha>,'' as \isa{\isactrlbold {\isasymalpha}}.%
\end{isamarkuptext}%
\isamarkuptrue%
%
\isadelimmlref
%
\endisadelimmlref
%
\isatagmlref
%
\begin{isamarkuptext}%
\begin{mldecls}
  \indexmltype{Symbol.symbol}\verb|type Symbol.symbol| \\
  \indexml{Symbol.explode}\verb|Symbol.explode: string -> Symbol.symbol list| \\
  \indexml{Symbol.is\_letter}\verb|Symbol.is_letter: Symbol.symbol -> bool| \\
  \indexml{Symbol.is\_digit}\verb|Symbol.is_digit: Symbol.symbol -> bool| \\
  \indexml{Symbol.is\_quasi}\verb|Symbol.is_quasi: Symbol.symbol -> bool| \\
  \indexml{Symbol.is\_blank}\verb|Symbol.is_blank: Symbol.symbol -> bool| \\
  \end{mldecls}
  \begin{mldecls}
  \indexmltype{Symbol.sym}\verb|type Symbol.sym| \\
  \indexml{Symbol.decode}\verb|Symbol.decode: Symbol.symbol -> Symbol.sym| \\
  \end{mldecls}

  \begin{description}

  \item \verb|Symbol.symbol| represents individual Isabelle
  symbols; this is an alias for \verb|string|.

  \item \verb|Symbol.explode|~\isa{str} produces a symbol list
  from the packed form.  This function supercedes \verb|String.explode| for virtually all purposes of manipulating text in
  Isabelle!

  \item \verb|Symbol.is_letter|, \verb|Symbol.is_digit|, \verb|Symbol.is_quasi|, \verb|Symbol.is_blank| classify standard
  symbols according to fixed syntactic conventions of Isabelle, cf.\
  \cite{isabelle-isar-ref}.

  \item \verb|Symbol.sym| is a concrete datatype that represents
  the different kinds of symbols explicitly, with constructors \verb|Symbol.Char|, \verb|Symbol.Sym|, \verb|Symbol.Ctrl|, \verb|Symbol.Raw|.

  \item \verb|Symbol.decode| converts the string representation of a
  symbol into the datatype version.

  \end{description}%
\end{isamarkuptext}%
\isamarkuptrue%
%
\endisatagmlref
{\isafoldmlref}%
%
\isadelimmlref
%
\endisadelimmlref
%
\isamarkupsubsection{Basic names \label{sec:basic-names}%
}
\isamarkuptrue%
%
\begin{isamarkuptext}%
A \emph{basic name} essentially consists of a single Isabelle
  identifier.  There are conventions to mark separate classes of basic
  names, by attaching a suffix of underscores (\isa{{\isacharunderscore}}): one
  underscore means \emph{internal name}, two underscores means
  \emph{Skolem name}, three underscores means \emph{internal Skolem
  name}.

  For example, the basic name \isa{foo} has the internal version
  \isa{foo{\isacharunderscore}}, with Skolem versions \isa{foo{\isacharunderscore}{\isacharunderscore}} and \isa{foo{\isacharunderscore}{\isacharunderscore}{\isacharunderscore}}, respectively.

  These special versions provide copies of the basic name space, apart
  from anything that normally appears in the user text.  For example,
  system generated variables in Isar proof contexts are usually marked
  as internal, which prevents mysterious name references like \isa{xaa} to appear in the text.

  \medskip Manipulating binding scopes often requires on-the-fly
  renamings.  A \emph{name context} contains a collection of already
  used names.  The \isa{declare} operation adds names to the
  context.

  The \isa{invents} operation derives a number of fresh names from
  a given starting point.  For example, the first three names derived
  from \isa{a} are \isa{a}, \isa{b}, \isa{c}.

  The \isa{variants} operation produces fresh names by
  incrementing tentative names as base-26 numbers (with digits \isa{a{\isachardot}{\isachardot}z}) until all clashes are resolved.  For example, name \isa{foo} results in variants \isa{fooa}, \isa{foob}, \isa{fooc}, \dots, \isa{fooaa}, \isa{fooab} etc.; each renaming
  step picks the next unused variant from this sequence.%
\end{isamarkuptext}%
\isamarkuptrue%
%
\isadelimmlref
%
\endisadelimmlref
%
\isatagmlref
%
\begin{isamarkuptext}%
\begin{mldecls}
  \indexml{Name.internal}\verb|Name.internal: string -> string| \\
  \indexml{Name.skolem}\verb|Name.skolem: string -> string| \\
  \end{mldecls}
  \begin{mldecls}
  \indexmltype{Name.context}\verb|type Name.context| \\
  \indexml{Name.context}\verb|Name.context: Name.context| \\
  \indexml{Name.declare}\verb|Name.declare: string -> Name.context -> Name.context| \\
  \indexml{Name.invents}\verb|Name.invents: Name.context -> string -> int -> string list| \\
  \indexml{Name.variants}\verb|Name.variants: string list -> Name.context -> string list * Name.context| \\
  \end{mldecls}

  \begin{description}

  \item \verb|Name.internal|~\isa{name} produces an internal name
  by adding one underscore.

  \item \verb|Name.skolem|~\isa{name} produces a Skolem name by
  adding two underscores.

  \item \verb|Name.context| represents the context of already used
  names; the initial value is \verb|Name.context|.

  \item \verb|Name.declare|~\isa{name} enters a used name into the
  context.

  \item \verb|Name.invents|~\isa{context\ name\ n} produces \isa{n} fresh names derived from \isa{name}.

  \item \verb|Name.variants|~\isa{names\ context} produces fresh
  varians of \isa{names}; the result is entered into the context.

  \end{description}%
\end{isamarkuptext}%
\isamarkuptrue%
%
\endisatagmlref
{\isafoldmlref}%
%
\isadelimmlref
%
\endisadelimmlref
%
\isamarkupsubsection{Indexed names%
}
\isamarkuptrue%
%
\begin{isamarkuptext}%
An \emph{indexed name} (or \isa{indexname}) is a pair of a basic
  name and a natural number.  This representation allows efficient
  renaming by incrementing the second component only.  The canonical
  way to rename two collections of indexnames apart from each other is
  this: determine the maximum index \isa{maxidx} of the first
  collection, then increment all indexes of the second collection by
  \isa{maxidx\ {\isacharplus}\ {\isadigit{1}}}; the maximum index of an empty collection is
  \isa{{\isacharminus}{\isadigit{1}}}.

  Occasionally, basic names and indexed names are injected into the
  same pair type: the (improper) indexname \isa{{\isacharparenleft}x{\isacharcomma}\ {\isacharminus}{\isadigit{1}}{\isacharparenright}} is used
  to encode basic names.

  \medskip Isabelle syntax observes the following rules for
  representing an indexname \isa{{\isacharparenleft}x{\isacharcomma}\ i{\isacharparenright}} as a packed string:

  \begin{itemize}

  \item \isa{{\isacharquery}x} if \isa{x} does not end with a digit and \isa{i\ {\isacharequal}\ {\isadigit{0}}},

  \item \isa{{\isacharquery}xi} if \isa{x} does not end with a digit,

  \item \isa{{\isacharquery}x{\isachardot}i} otherwise.

  \end{itemize}

  Indexnames may acquire large index numbers over time.  Results are
  normalized towards \isa{{\isadigit{0}}} at certain checkpoints, notably at
  the end of a proof.  This works by producing variants of the
  corresponding basic name components.  For example, the collection
  \isa{{\isacharquery}x{\isadigit{1}}{\isacharcomma}\ {\isacharquery}x{\isadigit{7}}{\isacharcomma}\ {\isacharquery}x{\isadigit{4}}{\isadigit{2}}} becomes \isa{{\isacharquery}x{\isacharcomma}\ {\isacharquery}xa{\isacharcomma}\ {\isacharquery}xb}.%
\end{isamarkuptext}%
\isamarkuptrue%
%
\isadelimmlref
%
\endisadelimmlref
%
\isatagmlref
%
\begin{isamarkuptext}%
\begin{mldecls}
  \indexmltype{indexname}\verb|type indexname| \\
  \end{mldecls}

  \begin{description}

  \item \verb|indexname| represents indexed names.  This is an
  abbreviation for \verb|string * int|.  The second component is
  usually non-negative, except for situations where \isa{{\isacharparenleft}x{\isacharcomma}\ {\isacharminus}{\isadigit{1}}{\isacharparenright}}
  is used to embed basic names into this type.

  \end{description}%
\end{isamarkuptext}%
\isamarkuptrue%
%
\endisatagmlref
{\isafoldmlref}%
%
\isadelimmlref
%
\endisadelimmlref
%
\isamarkupsubsection{Qualified names and name spaces%
}
\isamarkuptrue%
%
\begin{isamarkuptext}%
A \emph{qualified name} consists of a non-empty sequence of basic
  name components.  The packed representation uses a dot as separator,
  as in ``\isa{A{\isachardot}b{\isachardot}c}''.  The last component is called \emph{base}
  name, the remaining prefix \emph{qualifier} (which may be empty).
  The idea of qualified names is to encode nested structures by
  recording the access paths as qualifiers.  For example, an item
  named ``\isa{A{\isachardot}b{\isachardot}c}'' may be understood as a local entity \isa{c}, within a local structure \isa{b}, within a global
  structure \isa{A}.  Typically, name space hierarchies consist of
  1--2 levels of qualification, but this need not be always so.

  The empty name is commonly used as an indication of unnamed
  entities, whenever this makes any sense.  The basic operations on
  qualified names are smart enough to pass through such improper names
  unchanged.

  \medskip A \isa{naming} policy tells how to turn a name
  specification into a fully qualified internal name (by the \isa{full} operation), and how fully qualified names may be accessed
  externally.  For example, the default naming policy is to prefix an
  implicit path: \isa{full\ x} produces \isa{path{\isachardot}x}, and the
  standard accesses for \isa{path{\isachardot}x} include both \isa{x} and
  \isa{path{\isachardot}x}.  Normally, the naming is implicit in the theory or
  proof context; there are separate versions of the corresponding.

  \medskip A \isa{name\ space} manages a collection of fully
  internalized names, together with a mapping between external names
  and internal names (in both directions).  The corresponding \isa{intern} and \isa{extern} operations are mostly used for
  parsing and printing only!  The \isa{declare} operation augments
  a name space according to the accesses determined by the naming
  policy.

  \medskip As a general principle, there is a separate name space for
  each kind of formal entity, e.g.\ logical constant, type
  constructor, type class, theorem.  It is usually clear from the
  occurrence in concrete syntax (or from the scope) which kind of
  entity a name refers to.  For example, the very same name \isa{c} may be used uniformly for a constant, type constructor, and
  type class.

  There are common schemes to name theorems systematically, according
  to the name of the main logical entity involved, e.g.\ \isa{c{\isachardot}intro} for a canonical theorem related to constant \isa{c}.
  This technique of mapping names from one space into another requires
  some care in order to avoid conflicts.  In particular, theorem names
  derived from a type constructor or type class are better suffixed in
  addition to the usual qualification, e.g.\ \isa{c{\isacharunderscore}type{\isachardot}intro}
  and \isa{c{\isacharunderscore}class{\isachardot}intro} for theorems related to type \isa{c}
  and class \isa{c}, respectively.%
\end{isamarkuptext}%
\isamarkuptrue%
%
\isadelimmlref
%
\endisadelimmlref
%
\isatagmlref
%
\begin{isamarkuptext}%
\begin{mldecls}
  \indexml{NameSpace.base}\verb|NameSpace.base: string -> string| \\
  \indexml{NameSpace.qualifier}\verb|NameSpace.qualifier: string -> string| \\
  \indexml{NameSpace.append}\verb|NameSpace.append: string -> string -> string| \\
  \indexml{NameSpace.implode}\verb|NameSpace.implode: string list -> string| \\
  \indexml{NameSpace.explode}\verb|NameSpace.explode: string -> string list| \\
  \end{mldecls}
  \begin{mldecls}
  \indexmltype{NameSpace.naming}\verb|type NameSpace.naming| \\
  \indexml{NameSpace.default\_naming}\verb|NameSpace.default_naming: NameSpace.naming| \\
  \indexml{NameSpace.add\_path}\verb|NameSpace.add_path: string -> NameSpace.naming -> NameSpace.naming| \\
  \indexml{NameSpace.full}\verb|NameSpace.full: NameSpace.naming -> string -> string| \\
  \end{mldecls}
  \begin{mldecls}
  \indexmltype{NameSpace.T}\verb|type NameSpace.T| \\
  \indexml{NameSpace.empty}\verb|NameSpace.empty: NameSpace.T| \\
  \indexml{NameSpace.merge}\verb|NameSpace.merge: NameSpace.T * NameSpace.T -> NameSpace.T| \\
  \indexml{NameSpace.declare}\verb|NameSpace.declare: NameSpace.naming -> string -> NameSpace.T -> NameSpace.T| \\
  \indexml{NameSpace.intern}\verb|NameSpace.intern: NameSpace.T -> string -> string| \\
  \indexml{NameSpace.extern}\verb|NameSpace.extern: NameSpace.T -> string -> string| \\
  \end{mldecls}

  \begin{description}

  \item \verb|NameSpace.base|~\isa{name} returns the base name of a
  qualified name.

  \item \verb|NameSpace.qualifier|~\isa{name} returns the qualifier
  of a qualified name.

  \item \verb|NameSpace.append|~\isa{name\isactrlisub {\isadigit{1}}\ name\isactrlisub {\isadigit{2}}}
  appends two qualified names.

  \item \verb|NameSpace.implode|~\isa{name} and \verb|NameSpace.explode|~\isa{names} convert between the packed string
  representation and the explicit list form of qualified names.

  \item \verb|NameSpace.naming| represents the abstract concept of
  a naming policy.

  \item \verb|NameSpace.default_naming| is the default naming policy.
  In a theory context, this is usually augmented by a path prefix
  consisting of the theory name.

  \item \verb|NameSpace.add_path|~\isa{path\ naming} augments the
  naming policy by extending its path component.

  \item \verb|NameSpace.full|\isa{naming\ name} turns a name
  specification (usually a basic name) into the fully qualified
  internal version, according to the given naming policy.

  \item \verb|NameSpace.T| represents name spaces.

  \item \verb|NameSpace.empty| and \verb|NameSpace.merge|~\isa{{\isacharparenleft}space\isactrlisub {\isadigit{1}}{\isacharcomma}\ space\isactrlisub {\isadigit{2}}{\isacharparenright}} are the canonical operations for
  maintaining name spaces according to theory data management
  (\secref{sec:context-data}).

  \item \verb|NameSpace.declare|~\isa{naming\ name\ space} enters a
  fully qualified name into the name space, with external accesses
  determined by the naming policy.

  \item \verb|NameSpace.intern|~\isa{space\ name} internalizes a
  (partially qualified) external name.

  This operation is mostly for parsing!  Note that fully qualified
  names stemming from declarations are produced via \verb|NameSpace.full| (or its derivatives for \verb|theory| and
  \verb|Proof.context|).

  \item \verb|NameSpace.extern|~\isa{space\ name} externalizes a
  (fully qualified) internal name.

  This operation is mostly for printing!  Note unqualified names are
  produced via \verb|NameSpace.base|.

  \end{description}%
\end{isamarkuptext}%
\isamarkuptrue%
%
\endisatagmlref
{\isafoldmlref}%
%
\isadelimmlref
%
\endisadelimmlref
%
\isadelimtheory
%
\endisadelimtheory
%
\isatagtheory
\isacommand{end}\isamarkupfalse%
%
\endisatagtheory
{\isafoldtheory}%
%
\isadelimtheory
%
\endisadelimtheory
\isanewline
\end{isabellebody}%
%%% Local Variables:
%%% mode: latex
%%% TeX-master: "root"
%%% End:
