\part{Foundations} 
The following sections discuss Isabelle's logical foundations in detail:
representing logical syntax in the typed $\lambda$-calculus; expressing
inference rules in Isabelle's meta-logic; combining rules by resolution.

If you wish to use Isabelle immediately, please turn to
page~\pageref{chap:getting}.  You can always read about foundations later,
either by returning to this point or by looking up particular items in the
index.

\begin{figure} 
\begin{eqnarray*}
  \neg P   & \hbox{abbreviates} & P\imp\bot \\
  P\bimp Q & \hbox{abbreviates} & (P\imp Q) \conj (Q\imp P)
\end{eqnarray*}
\vskip 4ex

\(\begin{array}{c@{\qquad\qquad}c}
  \infer[({\conj}I)]{P\conj Q}{P & Q}  &
  \infer[({\conj}E1)]{P}{P\conj Q} \qquad 
  \infer[({\conj}E2)]{Q}{P\conj Q} \\[4ex]

  \infer[({\disj}I1)]{P\disj Q}{P} \qquad 
  \infer[({\disj}I2)]{P\disj Q}{Q} &
  \infer[({\disj}E)]{R}{P\disj Q & \infer*{R}{[P]} & \infer*{R}{[Q]}}\\[4ex]

  \infer[({\imp}I)]{P\imp Q}{\infer*{Q}{[P]}} &
  \infer[({\imp}E)]{Q}{P\imp Q & P}  \\[4ex]

  &
  \infer[({\bot}E)]{P}{\bot}\\[4ex]

  \infer[({\forall}I)*]{\forall x.P}{P} &
  \infer[({\forall}E)]{P[t/x]}{\forall x.P} \\[3ex]

  \infer[({\exists}I)]{\exists x.P}{P[t/x]} &
  \infer[({\exists}E)*]{Q}{{\exists x.P} & \infer*{Q}{[P]} } \\[3ex]

  {t=t} \,(refl)   &  \vcenter{\infer[(subst)]{P[u/x]}{t=u & P[t/x]}} 
\end{array} \)

\bigskip\bigskip
*{\em Eigenvariable conditions\/}:

$\forall I$: provided $x$ is not free in the assumptions

$\exists E$: provided $x$ is not free in $Q$ or any assumption except $P$
\caption{Intuitionistic first-order logic} \label{fol-fig}
\end{figure}

\section{Formalizing logical syntax in Isabelle}\label{sec:logical-syntax}
\index{first-order logic}

Figure~\ref{fol-fig} presents intuitionistic first-order logic,
including equality.  Let us see how to formalize
this logic in Isabelle, illustrating the main features of Isabelle's
polymorphic meta-logic.

\index{lambda calc@$\lambda$-calculus} 
Isabelle represents syntax using the simply typed $\lambda$-calculus.  We
declare a type for each syntactic category of the logic.  We declare a
constant for each symbol of the logic, giving each $n$-place operation an
$n$-argument curried function type.  Most importantly,
$\lambda$-abstraction represents variable binding in quantifiers.

\index{types!syntax of}\index{types!function}\index{*fun type} 
\index{type constructors}
Isabelle has \ML-style polymorphic types such as~$(\alpha)list$, where
$list$ is a type constructor and $\alpha$ is a type variable; for example,
$(bool)list$ is the type of lists of booleans.  Function types have the
form $(\sigma,\tau)fun$ or $\sigma\To\tau$, where $\sigma$ and $\tau$ are
types.  Curried function types may be abbreviated:
\[  \sigma@1\To (\cdots \sigma@n\To \tau\cdots)  \quad \hbox{as} \quad
[\sigma@1, \ldots, \sigma@n] \To \tau \]
 
\index{terms!syntax of} The syntax for terms is summarised below.
Note that there are two versions of function application syntax
available in Isabelle: either $t\,u$, which is the usual form for
higher-order languages, or $t(u)$, trying to look more like
first-order.  The latter syntax is used throughout the manual.
\[ 
\index{lambda abs@$\lambda$-abstractions}\index{function applications}
\begin{array}{ll}
  t :: \tau   & \hbox{type constraint, on a term or bound variable} \\
  \lambda x.t   & \hbox{abstraction} \\
  \lambda x@1\ldots x@n.t
        & \hbox{curried abstraction, $\lambda x@1. \ldots \lambda x@n.t$} \\
  t(u)          & \hbox{application} \\
  t (u@1, \ldots, u@n) & \hbox{curried application, $t(u@1)\ldots(u@n)$} 
\end{array}
\]


\subsection{Simple types and constants}\index{types!simple|bold} 

The syntactic categories of our logic (Fig.\ts\ref{fol-fig}) are {\bf
  formulae} and {\bf terms}.  Formulae denote truth values, so (following
tradition) let us call their type~$o$.  To allow~0 and~$Suc(t)$ as terms,
let us declare a type~$nat$ of natural numbers.  Later, we shall see
how to admit terms of other types.

\index{constants}\index{*nat type}\index{*o type}
After declaring the types~$o$ and~$nat$, we may declare constants for the
symbols of our logic.  Since $\bot$ denotes a truth value (falsity) and 0
denotes a number, we put \begin{eqnarray*}
  \bot  & :: & o \\
  0     & :: & nat.
\end{eqnarray*}
If a symbol requires operands, the corresponding constant must have a
function type.  In our logic, the successor function
($Suc$) is from natural numbers to natural numbers, negation ($\neg$) is a
function from truth values to truth values, and the binary connectives are
curried functions taking two truth values as arguments: 
\begin{eqnarray*}
  Suc    & :: & nat\To nat  \\
  {\neg} & :: & o\To o      \\
  \conj,\disj,\imp,\bimp  & :: & [o,o]\To o 
\end{eqnarray*}
The binary connectives can be declared as infixes, with appropriate
precedences, so that we write $P\conj Q\disj R$ instead of
$\disj(\conj(P,Q), R)$.

Section~\ref{sec:defining-theories} below describes the syntax of Isabelle
theory files and illustrates it by extending our logic with mathematical
induction.


\subsection{Polymorphic types and constants} \label{polymorphic}
\index{types!polymorphic|bold}
\index{equality!polymorphic}
\index{constants!polymorphic}

Which type should we assign to the equality symbol?  If we tried
$[nat,nat]\To o$, then equality would be restricted to the natural
numbers; we should have to declare different equality symbols for each
type.  Isabelle's type system is polymorphic, so we could declare
\begin{eqnarray*}
  {=}  & :: & [\alpha,\alpha]\To o,
\end{eqnarray*}
where the type variable~$\alpha$ ranges over all types.
But this is also wrong.  The declaration is too polymorphic; $\alpha$
includes types like~$o$ and $nat\To nat$.  Thus, it admits
$\bot=\neg(\bot)$ and $Suc=Suc$ as formulae, which is acceptable in
higher-order logic but not in first-order logic.

Isabelle's {\bf type classes}\index{classes} control
polymorphism~\cite{nipkow-prehofer}.  Each type variable belongs to a
class, which denotes a set of types.  Classes are partially ordered by the
subclass relation, which is essentially the subset relation on the sets of
types.  They closely resemble the classes of the functional language
Haskell~\cite{haskell-tutorial,haskell-report}.

\index{*logic class}\index{*term class}
Isabelle provides the built-in class $logic$, which consists of the logical
types: the ones we want to reason about.  Let us declare a class $term$, to
consist of all legal types of terms in our logic.  The subclass structure
is now $term\le logic$.

\index{*nat type}
We put $nat$ in class $term$ by declaring $nat{::}term$.  We declare the
equality constant by
\begin{eqnarray*}
  {=}  & :: & [\alpha{::}term,\alpha]\To o 
\end{eqnarray*}
where $\alpha{::}term$ constrains the type variable~$\alpha$ to class
$term$.  Such type variables resemble Standard~\ML's equality type
variables.

We give~$o$ and function types the class $logic$ rather than~$term$, since
they are not legal types for terms.  We may introduce new types of class
$term$ --- for instance, type $string$ or $real$ --- at any time.  We can
even declare type constructors such as~$list$, and state that type
$(\tau)list$ belongs to class~$term$ provided $\tau$ does; equality
applies to lists of natural numbers but not to lists of formulae.  We may
summarize this paragraph by a set of {\bf arity declarations} for type
constructors:\index{arities!declaring}
\begin{eqnarray*}\index{*o type}\index{*fun type}
  o             & :: & logic \\
  fun           & :: & (logic,logic)logic \\
  nat, string, real     & :: & term \\
  list          & :: & (term)term
\end{eqnarray*}
(Recall that $fun$ is the type constructor for function types.)
In \rmindex{higher-order logic}, equality does apply to truth values and
functions;  this requires the arity declarations ${o::term}$
and ${fun::(term,term)term}$.  The class system can also handle
overloading.\index{overloading|bold} We could declare $arith$ to be the
subclass of $term$ consisting of the `arithmetic' types, such as~$nat$.
Then we could declare the operators
\begin{eqnarray*}
  {+},{-},{\times},{/}  & :: & [\alpha{::}arith,\alpha]\To \alpha
\end{eqnarray*}
If we declare new types $real$ and $complex$ of class $arith$, then we
in effect have three sets of operators:
\begin{eqnarray*}
  {+},{-},{\times},{/}  & :: & [nat,nat]\To nat \\
  {+},{-},{\times},{/}  & :: & [real,real]\To real \\
  {+},{-},{\times},{/}  & :: & [complex,complex]\To complex 
\end{eqnarray*}
Isabelle will regard these as distinct constants, each of which can be defined
separately.  We could even introduce the type $(\alpha)vector$ and declare
its arity as $(arith)arith$.  Then we could declare the constant
\begin{eqnarray*}
  {+}  & :: & [(\alpha)vector,(\alpha)vector]\To (\alpha)vector 
\end{eqnarray*}
and specify it in terms of ${+} :: [\alpha,\alpha]\To \alpha$.

A type variable may belong to any finite number of classes.  Suppose that
we had declared yet another class $ord \le term$, the class of all
`ordered' types, and a constant
\begin{eqnarray*}
  {\le}  & :: & [\alpha{::}ord,\alpha]\To o.
\end{eqnarray*}
In this context the variable $x$ in $x \le (x+x)$ will be assigned type
$\alpha{::}\{arith,ord\}$, which means $\alpha$ belongs to both $arith$ and
$ord$.  Semantically the set $\{arith,ord\}$ should be understood as the
intersection of the sets of types represented by $arith$ and $ord$.  Such
intersections of classes are called \bfindex{sorts}.  The empty
intersection of classes, $\{\}$, contains all types and is thus the {\bf
  universal sort}.

Even with overloading, each term has a unique, most general type.  For this
to be possible, the class and type declarations must satisfy certain
technical constraints; see 
\iflabelundefined{sec:ref-defining-theories}%
           {Sect.\ Defining Theories in the {\em Reference Manual}}%
           {\S\ref{sec:ref-defining-theories}}.


\subsection{Higher types and quantifiers}
\index{types!higher|bold}\index{quantifiers}
Quantifiers are regarded as operations upon functions.  Ignoring polymorphism
for the moment, consider the formula $\forall x. P(x)$, where $x$ ranges
over type~$nat$.  This is true if $P(x)$ is true for all~$x$.  Abstracting
$P(x)$ into a function, this is the same as saying that $\lambda x.P(x)$
returns true for all arguments.  Thus, the universal quantifier can be
represented by a constant
\begin{eqnarray*}
  \forall  & :: & (nat\To o) \To o,
\end{eqnarray*}
which is essentially an infinitary truth table.  The representation of $\forall
x. P(x)$ is $\forall(\lambda x. P(x))$.  

The existential quantifier is treated
in the same way.  Other binding operators are also easily handled; for
instance, the summation operator $\Sigma@{k=i}^j f(k)$ can be represented as
$\Sigma(i,j,\lambda k.f(k))$, where
\begin{eqnarray*}
  \Sigma  & :: & [nat,nat, nat\To nat] \To nat.
\end{eqnarray*}
Quantifiers may be polymorphic.  We may define $\forall$ and~$\exists$ over
all legal types of terms, not just the natural numbers, and
allow summations over all arithmetic types:
\begin{eqnarray*}
   \forall,\exists      & :: & (\alpha{::}term\To o) \To o \\
   \Sigma               & :: & [nat,nat, nat\To \alpha{::}arith] \To \alpha
\end{eqnarray*}
Observe that the index variables still have type $nat$, while the values
being summed may belong to any arithmetic type.


\section{Formalizing logical rules in Isabelle}
\index{meta-implication|bold}
\index{meta-quantifiers|bold}
\index{meta-equality|bold}

Object-logics are formalized by extending Isabelle's
meta-logic~\cite{paulson-found}, which is intuitionistic higher-order logic.
The meta-level connectives are {\bf implication}, the {\bf universal
  quantifier}, and {\bf equality}.
\begin{itemize}
  \item The implication \(\phi\Imp \psi\) means `\(\phi\) implies
\(\psi\)', and expresses logical {\bf entailment}.  

  \item The quantification \(\Forall x.\phi\) means `\(\phi\) is true for
all $x$', and expresses {\bf generality} in rules and axiom schemes. 

\item The equality \(a\equiv b\) means `$a$ equals $b$', for expressing
  {\bf definitions} (see~\S\ref{definitions}).\index{definitions}
  Equalities left over from the unification process, so called {\bf
    flex-flex constraints},\index{flex-flex constraints} are written $a\qeq
  b$.  The two equality symbols have the same logical meaning.

\end{itemize}
The syntax of the meta-logic is formalized in the same manner
as object-logics, using the simply typed $\lambda$-calculus.  Analogous to
type~$o$ above, there is a built-in type $prop$ of meta-level truth values.
Meta-level formulae will have this type.  Type $prop$ belongs to
class~$logic$; also, $\sigma\To\tau$ belongs to $logic$ provided $\sigma$
and $\tau$ do.  Here are the types of the built-in connectives:
\begin{eqnarray*}\index{*prop type}\index{*logic class}
  \Imp     & :: & [prop,prop]\To prop \\
  \Forall  & :: & (\alpha{::}logic\To prop) \To prop \\
  {\equiv} & :: & [\alpha{::}\{\},\alpha]\To prop \\
  \qeq & :: & [\alpha{::}\{\},\alpha]\To prop
\end{eqnarray*}
The polymorphism in $\Forall$ is restricted to class~$logic$ to exclude
certain types, those used just for parsing.  The type variable
$\alpha{::}\{\}$ ranges over the universal sort.

In our formalization of first-order logic, we declared a type~$o$ of
object-level truth values, rather than using~$prop$ for this purpose.  If
we declared the object-level connectives to have types such as
${\neg}::prop\To prop$, then these connectives would be applicable to
meta-level formulae.  Keeping $prop$ and $o$ as separate types maintains
the distinction between the meta-level and the object-level.  To formalize
the inference rules, we shall need to relate the two levels; accordingly,
we declare the constant
\index{*Trueprop constant}
\begin{eqnarray*}
  Trueprop & :: & o\To prop.
\end{eqnarray*}
We may regard $Trueprop$ as a meta-level predicate, reading $Trueprop(P)$ as
`$P$ is true at the object-level.'  Put another way, $Trueprop$ is a coercion
from $o$ to $prop$.


\subsection{Expressing propositional rules}
\index{rules!propositional}
We shall illustrate the use of the meta-logic by formalizing the rules of
Fig.\ts\ref{fol-fig}.  Each object-level rule is expressed as a meta-level
axiom. 

One of the simplest rules is $(\conj E1)$.  Making
everything explicit, its formalization in the meta-logic is
$$
\Forall P\;Q. Trueprop(P\conj Q) \Imp Trueprop(P).   \eqno(\conj E1)
$$
This may look formidable, but it has an obvious reading: for all object-level
truth values $P$ and~$Q$, if $P\conj Q$ is true then so is~$P$.  The
reading is correct because the meta-logic has simple models, where
types denote sets and $\Forall$ really means `for all.'

\index{*Trueprop constant}
Isabelle adopts notational conventions to ease the writing of rules.  We may
hide the occurrences of $Trueprop$ by making it an implicit coercion.
Outer universal quantifiers may be dropped.  Finally, the nested implication
\index{meta-implication}
\[  \phi@1\Imp(\cdots \phi@n\Imp\psi\cdots) \]
may be abbreviated as $\List{\phi@1; \ldots; \phi@n} \Imp \psi$, which
formalizes a rule of $n$~premises.

Using these conventions, the conjunction rules become the following axioms.
These fully specify the properties of~$\conj$:
$$ \List{P; Q} \Imp P\conj Q                 \eqno(\conj I) $$
$$ P\conj Q \Imp P  \qquad  P\conj Q \Imp Q  \eqno(\conj E1,2) $$

\noindent
Next, consider the disjunction rules.  The discharge of assumption in
$(\disj E)$ is expressed  using $\Imp$:
\index{assumptions!discharge of}%
$$ P \Imp P\disj Q  \qquad  Q \Imp P\disj Q  \eqno(\disj I1,2) $$
$$ \List{P\disj Q; P\Imp R; Q\Imp R} \Imp R  \eqno(\disj E) $$
%
To understand this treatment of assumptions in natural
deduction, look at implication.  The rule $({\imp}I)$ is the classic
example of natural deduction: to prove that $P\imp Q$ is true, assume $P$
is true and show that $Q$ must then be true.  More concisely, if $P$
implies $Q$ (at the meta-level), then $P\imp Q$ is true (at the
object-level).  Showing the coercion explicitly, this is formalized as
\[ (Trueprop(P)\Imp Trueprop(Q)) \Imp Trueprop(P\imp Q). \]
The rule $({\imp}E)$ is straightforward; hiding $Trueprop$, the axioms to
specify $\imp$ are 
$$ (P \Imp Q)  \Imp  P\imp Q   \eqno({\imp}I) $$
$$ \List{P\imp Q; P}  \Imp Q.  \eqno({\imp}E) $$

\noindent
Finally, the intuitionistic contradiction rule is formalized as the axiom
$$ \bot \Imp P.   \eqno(\bot E) $$

\begin{warn}
Earlier versions of Isabelle, and certain
papers~\cite{paulson-found,paulson700}, use $\List{P}$ to mean $Trueprop(P)$.
\end{warn}

\subsection{Quantifier rules and substitution}
\index{quantifiers}\index{rules!quantifier}\index{substitution|bold}
\index{variables!bound}\index{lambda abs@$\lambda$-abstractions}
\index{function applications}

Isabelle expresses variable binding using $\lambda$-abstraction; for instance,
$\forall x.P$ is formalized as $\forall(\lambda x.P)$.  Recall that $F(t)$
is Isabelle's syntax for application of the function~$F$ to the argument~$t$;
it is not a meta-notation for substitution.  On the other hand, a substitution
will take place if $F$ has the form $\lambda x.P$;  Isabelle transforms
$(\lambda x.P)(t)$ to~$P[t/x]$ by $\beta$-conversion.  Thus, we can express
inference rules that involve substitution for bound variables.

\index{parameters|bold}\index{eigenvariables|see{parameters}}
A logic may attach provisos to certain of its rules, especially quantifier
rules.  We cannot hope to formalize arbitrary provisos.  Fortunately, those
typical of quantifier rules always have the same form, namely `$x$ not free in
\ldots {\it (some set of formulae)},' where $x$ is a variable (called a {\bf
parameter} or {\bf eigenvariable}) in some premise.  Isabelle treats
provisos using~$\Forall$, its inbuilt notion of `for all'.
\index{meta-quantifiers}

The purpose of the proviso `$x$ not free in \ldots' is
to ensure that the premise may not make assumptions about the value of~$x$,
and therefore holds for all~$x$.  We formalize $(\forall I)$ by
\[ \left(\Forall x. Trueprop(P(x))\right) \Imp Trueprop(\forall x.P(x)). \]
This means, `if $P(x)$ is true for all~$x$, then $\forall x.P(x)$ is true.'
The $\forall E$ rule exploits $\beta$-conversion.  Hiding $Trueprop$, the
$\forall$ axioms are
$$ \left(\Forall x. P(x)\right)  \Imp  \forall x.P(x)   \eqno(\forall I) $$
$$ (\forall x.P(x))  \Imp P(t).  \eqno(\forall E) $$

\noindent
We have defined the object-level universal quantifier~($\forall$)
using~$\Forall$.  But we do not require meta-level counterparts of all the
connectives of the object-logic!  Consider the existential quantifier: 
$$ P(t)  \Imp  \exists x.P(x)  \eqno(\exists I) $$
$$ \List{\exists x.P(x);\; \Forall x. P(x)\Imp Q} \Imp Q  \eqno(\exists E) $$
Let us verify $(\exists E)$ semantically.  Suppose that the premises
hold; since $\exists x.P(x)$ is true, we may choose an~$a$ such that $P(a)$ is
true.  Instantiating $\Forall x. P(x)\Imp Q$ with $a$ yields $P(a)\Imp Q$, and
we obtain the desired conclusion, $Q$.

The treatment of substitution deserves mention.  The rule
\[ \infer{P[u/t]}{t=u & P} \]
would be hard to formalize in Isabelle.  It calls for replacing~$t$ by $u$
throughout~$P$, which cannot be expressed using $\beta$-conversion.  Our
rule~$(subst)$ uses~$P$ as a template for substitution, inferring $P[u/x]$
from~$P[t/x]$.  When we formalize this as an axiom, the template becomes a
function variable:
$$ \List{t=u; P(t)} \Imp P(u).  \eqno(subst) $$


\subsection{Signatures and theories}
\index{signatures|bold}

A {\bf signature} contains the information necessary for type-checking,
parsing and pretty printing a term.  It specifies type classes and their
relationships, types and their arities, constants and their types, etc.  It
also contains grammar rules, specified using mixfix declarations.

Two signatures can be merged provided their specifications are compatible ---
they must not, for example, assign different types to the same constant.
Under similar conditions, a signature can be extended.  Signatures are
managed internally by Isabelle; users seldom encounter them.

\index{theories|bold} A {\bf theory} consists of a signature plus a collection
of axioms.  The Pure theory contains only the meta-logic.  Theories can be
combined provided their signatures are compatible.  A theory definition
extends an existing theory with further signature specifications --- classes,
types, constants and mixfix declarations --- plus lists of axioms and
definitions etc., expressed as strings to be parsed.  A theory can formalize a
small piece of mathematics, such as lists and their operations, or an entire
logic.  A mathematical development typically involves many theories in a
hierarchy.  For example, the Pure theory could be extended to form a theory
for Fig.\ts\ref{fol-fig}; this could be extended in two separate ways to form
a theory for natural numbers and a theory for lists; the union of these two
could be extended into a theory defining the length of a list:
\begin{tt}
\[
\begin{array}{c@{}c@{}c@{}c@{}c}
     {}   &     {}   &\hbox{Pure}&     {}  &     {}  \\
     {}   &     {}   &  \downarrow &     {}   &     {}   \\
     {}   &     {}   &\hbox{FOL} &     {}   &     {}   \\
     {}   & \swarrow &     {}    & \searrow &     {}   \\
 \hbox{Nat} &   {}   &     {}    &     {}   & \hbox{List} \\
     {}   & \searrow &     {}    & \swarrow &     {}   \\
     {}   &     {} &\hbox{Nat}+\hbox{List}&  {}   &     {}   \\
     {}   &     {}   &  \downarrow &     {}   &     {}   \\
     {}   &     {} & \hbox{Length} &  {}   &     {}
\end{array}
\]
\end{tt}%
Each Isabelle proof typically works within a single theory, which is
associated with the proof state.  However, many different theories may
coexist at the same time, and you may work in each of these during a single
session.  

\begin{warn}\index{constants!clashes with variables}%
  Confusing problems arise if you work in the wrong theory.  Each theory
  defines its own syntax.  An identifier may be regarded in one theory as a
  constant and in another as a variable, for example.
\end{warn}

\section{Proof construction in Isabelle}
I have elsewhere described the meta-logic and demonstrated it by
formalizing first-order logic~\cite{paulson-found}.  There is a one-to-one
correspondence between meta-level proofs and object-level proofs.  To each
use of a meta-level axiom, such as $(\forall I)$, there is a use of the
corresponding object-level rule.  Object-level assumptions and parameters
have meta-level counterparts.  The meta-level formalization is {\bf
  faithful}, admitting no incorrect object-level inferences, and {\bf
  adequate}, admitting all correct object-level inferences.  These
properties must be demonstrated separately for each object-logic.

The meta-logic is defined by a collection of inference rules, including
equational rules for the $\lambda$-calculus and logical rules.  The rules
for~$\Imp$ and~$\Forall$ resemble those for~$\imp$ and~$\forall$ in
Fig.\ts\ref{fol-fig}.  Proofs performed using the primitive meta-rules
would be lengthy; Isabelle proofs normally use certain derived rules.
{\bf Resolution}, in particular, is convenient for backward proof.

Unification is central to theorem proving.  It supports quantifier
reasoning by allowing certain `unknown' terms to be instantiated later,
possibly in stages.  When proving that the time required to sort $n$
integers is proportional to~$n^2$, we need not state the constant of
proportionality; when proving that a hardware adder will deliver the sum of
its inputs, we need not state how many clock ticks will be required.  Such
quantities often emerge from the proof.

Isabelle provides {\bf schematic variables}, or {\bf
  unknowns},\index{unknowns} for unification.  Logically, unknowns are free
variables.  But while ordinary variables remain fixed, unification may
instantiate unknowns.  Unknowns are written with a ?\ prefix and are
frequently subscripted: $\Var{a}$, $\Var{a@1}$, $\Var{a@2}$, \ldots,
$\Var{P}$, $\Var{P@1}$, \ldots.

Recall that an inference rule of the form
\[ \infer{\phi}{\phi@1 & \ldots & \phi@n} \]
is formalized in Isabelle's meta-logic as the axiom
$\List{\phi@1; \ldots; \phi@n} \Imp \phi$.\index{resolution}
Such axioms resemble Prolog's Horn clauses, and can be combined by
resolution --- Isabelle's principal proof method.  Resolution yields both
forward and backward proof.  Backward proof works by unifying a goal with
the conclusion of a rule, whose premises become new subgoals.  Forward proof
works by unifying theorems with the premises of a rule, deriving a new theorem.

Isabelle formulae require an extended notion of resolution.
They differ from Horn clauses in two major respects:
\begin{itemize}
  \item They are written in the typed $\lambda$-calculus, and therefore must be
resolved using higher-order unification.

\item The constituents of a clause need not be atomic formulae.  Any
  formula of the form $Trueprop(\cdots)$ is atomic, but axioms such as
  ${\imp}I$ and $\forall I$ contain non-atomic formulae.
\end{itemize}
Isabelle has little in common with classical resolution theorem provers
such as Otter~\cite{wos-bledsoe}.  At the meta-level, Isabelle proves
theorems in their positive form, not by refutation.  However, an
object-logic that includes a contradiction rule may employ a refutation
proof procedure.


\subsection{Higher-order unification}
\index{unification!higher-order|bold}
Unification is equation solving.  The solution of $f(\Var{x},c) \qeq
f(d,\Var{y})$ is $\Var{x}\equiv d$ and $\Var{y}\equiv c$.  {\bf
Higher-order unification} is equation solving for typed $\lambda$-terms.
To handle $\beta$-conversion, it must reduce $(\lambda x.t)u$ to $t[u/x]$.
That is easy --- in the typed $\lambda$-calculus, all reduction sequences
terminate at a normal form.  But it must guess the unknown
function~$\Var{f}$ in order to solve the equation
\begin{equation} \label{hou-eqn}
 \Var{f}(t) \qeq g(u@1,\ldots,u@k).
\end{equation}
Huet's~\cite{huet75} search procedure solves equations by imitation and
projection.  {\bf Imitation} makes~$\Var{f}$ apply the leading symbol (if a
constant) of the right-hand side.  To solve equation~(\ref{hou-eqn}), it
guesses
\[ \Var{f} \equiv \lambda x. g(\Var{h@1}(x),\ldots,\Var{h@k}(x)), \]
where $\Var{h@1}$, \ldots, $\Var{h@k}$ are new unknowns.  Assuming there are no
other occurrences of~$\Var{f}$, equation~(\ref{hou-eqn}) simplifies to the
set of equations
\[ \Var{h@1}(t)\qeq u@1 \quad\ldots\quad \Var{h@k}(t)\qeq u@k. \]
If the procedure solves these equations, instantiating $\Var{h@1}$, \ldots,
$\Var{h@k}$, then it yields an instantiation for~$\Var{f}$.

{\bf Projection} makes $\Var{f}$ apply one of its arguments.  To solve
equation~(\ref{hou-eqn}), if $t$ expects~$m$ arguments and delivers a
result of suitable type, it guesses
\[ \Var{f} \equiv \lambda x. x(\Var{h@1}(x),\ldots,\Var{h@m}(x)), \]
where $\Var{h@1}$, \ldots, $\Var{h@m}$ are new unknowns.  Assuming there are no
other occurrences of~$\Var{f}$, equation~(\ref{hou-eqn}) simplifies to the 
equation 
\[ t(\Var{h@1}(t),\ldots,\Var{h@m}(t)) \qeq g(u@1,\ldots,u@k). \]

\begin{warn}\index{unification!incompleteness of}%
Huet's unification procedure is complete.  Isabelle's polymorphic version,
which solves for type unknowns as well as for term unknowns, is incomplete.
The problem is that projection requires type information.  In
equation~(\ref{hou-eqn}), if the type of~$t$ is unknown, then projections
are possible for all~$m\geq0$, and the types of the $\Var{h@i}$ will be
similarly unconstrained.  Therefore, Isabelle never attempts such
projections, and may fail to find unifiers where a type unknown turns out
to be a function type.
\end{warn}

\index{unknowns!function|bold}
Given $\Var{f}(t@1,\ldots,t@n)\qeq u$, Huet's procedure could make up to
$n+1$ guesses.  The search tree and set of unifiers may be infinite.  But
higher-order unification can work effectively, provided you are careful
with {\bf function unknowns}:
\begin{itemize}
  \item Equations with no function unknowns are solved using first-order
unification, extended to treat bound variables.  For example, $\lambda x.x
\qeq \lambda x.\Var{y}$ has no solution because $\Var{y}\equiv x$ would
capture the free variable~$x$.

  \item An occurrence of the term $\Var{f}(x,y,z)$, where the arguments are
distinct bound variables, causes no difficulties.  Its projections can only
match the corresponding variables.

  \item Even an equation such as $\Var{f}(a)\qeq a+a$ is all right.  It has
four solutions, but Isabelle evaluates them lazily, trying projection before
imitation.  The first solution is usually the one desired:
\[ \Var{f}\equiv \lambda x. x+x \quad
   \Var{f}\equiv \lambda x. a+x \quad
   \Var{f}\equiv \lambda x. x+a \quad
   \Var{f}\equiv \lambda x. a+a \]
  \item  Equations such as $\Var{f}(\Var{x},\Var{y})\qeq t$ and
$\Var{f}(\Var{g}(x))\qeq t$ admit vast numbers of unifiers, and must be
avoided. 
\end{itemize}
In problematic cases, you may have to instantiate some unknowns before
invoking unification. 


\subsection{Joining rules by resolution} \label{joining}
\index{resolution|bold}
Let $\List{\psi@1; \ldots; \psi@m} \Imp \psi$ and $\List{\phi@1; \ldots;
\phi@n} \Imp \phi$ be two Isabelle theorems, representing object-level rules. 
Choosing some~$i$ from~1 to~$n$, suppose that $\psi$ and $\phi@i$ have a
higher-order unifier.  Writing $Xs$ for the application of substitution~$s$ to
expression~$X$, this means there is some~$s$ such that $\psi s\equiv \phi@i s$.
By resolution, we may conclude
\[ (\List{\phi@1; \ldots; \phi@{i-1}; \psi@1; \ldots; \psi@m;
          \phi@{i+1}; \ldots; \phi@n} \Imp \phi)s.
\]
The substitution~$s$ may instantiate unknowns in both rules.  In short,
resolution is the following rule:
\[ \infer[(\psi s\equiv \phi@i s)]
         {(\List{\phi@1; \ldots; \phi@{i-1}; \psi@1; \ldots; \psi@m;
          \phi@{i+1}; \ldots; \phi@n} \Imp \phi)s}
         {\List{\psi@1; \ldots; \psi@m} \Imp \psi & &
          \List{\phi@1; \ldots; \phi@n} \Imp \phi}
\]
It operates at the meta-level, on Isabelle theorems, and is justified by
the properties of $\Imp$ and~$\Forall$.  It takes the number~$i$ (for
$1\leq i\leq n$) as a parameter and may yield infinitely many conclusions,
one for each unifier of $\psi$ with $\phi@i$.  Isabelle returns these
conclusions as a sequence (lazy list).

Resolution expects the rules to have no outer quantifiers~($\Forall$).
It may rename or instantiate any schematic variables, but leaves free
variables unchanged.  When constructing a theory, Isabelle puts the
rules into a standard form with all free variables converted into
schematic ones; for instance, $({\imp}E)$ becomes
\[ \List{\Var{P}\imp \Var{Q}; \Var{P}}  \Imp \Var{Q}. 
\]
When resolving two rules, the unknowns in the first rule are renamed, by
subscripting, to make them distinct from the unknowns in the second rule.  To
resolve $({\imp}E)$ with itself, the first copy of the rule becomes
\[ \List{\Var{P@1}\imp \Var{Q@1}; \Var{P@1}}  \Imp \Var{Q@1}. \]
Resolving this with $({\imp}E)$ in the first premise, unifying $\Var{Q@1}$ with
$\Var{P}\imp \Var{Q}$, is the meta-level inference
\[ \infer{\List{\Var{P@1}\imp (\Var{P}\imp \Var{Q}); \Var{P@1}; \Var{P}} 
           \Imp\Var{Q}.}
         {\List{\Var{P@1}\imp \Var{Q@1}; \Var{P@1}}  \Imp \Var{Q@1} & &
          \List{\Var{P}\imp \Var{Q}; \Var{P}}  \Imp \Var{Q}}
\]
Renaming the unknowns in the resolvent, we have derived the
object-level rule\index{rules!derived}
\[ \infer{Q.}{R\imp (P\imp Q)  &  R  &  P}  \]
Joining rules in this fashion is a simple way of proving theorems.  The
derived rules are conservative extensions of the object-logic, and may permit
simpler proofs.  Let us consider another example.  Suppose we have the axiom
$$ \forall x\,y. Suc(x)=Suc(y)\imp x=y. \eqno (inject) $$

\noindent 
The standard form of $(\forall E)$ is
$\forall x.\Var{P}(x)  \Imp \Var{P}(\Var{t})$.
Resolving $(inject)$ with $(\forall E)$ replaces $\Var{P}$ by
$\lambda x. \forall y. Suc(x)=Suc(y)\imp x=y$ and leaves $\Var{t}$
unchanged, yielding  
\[ \forall y. Suc(\Var{t})=Suc(y)\imp \Var{t}=y. \]
Resolving this with $(\forall E)$ puts a subscript on~$\Var{t}$
and yields
\[ Suc(\Var{t@1})=Suc(\Var{t})\imp \Var{t@1}=\Var{t}. \]
Resolving this with $({\imp}E)$ increases the subscripts and yields
\[ Suc(\Var{t@2})=Suc(\Var{t@1})\Imp \Var{t@2}=\Var{t@1}. 
\]
We have derived the rule
\[ \infer{m=n,}{Suc(m)=Suc(n)} \]
which goes directly from $Suc(m)=Suc(n)$ to $m=n$.  It is handy for simplifying
an equation like $Suc(Suc(Suc(m)))=Suc(Suc(Suc(0)))$.  


\section{Lifting a rule into a context}
The rules $({\imp}I)$ and $(\forall I)$ may seem unsuitable for
resolution.  They have non-atomic premises, namely $P\Imp Q$ and $\Forall
x.P(x)$, while the conclusions of all the rules are atomic (they have the form
$Trueprop(\cdots)$).  Isabelle gets round the problem through a meta-inference
called \bfindex{lifting}.  Let us consider how to construct proofs such as
\[ \infer[({\imp}I)]{P\imp(Q\imp R)}
         {\infer[({\imp}I)]{Q\imp R}
                        {\infer*{R}{[P,Q]}}}
   \qquad
   \infer[(\forall I)]{\forall x\,y.P(x,y)}
         {\infer[(\forall I)]{\forall y.P(x,y)}{P(x,y)}}
\]

\subsection{Lifting over assumptions}
\index{assumptions!lifting over}
Lifting over $\theta\Imp{}$ is the following meta-inference rule:
\[ \infer{\List{\theta\Imp\phi@1; \ldots; \theta\Imp\phi@n} \Imp
          (\theta \Imp \phi)}
         {\List{\phi@1; \ldots; \phi@n} \Imp \phi} \]
This is clearly sound: if $\List{\phi@1; \ldots; \phi@n} \Imp \phi$ is true and
$\theta\Imp\phi@1$, \ldots, $\theta\Imp\phi@n$ and $\theta$ are all true then
$\phi$ must be true.  Iterated lifting over a series of meta-formulae
$\theta@k$, \ldots, $\theta@1$ yields an object-rule whose conclusion is
$\List{\theta@1; \ldots; \theta@k} \Imp \phi$.  Typically the $\theta@i$ are
the assumptions in a natural deduction proof; lifting copies them into a rule's
premises and conclusion.

When resolving two rules, Isabelle lifts the first one if necessary.  The
standard form of $({\imp}I)$ is
\[ (\Var{P} \Imp \Var{Q})  \Imp  \Var{P}\imp \Var{Q}.   \]
To resolve this rule with itself, Isabelle modifies one copy as follows: it
renames the unknowns to $\Var{P@1}$ and $\Var{Q@1}$, then lifts the rule over
$\Var{P}\Imp{}$ to obtain
\[ (\Var{P}\Imp (\Var{P@1} \Imp \Var{Q@1})) \Imp (\Var{P} \Imp 
   (\Var{P@1}\imp \Var{Q@1})).   \]
Using the $\List{\cdots}$ abbreviation, this can be written as
\[ \List{\List{\Var{P}; \Var{P@1}} \Imp \Var{Q@1}; \Var{P}} 
   \Imp  \Var{P@1}\imp \Var{Q@1}.   \]
Unifying $\Var{P}\Imp \Var{P@1}\imp\Var{Q@1}$ with $\Var{P} \Imp
\Var{Q}$ instantiates $\Var{Q}$ to ${\Var{P@1}\imp\Var{Q@1}}$.
Resolution yields
\[ (\List{\Var{P}; \Var{P@1}} \Imp \Var{Q@1}) \Imp
\Var{P}\imp(\Var{P@1}\imp\Var{Q@1}).   \]
This represents the derived rule
\[ \infer{P\imp(Q\imp R).}{\infer*{R}{[P,Q]}} \]

\subsection{Lifting over parameters}
\index{parameters!lifting over}
An analogous form of lifting handles premises of the form $\Forall x\ldots\,$. 
Here, lifting prefixes an object-rule's premises and conclusion with $\Forall
x$.  At the same time, lifting introduces a dependence upon~$x$.  It replaces
each unknown $\Var{a}$ in the rule by $\Var{a'}(x)$, where $\Var{a'}$ is a new
unknown (by subscripting) of suitable type --- necessarily a function type.  In
short, lifting is the meta-inference
\[ \infer{\List{\Forall x.\phi@1^x; \ldots; \Forall x.\phi@n^x} 
           \Imp \Forall x.\phi^x,}
         {\List{\phi@1; \ldots; \phi@n} \Imp \phi} \]
%
where $\phi^x$ stands for the result of lifting unknowns over~$x$ in
$\phi$.  It is not hard to verify that this meta-inference is sound.  If
$\phi\Imp\psi$ then $\phi^x\Imp\psi^x$ for all~$x$; so if $\phi^x$ is true
for all~$x$ then so is $\psi^x$.  Thus, from $\phi\Imp\psi$ we conclude
$(\Forall x.\phi^x) \Imp (\Forall x.\psi^x)$.

For example, $(\disj I)$ might be lifted to
\[ (\Forall x.\Var{P@1}(x)) \Imp (\Forall x. \Var{P@1}(x)\disj \Var{Q@1}(x))\]
and $(\forall I)$ to
\[ (\Forall x\,y.\Var{P@1}(x,y)) \Imp (\Forall x. \forall y.\Var{P@1}(x,y)). \]
Isabelle has renamed a bound variable in $(\forall I)$ from $x$ to~$y$,
avoiding a clash.  Resolving the above with $(\forall I)$ is the meta-inference
\[ \infer{\Forall x\,y.\Var{P@1}(x,y)) \Imp \forall x\,y.\Var{P@1}(x,y)) }
         {(\Forall x\,y.\Var{P@1}(x,y)) \Imp 
               (\Forall x. \forall y.\Var{P@1}(x,y))  &
          (\Forall x.\Var{P}(x)) \Imp (\forall x.\Var{P}(x))} \]
Here, $\Var{P}$ is replaced by $\lambda x.\forall y.\Var{P@1}(x,y)$; the
resolvent expresses the derived rule
\[ \vcenter{ \infer{\forall x\,y.Q(x,y)}{Q(x,y)} }
   \quad\hbox{provided $x$, $y$ not free in the assumptions} 
\] 
I discuss lifting and parameters at length elsewhere~\cite{paulson-found}.
Miller goes into even greater detail~\cite{miller-mixed}.


\section{Backward proof by resolution}
\index{resolution!in backward proof}

Resolution is convenient for deriving simple rules and for reasoning
forward from facts.  It can also support backward proof, where we start
with a goal and refine it to progressively simpler subgoals until all have
been solved.  {\sc lcf} and its descendants {\sc hol} and Nuprl provide
tactics and tacticals, which constitute a sophisticated language for
expressing proof searches.  {\bf Tactics} refine subgoals while {\bf
  tacticals} combine tactics.

\index{LCF system}
Isabelle's tactics and tacticals work differently from {\sc lcf}'s.  An
Isabelle rule is bidirectional: there is no distinction between
inputs and outputs.  {\sc lcf} has a separate tactic for each rule;
Isabelle performs refinement by any rule in a uniform fashion, using
resolution.

Isabelle works with meta-level theorems of the form
\( \List{\phi@1; \ldots; \phi@n} \Imp \phi \).
We have viewed this as the {\bf rule} with premises
$\phi@1$,~\ldots,~$\phi@n$ and conclusion~$\phi$.  It can also be viewed as
the {\bf proof state}\index{proof state}
with subgoals $\phi@1$,~\ldots,~$\phi@n$ and main
goal~$\phi$.

To prove the formula~$\phi$, take $\phi\Imp \phi$ as the initial proof
state.  This assertion is, trivially, a theorem.  At a later stage in the
backward proof, a typical proof state is $\List{\phi@1; \ldots; \phi@n}
\Imp \phi$.  This proof state is a theorem, ensuring that the subgoals
$\phi@1$,~\ldots,~$\phi@n$ imply~$\phi$.  If $n=0$ then we have
proved~$\phi$ outright.  If $\phi$ contains unknowns, they may become
instantiated during the proof; a proof state may be $\List{\phi@1; \ldots;
\phi@n} \Imp \phi'$, where $\phi'$ is an instance of~$\phi$.

\subsection{Refinement by resolution}
To refine subgoal~$i$ of a proof state by a rule, perform the following
resolution: 
\[ \infer{\hbox{new proof state}}{\hbox{rule} & & \hbox{proof state}} \]
Suppose the rule is $\List{\psi'@1; \ldots; \psi'@m} \Imp \psi'$ after
lifting over subgoal~$i$'s assumptions and parameters.  If the proof state
is $\List{\phi@1; \ldots; \phi@n} \Imp \phi$, then the new proof state is
(for~$1\leq i\leq n$)
\[ (\List{\phi@1; \ldots; \phi@{i-1}; \psi'@1;
          \ldots; \psi'@m; \phi@{i+1}; \ldots; \phi@n} \Imp \phi)s.  \]
Substitution~$s$ unifies $\psi'$ with~$\phi@i$.  In the proof state,
subgoal~$i$ is replaced by $m$ new subgoals, the rule's instantiated premises.
If some of the rule's unknowns are left un-instantiated, they become new
unknowns in the proof state.  Refinement by~$(\exists I)$, namely
\[ \Var{P}(\Var{t}) \Imp \exists x. \Var{P}(x), \]
inserts a new unknown derived from~$\Var{t}$ by subscripting and lifting.
We do not have to specify an `existential witness' when
applying~$(\exists I)$.  Further resolutions may instantiate unknowns in
the proof state.

\subsection{Proof by assumption}
\index{assumptions!use of}
In the course of a natural deduction proof, parameters $x@1$, \ldots,~$x@l$ and
assumptions $\theta@1$, \ldots, $\theta@k$ accumulate, forming a context for
each subgoal.  Repeated lifting steps can lift a rule into any context.  To
aid readability, Isabelle puts contexts into a normal form, gathering the
parameters at the front:
\begin{equation} \label{context-eqn}
\Forall x@1 \ldots x@l. \List{\theta@1; \ldots; \theta@k}\Imp\theta. 
\end{equation}
Under the usual reading of the connectives, this expresses that $\theta$
follows from $\theta@1$,~\ldots~$\theta@k$ for arbitrary
$x@1$,~\ldots,~$x@l$.  It is trivially true if $\theta$ equals any of
$\theta@1$,~\ldots~$\theta@k$, or is unifiable with any of them.  This
models proof by assumption in natural deduction.

Isabelle automates the meta-inference for proof by assumption.  Its arguments
are the meta-theorem $\List{\phi@1; \ldots; \phi@n} \Imp \phi$, and some~$i$
from~1 to~$n$, where $\phi@i$ has the form~(\ref{context-eqn}).  Its results
are meta-theorems of the form
\[ (\List{\phi@1; \ldots; \phi@{i-1}; \phi@{i+1}; \phi@n} \Imp \phi)s \]
for each $s$ and~$j$ such that $s$ unifies $\lambda x@1 \ldots x@l. \theta@j$
with $\lambda x@1 \ldots x@l. \theta$.  Isabelle supplies the parameters
$x@1$,~\ldots,~$x@l$ to higher-order unification as bound variables, which
regards them as unique constants with a limited scope --- this enforces
parameter provisos~\cite{paulson-found}.

The premise represents a proof state with~$n$ subgoals, of which the~$i$th
is to be solved by assumption.  Isabelle searches the subgoal's context for
an assumption~$\theta@j$ that can solve it.  For each unifier, the
meta-inference returns an instantiated proof state from which the $i$th
subgoal has been removed.  Isabelle searches for a unifying assumption; for
readability and robustness, proofs do not refer to assumptions by number.

Consider the proof state 
\[ (\List{P(a); P(b)} \Imp P(\Var{x})) \Imp Q(\Var{x}). \]
Proof by assumption (with $i=1$, the only possibility) yields two results:
\begin{itemize}
  \item $Q(a)$, instantiating $\Var{x}\equiv a$
  \item $Q(b)$, instantiating $\Var{x}\equiv b$
\end{itemize}
Here, proof by assumption affects the main goal.  It could also affect
other subgoals; if we also had the subgoal ${\List{P(b); P(c)} \Imp
  P(\Var{x})}$, then $\Var{x}\equiv a$ would transform it to ${\List{P(b);
    P(c)} \Imp P(a)}$, which might be unprovable.


\subsection{A propositional proof} \label{prop-proof}
\index{examples!propositional}
Our first example avoids quantifiers.  Given the main goal $P\disj P\imp
P$, Isabelle creates the initial state
\[ (P\disj P\imp P)\Imp (P\disj P\imp P). \] 
%
Bear in mind that every proof state we derive will be a meta-theorem,
expressing that the subgoals imply the main goal.  Our aim is to reach the
state $P\disj P\imp P$; this meta-theorem is the desired result.

The first step is to refine subgoal~1 by (${\imp}I)$, creating a new state
where $P\disj P$ is an assumption:
\[ (P\disj P\Imp P)\Imp (P\disj P\imp P) \]
The next step is $(\disj E)$, which replaces subgoal~1 by three new subgoals. 
Because of lifting, each subgoal contains a copy of the context --- the
assumption $P\disj P$.  (In fact, this assumption is now redundant; we shall
shortly see how to get rid of it!)  The new proof state is the following
meta-theorem, laid out for clarity:
\[ \begin{array}{l@{}l@{\qquad\qquad}l} 
  \lbrakk\;& P\disj P\Imp \Var{P@1}\disj\Var{Q@1}; & \hbox{(subgoal 1)} \\
           & \List{P\disj P; \Var{P@1}} \Imp P;    & \hbox{(subgoal 2)} \\
           & \List{P\disj P; \Var{Q@1}} \Imp P     & \hbox{(subgoal 3)} \\
  \rbrakk\;& \Imp (P\disj P\imp P)                 & \hbox{(main goal)}
   \end{array} 
\]
Notice the unknowns in the proof state.  Because we have applied $(\disj E)$,
we must prove some disjunction, $\Var{P@1}\disj\Var{Q@1}$.  Of course,
subgoal~1 is provable by assumption.  This instantiates both $\Var{P@1}$ and
$\Var{Q@1}$ to~$P$ throughout the proof state:
\[ \begin{array}{l@{}l@{\qquad\qquad}l} 
    \lbrakk\;& \List{P\disj P; P} \Imp P; & \hbox{(subgoal 1)} \\
             & \List{P\disj P; P} \Imp P  & \hbox{(subgoal 2)} \\
    \rbrakk\;& \Imp (P\disj P\imp P)      & \hbox{(main goal)}
   \end{array} \]
Both of the remaining subgoals can be proved by assumption.  After two such
steps, the proof state is $P\disj P\imp P$.


\subsection{A quantifier proof}
\index{examples!with quantifiers}
To illustrate quantifiers and $\Forall$-lifting, let us prove
$(\exists x.P(f(x)))\imp(\exists x.P(x))$.  The initial proof
state is the trivial meta-theorem 
\[ (\exists x.P(f(x)))\imp(\exists x.P(x)) \Imp 
   (\exists x.P(f(x)))\imp(\exists x.P(x)). \]
As above, the first step is refinement by (${\imp}I)$: 
\[ (\exists x.P(f(x))\Imp \exists x.P(x)) \Imp 
   (\exists x.P(f(x)))\imp(\exists x.P(x)) 
\]
The next step is $(\exists E)$, which replaces subgoal~1 by two new subgoals.
Both have the assumption $\exists x.P(f(x))$.  The new proof
state is the meta-theorem  
\[ \begin{array}{l@{}l@{\qquad\qquad}l} 
   \lbrakk\;& \exists x.P(f(x)) \Imp \exists x.\Var{P@1}(x); & \hbox{(subgoal 1)} \\
            & \Forall x.\List{\exists x.P(f(x)); \Var{P@1}(x)} \Imp 
                       \exists x.P(x)  & \hbox{(subgoal 2)} \\
    \rbrakk\;& \Imp (\exists x.P(f(x)))\imp(\exists x.P(x))  & \hbox{(main goal)}
   \end{array} 
\]
The unknown $\Var{P@1}$ appears in both subgoals.  Because we have applied
$(\exists E)$, we must prove $\exists x.\Var{P@1}(x)$, where $\Var{P@1}(x)$ may
become any formula possibly containing~$x$.  Proving subgoal~1 by assumption
instantiates $\Var{P@1}$ to~$\lambda x.P(f(x))$:  
\[ \left(\Forall x.\List{\exists x.P(f(x)); P(f(x))} \Imp 
         \exists x.P(x)\right) 
   \Imp (\exists x.P(f(x)))\imp(\exists x.P(x)) 
\]
The next step is refinement by $(\exists I)$.  The rule is lifted into the
context of the parameter~$x$ and the assumption $P(f(x))$.  This copies
the context to the subgoal and allows the existential witness to
depend upon~$x$: 
\[ \left(\Forall x.\List{\exists x.P(f(x)); P(f(x))} \Imp 
         P(\Var{x@2}(x))\right) 
   \Imp (\exists x.P(f(x)))\imp(\exists x.P(x)) 
\]
The existential witness, $\Var{x@2}(x)$, consists of an unknown
applied to a parameter.  Proof by assumption unifies $\lambda x.P(f(x))$ 
with $\lambda x.P(\Var{x@2}(x))$, instantiating $\Var{x@2}$ to $f$.  The final
proof state contains no subgoals: $(\exists x.P(f(x)))\imp(\exists x.P(x))$.


\subsection{Tactics and tacticals}
\index{tactics|bold}\index{tacticals|bold}
{\bf Tactics} perform backward proof.  Isabelle tactics differ from those
of {\sc lcf}, {\sc hol} and Nuprl by operating on entire proof states,
rather than on individual subgoals.  An Isabelle tactic is a function that
takes a proof state and returns a sequence (lazy list) of possible
successor states.  Lazy lists are coded in ML as functions, a standard
technique~\cite{paulson-ml2}.  Isabelle represents proof states by theorems.

Basic tactics execute the meta-rules described above, operating on a
given subgoal.  The {\bf resolution tactics} take a list of rules and
return next states for each combination of rule and unifier.  The {\bf
assumption tactic} examines the subgoal's assumptions and returns next
states for each combination of assumption and unifier.  Lazy lists are
essential because higher-order resolution may return infinitely many
unifiers.  If there are no matching rules or assumptions then no next
states are generated; a tactic application that returns an empty list is
said to {\bf fail}.

Sequences realize their full potential with {\bf tacticals} --- operators
for combining tactics.  Depth-first search, breadth-first search and
best-first search (where a heuristic function selects the best state to
explore) return their outcomes as a sequence.  Isabelle provides such
procedures in the form of tacticals.  Simpler procedures can be expressed
directly using the basic tacticals {\tt THEN}, {\tt ORELSE} and {\tt REPEAT}:
\begin{ttdescription}
\item[$tac1$ THEN $tac2$] is a tactic for sequential composition.  Applied
to a proof state, it returns all states reachable in two steps by applying
$tac1$ followed by~$tac2$.

\item[$tac1$ ORELSE $tac2$] is a choice tactic.  Applied to a state, it
tries~$tac1$ and returns the result if non-empty; otherwise, it uses~$tac2$.

\item[REPEAT $tac$] is a repetition tactic.  Applied to a state, it
returns all states reachable by applying~$tac$ as long as possible --- until
it would fail.  
\end{ttdescription}
For instance, this tactic repeatedly applies $tac1$ and~$tac2$, giving
$tac1$ priority:
\begin{center} \tt
REPEAT($tac1$ ORELSE $tac2$)
\end{center}


\section{Variations on resolution}
In principle, resolution and proof by assumption suffice to prove all
theorems.  However, specialized forms of resolution are helpful for working
with elimination rules.  Elim-resolution applies an elimination rule to an
assumption; destruct-resolution is similar, but applies a rule in a forward
style.

The last part of the section shows how the techniques for proving theorems
can also serve to derive rules.

\subsection{Elim-resolution}
\index{elim-resolution|bold}\index{assumptions!deleting}

Consider proving the theorem $((R\disj R)\disj R)\disj R\imp R$.  By
$({\imp}I)$, we prove~$R$ from the assumption $((R\disj R)\disj R)\disj R$.
Applying $(\disj E)$ to this assumption yields two subgoals, one that
assumes~$R$ (and is therefore trivial) and one that assumes $(R\disj
R)\disj R$.  This subgoal admits another application of $(\disj E)$.  Since
natural deduction never discards assumptions, we eventually generate a
subgoal containing much that is redundant:
\[ \List{((R\disj R)\disj R)\disj R; (R\disj R)\disj R; R\disj R; R} \Imp R. \]
In general, using $(\disj E)$ on the assumption $P\disj Q$ creates two new
subgoals with the additional assumption $P$ or~$Q$.  In these subgoals,
$P\disj Q$ is redundant.  Other elimination rules behave
similarly.  In first-order logic, only universally quantified
assumptions are sometimes needed more than once --- say, to prove
$P(f(f(a)))$ from the assumptions $\forall x.P(x)\imp P(f(x))$ and~$P(a)$.

Many logics can be formulated as sequent calculi that delete redundant
assumptions after use.  The rule $(\disj E)$ might become
\[ \infer[\disj\hbox{-left}]
         {\Gamma,P\disj Q,\Delta \turn \Theta}
         {\Gamma,P,\Delta \turn \Theta && \Gamma,Q,\Delta \turn \Theta}  \] 
In backward proof, a goal containing $P\disj Q$ on the left of the~$\turn$
(that is, as an assumption) splits into two subgoals, replacing $P\disj Q$
by $P$ or~$Q$.  But the sequent calculus, with its explicit handling of
assumptions, can be tiresome to use.

Elim-resolution is Isabelle's way of getting sequent calculus behaviour
from natural deduction rules.  It lets an elimination rule consume an
assumption.  Elim-resolution combines two meta-theorems:
\begin{itemize}
  \item a rule $\List{\psi@1; \ldots; \psi@m} \Imp \psi$
  \item a proof state $\List{\phi@1; \ldots; \phi@n} \Imp \phi$
\end{itemize}
The rule must have at least one premise, thus $m>0$.  Write the rule's
lifted form as $\List{\psi'@1; \ldots; \psi'@m} \Imp \psi'$.  Suppose we
wish to change subgoal number~$i$.

Ordinary resolution would attempt to reduce~$\phi@i$,
replacing subgoal~$i$ by $m$ new ones.  Elim-resolution tries
simultaneously to reduce~$\phi@i$ and to solve~$\psi'@1$ by assumption; it
returns a sequence of next states.  Each of these replaces subgoal~$i$ by
instances of $\psi'@2$, \ldots, $\psi'@m$ from which the selected
assumption has been deleted.  Suppose $\phi@i$ has the parameter~$x$ and
assumptions $\theta@1$,~\ldots,~$\theta@k$.  Then $\psi'@1$, the rule's first
premise after lifting, will be
\( \Forall x. \List{\theta@1; \ldots; \theta@k}\Imp \psi^{x}@1 \).
Elim-resolution tries to unify $\psi'\qeq\phi@i$ and
$\lambda x. \theta@j \qeq \lambda x. \psi^{x}@1$ simultaneously, for
$j=1$,~\ldots,~$k$. 

Let us redo the example from~\S\ref{prop-proof}.  The elimination rule
is~$(\disj E)$,
\[ \List{\Var{P}\disj \Var{Q};\; \Var{P}\Imp \Var{R};\; \Var{Q}\Imp \Var{R}}
      \Imp \Var{R}  \]
and the proof state is $(P\disj P\Imp P)\Imp (P\disj P\imp P)$.  The
lifted rule is
\[ \begin{array}{l@{}l}
  \lbrakk\;& P\disj P \Imp \Var{P@1}\disj\Var{Q@1}; \\
           & \List{P\disj P ;\; \Var{P@1}} \Imp \Var{R@1};    \\
           & \List{P\disj P ;\; \Var{Q@1}} \Imp \Var{R@1}     \\
  \rbrakk\;& \Imp (P\disj P \Imp \Var{R@1})
  \end{array} 
\]
Unification takes the simultaneous equations
$P\disj P \qeq \Var{P@1}\disj\Var{Q@1}$ and $\Var{R@1} \qeq P$, yielding
$\Var{P@1}\equiv\Var{Q@1}\equiv\Var{R@1} \equiv P$.  The new proof state
is simply
\[ \List{P \Imp P;\; P \Imp P} \Imp (P\disj P\imp P). 
\]
Elim-resolution's simultaneous unification gives better control
than ordinary resolution.  Recall the substitution rule:
$$ \List{\Var{t}=\Var{u}; \Var{P}(\Var{t})} \Imp \Var{P}(\Var{u}) 
\eqno(subst) $$
Unsuitable for ordinary resolution because $\Var{P}(\Var{u})$ admits many
unifiers, $(subst)$ works well with elim-resolution.  It deletes some
assumption of the form $x=y$ and replaces every~$y$ by~$x$ in the subgoal
formula.  The simultaneous unification instantiates $\Var{u}$ to~$y$; if
$y$ is not an unknown, then $\Var{P}(y)$ can easily be unified with another
formula.  

In logical parlance, the premise containing the connective to be eliminated
is called the \bfindex{major premise}.  Elim-resolution expects the major
premise to come first.  The order of the premises is significant in
Isabelle.

\subsection{Destruction rules} \label{destruct}
\index{rules!destruction}\index{rules!elimination}
\index{forward proof}

Looking back to Fig.\ts\ref{fol-fig}, notice that there are two kinds of
elimination rule.  The rules $({\conj}E1)$, $({\conj}E2)$, $({\imp}E)$ and
$({\forall}E)$ extract the conclusion from the major premise.  In Isabelle
parlance, such rules are called {\bf destruction rules}; they are readable
and easy to use in forward proof.  The rules $({\disj}E)$, $({\bot}E)$ and
$({\exists}E)$ work by discharging assumptions; they support backward proof
in a style reminiscent of the sequent calculus.

The latter style is the most general form of elimination rule.  In natural
deduction, there is no way to recast $({\disj}E)$, $({\bot}E)$ or
$({\exists}E)$ as destruction rules.  But we can write general elimination
rules for $\conj$, $\imp$ and~$\forall$:
\[
\infer{R}{P\conj Q & \infer*{R}{[P,Q]}} \qquad
\infer{R}{P\imp Q & P & \infer*{R}{[Q]}}  \qquad
\infer{Q}{\forall x.P & \infer*{Q}{[P[t/x]]}} 
\]
Because they are concise, destruction rules are simpler to derive than the
corresponding elimination rules.  To facilitate their use in backward
proof, Isabelle provides a means of transforming a destruction rule such as
\[ \infer[\quad\hbox{to the elimination rule}\quad]{Q}{P@1 & \ldots & P@m} 
   \infer{R.}{P@1 & \ldots & P@m & \infer*{R}{[Q]}} 
\]
{\bf Destruct-resolution}\index{destruct-resolution} combines this
transformation with elim-resolution.  It applies a destruction rule to some
assumption of a subgoal.  Given the rule above, it replaces the
assumption~$P@1$ by~$Q$, with new subgoals of showing instances of $P@2$,
\ldots,~$P@m$.  Destruct-resolution works forward from a subgoal's
assumptions.  Ordinary resolution performs forward reasoning from theorems,
as illustrated in~\S\ref{joining}.


\subsection{Deriving rules by resolution}  \label{deriving}
\index{rules!derived|bold}\index{meta-assumptions!syntax of}
The meta-logic, itself a form of the predicate calculus, is defined by a
system of natural deduction rules.  Each theorem may depend upon
meta-assumptions.  The theorem that~$\phi$ follows from the assumptions
$\phi@1$, \ldots, $\phi@n$ is written
\[ \phi \quad [\phi@1,\ldots,\phi@n]. \]
A more conventional notation might be $\phi@1,\ldots,\phi@n \turn \phi$,
but Isabelle's notation is more readable with large formulae.

Meta-level natural deduction provides a convenient mechanism for deriving
new object-level rules.  To derive the rule
\[ \infer{\phi,}{\theta@1 & \ldots & \theta@k} \]
assume the premises $\theta@1$,~\ldots,~$\theta@k$ at the
meta-level.  Then prove $\phi$, possibly using these assumptions.
Starting with a proof state $\phi\Imp\phi$, assumptions may accumulate,
reaching a final state such as
\[ \phi \quad [\theta@1,\ldots,\theta@k]. \]
The meta-rule for $\Imp$ introduction discharges an assumption.
Discharging them in the order $\theta@k,\ldots,\theta@1$ yields the
meta-theorem $\List{\theta@1; \ldots; \theta@k} \Imp \phi$, with no
assumptions.  This represents the desired rule.
Let us derive the general $\conj$ elimination rule:
$$ \infer{R}{P\conj Q & \infer*{R}{[P,Q]}}  \eqno(\conj E) $$
We assume $P\conj Q$ and $\List{P;Q}\Imp R$, and commence backward proof in
the state $R\Imp R$.  Resolving this with the second assumption yields the
state 
\[ \phantom{\List{P\conj Q;\; P\conj Q}}
   \llap{$\List{P;Q}$}\Imp R \quad [\,\List{P;Q}\Imp R\,]. \]
Resolving subgoals~1 and~2 with~$({\conj}E1)$ and~$({\conj}E2)$,
respectively, yields the state
\[ \List{P\conj \Var{Q@1};\; \Var{P@2}\conj Q}\Imp R 
   \quad [\,\List{P;Q}\Imp R\,]. 
\]
The unknowns $\Var{Q@1}$ and~$\Var{P@2}$ arise from unconstrained
subformulae in the premises of~$({\conj}E1)$ and~$({\conj}E2)$.  Resolving
both subgoals with the assumption $P\conj Q$ instantiates the unknowns to yield
\[ R \quad [\, \List{P;Q}\Imp R, P\conj Q \,]. \]
The proof may use the meta-assumptions in any order, and as often as
necessary; when finished, we discharge them in the correct order to
obtain the desired form:
\[ \List{P\conj Q;\; \List{P;Q}\Imp R} \Imp R \]
We have derived the rule using free variables, which prevents their
premature instantiation during the proof; we may now replace them by
schematic variables.

\begin{warn}
  Schematic variables are not allowed in meta-assumptions, for a variety of
  reasons.  Meta-assumptions remain fixed throughout a proof.
\end{warn}

