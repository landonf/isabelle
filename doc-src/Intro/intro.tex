\documentstyle[a4,12pt]{article}
\makeatletter
%       proof.sty       (Proof Figure Macros)
%
%       version 2.0
%       June 24, 1991
%       Copyright (C) 1990,1991 Makoto Tatsuta (tatsuta@riec.tohoku.ac.jp)
% 
% This program is free software; you can redistribute it or modify
% it under the terms of the GNU General Public License as published by
% the Free Software Foundation; either versions 1, or (at your option)
% any later version.
% 
% This program is distributed in the hope that it will be useful
% but WITHOUT ANY WARRANTY; without even the implied warranty of
% MERCHANTABILITY or FITNESS FOR A PARTICULAR PURPOSE.  See the
% GNU General Public License for more details.
%
%       Usage:
%               In \documentstyle, specify an optional style `proof', say,
%                       \documentstyle[proof]{article}.
%
%       The following macros are available:
%
%       In all the following macros, all the arguments such as
%       <Lowers> and <Uppers> are processed in math mode.
%
%       \infer<Lower><Uppers>
%               draws an inference.
%
%               Use & in <Uppers> to delimit upper formulae.
%               <Uppers> consists more than 0 formulae.
%
%               \infer returns \hbox{ ... } or \vbox{ ... } and
%               sets \@LeftOffset and \@RightOffset globally.
%
%       \infer[<Label>]<Lower><Uppers>
%               draws an inference labeled with <Label>.
%
%       \infer*<Lower><Uppers>
%               draws a many step deduction.
%
%       \infer*[<Label>]<Lower><Uppers>
%               draws a many step deduction labeled with <Label>.
%
%       \infer=<Lower><Uppers>
%               draws a double-ruled deduction.
%
%       \infer=[<Label>]<Lower><Uppers>
%               draws a double-ruled deduction labeled with <Label>.
%
%       \deduce<Lower><Uppers>
%               draws an inference without a rule.
%
%       \deduce[<Proof>]<Lower><Uppers>
%               draws a many step deduction with a proof name.
%
%       Example:
%               If you want to write
%                           B C
%                          -----
%                      A     D
%                     ----------
%                         E
%       use
%               \infer{E}{
%                       A
%                       &
%                       \infer{D}{B & C}
%               }
%

%       Style Parameters

\newdimen\inferLineSkip         \inferLineSkip=2pt
\newdimen\inferLabelSkip        \inferLabelSkip=5pt
\def\inferTabSkip{\quad}

%       Variables

\newdimen\@LeftOffset   % global
\newdimen\@RightOffset  % global
\newdimen\@SavedLeftOffset      % safe from users

\newdimen\UpperWidth
\newdimen\LowerWidth
\newdimen\LowerHeight
\newdimen\UpperLeftOffset
\newdimen\UpperRightOffset
\newdimen\UpperCenter
\newdimen\LowerCenter
\newdimen\UpperAdjust
\newdimen\RuleAdjust
\newdimen\LowerAdjust
\newdimen\RuleWidth
\newdimen\HLabelAdjust
\newdimen\VLabelAdjust
\newdimen\WidthAdjust

\newbox\@UpperPart
\newbox\@LowerPart
\newbox\@LabelPart
\newbox\ResultBox

%       Flags

\newif\if@inferRule     % whether \@infer draws a rule.
\newif\if@DoubleRule    % whether \@infer draws doulbe rules.
\newif\if@ReturnLeftOffset      % whether \@infer returns \@LeftOffset.
\newif\if@MathSaved     % whether inner math mode where \infer or
                        % \deduce appears.

%       Special Fonts

\def\DeduceSym{\vtop{\baselineskip4\p@ \lineskiplimit\z@
    \vbox{\hbox{.}\hbox{.}\hbox{.}}\hbox{.}}}

%       Math Save Macros
%
%       \@SaveMath is called in the very begining of toplevel macros
%       which are \infer and \deduce.
%       \@RestoreMath is called in the very last before toplevel macros end.
%       Remark \infer and \deduce ends calling \@infer.

\def\@SaveMath{\@MathSavedfalse \ifmmode \ifinner
        \relax $\relax \@MathSavedtrue \fi\fi }

\def\@RestoreMath{\if@MathSaved \relax $\relax\fi }

%       Macros

\def\@Ifnextchar#1#2#3{\let\@tempe=#1\def\@tempa{#2}\def\@tempb{#3}\futurelet
    \@tempc\@ifnch}
        % It can compare an = character, though the one in latex.tex cannot.

\def\@ifEmpty#1#2#3{\def\@tempa{\@empty}\def\@tempb{#1}\relax
        \ifx \@tempa \@tempb #2\else #3\fi }

\def\infer{\@SaveMath \@Ifnextchar *{\@inferSteps}{\relax
        \@Ifnextchar ={\@inferDoubleRule}{\@inferOneStep}}}

\def\@inferOneStep{\@inferRuletrue \@DoubleRulefalse
        \@Ifnextchar [{\@infer}{\@infer[\@empty]}}

\def\@inferDoubleRule={\@inferRuletrue \@DoubleRuletrue
        \@Ifnextchar [{\@infer}{\@infer[\@empty]}}

\def\@inferSteps*{\@Ifnextchar [{\@@inferSteps}{\@@inferSteps[\@empty]}}

\def\@@inferSteps[#1]{\@deduce{#1}[\DeduceSym]}

\def\deduce{\@SaveMath \@Ifnextchar [{\@deduce{\@empty}}
        {\@inferRulefalse \@infer[\@empty]}}

%       \@deduce<Proof Label>[<Proof>]<Lower><Uppers>

\def\@deduce#1[#2]#3#4{\@inferRulefalse
        \@infer[\@empty]{#3}{\@SaveMath \@infer[{#1}]{#2}{#4}}}

%       \@infer[<Label>]<Lower><Uppers>
%               If \@inferRuletrue, it draws a rule and <Label> is right to
%               a rule. In this case, if \@DoubleRuletrue, it draws
%               double rules.
%
%               Otherwise, draws no rule and <Label> is right to <Lower>.

\def\@infer[#1]#2#3{\relax
% Get parameters
        \if@ReturnLeftOffset \else \@SavedLeftOffset=\@LeftOffset \fi
        \setbox\@LabelPart=\hbox{$#1$}\relax
        \setbox\@LowerPart=\hbox{$#2$}\relax
%
        \global\@LeftOffset=0pt
        \setbox\@UpperPart=\vbox{\tabskip=0pt \halign{\relax
                \global\@RightOffset=0pt \@ReturnLeftOffsettrue $##$&&
                \inferTabSkip
                \global\@RightOffset=0pt \@ReturnLeftOffsetfalse $##$\cr
                #3\cr}}\relax
%                       Here is a little trick.
%                       \@ReturnLeftOffsettrue(false) influences on \infer or
%                       \deduce placed in ## locally
%                       because of \@SaveMath and \@RestoreMath.
        \UpperLeftOffset=\@LeftOffset
        \UpperRightOffset=\@RightOffset
% Calculate Adjustments
        \LowerWidth=\wd\@LowerPart
        \LowerHeight=\ht\@LowerPart
        \LowerCenter=0.5\LowerWidth
%
        \UpperWidth=\wd\@UpperPart \advance\UpperWidth by -\UpperLeftOffset
        \advance\UpperWidth by -\UpperRightOffset
        \UpperCenter=\UpperLeftOffset
        \advance\UpperCenter by 0.5\UpperWidth
%
        \ifdim \UpperWidth > \LowerWidth
                % \UpperCenter > \LowerCenter
        \UpperAdjust=0pt
        \RuleAdjust=\UpperLeftOffset
        \LowerAdjust=\UpperCenter \advance\LowerAdjust by -\LowerCenter
        \RuleWidth=\UpperWidth
        \global\@LeftOffset=\LowerAdjust
%
        \else   % \UpperWidth <= \LowerWidth
        \ifdim \UpperCenter > \LowerCenter
%
        \UpperAdjust=0pt
        \RuleAdjust=\UpperCenter \advance\RuleAdjust by -\LowerCenter
        \LowerAdjust=\RuleAdjust
        \RuleWidth=\LowerWidth
        \global\@LeftOffset=\LowerAdjust
%
        \else   % \UpperWidth <= \LowerWidth
                % \UpperCenter <= \LowerCenter
%
        \UpperAdjust=\LowerCenter \advance\UpperAdjust by -\UpperCenter
        \RuleAdjust=0pt
        \LowerAdjust=0pt
        \RuleWidth=\LowerWidth
        \global\@LeftOffset=0pt
%
        \fi\fi
% Make a box
        \if@inferRule
%
        \setbox\ResultBox=\vbox{
                \moveright \UpperAdjust \box\@UpperPart
                \nointerlineskip \kern\inferLineSkip
                \if@DoubleRule
                \moveright \RuleAdjust \vbox{\hrule width\RuleWidth
                        \kern 1pt\hrule width\RuleWidth}\relax
                \else
                \moveright \RuleAdjust \vbox{\hrule width\RuleWidth}\relax
                \fi
                \nointerlineskip \kern\inferLineSkip
                \moveright \LowerAdjust \box\@LowerPart }\relax
%
        \@ifEmpty{#1}{}{\relax
%
        \HLabelAdjust=\wd\ResultBox     \advance\HLabelAdjust by -\RuleAdjust
        \advance\HLabelAdjust by -\RuleWidth
        \WidthAdjust=\HLabelAdjust
        \advance\WidthAdjust by -\inferLabelSkip
        \advance\WidthAdjust by -\wd\@LabelPart
        \ifdim \WidthAdjust < 0pt \WidthAdjust=0pt \fi
%
        \VLabelAdjust=\dp\@LabelPart
        \advance\VLabelAdjust by -\ht\@LabelPart
        \VLabelAdjust=0.5\VLabelAdjust  \advance\VLabelAdjust by \LowerHeight
        \advance\VLabelAdjust by \inferLineSkip
%
        \setbox\ResultBox=\hbox{\box\ResultBox
                \kern -\HLabelAdjust \kern\inferLabelSkip
                \raise\VLabelAdjust \box\@LabelPart \kern\WidthAdjust}\relax
%
        }\relax % end @ifEmpty
%
        \else % \@inferRulefalse
%
        \setbox\ResultBox=\vbox{
                \moveright \UpperAdjust \box\@UpperPart
                \nointerlineskip \kern\inferLineSkip
                \moveright \LowerAdjust \hbox{\unhbox\@LowerPart
                        \@ifEmpty{#1}{}{\relax
                        \kern\inferLabelSkip \unhbox\@LabelPart}}}\relax
        \fi
%
        \global\@RightOffset=\wd\ResultBox
        \global\advance\@RightOffset by -\@LeftOffset
        \global\advance\@RightOffset by -\LowerWidth
        \if@ReturnLeftOffset \else \global\@LeftOffset=\@SavedLeftOffset \fi
%
        \box\ResultBox
        \@RestoreMath
}

\typeout{Isabelle Manual Page Layout}

% iman.sty 
%
\typeout{Document Style iman. Released 15 September 1992}

\hyphenation{Isa-belle}

%%%INDEXING  use sedindex to process the index
%index, putting page numbers of definitions in boldface
\newcommand\bold[1]{{\bf#1}}
\newcommand\indexbold[1]{\index{#1|bold}}

%for cross-references: 2nd argument (page number) is ignored
\newcommand\see[2]{{\it see \/}{#1}}
\newcommand\seealso[2]{{\it see also \/}{#1}}

%set argument in \tt font; at the sime time, index using * prefix
\newcommand\ttindex[1]{{\tt#1}\index{*#1}\@}
\newcommand\ttindexbold[1]{{\tt#1}\index{*#1|bold}\@}

%set argument in \bf font and index in ROMAN font (for definitions in text!)
\newcommand\bfindex[1]{{\bf#1}\index{#1|bold}\@}


%%Euro-style date: 20 September 1955
\def\today{\number\day\space\ifcase\month\or
January\or February\or March\or April\or May\or June\or
July\or August\or September\or October\or November\or December\fi
\space\number\year}

%%% underscores as ordinary characters, not for subscripting
%%  use @ or \sb for subscripting; use \at for @
%%  only works in \tt font
%%  must not make _ an active char; would make \ttindex fail!
\gdef\underscoreoff{\catcode`\@=8\catcode`\_=\other}
\gdef\underscoreon{\catcode`\_=8\makeatother}
\chardef\other=12
\chardef\at=`\@

% alternative underscore
\def\_{\leavevmode\kern.06em\vbox{\hrule height.2ex width.3em}\hskip0.1em}

%%% \dquotes permits usage of "..." for \hbox{...} -- also taken from under.sty
{\catcode`\"=\active
\gdef\dquotes{\catcode`\"=\active  \let"=\@mathText}%
\gdef\@mathText#1"{\hbox{\mathTextFont #1\/}}}
\def\mathTextFont{\frenchspacing\tt}

%%%% meta-logical connectives

\let\Forall=\bigwedge
\let\Imp=\Longrightarrow
\let\To=\Rightarrow
\newcommand\Var[1]{{?\!#1}}

%%%% ``WARNING'' environment
\def\dbend{\vtop to 0pt{\vss\hbox{\Huge\bf!}\vss}}
\newenvironment{warn}{\medskip\medbreak\begingroup \clubpenalty=10000 
  	 \baselineskip=0.9\baselineskip
	 \noindent \hangindent\parindent \hangafter=-2 
  	 \hbox to0pt{\hskip-\hangindent\dbend\hfill}\ignorespaces}%
	{\par\endgroup\medbreak}


%%%% Standard logical symbols
\let\turn=\vdash
\let\conj=\wedge
\let\disj=\vee
\let\imp=\rightarrow
\let\bimp=\leftrightarrow
\newcommand\all[1]{\forall#1.}	%quantification
\newcommand\ex[1]{\exists#1.}
\newcommand{\pair}[1]{\langle#1\rangle}

\newenvironment{example}{\begin{Example}\rm}{\end{Example}}
\newtheorem{Example}{Example}[chapter]

\newcommand\lbrakk{\mathopen{[\![}}
\newcommand\rbrakk{\mathclose{]\!]}}
\newcommand\List[1]{\lbrakk#1\rbrakk}  %was \obj
\newcommand\vpile[1]{\begin{array}{c}#1\end{array}}

\let\int=\cap
\let\un=\cup
\let\inter=\bigcap
\let\union=\bigcup

\newcommand{\rew}{\mathop{\longrightarrow}}
\newcommand{\rewer}{\mathop{\longleftrightarrow}}

\def\ML{{\sc ml}}
\def\OBJ{{\sc obj}}

\def\LCF{{\tt LCF}\@}
\def\FOL{{\tt FOL}\@}
\def\HOL{{\tt HOL}\@}
\def\LK{{\tt LK}\@}
\def\ZF{{\tt ZF}\@}
\def\CTT{{\tt CTT}\@}
\def\Cube{{\tt Cube}}
\def\Modal{{\tt Modal}}

%macros to change the treatment of symbols
\def\relsemicolon{\mathcode`\;="303B}   %treat ; like a relation
\def\binperiod{\mathcode`\.="213A}   %treat . like a binary operator
\def\binvert{\mathcode`\|="226A}     %treat | like a binary operator

%redefinition of \sloppy and \fussy to use \emergencystretch
\def\sloppy{\tolerance2000 \hfuzz.5pt \vfuzz.5pt \emergencystretch=15pt}
\def\fussy{\tolerance200 \hfuzz.1pt \vfuzz.1pt \emergencystretch=0pt}

\chardef\ttilde=`\~  	% A tilde for \tt font
\chardef\ttback=`\\  	% A backslash for \tt font
\chardef\ttlbrace=`\{ 	% A left brace for \tt font
\chardef\ttrbrace=`\} 	% A right brace for \tt font

\newfont{\sltt}{cmsltt10}     %% for output from terminal sessions
\newcommand\out{\ \sltt}

%Indented, boxed alltt environment; uses \small\tt font
%\leftmargini is LaTeX's first-level indentation for items (2.5em)
%@endparenv is LaTeX's trick for preventing indentation of next paragraph
\newenvironment{ttbox}{\par\nobreak\vskip-2pt
	   \vbox\bgroup \small\begin{alltt} \leftskip\leftmargini}%
	{\end{alltt}\egroup\vskip-7pt\@endparenv}
\newcommand\ttbreak{\end{ttbox}\vskip-10pt\begin{ttbox}}


%%%Put first chapter on odd page, with arabic numbering; like \cleardoublepage
\newcommand\clearfirst{\clearpage\ifodd\c@page\else
    \hbox{}\newpage\if@twocolumn\hbox{}\newpage\fi\fi
    \pagenumbering{arabic}}

%%%Ruled chapter headings 
\def\@rulehead#1{\hrule height1pt \vskip 14pt \Huge \bf 
   #1 \vskip 14pt\hrule height1pt}
\def\@makechapterhead#1{ { \parindent 0pt 
 \ifnum\c@secnumdepth >\m@ne \raggedleft\large\bf\@chapapp{} \thechapter \par 
 \vskip 20pt \fi \raggedright \@rulehead{#1} \par \nobreak \vskip 40pt } }

\def\@makeschapterhead#1{ { \parindent 0pt \raggedright 
 \@rulehead{#1} \par \nobreak \vskip 40pt } }

% "itmath.sty" use cmr italic for letters in math mode and get the
%	       usual letter spacing of text mode.
%
% Michael Lawley, April 1993
% (lawley@cit.gu.edu.au)
%
% Derived from itma.sty (of unknown origin).
%
% MATHCODES
%
% The mathcodes for the letters A, ..., Z, a, ..., z are changed to
% generate text italic rather than math italic by default. This makes
% multi-letter identifiers look better. The mathcode for character c
% is set to "7000 (variable class) + "400 (text italic) + c.
%
% For NFSS the mathcode is "7000 (variable class) + (hex)\itfam + c
% \itfam is probably equal to 7.
%

\ifx\undefined\hexnumber@
  \def\hexnumber@#1{\ifcase#1 \z@
  \or \@ne \or \tw@ \or \thr@@
  \or 4\or 5\or 6\or 7\or 8\or
  9\or A\or B\or C\or D\or E\or F\fi}
\fi

\def\@setmcodes#1#2#3{{\count0=#1 \count1=#3
        \loop \global\mathcode\count0=\count1 \ifnum \count0<#2
        \advance\count0 by1 \advance\count1 by1 \repeat}}

\edef\@tempa{\hexnumber@\itfam}

\@setmcodes{`A}{`Z}{"7\@tempa 41}
\@setmcodes{`a}{`z}{"7\@tempa 61}

\ifx\define@mathgroup\undefined\else
	\define@mathgroup\mv@normal{\itfam}{cmr}{m}{it}\fi

% extra.sty : Isabelle Manual extra macros for non-Springer version
%
\typeout{Document Style extra. Released 17/2/94, version of 22/8/00}

\usepackage{ttbox}
{\obeylines\gdef\ttbreak
{\allowbreak}}

%%Euro-style date: 20 September 1955
\def\today{\number\day\space\ifcase\month\or
January\or February\or March\or April\or May\or June\or
July\or August\or September\or October\or November\or December\fi
\space\number\year}

%%%Put first chapter on odd page, with arabic numbering; like \cleardoublepage
\newcommand\clearfirst{\clearpage\ifodd\c@page\else
    \hbox{}\newpage\if@twocolumn\hbox{}\newpage\fi\fi
    \pagenumbering{arabic}}

%%%Ruled chapter headings 
\def\@rulehead#1{\hrule height1pt \vskip 14pt \Huge \bf 
   #1 \vskip 14pt\hrule height1pt}
\def\@makechapterhead#1{ { \parindent 0pt 
 \ifnum\c@secnumdepth >\m@ne \raggedleft\large\bf\@chapapp{} \thechapter \par 
 \vskip 20pt \fi \raggedright \@rulehead{#1} \par \nobreak \vskip 40pt } }

\def\@makeschapterhead#1{ { \parindent 0pt \raggedright 
 \@rulehead{#1} \par \nobreak \vskip 40pt } }


\makeatother

%% $Id$
%% run    bibtex intro         to prepare bibliography
%% run    ../sedindex intro    to prepare index file
%prth *(\(.*\));          \1;      
%{\\out \(.*\)}          {\\out val it = "\1" : thm}

\title{Introduction to Isabelle}   
\author{{\em Lawrence C. Paulson}\\
        Computer Laboratory \\ University of Cambridge \\[2ex]
        {\small{\em Electronic mail\/}: {\tt lcp@cl.cam.ac.uk}}
}
\date{} 
\makeindex

\underscoreoff

\setcounter{secnumdepth}{2} \setcounter{tocdepth}{2}

\sloppy
\binperiod     %%%treat . like a binary operator

\newcommand\qeq{\stackrel{?}{\equiv}}  %for disagreement pairs in unification
\newcommand{\nand}{\mathbin{\lnot\&}} 
\newcommand{\xor}{\mathbin{\#}}

\pagenumbering{roman} 
\begin{document}
\pagestyle{empty}
\begin{titlepage}
\maketitle 
\thispagestyle{empty}
\vfill
{\small Copyright \copyright{} \number\year{} by Lawrence C. Paulson}
\end{titlepage}

\pagestyle{headings}
\part*{Preface}
\index{Isabelle!overview} \index{Isabelle!object-logics supported}
Isabelle~\cite{paulson-natural,paulson-found,paulson700} is a generic theorem
prover.  It has been instantiated to support reasoning in several
object-logics:
\begin{itemize}
\item first-order logic, constructive and classical versions
\item higher-order logic, similar to that of Gordon's {\sc
hol}~\cite{mgordon-hol}
\item Zermelo-Fraenkel set theory~\cite{suppes72}
\item an extensional version of Martin-L\"of's Type Theory~\cite{nordstrom90}
\item the classical first-order sequent calculus, {\sc lk}
\item the modal logics $T$, $S4$, and $S43$
\item the Logic for Computable Functions~\cite{paulson87}
\end{itemize}
A logic's syntax and inference rules are specified declaratively; this
allows single-step proof construction.  Isabelle provides control
structures for expressing search procedures.  Isabelle also provides
several generic tools, such as simplifiers and classical theorem provers,
which can be applied to object-logics.

\index{ML}
Isabelle is a large system, but beginners can get by with a small
repertoire of commands and a basic knowledge of how Isabelle works.  Some
knowledge of Standard~\ML{} is essential, because \ML{} is Isabelle's user
interface.  Advanced Isabelle theorem proving can involve writing \ML{}
code, possibly with Isabelle's sources at hand.  My book
on~\ML{}~\cite{paulson91} covers much material connected with Isabelle,
including a simple theorem prover.  Users must be familiar with logic as
used in computer science; there are many good
texts~\cite{galton90,reeves90}.

\index{LCF}
{\sc lcf}, developed by Robin Milner and colleagues~\cite{mgordon79}, is an
ancestor of {\sc hol}, Nuprl, and several other systems.  Isabelle borrows
ideas from {\sc lcf}: formulae are~\ML{} values; theorems belong to an
abstract type; tactics and tacticals support backward proof.  But {\sc lcf}
represents object-level rules by functions, while Isabelle represents them
by terms.  You may find my other writings~\cite{paulson87,paulson-handbook}
helpful in understanding the relationship between {\sc lcf} and Isabelle.

\index{Isabelle!release history} Isabelle was first distributed in 1986.
The 1987 version introduced a higher-order meta-logic with an improved
treatment of quantifiers.  The 1988 version added limited polymorphism and
support for natural deduction.  The 1989 version included a parser and
pretty printer generator.  The 1992 version introduced type classes, to
support many-sorted and higher-order logics.  The current version provides
greater support for theories and is much faster.  Isabelle is still under
development and will continue to change.

\subsubsection*{Overview} 
This manual consists of three parts.  Part~I discusses the Isabelle's
foundations.  Part~II, presents simple on-line sessions, starting with
forward proof.  It also covers basic tactics and tacticals, and some
commands for invoking them.  Part~III contains further examples for users
with a bit of experience.  It explains how to derive rules define theories,
and concludes with an extended example: a Prolog interpreter.

Isabelle's Reference Manual and Object-Logics manual contain more details.
They assume familiarity with the concepts presented here.


\subsubsection*{Acknowledgements} 
Tobias Nipkow contributed most of the section on defining theories.
Stefan Berghofer, Sara Kalvala and Markus Wenzel suggested improvements.

Tobias Nipkow has made immense contributions to Isabelle, including the
parser generator, type classes, and the simplifier.  Carsten Clasohm and
Markus Wenzel made major contributions; Sonia Mahjoub and Karin Nimmermann
also helped.  Isabelle was developed using Dave Matthews's Standard~{\sc
  ml} compiler, Poly/{\sc ml}.  Many people have contributed to Isabelle's
standard object-logics, including Martin Coen, Philippe de Groote, Philippe
No\"el.  The research has been funded by the SERC (grants GR/G53279,
GR/H40570) and by ESPRIT (projects 3245: Logical Frameworks, and 6453:
Types).

\newpage
\pagestyle{plain} \tableofcontents 
\newpage

\newfont{\sanssi}{cmssi12}
\vspace*{2.5cm}
\begin{quote}
\raggedleft
{\sanssi
You can only find truth with logic\\
if you have already found truth without it.}\\
\bigskip

G.K. Chesterton, {\em The Man who was Orthodox}
\end{quote}

\clearfirst  \pagestyle{headings}
\part{Foundations} 
The following sections discuss Isabelle's logical foundations in detail:
representing logical syntax in the typed $\lambda$-calculus; expressing
inference rules in Isabelle's meta-logic; combining rules by resolution.

If you wish to use Isabelle immediately, please turn to
page~\pageref{chap:getting}.  You can always read about foundations later,
either by returning to this point or by looking up particular items in the
index.

\begin{figure} 
\begin{eqnarray*}
  \neg P   & \hbox{abbreviates} & P\imp\bot \\
  P\bimp Q & \hbox{abbreviates} & (P\imp Q) \conj (Q\imp P)
\end{eqnarray*}
\vskip 4ex

\(\begin{array}{c@{\qquad\qquad}c}
  \infer[({\conj}I)]{P\conj Q}{P & Q}  &
  \infer[({\conj}E1)]{P}{P\conj Q} \qquad 
  \infer[({\conj}E2)]{Q}{P\conj Q} \\[4ex]

  \infer[({\disj}I1)]{P\disj Q}{P} \qquad 
  \infer[({\disj}I2)]{P\disj Q}{Q} &
  \infer[({\disj}E)]{R}{P\disj Q & \infer*{R}{[P]} & \infer*{R}{[Q]}}\\[4ex]

  \infer[({\imp}I)]{P\imp Q}{\infer*{Q}{[P]}} &
  \infer[({\imp}E)]{Q}{P\imp Q & P}  \\[4ex]

  &
  \infer[({\bot}E)]{P}{\bot}\\[4ex]

  \infer[({\forall}I)*]{\forall x.P}{P} &
  \infer[({\forall}E)]{P[t/x]}{\forall x.P} \\[3ex]

  \infer[({\exists}I)]{\exists x.P}{P[t/x]} &
  \infer[({\exists}E)*]{Q}{{\exists x.P} & \infer*{Q}{[P]} } \\[3ex]

  {t=t} \,(refl)   &  \vcenter{\infer[(subst)]{P[u/x]}{t=u & P[t/x]}} 
\end{array} \)

\bigskip\bigskip
*{\em Eigenvariable conditions\/}:

$\forall I$: provided $x$ is not free in the assumptions

$\exists E$: provided $x$ is not free in $Q$ or any assumption except $P$
\caption{Intuitionistic first-order logic} \label{fol-fig}
\end{figure}

\section{Formalizing logical syntax in Isabelle}\label{sec:logical-syntax}
\index{first-order logic}

Figure~\ref{fol-fig} presents intuitionistic first-order logic,
including equality.  Let us see how to formalize
this logic in Isabelle, illustrating the main features of Isabelle's
polymorphic meta-logic.

\index{lambda calc@$\lambda$-calculus} 
Isabelle represents syntax using the simply typed $\lambda$-calculus.  We
declare a type for each syntactic category of the logic.  We declare a
constant for each symbol of the logic, giving each $n$-place operation an
$n$-argument curried function type.  Most importantly,
$\lambda$-abstraction represents variable binding in quantifiers.

\index{types!syntax of}\index{types!function}\index{*fun type} 
\index{type constructors}
Isabelle has \ML-style polymorphic types such as~$(\alpha)list$, where
$list$ is a type constructor and $\alpha$ is a type variable; for example,
$(bool)list$ is the type of lists of booleans.  Function types have the
form $(\sigma,\tau)fun$ or $\sigma\To\tau$, where $\sigma$ and $\tau$ are
types.  Curried function types may be abbreviated:
\[  \sigma@1\To (\cdots \sigma@n\To \tau\cdots)  \quad \hbox{as} \quad
[\sigma@1, \ldots, \sigma@n] \To \tau \]
 
\index{terms!syntax of} The syntax for terms is summarised below.
Note that there are two versions of function application syntax
available in Isabelle: either $t\,u$, which is the usual form for
higher-order languages, or $t(u)$, trying to look more like
first-order.  The latter syntax is used throughout the manual.
\[ 
\index{lambda abs@$\lambda$-abstractions}\index{function applications}
\begin{array}{ll}
  t :: \tau   & \hbox{type constraint, on a term or bound variable} \\
  \lambda x.t   & \hbox{abstraction} \\
  \lambda x@1\ldots x@n.t
        & \hbox{curried abstraction, $\lambda x@1. \ldots \lambda x@n.t$} \\
  t(u)          & \hbox{application} \\
  t (u@1, \ldots, u@n) & \hbox{curried application, $t(u@1)\ldots(u@n)$} 
\end{array}
\]


\subsection{Simple types and constants}\index{types!simple|bold} 

The syntactic categories of our logic (Fig.\ts\ref{fol-fig}) are {\bf
  formulae} and {\bf terms}.  Formulae denote truth values, so (following
tradition) let us call their type~$o$.  To allow~0 and~$Suc(t)$ as terms,
let us declare a type~$nat$ of natural numbers.  Later, we shall see
how to admit terms of other types.

\index{constants}\index{*nat type}\index{*o type}
After declaring the types~$o$ and~$nat$, we may declare constants for the
symbols of our logic.  Since $\bot$ denotes a truth value (falsity) and 0
denotes a number, we put \begin{eqnarray*}
  \bot  & :: & o \\
  0     & :: & nat.
\end{eqnarray*}
If a symbol requires operands, the corresponding constant must have a
function type.  In our logic, the successor function
($Suc$) is from natural numbers to natural numbers, negation ($\neg$) is a
function from truth values to truth values, and the binary connectives are
curried functions taking two truth values as arguments: 
\begin{eqnarray*}
  Suc    & :: & nat\To nat  \\
  {\neg} & :: & o\To o      \\
  \conj,\disj,\imp,\bimp  & :: & [o,o]\To o 
\end{eqnarray*}
The binary connectives can be declared as infixes, with appropriate
precedences, so that we write $P\conj Q\disj R$ instead of
$\disj(\conj(P,Q), R)$.

Section~\ref{sec:defining-theories} below describes the syntax of Isabelle
theory files and illustrates it by extending our logic with mathematical
induction.


\subsection{Polymorphic types and constants} \label{polymorphic}
\index{types!polymorphic|bold}
\index{equality!polymorphic}
\index{constants!polymorphic}

Which type should we assign to the equality symbol?  If we tried
$[nat,nat]\To o$, then equality would be restricted to the natural
numbers; we should have to declare different equality symbols for each
type.  Isabelle's type system is polymorphic, so we could declare
\begin{eqnarray*}
  {=}  & :: & [\alpha,\alpha]\To o,
\end{eqnarray*}
where the type variable~$\alpha$ ranges over all types.
But this is also wrong.  The declaration is too polymorphic; $\alpha$
includes types like~$o$ and $nat\To nat$.  Thus, it admits
$\bot=\neg(\bot)$ and $Suc=Suc$ as formulae, which is acceptable in
higher-order logic but not in first-order logic.

Isabelle's {\bf type classes}\index{classes} control
polymorphism~\cite{nipkow-prehofer}.  Each type variable belongs to a
class, which denotes a set of types.  Classes are partially ordered by the
subclass relation, which is essentially the subset relation on the sets of
types.  They closely resemble the classes of the functional language
Haskell~\cite{haskell-tutorial,haskell-report}.

\index{*logic class}\index{*term class}
Isabelle provides the built-in class $logic$, which consists of the logical
types: the ones we want to reason about.  Let us declare a class $term$, to
consist of all legal types of terms in our logic.  The subclass structure
is now $term\le logic$.

\index{*nat type}
We put $nat$ in class $term$ by declaring $nat{::}term$.  We declare the
equality constant by
\begin{eqnarray*}
  {=}  & :: & [\alpha{::}term,\alpha]\To o 
\end{eqnarray*}
where $\alpha{::}term$ constrains the type variable~$\alpha$ to class
$term$.  Such type variables resemble Standard~\ML's equality type
variables.

We give~$o$ and function types the class $logic$ rather than~$term$, since
they are not legal types for terms.  We may introduce new types of class
$term$ --- for instance, type $string$ or $real$ --- at any time.  We can
even declare type constructors such as~$list$, and state that type
$(\tau)list$ belongs to class~$term$ provided $\tau$ does; equality
applies to lists of natural numbers but not to lists of formulae.  We may
summarize this paragraph by a set of {\bf arity declarations} for type
constructors:\index{arities!declaring}
\begin{eqnarray*}\index{*o type}\index{*fun type}
  o             & :: & logic \\
  fun           & :: & (logic,logic)logic \\
  nat, string, real     & :: & term \\
  list          & :: & (term)term
\end{eqnarray*}
(Recall that $fun$ is the type constructor for function types.)
In \rmindex{higher-order logic}, equality does apply to truth values and
functions;  this requires the arity declarations ${o::term}$
and ${fun::(term,term)term}$.  The class system can also handle
overloading.\index{overloading|bold} We could declare $arith$ to be the
subclass of $term$ consisting of the `arithmetic' types, such as~$nat$.
Then we could declare the operators
\begin{eqnarray*}
  {+},{-},{\times},{/}  & :: & [\alpha{::}arith,\alpha]\To \alpha
\end{eqnarray*}
If we declare new types $real$ and $complex$ of class $arith$, then we
in effect have three sets of operators:
\begin{eqnarray*}
  {+},{-},{\times},{/}  & :: & [nat,nat]\To nat \\
  {+},{-},{\times},{/}  & :: & [real,real]\To real \\
  {+},{-},{\times},{/}  & :: & [complex,complex]\To complex 
\end{eqnarray*}
Isabelle will regard these as distinct constants, each of which can be defined
separately.  We could even introduce the type $(\alpha)vector$ and declare
its arity as $(arith)arith$.  Then we could declare the constant
\begin{eqnarray*}
  {+}  & :: & [(\alpha)vector,(\alpha)vector]\To (\alpha)vector 
\end{eqnarray*}
and specify it in terms of ${+} :: [\alpha,\alpha]\To \alpha$.

A type variable may belong to any finite number of classes.  Suppose that
we had declared yet another class $ord \le term$, the class of all
`ordered' types, and a constant
\begin{eqnarray*}
  {\le}  & :: & [\alpha{::}ord,\alpha]\To o.
\end{eqnarray*}
In this context the variable $x$ in $x \le (x+x)$ will be assigned type
$\alpha{::}\{arith,ord\}$, which means $\alpha$ belongs to both $arith$ and
$ord$.  Semantically the set $\{arith,ord\}$ should be understood as the
intersection of the sets of types represented by $arith$ and $ord$.  Such
intersections of classes are called \bfindex{sorts}.  The empty
intersection of classes, $\{\}$, contains all types and is thus the {\bf
  universal sort}.

Even with overloading, each term has a unique, most general type.  For this
to be possible, the class and type declarations must satisfy certain
technical constraints; see 
\iflabelundefined{sec:ref-defining-theories}%
           {Sect.\ Defining Theories in the {\em Reference Manual}}%
           {\S\ref{sec:ref-defining-theories}}.


\subsection{Higher types and quantifiers}
\index{types!higher|bold}\index{quantifiers}
Quantifiers are regarded as operations upon functions.  Ignoring polymorphism
for the moment, consider the formula $\forall x. P(x)$, where $x$ ranges
over type~$nat$.  This is true if $P(x)$ is true for all~$x$.  Abstracting
$P(x)$ into a function, this is the same as saying that $\lambda x.P(x)$
returns true for all arguments.  Thus, the universal quantifier can be
represented by a constant
\begin{eqnarray*}
  \forall  & :: & (nat\To o) \To o,
\end{eqnarray*}
which is essentially an infinitary truth table.  The representation of $\forall
x. P(x)$ is $\forall(\lambda x. P(x))$.  

The existential quantifier is treated
in the same way.  Other binding operators are also easily handled; for
instance, the summation operator $\Sigma@{k=i}^j f(k)$ can be represented as
$\Sigma(i,j,\lambda k.f(k))$, where
\begin{eqnarray*}
  \Sigma  & :: & [nat,nat, nat\To nat] \To nat.
\end{eqnarray*}
Quantifiers may be polymorphic.  We may define $\forall$ and~$\exists$ over
all legal types of terms, not just the natural numbers, and
allow summations over all arithmetic types:
\begin{eqnarray*}
   \forall,\exists      & :: & (\alpha{::}term\To o) \To o \\
   \Sigma               & :: & [nat,nat, nat\To \alpha{::}arith] \To \alpha
\end{eqnarray*}
Observe that the index variables still have type $nat$, while the values
being summed may belong to any arithmetic type.


\section{Formalizing logical rules in Isabelle}
\index{meta-implication|bold}
\index{meta-quantifiers|bold}
\index{meta-equality|bold}

Object-logics are formalized by extending Isabelle's
meta-logic~\cite{paulson-found}, which is intuitionistic higher-order logic.
The meta-level connectives are {\bf implication}, the {\bf universal
  quantifier}, and {\bf equality}.
\begin{itemize}
  \item The implication \(\phi\Imp \psi\) means `\(\phi\) implies
\(\psi\)', and expresses logical {\bf entailment}.  

  \item The quantification \(\Forall x.\phi\) means `\(\phi\) is true for
all $x$', and expresses {\bf generality} in rules and axiom schemes. 

\item The equality \(a\equiv b\) means `$a$ equals $b$', for expressing
  {\bf definitions} (see~\S\ref{definitions}).\index{definitions}
  Equalities left over from the unification process, so called {\bf
    flex-flex constraints},\index{flex-flex constraints} are written $a\qeq
  b$.  The two equality symbols have the same logical meaning.

\end{itemize}
The syntax of the meta-logic is formalized in the same manner
as object-logics, using the simply typed $\lambda$-calculus.  Analogous to
type~$o$ above, there is a built-in type $prop$ of meta-level truth values.
Meta-level formulae will have this type.  Type $prop$ belongs to
class~$logic$; also, $\sigma\To\tau$ belongs to $logic$ provided $\sigma$
and $\tau$ do.  Here are the types of the built-in connectives:
\begin{eqnarray*}\index{*prop type}\index{*logic class}
  \Imp     & :: & [prop,prop]\To prop \\
  \Forall  & :: & (\alpha{::}logic\To prop) \To prop \\
  {\equiv} & :: & [\alpha{::}\{\},\alpha]\To prop \\
  \qeq & :: & [\alpha{::}\{\},\alpha]\To prop
\end{eqnarray*}
The polymorphism in $\Forall$ is restricted to class~$logic$ to exclude
certain types, those used just for parsing.  The type variable
$\alpha{::}\{\}$ ranges over the universal sort.

In our formalization of first-order logic, we declared a type~$o$ of
object-level truth values, rather than using~$prop$ for this purpose.  If
we declared the object-level connectives to have types such as
${\neg}::prop\To prop$, then these connectives would be applicable to
meta-level formulae.  Keeping $prop$ and $o$ as separate types maintains
the distinction between the meta-level and the object-level.  To formalize
the inference rules, we shall need to relate the two levels; accordingly,
we declare the constant
\index{*Trueprop constant}
\begin{eqnarray*}
  Trueprop & :: & o\To prop.
\end{eqnarray*}
We may regard $Trueprop$ as a meta-level predicate, reading $Trueprop(P)$ as
`$P$ is true at the object-level.'  Put another way, $Trueprop$ is a coercion
from $o$ to $prop$.


\subsection{Expressing propositional rules}
\index{rules!propositional}
We shall illustrate the use of the meta-logic by formalizing the rules of
Fig.\ts\ref{fol-fig}.  Each object-level rule is expressed as a meta-level
axiom. 

One of the simplest rules is $(\conj E1)$.  Making
everything explicit, its formalization in the meta-logic is
$$
\Forall P\;Q. Trueprop(P\conj Q) \Imp Trueprop(P).   \eqno(\conj E1)
$$
This may look formidable, but it has an obvious reading: for all object-level
truth values $P$ and~$Q$, if $P\conj Q$ is true then so is~$P$.  The
reading is correct because the meta-logic has simple models, where
types denote sets and $\Forall$ really means `for all.'

\index{*Trueprop constant}
Isabelle adopts notational conventions to ease the writing of rules.  We may
hide the occurrences of $Trueprop$ by making it an implicit coercion.
Outer universal quantifiers may be dropped.  Finally, the nested implication
\index{meta-implication}
\[  \phi@1\Imp(\cdots \phi@n\Imp\psi\cdots) \]
may be abbreviated as $\List{\phi@1; \ldots; \phi@n} \Imp \psi$, which
formalizes a rule of $n$~premises.

Using these conventions, the conjunction rules become the following axioms.
These fully specify the properties of~$\conj$:
$$ \List{P; Q} \Imp P\conj Q                 \eqno(\conj I) $$
$$ P\conj Q \Imp P  \qquad  P\conj Q \Imp Q  \eqno(\conj E1,2) $$

\noindent
Next, consider the disjunction rules.  The discharge of assumption in
$(\disj E)$ is expressed  using $\Imp$:
\index{assumptions!discharge of}%
$$ P \Imp P\disj Q  \qquad  Q \Imp P\disj Q  \eqno(\disj I1,2) $$
$$ \List{P\disj Q; P\Imp R; Q\Imp R} \Imp R  \eqno(\disj E) $$
%
To understand this treatment of assumptions in natural
deduction, look at implication.  The rule $({\imp}I)$ is the classic
example of natural deduction: to prove that $P\imp Q$ is true, assume $P$
is true and show that $Q$ must then be true.  More concisely, if $P$
implies $Q$ (at the meta-level), then $P\imp Q$ is true (at the
object-level).  Showing the coercion explicitly, this is formalized as
\[ (Trueprop(P)\Imp Trueprop(Q)) \Imp Trueprop(P\imp Q). \]
The rule $({\imp}E)$ is straightforward; hiding $Trueprop$, the axioms to
specify $\imp$ are 
$$ (P \Imp Q)  \Imp  P\imp Q   \eqno({\imp}I) $$
$$ \List{P\imp Q; P}  \Imp Q.  \eqno({\imp}E) $$

\noindent
Finally, the intuitionistic contradiction rule is formalized as the axiom
$$ \bot \Imp P.   \eqno(\bot E) $$

\begin{warn}
Earlier versions of Isabelle, and certain
papers~\cite{paulson-found,paulson700}, use $\List{P}$ to mean $Trueprop(P)$.
\end{warn}

\subsection{Quantifier rules and substitution}
\index{quantifiers}\index{rules!quantifier}\index{substitution|bold}
\index{variables!bound}\index{lambda abs@$\lambda$-abstractions}
\index{function applications}

Isabelle expresses variable binding using $\lambda$-abstraction; for instance,
$\forall x.P$ is formalized as $\forall(\lambda x.P)$.  Recall that $F(t)$
is Isabelle's syntax for application of the function~$F$ to the argument~$t$;
it is not a meta-notation for substitution.  On the other hand, a substitution
will take place if $F$ has the form $\lambda x.P$;  Isabelle transforms
$(\lambda x.P)(t)$ to~$P[t/x]$ by $\beta$-conversion.  Thus, we can express
inference rules that involve substitution for bound variables.

\index{parameters|bold}\index{eigenvariables|see{parameters}}
A logic may attach provisos to certain of its rules, especially quantifier
rules.  We cannot hope to formalize arbitrary provisos.  Fortunately, those
typical of quantifier rules always have the same form, namely `$x$ not free in
\ldots {\it (some set of formulae)},' where $x$ is a variable (called a {\bf
parameter} or {\bf eigenvariable}) in some premise.  Isabelle treats
provisos using~$\Forall$, its inbuilt notion of `for all'.
\index{meta-quantifiers}

The purpose of the proviso `$x$ not free in \ldots' is
to ensure that the premise may not make assumptions about the value of~$x$,
and therefore holds for all~$x$.  We formalize $(\forall I)$ by
\[ \left(\Forall x. Trueprop(P(x))\right) \Imp Trueprop(\forall x.P(x)). \]
This means, `if $P(x)$ is true for all~$x$, then $\forall x.P(x)$ is true.'
The $\forall E$ rule exploits $\beta$-conversion.  Hiding $Trueprop$, the
$\forall$ axioms are
$$ \left(\Forall x. P(x)\right)  \Imp  \forall x.P(x)   \eqno(\forall I) $$
$$ (\forall x.P(x))  \Imp P(t).  \eqno(\forall E) $$

\noindent
We have defined the object-level universal quantifier~($\forall$)
using~$\Forall$.  But we do not require meta-level counterparts of all the
connectives of the object-logic!  Consider the existential quantifier: 
$$ P(t)  \Imp  \exists x.P(x)  \eqno(\exists I) $$
$$ \List{\exists x.P(x);\; \Forall x. P(x)\Imp Q} \Imp Q  \eqno(\exists E) $$
Let us verify $(\exists E)$ semantically.  Suppose that the premises
hold; since $\exists x.P(x)$ is true, we may choose an~$a$ such that $P(a)$ is
true.  Instantiating $\Forall x. P(x)\Imp Q$ with $a$ yields $P(a)\Imp Q$, and
we obtain the desired conclusion, $Q$.

The treatment of substitution deserves mention.  The rule
\[ \infer{P[u/t]}{t=u & P} \]
would be hard to formalize in Isabelle.  It calls for replacing~$t$ by $u$
throughout~$P$, which cannot be expressed using $\beta$-conversion.  Our
rule~$(subst)$ uses~$P$ as a template for substitution, inferring $P[u/x]$
from~$P[t/x]$.  When we formalize this as an axiom, the template becomes a
function variable:
$$ \List{t=u; P(t)} \Imp P(u).  \eqno(subst) $$


\subsection{Signatures and theories}
\index{signatures|bold}

A {\bf signature} contains the information necessary for type-checking,
parsing and pretty printing a term.  It specifies type classes and their
relationships, types and their arities, constants and their types, etc.  It
also contains grammar rules, specified using mixfix declarations.

Two signatures can be merged provided their specifications are compatible ---
they must not, for example, assign different types to the same constant.
Under similar conditions, a signature can be extended.  Signatures are
managed internally by Isabelle; users seldom encounter them.

\index{theories|bold} A {\bf theory} consists of a signature plus a collection
of axioms.  The Pure theory contains only the meta-logic.  Theories can be
combined provided their signatures are compatible.  A theory definition
extends an existing theory with further signature specifications --- classes,
types, constants and mixfix declarations --- plus lists of axioms and
definitions etc., expressed as strings to be parsed.  A theory can formalize a
small piece of mathematics, such as lists and their operations, or an entire
logic.  A mathematical development typically involves many theories in a
hierarchy.  For example, the Pure theory could be extended to form a theory
for Fig.\ts\ref{fol-fig}; this could be extended in two separate ways to form
a theory for natural numbers and a theory for lists; the union of these two
could be extended into a theory defining the length of a list:
\begin{tt}
\[
\begin{array}{c@{}c@{}c@{}c@{}c}
     {}   &     {}   &\hbox{Pure}&     {}  &     {}  \\
     {}   &     {}   &  \downarrow &     {}   &     {}   \\
     {}   &     {}   &\hbox{FOL} &     {}   &     {}   \\
     {}   & \swarrow &     {}    & \searrow &     {}   \\
 \hbox{Nat} &   {}   &     {}    &     {}   & \hbox{List} \\
     {}   & \searrow &     {}    & \swarrow &     {}   \\
     {}   &     {} &\hbox{Nat}+\hbox{List}&  {}   &     {}   \\
     {}   &     {}   &  \downarrow &     {}   &     {}   \\
     {}   &     {} & \hbox{Length} &  {}   &     {}
\end{array}
\]
\end{tt}%
Each Isabelle proof typically works within a single theory, which is
associated with the proof state.  However, many different theories may
coexist at the same time, and you may work in each of these during a single
session.  

\begin{warn}\index{constants!clashes with variables}%
  Confusing problems arise if you work in the wrong theory.  Each theory
  defines its own syntax.  An identifier may be regarded in one theory as a
  constant and in another as a variable, for example.
\end{warn}

\section{Proof construction in Isabelle}
I have elsewhere described the meta-logic and demonstrated it by
formalizing first-order logic~\cite{paulson-found}.  There is a one-to-one
correspondence between meta-level proofs and object-level proofs.  To each
use of a meta-level axiom, such as $(\forall I)$, there is a use of the
corresponding object-level rule.  Object-level assumptions and parameters
have meta-level counterparts.  The meta-level formalization is {\bf
  faithful}, admitting no incorrect object-level inferences, and {\bf
  adequate}, admitting all correct object-level inferences.  These
properties must be demonstrated separately for each object-logic.

The meta-logic is defined by a collection of inference rules, including
equational rules for the $\lambda$-calculus and logical rules.  The rules
for~$\Imp$ and~$\Forall$ resemble those for~$\imp$ and~$\forall$ in
Fig.\ts\ref{fol-fig}.  Proofs performed using the primitive meta-rules
would be lengthy; Isabelle proofs normally use certain derived rules.
{\bf Resolution}, in particular, is convenient for backward proof.

Unification is central to theorem proving.  It supports quantifier
reasoning by allowing certain `unknown' terms to be instantiated later,
possibly in stages.  When proving that the time required to sort $n$
integers is proportional to~$n^2$, we need not state the constant of
proportionality; when proving that a hardware adder will deliver the sum of
its inputs, we need not state how many clock ticks will be required.  Such
quantities often emerge from the proof.

Isabelle provides {\bf schematic variables}, or {\bf
  unknowns},\index{unknowns} for unification.  Logically, unknowns are free
variables.  But while ordinary variables remain fixed, unification may
instantiate unknowns.  Unknowns are written with a ?\ prefix and are
frequently subscripted: $\Var{a}$, $\Var{a@1}$, $\Var{a@2}$, \ldots,
$\Var{P}$, $\Var{P@1}$, \ldots.

Recall that an inference rule of the form
\[ \infer{\phi}{\phi@1 & \ldots & \phi@n} \]
is formalized in Isabelle's meta-logic as the axiom
$\List{\phi@1; \ldots; \phi@n} \Imp \phi$.\index{resolution}
Such axioms resemble Prolog's Horn clauses, and can be combined by
resolution --- Isabelle's principal proof method.  Resolution yields both
forward and backward proof.  Backward proof works by unifying a goal with
the conclusion of a rule, whose premises become new subgoals.  Forward proof
works by unifying theorems with the premises of a rule, deriving a new theorem.

Isabelle formulae require an extended notion of resolution.
They differ from Horn clauses in two major respects:
\begin{itemize}
  \item They are written in the typed $\lambda$-calculus, and therefore must be
resolved using higher-order unification.

\item The constituents of a clause need not be atomic formulae.  Any
  formula of the form $Trueprop(\cdots)$ is atomic, but axioms such as
  ${\imp}I$ and $\forall I$ contain non-atomic formulae.
\end{itemize}
Isabelle has little in common with classical resolution theorem provers
such as Otter~\cite{wos-bledsoe}.  At the meta-level, Isabelle proves
theorems in their positive form, not by refutation.  However, an
object-logic that includes a contradiction rule may employ a refutation
proof procedure.


\subsection{Higher-order unification}
\index{unification!higher-order|bold}
Unification is equation solving.  The solution of $f(\Var{x},c) \qeq
f(d,\Var{y})$ is $\Var{x}\equiv d$ and $\Var{y}\equiv c$.  {\bf
Higher-order unification} is equation solving for typed $\lambda$-terms.
To handle $\beta$-conversion, it must reduce $(\lambda x.t)u$ to $t[u/x]$.
That is easy --- in the typed $\lambda$-calculus, all reduction sequences
terminate at a normal form.  But it must guess the unknown
function~$\Var{f}$ in order to solve the equation
\begin{equation} \label{hou-eqn}
 \Var{f}(t) \qeq g(u@1,\ldots,u@k).
\end{equation}
Huet's~\cite{huet75} search procedure solves equations by imitation and
projection.  {\bf Imitation} makes~$\Var{f}$ apply the leading symbol (if a
constant) of the right-hand side.  To solve equation~(\ref{hou-eqn}), it
guesses
\[ \Var{f} \equiv \lambda x. g(\Var{h@1}(x),\ldots,\Var{h@k}(x)), \]
where $\Var{h@1}$, \ldots, $\Var{h@k}$ are new unknowns.  Assuming there are no
other occurrences of~$\Var{f}$, equation~(\ref{hou-eqn}) simplifies to the
set of equations
\[ \Var{h@1}(t)\qeq u@1 \quad\ldots\quad \Var{h@k}(t)\qeq u@k. \]
If the procedure solves these equations, instantiating $\Var{h@1}$, \ldots,
$\Var{h@k}$, then it yields an instantiation for~$\Var{f}$.

{\bf Projection} makes $\Var{f}$ apply one of its arguments.  To solve
equation~(\ref{hou-eqn}), if $t$ expects~$m$ arguments and delivers a
result of suitable type, it guesses
\[ \Var{f} \equiv \lambda x. x(\Var{h@1}(x),\ldots,\Var{h@m}(x)), \]
where $\Var{h@1}$, \ldots, $\Var{h@m}$ are new unknowns.  Assuming there are no
other occurrences of~$\Var{f}$, equation~(\ref{hou-eqn}) simplifies to the 
equation 
\[ t(\Var{h@1}(t),\ldots,\Var{h@m}(t)) \qeq g(u@1,\ldots,u@k). \]

\begin{warn}\index{unification!incompleteness of}%
Huet's unification procedure is complete.  Isabelle's polymorphic version,
which solves for type unknowns as well as for term unknowns, is incomplete.
The problem is that projection requires type information.  In
equation~(\ref{hou-eqn}), if the type of~$t$ is unknown, then projections
are possible for all~$m\geq0$, and the types of the $\Var{h@i}$ will be
similarly unconstrained.  Therefore, Isabelle never attempts such
projections, and may fail to find unifiers where a type unknown turns out
to be a function type.
\end{warn}

\index{unknowns!function|bold}
Given $\Var{f}(t@1,\ldots,t@n)\qeq u$, Huet's procedure could make up to
$n+1$ guesses.  The search tree and set of unifiers may be infinite.  But
higher-order unification can work effectively, provided you are careful
with {\bf function unknowns}:
\begin{itemize}
  \item Equations with no function unknowns are solved using first-order
unification, extended to treat bound variables.  For example, $\lambda x.x
\qeq \lambda x.\Var{y}$ has no solution because $\Var{y}\equiv x$ would
capture the free variable~$x$.

  \item An occurrence of the term $\Var{f}(x,y,z)$, where the arguments are
distinct bound variables, causes no difficulties.  Its projections can only
match the corresponding variables.

  \item Even an equation such as $\Var{f}(a)\qeq a+a$ is all right.  It has
four solutions, but Isabelle evaluates them lazily, trying projection before
imitation.  The first solution is usually the one desired:
\[ \Var{f}\equiv \lambda x. x+x \quad
   \Var{f}\equiv \lambda x. a+x \quad
   \Var{f}\equiv \lambda x. x+a \quad
   \Var{f}\equiv \lambda x. a+a \]
  \item  Equations such as $\Var{f}(\Var{x},\Var{y})\qeq t$ and
$\Var{f}(\Var{g}(x))\qeq t$ admit vast numbers of unifiers, and must be
avoided. 
\end{itemize}
In problematic cases, you may have to instantiate some unknowns before
invoking unification. 


\subsection{Joining rules by resolution} \label{joining}
\index{resolution|bold}
Let $\List{\psi@1; \ldots; \psi@m} \Imp \psi$ and $\List{\phi@1; \ldots;
\phi@n} \Imp \phi$ be two Isabelle theorems, representing object-level rules. 
Choosing some~$i$ from~1 to~$n$, suppose that $\psi$ and $\phi@i$ have a
higher-order unifier.  Writing $Xs$ for the application of substitution~$s$ to
expression~$X$, this means there is some~$s$ such that $\psi s\equiv \phi@i s$.
By resolution, we may conclude
\[ (\List{\phi@1; \ldots; \phi@{i-1}; \psi@1; \ldots; \psi@m;
          \phi@{i+1}; \ldots; \phi@n} \Imp \phi)s.
\]
The substitution~$s$ may instantiate unknowns in both rules.  In short,
resolution is the following rule:
\[ \infer[(\psi s\equiv \phi@i s)]
         {(\List{\phi@1; \ldots; \phi@{i-1}; \psi@1; \ldots; \psi@m;
          \phi@{i+1}; \ldots; \phi@n} \Imp \phi)s}
         {\List{\psi@1; \ldots; \psi@m} \Imp \psi & &
          \List{\phi@1; \ldots; \phi@n} \Imp \phi}
\]
It operates at the meta-level, on Isabelle theorems, and is justified by
the properties of $\Imp$ and~$\Forall$.  It takes the number~$i$ (for
$1\leq i\leq n$) as a parameter and may yield infinitely many conclusions,
one for each unifier of $\psi$ with $\phi@i$.  Isabelle returns these
conclusions as a sequence (lazy list).

Resolution expects the rules to have no outer quantifiers~($\Forall$).
It may rename or instantiate any schematic variables, but leaves free
variables unchanged.  When constructing a theory, Isabelle puts the
rules into a standard form with all free variables converted into
schematic ones; for instance, $({\imp}E)$ becomes
\[ \List{\Var{P}\imp \Var{Q}; \Var{P}}  \Imp \Var{Q}. 
\]
When resolving two rules, the unknowns in the first rule are renamed, by
subscripting, to make them distinct from the unknowns in the second rule.  To
resolve $({\imp}E)$ with itself, the first copy of the rule becomes
\[ \List{\Var{P@1}\imp \Var{Q@1}; \Var{P@1}}  \Imp \Var{Q@1}. \]
Resolving this with $({\imp}E)$ in the first premise, unifying $\Var{Q@1}$ with
$\Var{P}\imp \Var{Q}$, is the meta-level inference
\[ \infer{\List{\Var{P@1}\imp (\Var{P}\imp \Var{Q}); \Var{P@1}; \Var{P}} 
           \Imp\Var{Q}.}
         {\List{\Var{P@1}\imp \Var{Q@1}; \Var{P@1}}  \Imp \Var{Q@1} & &
          \List{\Var{P}\imp \Var{Q}; \Var{P}}  \Imp \Var{Q}}
\]
Renaming the unknowns in the resolvent, we have derived the
object-level rule\index{rules!derived}
\[ \infer{Q.}{R\imp (P\imp Q)  &  R  &  P}  \]
Joining rules in this fashion is a simple way of proving theorems.  The
derived rules are conservative extensions of the object-logic, and may permit
simpler proofs.  Let us consider another example.  Suppose we have the axiom
$$ \forall x\,y. Suc(x)=Suc(y)\imp x=y. \eqno (inject) $$

\noindent 
The standard form of $(\forall E)$ is
$\forall x.\Var{P}(x)  \Imp \Var{P}(\Var{t})$.
Resolving $(inject)$ with $(\forall E)$ replaces $\Var{P}$ by
$\lambda x. \forall y. Suc(x)=Suc(y)\imp x=y$ and leaves $\Var{t}$
unchanged, yielding  
\[ \forall y. Suc(\Var{t})=Suc(y)\imp \Var{t}=y. \]
Resolving this with $(\forall E)$ puts a subscript on~$\Var{t}$
and yields
\[ Suc(\Var{t@1})=Suc(\Var{t})\imp \Var{t@1}=\Var{t}. \]
Resolving this with $({\imp}E)$ increases the subscripts and yields
\[ Suc(\Var{t@2})=Suc(\Var{t@1})\Imp \Var{t@2}=\Var{t@1}. 
\]
We have derived the rule
\[ \infer{m=n,}{Suc(m)=Suc(n)} \]
which goes directly from $Suc(m)=Suc(n)$ to $m=n$.  It is handy for simplifying
an equation like $Suc(Suc(Suc(m)))=Suc(Suc(Suc(0)))$.  


\section{Lifting a rule into a context}
The rules $({\imp}I)$ and $(\forall I)$ may seem unsuitable for
resolution.  They have non-atomic premises, namely $P\Imp Q$ and $\Forall
x.P(x)$, while the conclusions of all the rules are atomic (they have the form
$Trueprop(\cdots)$).  Isabelle gets round the problem through a meta-inference
called \bfindex{lifting}.  Let us consider how to construct proofs such as
\[ \infer[({\imp}I)]{P\imp(Q\imp R)}
         {\infer[({\imp}I)]{Q\imp R}
                        {\infer*{R}{[P,Q]}}}
   \qquad
   \infer[(\forall I)]{\forall x\,y.P(x,y)}
         {\infer[(\forall I)]{\forall y.P(x,y)}{P(x,y)}}
\]

\subsection{Lifting over assumptions}
\index{assumptions!lifting over}
Lifting over $\theta\Imp{}$ is the following meta-inference rule:
\[ \infer{\List{\theta\Imp\phi@1; \ldots; \theta\Imp\phi@n} \Imp
          (\theta \Imp \phi)}
         {\List{\phi@1; \ldots; \phi@n} \Imp \phi} \]
This is clearly sound: if $\List{\phi@1; \ldots; \phi@n} \Imp \phi$ is true and
$\theta\Imp\phi@1$, \ldots, $\theta\Imp\phi@n$ and $\theta$ are all true then
$\phi$ must be true.  Iterated lifting over a series of meta-formulae
$\theta@k$, \ldots, $\theta@1$ yields an object-rule whose conclusion is
$\List{\theta@1; \ldots; \theta@k} \Imp \phi$.  Typically the $\theta@i$ are
the assumptions in a natural deduction proof; lifting copies them into a rule's
premises and conclusion.

When resolving two rules, Isabelle lifts the first one if necessary.  The
standard form of $({\imp}I)$ is
\[ (\Var{P} \Imp \Var{Q})  \Imp  \Var{P}\imp \Var{Q}.   \]
To resolve this rule with itself, Isabelle modifies one copy as follows: it
renames the unknowns to $\Var{P@1}$ and $\Var{Q@1}$, then lifts the rule over
$\Var{P}\Imp{}$ to obtain
\[ (\Var{P}\Imp (\Var{P@1} \Imp \Var{Q@1})) \Imp (\Var{P} \Imp 
   (\Var{P@1}\imp \Var{Q@1})).   \]
Using the $\List{\cdots}$ abbreviation, this can be written as
\[ \List{\List{\Var{P}; \Var{P@1}} \Imp \Var{Q@1}; \Var{P}} 
   \Imp  \Var{P@1}\imp \Var{Q@1}.   \]
Unifying $\Var{P}\Imp \Var{P@1}\imp\Var{Q@1}$ with $\Var{P} \Imp
\Var{Q}$ instantiates $\Var{Q}$ to ${\Var{P@1}\imp\Var{Q@1}}$.
Resolution yields
\[ (\List{\Var{P}; \Var{P@1}} \Imp \Var{Q@1}) \Imp
\Var{P}\imp(\Var{P@1}\imp\Var{Q@1}).   \]
This represents the derived rule
\[ \infer{P\imp(Q\imp R).}{\infer*{R}{[P,Q]}} \]

\subsection{Lifting over parameters}
\index{parameters!lifting over}
An analogous form of lifting handles premises of the form $\Forall x\ldots\,$. 
Here, lifting prefixes an object-rule's premises and conclusion with $\Forall
x$.  At the same time, lifting introduces a dependence upon~$x$.  It replaces
each unknown $\Var{a}$ in the rule by $\Var{a'}(x)$, where $\Var{a'}$ is a new
unknown (by subscripting) of suitable type --- necessarily a function type.  In
short, lifting is the meta-inference
\[ \infer{\List{\Forall x.\phi@1^x; \ldots; \Forall x.\phi@n^x} 
           \Imp \Forall x.\phi^x,}
         {\List{\phi@1; \ldots; \phi@n} \Imp \phi} \]
%
where $\phi^x$ stands for the result of lifting unknowns over~$x$ in
$\phi$.  It is not hard to verify that this meta-inference is sound.  If
$\phi\Imp\psi$ then $\phi^x\Imp\psi^x$ for all~$x$; so if $\phi^x$ is true
for all~$x$ then so is $\psi^x$.  Thus, from $\phi\Imp\psi$ we conclude
$(\Forall x.\phi^x) \Imp (\Forall x.\psi^x)$.

For example, $(\disj I)$ might be lifted to
\[ (\Forall x.\Var{P@1}(x)) \Imp (\Forall x. \Var{P@1}(x)\disj \Var{Q@1}(x))\]
and $(\forall I)$ to
\[ (\Forall x\,y.\Var{P@1}(x,y)) \Imp (\Forall x. \forall y.\Var{P@1}(x,y)). \]
Isabelle has renamed a bound variable in $(\forall I)$ from $x$ to~$y$,
avoiding a clash.  Resolving the above with $(\forall I)$ is the meta-inference
\[ \infer{\Forall x\,y.\Var{P@1}(x,y)) \Imp \forall x\,y.\Var{P@1}(x,y)) }
         {(\Forall x\,y.\Var{P@1}(x,y)) \Imp 
               (\Forall x. \forall y.\Var{P@1}(x,y))  &
          (\Forall x.\Var{P}(x)) \Imp (\forall x.\Var{P}(x))} \]
Here, $\Var{P}$ is replaced by $\lambda x.\forall y.\Var{P@1}(x,y)$; the
resolvent expresses the derived rule
\[ \vcenter{ \infer{\forall x\,y.Q(x,y)}{Q(x,y)} }
   \quad\hbox{provided $x$, $y$ not free in the assumptions} 
\] 
I discuss lifting and parameters at length elsewhere~\cite{paulson-found}.
Miller goes into even greater detail~\cite{miller-mixed}.


\section{Backward proof by resolution}
\index{resolution!in backward proof}

Resolution is convenient for deriving simple rules and for reasoning
forward from facts.  It can also support backward proof, where we start
with a goal and refine it to progressively simpler subgoals until all have
been solved.  {\sc lcf} and its descendants {\sc hol} and Nuprl provide
tactics and tacticals, which constitute a sophisticated language for
expressing proof searches.  {\bf Tactics} refine subgoals while {\bf
  tacticals} combine tactics.

\index{LCF system}
Isabelle's tactics and tacticals work differently from {\sc lcf}'s.  An
Isabelle rule is bidirectional: there is no distinction between
inputs and outputs.  {\sc lcf} has a separate tactic for each rule;
Isabelle performs refinement by any rule in a uniform fashion, using
resolution.

Isabelle works with meta-level theorems of the form
\( \List{\phi@1; \ldots; \phi@n} \Imp \phi \).
We have viewed this as the {\bf rule} with premises
$\phi@1$,~\ldots,~$\phi@n$ and conclusion~$\phi$.  It can also be viewed as
the {\bf proof state}\index{proof state}
with subgoals $\phi@1$,~\ldots,~$\phi@n$ and main
goal~$\phi$.

To prove the formula~$\phi$, take $\phi\Imp \phi$ as the initial proof
state.  This assertion is, trivially, a theorem.  At a later stage in the
backward proof, a typical proof state is $\List{\phi@1; \ldots; \phi@n}
\Imp \phi$.  This proof state is a theorem, ensuring that the subgoals
$\phi@1$,~\ldots,~$\phi@n$ imply~$\phi$.  If $n=0$ then we have
proved~$\phi$ outright.  If $\phi$ contains unknowns, they may become
instantiated during the proof; a proof state may be $\List{\phi@1; \ldots;
\phi@n} \Imp \phi'$, where $\phi'$ is an instance of~$\phi$.

\subsection{Refinement by resolution}
To refine subgoal~$i$ of a proof state by a rule, perform the following
resolution: 
\[ \infer{\hbox{new proof state}}{\hbox{rule} & & \hbox{proof state}} \]
Suppose the rule is $\List{\psi'@1; \ldots; \psi'@m} \Imp \psi'$ after
lifting over subgoal~$i$'s assumptions and parameters.  If the proof state
is $\List{\phi@1; \ldots; \phi@n} \Imp \phi$, then the new proof state is
(for~$1\leq i\leq n$)
\[ (\List{\phi@1; \ldots; \phi@{i-1}; \psi'@1;
          \ldots; \psi'@m; \phi@{i+1}; \ldots; \phi@n} \Imp \phi)s.  \]
Substitution~$s$ unifies $\psi'$ with~$\phi@i$.  In the proof state,
subgoal~$i$ is replaced by $m$ new subgoals, the rule's instantiated premises.
If some of the rule's unknowns are left un-instantiated, they become new
unknowns in the proof state.  Refinement by~$(\exists I)$, namely
\[ \Var{P}(\Var{t}) \Imp \exists x. \Var{P}(x), \]
inserts a new unknown derived from~$\Var{t}$ by subscripting and lifting.
We do not have to specify an `existential witness' when
applying~$(\exists I)$.  Further resolutions may instantiate unknowns in
the proof state.

\subsection{Proof by assumption}
\index{assumptions!use of}
In the course of a natural deduction proof, parameters $x@1$, \ldots,~$x@l$ and
assumptions $\theta@1$, \ldots, $\theta@k$ accumulate, forming a context for
each subgoal.  Repeated lifting steps can lift a rule into any context.  To
aid readability, Isabelle puts contexts into a normal form, gathering the
parameters at the front:
\begin{equation} \label{context-eqn}
\Forall x@1 \ldots x@l. \List{\theta@1; \ldots; \theta@k}\Imp\theta. 
\end{equation}
Under the usual reading of the connectives, this expresses that $\theta$
follows from $\theta@1$,~\ldots~$\theta@k$ for arbitrary
$x@1$,~\ldots,~$x@l$.  It is trivially true if $\theta$ equals any of
$\theta@1$,~\ldots~$\theta@k$, or is unifiable with any of them.  This
models proof by assumption in natural deduction.

Isabelle automates the meta-inference for proof by assumption.  Its arguments
are the meta-theorem $\List{\phi@1; \ldots; \phi@n} \Imp \phi$, and some~$i$
from~1 to~$n$, where $\phi@i$ has the form~(\ref{context-eqn}).  Its results
are meta-theorems of the form
\[ (\List{\phi@1; \ldots; \phi@{i-1}; \phi@{i+1}; \phi@n} \Imp \phi)s \]
for each $s$ and~$j$ such that $s$ unifies $\lambda x@1 \ldots x@l. \theta@j$
with $\lambda x@1 \ldots x@l. \theta$.  Isabelle supplies the parameters
$x@1$,~\ldots,~$x@l$ to higher-order unification as bound variables, which
regards them as unique constants with a limited scope --- this enforces
parameter provisos~\cite{paulson-found}.

The premise represents a proof state with~$n$ subgoals, of which the~$i$th
is to be solved by assumption.  Isabelle searches the subgoal's context for
an assumption~$\theta@j$ that can solve it.  For each unifier, the
meta-inference returns an instantiated proof state from which the $i$th
subgoal has been removed.  Isabelle searches for a unifying assumption; for
readability and robustness, proofs do not refer to assumptions by number.

Consider the proof state 
\[ (\List{P(a); P(b)} \Imp P(\Var{x})) \Imp Q(\Var{x}). \]
Proof by assumption (with $i=1$, the only possibility) yields two results:
\begin{itemize}
  \item $Q(a)$, instantiating $\Var{x}\equiv a$
  \item $Q(b)$, instantiating $\Var{x}\equiv b$
\end{itemize}
Here, proof by assumption affects the main goal.  It could also affect
other subgoals; if we also had the subgoal ${\List{P(b); P(c)} \Imp
  P(\Var{x})}$, then $\Var{x}\equiv a$ would transform it to ${\List{P(b);
    P(c)} \Imp P(a)}$, which might be unprovable.


\subsection{A propositional proof} \label{prop-proof}
\index{examples!propositional}
Our first example avoids quantifiers.  Given the main goal $P\disj P\imp
P$, Isabelle creates the initial state
\[ (P\disj P\imp P)\Imp (P\disj P\imp P). \] 
%
Bear in mind that every proof state we derive will be a meta-theorem,
expressing that the subgoals imply the main goal.  Our aim is to reach the
state $P\disj P\imp P$; this meta-theorem is the desired result.

The first step is to refine subgoal~1 by (${\imp}I)$, creating a new state
where $P\disj P$ is an assumption:
\[ (P\disj P\Imp P)\Imp (P\disj P\imp P) \]
The next step is $(\disj E)$, which replaces subgoal~1 by three new subgoals. 
Because of lifting, each subgoal contains a copy of the context --- the
assumption $P\disj P$.  (In fact, this assumption is now redundant; we shall
shortly see how to get rid of it!)  The new proof state is the following
meta-theorem, laid out for clarity:
\[ \begin{array}{l@{}l@{\qquad\qquad}l} 
  \lbrakk\;& P\disj P\Imp \Var{P@1}\disj\Var{Q@1}; & \hbox{(subgoal 1)} \\
           & \List{P\disj P; \Var{P@1}} \Imp P;    & \hbox{(subgoal 2)} \\
           & \List{P\disj P; \Var{Q@1}} \Imp P     & \hbox{(subgoal 3)} \\
  \rbrakk\;& \Imp (P\disj P\imp P)                 & \hbox{(main goal)}
   \end{array} 
\]
Notice the unknowns in the proof state.  Because we have applied $(\disj E)$,
we must prove some disjunction, $\Var{P@1}\disj\Var{Q@1}$.  Of course,
subgoal~1 is provable by assumption.  This instantiates both $\Var{P@1}$ and
$\Var{Q@1}$ to~$P$ throughout the proof state:
\[ \begin{array}{l@{}l@{\qquad\qquad}l} 
    \lbrakk\;& \List{P\disj P; P} \Imp P; & \hbox{(subgoal 1)} \\
             & \List{P\disj P; P} \Imp P  & \hbox{(subgoal 2)} \\
    \rbrakk\;& \Imp (P\disj P\imp P)      & \hbox{(main goal)}
   \end{array} \]
Both of the remaining subgoals can be proved by assumption.  After two such
steps, the proof state is $P\disj P\imp P$.


\subsection{A quantifier proof}
\index{examples!with quantifiers}
To illustrate quantifiers and $\Forall$-lifting, let us prove
$(\exists x.P(f(x)))\imp(\exists x.P(x))$.  The initial proof
state is the trivial meta-theorem 
\[ (\exists x.P(f(x)))\imp(\exists x.P(x)) \Imp 
   (\exists x.P(f(x)))\imp(\exists x.P(x)). \]
As above, the first step is refinement by (${\imp}I)$: 
\[ (\exists x.P(f(x))\Imp \exists x.P(x)) \Imp 
   (\exists x.P(f(x)))\imp(\exists x.P(x)) 
\]
The next step is $(\exists E)$, which replaces subgoal~1 by two new subgoals.
Both have the assumption $\exists x.P(f(x))$.  The new proof
state is the meta-theorem  
\[ \begin{array}{l@{}l@{\qquad\qquad}l} 
   \lbrakk\;& \exists x.P(f(x)) \Imp \exists x.\Var{P@1}(x); & \hbox{(subgoal 1)} \\
            & \Forall x.\List{\exists x.P(f(x)); \Var{P@1}(x)} \Imp 
                       \exists x.P(x)  & \hbox{(subgoal 2)} \\
    \rbrakk\;& \Imp (\exists x.P(f(x)))\imp(\exists x.P(x))  & \hbox{(main goal)}
   \end{array} 
\]
The unknown $\Var{P@1}$ appears in both subgoals.  Because we have applied
$(\exists E)$, we must prove $\exists x.\Var{P@1}(x)$, where $\Var{P@1}(x)$ may
become any formula possibly containing~$x$.  Proving subgoal~1 by assumption
instantiates $\Var{P@1}$ to~$\lambda x.P(f(x))$:  
\[ \left(\Forall x.\List{\exists x.P(f(x)); P(f(x))} \Imp 
         \exists x.P(x)\right) 
   \Imp (\exists x.P(f(x)))\imp(\exists x.P(x)) 
\]
The next step is refinement by $(\exists I)$.  The rule is lifted into the
context of the parameter~$x$ and the assumption $P(f(x))$.  This copies
the context to the subgoal and allows the existential witness to
depend upon~$x$: 
\[ \left(\Forall x.\List{\exists x.P(f(x)); P(f(x))} \Imp 
         P(\Var{x@2}(x))\right) 
   \Imp (\exists x.P(f(x)))\imp(\exists x.P(x)) 
\]
The existential witness, $\Var{x@2}(x)$, consists of an unknown
applied to a parameter.  Proof by assumption unifies $\lambda x.P(f(x))$ 
with $\lambda x.P(\Var{x@2}(x))$, instantiating $\Var{x@2}$ to $f$.  The final
proof state contains no subgoals: $(\exists x.P(f(x)))\imp(\exists x.P(x))$.


\subsection{Tactics and tacticals}
\index{tactics|bold}\index{tacticals|bold}
{\bf Tactics} perform backward proof.  Isabelle tactics differ from those
of {\sc lcf}, {\sc hol} and Nuprl by operating on entire proof states,
rather than on individual subgoals.  An Isabelle tactic is a function that
takes a proof state and returns a sequence (lazy list) of possible
successor states.  Lazy lists are coded in ML as functions, a standard
technique~\cite{paulson-ml2}.  Isabelle represents proof states by theorems.

Basic tactics execute the meta-rules described above, operating on a
given subgoal.  The {\bf resolution tactics} take a list of rules and
return next states for each combination of rule and unifier.  The {\bf
assumption tactic} examines the subgoal's assumptions and returns next
states for each combination of assumption and unifier.  Lazy lists are
essential because higher-order resolution may return infinitely many
unifiers.  If there are no matching rules or assumptions then no next
states are generated; a tactic application that returns an empty list is
said to {\bf fail}.

Sequences realize their full potential with {\bf tacticals} --- operators
for combining tactics.  Depth-first search, breadth-first search and
best-first search (where a heuristic function selects the best state to
explore) return their outcomes as a sequence.  Isabelle provides such
procedures in the form of tacticals.  Simpler procedures can be expressed
directly using the basic tacticals {\tt THEN}, {\tt ORELSE} and {\tt REPEAT}:
\begin{ttdescription}
\item[$tac1$ THEN $tac2$] is a tactic for sequential composition.  Applied
to a proof state, it returns all states reachable in two steps by applying
$tac1$ followed by~$tac2$.

\item[$tac1$ ORELSE $tac2$] is a choice tactic.  Applied to a state, it
tries~$tac1$ and returns the result if non-empty; otherwise, it uses~$tac2$.

\item[REPEAT $tac$] is a repetition tactic.  Applied to a state, it
returns all states reachable by applying~$tac$ as long as possible --- until
it would fail.  
\end{ttdescription}
For instance, this tactic repeatedly applies $tac1$ and~$tac2$, giving
$tac1$ priority:
\begin{center} \tt
REPEAT($tac1$ ORELSE $tac2$)
\end{center}


\section{Variations on resolution}
In principle, resolution and proof by assumption suffice to prove all
theorems.  However, specialized forms of resolution are helpful for working
with elimination rules.  Elim-resolution applies an elimination rule to an
assumption; destruct-resolution is similar, but applies a rule in a forward
style.

The last part of the section shows how the techniques for proving theorems
can also serve to derive rules.

\subsection{Elim-resolution}
\index{elim-resolution|bold}\index{assumptions!deleting}

Consider proving the theorem $((R\disj R)\disj R)\disj R\imp R$.  By
$({\imp}I)$, we prove~$R$ from the assumption $((R\disj R)\disj R)\disj R$.
Applying $(\disj E)$ to this assumption yields two subgoals, one that
assumes~$R$ (and is therefore trivial) and one that assumes $(R\disj
R)\disj R$.  This subgoal admits another application of $(\disj E)$.  Since
natural deduction never discards assumptions, we eventually generate a
subgoal containing much that is redundant:
\[ \List{((R\disj R)\disj R)\disj R; (R\disj R)\disj R; R\disj R; R} \Imp R. \]
In general, using $(\disj E)$ on the assumption $P\disj Q$ creates two new
subgoals with the additional assumption $P$ or~$Q$.  In these subgoals,
$P\disj Q$ is redundant.  Other elimination rules behave
similarly.  In first-order logic, only universally quantified
assumptions are sometimes needed more than once --- say, to prove
$P(f(f(a)))$ from the assumptions $\forall x.P(x)\imp P(f(x))$ and~$P(a)$.

Many logics can be formulated as sequent calculi that delete redundant
assumptions after use.  The rule $(\disj E)$ might become
\[ \infer[\disj\hbox{-left}]
         {\Gamma,P\disj Q,\Delta \turn \Theta}
         {\Gamma,P,\Delta \turn \Theta && \Gamma,Q,\Delta \turn \Theta}  \] 
In backward proof, a goal containing $P\disj Q$ on the left of the~$\turn$
(that is, as an assumption) splits into two subgoals, replacing $P\disj Q$
by $P$ or~$Q$.  But the sequent calculus, with its explicit handling of
assumptions, can be tiresome to use.

Elim-resolution is Isabelle's way of getting sequent calculus behaviour
from natural deduction rules.  It lets an elimination rule consume an
assumption.  Elim-resolution combines two meta-theorems:
\begin{itemize}
  \item a rule $\List{\psi@1; \ldots; \psi@m} \Imp \psi$
  \item a proof state $\List{\phi@1; \ldots; \phi@n} \Imp \phi$
\end{itemize}
The rule must have at least one premise, thus $m>0$.  Write the rule's
lifted form as $\List{\psi'@1; \ldots; \psi'@m} \Imp \psi'$.  Suppose we
wish to change subgoal number~$i$.

Ordinary resolution would attempt to reduce~$\phi@i$,
replacing subgoal~$i$ by $m$ new ones.  Elim-resolution tries
simultaneously to reduce~$\phi@i$ and to solve~$\psi'@1$ by assumption; it
returns a sequence of next states.  Each of these replaces subgoal~$i$ by
instances of $\psi'@2$, \ldots, $\psi'@m$ from which the selected
assumption has been deleted.  Suppose $\phi@i$ has the parameter~$x$ and
assumptions $\theta@1$,~\ldots,~$\theta@k$.  Then $\psi'@1$, the rule's first
premise after lifting, will be
\( \Forall x. \List{\theta@1; \ldots; \theta@k}\Imp \psi^{x}@1 \).
Elim-resolution tries to unify $\psi'\qeq\phi@i$ and
$\lambda x. \theta@j \qeq \lambda x. \psi^{x}@1$ simultaneously, for
$j=1$,~\ldots,~$k$. 

Let us redo the example from~\S\ref{prop-proof}.  The elimination rule
is~$(\disj E)$,
\[ \List{\Var{P}\disj \Var{Q};\; \Var{P}\Imp \Var{R};\; \Var{Q}\Imp \Var{R}}
      \Imp \Var{R}  \]
and the proof state is $(P\disj P\Imp P)\Imp (P\disj P\imp P)$.  The
lifted rule is
\[ \begin{array}{l@{}l}
  \lbrakk\;& P\disj P \Imp \Var{P@1}\disj\Var{Q@1}; \\
           & \List{P\disj P ;\; \Var{P@1}} \Imp \Var{R@1};    \\
           & \List{P\disj P ;\; \Var{Q@1}} \Imp \Var{R@1}     \\
  \rbrakk\;& \Imp (P\disj P \Imp \Var{R@1})
  \end{array} 
\]
Unification takes the simultaneous equations
$P\disj P \qeq \Var{P@1}\disj\Var{Q@1}$ and $\Var{R@1} \qeq P$, yielding
$\Var{P@1}\equiv\Var{Q@1}\equiv\Var{R@1} \equiv P$.  The new proof state
is simply
\[ \List{P \Imp P;\; P \Imp P} \Imp (P\disj P\imp P). 
\]
Elim-resolution's simultaneous unification gives better control
than ordinary resolution.  Recall the substitution rule:
$$ \List{\Var{t}=\Var{u}; \Var{P}(\Var{t})} \Imp \Var{P}(\Var{u}) 
\eqno(subst) $$
Unsuitable for ordinary resolution because $\Var{P}(\Var{u})$ admits many
unifiers, $(subst)$ works well with elim-resolution.  It deletes some
assumption of the form $x=y$ and replaces every~$y$ by~$x$ in the subgoal
formula.  The simultaneous unification instantiates $\Var{u}$ to~$y$; if
$y$ is not an unknown, then $\Var{P}(y)$ can easily be unified with another
formula.  

In logical parlance, the premise containing the connective to be eliminated
is called the \bfindex{major premise}.  Elim-resolution expects the major
premise to come first.  The order of the premises is significant in
Isabelle.

\subsection{Destruction rules} \label{destruct}
\index{rules!destruction}\index{rules!elimination}
\index{forward proof}

Looking back to Fig.\ts\ref{fol-fig}, notice that there are two kinds of
elimination rule.  The rules $({\conj}E1)$, $({\conj}E2)$, $({\imp}E)$ and
$({\forall}E)$ extract the conclusion from the major premise.  In Isabelle
parlance, such rules are called {\bf destruction rules}; they are readable
and easy to use in forward proof.  The rules $({\disj}E)$, $({\bot}E)$ and
$({\exists}E)$ work by discharging assumptions; they support backward proof
in a style reminiscent of the sequent calculus.

The latter style is the most general form of elimination rule.  In natural
deduction, there is no way to recast $({\disj}E)$, $({\bot}E)$ or
$({\exists}E)$ as destruction rules.  But we can write general elimination
rules for $\conj$, $\imp$ and~$\forall$:
\[
\infer{R}{P\conj Q & \infer*{R}{[P,Q]}} \qquad
\infer{R}{P\imp Q & P & \infer*{R}{[Q]}}  \qquad
\infer{Q}{\forall x.P & \infer*{Q}{[P[t/x]]}} 
\]
Because they are concise, destruction rules are simpler to derive than the
corresponding elimination rules.  To facilitate their use in backward
proof, Isabelle provides a means of transforming a destruction rule such as
\[ \infer[\quad\hbox{to the elimination rule}\quad]{Q}{P@1 & \ldots & P@m} 
   \infer{R.}{P@1 & \ldots & P@m & \infer*{R}{[Q]}} 
\]
{\bf Destruct-resolution}\index{destruct-resolution} combines this
transformation with elim-resolution.  It applies a destruction rule to some
assumption of a subgoal.  Given the rule above, it replaces the
assumption~$P@1$ by~$Q$, with new subgoals of showing instances of $P@2$,
\ldots,~$P@m$.  Destruct-resolution works forward from a subgoal's
assumptions.  Ordinary resolution performs forward reasoning from theorems,
as illustrated in~\S\ref{joining}.


\subsection{Deriving rules by resolution}  \label{deriving}
\index{rules!derived|bold}\index{meta-assumptions!syntax of}
The meta-logic, itself a form of the predicate calculus, is defined by a
system of natural deduction rules.  Each theorem may depend upon
meta-assumptions.  The theorem that~$\phi$ follows from the assumptions
$\phi@1$, \ldots, $\phi@n$ is written
\[ \phi \quad [\phi@1,\ldots,\phi@n]. \]
A more conventional notation might be $\phi@1,\ldots,\phi@n \turn \phi$,
but Isabelle's notation is more readable with large formulae.

Meta-level natural deduction provides a convenient mechanism for deriving
new object-level rules.  To derive the rule
\[ \infer{\phi,}{\theta@1 & \ldots & \theta@k} \]
assume the premises $\theta@1$,~\ldots,~$\theta@k$ at the
meta-level.  Then prove $\phi$, possibly using these assumptions.
Starting with a proof state $\phi\Imp\phi$, assumptions may accumulate,
reaching a final state such as
\[ \phi \quad [\theta@1,\ldots,\theta@k]. \]
The meta-rule for $\Imp$ introduction discharges an assumption.
Discharging them in the order $\theta@k,\ldots,\theta@1$ yields the
meta-theorem $\List{\theta@1; \ldots; \theta@k} \Imp \phi$, with no
assumptions.  This represents the desired rule.
Let us derive the general $\conj$ elimination rule:
$$ \infer{R}{P\conj Q & \infer*{R}{[P,Q]}}  \eqno(\conj E) $$
We assume $P\conj Q$ and $\List{P;Q}\Imp R$, and commence backward proof in
the state $R\Imp R$.  Resolving this with the second assumption yields the
state 
\[ \phantom{\List{P\conj Q;\; P\conj Q}}
   \llap{$\List{P;Q}$}\Imp R \quad [\,\List{P;Q}\Imp R\,]. \]
Resolving subgoals~1 and~2 with~$({\conj}E1)$ and~$({\conj}E2)$,
respectively, yields the state
\[ \List{P\conj \Var{Q@1};\; \Var{P@2}\conj Q}\Imp R 
   \quad [\,\List{P;Q}\Imp R\,]. 
\]
The unknowns $\Var{Q@1}$ and~$\Var{P@2}$ arise from unconstrained
subformulae in the premises of~$({\conj}E1)$ and~$({\conj}E2)$.  Resolving
both subgoals with the assumption $P\conj Q$ instantiates the unknowns to yield
\[ R \quad [\, \List{P;Q}\Imp R, P\conj Q \,]. \]
The proof may use the meta-assumptions in any order, and as often as
necessary; when finished, we discharge them in the correct order to
obtain the desired form:
\[ \List{P\conj Q;\; \List{P;Q}\Imp R} \Imp R \]
We have derived the rule using free variables, which prevents their
premature instantiation during the proof; we may now replace them by
schematic variables.

\begin{warn}
  Schematic variables are not allowed in meta-assumptions, for a variety of
  reasons.  Meta-assumptions remain fixed throughout a proof.
\end{warn}


%% $Id$
\part{Getting Started with Isabelle}\label{chap:getting}
Let us consider how to perform simple proofs using Isabelle.  At
present, Isabelle's user interface is \ML.  Proofs are conducted by
applying certain \ML{} functions, which update a stored proof state.
Basically, all syntax must be expressed using plain {\sc ascii}
characters.  There are also mechanisms built into Isabelle that support
alternative syntaxes, for example using mathematical symbols from a
special screen font.  The meta-logic and major object-logics already
provide such fancy output as an option.

Object-logics are built upon Pure Isabelle, which implements the
meta-logic and provides certain fundamental data structures: types,
terms, signatures, theorems and theories, tactics and tacticals.
These data structures have the corresponding \ML{} types {\tt typ},
{\tt term}, {\tt Sign.sg}, {\tt thm}, {\tt theory} and {\tt tactic};
tacticals have function types such as {\tt tactic->tactic}.  Isabelle
users can operate on these data structures by writing \ML{} programs.

\section{Forward proof}\label{sec:forward} \index{forward proof|(}
This section describes the concrete syntax for types, terms and theorems,
and demonstrates forward proof.

\subsection{Lexical matters}
\index{identifiers}\index{reserved words} 
An {\bf identifier} is a string of letters, digits, underscores~(\verb|_|)
and single quotes~({\tt'}), beginning with a letter.  Single quotes are
regarded as primes; for instance {\tt x'} is read as~$x'$.  Identifiers are
separated by white space and special characters.  {\bf Reserved words} are
identifiers that appear in Isabelle syntax definitions.

An Isabelle theory can declare symbols composed of special characters, such
as {\tt=}, {\tt==}, {\tt=>} and {\tt==>}.  (The latter three are part of
the syntax of the meta-logic.)  Such symbols may be run together; thus if
\verb|}| and \verb|{| are used for set brackets then \verb|{{a},{a,b}}| is
valid notation for a set of sets --- but only if \verb|}}| and \verb|{{|
have not been declared as symbols!  The parser resolves any ambiguity by
taking the longest possible symbol that has been declared.  Thus the string
{\tt==>} is read as a single symbol.  But \hbox{\tt= =>} is read as two
symbols.

Identifiers that are not reserved words may serve as free variables or
constants.  A {\bf type identifier} consists of an identifier prefixed by a
prime, for example {\tt'a} and \hbox{\tt'hello}.  Type identifiers stand
for (free) type variables, which remain fixed during a proof.
\index{type identifiers}

An {\bf unknown}\index{unknowns} (or type unknown) consists of a question
mark, an identifier (or type identifier), and a subscript.  The subscript,
a non-negative integer,
allows the renaming of unknowns prior to unification.%
\footnote{The subscript may appear after the identifier, separated by a
  dot; this prevents ambiguity when the identifier ends with a digit.  Thus
  {\tt?z6.0} has identifier {\tt"z6"} and subscript~0, while {\tt?a0.5}
  has identifier {\tt"a0"} and subscript~5.  If the identifier does not
  end with a digit, then no dot appears and a subscript of~0 is omitted;
  for example, {\tt?hello} has identifier {\tt"hello"} and subscript
  zero, while {\tt?z6} has identifier {\tt"z"} and subscript~6.  The same
  conventions apply to type unknowns.  The question mark is {\it not\/}
  part of the identifier!}


\subsection{Syntax of types and terms}
\index{classes!built-in|bold}\index{syntax!of types and terms}

Classes are denoted by identifiers; the built-in class \cldx{logic}
contains the `logical' types.  Sorts are lists of classes enclosed in
braces~\} and \{; singleton sorts may be abbreviated by dropping the braces.

\index{types!syntax of|bold}\index{sort constraints} Types are written
with a syntax like \ML's.  The built-in type \tydx{prop} is the type
of propositions.  Type variables can be constrained to particular
classes or sorts, for example {\tt 'a::term} and {\tt ?'b::\ttlbrace
  ord, arith\ttrbrace}.
\[\dquotes
\index{*:: symbol}\index{*=> symbol}
\index{{}@{\tt\ttlbrace} symbol}\index{{}@{\tt\ttrbrace} symbol}
\index{*[ symbol}\index{*] symbol}
\begin{array}{lll}
    \multicolumn{3}{c}{\hbox{ASCII Notation for Types}} \\ \hline
  \alpha "::" C              & \alpha :: C        & \hbox{class constraint} \\
  \alpha "::" "\ttlbrace" C@1 "," \ldots "," C@n "\ttrbrace" &
     \alpha :: \{C@1,\dots,C@n\}             & \hbox{sort constraint} \\
  \sigma " => " \tau        & \sigma\To\tau & \hbox{function type} \\
  "[" \sigma@1 "," \ldots "," \sigma@n "] => " \tau &
     [\sigma@1,\ldots,\sigma@n] \To\tau & \hbox{curried function type} \\
  "(" \tau@1"," \ldots "," \tau@n ")" tycon & 
     (\tau@1, \ldots, \tau@n)tycon      & \hbox{type construction}
\end{array} 
\]
Terms are those of the typed $\lambda$-calculus.
\index{terms!syntax of|bold}\index{type constraints}
\[\dquotes
\index symbol}\index{lambda abs@$\lambda$-abstractions}
\index{*:: symbol}
\begin{array}{lll}
    \multicolumn{3}{c}{\hbox{ASCII Notation for Terms}} \\ \hline
  t "::" \sigma         & t :: \sigma   & \hbox{type constraint} \\
  "\%" x "." t          & \lambda x.t   & \hbox{abstraction} \\
  "\%" x@1\ldots x@n "." t  & \lambda x@1\ldots x@n.t & 
     \hbox{curried abstraction} \\
  t "(" u@1"," \ldots "," u@n ")" & 
  t (u@1, \ldots, u@n) & \hbox{curried application}
\end{array}  
\]
The theorems and rules of an object-logic are represented by theorems in
the meta-logic, which are expressed using meta-formulae.  Since the
meta-logic is higher-order, meta-formulae~$\phi$, $\psi$, $\theta$,~\ldots{}
are just terms of type~{\tt prop}.  
\index{meta-implication}
\index{meta-quantifiers}\index{meta-equality}
\index{*"!"! symbol}\index{*"["| symbol}\index{*"|"] symbol}
\index{*== symbol}\index{*=?= symbol}\index{*==> symbol}
\[\dquotes
  \begin{array}{l@{\quad}l@{\quad}l}
    \multicolumn{3}{c}{\hbox{ASCII Notation for Meta-Formulae}} \\ \hline
  a " == " b    & a\equiv b &   \hbox{meta-equality} \\
  a " =?= " b   & a\qeq b &     \hbox{flex-flex constraint} \\
  \phi " ==> " \psi & \phi\Imp \psi & \hbox{meta-implication} \\
  "[|" \phi@1 ";" \ldots ";" \phi@n "|] ==> " \psi & 
  \List{\phi@1;\ldots;\phi@n} \Imp \psi & \hbox{nested implication} \\
  "!!" x "." \phi & \Forall x.\phi & \hbox{meta-quantification} \\
  "!!" x@1\ldots x@n "." \phi & 
  \Forall x@1. \ldots x@n.\phi & \hbox{nested quantification}
  \end{array}
\]
Flex-flex constraints are meta-equalities arising from unification; they
require special treatment.  See~\S\ref{flexflex}.
\index{flex-flex constraints}

\index{*Trueprop constant}
Most logics define the implicit coercion $Trueprop$ from object-formulae to
propositions.  This could cause an ambiguity: in $P\Imp Q$, do the
variables $P$ and $Q$ stand for meta-formulae or object-formulae?  If the
latter, $P\Imp Q$ really abbreviates $Trueprop(P)\Imp Trueprop(Q)$.  To
prevent such ambiguities, Isabelle's syntax does not allow a meta-formula
to consist of a variable.  Variables of type~\tydx{prop} are seldom
useful, but you can make a variable stand for a meta-formula by prefixing
it with the symbol {\tt PROP}:\index{*PROP symbol}
\begin{ttbox} 
PROP ?psi ==> PROP ?theta 
\end{ttbox}

Symbols of object-logics are typically rendered into {\sc ascii} as
follows:
\[ \begin{tabular}{l@{\quad}l@{\quad}l}
      \tt True          & $\top$        & true \\
      \tt False         & $\bot$        & false \\
      \tt $P$ \& $Q$    & $P\conj Q$    & conjunction \\
      \tt $P$ | $Q$     & $P\disj Q$    & disjunction \\
      \verb'~' $P$      & $\neg P$      & negation \\
      \tt $P$ --> $Q$   & $P\imp Q$     & implication \\
      \tt $P$ <-> $Q$   & $P\bimp Q$    & bi-implication \\
      \tt ALL $x\,y\,z$ .\ $P$  & $\forall x\,y\,z.P$   & for all \\
      \tt EX  $x\,y\,z$ .\ $P$  & $\exists x\,y\,z.P$   & there exists
   \end{tabular}
\]
To illustrate the notation, consider two axioms for first-order logic:
$$ \List{P; Q} \Imp P\conj Q                 \eqno(\conj I) $$
$$ \List{\exists x.P(x); \Forall x. P(x)\imp Q} \Imp Q \eqno(\exists E) $$
$({\conj}I)$ translates into {\sc ascii} characters as
\begin{ttbox}
[| ?P; ?Q |] ==> ?P & ?Q
\end{ttbox}
The schematic variables let unification instantiate the rule.  To avoid
cluttering logic definitions with question marks, Isabelle converts any
free variables in a rule to schematic variables; we normally declare
$({\conj}I)$ as
\begin{ttbox}
[| P; Q |] ==> P & Q
\end{ttbox}
This variables convention agrees with the treatment of variables in goals.
Free variables in a goal remain fixed throughout the proof.  After the
proof is finished, Isabelle converts them to scheme variables in the
resulting theorem.  Scheme variables in a goal may be replaced by terms
during the proof, supporting answer extraction, program synthesis, and so
forth.

For a final example, the rule $(\exists E)$ is rendered in {\sc ascii} as
\begin{ttbox}
[| EX x.P(x);  !!x. P(x) ==> Q |] ==> Q
\end{ttbox}


\subsection{Basic operations on theorems}
\index{theorems!basic operations on|bold}
\index{LCF system}
Meta-level theorems have the \ML{} type \mltydx{thm}.  They represent the
theorems and inference rules of object-logics.  Isabelle's meta-logic is
implemented using the {\sc lcf} approach: each meta-level inference rule is
represented by a function from theorems to theorems.  Object-level rules
are taken as axioms.

The main theorem printing commands are {\tt prth}, {\tt prths} and~{\tt
  prthq}.  Of the other operations on theorems, most useful are {\tt RS}
and {\tt RSN}, which perform resolution.

\index{theorems!printing of}
\begin{ttdescription}
\item[\ttindex{prth} {\it thm};]
  pretty-prints {\it thm\/} at the terminal.

\item[\ttindex{prths} {\it thms};]
  pretty-prints {\it thms}, a list of theorems.

\item[\ttindex{prthq} {\it thmq};]
  pretty-prints {\it thmq}, a sequence of theorems; this is useful for
  inspecting the output of a tactic.

\item[$thm1$ RS $thm2$] \index{*RS} 
  resolves the conclusion of $thm1$ with the first premise of~$thm2$.

\item[$thm1$ RSN $(i,thm2)$] \index{*RSN} 
  resolves the conclusion of $thm1$ with the $i$th premise of~$thm2$.

\item[\ttindex{standard} $thm$]  
  puts $thm$ into a standard format.  It also renames schematic variables
  to have subscript zero, improving readability and reducing subscript
  growth.
\end{ttdescription}
The rules of a theory are normally bound to \ML\ identifiers.  Suppose we
are running an Isabelle session containing theory~\FOL, natural deduction
first-order logic.\footnote{For a listing of the \FOL{} rules and their
  \ML{} names, turn to
\iflabelundefined{fol-rules}{{\em Isabelle's Object-Logics}}%
           {page~\pageref{fol-rules}}.}
Let us try an example given in~\S\ref{joining}.  We
first print \tdx{mp}, which is the rule~$({\imp}E)$, then resolve it with
itself.
\begin{ttbox} 
prth mp; 
{\out [| ?P --> ?Q; ?P |] ==> ?Q}
{\out val it = "[| ?P --> ?Q; ?P |] ==> ?Q" : thm}
prth (mp RS mp);
{\out [| ?P1 --> ?P --> ?Q; ?P1; ?P |] ==> ?Q}
{\out val it = "[| ?P1 --> ?P --> ?Q; ?P1; ?P |] ==> ?Q" : thm}
\end{ttbox}
User input appears in {\footnotesize\tt typewriter characters}, and output
appears in{\out slanted typewriter characters}.  \ML's response {\out val
  }~\ldots{} is compiler-dependent and will sometimes be suppressed.  This
session illustrates two formats for the display of theorems.  Isabelle's
top-level displays theorems as \ML{} values, enclosed in quotes.  Printing
commands like {\tt prth} omit the quotes and the surrounding {\tt val
  \ldots :\ thm}.  Ignoring their side-effects, the commands are identity
functions.

To contrast {\tt RS} with {\tt RSN}, we resolve
\tdx{conjunct1}, which stands for~$(\conj E1)$, with~\tdx{mp}.
\begin{ttbox} 
conjunct1 RS mp;
{\out val it = "[| (?P --> ?Q) & ?Q1; ?P |] ==> ?Q" : thm}
conjunct1 RSN (2,mp);
{\out val it = "[| ?P --> ?Q; ?P & ?Q1 |] ==> ?Q" : thm}
\end{ttbox}
These correspond to the following proofs:
\[ \infer[({\imp}E)]{Q}{\infer[({\conj}E1)]{P\imp Q}{(P\imp Q)\conj Q@1} & P}
   \qquad
   \infer[({\imp}E)]{Q}{P\imp Q & \infer[({\conj}E1)]{P}{P\conj Q@1}} 
\]
%
Rules can be derived by pasting other rules together.  Let us join
\tdx{spec}, which stands for~$(\forall E)$, with {\tt mp} and {\tt
  conjunct1}.  In \ML{}, the identifier~{\tt it} denotes the value just
printed.
\begin{ttbox} 
spec;
{\out val it = "ALL x. ?P(x) ==> ?P(?x)" : thm}
it RS mp;
{\out val it = "[| ALL x. ?P3(x) --> ?Q2(x); ?P3(?x1) |] ==>}
{\out           ?Q2(?x1)" : thm}
it RS conjunct1;
{\out val it = "[| ALL x. ?P4(x) --> ?P6(x) & ?Q5(x); ?P4(?x2) |] ==>}
{\out           ?P6(?x2)" : thm}
standard it;
{\out val it = "[| ALL x. ?P(x) --> ?Pa(x) & ?Q(x); ?P(?x) |] ==>}
{\out           ?Pa(?x)" : thm}
\end{ttbox}
By resolving $(\forall E)$ with (${\imp}E)$ and (${\conj}E1)$, we have
derived a destruction rule for formulae of the form $\forall x.
P(x)\imp(Q(x)\conj R(x))$.  Used with destruct-resolution, such specialized
rules provide a way of referring to particular assumptions.
\index{assumptions!use of}

\subsection{*Flex-flex constraints} \label{flexflex}
\index{flex-flex constraints|bold}\index{unknowns!function}
In higher-order unification, {\bf flex-flex} equations are those where both
sides begin with a function unknown, such as $\Var{f}(0)\qeq\Var{g}(0)$.
They admit a trivial unifier, here $\Var{f}\equiv \lambda x.\Var{a}$ and
$\Var{g}\equiv \lambda y.\Var{a}$, where $\Var{a}$ is a new unknown.  They
admit many other unifiers, such as $\Var{f} \equiv \lambda x.\Var{g}(0)$
and $\{\Var{f} \equiv \lambda x.x,\, \Var{g} \equiv \lambda x.0\}$.  Huet's
procedure does not enumerate the unifiers; instead, it retains flex-flex
equations as constraints on future unifications.  Flex-flex constraints
occasionally become attached to a proof state; more frequently, they appear
during use of {\tt RS} and~{\tt RSN}:
\begin{ttbox} 
refl;
{\out val it = "?a = ?a" : thm}
exI;
{\out val it = "?P(?x) ==> EX x. ?P(x)" : thm}
refl RS exI;
{\out val it = "?a3(?x) =?= ?a2(?x) ==> EX x. ?a3(x) = ?a2(x)" : thm}
\end{ttbox}

\noindent
Renaming variables, this is $\exists x.\Var{f}(x)=\Var{g}(x)$ with
the constraint ${\Var{f}(\Var{u})\qeq\Var{g}(\Var{u})}$.  Instances
satisfying the constraint include $\exists x.\Var{f}(x)=\Var{f}(x)$ and
$\exists x.x=\Var{u}$.  Calling \ttindex{flexflex_rule} removes all
constraints by applying the trivial unifier:\index{*prthq}
\begin{ttbox} 
prthq (flexflex_rule it);
{\out EX x. ?a4 = ?a4}
\end{ttbox} 
Isabelle simplifies flex-flex equations to eliminate redundant bound
variables.  In $\lambda x\,y.\Var{f}(k(y),x) \qeq \lambda x\,y.\Var{g}(y)$,
there is no bound occurrence of~$x$ on the right side; thus, there will be
none on the left in a common instance of these terms.  Choosing a new
variable~$\Var{h}$, Isabelle assigns $\Var{f}\equiv \lambda u\,v.?h(u)$,
simplifying the left side to $\lambda x\,y.\Var{h}(k(y))$.  Dropping $x$
from the equation leaves $\lambda y.\Var{h}(k(y)) \qeq \lambda
y.\Var{g}(y)$.  By $\eta$-conversion, this simplifies to the assignment
$\Var{g}\equiv\lambda y.?h(k(y))$.

\begin{warn}
\ttindex{RS} and \ttindex{RSN} fail (by raising exception {\tt THM}) unless
the resolution delivers {\bf exactly one} resolvent.  For multiple results,
use \ttindex{RL} and \ttindex{RLN}, which operate on theorem lists.  The
following example uses \ttindex{read_instantiate} to create an instance
of \tdx{refl} containing no schematic variables.
\begin{ttbox} 
val reflk = read_instantiate [("a","k")] refl;
{\out val reflk = "k = k" : thm}
\end{ttbox}

\noindent
A flex-flex constraint is no longer possible; resolution does not find a
unique unifier:
\begin{ttbox} 
reflk RS exI;
{\out uncaught exception THM}
\end{ttbox}
Using \ttindex{RL} this time, we discover that there are four unifiers, and
four resolvents:
\begin{ttbox} 
[reflk] RL [exI];
{\out val it = ["EX x. x = x", "EX x. k = x",}
{\out           "EX x. x = k", "EX x. k = k"] : thm list}
\end{ttbox} 
\end{warn}

\index{forward proof|)}

\section{Backward proof}
Although {\tt RS} and {\tt RSN} are fine for simple forward reasoning,
large proofs require tactics.  Isabelle provides a suite of commands for
conducting a backward proof using tactics.

\subsection{The basic tactics}
The tactics {\tt assume_tac}, {\tt
resolve_tac}, {\tt eresolve_tac}, and {\tt dresolve_tac} suffice for most
single-step proofs.  Although {\tt eresolve_tac} and {\tt dresolve_tac} are
not strictly necessary, they simplify proofs involving elimination and
destruction rules.  All the tactics act on a subgoal designated by a
positive integer~$i$, failing if~$i$ is out of range.  The resolution
tactics try their list of theorems in left-to-right order.

\begin{ttdescription}
\item[\ttindex{assume_tac} {\it i}] \index{tactics!assumption}
  is the tactic that attempts to solve subgoal~$i$ by assumption.  Proof by
  assumption is not a trivial step; it can falsify other subgoals by
  instantiating shared variables.  There may be several ways of solving the
  subgoal by assumption.

\item[\ttindex{resolve_tac} {\it thms} {\it i}]\index{tactics!resolution}
  is the basic resolution tactic, used for most proof steps.  The $thms$
  represent object-rules, which are resolved against subgoal~$i$ of the
  proof state.  For each rule, resolution forms next states by unifying the
  conclusion with the subgoal and inserting instantiated premises in its
  place.  A rule can admit many higher-order unifiers.  The tactic fails if
  none of the rules generates next states.

\item[\ttindex{eresolve_tac} {\it thms} {\it i}] \index{elim-resolution}
  performs elim-resolution.  Like {\tt resolve_tac~{\it thms}~{\it i\/}}
  followed by {\tt assume_tac~{\it i}}, it applies a rule then solves its
  first premise by assumption.  But {\tt eresolve_tac} additionally deletes
  that assumption from any subgoals arising from the resolution.

\item[\ttindex{dresolve_tac} {\it thms} {\it i}]
  \index{forward proof}\index{destruct-resolution}
  performs destruct-resolution with the~$thms$, as described
  in~\S\ref{destruct}.  It is useful for forward reasoning from the
  assumptions.
\end{ttdescription}

\subsection{Commands for backward proof}
\index{proofs!commands for}
Tactics are normally applied using the subgoal module, which maintains a
proof state and manages the proof construction.  It allows interactive
backtracking through the proof space, going away to prove lemmas, etc.; of
its many commands, most important are the following:
\begin{ttdescription}
\item[\ttindex{goal} {\it theory} {\it formula}; ] 
begins a new proof, where $theory$ is usually an \ML\ identifier
and the {\it formula\/} is written as an \ML\ string.

\item[\ttindex{by} {\it tactic}; ] 
applies the {\it tactic\/} to the current proof
state, raising an exception if the tactic fails.

\item[\ttindex{undo}(); ]
  reverts to the previous proof state.  Undo can be repeated but cannot be
  undone.  Do not omit the parentheses; typing {\tt\ \ undo;\ \ } merely
  causes \ML\ to echo the value of that function.

\item[\ttindex{result}();]
returns the theorem just proved, in a standard format.  It fails if
unproved subgoals are left, etc.

\item[\ttindex{qed} {\it name};] is the usual way of ending a proof.
  It gets the theorem using {\tt result}, stores it in Isabelle's
  theorem database and binds it to an \ML{} identifier.

\end{ttdescription}
The commands and tactics given above are cumbersome for interactive use.
Although our examples will use the full commands, you may prefer Isabelle's
shortcuts:
\begin{center} \tt
\index{*br} \index{*be} \index{*bd} \index{*ba}
\begin{tabular}{l@{\qquad\rm abbreviates\qquad}l}
    ba {\it i};           & by (assume_tac {\it i}); \\

    br {\it thm} {\it i}; & by (resolve_tac [{\it thm}] {\it i}); \\

    be {\it thm} {\it i}; & by (eresolve_tac [{\it thm}] {\it i}); \\

    bd {\it thm} {\it i}; & by (dresolve_tac [{\it thm}] {\it i}); 
\end{tabular}
\end{center}

\subsection{A trivial example in propositional logic}
\index{examples!propositional}

Directory {\tt FOL} of the Isabelle distribution defines the theory of
first-order logic.  Let us try the example from \S\ref{prop-proof},
entering the goal $P\disj P\imp P$ in that theory.\footnote{To run these
  examples, see the file {\tt FOL/ex/intro.ML}.  The files {\tt README} and
  {\tt Makefile} on the directories {\tt Pure} and {\tt FOL} explain how to
  build first-order logic.}
\begin{ttbox}
goal FOL.thy "P|P --> P"; 
{\out Level 0} 
{\out P | P --> P} 
{\out 1. P | P --> P} 
\end{ttbox}\index{level of a proof}
Isabelle responds by printing the initial proof state, which has $P\disj
P\imp P$ as the main goal and the only subgoal.  The {\bf level} of the
state is the number of {\tt by} commands that have been applied to reach
it.  We now use \ttindex{resolve_tac} to apply the rule \tdx{impI},
or~$({\imp}I)$, to subgoal~1:
\begin{ttbox}
by (resolve_tac [impI] 1); 
{\out Level 1} 
{\out P | P --> P} 
{\out 1. P | P ==> P}
\end{ttbox}
In the new proof state, subgoal~1 is $P$ under the assumption $P\disj P$.
(The meta-implication {\tt==>} indicates assumptions.)  We apply
\tdx{disjE}, or~(${\disj}E)$, to that subgoal:
\begin{ttbox}
by (resolve_tac [disjE] 1); 
{\out Level 2} 
{\out P | P --> P} 
{\out 1. P | P ==> ?P1 | ?Q1} 
{\out 2. [| P | P; ?P1 |] ==> P} 
{\out 3. [| P | P; ?Q1 |] ==> P}
\end{ttbox}
At Level~2 there are three subgoals, each provable by assumption.  We
deviate from~\S\ref{prop-proof} by tackling subgoal~3 first, using
\ttindex{assume_tac}.  This affects subgoal~1, updating {\tt?Q1} to~{\tt
  P}.
\begin{ttbox}
by (assume_tac 3); 
{\out Level 3} 
{\out P | P --> P} 
{\out 1. P | P ==> ?P1 | P} 
{\out 2. [| P | P; ?P1 |] ==> P}
\end{ttbox}
Next we tackle subgoal~2, instantiating {\tt?P1} to~{\tt P} in subgoal~1.
\begin{ttbox}
by (assume_tac 2); 
{\out Level 4} 
{\out P | P --> P} 
{\out 1. P | P ==> P | P}
\end{ttbox}
Lastly we prove the remaining subgoal by assumption:
\begin{ttbox}
by (assume_tac 1); 
{\out Level 5} 
{\out P | P --> P} 
{\out No subgoals!}
\end{ttbox}
Isabelle tells us that there are no longer any subgoals: the proof is
complete.  Calling {\tt qed} stores the theorem.
\begin{ttbox}
qed "mythm";
{\out val mythm = "?P | ?P --> ?P" : thm} 
\end{ttbox}
Isabelle has replaced the free variable~{\tt P} by the scheme
variable~{\tt?P}\@.  Free variables in the proof state remain fixed
throughout the proof.  Isabelle finally converts them to scheme variables
so that the resulting theorem can be instantiated with any formula.

As an exercise, try doing the proof as in \S\ref{prop-proof}, observing how
instantiations affect the proof state.


\subsection{Part of a distributive law}
\index{examples!propositional}
To demonstrate the tactics \ttindex{eresolve_tac}, \ttindex{dresolve_tac}
and the tactical {\tt REPEAT}, let us prove part of the distributive
law 
\[ (P\conj Q)\disj R \,\bimp\, (P\disj R)\conj (Q\disj R). \]
We begin by stating the goal to Isabelle and applying~$({\imp}I)$ to it:
\begin{ttbox}
goal FOL.thy "(P & Q) | R  --> (P | R)";
{\out Level 0}
{\out P & Q | R --> P | R}
{\out  1. P & Q | R --> P | R}
\ttbreak
by (resolve_tac [impI] 1);
{\out Level 1}
{\out P & Q | R --> P | R}
{\out  1. P & Q | R ==> P | R}
\end{ttbox}
Previously we applied~(${\disj}E)$ using {\tt resolve_tac}, but 
\ttindex{eresolve_tac} deletes the assumption after use.  The resulting proof
state is simpler.
\begin{ttbox}
by (eresolve_tac [disjE] 1);
{\out Level 2}
{\out P & Q | R --> P | R}
{\out  1. P & Q ==> P | R}
{\out  2. R ==> P | R}
\end{ttbox}
Using \ttindex{dresolve_tac}, we can apply~(${\conj}E1)$ to subgoal~1,
replacing the assumption $P\conj Q$ by~$P$.  Normally we should apply the
rule~(${\conj}E)$, given in~\S\ref{destruct}.  That is an elimination rule
and requires {\tt eresolve_tac}; it would replace $P\conj Q$ by the two
assumptions~$P$ and~$Q$.  Because the present example does not need~$Q$, we
may try out {\tt dresolve_tac}.
\begin{ttbox}
by (dresolve_tac [conjunct1] 1);
{\out Level 3}
{\out P & Q | R --> P | R}
{\out  1. P ==> P | R}
{\out  2. R ==> P | R}
\end{ttbox}
The next two steps apply~(${\disj}I1$) and~(${\disj}I2$) in an obvious manner.
\begin{ttbox}
by (resolve_tac [disjI1] 1);
{\out Level 4}
{\out P & Q | R --> P | R}
{\out  1. P ==> P}
{\out  2. R ==> P | R}
\ttbreak
by (resolve_tac [disjI2] 2);
{\out Level 5}
{\out P & Q | R --> P | R}
{\out  1. P ==> P}
{\out  2. R ==> R}
\end{ttbox}
Two calls of {\tt assume_tac} can finish the proof.  The
tactical~\ttindex{REPEAT} here expresses a tactic that calls {\tt assume_tac~1}
as many times as possible.  We can restrict attention to subgoal~1 because
the other subgoals move up after subgoal~1 disappears.
\begin{ttbox}
by (REPEAT (assume_tac 1));
{\out Level 6}
{\out P & Q | R --> P | R}
{\out No subgoals!}
\end{ttbox}


\section{Quantifier reasoning}
\index{quantifiers}\index{parameters}\index{unknowns}\index{unknowns!function}
This section illustrates how Isabelle enforces quantifier provisos.
Suppose that $x$, $y$ and~$z$ are parameters of a subgoal.  Quantifier
rules create terms such as~$\Var{f}(x,z)$, where~$\Var{f}$ is a function
unknown.  Instantiating $\Var{f}$ to $\lambda x\,z.t$ has the effect of
replacing~$\Var{f}(x,z)$ by~$t$, where the term~$t$ may contain free
occurrences of~$x$ and~$z$.  On the other hand, no instantiation
of~$\Var{f}$ can replace~$\Var{f}(x,z)$ by a term containing free
occurrences of~$y$, since parameters are bound variables.

\subsection{Two quantifier proofs: a success and a failure}
\index{examples!with quantifiers}
Let us contrast a proof of the theorem $\forall x.\exists y.x=y$ with an
attempted proof of the non-theorem $\exists y.\forall x.x=y$.  The former
proof succeeds, and the latter fails, because of the scope of quantified
variables~\cite{paulson-found}.  Unification helps even in these trivial
proofs.  In $\forall x.\exists y.x=y$ the $y$ that `exists' is simply $x$,
but we need never say so.  This choice is forced by the reflexive law for
equality, and happens automatically.

\paragraph{The successful proof.}
The proof of $\forall x.\exists y.x=y$ demonstrates the introduction rules
$(\forall I)$ and~$(\exists I)$.  We state the goal and apply $(\forall I)$:
\begin{ttbox}
goal FOL.thy "ALL x. EX y. x=y";
{\out Level 0}
{\out ALL x. EX y. x = y}
{\out  1. ALL x. EX y. x = y}
\ttbreak
by (resolve_tac [allI] 1);
{\out Level 1}
{\out ALL x. EX y. x = y}
{\out  1. !!x. EX y. x = y}
\end{ttbox}
The variable~{\tt x} is no longer universally quantified, but is a
parameter in the subgoal; thus, it is universally quantified at the
meta-level.  The subgoal must be proved for all possible values of~{\tt x}.

To remove the existential quantifier, we apply the rule $(\exists I)$:
\begin{ttbox}
by (resolve_tac [exI] 1);
{\out Level 2}
{\out ALL x. EX y. x = y}
{\out  1. !!x. x = ?y1(x)}
\end{ttbox}
The bound variable {\tt y} has become {\tt?y1(x)}.  This term consists of
the function unknown~{\tt?y1} applied to the parameter~{\tt x}.
Instances of {\tt?y1(x)} may or may not contain~{\tt x}.  We resolve the
subgoal with the reflexivity axiom.
\begin{ttbox}
by (resolve_tac [refl] 1);
{\out Level 3}
{\out ALL x. EX y. x = y}
{\out No subgoals!}
\end{ttbox}
Let us consider what has happened in detail.  The reflexivity axiom is
lifted over~$x$ to become $\Forall x.\Var{f}(x)=\Var{f}(x)$, which is
unified with $\Forall x.x=\Var{y@1}(x)$.  The function unknowns $\Var{f}$
and~$\Var{y@1}$ are both instantiated to the identity function, and
$x=\Var{y@1}(x)$ collapses to~$x=x$ by $\beta$-reduction.

\paragraph{The unsuccessful proof.}
We state the goal $\exists y.\forall x.x=y$, which is not a theorem, and
try~$(\exists I)$:
\begin{ttbox}
goal FOL.thy "EX y. ALL x. x=y";
{\out Level 0}
{\out EX y. ALL x. x = y}
{\out  1. EX y. ALL x. x = y}
\ttbreak
by (resolve_tac [exI] 1);
{\out Level 1}
{\out EX y. ALL x. x = y}
{\out  1. ALL x. x = ?y}
\end{ttbox}
The unknown {\tt ?y} may be replaced by any term, but this can never
introduce another bound occurrence of~{\tt x}.  We now apply~$(\forall I)$:
\begin{ttbox}
by (resolve_tac [allI] 1);
{\out Level 2}
{\out EX y. ALL x. x = y}
{\out  1. !!x. x = ?y}
\end{ttbox}
Compare our position with the previous Level~2.  Instead of {\tt?y1(x)} we
have~{\tt?y}, whose instances may not contain the bound variable~{\tt x}.
The reflexivity axiom does not unify with subgoal~1.
\begin{ttbox}
by (resolve_tac [refl] 1);
{\out by: tactic failed}
\end{ttbox}
There can be no proof of $\exists y.\forall x.x=y$ by the soundness of
first-order logic.  I have elsewhere proved the faithfulness of Isabelle's
encoding of first-order logic~\cite{paulson-found}; there could, of course, be
faults in the implementation.


\subsection{Nested quantifiers}
\index{examples!with quantifiers}
Multiple quantifiers create complex terms.  Proving 
\[ (\forall x\,y.P(x,y)) \imp (\forall z\,w.P(w,z)) \] 
will demonstrate how parameters and unknowns develop.  If they appear in
the wrong order, the proof will fail.

This section concludes with a demonstration of {\tt REPEAT}
and~{\tt ORELSE}.  
\begin{ttbox}
goal FOL.thy "(ALL x y.P(x,y))  -->  (ALL z w.P(w,z))";
{\out Level 0}
{\out (ALL x y. P(x,y)) --> (ALL z w. P(w,z))}
{\out  1. (ALL x y. P(x,y)) --> (ALL z w. P(w,z))}
\ttbreak
by (resolve_tac [impI] 1);
{\out Level 1}
{\out (ALL x y. P(x,y)) --> (ALL z w. P(w,z))}
{\out  1. ALL x y. P(x,y) ==> ALL z w. P(w,z)}
\end{ttbox}

\paragraph{The wrong approach.}
Using {\tt dresolve_tac}, we apply the rule $(\forall E)$, bound to the
\ML\ identifier \tdx{spec}.  Then we apply $(\forall I)$.
\begin{ttbox}
by (dresolve_tac [spec] 1);
{\out Level 2}
{\out (ALL x y. P(x,y)) --> (ALL z w. P(w,z))}
{\out  1. ALL y. P(?x1,y) ==> ALL z w. P(w,z)}
\ttbreak
by (resolve_tac [allI] 1);
{\out Level 3}
{\out (ALL x y. P(x,y)) --> (ALL z w. P(w,z))}
{\out  1. !!z. ALL y. P(?x1,y) ==> ALL w. P(w,z)}
\end{ttbox}
The unknown {\tt ?x1} and the parameter {\tt z} have appeared.  We again
apply $(\forall E)$ and~$(\forall I)$.
\begin{ttbox}
by (dresolve_tac [spec] 1);
{\out Level 4}
{\out (ALL x y. P(x,y)) --> (ALL z w. P(w,z))}
{\out  1. !!z. P(?x1,?y3(z)) ==> ALL w. P(w,z)}
\ttbreak
by (resolve_tac [allI] 1);
{\out Level 5}
{\out (ALL x y. P(x,y)) --> (ALL z w. P(w,z))}
{\out  1. !!z w. P(?x1,?y3(z)) ==> P(w,z)}
\end{ttbox}
The unknown {\tt ?y3} and the parameter {\tt w} have appeared.  Each
unknown is applied to the parameters existing at the time of its creation;
instances of~{\tt ?x1} cannot contain~{\tt z} or~{\tt w}, while instances
of {\tt?y3(z)} can only contain~{\tt z}.  Due to the restriction on~{\tt ?x1},
proof by assumption will fail.
\begin{ttbox}
by (assume_tac 1);
{\out by: tactic failed}
{\out uncaught exception ERROR}
\end{ttbox}

\paragraph{The right approach.}
To do this proof, the rules must be applied in the correct order.
Parameters should be created before unknowns.  The
\ttindex{choplev} command returns to an earlier stage of the proof;
let us return to the result of applying~$({\imp}I)$:
\begin{ttbox}
choplev 1;
{\out Level 1}
{\out (ALL x y. P(x,y)) --> (ALL z w. P(w,z))}
{\out  1. ALL x y. P(x,y) ==> ALL z w. P(w,z)}
\end{ttbox}
Previously we made the mistake of applying $(\forall E)$ before $(\forall I)$.
\begin{ttbox}
by (resolve_tac [allI] 1);
{\out Level 2}
{\out (ALL x y. P(x,y)) --> (ALL z w. P(w,z))}
{\out  1. !!z. ALL x y. P(x,y) ==> ALL w. P(w,z)}
\ttbreak
by (resolve_tac [allI] 1);
{\out Level 3}
{\out (ALL x y. P(x,y)) --> (ALL z w. P(w,z))}
{\out  1. !!z w. ALL x y. P(x,y) ==> P(w,z)}
\end{ttbox}
The parameters {\tt z} and~{\tt w} have appeared.  We now create the
unknowns:
\begin{ttbox}
by (dresolve_tac [spec] 1);
{\out Level 4}
{\out (ALL x y. P(x,y)) --> (ALL z w. P(w,z))}
{\out  1. !!z w. ALL y. P(?x3(z,w),y) ==> P(w,z)}
\ttbreak
by (dresolve_tac [spec] 1);
{\out Level 5}
{\out (ALL x y. P(x,y)) --> (ALL z w. P(w,z))}
{\out  1. !!z w. P(?x3(z,w),?y4(z,w)) ==> P(w,z)}
\end{ttbox}
Both {\tt?x3(z,w)} and~{\tt?y4(z,w)} could become any terms containing {\tt
z} and~{\tt w}:
\begin{ttbox}
by (assume_tac 1);
{\out Level 6}
{\out (ALL x y. P(x,y)) --> (ALL z w. P(w,z))}
{\out No subgoals!}
\end{ttbox}

\paragraph{A one-step proof using tacticals.}
\index{tacticals} \index{examples!of tacticals} 

Repeated application of rules can be effective, but the rules should be
attempted in the correct order.  Let us return to the original goal using
\ttindex{choplev}:
\begin{ttbox}
choplev 0;
{\out Level 0}
{\out (ALL x y. P(x,y)) --> (ALL z w. P(w,z))}
{\out  1. (ALL x y. P(x,y)) --> (ALL z w. P(w,z))}
\end{ttbox}
As we have just seen, \tdx{allI} should be attempted
before~\tdx{spec}, while \ttindex{assume_tac} generally can be
attempted first.  Such priorities can easily be expressed
using~\ttindex{ORELSE}, and repeated using~\ttindex{REPEAT}.
\begin{ttbox}
by (REPEAT (assume_tac 1 ORELSE resolve_tac [impI,allI] 1
     ORELSE dresolve_tac [spec] 1));
{\out Level 1}
{\out (ALL x y. P(x,y)) --> (ALL z w. P(w,z))}
{\out No subgoals!}
\end{ttbox}


\subsection{A realistic quantifier proof}
\index{examples!with quantifiers}
To see the practical use of parameters and unknowns, let us prove half of
the equivalence 
\[ (\forall x. P(x) \imp Q) \,\bimp\, ((\exists x. P(x)) \imp Q). \]
We state the left-to-right half to Isabelle in the normal way.
Since $\imp$ is nested to the right, $({\imp}I)$ can be applied twice; we
use {\tt REPEAT}:
\begin{ttbox}
goal FOL.thy "(ALL x.P(x) --> Q) --> (EX x.P(x)) --> Q";
{\out Level 0}
{\out (ALL x. P(x) --> Q) --> (EX x. P(x)) --> Q}
{\out  1. (ALL x. P(x) --> Q) --> (EX x. P(x)) --> Q}
\ttbreak
by (REPEAT (resolve_tac [impI] 1));
{\out Level 1}
{\out (ALL x. P(x) --> Q) --> (EX x. P(x)) --> Q}
{\out  1. [| ALL x. P(x) --> Q; EX x. P(x) |] ==> Q}
\end{ttbox}
We can eliminate the universal or the existential quantifier.  The
existential quantifier should be eliminated first, since this creates a
parameter.  The rule~$(\exists E)$ is bound to the
identifier~\tdx{exE}.
\begin{ttbox}
by (eresolve_tac [exE] 1);
{\out Level 2}
{\out (ALL x. P(x) --> Q) --> (EX x. P(x)) --> Q}
{\out  1. !!x. [| ALL x. P(x) --> Q; P(x) |] ==> Q}
\end{ttbox}
The only possibility now is $(\forall E)$, a destruction rule.  We use 
\ttindex{dresolve_tac}, which discards the quantified assumption; it is
only needed once.
\begin{ttbox}
by (dresolve_tac [spec] 1);
{\out Level 3}
{\out (ALL x. P(x) --> Q) --> (EX x. P(x)) --> Q}
{\out  1. !!x. [| P(x); P(?x3(x)) --> Q |] ==> Q}
\end{ttbox}
Because we applied $(\exists E)$ before $(\forall E)$, the unknown
term~{\tt?x3(x)} may depend upon the parameter~{\tt x}.

Although $({\imp}E)$ is a destruction rule, it works with 
\ttindex{eresolve_tac} to perform backward chaining.  This technique is
frequently useful.  
\begin{ttbox}
by (eresolve_tac [mp] 1);
{\out Level 4}
{\out (ALL x. P(x) --> Q) --> (EX x. P(x)) --> Q}
{\out  1. !!x. P(x) ==> P(?x3(x))}
\end{ttbox}
The tactic has reduced~{\tt Q} to~{\tt P(?x3(x))}, deleting the
implication.  The final step is trivial, thanks to the occurrence of~{\tt x}.
\begin{ttbox}
by (assume_tac 1);
{\out Level 5}
{\out (ALL x. P(x) --> Q) --> (EX x. P(x)) --> Q}
{\out No subgoals!}
\end{ttbox}


\subsection{The classical reasoner}
\index{classical reasoner}
Although Isabelle cannot compete with fully automatic theorem provers, it
provides enough automation to tackle substantial examples.  The classical
reasoner can be set up for any classical natural deduction logic;
see \iflabelundefined{chap:classical}{the {\em Reference Manual\/}}%
        {Chap.\ts\ref{chap:classical}}. 

Rules are packaged into {\bf classical sets}.  The classical reasoner
provides several tactics, which apply rules using naive algorithms.
Unification handles quantifiers as shown above.  The most useful tactic
is~\ttindex{Blast_tac}.  

Let us solve problems~40 and~60 of Pelletier~\cite{pelletier86}.  (The
backslashes~\hbox{\verb|\|\ldots\verb|\|} are an \ML{} string escape
sequence, to break the long string over two lines.)
\begin{ttbox}
goal FOL.thy "(EX y. ALL x. J(y,x) <-> ~J(x,x))  \ttback
\ttback       -->  ~ (ALL x. EX y. ALL z. J(z,y) <-> ~ J(z,x))";
{\out Level 0}
{\out (EX y. ALL x. J(y,x) <-> ~J(x,x)) -->}
{\out ~(ALL x. EX y. ALL z. J(z,y) <-> ~J(z,x))}
{\out  1. (EX y. ALL x. J(y,x) <-> ~J(x,x)) -->}
{\out     ~(ALL x. EX y. ALL z. J(z,y) <-> ~J(z,x))}
\end{ttbox}
\ttindex{Blast_tac} proves subgoal~1 at a stroke.
\begin{ttbox}
by (Blast_tac 1);
{\out Depth = 0}
{\out Depth = 1}
{\out Level 1}
{\out (EX y. ALL x. J(y,x) <-> ~J(x,x)) -->}
{\out ~(ALL x. EX y. ALL z. J(z,y) <-> ~J(z,x))}
{\out No subgoals!}
\end{ttbox}
Sceptics may examine the proof by calling the package's single-step
tactics, such as~{\tt step_tac}.  This would take up much space, however,
so let us proceed to the next example:
\begin{ttbox}
goal FOL.thy "ALL x. P(x,f(x)) <-> \ttback
\ttback       (EX y. (ALL z. P(z,y) --> P(z,f(x))) & P(x,y))";
{\out Level 0}
{\out ALL x. P(x,f(x)) <-> (EX y. (ALL z. P(z,y) --> P(z,f(x))) & P(x,y))}
{\out  1. ALL x. P(x,f(x)) <->}
{\out     (EX y. (ALL z. P(z,y) --> P(z,f(x))) & P(x,y))}
\end{ttbox}
Again, subgoal~1 succumbs immediately.
\begin{ttbox}
by (Blast_tac 1);
{\out Depth = 0}
{\out Depth = 1}
{\out Level 1}
{\out ALL x. P(x,f(x)) <-> (EX y. (ALL z. P(z,y) --> P(z,f(x))) & P(x,y))}
{\out No subgoals!}
\end{ttbox}
The classical reasoner is not restricted to the usual logical connectives.
The natural deduction rules for unions and intersections resemble those for
disjunction and conjunction.  The rules for infinite unions and
intersections resemble those for quantifiers.  Given such rules, the classical
reasoner is effective for reasoning in set theory.
  

%% $Id$
\part{Advanced Methods}
Before continuing, it might be wise to try some of your own examples in
Isabelle, reinforcing your knowledge of the basic functions.

Look through {\em Isabelle's Object-Logics\/} and try proving some simple
theorems.  You probably should begin with first-order logic ({\tt FOL}
or~{\tt LK}).  Try working some of the examples provided, and others from
the literature.  Set theory~({\tt ZF}) and Constructive Type Theory~({\tt
  CTT}) form a richer world for mathematical reasoning and, again, many
examples are in the literature.  Higher-order logic~({\tt HOL}) is
Isabelle's most sophisticated logic because its types and functions are
identified with those of the meta-logic.

Choose a logic that you already understand.  Isabelle is a proof
tool, not a teaching tool; if you do not know how to do a particular proof
on paper, then you certainly will not be able to do it on the machine.
Even experienced users plan large proofs on paper.

We have covered only the bare essentials of Isabelle, but enough to perform
substantial proofs.  By occasionally dipping into the {\em Reference
Manual}, you can learn additional tactics, subgoal commands and tacticals.


\section{Deriving rules in Isabelle}
\index{rules!derived}
A mathematical development goes through a progression of stages.  Each
stage defines some concepts and derives rules about them.  We shall see how
to derive rules, perhaps involving definitions, using Isabelle.  The
following section will explain how to declare types, constants, rules and
definitions.


\subsection{Deriving a rule using tactics and meta-level assumptions} 
\label{deriving-example}
\index{examples!of deriving rules}\index{assumptions!of main goal}

The subgoal module supports the derivation of rules, as discussed in
\S\ref{deriving}.  The \ttindex{goal} command, when supplied a goal of the
form $\List{\theta@1; \ldots; \theta@k} \Imp \phi$, creates $\phi\Imp\phi$
as the initial proof state and returns a list consisting of the theorems
${\theta@i\;[\theta@i]}$, for $i=1$, \ldots,~$k$.  These meta-assumptions
are also recorded internally, allowing {\tt result} to discharge them
in the original order.

Let us derive $\conj$ elimination using Isabelle.
Until now, calling {\tt goal} has returned an empty list, which we have
thrown away.  In this example, the list contains the two premises of the
rule.  We bind them to the \ML\ identifiers {\tt major} and {\tt
minor}:\footnote{Some ML compilers will print a message such as {\em
binding not exhaustive}.  This warns that {\tt goal} must return a
2-element list.  Otherwise, the pattern-match will fail; ML will
raise exception \xdx{Match}.}
\begin{ttbox}
val [major,minor] = goal FOL.thy
    "[| P&Q;  [| P; Q |] ==> R |] ==> R";
{\out Level 0}
{\out R}
{\out  1. R}
{\out val major = "P & Q  [P & Q]" : thm}
{\out val minor = "[| P; Q |] ==> R  [[| P; Q |] ==> R]" : thm}
\end{ttbox}
Look at the minor premise, recalling that meta-level assumptions are
shown in brackets.  Using {\tt minor}, we reduce $R$ to the subgoals
$P$ and~$Q$:
\begin{ttbox}
by (resolve_tac [minor] 1);
{\out Level 1}
{\out R}
{\out  1. P}
{\out  2. Q}
\end{ttbox}
Deviating from~\S\ref{deriving}, we apply $({\conj}E1)$ forwards from the
assumption $P\conj Q$ to obtain the theorem~$P\;[P\conj Q]$.
\begin{ttbox}
major RS conjunct1;
{\out val it = "P  [P & Q]" : thm}
\ttbreak
by (resolve_tac [major RS conjunct1] 1);
{\out Level 2}
{\out R}
{\out  1. Q}
\end{ttbox}
Similarly, we solve the subgoal involving~$Q$.
\begin{ttbox}
major RS conjunct2;
{\out val it = "Q  [P & Q]" : thm}
by (resolve_tac [major RS conjunct2] 1);
{\out Level 3}
{\out R}
{\out No subgoals!}
\end{ttbox}
Calling \ttindex{topthm} returns the current proof state as a theorem.
Note that it contains assumptions.  Calling \ttindex{result} discharges the
assumptions --- both occurrences of $P\conj Q$ are discharged as one ---
and makes the variables schematic.
\begin{ttbox}
topthm();
{\out val it = "R  [P & Q, P & Q, [| P; Q |] ==> R]" : thm}
val conjE = result();
{\out val conjE = "[| ?P & ?Q; [| ?P; ?Q |] ==> ?R |] ==> ?R" : thm}
\end{ttbox}


\subsection{Definitions and derived rules} \label{definitions}
\index{rules!derived}\index{definitions!and derived rules|(}

Definitions are expressed as meta-level equalities.  Let us define negation
and the if-and-only-if connective:
\begin{eqnarray*}
  \neg \Var{P}          & \equiv & \Var{P}\imp\bot \\
  \Var{P}\bimp \Var{Q}  & \equiv & 
                (\Var{P}\imp \Var{Q}) \conj (\Var{Q}\imp \Var{P})
\end{eqnarray*}
\index{meta-rewriting}%
Isabelle permits {\bf meta-level rewriting} using definitions such as
these.  {\bf Unfolding} replaces every instance
of $\neg \Var{P}$ by the corresponding instance of ${\Var{P}\imp\bot}$.  For
example, $\forall x.\neg (P(x)\conj \neg R(x,0))$ unfolds to
\[ \forall x.(P(x)\conj R(x,0)\imp\bot)\imp\bot.  \]
{\bf Folding} a definition replaces occurrences of the right-hand side by
the left.  The occurrences need not be free in the entire formula.

When you define new concepts, you should derive rules asserting their
abstract properties, and then forget their definitions.  This supports
modularity: if you later change the definitions without affecting their
abstract properties, then most of your proofs will carry through without
change.  Indiscriminate unfolding makes a subgoal grow exponentially,
becoming unreadable.

Taking this point of view, Isabelle does not unfold definitions
automatically during proofs.  Rewriting must be explicit and selective.
Isabelle provides tactics and meta-rules for rewriting, and a version of
the {\tt goal} command that unfolds the conclusion and premises of the rule
being derived.

For example, the intuitionistic definition of negation given above may seem
peculiar.  Using Isabelle, we shall derive pleasanter negation rules:
\[  \infer[({\neg}I)]{\neg P}{\infer*{\bot}{[P]}}   \qquad
    \infer[({\neg}E)]{Q}{\neg P & P}  \]
This requires proving the following meta-formulae:
$$ (P\Imp\bot)    \Imp \neg P   \eqno(\neg I)$$
$$ \List{\neg P; P} \Imp Q.       \eqno(\neg E)$$


\subsection{Deriving the $\neg$ introduction rule}
To derive $(\neg I)$, we may call {\tt goal} with the appropriate
formula.  Again, {\tt goal} returns a list consisting of the rule's
premises.  We bind this one-element list to the \ML\ identifier {\tt
  prems}.
\begin{ttbox}
val prems = goal FOL.thy "(P ==> False) ==> ~P";
{\out Level 0}
{\out ~P}
{\out  1. ~P}
{\out val prems = ["P ==> False  [P ==> False]"] : thm list}
\end{ttbox}
Calling \ttindex{rewrite_goals_tac} with \tdx{not_def}, which is the
definition of negation, unfolds that definition in the subgoals.  It leaves
the main goal alone.
\begin{ttbox}
not_def;
{\out val it = "~?P == ?P --> False" : thm}
by (rewrite_goals_tac [not_def]);
{\out Level 1}
{\out ~P}
{\out  1. P --> False}
\end{ttbox}
Using \tdx{impI} and the premise, we reduce subgoal~1 to a triviality:
\begin{ttbox}
by (resolve_tac [impI] 1);
{\out Level 2}
{\out ~P}
{\out  1. P ==> False}
\ttbreak
by (resolve_tac prems 1);
{\out Level 3}
{\out ~P}
{\out  1. P ==> P}
\end{ttbox}
The rest of the proof is routine.  Note the form of the final result.
\begin{ttbox}
by (assume_tac 1);
{\out Level 4}
{\out ~P}
{\out No subgoals!}
\ttbreak
val notI = result();
{\out val notI = "(?P ==> False) ==> ~?P" : thm}
\end{ttbox}
\indexbold{*notI theorem}

There is a simpler way of conducting this proof.  The \ttindex{goalw}
command starts a backward proof, as does {\tt goal}, but it also
unfolds definitions.  Thus there is no need to call
\ttindex{rewrite_goals_tac}:
\begin{ttbox}
val prems = goalw FOL.thy [not_def]
    "(P ==> False) ==> ~P";
{\out Level 0}
{\out ~P}
{\out  1. P --> False}
{\out val prems = ["P ==> False  [P ==> False]"] : thm list}
\end{ttbox}


\subsection{Deriving the $\neg$ elimination rule}
Let us derive the rule $(\neg E)$.  The proof follows that of~{\tt conjE}
above, with an additional step to unfold negation in the major premise.
Although the {\tt goalw} command is best for this, let us
try~{\tt goal} to see another way of unfolding definitions.  After
binding the premises to \ML\ identifiers, we apply \tdx{FalseE}:
\begin{ttbox}
val [major,minor] = goal FOL.thy "[| ~P;  P |] ==> R";
{\out Level 0}
{\out R}
{\out  1. R}
{\out val major = "~ P  [~ P]" : thm}
{\out val minor = "P  [P]" : thm}
\ttbreak
by (resolve_tac [FalseE] 1);
{\out Level 1}
{\out R}
{\out  1. False}
\end{ttbox}
Everything follows from falsity.  And we can prove falsity using the
premises and Modus Ponens:
\begin{ttbox}
by (resolve_tac [mp] 1);
{\out Level 2}
{\out R}
{\out  1. ?P1 --> False}
{\out  2. ?P1}
\end{ttbox}
For subgoal~1, we transform the major premise from~$\neg P$
to~${P\imp\bot}$.  The function \ttindex{rewrite_rule}, given a list of
definitions, unfolds them in a theorem.  Rewriting does not
affect the theorem's hypothesis, which remains~$\neg P$:
\begin{ttbox}
rewrite_rule [not_def] major;
{\out val it = "P --> False  [~P]" : thm}
by (resolve_tac [it] 1);
{\out Level 3}
{\out R}
{\out  1. P}
\end{ttbox}
The subgoal {\tt?P1} has been instantiated to~{\tt P}, which we can prove
using the minor premise:
\begin{ttbox}
by (resolve_tac [minor] 1);
{\out Level 4}
{\out R}
{\out No subgoals!}
val notE = result();
{\out val notE = "[| ~?P; ?P |] ==> ?R" : thm}
\end{ttbox}
\indexbold{*notE theorem}

\medskip
Again, there is a simpler way of conducting this proof.  Recall that
the \ttindex{goalw} command unfolds definitions the conclusion; it also
unfolds definitions in the premises:
\begin{ttbox}
val [major,minor] = goalw FOL.thy [not_def]
    "[| ~P;  P |] ==> R";
{\out val major = "P --> False  [~ P]" : thm}
{\out val minor = "P  [P]" : thm}
\end{ttbox}
Observe the difference in {\tt major}; the premises are unfolded without
calling~\ttindex{rewrite_rule}.  Incidentally, the four calls to
\ttindex{resolve_tac} above can be collapsed to one, with the help
of~\ttindex{RS}; this is a typical example of forward reasoning from a
complex premise.
\begin{ttbox}
minor RS (major RS mp RS FalseE);
{\out val it = "?P  [P, ~P]" : thm}
by (resolve_tac [it] 1);
{\out Level 1}
{\out R}
{\out No subgoals!}
\end{ttbox}
\index{definitions!and derived rules|)}

\goodbreak\medskip\index{*"!"! symbol!in main goal}
Finally, here is a trick that is sometimes useful.  If the goal
has an outermost meta-quantifier, then \ttindex{goal} and \ttindex{goalw}
do not return the rule's premises in the list of theorems;  instead, the
premises become assumptions in subgoal~1.  
%%%It does not matter which variables are quantified over.
\begin{ttbox}
goalw FOL.thy [not_def] "!!P R. [| ~P;  P |] ==> R";
{\out Level 0}
{\out !!P R. [| ~ P; P |] ==> R}
{\out  1. !!P R. [| P --> False; P |] ==> R}
val it = [] : thm list
\end{ttbox}
The proof continues as before.  But instead of referring to \ML\
identifiers, we refer to assumptions using {\tt eresolve_tac} or
{\tt assume_tac}: 
\begin{ttbox}
by (resolve_tac [FalseE] 1);
{\out Level 1}
{\out !!P R. [| ~ P; P |] ==> R}
{\out  1. !!P R. [| P --> False; P |] ==> False}
\ttbreak
by (eresolve_tac [mp] 1);
{\out Level 2}
{\out !!P R. [| ~ P; P |] ==> R}
{\out  1. !!P R. P ==> P}
\ttbreak
by (assume_tac 1);
{\out Level 3}
{\out !!P R. [| ~ P; P |] ==> R}
{\out No subgoals!}
\end{ttbox}
Calling \ttindex{result} strips the meta-quantifiers, so the resulting
theorem is the same as before.
\begin{ttbox}
val notE = result();
{\out val notE = "[| ~?P; ?P |] ==> ?R" : thm}
\end{ttbox}
Do not use the {\tt!!}\ trick if the premises contain meta-level
connectives, because \ttindex{eresolve_tac} and \ttindex{assume_tac} would
not be able to handle the resulting assumptions.  The trick is not suitable
for deriving the introduction rule~$(\neg I)$.


\section{Defining theories}\label{sec:defining-theories}
\index{theories!defining|(}

Isabelle makes no distinction between simple extensions of a logic --- like
defining a type~$bool$ with constants~$true$ and~$false$ --- and defining
an entire logic.  A theory definition has the form
\begin{ttbox}
\(T\) = \(S@1\) + \(\cdots\) + \(S@n\) +
classes      {\it class declarations}
default      {\it sort}
types        {\it type declarations and synonyms}
arities      {\it arity declarations}
consts       {\it constant declarations}
rules        {\it rule declarations}
translations {\it translation declarations}
end
ML           {\it ML code}
\end{ttbox}
This declares the theory $T$ to extend the existing theories
$S@1$,~\ldots,~$S@n$.  It may declare new classes, types, arities
(overloadings of existing types), constants and rules; it can specify the
default sort for type variables.  A constant declaration can specify an
associated concrete syntax.  The translations section specifies rewrite
rules on abstract syntax trees, for defining notations and abbreviations.
\index{*ML section}
The {\tt ML} section contains code to perform arbitrary syntactic
transformations.  The main declaration forms are discussed below.
The full syntax can be found in \iflabelundefined{app:TheorySyntax}{the
  appendix of the {\it Reference Manual}}{App.\ts\ref{app:TheorySyntax}}.

All the declaration parts can be omitted.  In the simplest case, $T$ is
just the union of $S@1$,~\ldots,~$S@n$.  New theories always extend one
or more other theories, inheriting their types, constants, syntax, etc.
The theory \thydx{Pure} contains nothing but Isabelle's meta-logic.

Each theory definition must reside in a separate file, whose name is the
theory's with {\tt.thy} appended.  For example, theory {\tt ListFn} resides
on a file named {\tt ListFn.thy}.  Isabelle uses this convention to locate the
file containing a given theory; \ttindexbold{use_thy} automatically loads a
theory's parents before loading the theory itself.

Calling \ttindexbold{use_thy}~{\tt"{\it T\/}"} reads a theory from the
file {\it T}{\tt.thy}, writes the corresponding {\ML} code to the file
{\tt.{\it T}.thy.ML}, reads the latter file, and deletes it if no errors
occurred.  This declares the {\ML} structure~$T$, which contains a component
{\tt thy} denoting the new theory, a component for each rule, and everything
declared in {\it ML code}.

Errors may arise during the translation to {\ML} (say, a misspelled keyword)
or during creation of the new theory (say, a type error in a rule).  But if
all goes well, {\tt use_thy} will finally read the file {\it T}{\tt.ML}, if
it exists.  This file typically begins with the {\ML} declaration {\tt
open}~$T$ and contains proofs that refer to the components of~$T$.

When a theory file is modified, many theories may have to be reloaded.
Isabelle records the modification times and dependencies of theory files.
See 
\iflabelundefined{sec:reloading-theories}{the {\em Reference Manual\/}}%
                 {\S\ref{sec:reloading-theories}}
for more details.


\subsection{Declaring constants and rules}
\indexbold{constants!declaring}\index{rules!declaring}

Most theories simply declare constants and rules.  The {\bf constant
declaration part} has the form
\begin{ttbox}
consts  \(c@1\) :: "\(\tau@1\)"
        \vdots
        \(c@n\) :: "\(\tau@n\)"
\end{ttbox}
where $c@1$, \ldots, $c@n$ are constants and $\tau@1$, \ldots, $\tau@n$ are
types.  Each type {\em must\/} be enclosed in quotation marks.  Each
constant must be enclosed in quotation marks unless it is a valid
identifier.  To declare $c@1$, \ldots, $c@n$ as constants of type $\tau$,
the $n$ declarations may be abbreviated to a single line:
\begin{ttbox}
        \(c@1\), \ldots, \(c@n\) :: "\(\tau\)"
\end{ttbox}
The {\bf rule declaration part} has the form
\begin{ttbox}
rules   \(id@1\) "\(rule@1\)"
        \vdots
        \(id@n\) "\(rule@n\)"
\end{ttbox}
where $id@1$, \ldots, $id@n$ are \ML{} identifiers and $rule@1$, \ldots,
$rule@n$ are expressions of type~$prop$.  Each rule {\em must\/} be
enclosed in quotation marks.

\indexbold{definitions}
{\bf Definitions} are rules of the form $t\equiv u$.  Normally definitions
should be conservative, serving only as abbreviations.  As of this writing,
Isabelle does not provide a separate declaration part for definitions; it
is your responsibility to ensure that your definitions are conservative.
However, Isabelle's rewriting primitives will reject $t\equiv u$ unless all
variables free in~$u$ are also free in~$t$.

\index{examples!of theories}
This theory extends first-order logic with two constants {\em nand} and
{\em xor}, and declares rules to define them:
\begin{ttbox}
Gate = FOL +
consts  nand,xor :: "[o,o] => o"
rules   nand_def "nand(P,Q) == ~(P & Q)"
        xor_def  "xor(P,Q)  == P & ~Q | ~P & Q"
end
\end{ttbox}


\subsection{Declaring type constructors}
\indexbold{types!declaring}\indexbold{arities!declaring}
%
Types are composed of type variables and {\bf type constructors}.  Each
type constructor takes a fixed number of arguments.  They are declared
with an \ML-like syntax.  If $list$ takes one type argument, $tree$ takes
two arguments and $nat$ takes no arguments, then these type constructors
can be declared by
\begin{ttbox}
types 'a list
      ('a,'b) tree
      nat
\end{ttbox}

The {\bf type declaration part} has the general form
\begin{ttbox}
types   \(tids@1\) \(id@1\)
        \vdots
        \(tids@1\) \(id@n\)
\end{ttbox}
where $id@1$, \ldots, $id@n$ are identifiers and $tids@1$, \ldots, $tids@n$
are type argument lists as shown in the example above.  It declares each
$id@i$ as a type constructor with the specified number of argument places.

The {\bf arity declaration part} has the form
\begin{ttbox}
arities \(tycon@1\) :: \(arity@1\)
        \vdots
        \(tycon@n\) :: \(arity@n\)
\end{ttbox}
where $tycon@1$, \ldots, $tycon@n$ are identifiers and $arity@1$, \ldots,
$arity@n$ are arities.  Arity declarations add arities to existing
types; they do not declare the types themselves.
In the simplest case, for an 0-place type constructor, an arity is simply
the type's class.  Let us declare a type~$bool$ of class $term$, with
constants $tt$ and~$ff$.  (In first-order logic, booleans are
distinct from formulae, which have type $o::logic$.)
\index{examples!of theories}
\begin{ttbox}
Bool = FOL +
types   bool
arities bool    :: term
consts  tt,ff   :: "bool"
end
\end{ttbox}
A $k$-place type constructor may have arities of the form
$(s@1,\ldots,s@k)c$, where $s@1,\ldots,s@n$ are sorts and $c$ is a class.
Each sort specifies a type argument; it has the form $\{c@1,\ldots,c@m\}$,
where $c@1$, \dots,~$c@m$ are classes.  Mostly we deal with singleton
sorts, and may abbreviate them by dropping the braces.  The arity
$(term)term$ is short for $(\{term\})term$.  Recall the discussion in
\S\ref{polymorphic}.

A type constructor may be overloaded (subject to certain conditions) by
appearing in several arity declarations.  For instance, the function type
constructor~$fun$ has the arity $(logic,logic)logic$; in higher-order
logic, it is declared also to have arity $(term,term)term$.

Theory {\tt List} declares the 1-place type constructor $list$, gives
it arity $(term)term$, and declares constants $Nil$ and $Cons$ with
polymorphic types:%
\footnote{In the {\tt consts} part, type variable {\tt'a} has the default
  sort, which is {\tt term}.  See the {\em Reference Manual\/}
\iflabelundefined{sec:ref-defining-theories}{}%
{(\S\ref{sec:ref-defining-theories})} for more information.}
\index{examples!of theories}
\begin{ttbox}
List = FOL +
types   'a list
arities list    :: (term)term
consts  Nil     :: "'a list"
        Cons    :: "['a, 'a list] => 'a list" 
end
\end{ttbox}
Multiple arity declarations may be abbreviated to a single line:
\begin{ttbox}
arities \(tycon@1\), \ldots, \(tycon@n\) :: \(arity\)
\end{ttbox}

\begin{warn}
Arity declarations resemble constant declarations, but there are {\it no\/}
quotation marks!  Types and rules must be quoted because the theory
translator passes them verbatim to the {\ML} output file.
\end{warn}

\subsection{Type synonyms}\indexbold{type synonyms}
Isabelle supports {\bf type synonyms} ({\bf abbreviations}) which are similar
to those found in \ML.  Such synonyms are defined in the type declaration part
and are fairly self explanatory:
\begin{ttbox}
types gate       = "[o,o] => o"
      'a pred    = "'a => o"
      ('a,'b)nuf = "'b => 'a"
\end{ttbox}
Type declarations and synonyms can be mixed arbitrarily:
\begin{ttbox}
types nat
      'a stream = "nat => 'a"
      signal    = "nat stream"
      'a list
\end{ttbox}
A synonym is merely an abbreviation for some existing type expression.  Hence
synonyms may not be recursive!  Internally all synonyms are fully expanded.  As
a consequence Isabelle output never contains synonyms.  Their main purpose is
to improve the readability of theories.  Synonyms can be used just like any
other type:
\begin{ttbox}
consts and,or :: "gate"
       negate :: "signal => signal"
\end{ttbox}

\subsection{Infix and mixfix operators}
\index{infixes}\index{examples!of theories}

Infix or mixfix syntax may be attached to constants.  Consider the
following theory:
\begin{ttbox}
Gate2 = FOL +
consts  "~&"     :: "[o,o] => o"         (infixl 35)
        "#"      :: "[o,o] => o"         (infixl 30)
rules   nand_def "P ~& Q == ~(P & Q)"    
        xor_def  "P # Q  == P & ~Q | ~P & Q"
end
\end{ttbox}
The constant declaration part declares two left-associating infix operators
with their priorities, or precedences; they are $\nand$ of priority~35 and
$\xor$ of priority~30.  Hence $P \xor Q \xor R$ is parsed as $(P\xor Q)
\xor R$ and $P \xor Q \nand R$ as $P \xor (Q \nand R)$.  Note the quotation
marks in \verb|"~&"| and \verb|"#"|.

The constants \hbox{\verb|op ~&|} and \hbox{\verb|op #|} are declared
automatically, just as in \ML.  Hence you may write propositions like
\verb|op #(True) == op ~&(True)|, which asserts that the functions $\lambda
Q.True \xor Q$ and $\lambda Q.True \nand Q$ are identical.

\bigskip\index{mixfix declarations}
{\bf Mixfix} operators may have arbitrary context-free syntaxes.  Let us
add a line to the constant declaration part:
\begin{ttbox}
        If :: "[o,o,o] => o"       ("if _ then _ else _")
\end{ttbox}
This declares a constant $If$ of type $[o,o,o] \To o$ with concrete syntax {\tt
  if~$P$ then~$Q$ else~$R$} as well as {\tt If($P$,$Q$,$R$)}.  Underscores
denote argument positions.  

The declaration above does not allow the {\tt if}-{\tt then}-{\tt else}
construct to be split across several lines, even if it is too long to fit
on one line.  Pretty-printing information can be added to specify the
layout of mixfix operators.  For details, see
\iflabelundefined{Defining-Logics}%
    {the {\it Reference Manual}, chapter `Defining Logics'}%
    {Chap.\ts\ref{Defining-Logics}}.

Mixfix declarations can be annotated with priorities, just like
infixes.  The example above is just a shorthand for
\begin{ttbox}
        If :: "[o,o,o] => o"       ("if _ then _ else _" [0,0,0] 1000)
\end{ttbox}
The numeric components determine priorities.  The list of integers
defines, for each argument position, the minimal priority an expression
at that position must have.  The final integer is the priority of the
construct itself.  In the example above, any argument expression is
acceptable because priorities are non-negative, and conditionals may
appear everywhere because 1000 is the highest priority.  On the other
hand, the declaration
\begin{ttbox}
        If :: "[o,o,o] => o"       ("if _ then _ else _" [100,0,0] 99)
\end{ttbox}
defines concrete syntax for a conditional whose first argument cannot have
the form {\tt if~$P$ then~$Q$ else~$R$} because it must have a priority
of at least~100.  We may of course write
\begin{quote}\tt
if (if $P$ then $Q$ else $R$) then $S$ else $T$
\end{quote}
because expressions in parentheses have maximal priority.  

Binary type constructors, like products and sums, may also be declared as
infixes.  The type declaration below introduces a type constructor~$*$ with
infix notation $\alpha*\beta$, together with the mixfix notation
${<}\_,\_{>}$ for pairs.  
\index{examples!of theories}\index{mixfix declarations}
\begin{ttbox}
Prod = FOL +
types   ('a,'b) "*"                           (infixl 20)
arities "*"     :: (term,term)term
consts  fst     :: "'a * 'b => 'a"
        snd     :: "'a * 'b => 'b"
        Pair    :: "['a,'b] => 'a * 'b"       ("(1<_,/_>)")
rules   fst     "fst(<a,b>) = a"
        snd     "snd(<a,b>) = b"
end
\end{ttbox}

\begin{warn}
The name of the type constructor is~{\tt *} and not {\tt op~*}, as it would
be in the case of an infix constant.  Only infix type constructors can have
symbolic names like~{\tt *}.  There is no general mixfix syntax for types.
\end{warn}


\subsection{Overloading}
\index{overloading}\index{examples!of theories}
The {\bf class declaration part} has the form
\begin{ttbox}
classes \(id@1\) < \(c@1\)
        \vdots
        \(id@n\) < \(c@n\)
\end{ttbox}
where $id@1$, \ldots, $id@n$ are identifiers and $c@1$, \ldots, $c@n$ are
existing classes.  It declares each $id@i$ as a new class, a subclass
of~$c@i$.  In the general case, an identifier may be declared to be a
subclass of $k$ existing classes:
\begin{ttbox}
        \(id\) < \(c@1\), \ldots, \(c@k\)
\end{ttbox}
Type classes allow constants to be overloaded.  As suggested in
\S\ref{polymorphic}, let us define the class $arith$ of arithmetic
types with the constants ${+} :: [\alpha,\alpha]\To \alpha$ and $0,1 {::}
\alpha$, for $\alpha{::}arith$.  We introduce $arith$ as a subclass of
$term$ and add the three polymorphic constants of this class.
\index{examples!of theories}\index{constants!overloaded}
\begin{ttbox}
Arith = FOL +
classes arith < term
consts  "0"     :: "'a::arith"                  ("0")
        "1"     :: "'a::arith"                  ("1")
        "+"     :: "['a::arith,'a] => 'a"       (infixl 60)
end
\end{ttbox}
No rules are declared for these constants: we merely introduce their
names without specifying properties.  On the other hand, classes
with rules make it possible to prove {\bf generic} theorems.  Such
theorems hold for all instances, all types in that class.

We can now obtain distinct versions of the constants of $arith$ by
declaring certain types to be of class $arith$.  For example, let us
declare the 0-place type constructors $bool$ and $nat$:
\index{examples!of theories}
\begin{ttbox}
BoolNat = Arith +
types   bool  nat
arities bool, nat   :: arith
consts  Suc         :: "nat=>nat"
\ttbreak
rules   add0        "0 + n = n::nat"
        addS        "Suc(m)+n = Suc(m+n)"
        nat1        "1 = Suc(0)"
        or0l        "0 + x = x::bool"
        or0r        "x + 0 = x::bool"
        or1l        "1 + x = 1::bool"
        or1r        "x + 1 = 1::bool"
end
\end{ttbox}
Because $nat$ and $bool$ have class $arith$, we can use $0$, $1$ and $+$ at
either type.  The type constraints in the axioms are vital.  Without
constraints, the $x$ in $1+x = x$ would have type $\alpha{::}arith$
and the axiom would hold for any type of class $arith$.  This would
collapse $nat$ to a trivial type:
\[ Suc(1) = Suc(0+1) = Suc(0)+1 = 1+1 = 1! \]


\section{Theory example: the natural numbers}

We shall now work through a small example of formalized mathematics
demonstrating many of the theory extension features.


\subsection{Extending first-order logic with the natural numbers}
\index{examples!of theories}

Section\ts\ref{sec:logical-syntax} has formalized a first-order logic,
including a type~$nat$ and the constants $0::nat$ and $Suc::nat\To nat$.
Let us introduce the Peano axioms for mathematical induction and the
freeness of $0$ and~$Suc$:\index{axioms!Peano}
\[ \vcenter{\infer[(induct)]{P[n/x]}{P[0/x] & \infer*{P[Suc(x)/x]}{[P]}}}
 \qquad \parbox{4.5cm}{provided $x$ is not free in any assumption except~$P$}
\]
\[ \infer[(Suc\_inject)]{m=n}{Suc(m)=Suc(n)} \qquad
   \infer[(Suc\_neq\_0)]{R}{Suc(m)=0}
\]
Mathematical induction asserts that $P(n)$ is true, for any $n::nat$,
provided $P(0)$ holds and that $P(x)$ implies $P(Suc(x))$ for all~$x$.
Some authors express the induction step as $\forall x. P(x)\imp P(Suc(x))$.
To avoid making induction require the presence of other connectives, we
formalize mathematical induction as
$$ \List{P(0); \Forall x. P(x)\Imp P(Suc(x))} \Imp P(n). \eqno(induct) $$

\noindent
Similarly, to avoid expressing the other rules using~$\forall$, $\imp$
and~$\neg$, we take advantage of the meta-logic;\footnote
{On the other hand, the axioms $Suc(m)=Suc(n) \bimp m=n$
and $\neg(Suc(m)=0)$ are logically equivalent to those given, and work
better with Isabelle's simplifier.} 
$(Suc\_neq\_0)$ is
an elimination rule for $Suc(m)=0$:
$$ Suc(m)=Suc(n) \Imp m=n  \eqno(Suc\_inject) $$
$$ Suc(m)=0      \Imp R    \eqno(Suc\_neq\_0) $$

\noindent
We shall also define a primitive recursion operator, $rec$.  Traditionally,
primitive recursion takes a natural number~$a$ and a 2-place function~$f$,
and obeys the equations
\begin{eqnarray*}
  rec(0,a,f)            & = & a \\
  rec(Suc(m),a,f)       & = & f(m, rec(m,a,f))
\end{eqnarray*}
Addition, defined by $m+n \equiv rec(m,n,\lambda x\,y.Suc(y))$,
should satisfy
\begin{eqnarray*}
  0+n      & = & n \\
  Suc(m)+n & = & Suc(m+n)
\end{eqnarray*}
Primitive recursion appears to pose difficulties: first-order logic has no
function-valued expressions.  We again take advantage of the meta-logic,
which does have functions.  We also generalise primitive recursion to be
polymorphic over any type of class~$term$, and declare the addition
function:
\begin{eqnarray*}
  rec   & :: & [nat, \alpha{::}term, [nat,\alpha]\To\alpha] \To\alpha \\
  +     & :: & [nat,nat]\To nat 
\end{eqnarray*}


\subsection{Declaring the theory to Isabelle}
\index{examples!of theories}
Let us create the theory \thydx{Nat} starting from theory~\verb$FOL$,
which contains only classical logic with no natural numbers.  We declare
the 0-place type constructor $nat$ and the associated constants.  Note that
the constant~0 requires a mixfix annotation because~0 is not a legal
identifier, and could not otherwise be written in terms:
\begin{ttbox}\index{mixfix declarations}
Nat = FOL +
types   nat
arities nat         :: term
consts  "0"         :: "nat"                              ("0")
        Suc         :: "nat=>nat"
        rec         :: "[nat, 'a, [nat,'a]=>'a] => 'a"
        "+"         :: "[nat, nat] => nat"                (infixl 60)
rules   Suc_inject  "Suc(m)=Suc(n) ==> m=n"
        Suc_neq_0   "Suc(m)=0      ==> R"
        induct      "[| P(0);  !!x. P(x) ==> P(Suc(x)) |]  ==> P(n)"
        rec_0       "rec(0,a,f) = a"
        rec_Suc     "rec(Suc(m), a, f) = f(m, rec(m,a,f))"
        add_def     "m+n == rec(m, n, \%x y. Suc(y))"
end
\end{ttbox}
In axiom {\tt add_def}, recall that \verb|%| stands for~$\lambda$.
Loading this theory file creates the \ML\ structure {\tt Nat}, which
contains the theory and axioms.  Opening structure {\tt Nat} lets us write
{\tt induct} instead of {\tt Nat.induct}, and so forth.
\begin{ttbox}
open Nat;
\end{ttbox}

\subsection{Proving some recursion equations}
File {\tt FOL/ex/Nat.ML} contains proofs involving this theory of the
natural numbers.  As a trivial example, let us derive recursion equations
for \verb$+$.  Here is the zero case:
\begin{ttbox}
goalw Nat.thy [add_def] "0+n = n";
{\out Level 0}
{\out 0 + n = n}
{\out  1. rec(0,n,\%x y. Suc(y)) = n}
\ttbreak
by (resolve_tac [rec_0] 1);
{\out Level 1}
{\out 0 + n = n}
{\out No subgoals!}
val add_0 = result();
\end{ttbox}
And here is the successor case:
\begin{ttbox}
goalw Nat.thy [add_def] "Suc(m)+n = Suc(m+n)";
{\out Level 0}
{\out Suc(m) + n = Suc(m + n)}
{\out  1. rec(Suc(m),n,\%x y. Suc(y)) = Suc(rec(m,n,\%x y. Suc(y)))}
\ttbreak
by (resolve_tac [rec_Suc] 1);
{\out Level 1}
{\out Suc(m) + n = Suc(m + n)}
{\out No subgoals!}
val add_Suc = result();
\end{ttbox}
The induction rule raises some complications, which are discussed next.
\index{theories!defining|)}


\section{Refinement with explicit instantiation}
\index{resolution!with instantiation}
\index{instantiation|(}

In order to employ mathematical induction, we need to refine a subgoal by
the rule~$(induct)$.  The conclusion of this rule is $\Var{P}(\Var{n})$,
which is highly ambiguous in higher-order unification.  It matches every
way that a formula can be regarded as depending on a subterm of type~$nat$.
To get round this problem, we could make the induction rule conclude
$\forall n.\Var{P}(n)$ --- but putting a subgoal into this form requires
refinement by~$(\forall E)$, which is equally hard!

The tactic {\tt res_inst_tac}, like {\tt resolve_tac}, refines a subgoal by
a rule.  But it also accepts explicit instantiations for the rule's
schematic variables.  
\begin{description}
\item[\ttindex{res_inst_tac} {\it insts} {\it thm} {\it i}]
instantiates the rule {\it thm} with the instantiations {\it insts}, and
then performs resolution on subgoal~$i$.

\item[\ttindex{eres_inst_tac}] 
and \ttindex{dres_inst_tac} are similar, but perform elim-resolution
and destruct-resolution, respectively.
\end{description}
The list {\it insts} consists of pairs $[(v@1,e@1), \ldots, (v@n,e@n)]$,
where $v@1$, \ldots, $v@n$ are names of schematic variables in the rule ---
with no leading question marks! --- and $e@1$, \ldots, $e@n$ are
expressions giving their instantiations.  The expressions are type-checked
in the context of a particular subgoal: free variables receive the same
types as they have in the subgoal, and parameters may appear.  Type
variable instantiations may appear in~{\it insts}, but they are seldom
required: {\tt res_inst_tac} instantiates type variables automatically
whenever the type of~$e@i$ is an instance of the type of~$\Var{v@i}$.

\subsection{A simple proof by induction}
\index{examples!of induction}
Let us prove that no natural number~$k$ equals its own successor.  To
use~$(induct)$, we instantiate~$\Var{n}$ to~$k$; Isabelle finds a good
instantiation for~$\Var{P}$.
\begin{ttbox}
goal Nat.thy "~ (Suc(k) = k)";
{\out Level 0}
{\out ~Suc(k) = k}
{\out  1. ~Suc(k) = k}
\ttbreak
by (res_inst_tac [("n","k")] induct 1);
{\out Level 1}
{\out ~Suc(k) = k}
{\out  1. ~Suc(0) = 0}
{\out  2. !!x. ~Suc(x) = x ==> ~Suc(Suc(x)) = Suc(x)}
\end{ttbox}
We should check that Isabelle has correctly applied induction.  Subgoal~1
is the base case, with $k$ replaced by~0.  Subgoal~2 is the inductive step,
with $k$ replaced by~$Suc(x)$ and with an induction hypothesis for~$x$.
The rest of the proof demonstrates~\tdx{notI}, \tdx{notE} and the
other rules of theory {\tt Nat}.  The base case holds by~\ttindex{Suc_neq_0}:
\begin{ttbox}
by (resolve_tac [notI] 1);
{\out Level 2}
{\out ~Suc(k) = k}
{\out  1. Suc(0) = 0 ==> False}
{\out  2. !!x. ~Suc(x) = x ==> ~Suc(Suc(x)) = Suc(x)}
\ttbreak
by (eresolve_tac [Suc_neq_0] 1);
{\out Level 3}
{\out ~Suc(k) = k}
{\out  1. !!x. ~Suc(x) = x ==> ~Suc(Suc(x)) = Suc(x)}
\end{ttbox}
The inductive step holds by the contrapositive of~\ttindex{Suc_inject}.
Negation rules transform the subgoal into that of proving $Suc(x)=x$ from
$Suc(Suc(x)) = Suc(x)$:
\begin{ttbox}
by (resolve_tac [notI] 1);
{\out Level 4}
{\out ~Suc(k) = k}
{\out  1. !!x. [| ~Suc(x) = x; Suc(Suc(x)) = Suc(x) |] ==> False}
\ttbreak
by (eresolve_tac [notE] 1);
{\out Level 5}
{\out ~Suc(k) = k}
{\out  1. !!x. Suc(Suc(x)) = Suc(x) ==> Suc(x) = x}
\ttbreak
by (eresolve_tac [Suc_inject] 1);
{\out Level 6}
{\out ~Suc(k) = k}
{\out No subgoals!}
\end{ttbox}


\subsection{An example of ambiguity in {\tt resolve_tac}}
\index{examples!of induction}\index{unification!higher-order}
If you try the example above, you may observe that {\tt res_inst_tac} is
not actually needed.  Almost by chance, \ttindex{resolve_tac} finds the right
instantiation for~$(induct)$ to yield the desired next state.  With more
complex formulae, our luck fails.  
\begin{ttbox}
goal Nat.thy "(k+m)+n = k+(m+n)";
{\out Level 0}
{\out k + m + n = k + (m + n)}
{\out  1. k + m + n = k + (m + n)}
\ttbreak
by (resolve_tac [induct] 1);
{\out Level 1}
{\out k + m + n = k + (m + n)}
{\out  1. k + m + n = 0}
{\out  2. !!x. k + m + n = x ==> k + m + n = Suc(x)}
\end{ttbox}
This proof requires induction on~$k$.  The occurrence of~0 in subgoal~1
indicates that induction has been applied to the term~$k+(m+n)$; this
application is sound but will not lead to a proof here.  Fortunately,
Isabelle can (lazily!) generate all the valid applications of induction.
The \ttindex{back} command causes backtracking to an alternative outcome of
the tactic.
\begin{ttbox}
back();
{\out Level 1}
{\out k + m + n = k + (m + n)}
{\out  1. k + m + n = k + 0}
{\out  2. !!x. k + m + n = k + x ==> k + m + n = k + Suc(x)}
\end{ttbox}
Now induction has been applied to~$m+n$.  This is equally useless.  Let us
call \ttindex{back} again.
\begin{ttbox}
back();
{\out Level 1}
{\out k + m + n = k + (m + n)}
{\out  1. k + m + 0 = k + (m + 0)}
{\out  2. !!x. k + m + x = k + (m + x) ==>}
{\out          k + m + Suc(x) = k + (m + Suc(x))}
\end{ttbox}
Now induction has been applied to~$n$.  What is the next alternative?
\begin{ttbox}
back();
{\out Level 1}
{\out k + m + n = k + (m + n)}
{\out  1. k + m + n = k + (m + 0)}
{\out  2. !!x. k + m + n = k + (m + x) ==> k + m + n = k + (m + Suc(x))}
\end{ttbox}
Inspecting subgoal~1 reveals that induction has been applied to just the
second occurrence of~$n$.  This perfectly legitimate induction is useless
here.  

The main goal admits fourteen different applications of induction.  The
number is exponential in the size of the formula.

\subsection{Proving that addition is associative}
Let us invoke the induction rule properly, using~{\tt
  res_inst_tac}.  At the same time, we shall have a glimpse at Isabelle's
simplification tactics, which are described in 
\iflabelundefined{simp-chap}%
    {the {\em Reference Manual}}{Chap.\ts\ref{simp-chap}}.

\index{simplification}\index{examples!of simplification} 

Isabelle's simplification tactics repeatedly apply equations to a subgoal,
perhaps proving it.  For efficiency, the rewrite rules must be
packaged into a {\bf simplification set},\index{simplification sets} 
or {\bf simpset}.  We take the standard simpset for first-order logic and
insert the equations proved in the previous section, namely
$0+n=n$ and ${\tt Suc}(m)+n={\tt Suc}(m+n)$:
\begin{ttbox}
val add_ss = FOL_ss addrews [add_0, add_Suc];
\end{ttbox}
We state the goal for associativity of addition, and
use \ttindex{res_inst_tac} to invoke induction on~$k$:
\begin{ttbox}
goal Nat.thy "(k+m)+n = k+(m+n)";
{\out Level 0}
{\out k + m + n = k + (m + n)}
{\out  1. k + m + n = k + (m + n)}
\ttbreak
by (res_inst_tac [("n","k")] induct 1);
{\out Level 1}
{\out k + m + n = k + (m + n)}
{\out  1. 0 + m + n = 0 + (m + n)}
{\out  2. !!x. x + m + n = x + (m + n) ==>}
{\out          Suc(x) + m + n = Suc(x) + (m + n)}
\end{ttbox}
The base case holds easily; both sides reduce to $m+n$.  The
tactic~\ttindex{simp_tac} rewrites with respect to the given simplification
set, applying the rewrite rules for addition:
\begin{ttbox}
by (simp_tac add_ss 1);
{\out Level 2}
{\out k + m + n = k + (m + n)}
{\out  1. !!x. x + m + n = x + (m + n) ==>}
{\out          Suc(x) + m + n = Suc(x) + (m + n)}
\end{ttbox}
The inductive step requires rewriting by the equations for addition
together the induction hypothesis, which is also an equation.  The
tactic~\ttindex{asm_simp_tac} rewrites using a simplification set and any
useful assumptions:
\begin{ttbox}
by (asm_simp_tac add_ss 1);
{\out Level 3}
{\out k + m + n = k + (m + n)}
{\out No subgoals!}
\end{ttbox}
\index{instantiation|)}


\section{A Prolog interpreter}
\index{Prolog interpreter|bold}
To demonstrate the power of tacticals, let us construct a Prolog
interpreter and execute programs involving lists.\footnote{To run these
examples, see the file {\tt FOL/ex/Prolog.ML}.} The Prolog program
consists of a theory.  We declare a type constructor for lists, with an
arity declaration to say that $(\tau)list$ is of class~$term$
provided~$\tau$ is:
\begin{eqnarray*}
  list  & :: & (term)term
\end{eqnarray*}
We declare four constants: the empty list~$Nil$; the infix list
constructor~{:}; the list concatenation predicate~$app$; the list reverse
predicate~$rev$.  (In Prolog, functions on lists are expressed as
predicates.)
\begin{eqnarray*}
    Nil         & :: & \alpha list \\
    {:}         & :: & [\alpha,\alpha list] \To \alpha list \\
    app & :: & [\alpha list,\alpha list,\alpha list] \To o \\
    rev & :: & [\alpha list,\alpha list] \To o 
\end{eqnarray*}
The predicate $app$ should satisfy the Prolog-style rules
\[ {app(Nil,ys,ys)} \qquad
   {app(xs,ys,zs) \over app(x:xs, ys, x:zs)} \]
We define the naive version of $rev$, which calls~$app$:
\[ {rev(Nil,Nil)} \qquad
   {rev(xs,ys)\quad  app(ys, x:Nil, zs) \over
    rev(x:xs, zs)} 
\]

\index{examples!of theories}
Theory \thydx{Prolog} extends first-order logic in order to make use
of the class~$term$ and the type~$o$.  The interpreter does not use the
rules of~{\tt FOL}.
\begin{ttbox}
Prolog = FOL +
types   'a list
arities list    :: (term)term
consts  Nil     :: "'a list"
        ":"     :: "['a, 'a list]=> 'a list"            (infixr 60)
        app     :: "['a list, 'a list, 'a list] => o"
        rev     :: "['a list, 'a list] => o"
rules   appNil  "app(Nil,ys,ys)"
        appCons "app(xs,ys,zs) ==> app(x:xs, ys, x:zs)"
        revNil  "rev(Nil,Nil)"
        revCons "[| rev(xs,ys); app(ys,x:Nil,zs) |] ==> rev(x:xs,zs)"
end
\end{ttbox}
\subsection{Simple executions}
Repeated application of the rules solves Prolog goals.  Let us
append the lists $[a,b,c]$ and~$[d,e]$.  As the rules are applied, the
answer builds up in~{\tt ?x}.
\begin{ttbox}
goal Prolog.thy "app(a:b:c:Nil, d:e:Nil, ?x)";
{\out Level 0}
{\out app(a : b : c : Nil, d : e : Nil, ?x)}
{\out  1. app(a : b : c : Nil, d : e : Nil, ?x)}
\ttbreak
by (resolve_tac [appNil,appCons] 1);
{\out Level 1}
{\out app(a : b : c : Nil, d : e : Nil, a : ?zs1)}
{\out  1. app(b : c : Nil, d : e : Nil, ?zs1)}
\ttbreak
by (resolve_tac [appNil,appCons] 1);
{\out Level 2}
{\out app(a : b : c : Nil, d : e : Nil, a : b : ?zs2)}
{\out  1. app(c : Nil, d : e : Nil, ?zs2)}
\end{ttbox}
At this point, the first two elements of the result are~$a$ and~$b$.
\begin{ttbox}
by (resolve_tac [appNil,appCons] 1);
{\out Level 3}
{\out app(a : b : c : Nil, d : e : Nil, a : b : c : ?zs3)}
{\out  1. app(Nil, d : e : Nil, ?zs3)}
\ttbreak
by (resolve_tac [appNil,appCons] 1);
{\out Level 4}
{\out app(a : b : c : Nil, d : e : Nil, a : b : c : d : e : Nil)}
{\out No subgoals!}
\end{ttbox}

Prolog can run functions backwards.  Which list can be appended
with $[c,d]$ to produce $[a,b,c,d]$?
Using \ttindex{REPEAT}, we find the answer at once, $[a,b]$:
\begin{ttbox}
goal Prolog.thy "app(?x, c:d:Nil, a:b:c:d:Nil)";
{\out Level 0}
{\out app(?x, c : d : Nil, a : b : c : d : Nil)}
{\out  1. app(?x, c : d : Nil, a : b : c : d : Nil)}
\ttbreak
by (REPEAT (resolve_tac [appNil,appCons] 1));
{\out Level 1}
{\out app(a : b : Nil, c : d : Nil, a : b : c : d : Nil)}
{\out No subgoals!}
\end{ttbox}


\subsection{Backtracking}\index{backtracking!Prolog style}
Prolog backtracking can answer questions that have multiple solutions.
Which lists $x$ and $y$ can be appended to form the list $[a,b,c,d]$?  This
question has five solutions.  Using \ttindex{REPEAT} to apply the rules, we
quickly find the first solution, namely $x=[]$ and $y=[a,b,c,d]$:
\begin{ttbox}
goal Prolog.thy "app(?x, ?y, a:b:c:d:Nil)";
{\out Level 0}
{\out app(?x, ?y, a : b : c : d : Nil)}
{\out  1. app(?x, ?y, a : b : c : d : Nil)}
\ttbreak
by (REPEAT (resolve_tac [appNil,appCons] 1));
{\out Level 1}
{\out app(Nil, a : b : c : d : Nil, a : b : c : d : Nil)}
{\out No subgoals!}
\end{ttbox}
Isabelle can lazily generate all the possibilities.  The \ttindex{back}
command returns the tactic's next outcome, namely $x=[a]$ and $y=[b,c,d]$:
\begin{ttbox}
back();
{\out Level 1}
{\out app(a : Nil, b : c : d : Nil, a : b : c : d : Nil)}
{\out No subgoals!}
\end{ttbox}
The other solutions are generated similarly.
\begin{ttbox}
back();
{\out Level 1}
{\out app(a : b : Nil, c : d : Nil, a : b : c : d : Nil)}
{\out No subgoals!}
\ttbreak
back();
{\out Level 1}
{\out app(a : b : c : Nil, d : Nil, a : b : c : d : Nil)}
{\out No subgoals!}
\ttbreak
back();
{\out Level 1}
{\out app(a : b : c : d : Nil, Nil, a : b : c : d : Nil)}
{\out No subgoals!}
\end{ttbox}


\subsection{Depth-first search}
\index{search!depth-first}
Now let us try $rev$, reversing a list.
Bundle the rules together as the \ML{} identifier {\tt rules}.  Naive
reverse requires 120 inferences for this 14-element list, but the tactic
terminates in a few seconds.
\begin{ttbox}
goal Prolog.thy "rev(a:b:c:d:e:f:g:h:i:j:k:l:m:n:Nil, ?w)";
{\out Level 0}
{\out rev(a : b : c : d : e : f : g : h : i : j : k : l : m : n : Nil, ?w)}
{\out  1. rev(a : b : c : d : e : f : g : h : i : j : k : l : m : n : Nil,}
{\out         ?w)}
\ttbreak
val rules = [appNil,appCons,revNil,revCons];
\ttbreak
by (REPEAT (resolve_tac rules 1));
{\out Level 1}
{\out rev(a : b : c : d : e : f : g : h : i : j : k : l : m : n : Nil,}
{\out     n : m : l : k : j : i : h : g : f : e : d : c : b : a : Nil)}
{\out No subgoals!}
\end{ttbox}
We may execute $rev$ backwards.  This, too, should reverse a list.  What
is the reverse of $[a,b,c]$?
\begin{ttbox}
goal Prolog.thy "rev(?x, a:b:c:Nil)";
{\out Level 0}
{\out rev(?x, a : b : c : Nil)}
{\out  1. rev(?x, a : b : c : Nil)}
\ttbreak
by (REPEAT (resolve_tac rules 1));
{\out Level 1}
{\out rev(?x1 : Nil, a : b : c : Nil)}
{\out  1. app(Nil, ?x1 : Nil, a : b : c : Nil)}
\end{ttbox}
The tactic has failed to find a solution!  It reached a dead end at
subgoal~1: there is no~$\Var{x@1}$ such that [] appended with~$[\Var{x@1}]$
equals~$[a,b,c]$.  Backtracking explores other outcomes.
\begin{ttbox}
back();
{\out Level 1}
{\out rev(?x1 : a : Nil, a : b : c : Nil)}
{\out  1. app(Nil, ?x1 : Nil, b : c : Nil)}
\end{ttbox}
This too is a dead end, but the next outcome is successful.
\begin{ttbox}
back();
{\out Level 1}
{\out rev(c : b : a : Nil, a : b : c : Nil)}
{\out No subgoals!}
\end{ttbox}
\ttindex{REPEAT} goes wrong because it is only a repetition tactical, not a
search tactical.  {\tt REPEAT} stops when it cannot continue, regardless of
which state is reached.  The tactical \ttindex{DEPTH_FIRST} searches for a
satisfactory state, as specified by an \ML{} predicate.  Below,
\ttindex{has_fewer_prems} specifies that the proof state should have no
subgoals.
\begin{ttbox}
val prolog_tac = DEPTH_FIRST (has_fewer_prems 1) 
                             (resolve_tac rules 1);
\end{ttbox}
Since Prolog uses depth-first search, this tactic is a (slow!) 
Prolog interpreter.  We return to the start of the proof using
\ttindex{choplev}, and apply {\tt prolog_tac}:
\begin{ttbox}
choplev 0;
{\out Level 0}
{\out rev(?x, a : b : c : Nil)}
{\out  1. rev(?x, a : b : c : Nil)}
\ttbreak
by (DEPTH_FIRST (has_fewer_prems 1) (resolve_tac rules 1));
{\out Level 1}
{\out rev(c : b : a : Nil, a : b : c : Nil)}
{\out No subgoals!}
\end{ttbox}
Let us try {\tt prolog_tac} on one more example, containing four unknowns:
\begin{ttbox}
goal Prolog.thy "rev(a:?x:c:?y:Nil, d:?z:b:?u)";
{\out Level 0}
{\out rev(a : ?x : c : ?y : Nil, d : ?z : b : ?u)}
{\out  1. rev(a : ?x : c : ?y : Nil, d : ?z : b : ?u)}
\ttbreak
by prolog_tac;
{\out Level 1}
{\out rev(a : b : c : d : Nil, d : c : b : a : Nil)}
{\out No subgoals!}
\end{ttbox}
Although Isabelle is much slower than a Prolog system, Isabelle
tactics can exploit logic programming techniques.  



\bibliographystyle{plain} \small\raggedright\frenchspacing
\bibliography{string,atp,funprog,general,logicprog,isabelle,theory,crossref}


\chapter{Introduction}

\section{Quick start}

FIXME examples, ProofGeneral setup

\section{Examples}

\section{How to write Isar proofs anyway?}


%%% Local Variables: 
%%% mode: latex
%%% TeX-master: "isar-ref"
%%% End: 

\end{document}
