%% $Id$
\chapter{Higher-Order Logic}
\index{higher-order logic|(}
\index{HOL system@{\sc hol} system}

The theory~\thydx{HOL} implements higher-order logic.  It is based on
Gordon's~{\sc hol} system~\cite{mgordon-hol}, which itself is based on
Church's original paper~\cite{church40}.  Andrews's book~\cite{andrews86} is a
full description of higher-order logic.  Experience with the {\sc hol} system
has demonstrated that higher-order logic is useful for hardware verification;
beyond this, it is widely applicable in many areas of mathematics.  It is
weaker than {\ZF} set theory but for most applications this does not matter.
If you prefer {\ML} to Lisp, you will probably prefer \HOL\ to~{\ZF}.

The syntax of Isabelle's \HOL\ has recently been changed to look more like the
traditional syntax of higher-order logic.  Function application is now
curried.  To apply the function~$f$ to the arguments~$a$ and~$b$ in \HOL, you
must write $f\,a\,b$.  Note that $f(a,b)$ means ``$f$ applied to the pair
$(a,b)$'' in \HOL.  We write ordered pairs as $(a,b)$, not $\langle
a,b\rangle$ as in {\ZF} and earlier versions of \HOL.  Early releases of
Isabelle included still another version of~\HOL, with explicit type inference
rules~\cite{paulson-COLOG}.  This version no longer exists, but \thydx{ZF}
supports a similar style of reasoning.

\HOL\ has a distinct feel, compared with {\ZF} and {\CTT}.  It
identifies object-level types with meta-level types, taking advantage of
Isabelle's built-in type checker.  It identifies object-level functions
with meta-level functions, so it uses Isabelle's operations for abstraction
and application.  There is no `apply' operator: function applications are
written as simply~$f~a$ rather than $f{\tt`}a$.

These identifications allow Isabelle to support \HOL\ particularly nicely,
but they also mean that \HOL\ requires more sophistication from the user
--- in particular, an understanding of Isabelle's type system.  Beginners
should work with {\tt show_types} set to {\tt true}.  Gain experience by
working in first-order logic before attempting to use higher-order logic.
This chapter assumes familiarity with~{\FOL{}}.


\begin{figure} 
\begin{constants}
  \it name      &\it meta-type  & \it description \\ 
  \cdx{Trueprop}& $bool\To prop$                & coercion to $prop$\\
  \cdx{not}     & $bool\To bool$                & negation ($\neg$) \\
  \cdx{True}    & $bool$                        & tautology ($\top$) \\
  \cdx{False}   & $bool$                        & absurdity ($\bot$) \\
  \cdx{If}      & $[bool,\alpha,\alpha]\To\alpha::term$ & conditional \\
  \cdx{Inv}     & $(\alpha\To\beta)\To(\beta\To\alpha)$ & function inversion\\
  \cdx{Let}     & $[\alpha,\alpha\To\beta]\To\beta$ & let binder
\end{constants}
\subcaption{Constants}

\begin{constants}
\index{"@@{\tt\at} symbol}
\index{*"! symbol}\index{*"? symbol}
\index{*"?"! symbol}\index{*"E"X"! symbol}
  \it symbol &\it name     &\it meta-type & \it description \\
  \tt\at & \cdx{Eps}  & $(\alpha\To bool)\To\alpha::term$ & 
        Hilbert description ($\epsilon$) \\
  {\tt!~} or \sdx{ALL}  & \cdx{All}  & $(\alpha::term\To bool)\To bool$ & 
        universal quantifier ($\forall$) \\
  {\tt?~} or \sdx{EX}   & \cdx{Ex}   & $(\alpha::term\To bool)\To bool$ & 
        existential quantifier ($\exists$) \\
  {\tt?!} or {\tt EX!}  & \cdx{Ex1}  & $(\alpha::term\To bool)\To bool$ & 
        unique existence ($\exists!$)
\end{constants}
\subcaption{Binders} 

\begin{constants}
\index{*"= symbol}
\index{&@{\tt\&} symbol}
\index{*"| symbol}
\index{*"-"-"> symbol}
  \it symbol    & \it meta-type & \it priority & \it description \\ 
  \sdx{o}       & $[\beta\To\gamma,\alpha\To\beta]\To (\alpha\To\gamma)$ & 
        Left 55 & composition ($\circ$) \\
  \tt =         & $[\alpha::term,\alpha]\To bool$ & Left 50 & equality ($=$) \\
  \tt <         & $[\alpha::ord,\alpha]\To bool$ & Left 50 & less than ($<$) \\
  \tt <=        & $[\alpha::ord,\alpha]\To bool$ & Left 50 & 
                less than or equals ($\leq$)\\
  \tt \&        & $[bool,bool]\To bool$ & Right 35 & conjunction ($\conj$) \\
  \tt |         & $[bool,bool]\To bool$ & Right 30 & disjunction ($\disj$) \\
  \tt -->       & $[bool,bool]\To bool$ & Right 25 & implication ($\imp$)
\end{constants}
\subcaption{Infixes}
\caption{Syntax of {\tt HOL}} \label{hol-constants}
\end{figure}


\begin{figure}
\index{*let symbol}
\index{*in symbol}
\dquotes
\[\begin{array}{rclcl}
    term & = & \hbox{expression of class~$term$} \\
         & | & "\at~" id~id^* " . " formula \\
         & | & 
    \multicolumn{3}{l}{"let"~id~"="~term";"\dots";"~id~"="~term~"in"~term} \\
         & | & 
    \multicolumn{3}{l}{"if"~formula~"then"~term~"else"~term} \\[2ex]
 formula & = & \hbox{expression of type~$bool$} \\
         & | & term " = " term \\
         & | & term " \ttilde= " term \\
         & | & term " < " term \\
         & | & term " <= " term \\
         & | & "\ttilde\ " formula \\
         & | & formula " \& " formula \\
         & | & formula " | " formula \\
         & | & formula " --> " formula \\
         & | & "!~~~" id~id^* " . " formula 
         & | & "ALL~" id~id^* " . " formula \\
         & | & "?~~~" id~id^* " . " formula 
         & | & "EX~~" id~id^* " . " formula \\
         & | & "?!~~" id~id^* " . " formula 
         & | & "EX!~" id~id^* " . " formula
  \end{array}
\]
\caption{Full grammar for \HOL} \label{hol-grammar}
\end{figure} 


\section{Syntax}
The type class of higher-order terms is called~\cldx{term}.  Type variables
range over this class by default.  The equality symbol and quantifiers are
polymorphic over class {\tt term}.

Class \cldx{ord} consists of all ordered types; the relations $<$ and
$\leq$ are polymorphic over this class, as are the functions
\cdx{mono}, \cdx{min} and \cdx{max}.  Three other
type classes --- \cldx{plus}, \cldx{minus} and \cldx{times} --- permit
overloading of the operators {\tt+}, {\tt-} and {\tt*}.  In particular,
{\tt-} is overloaded for set difference and subtraction.
\index{*"+ symbol}
\index{*"- symbol}
\index{*"* symbol}

Figure~\ref{hol-constants} lists the constants (including infixes and
binders), while Fig.\ts\ref{hol-grammar} presents the grammar of
higher-order logic.  Note that $a$\verb|~=|$b$ is translated to
$\neg(a=b)$.

\begin{warn}
  \HOL\ has no if-and-only-if connective; logical equivalence is expressed
  using equality.  But equality has a high priority, as befitting a
  relation, while if-and-only-if typically has the lowest priority.  Thus,
  $\neg\neg P=P$ abbreviates $\neg\neg (P=P)$ and not $(\neg\neg P)=P$.
  When using $=$ to mean logical equivalence, enclose both operands in
  parentheses.
\end{warn}

\subsection{Types}\label{hol-types}
The type of formulae, \tydx{bool}, belongs to class \cldx{term}; thus,
formulae are terms.  The built-in type~\tydx{fun}, which constructs function
types, is overloaded with arity {\tt(term,term)term}.  Thus, $\sigma\To\tau$
belongs to class~{\tt term} if $\sigma$ and~$\tau$ do, allowing quantification
over functions.

HOL offers various methods for introducing new types. For details
see~\S\ref{sec:HOL:Types} and~\S\ref{sec:HOL:datatype}.


\subsection{Binders}
Hilbert's {\bf description} operator~$\epsilon x.P[x]$ stands for some~$a$
satisfying~$P[a]$, if such exists.  Since all terms in \HOL\ denote
something, a description is always meaningful, but we do not know its value
unless $P[x]$ defines it uniquely.  We may write descriptions as
\cdx{Eps}($P$) or use the syntax
\hbox{\tt \at $x$.$P[x]$}.

Existential quantification is defined by
\[ \exists x.P~x \;\equiv\; P(\epsilon x.P~x). \]
The unique existence quantifier, $\exists!x.P[x]$, is defined in terms
of~$\exists$ and~$\forall$.  An Isabelle binder, it admits nested
quantifications.  For instance, $\exists!x y.P~x~y$ abbreviates
$\exists!x. \exists!y.P~x~y$; note that this does not mean that there
exists a unique pair $(x,y)$ satisfying~$P~x~y$.

\index{*"! symbol}\index{*"? symbol}\index{HOL system@{\sc hol} system}
Quantifiers have two notations.  As in Gordon's {\sc hol} system, \HOL\
uses~{\tt!}\ and~{\tt?}\ to stand for $\forall$ and $\exists$.  The
existential quantifier must be followed by a space; thus {\tt?x} is an
unknown, while \verb'? x.f x=y' is a quantification.  Isabelle's usual
notation for quantifiers, \sdx{ALL} and \sdx{EX}, is also
available.  Both notations are accepted for input.  The {\ML} reference
\ttindexbold{HOL_quantifiers} governs the output notation.  If set to {\tt
true}, then~{\tt!}\ and~{\tt?}\ are displayed; this is the default.  If set
to {\tt false}, then~{\tt ALL} and~{\tt EX} are displayed.

All these binders have priority 10. 


\subsection{The \sdx{let} and \sdx{case} constructions}
Local abbreviations can be introduced by a {\tt let} construct whose
syntax appears in Fig.\ts\ref{hol-grammar}.  Internally it is translated into
the constant~\cdx{Let}.  It can be expanded by rewriting with its
definition, \tdx{Let_def}.

\HOL\ also defines the basic syntax
\[\dquotes"case"~e~"of"~c@1~"=>"~e@1~"|" \dots "|"~c@n~"=>"~e@n\] 
as a uniform means of expressing {\tt case} constructs.  Therefore {\tt case}
and \sdx{of} are reserved words.  Initially, this is mere syntax and has no
logical meaning.  By declaring translations, you can cause instances of the
{\tt case} construct to denote applications of particular case operators.
This is what happens automatically for each {\tt datatype} declaration. For
example \verb$datatype nat = Z | S nat$ declares a translation between
\verb$case x of Z => a | S n => b$ and \verb$nat_case a (%n.b) x$, where
\verb$nat_case$ is some appropriate function.


\begin{figure}
\begin{ttbox}\makeatother
\tdx{refl}           t = t
\tdx{subst}          [| s=t; P s |] ==> P t
\tdx{ext}            (!!x. f x = g x) ==> (\%x.f x) = (\%x.g x)
\tdx{impI}           (P ==> Q) ==> P-->Q
\tdx{mp}             [| P-->Q;  P |] ==> Q
\tdx{iff}            (P-->Q) --> (Q-->P) --> (P=Q)
\tdx{selectI}        P(x) ==> P(@x.P x)
\tdx{True_or_False}  (P=True) | (P=False)
\end{ttbox}
\caption{The {\tt HOL} rules} \label{hol-rules}
\end{figure}


\begin{figure}\hfuzz=4pt%suppress "Overfull \hbox" message
\begin{ttbox}\makeatother
\tdx{True_def}   True     == ((\%x::bool.x)=(\%x.x))
\tdx{All_def}    All      == (\%P. P = (\%x.True))
\tdx{Ex_def}     Ex       == (\%P. P(@x.P x))
\tdx{False_def}  False    == (!P.P)
\tdx{not_def}    not      == (\%P. P-->False)
\tdx{and_def}    op &     == (\%P Q. !R. (P-->Q-->R) --> R)
\tdx{or_def}     op |     == (\%P Q. !R. (P-->R) --> (Q-->R) --> R)
\tdx{Ex1_def}    Ex1      == (\%P. ? x. P x & (! y. P y --> y=x))

\tdx{Inv_def}    Inv      == (\%f y. @x. f x = y)
\tdx{o_def}      op o     == (\%f g x. f(g x))
\tdx{if_def}     If P x y == (\%P x y.@z.(P=True --> z=x) & (P=False --> z=y))
\tdx{Let_def}    Let s f  == f s
\end{ttbox}
\caption{The {\tt HOL} definitions} \label{hol-defs}
\end{figure}


\section{Rules of inference}
Figure~\ref{hol-rules} shows the inference rules of~\HOL{}, with
their~{\ML} names.  Some of the rules deserve additional comments:
\begin{ttdescription}
\item[\tdx{ext}] expresses extensionality of functions.
\item[\tdx{iff}] asserts that logically equivalent formulae are
  equal.
\item[\tdx{selectI}] gives the defining property of the Hilbert
  $\epsilon$-operator.  It is a form of the Axiom of Choice.  The derived rule
  \tdx{select_equality} (see below) is often easier to use.
\item[\tdx{True_or_False}] makes the logic classical.\footnote{In
    fact, the $\epsilon$-operator already makes the logic classical, as
    shown by Diaconescu; see Paulson~\cite{paulson-COLOG} for details.}
\end{ttdescription}

\HOL{} follows standard practice in higher-order logic: only a few
connectives are taken as primitive, with the remainder defined obscurely
(Fig.\ts\ref{hol-defs}).  Gordon's {\sc hol} system expresses the
corresponding definitions \cite[page~270]{mgordon-hol} using
object-equality~({\tt=}), which is possible because equality in
higher-order logic may equate formulae and even functions over formulae.
But theory~\HOL{}, like all other Isabelle theories, uses
meta-equality~({\tt==}) for definitions.

Some of the rules mention type variables; for example, {\tt refl}
mentions the type variable~{\tt'a}.  This allows you to instantiate
type variables explicitly by calling {\tt res_inst_tac}.  By default,
explicit type variables have class \cldx{term}.

Include type constraints whenever you state a polymorphic goal.  Type
inference may otherwise make the goal more polymorphic than you intended,
with confusing results.

\begin{warn}
  If resolution fails for no obvious reason, try setting
  \ttindex{show_types} to {\tt true}, causing Isabelle to display types of
  terms.  Possibly set \ttindex{show_sorts} to {\tt true} as well, causing
  Isabelle to display sorts.

  \index{unification!incompleteness of}
  Where function types are involved, Isabelle's unification code does not
  guarantee to find instantiations for type variables automatically.  Be
  prepared to use \ttindex{res_inst_tac} instead of {\tt resolve_tac},
  possibly instantiating type variables.  Setting
  \ttindex{Unify.trace_types} to {\tt true} causes Isabelle to report
  omitted search paths during unification.\index{tracing!of unification}
\end{warn}


\begin{figure}
\begin{ttbox}
\tdx{sym}         s=t ==> t=s
\tdx{trans}       [| r=s; s=t |] ==> r=t
\tdx{ssubst}      [| t=s; P s |] ==> P t
\tdx{box_equals}  [| a=b;  a=c;  b=d |] ==> c=d  
\tdx{arg_cong}    x = y ==> f x = f y
\tdx{fun_cong}    f = g ==> f x = g x
\tdx{cong}        [| f = g; x = y |] ==> f x = g y
\tdx{not_sym}     t ~= s ==> s ~= t
\subcaption{Equality}

\tdx{TrueI}       True 
\tdx{FalseE}      False ==> P

\tdx{conjI}       [| P; Q |] ==> P&Q
\tdx{conjunct1}   [| P&Q |] ==> P
\tdx{conjunct2}   [| P&Q |] ==> Q 
\tdx{conjE}       [| P&Q;  [| P; Q |] ==> R |] ==> R

\tdx{disjI1}      P ==> P|Q
\tdx{disjI2}      Q ==> P|Q
\tdx{disjE}       [| P | Q; P ==> R; Q ==> R |] ==> R

\tdx{notI}        (P ==> False) ==> ~ P
\tdx{notE}        [| ~ P;  P |] ==> R
\tdx{impE}        [| P-->Q;  P;  Q ==> R |] ==> R
\subcaption{Propositional logic}

\tdx{iffI}        [| P ==> Q;  Q ==> P |] ==> P=Q
\tdx{iffD1}       [| P=Q; P |] ==> Q
\tdx{iffD2}       [| P=Q; Q |] ==> P
\tdx{iffE}        [| P=Q; [| P --> Q; Q --> P |] ==> R |] ==> R
%
%\tdx{eqTrueI}     P ==> P=True 
%\tdx{eqTrueE}     P=True ==> P 
\subcaption{Logical equivalence}

\end{ttbox}
\caption{Derived rules for \HOL} \label{hol-lemmas1}
\end{figure}


\begin{figure}
\begin{ttbox}\makeatother
\tdx{allI}      (!!x. P x) ==> !x. P x
\tdx{spec}      !x.P x ==> P x
\tdx{allE}      [| !x.P x;  P x ==> R |] ==> R
\tdx{all_dupE}  [| !x.P x;  [| P x; !x.P x |] ==> R |] ==> R

\tdx{exI}       P x ==> ? x. P x
\tdx{exE}       [| ? x. P x; !!x. P x ==> Q |] ==> Q

\tdx{ex1I}      [| P a;  !!x. P x ==> x=a |] ==> ?! x. P x
\tdx{ex1E}      [| ?! x.P x;  !!x. [| P x;  ! y. P y --> y=x |] ==> R 
          |] ==> R

\tdx{select_equality} [| P a;  !!x. P x ==> x=a |] ==> (@x.P x) = a
\subcaption{Quantifiers and descriptions}

\tdx{ccontr}          (~P ==> False) ==> P
\tdx{classical}       (~P ==> P) ==> P
\tdx{excluded_middle} ~P | P

\tdx{disjCI}          (~Q ==> P) ==> P|Q
\tdx{exCI}            (! x. ~ P x ==> P a) ==> ? x.P x
\tdx{impCE}           [| P-->Q; ~ P ==> R; Q ==> R |] ==> R
\tdx{iffCE}           [| P=Q;  [| P;Q |] ==> R;  [| ~P; ~Q |] ==> R |] ==> R
\tdx{notnotD}         ~~P ==> P
\tdx{swap}            ~P ==> (~Q ==> P) ==> Q
\subcaption{Classical logic}

%\tdx{if_True}         (if True then x else y) = x
%\tdx{if_False}        (if False then x else y) = y
\tdx{if_P}            P ==> (if P then x else y) = x
\tdx{if_not_P}        ~ P ==> (if P then x else y) = y
\tdx{expand_if}       P(if Q then x else y) = ((Q --> P x) & (~Q --> P y))
\subcaption{Conditionals}
\end{ttbox}
\caption{More derived rules} \label{hol-lemmas2}
\end{figure}


Some derived rules are shown in Figures~\ref{hol-lemmas1}
and~\ref{hol-lemmas2}, with their {\ML} names.  These include natural rules
for the logical connectives, as well as sequent-style elimination rules for
conjunctions, implications, and universal quantifiers.  

Note the equality rules: \tdx{ssubst} performs substitution in
backward proofs, while \tdx{box_equals} supports reasoning by
simplifying both sides of an equation.

The following simple tactics are occasionally useful:
\begin{ttdescription}
\item[\ttindexbold{strip_tac} $i$] applies {\tt allI} and {\tt impI}
  repeatedly to remove all outermost universal quantifiers and implications
  from subgoal $i$.
\item[\ttindexbold{case_tac} {\tt"}$P${\tt"} $i$] performs case distinction
  on $P$ for subgoal $i$: the latter is replaced by two identical subgoals
  with the added assumptions $P$ and $\neg P$, respectively.
\end{ttdescription}


\begin{figure} 
\begin{center}
\begin{tabular}{rrr}
  \it name      &\it meta-type  & \it description \\ 
\index{{}@\verb'{}' symbol}
  \verb|{}|     & $\alpha\,set$         & the empty set \\
  \cdx{insert}  & $[\alpha,\alpha\,set]\To \alpha\,set$
        & insertion of element \\
  \cdx{Collect} & $(\alpha\To bool)\To\alpha\,set$
        & comprehension \\
  \cdx{Compl}   & $(\alpha\,set)\To\alpha\,set$
        & complement \\
  \cdx{INTER} & $[\alpha\,set,\alpha\To\beta\,set]\To\beta\,set$
        & intersection over a set\\
  \cdx{UNION} & $[\alpha\,set,\alpha\To\beta\,set]\To\beta\,set$
        & union over a set\\
  \cdx{Inter} & $(\alpha\,set)set\To\alpha\,set$
        &set of sets intersection \\
  \cdx{Union} & $(\alpha\,set)set\To\alpha\,set$
        &set of sets union \\
  \cdx{Pow}   & $\alpha\,set \To (\alpha\,set)set$
        & powerset \\[1ex]
  \cdx{range}   & $(\alpha\To\beta )\To\beta\,set$
        & range of a function \\[1ex]
  \cdx{Ball}~~\cdx{Bex} & $[\alpha\,set,\alpha\To bool]\To bool$
        & bounded quantifiers \\
  \cdx{mono}    & $(\alpha\,set\To\beta\,set)\To bool$
        & monotonicity \\
  \cdx{inj}~~\cdx{surj}& $(\alpha\To\beta )\To bool$
        & injective/surjective \\
  \cdx{inj_onto}        & $[\alpha\To\beta ,\alpha\,set]\To bool$
        & injective over subset
\end{tabular}
\end{center}
\subcaption{Constants}

\begin{center}
\begin{tabular}{llrrr} 
  \it symbol &\it name     &\it meta-type & \it priority & \it description \\
  \sdx{INT}  & \cdx{INTER1}  & $(\alpha\To\beta\,set)\To\beta\,set$ & 10 & 
        intersection over a type\\
  \sdx{UN}  & \cdx{UNION1}  & $(\alpha\To\beta\,set)\To\beta\,set$ & 10 & 
        union over a type
\end{tabular}
\end{center}
\subcaption{Binders} 

\begin{center}
\index{*"`"` symbol}
\index{*": symbol}
\index{*"<"= symbol}
\begin{tabular}{rrrr} 
  \it symbol    & \it meta-type & \it priority & \it description \\ 
  \tt ``        & $[\alpha\To\beta ,\alpha\,set]\To  (\beta\,set)$
        & Left 90 & image \\
  \sdx{Int}     & $[\alpha\,set,\alpha\,set]\To\alpha\,set$
        & Left 70 & intersection ($\inter$) \\
  \sdx{Un}      & $[\alpha\,set,\alpha\,set]\To\alpha\,set$
        & Left 65 & union ($\union$) \\
  \tt:          & $[\alpha ,\alpha\,set]\To bool$       
        & Left 50 & membership ($\in$) \\
  \tt <=        & $[\alpha\,set,\alpha\,set]\To bool$
        & Left 50 & subset ($\subseteq$) 
\end{tabular}
\end{center}
\subcaption{Infixes}
\caption{Syntax of the theory {\tt Set}} \label{hol-set-syntax}
\end{figure} 


\begin{figure} 
\begin{center} \tt\frenchspacing
\index{*"! symbol}
\begin{tabular}{rrr} 
  \it external          & \it internal  & \it description \\ 
  $a$ \ttilde: $b$      & \ttilde($a$ : $b$)    & \rm non-membership\\
  \{$a@1$, $\ldots$\}  &  insert $a@1$ $\ldots$ \{\} & \rm finite set \\
  \{$x$.$P[x]$\}        &  Collect($\lambda x.P[x]$) &
        \rm comprehension \\
  \sdx{INT} $x$:$A$.$B[x]$      & INTER $A$ $\lambda x.B[x]$ &
        \rm intersection \\
  \sdx{UN}{\tt\ }  $x$:$A$.$B[x]$      & UNION $A$ $\lambda x.B[x]$ &
        \rm union \\
  \tt ! $x$:$A$.$P[x]$ or \sdx{ALL} $x$:$A$.$P[x]$ & 
        Ball $A$ $\lambda x.P[x]$ & 
        \rm bounded $\forall$ \\
  \sdx{?} $x$:$A$.$P[x]$ or \sdx{EX}{\tt\ } $x$:$A$.$P[x]$ & 
        Bex $A$ $\lambda x.P[x]$ & \rm bounded $\exists$
\end{tabular}
\end{center}
\subcaption{Translations}

\dquotes
\[\begin{array}{rclcl}
    term & = & \hbox{other terms\ldots} \\
         & | & "\{\}" \\
         & | & "\{ " term\; ("," term)^* " \}" \\
         & | & "\{ " id " . " formula " \}" \\
         & | & term " `` " term \\
         & | & term " Int " term \\
         & | & term " Un " term \\
         & | & "INT~~"  id ":" term " . " term \\
         & | & "UN~~~"  id ":" term " . " term \\
         & | & "INT~~"  id~id^* " . " term \\
         & | & "UN~~~"  id~id^* " . " term \\[2ex]
 formula & = & \hbox{other formulae\ldots} \\
         & | & term " : " term \\
         & | & term " \ttilde: " term \\
         & | & term " <= " term \\
         & | & "!~" id ":" term " . " formula 
         & | & "ALL " id ":" term " . " formula \\
         & | & "?~" id ":" term " . " formula 
         & | & "EX~~" id ":" term " . " formula
  \end{array}
\]
\subcaption{Full Grammar}
\caption{Syntax of the theory {\tt Set} (continued)} \label{hol-set-syntax2}
\end{figure} 


\section{A formulation of set theory}
Historically, higher-order logic gives a foundation for Russell and
Whitehead's theory of classes.  Let us use modern terminology and call them
{\bf sets}, but note that these sets are distinct from those of {\ZF} set
theory, and behave more like {\ZF} classes.
\begin{itemize}
\item
Sets are given by predicates over some type~$\sigma$.  Types serve to
define universes for sets, but type checking is still significant.
\item
There is a universal set (for each type).  Thus, sets have complements, and
may be defined by absolute comprehension.
\item
Although sets may contain other sets as elements, the containing set must
have a more complex type.
\end{itemize}
Finite unions and intersections have the same behaviour in \HOL\ as they
do in~{\ZF}.  In \HOL\ the intersection of the empty set is well-defined,
denoting the universal set for the given type.

% FIXME: define set via typedef
\subsection{Syntax of set theory}\index{*set type}
\HOL's set theory is called \thydx{Set}.  The type $\alpha\,set$ is
essentially the same as $\alpha\To bool$.  The new type is defined for
clarity and to avoid complications involving function types in unification.
Since Isabelle does not support type definitions (as mentioned in
\S\ref{hol-types}), the isomorphisms between the two types are declared
explicitly.  Here they are natural: {\tt Collect} maps $\alpha\To bool$ to
$\alpha\,set$, while \hbox{\tt op :} maps in the other direction (ignoring
argument order).

Figure~\ref{hol-set-syntax} lists the constants, infixes, and syntax
translations.  Figure~\ref{hol-set-syntax2} presents the grammar of the new
constructs.  Infix operators include union and intersection ($A\union B$
and $A\inter B$), the subset and membership relations, and the image
operator~{\tt``}\@.  Note that $a$\verb|~:|$b$ is translated to
$\neg(a\in b)$.  

The {\tt\{\ldots\}} notation abbreviates finite sets constructed in the
obvious manner using~{\tt insert} and~$\{\}$:
\begin{eqnarray*}
  \{a@1, \ldots, a@n\}  & \equiv &  
  {\tt insert}~a@1~({\tt insert}\ldots({\tt insert}~a@n~\{\})\ldots)
\end{eqnarray*}

The set \hbox{\tt\{$x$.$P[x]$\}} consists of all $x$ (of suitable type)
that satisfy~$P[x]$, where $P[x]$ is a formula that may contain free
occurrences of~$x$.  This syntax expands to \cdx{Collect}$(\lambda
x.P[x])$.  It defines sets by absolute comprehension, which is impossible
in~{\ZF}; the type of~$x$ implicitly restricts the comprehension.

The set theory defines two {\bf bounded quantifiers}:
\begin{eqnarray*}
   \forall x\in A.P[x] &\hbox{abbreviates}& \forall x. x\in A\imp P[x] \\
   \exists x\in A.P[x] &\hbox{abbreviates}& \exists x. x\in A\conj P[x]
\end{eqnarray*}
The constants~\cdx{Ball} and~\cdx{Bex} are defined
accordingly.  Instead of {\tt Ball $A$ $P$} and {\tt Bex $A$ $P$} we may
write\index{*"! symbol}\index{*"? symbol}
\index{*ALL symbol}\index{*EX symbol} 
%
\hbox{\tt !~$x$:$A$.$P[x]$} and \hbox{\tt ?~$x$:$A$.$P[x]$}.  Isabelle's
usual quantifier symbols, \sdx{ALL} and \sdx{EX}, are also accepted
for input.  As with the primitive quantifiers, the {\ML} reference
\ttindex{HOL_quantifiers} specifies which notation to use for output.

Unions and intersections over sets, namely $\bigcup@{x\in A}B[x]$ and
$\bigcap@{x\in A}B[x]$, are written 
\sdx{UN}~\hbox{\tt$x$:$A$.$B[x]$} and
\sdx{INT}~\hbox{\tt$x$:$A$.$B[x]$}.  

Unions and intersections over types, namely $\bigcup@x B[x]$ and $\bigcap@x
B[x]$, are written \sdx{UN}~\hbox{\tt$x$.$B[x]$} and
\sdx{INT}~\hbox{\tt$x$.$B[x]$}.  They are equivalent to the previous
union and intersection operators when $A$ is the universal set.

The operators $\bigcup A$ and $\bigcap A$ act upon sets of sets.  They are
not binders, but are equal to $\bigcup@{x\in A}x$ and $\bigcap@{x\in A}x$,
respectively.


% FIXME: remove the two laws connecting mem and Collect
\begin{figure} \underscoreon
\begin{ttbox}
\tdx{mem_Collect_eq}    (a : \{x.P x\}) = P a
\tdx{Collect_mem_eq}    \{x.x:A\} = A

\tdx{empty_def}         \{\}          == \{x.False\}
\tdx{insert_def}        insert a B  == \{x.x=a\} Un B
\tdx{Ball_def}          Ball A P    == ! x. x:A --> P x
\tdx{Bex_def}           Bex A P     == ? x. x:A & P x
\tdx{subset_def}        A <= B      == ! x:A. x:B
\tdx{Un_def}            A Un B      == \{x.x:A | x:B\}
\tdx{Int_def}           A Int B     == \{x.x:A & x:B\}
\tdx{set_diff_def}      A - B       == \{x.x:A & x~:B\}
\tdx{Compl_def}         Compl A     == \{x. ~ x:A\}
\tdx{INTER_def}         INTER A B   == \{y. ! x:A. y: B x\}
\tdx{UNION_def}         UNION A B   == \{y. ? x:A. y: B x\}
\tdx{INTER1_def}        INTER1 B    == INTER \{x.True\} B 
\tdx{UNION1_def}        UNION1 B    == UNION \{x.True\} B 
\tdx{Inter_def}         Inter S     == (INT x:S. x)
\tdx{Union_def}         Union S     == (UN  x:S. x)
\tdx{Pow_def}           Pow A       == \{B. B <= A\}
\tdx{image_def}         f``A        == \{y. ? x:A. y=f x\}
\tdx{range_def}         range f     == \{y. ? x. y=f x\}
\tdx{mono_def}          mono f      == !A B. A <= B --> f A <= f B
\tdx{inj_def}           inj f       == ! x y. f x=f y --> x=y
\tdx{surj_def}          surj f      == ! y. ? x. y=f x
\tdx{inj_onto_def}      inj_onto f A == !x:A. !y:A. f x=f y --> x=y
\end{ttbox}
\caption{Rules of the theory {\tt Set}} \label{hol-set-rules}
\end{figure}


\begin{figure} \underscoreon
\begin{ttbox}
\tdx{CollectI}        [| P a |] ==> a : \{x.P x\}
\tdx{CollectD}        [| a : \{x.P x\} |] ==> P a
\tdx{CollectE}        [| a : \{x.P x\};  P a ==> W |] ==> W

\tdx{ballI}           [| !!x. x:A ==> P x |] ==> ! x:A. P x
\tdx{bspec}           [| ! x:A. P x;  x:A |] ==> P x
\tdx{ballE}           [| ! x:A. P x;  P x ==> Q;  ~ x:A ==> Q |] ==> Q

\tdx{bexI}            [| P x;  x:A |] ==> ? x:A. P x
\tdx{bexCI}           [| ! x:A. ~ P x ==> P a;  a:A |] ==> ? x:A.P x
\tdx{bexE}            [| ? x:A. P x;  !!x. [| x:A; P x |] ==> Q  |] ==> Q
\subcaption{Comprehension and Bounded quantifiers}

\tdx{subsetI}         (!!x.x:A ==> x:B) ==> A <= B
\tdx{subsetD}         [| A <= B;  c:A |] ==> c:B
\tdx{subsetCE}        [| A <= B;  ~ (c:A) ==> P;  c:B ==> P |] ==> P

\tdx{subset_refl}     A <= A
\tdx{subset_trans}    [| A<=B;  B<=C |] ==> A<=C

\tdx{equalityI}       [| A <= B;  B <= A |] ==> A = B
\tdx{equalityD1}      A = B ==> A<=B
\tdx{equalityD2}      A = B ==> B<=A
\tdx{equalityE}       [| A = B;  [| A<=B; B<=A |] ==> P |]  ==>  P

\tdx{equalityCE}      [| A = B;  [| c:A; c:B |] ==> P;  
                           [| ~ c:A; ~ c:B |] ==> P 
                |]  ==>  P
\subcaption{The subset and equality relations}
\end{ttbox}
\caption{Derived rules for set theory} \label{hol-set1}
\end{figure}


\begin{figure} \underscoreon
\begin{ttbox}
\tdx{emptyE}   a : \{\} ==> P

\tdx{insertI1} a : insert a B
\tdx{insertI2} a : B ==> a : insert b B
\tdx{insertE}  [| a : insert b A;  a=b ==> P;  a:A ==> P |] ==> P

\tdx{ComplI}   [| c:A ==> False |] ==> c : Compl A
\tdx{ComplD}   [| c : Compl A |] ==> ~ c:A

\tdx{UnI1}     c:A ==> c : A Un B
\tdx{UnI2}     c:B ==> c : A Un B
\tdx{UnCI}     (~c:B ==> c:A) ==> c : A Un B
\tdx{UnE}      [| c : A Un B;  c:A ==> P;  c:B ==> P |] ==> P

\tdx{IntI}     [| c:A;  c:B |] ==> c : A Int B
\tdx{IntD1}    c : A Int B ==> c:A
\tdx{IntD2}    c : A Int B ==> c:B
\tdx{IntE}     [| c : A Int B;  [| c:A; c:B |] ==> P |] ==> P

\tdx{UN_I}     [| a:A;  b: B a |] ==> b: (UN x:A. B x)
\tdx{UN_E}     [| b: (UN x:A. B x);  !!x.[| x:A;  b:B x |] ==> R |] ==> R

\tdx{INT_I}    (!!x. x:A ==> b: B x) ==> b : (INT x:A. B x)
\tdx{INT_D}    [| b: (INT x:A. B x);  a:A |] ==> b: B a
\tdx{INT_E}    [| b: (INT x:A. B x);  b: B a ==> R;  ~ a:A ==> R |] ==> R

\tdx{UnionI}   [| X:C;  A:X |] ==> A : Union C
\tdx{UnionE}   [| A : Union C;  !!X.[| A:X;  X:C |] ==> R |] ==> R

\tdx{InterI}   [| !!X. X:C ==> A:X |] ==> A : Inter C
\tdx{InterD}   [| A : Inter C;  X:C |] ==> A:X
\tdx{InterE}   [| A : Inter C;  A:X ==> R;  ~ X:C ==> R |] ==> R

\tdx{PowI}     A<=B ==> A: Pow B
\tdx{PowD}     A: Pow B ==> A<=B
\end{ttbox}
\caption{Further derived rules for set theory} \label{hol-set2}
\end{figure}


\subsection{Axioms and rules of set theory}
Figure~\ref{hol-set-rules} presents the rules of theory \thydx{Set}.  The
axioms \tdx{mem_Collect_eq} and \tdx{Collect_mem_eq} assert
that the functions {\tt Collect} and \hbox{\tt op :} are isomorphisms.  Of
course, \hbox{\tt op :} also serves as the membership relation.

All the other axioms are definitions.  They include the empty set, bounded
quantifiers, unions, intersections, complements and the subset relation.
They also include straightforward properties of functions: image~({\tt``}) and
{\tt range}, and predicates concerning monotonicity, injectiveness and
surjectiveness.  

The predicate \cdx{inj_onto} is used for simulating type definitions.
The statement ${\tt inj_onto}~f~A$ asserts that $f$ is injective on the
set~$A$, which specifies a subset of its domain type.  In a type
definition, $f$ is the abstraction function and $A$ is the set of valid
representations; we should not expect $f$ to be injective outside of~$A$.

\begin{figure} \underscoreon
\begin{ttbox}
\tdx{Inv_f_f}    inj f ==> Inv f (f x) = x
\tdx{f_Inv_f}    y : range f ==> f(Inv f y) = y

%\tdx{Inv_injective}
%    [| Inv f x=Inv f y; x: range f;  y: range f |] ==> x=y
%
\tdx{imageI}     [| x:A |] ==> f x : f``A
\tdx{imageE}     [| b : f``A;  !!x.[| b=f x;  x:A |] ==> P |] ==> P

\tdx{rangeI}     f x : range f
\tdx{rangeE}     [| b : range f;  !!x.[| b=f x |] ==> P |] ==> P

\tdx{monoI}      [| !!A B. A <= B ==> f A <= f B |] ==> mono f
\tdx{monoD}      [| mono f;  A <= B |] ==> f A <= f B

\tdx{injI}       [| !! x y. f x = f y ==> x=y |] ==> inj f
\tdx{inj_inverseI}              (!!x. g(f x) = x) ==> inj f
\tdx{injD}       [| inj f; f x = f y |] ==> x=y

\tdx{inj_ontoI}  (!!x y. [| f x=f y; x:A; y:A |] ==> x=y) ==> inj_onto f A
\tdx{inj_ontoD}  [| inj_onto f A;  f x=f y;  x:A;  y:A |] ==> x=y

\tdx{inj_onto_inverseI}
    (!!x. x:A ==> g(f x) = x) ==> inj_onto f A
\tdx{inj_onto_contraD}
    [| inj_onto f A;  x~=y;  x:A;  y:A |] ==> ~ f x=f y
\end{ttbox}
\caption{Derived rules involving functions} \label{hol-fun}
\end{figure}


\begin{figure} \underscoreon
\begin{ttbox}
\tdx{Union_upper}     B:A ==> B <= Union A
\tdx{Union_least}     [| !!X. X:A ==> X<=C |] ==> Union A <= C

\tdx{Inter_lower}     B:A ==> Inter A <= B
\tdx{Inter_greatest}  [| !!X. X:A ==> C<=X |] ==> C <= Inter A

\tdx{Un_upper1}       A <= A Un B
\tdx{Un_upper2}       B <= A Un B
\tdx{Un_least}        [| A<=C;  B<=C |] ==> A Un B <= C

\tdx{Int_lower1}      A Int B <= A
\tdx{Int_lower2}      A Int B <= B
\tdx{Int_greatest}    [| C<=A;  C<=B |] ==> C <= A Int B
\end{ttbox}
\caption{Derived rules involving subsets} \label{hol-subset}
\end{figure}


\begin{figure} \underscoreon   \hfuzz=4pt%suppress "Overfull \hbox" message
\begin{ttbox}
\tdx{Int_absorb}        A Int A = A
\tdx{Int_commute}       A Int B = B Int A
\tdx{Int_assoc}         (A Int B) Int C  =  A Int (B Int C)
\tdx{Int_Un_distrib}    (A Un B)  Int C  =  (A Int C) Un (B Int C)

\tdx{Un_absorb}         A Un A = A
\tdx{Un_commute}        A Un B = B Un A
\tdx{Un_assoc}          (A Un B)  Un C  =  A Un (B Un C)
\tdx{Un_Int_distrib}    (A Int B) Un C  =  (A Un C) Int (B Un C)

\tdx{Compl_disjoint}    A Int (Compl A) = \{x.False\}
\tdx{Compl_partition}   A Un  (Compl A) = \{x.True\}
\tdx{double_complement} Compl(Compl A) = A
\tdx{Compl_Un}          Compl(A Un B)  = (Compl A) Int (Compl B)
\tdx{Compl_Int}         Compl(A Int B) = (Compl A) Un (Compl B)

\tdx{Union_Un_distrib}  Union(A Un B) = (Union A) Un (Union B)
\tdx{Int_Union}         A Int (Union B) = (UN C:B. A Int C)
\tdx{Un_Union_image}    (UN x:C.(A x) Un (B x)) = Union(A``C) Un Union(B``C)

\tdx{Inter_Un_distrib}  Inter(A Un B) = (Inter A) Int (Inter B)
\tdx{Un_Inter}          A Un (Inter B) = (INT C:B. A Un C)
\tdx{Int_Inter_image}   (INT x:C.(A x) Int (B x)) = Inter(A``C) Int Inter(B``C)
\end{ttbox}
\caption{Set equalities} \label{hol-equalities}
\end{figure}


Figures~\ref{hol-set1} and~\ref{hol-set2} present derived rules.  Most are
obvious and resemble rules of Isabelle's {\ZF} set theory.  Certain rules,
such as \tdx{subsetCE}, \tdx{bexCI} and \tdx{UnCI},
are designed for classical reasoning; the rules \tdx{subsetD},
\tdx{bexI}, \tdx{Un1} and~\tdx{Un2} are not
strictly necessary but yield more natural proofs.  Similarly,
\tdx{equalityCE} supports classical reasoning about extensionality,
after the fashion of \tdx{iffCE}.  See the file {\tt HOL/Set.ML} for
proofs pertaining to set theory.

Figure~\ref{hol-fun} presents derived inference rules involving functions.
They also include rules for \cdx{Inv}, which is defined in theory~{\tt
  HOL}; note that ${\tt Inv}~f$ applies the Axiom of Choice to yield an
inverse of~$f$.  They also include natural deduction rules for the image
and range operators, and for the predicates {\tt inj} and {\tt inj_onto}.
Reasoning about function composition (the operator~\sdx{o}) and the
predicate~\cdx{surj} is done simply by expanding the definitions.  See
the file {\tt HOL/fun.ML} for a complete listing of the derived rules.

Figure~\ref{hol-subset} presents lattice properties of the subset relation.
Unions form least upper bounds; non-empty intersections form greatest lower
bounds.  Reasoning directly about subsets often yields clearer proofs than
reasoning about the membership relation.  See the file {\tt HOL/subset.ML}.

Figure~\ref{hol-equalities} presents many common set equalities.  They
include commutative, associative and distributive laws involving unions,
intersections and complements.  The proofs are mostly trivial, using the
classical reasoner; see file {\tt HOL/equalities.ML}.


\section{Generic packages}
\HOL\ instantiates most of Isabelle's generic packages;
see {\tt HOL/ROOT.ML} for details.

\subsection{Substitution and simplification}

Because it includes a general substitution rule, \HOL\ instantiates the
tactic {\tt hyp_subst_tac}, which substitutes for an equality throughout a
subgoal and its hypotheses.

It instantiates the simplifier, defining~\ttindexbold{HOL_ss} as the
simplification set for higher-order logic.  Equality~($=$), which also
expresses logical equivalence, may be used for rewriting.  See the file {\tt
HOL/simpdata.ML} for a complete listing of the simplification rules.

See \iflabelundefined{chap:classical}{the {\em Reference Manual\/}}%
{Chaps.\ts\ref{substitution} and~\ref{simp-chap}} for details of substitution
and simplification.

\begin{warn}\index{simplification!of conjunctions}
  The simplifier is not set up to reduce, for example, \verb$a = b & ...a...$
  to \verb$a = b & ...b...$: it does not use the left part of a conjunction
  while simplifying the right part. This can be changed by including
  \ttindex{conj_cong} in a simpset: \verb$addcongs [conj_cong]$. It can slow
  down rewriting and is therefore not included by default.
\end{warn}

In case a rewrite rule cannot be dealt with by the simplifier (either because
of nontermination or because its left-hand side is too flexible), HOL
provides {\tt stac}:
\begin{ttdescription}
\item[\ttindexbold{stac} $thm$ $i,$] where $thm$ is of the form $lhs = rhs$,
  replaces in subgoal $i$ instances of $lhs$ by corresponding instances of
  $rhs$. In case of multiple instances of $lhs$ in subgoal $i$, backtracking
  may be necessary to select the desired ones.
\end{ttdescription}


\subsection{Classical reasoning}

\HOL\ derives classical introduction rules for $\disj$ and~$\exists$, as
well as classical elimination rules for~$\imp$ and~$\bimp$, and the swap
rule; recall Fig.\ts\ref{hol-lemmas2} above.

The classical reasoner is set up as the structure
{\tt Classical}.  This structure is open, so {\ML} identifiers such
as {\tt step_tac}, {\tt fast_tac}, {\tt best_tac}, etc., refer to it.
\HOL\ defines the following classical rule sets:
\begin{ttbox} 
prop_cs    : claset
HOL_cs     : claset
set_cs     : claset
\end{ttbox}
\begin{ttdescription}
\item[\ttindexbold{prop_cs}] contains the propositional rules, namely
those for~$\top$, $\bot$, $\conj$, $\disj$, $\neg$, $\imp$ and~$\bimp$,
along with the rule~{\tt refl}.

\item[\ttindexbold{HOL_cs}] extends {\tt prop_cs} with the safe rules
  {\tt allI} and~{\tt exE} and the unsafe rules {\tt allE}
  and~{\tt exI}, as well as rules for unique existence.  Search using
  this classical set is incomplete: quantified formulae are used at most
  once.

\item[\ttindexbold{set_cs}] extends {\tt HOL_cs} with rules for the bounded
  quantifiers, subsets, comprehensions, unions and intersections,
  complements, finite sets, images and ranges.
\end{ttdescription}
\noindent
See \iflabelundefined{chap:classical}{the {\em Reference Manual\/}}%
        {Chap.\ts\ref{chap:classical}} 
for more discussion of classical proof methods.


\section{Types}\label{sec:HOL:Types}
This section describes HOL's basic predefined types (\verb$*$, \verb$+$, {\tt
  nat} and {\tt list}) and ways for introducing new types. The most important
type construction, the {\tt datatype}, is treated separately in
\S\ref{sec:HOL:datatype}.

\subsection{Product and sum types}\index{*"* type}\index{*"+ type}

\begin{figure}
\begin{constants}
  \it symbol    & \it meta-type &           & \it description \\ 
  \cdx{Pair}    & $[\alpha,\beta]\To \alpha\times\beta$
        & & ordered pairs $(a,b)$ \\
  \cdx{fst}     & $\alpha\times\beta \To \alpha$        & & first projection\\
  \cdx{snd}     & $\alpha\times\beta \To \beta$         & & second projection\\
  \cdx{split}   & $[[\alpha,\beta]\To\gamma, \alpha\times\beta] \To \gamma$ 
        & & generalized projection\\
  \cdx{Sigma}  & 
        $[\alpha\,set, \alpha\To\beta\,set]\To(\alpha\times\beta)set$ &
        & general sum of sets
\end{constants}
\begin{ttbox}\makeatletter
%\tdx{fst_def}      fst p     == @a. ? b. p = (a,b)
%\tdx{snd_def}      snd p     == @b. ? a. p = (a,b)
%\tdx{split_def}    split c p == c (fst p) (snd p)
\tdx{Sigma_def}    Sigma A B == UN x:A. UN y:B x. \{(x,y)\}

\tdx{Pair_inject}  [| (a, b) = (a',b');  [| a=a';  b=b' |] ==> R |] ==> R
\tdx{PairE}        [| !!x y. p = (x,y) ==> Q |] ==> Q
\tdx{Pair_eq}      ((a,b) = (a',b')) = (a=a' & b=b')

\tdx{fst_conv}     fst (a,b) = a
\tdx{snd_conv}     snd (a,b) = b
\tdx{surjective_pairing}  p = (fst p,snd p)

\tdx{split}        split c (a,b) = c a b
\tdx{expand_split} R(split c p) = (! x y. p = (x,y) --> R(c x y))

\tdx{SigmaI}       [| a:A;  b:B a |] ==> (a,b) : Sigma A B
\tdx{SigmaE}       [| c: Sigma A B;  
                !!x y.[| x:A; y:B x; c=(x,y) |] ==> P |] ==> P
\end{ttbox}
\caption{Type $\alpha\times\beta$}\label{hol-prod}
\end{figure} 

Theory \thydx{Prod} (Fig.\ts\ref{hol-prod}) defines the product type
$\alpha\times\beta$, with the ordered pair syntax {\tt($a$,$b$)}. Tuples are
simulated by pairs nested to the right:
\begin{center}
\begin{tabular}{|c|c|}
\hline
external & internal \\
\hline
$\tau@1 * \dots * \tau@n$ & $\tau@1 * (\dots (\tau@{n-1} * \tau@n)\dots)$ \\
\hline
$(t@1,\dots,t@n)$ & $(t@1,(\dots,(t@{n-1},t@n)\dots)$ \\
\hline
\end{tabular}
\end{center}
In addition, it is possible to use tuples
as patterns in abstractions:
\begin{center}
{\tt\%($x$,$y$).$t$} \quad stands for\quad {\tt split(\%$x$ $y$.$t$)} 
\end{center}
Nested patterns are possible and are translated stepwise:
{\tt\%($x$,$y$,$z$).$t$} $\leadsto$ {\tt\%($x$,($y$,$z$)).$t$} $\leadsto$
{\tt split(\%$x$.\%($y$,$z$).$t$)} $\leadsto$ {\tt split(\%$x$.split(\%$y$
  $z$.$t$))}. The reverse translation is performed upon printing.
\begin{warn}
  The translation between patterns and {\tt split} is performed automatically
  by the parser and printer.  Thus the internal and external form of a term
  may differ, whichs affects proofs.  For example the term {\tt
    (\%(x,y).(y,x))(a,b)} requires the theorem {\tt split} to rewrite to
  {\tt(b,a)}.
\end{warn}
In addition to explicit $\lambda$-abstractions, patterns can be used in any
variable binding construct which is internally described by a
$\lambda$-abstraction. Some important examples are
\begin{description}
\item[Let:] {\tt let {\it pattern} = $t$ in $u$}
\item[Quantifiers:] {\tt !~{\it pattern}:$A$.~$P$}
\item[Choice:] {\underscoreon \tt @~{\it pattern}~.~$P$}
\item[Set operations:] {\tt UN~{\it pattern}:$A$.~$B$}
\item[Sets:] {\tt \{~{\it pattern}~.~$P$~\}}
\end{description}

There is a simple tactic which supports reasoning about patterns:
\begin{ttdescription}
\item[\ttindexbold{split_all_tac} $i$] replaces in subgoal $i$ all
  {\tt!!}-quantified variables of product type by individual variables for
  each component. A simple example:
\begin{ttbox}
{\out 1. !!p. (\%(x,y,z). (x, y, z)) p = p}
by(split_all_tac 1);
{\out 1. !!x xa ya. (\%(x,y,z). (x, y, z)) (x, xa, ya) = (x, xa, ya)}
\end{ttbox}
\end{ttdescription}

Theory {\tt Prod} also introduces the degenerate product type {\tt unit}
which contains only a single element named {\tt()} with the property
\begin{ttbox}
\tdx{unit_eq}       u = ()
\end{ttbox}
\bigskip

Theory \thydx{Sum} (Fig.~\ref{hol-sum}) defines the sum type $\alpha+\beta$
which associates to the right and has a lower priority than $*$: $\tau@1 +
\tau@2 + \tau@3*\tau@4$ means $\tau@1 + (\tau@2 + (\tau@3*\tau@4))$.

The definition of products and sums in terms of existing types is not shown.
The constructions are fairly standard and can be found in the respective {\tt
  thy}-files.

\begin{figure}
\begin{constants}
  \it symbol    & \it meta-type &           & \it description \\ 
  \cdx{Inl}     & $\alpha \To \alpha+\beta$    & & first injection\\
  \cdx{Inr}     & $\beta \To \alpha+\beta$     & & second injection\\
  \cdx{sum_case} & $[\alpha\To\gamma, \beta\To\gamma, \alpha+\beta] \To\gamma$
        & & conditional
\end{constants}
\begin{ttbox}\makeatletter
%\tdx{sum_case_def}   sum_case == (\%f g p. @z. (!x. p=Inl x --> z=f x) &
%                                        (!y. p=Inr y --> z=g y))
%
\tdx{Inl_not_Inr}    ~ Inl a=Inr b

\tdx{inj_Inl}        inj Inl
\tdx{inj_Inr}        inj Inr

\tdx{sumE}           [| !!x. P(Inl x);  !!y. P(Inr y) |] ==> P s

\tdx{sum_case_Inl}   sum_case f g (Inl x) = f x
\tdx{sum_case_Inr}   sum_case f g (Inr x) = g x

\tdx{surjective_sum} sum_case (\%x. f(Inl x)) (\%y. f(Inr y)) s = f s
\tdx{expand_sum_case} R(sum_case f g s) = ((! x. s = Inl(x) --> R(f(x))) &
                                     (! y. s = Inr(y) --> R(g(y))))
\end{ttbox}
\caption{Type $\alpha+\beta$}\label{hol-sum}
\end{figure}

\begin{figure}
\index{*"< symbol}
\index{*"* symbol}
\index{*div symbol}
\index{*mod symbol}
\index{*"+ symbol}
\index{*"- symbol}
\begin{constants}
  \it symbol    & \it meta-type & \it priority & \it description \\ 
  \cdx{0}       & $nat$         & & zero \\
  \cdx{Suc}     & $nat \To nat$ & & successor function\\
  \cdx{nat_case} & $[\alpha, nat\To\alpha, nat] \To\alpha$
        & & conditional\\
  \cdx{nat_rec} & $[nat, \alpha, [nat, \alpha]\To\alpha] \To \alpha$
        & & primitive recursor\\
  \tt *         & $[nat,nat]\To nat$    &  Left 70      & multiplication \\
  \tt div       & $[nat,nat]\To nat$    &  Left 70      & division\\
  \tt mod       & $[nat,nat]\To nat$    &  Left 70      & modulus\\
  \tt +         & $[nat,nat]\To nat$    &  Left 65      & addition\\
  \tt -         & $[nat,nat]\To nat$    &  Left 65      & subtraction
\end{constants}
\subcaption{Constants and infixes}

\begin{ttbox}\makeatother
\tdx{nat_induct}     [| P 0; !!k. [| P k |] ==> P(Suc k) |]  ==> P n

\tdx{Suc_not_Zero}   Suc m ~= 0
\tdx{inj_Suc}        inj Suc
\tdx{n_not_Suc_n}    n~=Suc n
\subcaption{Basic properties}
\end{ttbox}
\caption{Defining {\tt nat}, the type of natural numbers} \label{hol-nat1}
\end{figure}


\begin{figure}
\begin{ttbox}\makeatother
\tdx{nat_case_0}     nat_case a f 0 = a
\tdx{nat_case_Suc}   nat_case a f (Suc k) = f k

\tdx{nat_rec_0}      nat_rec 0 c h = c
\tdx{nat_rec_Suc}    nat_rec (Suc n) c h = h n (nat_rec n c h)

\tdx{add_0}        0+n           = n
\tdx{add_Suc}      (Suc m)+n     = Suc(m+n)
\tdx{diff_0}       m-0           = m
\tdx{diff_0_eq_0}  0-n           = n
\tdx{diff_Suc_Suc} Suc(m)-Suc(n) = m-n
\tdx{mult_def}     0*n           = 0
\tdx{mult_Suc}     Suc(m)*n      = n + m*n

\tdx{mod_less}     m<n ==> m mod n = m
\tdx{mod_geq}      [| 0<n;  ~m<n |] ==> m mod n = (m-n) mod n
\tdx{div_less}     m<n ==> m div n = 0
\tdx{div_geq}      [| 0<n;  ~m<n |] ==> m div n = Suc((m-n) div n)
\subcaption{Recursion equations}

\tdx{less_trans}     [| i<j;  j<k |] ==> i<k
\tdx{lessI}          n < Suc n
\tdx{zero_less_Suc}  0 < Suc n

\tdx{less_not_sym}   n<m --> ~ m<n 
\tdx{less_not_refl}  ~ n<n
\tdx{not_less0}      ~ n<0

\tdx{Suc_less_eq}    (Suc m < Suc n) = (m<n)
\tdx{less_induct}    [| !!n. [| ! m. m<n --> P m |] ==> P n |]  ==>  P n

\tdx{less_linear}    m<n | m=n | n<m
\subcaption{The less-than relation}
\end{ttbox}
\caption{Derived rules for {\tt nat}} \label{hol-nat2}
\end{figure}

\subsection{The type of natural numbers, {\tt nat}}
%FIXME: introduce separate type proto_nat

The theory \thydx{Nat} defines the natural numbers in a roundabout but
traditional way.  The axiom of infinity postulates an type~\tydx{ind} of
individuals, which is non-empty and closed under an injective operation.
The natural numbers are inductively generated by choosing an arbitrary
individual for~0 and using the injective operation to take successors.  As
usual, the isomorphisms between~\tydx{nat} and its representation are made
explicitly. For details see the file {\tt Nat.thy}.

%The definition makes use of a least fixed point operator \cdx{lfp},
%defined using the Knaster-Tarski theorem.  This is used to define the
%operator \cdx{trancl}, for taking the transitive closure of a relation.
%Primitive recursion makes use of \cdx{wfrec}, an operator for recursion
%along arbitrary well-founded relations.  The corresponding theories are
%called {\tt Lfp}, {\tt Trancl} and {\tt WF}\@.  Elsewhere I have described
%similar constructions in the context of set theory~\cite{paulson-set-II}.

Type~\tydx{nat} is postulated to belong to class~\cldx{ord}, which
overloads $<$ and $\leq$ on the natural numbers.

Theory \thydx{Arith} develops arithmetic on the natural numbers.  It
defines addition, multiplication, subtraction, division, and remainder.
Many of their properties are proved: commutative, associative and
distributive laws, identity and cancellation laws, etc.  The most
interesting result is perhaps the theorem $a \bmod b + (a/b)\times b = a$.
Division and remainder are defined by repeated subtraction, which requires
well-founded rather than primitive recursion.  See Figs.\ts\ref{hol-nat1}
and~\ref{hol-nat2}.

%The predecessor relation, \cdx{pred_nat}, is shown to be well-founded.
%Recursion along this relation resembles primitive recursion, but is
%stronger because we are in higher-order logic; using primitive recursion to
%define a higher-order function, we can easily Ackermann's function, which
%is not primitive recursive \cite[page~104]{thompson91}.
%The transitive closure of \cdx{pred_nat} is~$<$.  Many functions on the
%natural numbers are most easily expressed using recursion along~$<$.

The tactic {\tt\ttindex{nat_ind_tac} "$n$" $i$} performs induction over the
variable~$n$ in subgoal~$i$.

%FIXME add nth
\begin{figure}
\index{#@{\tt[]} symbol}
\index{#@{\tt\#} symbol}
\index{"@@{\tt\at} symbol}
\begin{constants}
  \it symbol & \it meta-type & \it priority & \it description \\
  \tt[]    & $\alpha list$ & & empty list\\
  \tt \#   & $[\alpha,\alpha list]\To \alpha list$ & Right 65 & 
        list constructor \\
  \cdx{null}    & $\alpha list \To bool$ & & emptiness test\\
  \cdx{hd}      & $\alpha list \To \alpha$ & & head \\
  \cdx{tl}      & $\alpha list \To \alpha list$ & & tail \\
  \cdx{ttl}     & $\alpha list \To \alpha list$ & & total tail \\
  \tt\at  & $[\alpha list,\alpha list]\To \alpha list$ & Left 65 & append \\
  \cdx{map}     & $(\alpha\To\beta) \To (\alpha list \To \beta list)$
        & & mapping functional\\
  \cdx{filter}  & $(\alpha \To bool) \To (\alpha list \To \alpha list)$
        & & filter functional\\
  \cdx{list_all}& $(\alpha \To bool) \To (\alpha list \To bool)$
        & & forall functional\\
  \sdx{mem}  & $[\alpha,\alpha list]\To bool$    &  Left 55   & membership\\
  \cdx{length}  & $\alpha list \To nat$ & & length \\
%  \cdx{nth}  & $nat \To \alpha list \To \alpha$ & & indexing \\
  \cdx{foldl}   & $(\beta\To\alpha\To\beta) \To \beta \To \alpha list \To \beta$ &
  & iteration \\
  \cdx{flat}   & $(\alpha list) list\To \alpha list$ & & flattening \\
  \cdx{rev}     & $\alpha list \To \alpha list$ & & reverse \\
\end{constants}
\subcaption{Constants and infixes}

\begin{center} \tt\frenchspacing
\begin{tabular}{rrr} 
  \it external        & \it internal  & \it description \\{}
  [$x@1$, $\dots$, $x@n$]  &  $x@1$ \# $\cdots$ \# $x@n$ \# [] &
        \rm finite list \\{}
  [$x$:$l$. $P$]  & filter ($\lambda x{.}P$) $l$ & 
        \rm list comprehension
\end{tabular}
\end{center}
\subcaption{Translations}
\caption{The theory \thydx{List}} \label{hol-list}
\end{figure}


\begin{figure}
\begin{ttbox}\makeatother
\tdx{null_Nil}        null [] = True
\tdx{null_Cons}       null (x#xs) = False

\tdx{hd_Cons}         hd (x#xs) = x
\tdx{tl_Cons}         tl (x#xs) = xs

\tdx{ttl_Nil}         ttl [] = []
\tdx{ttl_Cons}        ttl (x#xs) = xs

\tdx{append_Nil}      [] @ ys = ys
\tdx{append_Cons}     (x#xs) @ ys = x # xs @ ys

\tdx{map_Nil}         map f [] = []
\tdx{map_Cons}        map f (x#xs) = f x # map f xs

\tdx{filter_Nil}      filter P [] = []
\tdx{filter_Cons}     filter P (x#xs) = (if P x then x#filter P xs else filter P xs)

\tdx{list_all_Nil}    list_all P [] = True
\tdx{list_all_Cons}   list_all P (x#xs) = (P x & list_all P xs)

\tdx{mem_Nil}         x mem [] = False
\tdx{mem_Cons}        x mem (y#ys) = (if y=x then True else x mem ys)

\tdx{length_Nil}      length([]) = 0
\tdx{length_Cons}     length(x#xs) = Suc(length(xs))

\tdx{foldl_Nil}       foldl f a [] = a
\tdx{foldl_Cons}      foldl f a (x#xs) = foldl f (f a x) xs

\tdx{flat_Nil}        flat([]) = []
\tdx{flat_Cons}       flat(x#xs) = x @ flat(xs)

\tdx{rev_Nil}         rev([]) = []
\tdx{rev_Cons}        rev(x#xs) = rev(xs) @ [x]
\end{ttbox}
\caption{Rewrite rules for lists} \label{fig:HOL:list-simps}
\end{figure}


\subsection{The type constructor for lists, {\tt list}}
\index{*list type}

Figure~\ref{hol-list} presents the theory \thydx{List}: the basic list
operations with their types and syntax. The type constructor {\tt list} is
defined as a {\tt datatype} with the constructors {\tt[]} and {\tt\#}.  This
yields an induction tactic {\tt list.induct_tac} and a list of freeness
theorems {\tt list.simps}.
A \sdx{case} construct of the form
\begin{center}\tt
case $e$ of [] => $a$  |  x\#xs => b
\end{center}
is defined by translation. For details see~\S\ref{sec:HOL:datatype}.

{\tt List} provides a basic library of list processing functions defined by
primitive recursion (see~\S\ref{sec:HOL:primrec}). The recursion equations
are shown in Fig.\ts\ref{fig:HOL:list-simps}.


\subsection{Introducing new types}

The \HOL-methodology dictates that all extension to a theory should be
conservative and thus preserve consistency. There are two basic type
extension mechanisms which meet this criterion: {\em type synonyms\/} and
{\em type definitions\/}. The former are inherited from {\tt Pure} and are
described elsewhere.
\begin{warn}
  Types in \HOL\ must be non-empty; otherwise the quantifier rules would be
  unsound, because $\exists x. x=x$ is a theorem \cite[\S7]{paulson-COLOG}.
\end{warn}
A \bfindex{type definition} identifies the new type with a subset of an
existing type. More precisely, the new type is defined by exhibiting an
existing type~$\tau$, a set~$A::(\tau)set$, and a theorem of the form $x:A$.
Thus~$A$ is a non-empty subset of~$\tau$, and the new type denotes this
subset.  New functions are generated to establish an isomorphism between the
new type and the subset.  If type~$\tau$ involves type variables $\alpha@1$,
\ldots, $\alpha@n$, then the type definition creates a type constructor
$(\alpha@1,\ldots,\alpha@n)ty$ rather than a particular type.

\begin{figure}[htbp]
\begin{rail}
typedef  : 'typedef' ( () | '(' tname ')') type '=' set witness;
type    : typevarlist name ( () | '(' infix ')' );
tname   : name;
set     : string;
witness : () | '(' id ')';
\end{rail}
\caption{Syntax of type definition}
\label{fig:HOL:typedef}
\end{figure}

The syntax for type definitions is shown in Fig.~\ref{fig:HOL:typedef}. For
the definition of ``typevarlist'' and ``infix'' see
\iflabelundefined{chap:classical}
{the appendix of the {\em Reference Manual\/}}%
{Appendix~\ref{app:TheorySyntax}}. The remaining nonterminals have the
following meaning:
\begin{description}
\item[\it type]: the new type constructor $(\alpha@1,\dots,\alpha@n)ty$ with
  optional infix annotation.
\item[\it tname]: an alphanumeric name $T$ for the type constructor $ty$, in
  case $ty$ is a symbolic name. Default: $ty$.
\item[\it set]: the representing subset $A$.
\item[\it witness]: name of a theorem of the form $a:A$ proving
  non-emptiness. Can be omitted in case Isabelle manages to prove
  non-emptiness automatically.
\end{description}
If all context conditions are met (no duplicate type variables in
'typevarlist', no extra type variables in 'set', and no free term variables
in 'set'), the following components are added to the theory:
\begin{itemize}
\item a type $ty :: (term,\dots)term$;
\item constants
\begin{eqnarray*}
T &::& (\tau)set \\
Rep_T &::& (\alpha@1,\dots,\alpha@n)ty \To \tau \\
Abs_T &::& \tau \To (\alpha@1,\dots,\alpha@n)ty
\end{eqnarray*}
\item a definition and three axioms
\[
\begin{array}{ll}
T{\tt_def} & T \equiv A \\
{\tt Rep_}T & Rep_T(x) : T \\
{\tt Rep_}T{\tt_inverse} & Abs_T(Rep_T(x)) = x \\
{\tt Abs_}T{\tt_inverse} & y:T \Imp Rep_T(Abs_T(y)) = y
\end{array}
\]
stating that $(\alpha@1,\dots,\alpha@n)ty$ is isomorphic to $A$ by $Rep_T$
and its inverse $Abs_T$.
\end{itemize}
Here are two simple examples where emptiness is proved automatically:
\begin{ttbox}
typedef unit = "\{False\}"

typedef (prod)
  ('a, 'b) "*"    (infixr 20)
      = "\{f . EX (a::'a) (b::'b). f = (\%x y. x = a & y = b)\}"
\end{ttbox}

Type definitions permit the introduction of abstract data types in a safe
way, namely by providing models based on already existing types. Given some
abstract axiomatic description $P$ of a type, this involves two steps:
\begin{enumerate}
\item Find an appropriate type $\tau$ and subset $A$ which has the desired
  properties $P$, and make the above type definition based on this
  representation.
\item Prove that $P$ holds for $ty$ by lifting $P$ from the representation.
\end{enumerate}
You can now forget about the representation and work solely in terms of the
abstract properties $P$.

\begin{warn}
If you introduce a new type (constructor) $ty$ axiomatically, i.e.\ by
declaring the type and its operations and by stating the desired axioms, you
should make sure the type has a non-empty model. You must also have a clause
\par
\begin{ttbox}
arities \(ty\): (term,\(\dots\),term)term
\end{ttbox}
in your theory file to tell Isabelle that elements of type $ty$ are in class
{\tt term}, the class of all HOL terms.
\end{warn}

\section{Datatype declarations}
\label{sec:HOL:datatype}
\index{*datatype|(}

\underscoreon

It is often necessary to extend a theory with \ML-like datatypes.  This
extension consists of the new type, declarations of its constructors and
rules that describe the new type. The theory definition section {\tt
  datatype} represents a compact way of doing this.


\subsection{Foundations}

A datatype declaration has the following general structure:
\[ \mbox{\tt datatype}~ (\alpha_1,\dots,\alpha_n)t ~=~
      C_1~\tau_{11}~\dots~\tau_{1k_1} ~\mid~ \dots ~\mid~
      C_m~\tau_{m1}~\dots~\tau_{mk_m} 
\]
where $\alpha_i$ are type variables, $C_i$ are distinct constructor names and
$\tau_{ij}$ are one of the following:
\begin{itemize}
\item type variables $\alpha_1,\dots,\alpha_n$,
\item types $(\beta_1,\dots,\beta_l)s$ where $s$ is a previously declared
  type or type synonym and $\{\beta_1,\dots,\beta_l\} \subseteq
  \{\alpha_1,\dots,\alpha_n\}$,
\item the newly defined type $(\alpha_1,\dots,\alpha_n)t$ \footnote{This
    makes it a recursive type. To ensure that the new type is not empty at
    least one constructor must consist of only non-recursive type
    components.}
\end{itemize}
If you would like one of the $\tau_{ij}$ to be a complex type expression
$\tau$ you need to declare a new type synonym $syn = \tau$ first and use
$syn$ in place of $\tau$. Of course this does not work if $\tau$ mentions the
recursive type itself, thus ruling out problematic cases like \[ \mbox{\tt
  datatype}~ t ~=~ C(t \To t) \] together with unproblematic ones like \[
\mbox{\tt datatype}~ t ~=~ C(t~list). \]

The constructors are automatically defined as functions of their respective
type:
\[ C_j : [\tau_{j1},\dots,\tau_{jk_j}] \To (\alpha_1,\dots,\alpha_n)t \]
These functions have certain {\em freeness} properties:
\begin{description}
\item[\tt distinct] They are distinct:
\[ C_i~x_1~\dots~x_{k_i} \neq C_j~y_1~\dots~y_{k_j} \qquad
   \mbox{for all}~ i \neq j.
\]
\item[\tt inject] They are injective:
\[ (C_j~x_1~\dots~x_{k_j} = C_j~y_1~\dots~y_{k_j}) =
   (x_1 = y_1 \land \dots \land x_{k_j} = y_{k_j})
\]
\end{description}
Because the number of inequalities is quadratic in the number of
constructors, a different method is used if their number exceeds
a certain value, currently 4. In that case every constructor is mapped to a
natural number
\[
\begin{array}{lcl}
\mbox{\it t\_ord}(C_1~x_1~\dots~x_{k_1}) & = & 0 \\
& \vdots & \\
\mbox{\it t\_ord}(C_m x_1~\dots~x_{k_m}) & = & m-1
\end{array}
\]
and distinctness of constructors is expressed by:
\[
\mbox{\it t\_ord}~x \neq \mbox{\it t\_ord}~y \Imp x \neq y.
\]
In addition a structural induction axiom {\tt induct} is provided: 
\[
\infer{P x}
{\begin{array}{lcl}
\Forall x_1\dots x_{k_1}.
  \List{P~x_{r_{11}}; \dots; P~x_{r_{1l_1}}} &
  \Imp  & P(C_1~x_1~\dots~x_{k_1}) \\
 & \vdots & \\
\Forall x_1\dots x_{k_m}.
  \List{P~x_{r_{m1}}; \dots; P~x_{r_{ml_m}}} &
  \Imp & P(C_m~x_1~\dots~x_{k_m})
\end{array}}
\]
where $\{r_{j1},\dots,r_{jl_j}\} = \{i \in \{1,\dots k_j\} ~\mid~ \tau_{ji}
= (\alpha_1,\dots,\alpha_n)t \}$, i.e.\ the property $P$ can be assumed for
all arguments of the recursive type.

The type also comes with an \ML-like \sdx{case}-construct:
\[
\begin{array}{rrcl}
\mbox{\tt case}~e~\mbox{\tt of} & C_1~x_{11}~\dots~x_{1k_1} & \To & e_1 \\
                           \vdots \\
                           \mid & C_m~x_{m1}~\dots~x_{mk_m} & \To & e_m
\end{array}
\]
In contrast to \ML, {\em all} constructors must be present, their order is
fixed, and nested patterns are not supported.


\subsection{Defining datatypes}

A datatype is defined in a theory definition file using the keyword {\tt
  datatype}. The definition following {\tt datatype} must conform to the
syntax of {\em typedecl} specified in Fig.~\ref{datatype-grammar} and must
obey the rules in the previous section. As a result the theory is extended
with the new type, the constructors, and the theorems listed in the previous
section.

\begin{figure}
\begin{rail}
typedecl : typevarlist id '=' (cons + '|')
         ;
cons     : name (typ *) ( () | mixfix )
         ;
typ      : id | tid | ('(' typevarlist id ')')
         ;
\end{rail}
\caption{Syntax of datatype declarations}
\label{datatype-grammar}
\end{figure}

Reading the theory file produces a structure which, in addition to the usual
components, contains a structure named $t$ for each datatype $t$ defined in
the file.\footnote{Otherwise multiple datatypes in the same theory file would
  lead to name clashes.} Each structure $t$ contains the following elements:
\begin{ttbox}
val distinct : thm list
val inject : thm list
val induct : thm
val cases : thm list
val simps : thm list
val induct_tac : string -> int -> tactic
\end{ttbox}
{\tt distinct}, {\tt inject} and {\tt induct} contain the theorems described
above. For convenience {\tt distinct} contains inequalities in both
directions.
\begin{warn}
  If there are five or more constructors, the {\em t\_ord} scheme is used for
  {\tt distinct}.  In this case the theory {\tt Arith} must be contained
  in the current theory, if necessary by including it explicitly.
\end{warn}
The reduction rules of the {\tt case}-construct are in {\tt cases}.  All
theorems from {\tt distinct}, {\tt inject} and {\tt cases} are combined in
{\tt simps} for use with the simplifier. The tactic {\verb$induct_tac$~{\em
    var i}\/} applies structural induction over variable {\em var} to
subgoal {\em i}.


\subsection{Examples}

\subsubsection{The datatype $\alpha~list$}

We want to define the type $\alpha~list$.\footnote{Of course there is a list
  type in HOL already. This is only an example.} To do this we have to build
a new theory that contains the type definition. We start from {\tt HOL}.
\begin{ttbox}
MyList = HOL +
  datatype 'a list = Nil | Cons 'a ('a list)
end
\end{ttbox}
After loading the theory (\verb$use_thy "MyList"$), we can prove
$Cons~x~xs\neq xs$.  First we build a suitable simpset for the simplifier:
\begin{ttbox}
val mylist_ss = HOL_ss addsimps MyList.list.simps;
goal MyList.thy "!x. Cons x xs ~= xs";
{\out Level 0}
{\out ! x. Cons x xs ~= xs}
{\out  1. ! x. Cons x xs ~= xs}
\end{ttbox}
This can be proved by the structural induction tactic:
\begin{ttbox}
by (MyList.list.induct_tac "xs" 1);
{\out Level 1}
{\out ! x. Cons x xs ~= xs}
{\out  1. ! x. Cons x Nil ~= Nil}
{\out  2. !!a list.}
{\out        ! x. Cons x list ~= list ==>}
{\out        ! x. Cons x (Cons a list) ~= Cons a list}
\end{ttbox}
The first subgoal can be proved with the simplifier and the distinctness
axioms which are part of \verb$mylist_ss$.
\begin{ttbox}
by (simp_tac mylist_ss 1);
{\out Level 2}
{\out ! x. Cons x xs ~= xs}
{\out  1. !!a list.}
{\out        ! x. Cons x list ~= list ==>}
{\out        ! x. Cons x (Cons a list) ~= Cons a list}
\end{ttbox}
Using the freeness axioms we can quickly prove the remaining goal.
\begin{ttbox}
by (asm_simp_tac mylist_ss 1);
{\out Level 3}
{\out ! x. Cons x xs ~= xs}
{\out No subgoals!}
\end{ttbox}
Because both subgoals were proved by almost the same tactic we could have
done that in one step using
\begin{ttbox}
by (ALLGOALS (asm_simp_tac mylist_ss));
\end{ttbox}


\subsubsection{The datatype $\alpha~list$ with mixfix syntax}

In this example we define the type $\alpha~list$ again but this time we want
to write {\tt []} instead of {\tt Nil} and we want to use the infix operator
\verb|#| instead of {\tt Cons}. To do this we simply add mixfix annotations
after the constructor declarations as follows:
\begin{ttbox}
MyList = HOL +
  datatype 'a list = "[]" ("[]") 
                   | "#" 'a ('a list) (infixr 70)
end
\end{ttbox}
Now the theorem in the previous example can be written \verb|x#xs ~= xs|. The
proof is the same.


\subsubsection{A datatype for weekdays}

This example shows a datatype that consists of more than four constructors:
\begin{ttbox}
Days = Arith +
  datatype days = Mo | Tu | We | Th | Fr | Sa | So
end
\end{ttbox}
Because there are more than four constructors, the theory must be based on
{\tt Arith}. Inequality is defined via a function \verb|days_ord|. Although
the expression \verb|Mo ~= Tu| is not directly contained in {\tt distinct},
it can be proved by the simplifier if \verb$arith_ss$ is used:
\begin{ttbox}
val days_ss = arith_ss addsimps Days.days.simps;

goal Days.thy "Mo ~= Tu";
by (simp_tac days_ss 1);
\end{ttbox}
Note that usually it is not necessary to derive these inequalities explicitly
because the simplifier will dispose of them automatically.

\subsection{Primitive recursive functions}
\label{sec:HOL:primrec}
\index{primitive recursion|(}
\index{*primrec|(}

Datatypes come with a uniform way of defining functions, {\bf primitive
  recursion}. Although it is possible to define primitive recursive functions
by asserting their reduction rules as new axioms, e.g.\
\begin{ttbox}
Append = MyList +
consts app :: ['a list,'a list] => 'a list
rules 
   app_Nil   "app [] ys = ys"
   app_Cons  "app (x#xs) ys = x#app xs ys"
end
\end{ttbox}
this carries with it the danger of accidentally asserting an inconsistency,
as in \verb$app [] ys = us$. Therefore primitive recursive functions on
datatypes can be defined with a special syntax:
\begin{ttbox}
Append = MyList +
consts app :: ['a list,'a list] => 'a list
primrec app MyList.list
   app_Nil   "app [] ys = ys"
   app_Cons  "app (x#xs) ys = x#app xs ys"
end
\end{ttbox}
The system will now check that the two rules \verb$app_Nil$ and
\verb$app_Cons$ do indeed form a primitive recursive definition, thus
ensuring that consistency is maintained. For example
\begin{ttbox}
primrec app MyList.list
    app_Nil   "app [] ys = us"
\end{ttbox}
is rejected:
\begin{ttbox}
Extra variables on rhs
\end{ttbox}
\bigskip

The general form of a primitive recursive definition is
\begin{ttbox}
primrec {\it function} {\it type}
    {\it reduction rules}
\end{ttbox}
where
\begin{itemize}
\item {\it function} is the name of the function, either as an {\it id} or a
  {\it string}. The function must already have been declared.
\item {\it type} is the name of the datatype, either as an {\it id} or in the
  long form {\it Thy.t}, where {\it Thy} is the name of the parent theory the
  datatype was declared in, and $t$ the name of the datatype. The long form
  is required if the {\tt datatype} and the {\tt primrec} sections are in
  different theories.
\item {\it reduction rules} specify one or more named equations of the form
  {\it id\/}~{\it string}, where the identifier gives the name of the rule in
  the result structure, and {\it string} is a reduction rule of the form \[
  f~x_1~\dots~x_m~(C~y_1~\dots~y_k)~z_1~\dots~z_n = r \] such that $C$ is a
  constructor of the datatype, $r$ contains only the free variables on the
  left-hand side, and all recursive calls in $r$ are of the form
  $f~\dots~y_i~\dots$ for some $i$. There must be exactly one reduction
  rule for each constructor. Since these reduction rules are mainly used via
  the implicit simpset, their identifiers may also be omitted.
\end{itemize}
A theory file may contain any number of {\tt primrec} sections which may be
intermixed with other declarations.

For the consistency-sensitive user it may be reassuring to know that {\tt
  primrec} does not assert the reduction rules as new axioms but derives them
as theorems from an explicit definition of the recursive function in terms of
a recursion operator on the datatype.

The primitive recursive function can also use infix or mixfix syntax:
\begin{ttbox}
Append = MyList +
consts "@"  :: ['a list,'a list] => 'a list  (infixr 60)
primrec "op @" MyList.list
   app_Nil   "[] @ ys = ys"
   app_Cons  "(x#xs) @ ys = x#(xs @ ys)"
end
\end{ttbox}

The reduction rules become part of the ML structure \verb$Append$ and can
be used to prove theorems about the function:
\begin{ttbox}
val append_ss = HOL_ss addsimps [Append.app_Nil,Append.app_Cons];

goal Append.thy "(xs @ ys) @ zs = xs @ (ys @ zs)";
by (MyList.list.induct_tac "xs" 1);
by (ALLGOALS(asm_simp_tac append_ss));
\end{ttbox}

%Note that underdefined primitive recursive functions are allowed:
%\begin{ttbox}
%Tl = MyList +
%consts tl  :: 'a list => 'a list
%primrec tl MyList.list
%   tl_Cons "tl(x#xs) = xs"
%end
%\end{ttbox}
%Nevertheless {\tt tl} is total, although we do not know what the result of
%\verb$tl([])$ is.

\index{primitive recursion|)}
\index{*primrec|)}
\index{*datatype|)}


\section{Inductive and coinductive definitions}
\index{*inductive|(}
\index{*coinductive|(}

An {\bf inductive definition} specifies the least set closed under given
rules.  For example, a structural operational semantics is an inductive
definition of an evaluation relation.  Dually, a {\bf coinductive
  definition} specifies the greatest set closed under given rules.  An
important example is using bisimulation relations to formalize equivalence
of processes and infinite data structures.

A theory file may contain any number of inductive and coinductive
definitions.  They may be intermixed with other declarations; in
particular, the (co)inductive sets {\bf must} be declared separately as
constants, and may have mixfix syntax or be subject to syntax translations.

Each (co)inductive definition adds definitions to the theory and also
proves some theorems.  Each definition creates an ML structure, which is a
substructure of the main theory structure.

This package is derived from the ZF one, described in a
separate paper,\footnote{It appeared in CADE~\cite{paulson-CADE} and a
  longer version is distributed with Isabelle.} which you should refer to
in case of difficulties.  The package is simpler than ZF's, thanks to HOL's
automatic type-checking.  The type of the (co)inductive determines the
domain of the fixedpoint definition, and the package does not use inference
rules for type-checking.


\subsection{The result structure}
Many of the result structure's components have been discussed in the paper;
others are self-explanatory.
\begin{description}
\item[\tt thy] is the new theory containing the recursive sets.

\item[\tt defs] is the list of definitions of the recursive sets.

\item[\tt mono] is a monotonicity theorem for the fixedpoint operator.

\item[\tt unfold] is a fixedpoint equation for the recursive set (the union of
the recursive sets, in the case of mutual recursion).

\item[\tt intrs] is the list of introduction rules, now proved as theorems, for
the recursive sets.  The rules are also available individually, using the
names given them in the theory file. 

\item[\tt elim] is the elimination rule.

\item[\tt mk\_cases] is a function to create simplified instances of {\tt
elim}, using freeness reasoning on some underlying datatype.
\end{description}

For an inductive definition, the result structure contains two induction rules,
{\tt induct} and \verb|mutual_induct|.  For a coinductive definition, it
contains the rule \verb|coinduct|.

Figure~\ref{def-result-fig} summarizes the two result signatures,
specifying the types of all these components.

\begin{figure}
\begin{ttbox}
sig
val thy          : theory
val defs         : thm list
val mono         : thm
val unfold       : thm
val intrs        : thm list
val elim         : thm
val mk_cases     : thm list -> string -> thm
{\it(Inductive definitions only)} 
val induct       : thm
val mutual_induct: thm
{\it(Coinductive definitions only)}
val coinduct    : thm
end
\end{ttbox}
\hrule
\caption{The result of a (co)inductive definition} \label{def-result-fig}
\end{figure}

\subsection{The syntax of a (co)inductive definition}
An inductive definition has the form
\begin{ttbox}
inductive    {\it inductive sets}
  intrs      {\it introduction rules}
  monos      {\it monotonicity theorems}
  con_defs   {\it constructor definitions}
\end{ttbox}
A coinductive definition is identical, except that it starts with the keyword
{\tt coinductive}.  

The {\tt monos} and {\tt con\_defs} sections are optional.  If present,
each is specified as a string, which must be a valid ML expression of type
{\tt thm list}.  It is simply inserted into the {\tt .thy.ML} file; if it
is ill-formed, it will trigger ML error messages.  You can then inspect the
file on your directory.

\begin{itemize}
\item The {\it inductive sets} are specified by one or more strings.

\item The {\it introduction rules} specify one or more introduction rules in
  the form {\it ident\/}~{\it string}, where the identifier gives the name of
  the rule in the result structure.

\item The {\it monotonicity theorems} are required for each operator
  applied to a recursive set in the introduction rules.  There {\bf must}
  be a theorem of the form $A\subseteq B\Imp M(A)\subseteq M(B)$, for each
  premise $t\in M(R_i)$ in an introduction rule!

\item The {\it constructor definitions} contain definitions of constants
  appearing in the introduction rules.  In most cases it can be omitted.
\end{itemize}

The package has a few notable restrictions:
\begin{itemize}
\item The theory must separately declare the recursive sets as
  constants.

\item The names of the recursive sets must be identifiers, not infix
operators.  

\item Side-conditions must not be conjunctions.  However, an introduction rule
may contain any number of side-conditions.

\item Side-conditions of the form $x=t$, where the variable~$x$ does not
  occur in~$t$, will be substituted through the rule \verb|mutual_induct|.
\end{itemize}


\subsection{Example of an inductive definition}
Two declarations, included in a theory file, define the finite powerset
operator.  First we declare the constant~{\tt Fin}.  Then we declare it
inductively, with two introduction rules:
\begin{ttbox}
consts Fin :: 'a set => 'a set set
inductive "Fin A"
  intrs
    emptyI  "\{\} : Fin A"
    insertI "[| a: A;  b: Fin A |] ==> insert a b : Fin A"
\end{ttbox}
The resulting theory structure contains a substructure, called~{\tt Fin}.
It contains the {\tt Fin}$~A$ introduction rules as the list {\tt Fin.intrs},
and also individually as {\tt Fin.emptyI} and {\tt Fin.consI}.  The induction
rule is {\tt Fin.induct}.

For another example, here is a theory file defining the accessible part of a
relation.  The main thing to note is the use of~{\tt Pow} in the sole
introduction rule, and the corresponding mention of the rule
\verb|Pow_mono| in the {\tt monos} list.  The paper discusses a ZF version
of this example in more detail.
\begin{ttbox}
Acc = WF + 
consts pred :: "['b, ('a * 'b)set] => 'a set"   (*Set of predecessors*)
       acc  :: "('a * 'a)set => 'a set"         (*Accessible part*)
defs   pred_def  "pred x r == {y. (y,x):r}"
inductive "acc r"
  intrs
     pred "pred a r: Pow(acc r) ==> a: acc r"
  monos   "[Pow_mono]"
end
\end{ttbox}
The HOL distribution contains many other inductive definitions, such as the
theory {\tt HOL/ex/PropLog.thy} and the directory {\tt HOL/IMP}.  The
theory {\tt HOL/ex/LList.thy} contains coinductive definitions.

\index{*coinductive|)} \index{*inductive|)} \underscoreoff


\section{The examples directories}
Directory {\tt HOL/Subst} contains Martin Coen's mechanisation of a theory of
substitutions and unifiers.  It is based on Paulson's previous
mechanisation in {\LCF}~\cite{paulson85} of Manna and Waldinger's
theory~\cite{mw81}. 

Directory {\tt HOL/IMP} contains a formalization of the denotational,
operational and axiomatic semantics of a simple while-language, including an
equivalence proof between denotational and operational semantics and a
soundness and part of a completeness proof of the Hoare rules w.r.t.\ the
denotational semantics. The whole development is taken from
Winskel~\cite{winskel93}.  In addition, a verification-condition-generator is
proved sound and complete w.r.t. the Hoare rules.

Directory {\tt HOL/Hoare} contains a user friendly surface syntax for Hoare
logic, including a tactic for generating verification-conditions.

Directory {\tt HOL/ex} contains other examples and experimental proofs in
{\HOL}.  Here is an overview of the more interesting files.
\begin{itemize}
\item File {\tt cla.ML} demonstrates the classical reasoner on over sixty
  predicate calculus theorems, ranging from simple tautologies to
  moderately difficult problems involving equality and quantifiers.

\item File {\tt meson.ML} contains an experimental implementation of the {\sc
    meson} proof procedure, inspired by Plaisted~\cite{plaisted90}.  It is
  much more powerful than Isabelle's classical reasoner.  But it is less
  useful in practice because it works only for pure logic; it does not
  accept derived rules for the set theory primitives, for example.

\item File {\tt mesontest.ML} contains test data for the {\sc meson} proof
  procedure.  These are mostly taken from Pelletier \cite{pelletier86}.

\item File {\tt set.ML} proves Cantor's Theorem, which is presented in
  \S\ref{sec:hol-cantor} below, and the Schr\"oder-Bernstein Theorem.

\item The definition of lazy lists demonstrates methods for handling
  infinite data structures and coinduction in higher-order
  logic~\cite{paulson-coind}.  Theory \thydx{LList} defines an operator for
  corecursion on lazy lists, which is used to define a few simple functions
  such as map and append.  Corecursion cannot easily define operations such
  as filter, which can compute indefinitely before yielding the next
  element (if any!) of the lazy list.  A coinduction principle is defined
  for proving equations on lazy lists.

\item Theory {\tt PropLog} proves the soundness and completeness of
  classical propositional logic, given a truth table semantics.  The only
  connective is $\imp$.  A Hilbert-style axiom system is specified, and its
  set of theorems defined inductively.  A similar proof in \ZF{} is
  described elsewhere~\cite{paulson-set-II}.

\item Theory {\tt Term} develops an experimental recursive type definition;
  the recursion goes through the type constructor~\tydx{list}.

\item Theory {\tt Simult} constructs mutually recursive sets of trees and
  forests, including induction and recursion rules.

\item Theory {\tt MT} contains Jacob Frost's formalization~\cite{frost93} of
  Milner and Tofte's coinduction example~\cite{milner-coind}.  This
  substantial proof concerns the soundness of a type system for a simple
  functional language.  The semantics of recursion is given by a cyclic
  environment, which makes a coinductive argument appropriate.
\end{itemize}


\goodbreak
\section{Example: Cantor's Theorem}\label{sec:hol-cantor}
Cantor's Theorem states that every set has more subsets than it has
elements.  It has become a favourite example in higher-order logic since
it is so easily expressed:
\[  \forall f::[\alpha,\alpha]\To bool. \exists S::\alpha\To bool.
    \forall x::\alpha. f~x \not= S 
\] 
%
Viewing types as sets, $\alpha\To bool$ represents the powerset
of~$\alpha$.  This version states that for every function from $\alpha$ to
its powerset, some subset is outside its range.  

The Isabelle proof uses \HOL's set theory, with the type $\alpha\,set$ and
the operator \cdx{range}.  The set~$S$ is given as an unknown instead of a
quantified variable so that we may inspect the subset found by the proof.
\begin{ttbox}
goal Set.thy "~ ?S : range(f :: 'a=>'a set)";
{\out Level 0}
{\out ~ ?S : range f}
{\out  1. ~ ?S : range f}
\end{ttbox}
The first two steps are routine.  The rule \tdx{rangeE} replaces
$\Var{S}\in {\tt range} f$ by $\Var{S}=f~x$ for some~$x$.
\begin{ttbox}
by (resolve_tac [notI] 1);
{\out Level 1}
{\out ~ ?S : range f}
{\out  1. ?S : range f ==> False}
\ttbreak
by (eresolve_tac [rangeE] 1);
{\out Level 2}
{\out ~ ?S : range f}
{\out  1. !!x. ?S = f x ==> False}
\end{ttbox}
Next, we apply \tdx{equalityCE}, reasoning that since $\Var{S}=f~x$,
we have $\Var{c}\in \Var{S}$ if and only if $\Var{c}\in f~x$ for
any~$\Var{c}$.
\begin{ttbox}
by (eresolve_tac [equalityCE] 1);
{\out Level 3}
{\out ~ ?S : range f}
{\out  1. !!x. [| ?c3 x : ?S; ?c3 x : f x |] ==> False}
{\out  2. !!x. [| ~ ?c3 x : ?S; ~ ?c3 x : f x |] ==> False}
\end{ttbox}
Now we use a bit of creativity.  Suppose that~$\Var{S}$ has the form of a
comprehension.  Then $\Var{c}\in\{x.\Var{P}~x\}$ implies
$\Var{P}~\Var{c}$.   Destruct-resolution using \tdx{CollectD}
instantiates~$\Var{S}$ and creates the new assumption.
\begin{ttbox}
by (dresolve_tac [CollectD] 1);
{\out Level 4}
{\out ~ \{x. ?P7 x\} : range f}
{\out  1. !!x. [| ?c3 x : f x; ?P7(?c3 x) |] ==> False}
{\out  2. !!x. [| ~ ?c3 x : \{x. ?P7 x\}; ~ ?c3 x : f x |] ==> False}
\end{ttbox}
Forcing a contradiction between the two assumptions of subgoal~1 completes
the instantiation of~$S$.  It is now the set $\{x. x\not\in f~x\}$, which
is the standard diagonal construction.
\begin{ttbox}
by (contr_tac 1);
{\out Level 5}
{\out ~ \{x. ~ x : f x\} : range f}
{\out  1. !!x. [| ~ x : \{x. ~ x : f x\}; ~ x : f x |] ==> False}
\end{ttbox}
The rest should be easy.  To apply \tdx{CollectI} to the negated
assumption, we employ \ttindex{swap_res_tac}:
\begin{ttbox}
by (swap_res_tac [CollectI] 1);
{\out Level 6}
{\out ~ \{x. ~ x : f x\} : range f}
{\out  1. !!x. [| ~ x : f x; ~ False |] ==> ~ x : f x}
\ttbreak
by (assume_tac 1);
{\out Level 7}
{\out ~ \{x. ~ x : f x\} : range f}
{\out No subgoals!}
\end{ttbox}
How much creativity is required?  As it happens, Isabelle can prove this
theorem automatically.  The classical set \ttindex{set_cs} contains rules
for most of the constructs of \HOL's set theory.  We must augment it with
\tdx{equalityCE} to break up set equalities, and then apply best-first
search.  Depth-first search would diverge, but best-first search
successfully navigates through the large search space.
\index{search!best-first}
\begin{ttbox}
choplev 0;
{\out Level 0}
{\out ~ ?S : range f}
{\out  1. ~ ?S : range f}
\ttbreak
by (best_tac (set_cs addSEs [equalityCE]) 1);
{\out Level 1}
{\out ~ \{x. ~ x : f x\} : range f}
{\out No subgoals!}
\end{ttbox}

\index{higher-order logic|)}
