%% THIS FILE IS COMMON TO ALL LOGIC MANUALS

\chapter{Syntax definitions}
The syntax of each logic is presented using a context-free grammar.
These grammars obey the following conventions:
\begin{itemize}
\item identifiers denote nonterminal symbols
\item \texttt{typewriter} font denotes terminal symbols
\item parentheses $(\ldots)$ express grouping
\item constructs followed by a Kleene star, such as $id^*$ and $(\ldots)^*$
can be repeated~0 or more times 
\item alternatives are separated by a vertical bar,~$|$
\item the symbol for alphanumeric identifiers is~{\it id\/} 
\item the symbol for scheme variables is~{\it var}
\end{itemize}
To reduce the number of nonterminals and grammar rules required, Isabelle's
syntax module employs {\bf priorities},\index{priorities} or precedences.
Each grammar rule is given by a mixfix declaration, which has a priority,
and each argument place has a priority.  This general approach handles
infix operators that associate either to the left or to the right, as well
as prefix and binding operators.

In a syntactically valid expression, an operator's arguments never involve
an operator of lower priority unless brackets are used.  Consider
first-order logic, where $\exists$ has lower priority than $\disj$,
which has lower priority than $\conj$.  There, $P\conj Q \disj R$
abbreviates $(P\conj Q) \disj R$ rather than $P\conj (Q\disj R)$.  Also,
$\exists x.P\disj Q$ abbreviates $\exists x.(P\disj Q)$ rather than
$(\exists x.P)\disj Q$.  Note especially that $P\disj(\exists x.Q)$
becomes syntactically invalid if the brackets are removed.

A {\bf binder} is a symbol associated with a constant of type
$(\sigma\To\tau)\To\tau'$.  For instance, we may declare~$\forall$ as a binder
for the constant~$All$, which has type $(\alpha\To o)\To o$.  This defines the
syntax $\forall x.t$ to mean $All(\lambda x.t)$.  We can also write $\forall
x@1\ldots x@m.t$ to abbreviate $\forall x@1.  \ldots \forall x@m.t$; this is
possible for any constant provided that $\tau$ and $\tau'$ are the same type.
The Hilbert description operator $\varepsilon x.P\,x$ has type $(\alpha\To
bool)\To\alpha$ and normally binds only one variable.  
ZF's bounded quantifier $\forall x\in A.P(x)$ cannot be declared as a
binder because it has type $[i, i\To o]\To o$.  The syntax for binders allows
type constraints on bound variables, as in
\[ \forall (x{::}\alpha) \; (y{::}\beta) \; z{::}\gamma. Q(x,y,z) \]

To avoid excess detail, the logic descriptions adopt a semi-formal style.
Infix operators and binding operators are listed in separate tables, which
include their priorities.  Grammar descriptions do not include numeric
priorities; instead, the rules appear in order of decreasing priority.
This should suffice for most purposes; for full details, please consult the
actual syntax definitions in the {\tt.thy} files.

Each nonterminal symbol is associated with some Isabelle type.  For
example, the formulae of first-order logic have type~$o$.  Every
Isabelle expression of type~$o$ is therefore a formula.  These include
atomic formulae such as $P$, where $P$ is a variable of type~$o$, and more
generally expressions such as $P(t,u)$, where $P$, $t$ and~$u$ have
suitable types.  Therefore, `expression of type~$o$' is listed as a
separate possibility in the grammar for formulae.


