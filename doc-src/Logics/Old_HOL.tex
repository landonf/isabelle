%% $Id$
\chapter{Higher-order logic}
The directory~\ttindexbold{HOL} contains a theory for higher-order logic
based on Gordon's~{\sc hol} system~\cite{gordon88a}, which itself is based on
Church~\cite{church40}.  Andrews~\cite{andrews86} is a full description of
higher-order logic.  Gordon's work has demonstrated that higher-order logic
is useful for hardware verification; beyond this, it is widely applicable
in many areas of mathematics.  It is weaker than {\ZF} set theory but for
most applications this does not matter.  If you prefer {\ML} to Lisp, you
will probably prefer {\HOL} to~{\ZF}.

Previous releases of Isabelle included a completely different version
of~{\HOL}, with explicit type inference rules~\cite{paulson-COLOG}.  This
version no longer exists, but \ttindex{ZF} supports a similar style of
reasoning.

{\HOL} has a distinct feel, compared with {\ZF} and {\CTT}.  It
identifies object-level types with meta-level types, taking advantage of
Isabelle's built-in type checker.  It identifies object-level functions
with meta-level functions, so it uses Isabelle's operations for abstraction
and application.  There is no ``apply'' operator: function applications are
written as simply~$f(a)$ rather than $f{\tt`}a$.

These identifications allow Isabelle to support {\HOL} particularly nicely,
but they also mean that {\HOL} requires more sophistication from the user
--- in particular, an understanding of Isabelle's type system.  Beginners
should gain experience by working in first-order logic, before attempting
to use higher-order logic.  This chapter assumes familiarity with~{\FOL{}}.


\begin{figure} 
\begin{center}
\begin{tabular}{rrr} 
  \it name    	&\it meta-type 	& \it description \\ 
  \idx{Trueprop}& $bool\To prop$		& coercion to $prop$\\
  \idx{not}	& $bool\To bool$		& negation ($\neg$) \\
  \idx{True}	& $bool$			& tautology ($\top$) \\
  \idx{False}	& $bool$			& absurdity ($\bot$) \\
  \idx{if}	& $[bool,\alpha,\alpha]\To\alpha::term$	& conditional \\
  \idx{Inv}	& $(\alpha\To\beta)\To(\beta\To\alpha)$ & function inversion
\end{tabular}
\end{center}
\subcaption{Constants}

\index{"@@{\tt\at}}
\begin{center}
\begin{tabular}{llrrr} 
  \it symbol &\it name	   &\it meta-type & \it prec & \it description \\
  \tt\at   & \idx{Eps}  & $(\alpha\To bool)\To\alpha::term$ & 10 & 
	Hilbert description ($\epsilon$) \\
  \idx{!}  & \idx{All}  & $(\alpha::term\To bool)\To bool$ & 10 & 
	universal quantifier ($\forall$) \\
  \idx{?}  & \idx{Ex}   & $(\alpha::term\To bool)\To bool$ & 10 & 
	existential quantifier ($\exists$) \\
  \idx{?!} & \idx{Ex1}  & $(\alpha::term\To bool)\To bool$ & 10 & 
	unique existence ($\exists!$)
\end{tabular}
\end{center}
\subcaption{Binders} 

\begin{center}
\begin{tabular}{llrrr} 
  \it symbol &\it name	   &\it meta-type & \it prec & \it description \\
  \idx{ALL}  & \idx{All}  & $(\alpha::term\To bool)\To bool$ & 10 & 
	universal quantifier ($\forall$) \\
  \idx{EX}   & \idx{Ex}   & $(\alpha::term\To bool)\To bool$ & 10 & 
	existential quantifier ($\exists$) \\
  \idx{EX!}  & \idx{Ex1}  & $(\alpha::term\To bool)\To bool$ & 10 & 
	unique existence ($\exists!$)
\end{tabular}
\end{center}
\subcaption{Alternative quantifiers} 

\begin{center}
\indexbold{*"=}
\indexbold{&@{\tt\&}}
\indexbold{*"|}
\indexbold{*"-"-">}
\begin{tabular}{rrrr} 
  \it symbol  	& \it meta-type & \it precedence & \it description \\ 
  \idx{o}	& $[\beta\To\gamma,\alpha\To\beta]\To (\alpha\To\gamma)$ & 
	Right 50 & composition ($\circ$) \\
  \tt =		& $[\alpha::term,\alpha]\To bool$ & Left 50 & equality ($=$) \\
  \tt \& 	& $[bool,bool]\To bool$	& Right 35 & conjunction ($\conj$) \\
  \tt |		& $[bool,bool]\To bool$	& Right 30 & disjunction ($\disj$) \\
  \tt --> 	& $[bool,bool]\To bool$	& Right 25 & implication ($\imp$) \\
  \tt <		& $[\alpha::ord,\alpha]\To bool$ & Left 50 & less than ($<$) \\
  \tt <=	& $[\alpha::ord,\alpha]\To bool$ & Left 50 & 
		less than or equals ($\leq$)
\end{tabular}
\end{center}
\subcaption{Infixes}
\caption{Syntax of {\tt HOL}} \label{hol-constants}
\end{figure}


\begin{figure} 
\dquotes
\[\begin{array}{rcl}
    term & = & \hbox{expression of class~$term$} \\
	 & | & "\at~~" id~id^* " . " formula \\[2ex]
 formula & = & \hbox{expression of type~$bool$} \\
	 & | & term " = " term \\
	 & | & term " \ttilde= " term \\
	 & | & term " < " term \\
	 & | & term " <= " term \\
	 & | & "\ttilde\ " formula \\
	 & | & formula " \& " formula \\
	 & | & formula " | " formula \\
	 & | & formula " --> " formula \\
	 & | & "!~~~" id~id^* " . " formula \\
	 & | & "?~~~" id~id^* " . " formula \\
	 & | & "?!~~" id~id^* " . " formula \\
	 & | & "ALL~" id~id^* " . " formula \\
	 & | & "EX~~" id~id^* " . " formula \\
	 & | & "EX!~" id~id^* " . " formula
  \end{array}
\]
\subcaption{Grammar}
\caption{Full grammar for {\HOL}} \label{hol-grammar}
\end{figure} 


\section{Syntax}
Type inference is automatic, exploiting Isabelle's type classes.  The class
of higher-order terms is called {\it term\/}; type variables range over
this class by default.  The equality symbol and quantifiers are polymorphic
over class {\it term}.  

Class {\it ord\/} consists of all ordered types; the relations $<$ and
$\leq$ are polymorphic over this class, as are the functions
\ttindex{mono}, \ttindex{min} and \ttindex{max}.  Three other
type classes --- {\it plus}, {\it minus\/} and {\it times} --- permit
overloading of the operators {\tt+}, {\tt-} and {\tt*}.  In particular,
{\tt-} is overloaded for set difference and subtraction.
\index{*"+}\index{-@{\tt-}}\index{*@{\tt*}}

Figure~\ref{hol-constants} lists the constants (including infixes and
binders), while Figure~\ref{hol-grammar} presents the grammar.  Note that
$a$\verb|~=|$b$ is translated to \verb|~(|$a$=$b$\verb|)|.

\subsection{Types}
The type of formulae, {\it bool} belongs to class {\it term}; thus,
formulae are terms.  The built-in type~$fun$, which constructs function
types, is overloaded such that $\sigma\To\tau$ belongs to class~$term$ if
$\sigma$ and~$\tau$ do; this allows quantification over functions.  Types
in {\HOL} must be non-empty because of the form of quantifier
rules~\cite[\S7]{paulson-COLOG}.

Gordon's {\sc hol} system supports {\bf type definitions}.  A type is
defined by exhibiting an existing type~$\sigma$, a predicate~$P::\sigma\To
bool$, and a theorem of the form $\exists x::\sigma.P(x)$.  Thus~$P$
specifies a non-empty subset of~$\sigma$, and the new type denotes this
subset.  New function constants are generated to establish an isomorphism
between the new type and the subset.  If type~$\sigma$ involves type
variables $\alpha@1$, \ldots, $\alpha@n$, then the type definition creates
a type constructor $(\alpha@1,\ldots,\alpha@n)ty$ rather than a particular
type.

Isabelle does not support type definitions at present.  Instead, they are
mimicked by explicit definitions of isomorphism functions.  These should be
accompanied by theorems of the form $\exists x::\sigma.P(x)$, but this is
not checked.


\subsection{Binders}
Hilbert's {\bf description} operator~$\epsilon x.P[x]$ stands for some~$a$
satisfying~$P[a]$, if such exists.  Since all terms in {\HOL} denote
something, a description is always meaningful, but we do not know its value
unless $P[x]$ defines it uniquely.  We may write descriptions as
\ttindexbold{Eps}($P$) or use the syntax
\hbox{\tt \at $x$.$P[x]$}.  Existential quantification is defined
by
\[ \exists x.P(x) \equiv P(\epsilon x.P(x)) \]
The unique existence quantifier, $\exists!x.P[x]$, is defined in terms
of~$\exists$ and~$\forall$.  An Isabelle binder, it admits nested
quantifications.  For instance, $\exists!x y.P(x,y)$ abbreviates
$\exists!x. \exists!y.P(x,y)$; note that this does not mean that there
exists a unique pair $(x,y)$ satisfying~$P(x,y)$.

\index{*"!}\index{*"?}
Quantifiers have two notations.  As in Gordon's {\sc hol} system, {\HOL}
uses~{\tt!}\ and~{\tt?}\ to stand for $\forall$ and $\exists$.  The
existential quantifier must be followed by a space; thus {\tt?x} is an
unknown, while \verb'? x.f(x)=y' is a quantification.  Isabelle's usual
notation for quantifiers, \ttindex{ALL} and \ttindex{EX}, is also
available.  Both notations are accepted for input.  The {\ML} reference
\ttindexbold{HOL_quantifiers} governs the output notation.  If set to {\tt
true}, then~{\tt!}\ and~{\tt?}\ are displayed; this is the default.  If set
to {\tt false}, then~{\tt ALL} and~{\tt EX} are displayed.

\begin{warn}
Although the description operator does not usually allow iteration of the
form \hbox{\tt \at $x@1 \dots x@n$.$P[x@1,\dots,x@n]$}, there are cases where
this is legal.  If \hbox{\tt \at $y$.$P[x,y]$} is of type~{\it bool},
then \hbox{\tt \at $x\,y$.$P[x,y]$} is legal.  The pretty printer will
display \hbox{\tt \at $x$.\at $y$.$P[x,y]$} as
\hbox{\tt \at $x\,y$.$P[x,y]$}.
\end{warn}

\begin{figure} \makeatother
\begin{ttbox}
\idx{refl}           t = t::'a
\idx{subst}          [| s=t; P(s) |] ==> P(t::'a)
\idx{ext}            (!!x::'a. f(x)::'b = g(x)) ==> (%x.f(x)) = (%x.g(x))
\idx{impI}           (P ==> Q) ==> P-->Q
\idx{mp}             [| P-->Q;  P |] ==> Q
\idx{iff}            (P-->Q) --> (Q-->P) --> (P=Q)
\idx{selectI}        P(x::'a) ==> P(@x.P(x))
\idx{True_or_False}  (P=True) | (P=False)
\subcaption{basic rules}

\idx{True_def}       True  = ((%x.x)=(%x.x))
\idx{All_def}        All   = (%P. P = (%x.True))
\idx{Ex_def}         Ex    = (%P. P(@x.P(x)))
\idx{False_def}      False = (!P.P)
\idx{not_def}        not   = (%P. P-->False)
\idx{and_def}        op &  = (%P Q. !R. (P-->Q-->R) --> R)
\idx{or_def}         op |  = (%P Q. !R. (P-->R) --> (Q-->R) --> R)
\idx{Ex1_def}        Ex1   = (%P. ? x. P(x) & (! y. P(y) --> y=x))
\subcaption{Definitions of the logical constants}

\idx{Inv_def}   Inv  = (%(f::'a=>'b) y. @x. f(x)=y)
\idx{o_def}     op o = (%(f::'b=>'c) g (x::'a). f(g(x)))
\idx{if_def}    if   = (%P x y.@z::'a.(P=True --> z=x) & (P=False --> z=y))
\subcaption{Further definitions}
\end{ttbox}
\caption{Rules of {\tt HOL}} \label{hol-rules}
\end{figure}


\begin{figure} \makeatother
\begin{ttbox}
\idx{sym}         s=t ==> t=s
\idx{trans}       [| r=s; s=t |] ==> r=t
\idx{ssubst}      [| t=s; P(s) |] ==> P(t::'a)
\idx{box_equals}  [| a=b;  a=c;  b=d |] ==> c=d  
\idx{arg_cong}    s=t ==> f(s)=f(t)
\idx{fun_cong}    s::'a=>'b = t ==> s(x)=t(x)
\subcaption{Equality}

\idx{TrueI}       True 
\idx{FalseE}      False ==> P

\idx{conjI}       [| P; Q |] ==> P&Q
\idx{conjunct1}   [| P&Q |] ==> P
\idx{conjunct2}   [| P&Q |] ==> Q 
\idx{conjE}       [| P&Q;  [| P; Q |] ==> R |] ==> R

\idx{disjI1}      P ==> P|Q
\idx{disjI2}      Q ==> P|Q
\idx{disjE}       [| P | Q; P ==> R; Q ==> R |] ==> R

\idx{notI}        (P ==> False) ==> ~ P
\idx{notE}        [| ~ P;  P |] ==> R
\idx{impE}        [| P-->Q;  P;  Q ==> R |] ==> R
\subcaption{Propositional logic}

\idx{iffI}        [| P ==> Q;  Q ==> P |] ==> P=Q
\idx{iffD1}       [| P=Q; P |] ==> Q
\idx{iffD2}       [| P=Q; Q |] ==> P
\idx{iffE}        [| P=Q; [| P --> Q; Q --> P |] ==> R |] ==> R

\idx{eqTrueI}     P ==> P=True 
\idx{eqTrueE}     P=True ==> P 
\subcaption{Logical equivalence}

\end{ttbox}
\caption{Derived rules for {\HOL}} \label{hol-lemmas1}
\end{figure}


\begin{figure} \makeatother
\begin{ttbox}
\idx{allI}      (!!x::'a. P(x)) ==> !x. P(x)
\idx{spec}      !x::'a.P(x) ==> P(x)
\idx{allE}      [| !x.P(x);  P(x) ==> R |] ==> R
\idx{all_dupE}  [| !x.P(x);  [| P(x); !x.P(x) |] ==> R |] ==> R

\idx{exI}       P(x) ==> ? x::'a.P(x)
\idx{exE}       [| ? x::'a.P(x); !!x. P(x) ==> Q |] ==> Q

\idx{ex1I}      [| P(a);  !!x. P(x) ==> x=a |] ==> ?! x. P(x)
\idx{ex1E}      [| ?! x.P(x);  !!x. [| P(x);  ! y. P(y) --> y=x |] ==> R 
          |] ==> R

\idx{select_equality}  [| P(a);  !!x. P(x) ==> x=a |] ==> (@x.P(x)) = a
\subcaption{Quantifiers and descriptions}

\idx{ccontr}             (~P ==> False) ==> P
\idx{classical}          (~P ==> P) ==> P
\idx{excluded_middle}    ~P | P

\idx{disjCI}    (~Q ==> P) ==> P|Q
\idx{exCI}      (! x. ~ P(x) ==> P(a)) ==> ? x.P(x)
\idx{impCE}     [| P-->Q; ~ P ==> R; Q ==> R |] ==> R
\idx{iffCE}     [| P<->Q;  [| P; Q |] ==> R;  [| ~P; ~Q |] ==> R |] ==> R
\idx{notnotD}   ~~P ==> P
\idx{swap}      ~P ==> (~Q ==> P) ==> Q
\subcaption{Classical logic}

\idx{if_True}    if(True,x,y) = x
\idx{if_False}   if(False,x,y) = y
\idx{if_P}       P ==> if(P,x,y) = x
\idx{if_not_P}   ~ P ==> if(P,x,y) = y
\idx{expand_if}  P(if(Q,x,y)) = ((Q --> P(x)) & (~Q --> P(y)))
\subcaption{Conditionals}
\end{ttbox}
\caption{More derived rules} \label{hol-lemmas2}
\end{figure}


\section{Rules of inference}
The basic theory has the {\ML} identifier \ttindexbold{HOL.thy}.  However,
many further theories are defined, introducing product and sum types, the
natural numbers, etc.

Figure~\ref{hol-rules} shows the inference rules with their~{\ML} names.
They follow standard practice in higher-order logic: only a few connectives
are taken as primitive, with the remainder defined obscurely.  

Unusually, the definitions use object-equality~({\tt=}) rather than
meta-equality~({\tt==}).  This is possible because equality in higher-order
logic may equate formulae and even functions over formulae.  On the other
hand, meta-equality is Isabelle's usual symbol for making definitions.
Take care to note which form of equality is used before attempting a proof.

Some of the rules mention type variables; for example, {\tt refl} mentions
the type variable~{\tt'a}.  This facilitates explicit instantiation of the
type variables.  By default, such variables range over class {\it term}.  

\begin{warn}
Where function types are involved, Isabelle's unification code does not
guarantee to find instantiations for type variables automatically.  If
resolution fails for no obvious reason, try setting \ttindex{show_types} to
{\tt true}, causing Isabelle to display types of terms.  (Possibly, set
\ttindex{show_sorts} to {\tt true} also, causing Isabelle to display sorts.)
Be prepared to use \ttindex{res_inst_tac} instead of {\tt resolve_tac}.
Setting \ttindex{Unify.trace_types} to {\tt true} causes Isabelle to
report omitted search paths during unification.
\end{warn}

Here are further comments on the rules:
\begin{description}
\item[\ttindexbold{ext}] 
expresses extensionality of functions.
\item[\ttindexbold{iff}] 
asserts that logically equivalent formulae are equal.
\item[\ttindexbold{selectI}] 
gives the defining property of the Hilbert $\epsilon$-operator.  The
derived rule \ttindexbold{select_equality} (see below) is often easier to use.
\item[\ttindexbold{True_or_False}] 
makes the logic classical.\footnote{In fact, the $\epsilon$-operator
already makes the logic classical, as shown by Diaconescu;
see Paulson~\cite{paulson-COLOG} for details.}
\end{description}

\begin{warn}
{\HOL} has no if-and-only-if connective; logical equivalence is expressed
using equality.  But equality has a high precedence, as befitting a
relation, while if-and-only-if typically has the lowest precedence.  Thus,
$\neg\neg P=P$ abbreviates $\neg\neg (P=P)$ and not $(\neg\neg P)=P$.  When
using $=$ to mean logical equivalence, enclose both operands in
parentheses.
\end{warn}

Some derived rules are shown in Figures~\ref{hol-lemmas1}
and~\ref{hol-lemmas2}, with their {\ML} names.  These include natural rules
for the logical connectives, as well as sequent-style elimination rules for
conjunctions, implications, and universal quantifiers.  

Note the equality rules: \ttindexbold{ssubst} performs substitution in
backward proofs, while \ttindexbold{box_equals} supports reasoning by
simplifying both sides of an equation.

See the files \ttindexbold{HOL/hol.thy} and
\ttindexbold{HOL/hol.ML} for complete listings of the rules and
derived rules.


\section{Generic packages}
{\HOL} instantiates most of Isabelle's generic packages;
see \ttindexbold{HOL/ROOT.ML} for details.
\begin{itemize}
\item 
Because it includes a general substitution rule, {\HOL} instantiates the
tactic \ttindex{hyp_subst_tac}, which substitutes for an equality
throughout a subgoal and its hypotheses.
\item 
It instantiates the simplifier, defining~\ttindexbold{HOL_ss} as the
simplification set for higher-order logic.  Equality~($=$), which also
expresses logical equivalence, may be used for rewriting.  See the file
\ttindexbold{HOL/simpdata.ML} for a complete listing of the simplification
rules. 
\item 
It instantiates the classical reasoning module.  See~\S\ref{hol-cla-prover}
for details. 
\end{itemize}


\begin{figure} 
\begin{center}
\begin{tabular}{rrr} 
  \it name    	&\it meta-type 	& \it description \\ 
\index{"{"}@{\tt\{\}}}
  {\tt\{\}}	& $\alpha\,set$ 	& the empty set \\
  \idx{insert}	& $[\alpha,\alpha\,set]\To \alpha\,set$
	& insertion of element \\
  \idx{Collect}	& $(\alpha\To bool)\To\alpha\,set$
	& comprehension \\
  \idx{Compl}	& $(\alpha\,set)\To\alpha\,set$
	& complement \\
  \idx{INTER} & $[\alpha\,set,\alpha\To\beta\,set]\To\beta\,set$
	& intersection over a set\\
  \idx{UNION} & $[\alpha\,set,\alpha\To\beta\,set]\To\beta\,set$
	& union over a set\\
  \idx{Inter} & $((\alpha\,set)set)\To\alpha\,set$
	&set of sets intersection \\
  \idx{Union} & $((\alpha\,set)set)\To\alpha\,set$
	&set of sets union \\
  \idx{range}	& $(\alpha\To\beta )\To\beta\,set$
	& range of a function \\[1ex]
  \idx{Ball}~~\idx{Bex}	& $[\alpha\,set,\alpha\To bool]\To bool$
	& bounded quantifiers \\
  \idx{mono} 	& $(\alpha\,set\To\beta\,set)\To bool$
	& monotonicity \\
  \idx{inj}~~\idx{surj}& $(\alpha\To\beta )\To bool$
	& injective/surjective \\
  \idx{inj_onto}	& $[\alpha\To\beta ,\alpha\,set]\To bool$
	& injective over subset
\end{tabular}
\end{center}
\subcaption{Constants}

\begin{center}
\begin{tabular}{llrrr} 
  \it symbol &\it name	   &\it meta-type & \it prec & \it description \\
  \idx{INT}  & \idx{INTER1}  & $(\alpha\To\beta\,set)\To\beta\,set$ & 10 & 
	intersection over a type\\
  \idx{UN}  & \idx{UNION1}  & $(\alpha\To\beta\,set)\To\beta\,set$ & 10 & 
	union over a type
\end{tabular}
\end{center}
\subcaption{Binders} 

\begin{center}
\indexbold{*"`"`}
\indexbold{*":}
\indexbold{*"<"=}
\begin{tabular}{rrrr} 
  \it symbol	& \it meta-type & \it precedence & \it description \\ 
  \tt ``	& $[\alpha\To\beta ,\alpha\,set]\To  (\beta\,set)$
	& Left 90 & image \\
  \idx{Int}	& $[\alpha\,set,\alpha\,set]\To\alpha\,set$
	& Left 70 & intersection ($\inter$) \\
  \idx{Un}	& $[\alpha\,set,\alpha\,set]\To\alpha\,set$
	& Left 65 & union ($\union$) \\
  \tt:		& $[\alpha ,\alpha\,set]\To bool$	
	& Left 50 & membership ($\in$) \\
  \tt <=	& $[\alpha\,set,\alpha\,set]\To bool$
	& Left 50 & subset ($\subseteq$) 
\end{tabular}
\end{center}
\subcaption{Infixes}
\caption{Syntax of {\HOL}'s set theory} \label{hol-set-syntax}
\end{figure} 


\begin{figure} 
\begin{center} \tt\frenchspacing
\begin{tabular}{rrr} 
  \it external		& \it internal	& \it description \\ 
  $a$ \ttilde: $b$      & \ttilde($a$ : $b$)    & \rm negated membership\\
  \{$a@1$, $\ldots$, $a@n$\}  &  insert($a@1$,$\cdots$,insert($a@n$,0)) &
        \rm finite set \\
  \{$x$.$P[x]$\}	&  Collect($\lambda x.P[x]$) &
        \rm comprehension \\
  \idx{INT} $x$:$A$.$B[x]$	& INTER($A$,$\lambda x.B[x]$) &
	\rm intersection over a set \\
  \idx{UN}  $x$:$A$.$B[x]$	& UNION($A$,$\lambda x.B[x]$) &
	\rm union over a set \\
  \idx{!} $x$:$A$.$P[x]$	& Ball($A$,$\lambda x.P[x]$) & 
	\rm bounded $\forall$ \\
  \idx{?} $x$:$A$.$P[x]$	& Bex($A$,$\lambda x.P[x]$) & 
	\rm bounded $\exists$ \\[1ex]
  \idx{ALL} $x$:$A$.$P[x]$	& Ball($A$,$\lambda x.P[x]$) & 
	\rm bounded $\forall$ \\
  \idx{EX} $x$:$A$.$P[x]$	& Bex($A$,$\lambda x.P[x]$) & 
	\rm bounded $\exists$
\end{tabular}
\end{center}
\subcaption{Translations}

\dquotes
\[\begin{array}{rcl}
    term & = & \hbox{other terms\ldots} \\
	 & | & "\{\}" \\
	 & | & "\{ " term\; ("," term)^* " \}" \\
	 & | & "\{ " id " . " formula " \}" \\
	 & | & term " `` " term \\
	 & | & term " Int " term \\
	 & | & term " Un " term \\
	 & | & "INT~~"  id ":" term " . " term \\
	 & | & "UN~~~"  id ":" term " . " term \\
	 & | & "INT~~"  id~id^* " . " term \\
	 & | & "UN~~~"  id~id^* " . " term \\[2ex]
 formula & = & \hbox{other formulae\ldots} \\
	 & | & term " : " term \\
	 & | & term " \ttilde: " term \\
	 & | & term " <= " term \\
	 & | & "!~~~" id ":" term " . " formula \\
	 & | & "?~~~" id ":" term " . " formula \\
	 & | & "ALL " id ":" term " . " formula \\
	 & | & "EX~~" id ":" term " . " formula
  \end{array}
\]
\subcaption{Full Grammar}
\caption{Syntax of {\HOL}'s set theory (continued)} \label{hol-set-syntax2}
\end{figure} 


\begin{figure} \makeatother
\begin{ttbox}
\idx{mem_Collect_eq}    (a : \{x.P(x)\}) = P(a)
\idx{Collect_mem_eq}    \{x.x:A\} = A
\subcaption{Isomorphisms between predicates and sets}

\idx{empty_def}         \{\}         == \{x.x= False\}
\idx{insert_def}        insert(a,B) == \{x.x=a\} Un B
\idx{Ball_def}          Ball(A,P)   == ! x. x:A --> P(x)
\idx{Bex_def}           Bex(A,P)    == ? x. x:A & P(x)
\idx{subset_def}        A <= B      == ! x:A. x:B
\idx{Un_def}            A Un B      == \{x.x:A | x:B\}
\idx{Int_def}           A Int B     == \{x.x:A & x:B\}
\idx{set_diff_def}      A - B       == \{x.x:A & x~:B\}
\idx{Compl_def}         Compl(A)    == \{x. ~ x:A\}
\idx{INTER_def}         INTER(A,B)  == \{y. ! x:A. y: B(x)\}
\idx{UNION_def}         UNION(A,B)  == \{y. ? x:A. y: B(x)\}
\idx{INTER1_def}        INTER1(B)   == INTER(\{x.True\}, B)
\idx{UNION1_def}        UNION1(B)   == UNION(\{x.True\}, B)
\idx{Inter_def}         Inter(S)    == (INT x:S. x)
\idx{Union_def}         Union(S)    ==  (UN x:S. x)
\idx{image_def}         f``A        == \{y. ? x:A. y=f(x)\}
\idx{range_def}         range(f)    == \{y. ? x. y=f(x)\}
\idx{mono_def}          mono(f)     == !A B. A <= B --> f(A) <= f(B)
\idx{inj_def}           inj(f)      == ! x y. f(x)=f(y) --> x=y
\idx{surj_def}          surj(f)     == ! y. ? x. y=f(x)
\idx{inj_onto_def}      inj_onto(f,A) == !x:A. !y:A. f(x)=f(y) --> x=y
\subcaption{Definitions}
\end{ttbox}
\caption{Rules of {\HOL}'s set theory} \label{hol-set-rules}
\end{figure}


\begin{figure} \makeatother
\begin{ttbox}
\idx{CollectI}      [| P(a) |] ==> a : \{x.P(x)\}
\idx{CollectD}      [| a : \{x.P(x)\} |] ==> P(a)
\idx{CollectE}      [| a : \{x.P(x)\};  P(a) ==> W |] ==> W
\idx{Collect_cong}  [| !!x. P(x)=Q(x) |] ==> \{x. P(x)\} = \{x. Q(x)\}
\subcaption{Comprehension}

\idx{ballI}         [| !!x. x:A ==> P(x) |] ==> ! x:A. P(x)
\idx{bspec}         [| ! x:A. P(x);  x:A |] ==> P(x)
\idx{ballE}         [| ! x:A. P(x);  P(x) ==> Q;  ~ x:A ==> Q |] ==> Q
\idx{ball_cong}     [| A=A';  !!x. x:A' ==> P(x) = P'(x) |] ==>
              (! x:A. P(x)) = (! x:A'. P'(x))

\idx{bexI}          [| P(x);  x:A |] ==> ? x:A. P(x)
\idx{bexCI}         [| ! x:A. ~ P(x) ==> P(a);  a:A |] ==> ? x:A.P(x)
\idx{bexE}          [| ? x:A. P(x);  !!x. [| x:A; P(x) |] ==> Q  |] ==> Q
\subcaption{Bounded quantifiers}

\idx{subsetI}         (!!x.x:A ==> x:B) ==> A <= B
\idx{subsetD}         [| A <= B;  c:A |] ==> c:B
\idx{subsetCE}        [| A <= B;  ~ (c:A) ==> P;  c:B ==> P |] ==> P

\idx{subset_refl}     A <= A
\idx{subset_antisym}  [| A <= B;  B <= A |] ==> A = B
\idx{subset_trans}    [| A<=B;  B<=C |] ==> A<=C

\idx{set_ext}         [| !!x. (x:A) = (x:B) |] ==> A = B
\idx{equalityD1}      A = B ==> A<=B
\idx{equalityD2}      A = B ==> B<=A
\idx{equalityE}       [| A = B;  [| A<=B; B<=A |] ==> P |]  ==>  P

\idx{equalityCE}      [| A = B;  [| c:A; c:B |] ==> P;  
                           [| ~ c:A; ~ c:B |] ==> P 
                |]  ==>  P
\subcaption{The subset and equality relations}
\end{ttbox}
\caption{Derived rules for set theory} \label{hol-set1}
\end{figure}


\begin{figure} \makeatother
\begin{ttbox}
\idx{emptyE}   a : \{\} ==> P

\idx{insertI1} a : insert(a,B)
\idx{insertI2} a : B ==> a : insert(b,B)
\idx{insertE}  [| a : insert(b,A);  a=b ==> P;  a:A ==> P |] ==> 

\idx{ComplI}   [| c:A ==> False |] ==> c : Compl(A)
\idx{ComplD}   [| c : Compl(A) |] ==> ~ c:A

\idx{UnI1}     c:A ==> c : A Un B
\idx{UnI2}     c:B ==> c : A Un B
\idx{UnCI}     (~c:B ==> c:A) ==> c : A Un B
\idx{UnE}      [| c : A Un B;  c:A ==> P;  c:B ==> P |] ==> P

\idx{IntI}     [| c:A;  c:B |] ==> c : A Int B
\idx{IntD1}    c : A Int B ==> c:A
\idx{IntD2}    c : A Int B ==> c:B
\idx{IntE}     [| c : A Int B;  [| c:A; c:B |] ==> P |] ==> P

\idx{UN_I}     [| a:A;  b: B(a) |] ==> b: (UN x:A. B(x))
\idx{UN_E}     [| b: (UN x:A. B(x));  !!x.[| x:A;  b:B(x) |] ==> R |] ==> R

\idx{INT_I}    (!!x. x:A ==> b: B(x)) ==> b : (INT x:A. B(x))
\idx{INT_D}    [| b: (INT x:A. B(x));  a:A |] ==> b: B(a)
\idx{INT_E}    [| b: (INT x:A. B(x));  b: B(a) ==> R;  ~ a:A ==> R |] ==> R

\idx{UnionI}   [| X:C;  A:X |] ==> A : Union(C)
\idx{UnionE}   [| A : Union(C);  !!X.[| A:X;  X:C |] ==> R |] ==> R

\idx{InterI}   [| !!X. X:C ==> A:X |] ==> A : Inter(C)
\idx{InterD}   [| A : Inter(C);  X:C |] ==> A:X
\idx{InterE}   [| A : Inter(C);  A:X ==> R;  ~ X:C ==> R |] ==> R
\end{ttbox}
\caption{Further derived rules for set theory} \label{hol-set2}
\end{figure}


\section{A formulation of set theory}
Historically, higher-order logic gives a foundation for Russell and
Whitehead's theory of classes.  Let us use modern terminology and call them
{\bf sets}, but note that these sets are distinct from those of {\ZF} set
theory, and behave more like {\ZF} classes.
\begin{itemize}
\item
Sets are given by predicates over some type~$\sigma$.  Types serve to
define universes for sets, but type checking is still significant.
\item
There is a universal set (for each type).  Thus, sets have complements, and
may be defined by absolute comprehension.
\item
Although sets may contain other sets as elements, the containing set must
have a more complex type.
\end{itemize}
Finite unions and intersections have the same behaviour in {\HOL} as they
do in~{\ZF}.  In {\HOL} the intersection of the empty set is well-defined,
denoting the universal set for the given type.

\subsection{Syntax of set theory}
The type $\alpha\,set$ is essentially the same as $\alpha\To bool$.  The new
type is defined for clarity and to avoid complications involving function
types in unification.  Since Isabelle does not support type definitions (as
discussed above), the isomorphisms between the two types are declared
explicitly.  Here they are natural: {\tt Collect} maps $\alpha\To bool$ to
$\alpha\,set$, while \hbox{\tt op :} maps in the other direction (ignoring
argument order).  

Figure~\ref{hol-set-syntax} lists the constants, infixes, and syntax
translations.  Figure~\ref{hol-set-syntax2} presents the grammar of the new
constructs.  Infix operators include union and intersection ($A\union B$
and $A\inter B$), the subset and membership relations, and the image
operator~{\tt``}.  Note that $a$\verb|~:|$b$ is translated to
\verb|~(|$a$:$b$\verb|)|.  The {\tt\{\ldots\}} notation abbreviates finite
sets constructed in the obvious manner using~{\tt insert} and~$\{\}$ (the
empty set):
\begin{eqnarray*}
 \{a,b,c\} & \equiv & {\tt insert}(a,{\tt insert}(b,{\tt insert}(c,\emptyset)))
\end{eqnarray*}

The set \hbox{\tt\{$x$.$P[x]$\}} consists of all $x$ (of suitable type)
that satisfy~$P[x]$, where $P[x]$ is a formula that may contain free
occurrences of~$x$.  This syntax expands to \ttindexbold{Collect}$(\lambda
x.P[x])$. 

The set theory defines two {\bf bounded quantifiers}:
\begin{eqnarray*}
   \forall x\in A.P[x] &\hbox{which abbreviates}& \forall x. x\in A\imp P[x] \\
   \exists x\in A.P[x] &\hbox{which abbreviates}& \exists x. x\in A\conj P[x]
\end{eqnarray*}
The constants~\ttindexbold{Ball} and~\ttindexbold{Bex} are defined
accordingly.  Instead of {\tt Ball($A$,$P$)} and {\tt Bex($A$,$P$)} we may
write\index{*"!}\index{*"?}\index{*ALL}\index{*EX}
\hbox{\tt !~$x$:$A$.$P[x]$} and \hbox{\tt ?~$x$:$A$.$P[x]$}. 
Isabelle's usual notation, \ttindex{ALL} and \ttindex{EX}, is also
available.  As with
ordinary quantifiers, the contents of \ttindexbold{HOL_quantifiers} specifies
which notation should be used for output.

Unions and intersections over sets, namely $\bigcup@{x\in A}B[x]$ and
$\bigcap@{x\in A}B[x]$, are written 
\ttindexbold{UN}~\hbox{\tt$x$:$A$.$B[x]$} and
\ttindexbold{INT}~\hbox{\tt$x$:$A$.$B[x]$}.  
Unions and intersections over types, namely $\bigcup@x B[x]$ and
$\bigcap@x B[x]$, are written 
\ttindexbold{UN}~\hbox{\tt$x$.$B[x]$} and
\ttindexbold{INT}~\hbox{\tt$x$.$B[x]$}; they are equivalent to the previous
union/intersection operators when $A$ is the universal set.
The set of set union and intersection operators ($\bigcup A$ and $\bigcap
A$) are not binders, but equals $\bigcup@{x\in A}x$ and $\bigcap@{x\in
  A}x$, respectively.

\subsection{Axioms and rules of set theory}
The axioms \ttindexbold{mem_Collect_eq} and
\ttindexbold{Collect_mem_eq} assert that the functions {\tt Collect} and
\hbox{\tt op :} are isomorphisms. 
All the other axioms are definitions; see Figure~\ref{hol-set-rules}.
These include straightforward properties of functions: image~({\tt``}) and
{\tt range}, and predicates concerning monotonicity, injectiveness, etc.

{\HOL}'s set theory has the {\ML} identifier \ttindexbold{Set.thy}.

\begin{figure} \makeatother
\begin{ttbox}
\idx{imageI}     [| x:A |] ==> f(x) : f``A
\idx{imageE}     [| b : f``A;  !!x.[| b=f(x);  x:A |] ==> P |] ==> P

\idx{rangeI}     f(x) : range(f)
\idx{rangeE}     [| b : range(f);  !!x.[| b=f(x) |] ==> P |] ==> P

\idx{monoI}      [| !!A B. A <= B ==> f(A) <= f(B) |] ==> mono(f)
\idx{monoD}      [| mono(f);  A <= B |] ==> f(A) <= f(B)

\idx{injI}       [| !! x y. f(x) = f(y) ==> x=y |] ==> inj(f)
\idx{inj_inverseI}              (!!x. g(f(x)) = x) ==> inj(f)
\idx{injD}       [| inj(f); f(x) = f(y) |] ==> x=y

\idx{Inv_f_f}    inj(f) ==> Inv(f,f(x)) = x
\idx{f_Inv_f}    y : range(f) ==> f(Inv(f,y)) = y

\idx{Inv_injective}
    [| Inv(f,x)=Inv(f,y); x: range(f);  y: range(f) |] ==> x=y

\idx{inj_ontoI}
    (!! x y. [| f(x) = f(y); x:A; y:A |] ==> x=y) ==> inj_onto(f,A)

\idx{inj_onto_inverseI}
    (!!x. x:A ==> g(f(x)) = x) ==> inj_onto(f,A)

\idx{inj_ontoD}
    [| inj_onto(f,A);  f(x)=f(y);  x:A;  y:A |] ==> x=y

\idx{inj_onto_contraD}
    [| inj_onto(f,A);  x~=y;  x:A;  y:A |] ==> ~ f(x)=f(y)
\end{ttbox}
\caption{Derived rules involving functions} \label{hol-fun}
\end{figure}


\begin{figure} \makeatother
\begin{ttbox}
\idx{Union_upper}     B:A ==> B <= Union(A)
\idx{Union_least}     [| !!X. X:A ==> X<=C |] ==> Union(A) <= C

\idx{Inter_lower}     B:A ==> Inter(A) <= B
\idx{Inter_greatest}  [| !!X. X:A ==> C<=X |] ==> C <= Inter(A)

\idx{Un_upper1}       A <= A Un B
\idx{Un_upper2}       B <= A Un B
\idx{Un_least}        [| A<=C;  B<=C |] ==> A Un B <= C

\idx{Int_lower1}      A Int B <= A
\idx{Int_lower2}      A Int B <= B
\idx{Int_greatest}    [| C<=A;  C<=B |] ==> C <= A Int B
\end{ttbox}
\caption{Derived rules involving subsets} \label{hol-subset}
\end{figure}


\begin{figure} \makeatother
\begin{ttbox}
\idx{Int_absorb}         A Int A = A
\idx{Int_commute}        A Int B = B Int A
\idx{Int_assoc}          (A Int B) Int C  =  A Int (B Int C)
\idx{Int_Un_distrib}     (A Un B)  Int C  =  (A Int C) Un (B Int C)

\idx{Un_absorb}          A Un A = A
\idx{Un_commute}         A Un B = B Un A
\idx{Un_assoc}           (A Un B)  Un C  =  A Un (B Un C)
\idx{Un_Int_distrib}     (A Int B) Un C  =  (A Un C) Int (B Un C)

\idx{Compl_disjoint}     A Int Compl(A) = \{x.False\} 
\idx{Compl_partition}    A Un  Compl(A) = \{x.True\}
\idx{double_complement}  Compl(Compl(A)) = A
\idx{Compl_Un}           Compl(A Un B)  = Compl(A) Int Compl(B)
\idx{Compl_Int}          Compl(A Int B) = Compl(A) Un Compl(B)

\idx{Union_Un_distrib}   Union(A Un B) = Union(A) Un Union(B)
\idx{Int_Union_image}    A Int Union(B) = (UN C:B. A Int C)
\idx{Un_Union_image} 
    (UN x:C. A(x) Un B(x)) = Union(A``C)  Un  Union(B``C)

\idx{Inter_Un_distrib}   Inter(A Un B) = Inter(A) Int Inter(B)
\idx{Un_Inter_image}     A Un Inter(B) = (INT C:B. A Un C)
\idx{Int_Inter_image}
   (INT x:C. A(x) Int B(x)) = Inter(A``C) Int Inter(B``C)
\end{ttbox}
\caption{Set equalities} \label{hol-equalities}
\end{figure}


\subsection{Derived rules for sets}
Figures~\ref{hol-set1} and~\ref{hol-set2} present derived rules.  Most
are obvious and resemble rules of Isabelle's {\ZF} set theory.  The
rules named $XXX${\tt_cong} break down equalities.  Certain rules, such as
\ttindexbold{subsetCE}, \ttindexbold{bexCI} and \ttindexbold{UnCI}, are
designed for classical reasoning; the more natural rules \ttindexbold{subsetD},
\ttindexbold{bexI}, \ttindexbold{Un1} and~\ttindexbold{Un2} are not
strictly necessary.  Similarly, \ttindexbold{equalityCE} supports classical
reasoning about extensionality, after the fashion of \ttindex{iffCE}.  See
the file \ttindexbold{HOL/set.ML} for proofs pertaining to set theory.

Figure~\ref{hol-fun} presents derived rules involving functions.  See
the file \ttindexbold{HOL/fun.ML} for a complete listing.

Figure~\ref{hol-subset} presents lattice properties of the subset relation.
See \ttindexbold{HOL/subset.ML}.

Figure~\ref{hol-equalities} presents set equalities.  See
\ttindexbold{HOL/equalities.ML}.


\begin{figure} \makeatother
\begin{center}
\begin{tabular}{rrr} 
  \it name    	&\it meta-type 	& \it description \\ 
  \idx{Pair}	& $[\alpha,\beta]\To \alpha\times\beta$
	& ordered pairs $\langle a,b\rangle$ \\
  \idx{fst}	& $\alpha\times\beta \To \alpha$		& first projection\\
  \idx{snd}	& $\alpha\times\beta \To \beta$		& second projection\\
  \idx{split}	& $[\alpha\times\beta, [\alpha,\beta]\To\gamma] \To \gamma$ 
	& generalized projection
\end{tabular}
\end{center}
\subcaption{Constants}

\begin{ttbox}
\idx{fst_def}      fst(p)     == @a. ? b. p = <a,b>
\idx{snd_def}      snd(p)     == @b. ? a. p = <a,b>
\idx{split_def}    split(p,c) == c(fst(p),snd(p))
\subcaption{Definitions}

\idx{Pair_inject}  [| <a, b> = <a',b'>;  [| a=a';  b=b' |] ==> R |] ==> R

\idx{fst}          fst(<a,b>) = a
\idx{snd}          snd(<a,b>) = b
\idx{split}        split(<a,b>, c) = c(a,b)

\idx{surjective_pairing}  p = <fst(p),snd(p)>
\subcaption{Derived rules}
\end{ttbox}
\caption{Type $\alpha\times\beta$} 
\label{hol-prod}
\end{figure} 


\begin{figure} \makeatother
\begin{center}
\begin{tabular}{rrr} 
  \it name    	&\it meta-type 	& \it description \\ 
  \idx{Inl}	& $\alpha \To \alpha+\beta$			& first injection\\
  \idx{Inr}	& $\beta \To \alpha+\beta$			& second injection\\
  \idx{case}	& $[\alpha+\beta, \alpha\To\gamma, \beta\To\gamma] \To\gamma$
	& conditional
\end{tabular}
\end{center}
\subcaption{Constants}

\begin{ttbox}
\idx{case_def}     case == (%p f g. @z. (!x. p=Inl(x) --> z=f(x)) &
                                  (!y. p=Inr(y) --> z=g(y)))
\subcaption{Definition}

\idx{Inl_not_Inr}    ~ Inl(a)=Inr(b)

\idx{inj_Inl}        inj(Inl)
\idx{inj_Inr}        inj(Inr)

\idx{sumE}           [| !!x::'a. P(Inl(x));  !!y::'b. P(Inr(y)) |] ==> P(s)

\idx{case_Inl}       case(Inl(x), f, g) = f(x)
\idx{case_Inr}       case(Inr(x), f, g) = g(x)

\idx{surjective_sum} case(s, %x::'a. f(Inl(x)), %y::'b. f(Inr(y))) = f(s)
\subcaption{Derived rules}
\end{ttbox}
\caption{Rules for type $\alpha+\beta$} 
\label{hol-sum}
\end{figure}


\section{Types}
The basic higher-order logic is augmented with a tremendous amount of
material, including support for recursive function and type definitions.
Space does not permit a detailed discussion.  The present section describes
product, sum, natural number and list types.

\subsection{Product and sum types}
{\HOL} defines the product type $\alpha\times\beta$ and the sum type
$\alpha+\beta$, with the ordered pair syntax {\tt<$a$,$b$>}, using fairly
standard constructions (Figures~\ref{hol-prod} and~\ref{hol-sum}).  Because
Isabelle does not support type definitions, the isomorphisms between these
types and their representations are made explicitly.

Most of the definitions are suppressed, but observe that the projections
and conditionals are defined as descriptions.  Their properties are easily
proved using \ttindex{select_equality}.  See \ttindexbold{HOL/prod.thy} and
\ttindexbold{HOL/sum.thy} for details.

\begin{figure} \makeatother
\indexbold{*"<}
\begin{center}
\begin{tabular}{rrr} 
  \it symbol  	& \it meta-type & \it description \\ 
  \idx{0}	& $nat$		& zero \\
  \idx{Suc}	& $nat \To nat$	& successor function\\
  \idx{nat_case} & $[nat, \alpha, nat\To\alpha] \To\alpha$
	& conditional\\
  \idx{nat_rec} & $[nat, \alpha, [nat, \alpha]\To\alpha] \To \alpha$
	& primitive recursor\\
  \idx{pred_nat} & $(nat\times nat) set$ & predecessor relation
\end{tabular}
\end{center}

\begin{center}
\indexbold{*"+}
\index{*@{\tt*}|bold}
\index{/@{\tt/}|bold}
\index{//@{\tt//}|bold}
\index{+@{\tt+}|bold}
\index{-@{\tt-}|bold}
\begin{tabular}{rrrr} 
  \it symbol  	& \it meta-type & \it precedence & \it description \\ 
  \tt *		& $[nat,nat]\To nat$	&  Left 70	& multiplication \\
  \tt /		& $[nat,nat]\To nat$	&  Left 70	& division\\
  \tt //	& $[nat,nat]\To nat$	&  Left 70	& modulus\\
  \tt +		& $[nat,nat]\To nat$	&  Left 65	& addition\\
  \tt -		& $[nat,nat]\To nat$ 	&  Left 65	& subtraction
\end{tabular}
\end{center}
\subcaption{Constants and infixes}

\begin{ttbox}
\idx{nat_case_def}  nat_case == (%n a f. @z. (n=0 --> z=a) & 
                                        (!x. n=Suc(x) --> z=f(x)))
\idx{pred_nat_def}  pred_nat == \{p. ? n. p = <n, Suc(n)>\} 
\idx{less_def}      m<n      == <m,n>:pred_nat^+
\idx{nat_rec_def}   nat_rec(n,c,d) == 
               wfrec(pred_nat, n, %l g.nat_case(l, c, %m.d(m,g(m))))

\idx{add_def}   m+n  == nat_rec(m, n, %u v.Suc(v))
\idx{diff_def}  m-n  == nat_rec(n, m, %u v. nat_rec(v, 0, %x y.x))
\idx{mult_def}  m*n  == nat_rec(m, 0, %u v. n + v)
\idx{mod_def}   m//n == wfrec(trancl(pred_nat), m, %j f. if(j<n,j,f(j-n)))
\idx{quo_def}   m/n  == wfrec(trancl(pred_nat), 
                        m, %j f. if(j<n,0,Suc(f(j-n))))
\subcaption{Definitions}
\end{ttbox}
\caption{Defining $nat$, the type of natural numbers} \label{hol-nat1}
\end{figure}


\begin{figure} \makeatother
\begin{ttbox}
\idx{nat_induct}     [| P(0); !!k. [| P(k) |] ==> P(Suc(k)) |]  ==> P(n)

\idx{Suc_not_Zero}   Suc(m) ~= 0
\idx{inj_Suc}        inj(Suc)
\idx{n_not_Suc_n}    n~=Suc(n)
\subcaption{Basic properties}

\idx{pred_natI}      <n, Suc(n)> : pred_nat
\idx{pred_natE}
    [| p : pred_nat;  !!x n. [| p = <n, Suc(n)> |] ==> R |] ==> R

\idx{nat_case_0}     nat_case(0, a, f) = a
\idx{nat_case_Suc}   nat_case(Suc(k), a, f) = f(k)

\idx{wf_pred_nat}    wf(pred_nat)
\idx{nat_rec_0}      nat_rec(0,c,h) = c
\idx{nat_rec_Suc}    nat_rec(Suc(n), c, h) = h(n, nat_rec(n,c,h))
\subcaption{Case analysis and primitive recursion}

\idx{less_trans}     [| i<j;  j<k |] ==> i<k
\idx{lessI}          n < Suc(n)
\idx{zero_less_Suc}  0 < Suc(n)

\idx{less_not_sym}   n<m --> ~ m<n 
\idx{less_not_refl}  ~ n<n
\idx{not_less0}      ~ n<0

\idx{Suc_less_eq}    (Suc(m) < Suc(n)) = (m<n)
\idx{less_induct}    [| !!n. [| ! m. m<n --> P(m) |] ==> P(n) |]  ==>  P(n)

\idx{less_linear}    m<n | m=n | n<m
\subcaption{The less-than relation}
\end{ttbox}
\caption{Derived rules for~$nat$} \label{hol-nat2}
\end{figure}


\subsection{The type of natural numbers, $nat$}
{\HOL} defines the natural numbers in a roundabout but traditional way.
The axiom of infinity postulates an type~$ind$ of individuals, which is
non-empty and closed under an injective operation.  The natural numbers are
inductively generated by choosing an arbitrary individual for~0 and using
the injective operation to take successors.  As usual, the isomorphisms
between~$nat$ and its representation are made explicitly.

The definition makes use of a least fixed point operator \ttindex{lfp},
defined using the Knaster-Tarski theorem.  This in turn defines an operator
\ttindex{trancl} for taking the transitive closure of a relation.  See
files \ttindexbold{HOL/lfp.thy} and \ttindexbold{HOL/trancl.thy} for
details.  The definition of~$nat$ resides on \ttindexbold{HOL/nat.thy}.  

Type $nat$ is postulated to belong to class~$ord$, which overloads $<$ and
$\leq$ on the natural numbers.  As of this writing, Isabelle provides no
means of verifying that such overloading is sensible.  On the other hand,
the {\HOL} theory includes no polymorphic axioms stating general properties
of $<$ and $\leq$.

File \ttindexbold{HOL/arith.ML} develops arithmetic on the natural numbers.
It defines addition, multiplication, subtraction, division, and remainder,
proving the theorem $a \bmod b + (a/b)\times b = a$.  Division and
remainder are defined by repeated subtraction, which requires well-founded
rather than primitive recursion.

Primitive recursion makes use of \ttindex{wfrec}, an operator for recursion
along arbitrary well-founded relations; see \ttindexbold{HOL/wf.ML} for the
development.  The predecessor relation, \ttindexbold{pred_nat}, is shown to
be well-founded; recursion along this relation is primitive recursion,
while its transitive closure is~$<$.


\begin{figure} \makeatother
\begin{center}
\begin{tabular}{rrr} 
  \it symbol  	& \it meta-type & \it description \\ 
  \idx{Nil}	& $\alpha list$	& the empty list\\
  \idx{Cons}	& $[\alpha, \alpha list] \To \alpha list$
	& list constructor\\
  \idx{list_rec}	& $[\alpha list, \beta, [\alpha ,\alpha list,
\beta]\To\beta] \To \beta$
	& list recursor\\
  \idx{map}	& $(\alpha\To\beta) \To (\alpha list \To \beta list)$
	& mapping functional
\end{tabular}
\end{center}
\subcaption{Constants}

\begin{ttbox}
\idx{map_def}     map(f,xs) == list_rec(xs, Nil, %x l r. Cons(f(x), r))
\subcaption{Definition}

\idx{list_induct}
    [| P(Nil);  !!x xs. [| P(xs) |] ==> P(Cons(x,xs)) |]  ==> P(l)

\idx{Cons_not_Nil}   ~ Cons(x,xs) = Nil
\idx{Cons_Cons_eq}   (Cons(x,xs)=Cons(y,ys)) = (x=y & xs=ys)

\idx{list_rec_Nil}   list_rec(Nil,c,h) = c
\idx{list_rec_Cons}  list_rec(Cons(a,l), c, h) = h(a, l, list_rec(l,c,h))

\idx{map_Nil}        map(f,Nil) = Nil
\idx{map_Cons}       map(f, Cons(x,xs)) = Cons(f(x), map(f,xs))
\end{ttbox}
\caption{The type of lists and its operations} \label{hol-list}
\end{figure}


\subsection{The type constructor for lists, $\alpha\,list$}
{\HOL}'s definition of lists is an example of an experimental method for
handling recursive data types.  The details need not concern us here; see
the file \ttindexbold{HOL/list.ML}.  Figure~\ref{hol-list} presents the
basic list operations, with their types and properties.  In particular,
\ttindexbold{list_rec} is a primitive recursion operator for lists, in the
style of Martin-L\"of type theory.  It is derived from well-founded
recursion, a general principle that can express arbitrary total recursive
functions. 


\subsection{The type constructor for lazy lists, $\alpha\,llist$}
The definition of lazy lists demonstrates methods for handling infinite
data structures and co-induction in higher-order logic.  It defines an
operator for co-recursion on lazy lists, which is used to define a few
simple functions such as map and append.  Co-recursion cannot easily define
operations such as filter, which can compute indefinitely before yielding
the next element (if any!) of the lazy list.  A co-induction principle is
defined for proving equations on lazy lists.  See the files
\ttindexbold{HOL/llist.thy} and \ttindexbold{HOL/llist.ML} for the formal
derivations.  I have written a report discussing the treatment of lazy
lists, and finite lists also~\cite{paulson-coind}.


\section{Classical proof procedures} \label{hol-cla-prover}
{\HOL} derives classical introduction rules for $\disj$ and~$\exists$, as
well as classical elimination rules for~$\imp$ and~$\bimp$, and the swap
rule (Figure~\ref{hol-lemmas2}).

The classical reasoning module is set up for \HOL, as the structure 
\ttindexbold{Classical}.  This structure is open, so {\ML} identifiers such
as {\tt step_tac}, {\tt fast_tac}, {\tt best_tac}, etc., refer to it.

{\HOL} defines the following classical rule sets:
\begin{ttbox} 
prop_cs    : claset
HOL_cs     : claset
HOL_dup_cs : claset
set_cs     : claset
\end{ttbox}
\begin{description}
\item[\ttindexbold{prop_cs}] contains the propositional rules, namely
those for~$\top$, $\bot$, $\conj$, $\disj$, $\neg$, $\imp$ and~$\bimp$,
along with the rule~\ttindex{refl}.

\item[\ttindexbold{HOL_cs}] 
extends {\tt prop_cs} with the safe rules \ttindex{allI} and~\ttindex{exE}
and the unsafe rules \ttindex{allE} and~\ttindex{exI}, as well as rules for
unique existence.  Search using this is incomplete since quantified
formulae are used at most once.

\item[\ttindexbold{HOL_dup_cs}] 
extends {\tt prop_cs} with the safe rules \ttindex{allI} and~\ttindex{exE}
and the unsafe rules \ttindex{all_dupE} and~\ttindex{exCI}, as well as
rules for unique existence.  Search using this is complete --- quantified
formulae may be duplicated --- but frequently fails to terminate.  It is
generally unsuitable for depth-first search.

\item[\ttindexbold{set_cs}] 
extends {\tt HOL_cs} with rules for the bounded quantifiers, subsets,
comprehensions, unions/intersections, complements, finite setes, images and
ranges.
\end{description}
\noindent
See the {\em Reference Manual} for more discussion of classical proof
methods.


\section{The examples directory}
This directory contains examples and experimental proofs in {\HOL}.  Here
is an overview of the more interesting files.
\begin{description}
\item[\ttindexbold{HOL/ex/meson.ML}]
contains an experimental implementation of the MESON proof procedure,
inspired by Plaisted~\cite{plaisted90}.  It is much more powerful than
Isabelle's classical module.  

\item[\ttindexbold{HOL/ex/meson-test.ML}]
contains test data for the MESON proof procedure.

\item[\ttindexbold{HOL/ex/set.ML}]
proves Cantor's Theorem (see below) and the Schr\"oder-Bernstein Theorem.

\item[\ttindexbold{HOL/ex/prop-log.ML}]
proves the soundness and completeness of classical propositional logic,
given a truth table semantics.  The only connective is $\imp$.  A
Hilbert-style axiom system is specified, and its set of theorems defined
inductively.

\item[\ttindexbold{HOL/ex/term.ML}]
is an experimental recursive type definition, where the recursion goes
through the type constructor~$list$.

\item[\ttindexbold{HOL/ex/simult.ML}]
defines primitives for solving mutually recursive equations over sets.
It constructs sets of trees and forests as an example, including induction
and recursion rules that handle the mutual recursion.
\end{description}


\section{Example: deriving the conjunction rules}
{\HOL} comes with a body of derived rules, ranging from simple properties
of the logical constants and set theory to well-founded recursion.  Many of
them are worth studying.

Deriving natural deduction rules for the logical constants from their
definitions is an archetypal example of higher-order reasoning.  Let us
verify two conjunction rules:
\[ \infer[({\conj}I)]{P\conj Q}{P & Q} \qquad\qquad
   \infer[({\conj}E1)]{P}{P\conj Q}  
\]

\subsection{The introduction rule}
We begin by stating the rule as the goal.  The list of premises $[P,Q]$ is
bound to the {\ML} variable~{\tt prems}.
\begin{ttbox}
val prems = goal HOL.thy "[| P; Q |] ==> P&Q";
{\out Level 0}
{\out P & Q}
{\out  1. P & Q}
\end{ttbox}
The next step is to unfold the definition of conjunction.  But
\ttindex{and_def} uses {\HOL}'s internal equality, so
\ttindex{rewrite_goals_tac} is unsuitable.
Instead, we perform substitution using the rule \ttindex{ssubst}:
\begin{ttbox}
by (resolve_tac [and_def RS ssubst] 1);
{\out Level 1}
{\out P & Q}
{\out  1. ! R. (P --> Q --> R) --> R}
\end{ttbox}
We now apply $(\forall I)$ and $({\imp}I)$:
\begin{ttbox}
by (resolve_tac [allI] 1);
{\out Level 2}
{\out P & Q}
{\out  1. !!R. (P --> Q --> R) --> R}
by (resolve_tac [impI] 1);
{\out Level 3}
{\out P & Q}
{\out  1. !!R. P --> Q --> R ==> R}
\end{ttbox}
The assumption is a nested implication, which may be eliminated
using~\ttindex{mp} resolved with itself.  Elim-resolution, here, performs
backwards chaining.  More straightforward would be to use~\ttindex{impE}
twice.
\index{*RS}
\begin{ttbox}
by (eresolve_tac [mp RS mp] 1);
{\out Level 4}
{\out P & Q}
{\out  1. !!R. P}
{\out  2. !!R. Q}
\end{ttbox}
These two subgoals are simply the premises:
\begin{ttbox}
by (REPEAT (resolve_tac prems 1));
{\out Level 5}
{\out P & Q}
{\out No subgoals!}
\end{ttbox}


\subsection{The elimination rule}
Again, we bind the list of premises (in this case $[P\conj Q]$)
to~{\tt prems}.
\begin{ttbox}
val prems = goal HOL.thy "[| P & Q |] ==> P";
{\out Level 0}
{\out P}
{\out  1. P}
\end{ttbox}
Working with premises that involve defined constants can be tricky.  We
must expand the definition of conjunction in the meta-assumption $P\conj
Q$.  The rule \ttindex{subst} performs substitution in forward proofs.
We get two resolvents, since the vacuous substitution is valid:
\begin{ttbox}
prems RL [and_def RS subst];
{\out val it = ["! R. (P --> Q --> R) --> R  [P & Q]",}
{\out           "P & Q  [P & Q]"] : thm list}
\end{ttbox}
By applying $(\forall E)$ and $({\imp}E)$ to the resolvents, we dispose of
the vacuous one and put the other into a convenient form:
\index{*RL}
\begin{ttbox}
prems RL [and_def RS subst] RL [spec] RL [mp];
{\out val it = ["P --> Q --> ?Q ==> ?Q  [P & Q]"] : thm list}
\end{ttbox}
This is a list containing a single rule, which is directly applicable to
our goal:
\begin{ttbox}
by (resolve_tac it 1);
{\out Level 1}
{\out P}
{\out  1. P --> Q --> P}
\end{ttbox}
The subgoal is a trivial implication.  Recall that \ttindex{ares_tac} is a
combination of \ttindex{assume_tac} and \ttindex{resolve_tac}.
\begin{ttbox}
by (REPEAT (ares_tac [impI] 1));
{\out Level 2}
{\out P}
{\out No subgoals!}
\end{ttbox}


\section{Example: Cantor's Theorem}
Cantor's Theorem states that every set has more subsets than it has
elements.  It has become a favourite example in higher-order logic since
it is so easily expressed:
\[  \forall f::[\alpha,\alpha]\To bool. \exists S::\alpha\To bool.
    \forall x::\alpha. f(x) \not= S 
\] 
Viewing types as sets, $\alpha\To bool$ represents the powerset
of~$\alpha$.  This version states that for every function from $\alpha$ to
its powerset, some subset is outside its range.

The Isabelle proof uses {\HOL}'s set theory, with the type $\alpha\,set$ and
the operator \ttindex{range}.  Since it avoids quantification, we may
inspect the subset found by the proof.
\begin{ttbox}
goal Set.thy "~ ?S : range(f :: 'a=>'a set)";
{\out Level 0}
{\out ~ ?S : range(f)}
{\out  1. ~ ?S : range(f)}
\end{ttbox}
The first two steps are routine.  The rule \ttindex{rangeE} reasons that,
since $\Var{S}\in range(f)$, we have $\Var{S}=f(x)$ for some~$x$.
\begin{ttbox}
by (resolve_tac [notI] 1);
{\out Level 1}
{\out ~ ?S : range(f)}
{\out  1. ?S : range(f) ==> False}
by (eresolve_tac [rangeE] 1);
{\out Level 2}
{\out ~ ?S : range(f)}
{\out  1. !!x. ?S = f(x) ==> False}
\end{ttbox}
Next, we apply \ttindex{equalityCE}, reasoning that since $\Var{S}=f(x)$,
we have $\Var{c}\in \Var{S}$ if and only if $\Var{c}\in f(x)$ for
any~$\Var{c}$.
\begin{ttbox}
by (eresolve_tac [equalityCE] 1);
{\out Level 3}
{\out ~ ?S : range(f)}
{\out  1. !!x. [| ?c3(x) : ?S; ?c3(x) : f(x) |] ==> False}
{\out  2. !!x. [| ~ ?c3(x) : ?S; ~ ?c3(x) : f(x) |] ==> False}
\end{ttbox}
Now we use a bit of creativity.  Suppose that $\Var{S}$ has the form of a
comprehension.  Then $\Var{c}\in\{x.\Var{P}(x)\}$ implies
$\Var{P}(\Var{c})\}$.\index{*CollectD}
\begin{ttbox}
by (dresolve_tac [CollectD] 1);
{\out Level 4}
{\out ~ \{x. ?P7(x)\} : range(f)}
{\out  1. !!x. [| ?c3(x) : f(x); ?P7(?c3(x)) |] ==> False}
{\out  2. !!x. [| ~ ?c3(x) : \{x. ?P7(x)\}; ~ ?c3(x) : f(x) |] ==> False}
\end{ttbox}
Forcing a contradiction between the two assumptions of subgoal~1 completes
the instantiation of~$S$.  It is now $\{x. x\not\in f(x)\}$, the standard
diagonal construction.
\begin{ttbox}
by (contr_tac 1);
{\out Level 5}
{\out ~ \{x. ~ x : f(x)\} : range(f)}
{\out  1. !!x. [| ~ x : \{x. ~ x : f(x)\}; ~ x : f(x) |] ==> False}
\end{ttbox}
The rest should be easy.  To apply \ttindex{CollectI} to the negated
assumption, we employ \ttindex{swap_res_tac}:
\begin{ttbox}
by (swap_res_tac [CollectI] 1);
{\out Level 6}
{\out ~ \{x. ~ x : f(x)\} : range(f)}
{\out  1. !!x. [| ~ x : f(x); ~ False |] ==> ~ x : f(x)}
by (assume_tac 1);
{\out Level 7}
{\out ~ \{x. ~ x : f(x)\} : range(f)}
{\out No subgoals!}
\end{ttbox}
How much creativity is required?  As it happens, Isabelle can prove this
theorem automatically.  The classical set \ttindex{set_cs} contains rules
for most of the constructs of {\HOL}'s set theory.  We augment it with
\ttindex{equalityCE} --- set equalities are not broken up by default ---
and apply best-first search.  Depth-first search would diverge, but
best-first search successfully navigates through the large search space.
\begin{ttbox}
choplev 0;
{\out Level 0}
{\out ~ ?S : range(f)}
{\out  1. ~ ?S : range(f)}
by (best_tac (set_cs addSEs [equalityCE]) 1);
{\out Level 1}
{\out ~ \{x. ~ x : f(x)\} : range(f)}
{\out No subgoals!}
\end{ttbox}
