%process by     nlatex cade; dvips -o s.ps -Ppreview cade
%               ghostview -magstep -2 -landscape s.ps
\documentstyle[fancybox,newlfont,lcp,proof,semhelv,sem-a4,%
%  article,%
  slidesonly,%Try notes or notesonly instead
  semlayer%     This must be included, but you need the semcolor option to
  ]{seminar}                                  % actually see the overlays.

\def\printlandscape{\special{landscape}}    % Works with dvips.

\slidesmag{5}\articlemag{2}    %the difference is 3 instead of 4!
\extraslideheight{30pt}

\renewcommand\slidefuzz{6pt}
\sloppy\hfuzz2pt   %sloppy defines \hfuzz0.5pt but it's mainly for text

\newcommand\sbs{\subseteq}
\newcommand\Pow{{\cal P}}
\newcommand\lfp{\hbox{\tt lfp}}
\newcommand\gfp{\hbox{\tt gfp}}
\newcommand\lst{\hbox{\tt list}}
\newcommand\term{\hbox{\tt term}}

\newcommand\heading[1]{%
  \begin{center}\large\bf\shadowbox{#1}\end{center}
  \vspace{1ex minus 1ex}}

\newpagestyle{mine}%
  {L. Paulson\hfil A Fixedpoint Approach to (Co)Inductive Definitions\hfil
      \thepage}%
  {\hfil\special{psfile=cuarms.eps hscale=20 vscale=20 voffset=-6
      hoffset=-14}\hfil}
\pagestyle{mine}


\begin{document}
\slidefonts

\begin{slide}\centering
\shadowbox{% 
    \begin{Bcenter}
    {\Large\bf A Fixedpoint Approach to}\\[2ex]
    {\Large\bf (Co)Inductive Definitions}
    \end{Bcenter}}
\bigskip

    \begin{Bcenter}
    Lawrence C. Paulson\\
    Computer Laboratory\\
    University of Cambridge\\
    England\\[1ex]
    \verb|lcp@cl.cam.ac.uk|
    \end{Bcenter}
\bigskip

{\footnotesize Thanks: SERC grants GR/G53279, GR/H40570; ESPRIT Project 6453
  `Types'} 
\end{slide}


\begin{slide}
\heading{Inductive Definitions}
\begin{itemize}
  \item {\bf datatypes}
    \begin{itemize}
      \item finite lists, trees
      \item syntax of expressions, \ldots
    \end{itemize}
  \item {\bf inference systems}
    \begin{itemize}
      \item transitive closure of a relation
      \item transition systems
      \item structural operational semantics
    \end{itemize}
\end{itemize}

Supported by Boyer/Moore, HOL, Coq, \ldots, Isabelle/ZF
\end{slide}


\begin{slide}
\heading{Coinductive Definitions}
\begin{itemize}
  \item {\bf codatatypes}
    \begin{itemize}
      \item {\it infinite\/} lists, trees
      \item  syntax of {\it infinite\/} expressions, \ldots
    \end{itemize}
  \item {\bf bisimulation relations}
    \begin{itemize}
      \item process equivalence
      \item uses in functional programming (Abramksy, Howe)
    \end{itemize}
\end{itemize}

Supported by \ldots ?, \ldots, Isabelle/ZF
\end{slide}


\begin{slide}
\heading{The Knaster-Tarksi Fixedpoint Theorem}
$h$ a monotone function

$D$ a set such that $h(D)\sbs D$

The {\bf least} fixedpoint $\lfp(D,h)$ yields {\bf inductive} definitions

The {\bf greatest} fixedpoint $\gfp(D,h)$ yields {\bf coinductive} definitions

{\it A general approach\/}:
\begin{itemize}
  \item handles all provably monotone definitions
  \item works for set theory, higher-order logic, \ldots
\end{itemize}
\end{slide}


\begin{slide}
\heading{An Implementation in Isabelle/ZF}\centering
\begin{itemize}
  \item {\bf Input} 
     \begin{itemize}
      \item description of {\it introduction rules\/} \& tree's {\it
          constructors\/} 
      \item {\it theorems\/} implying that the definition is {\it monotonic\/}
    \end{itemize}
  \item {\bf Output} 
     \begin{itemize}
      \item (co)induction rules
      \item case analysis rule and {\it rule inversion\/} tools, \ldots
    \end{itemize}
\end{itemize}

\vfill
{\it flexible, secure, \ldots but fast\/}
\end{slide}


\begin{slide}
\heading{Working Examples}
\begin{itemize}
  \item lists
  \item terms recursive over lists: $\term(A) = A\times\lst(\term(A))$
  \item primitive recursive functions
  \item lazy lists
  \item bisimulations for lazy lists
  \item combinator reductions; Church-Rosser Theorem
  \item mutually recursive trees \& forests
\end{itemize}
\end{slide}


\begin{slide}
\heading{Other Work Using Fixedpoints}
{\bf The HOL system}:
\begin{itemize}
  \item Melham's induction package: special case of Fixedpoint Theorem
  \item Andersen \& Petersen's induction package
  \item (no HOL datatype package uses fixedpoints)
\end{itemize}

{\bf Coq and LEGO}:
\begin{itemize}
  \item (Co)induction {\it almost\/} expressible in base logic (CoC)
  \item \ldots{} inductive definitions are built-in
\end{itemize}
\end{slide}


\begin{slide}
\heading{Limitations \& Future Developments}\centering
\begin{itemize}
  \item {\bf infinite-branching trees}
    \begin{itemize}
      \item justification requires proof
      \item would be easier to {\it build them in\/}!
    \end{itemize}
  \item {\bf recursive function definitions}
    \begin{itemize}
      \item use {\it well-founded\/} recursion
      \item distinct from datatype definitions
    \end{itemize}
  \item {\bf port to Isabelle/HOL}
\end{itemize}
\end{slide}

\end{document}


{\it flat\/} ordered pairs used to define infinite lists, \ldots

\begin{slide}
\heading{}\centering
\end{slide}

